%% Generated by Sphinx.
\def\sphinxdocclass{report}
\documentclass[letterpaper,10pt,spanish]{sphinxmanual}
\ifdefined\pdfpxdimen
   \let\sphinxpxdimen\pdfpxdimen\else\newdimen\sphinxpxdimen
\fi \sphinxpxdimen=.75bp\relax

\usepackage[utf8]{inputenc}
\ifdefined\DeclareUnicodeCharacter
 \ifdefined\DeclareUnicodeCharacterAsOptional
  \DeclareUnicodeCharacter{"00A0}{\nobreakspace}
  \DeclareUnicodeCharacter{"2500}{\sphinxunichar{2500}}
  \DeclareUnicodeCharacter{"2502}{\sphinxunichar{2502}}
  \DeclareUnicodeCharacter{"2514}{\sphinxunichar{2514}}
  \DeclareUnicodeCharacter{"251C}{\sphinxunichar{251C}}
  \DeclareUnicodeCharacter{"2572}{\textbackslash}
 \else
  \DeclareUnicodeCharacter{00A0}{\nobreakspace}
  \DeclareUnicodeCharacter{2500}{\sphinxunichar{2500}}
  \DeclareUnicodeCharacter{2502}{\sphinxunichar{2502}}
  \DeclareUnicodeCharacter{2514}{\sphinxunichar{2514}}
  \DeclareUnicodeCharacter{251C}{\sphinxunichar{251C}}
  \DeclareUnicodeCharacter{2572}{\textbackslash}
 \fi
\fi
\usepackage{cmap}
\usepackage[T1]{fontenc}
\usepackage{amsmath,amssymb,amstext}
\usepackage{babel}
\usepackage{times}
\usepackage[Sonny]{fncychap}
\usepackage[dontkeepoldnames]{sphinx}

\usepackage{geometry}

% Include hyperref last.
\usepackage{hyperref}
% Fix anchor placement for figures with captions.
\usepackage{hypcap}% it must be loaded after hyperref.
% Set up styles of URL: it should be placed after hyperref.
\urlstyle{same}

\addto\captionsspanish{\renewcommand{\figurename}{Figura}}
\addto\captionsspanish{\renewcommand{\tablename}{Tabla}}
\addto\captionsspanish{\renewcommand{\literalblockname}{Lista}}

\addto\captionsspanish{\renewcommand{\literalblockcontinuedname}{continued from previous page}}
\addto\captionsspanish{\renewcommand{\literalblockcontinuesname}{continues on next page}}

\addto\extrasspanish{\def\pageautorefname{página}}

\setcounter{tocdepth}{2}



\title{Apuntes de Lenguajes de Marcas Documentation}
\date{29 de mayo de 2018}
\release{1.3}
\author{Oscar Gomez}
\newcommand{\sphinxlogo}{\vbox{}}
\renewcommand{\releasename}{Versión}
\makeindex

\begin{document}
\ifnum\catcode`\"=\active\shorthandoff{"}\fi
\maketitle
\sphinxtableofcontents
\phantomsection\label{\detokenize{index::doc}}



\chapter{Introducción}
\label{\detokenize{tema1::doc}}\label{\detokenize{tema1:apuntes-de-lenguajes-de-marcas}}\label{\detokenize{tema1:introduccion}}

\section{Historia}
\label{\detokenize{tema1:historia}}

\subsection{El pasado}
\label{\detokenize{tema1:el-pasado}}
Los lenguajes de marcas son bastante antiguos aunque solo se han popularizado con la llegada de Internet.  De hecho, los comienzos de los lenguajes de marcas se pueden situar en el lenguaje SGML (Standard Generalized Markup Language). Dicho lenguaje en realidad nunca llegó al gran público, aunque sigue existiendo y usándose en entornos muy específicos. Lo más usado y conocido hoy día es HTML (HyperText Markup Language).


\subsection{El presente}
\label{\detokenize{tema1:el-presente}}
El lenguaje HTML ha ido cambiando mucho a lo largo del tiempo. La versión más actual es la conocida como HTML5 y es la que se contará en estos apuntes.
El fundamento de Internet y todas las tecnologías asociadas se basa en estándares abiertos e
independientes de la tecnología. El organismo que regula estos estándares sin ninguna contrapartida a cambio es el \sphinxhref{http://www.w3c.org}{World Wide Web Consortium (W3C)} . Todos sus estándares (que ellos llaman «Technical Reports») pueden descargarse gratuitamente en su web.


\subsection{El futuro}
\label{\detokenize{tema1:el-futuro}}
El futuro de todas estas tecnologías sigue estando en sistemas abiertos, que facilitan la competencia y por tanto benefician al usuario. La web ha tenido un crecimiento exponencial debido a la existencias de las plataformas móviles, que han multiplicado los accesos a las páginas web (y que también han dificultado el diseño web, hablaremos más de ello en el tema sobre CSS).


\section{Servidores web}
\label{\detokenize{tema1:servidores-web}}
Este apartado intenta ilustrar muy por encima como funciona un servidor web. Un servidor web es un programa que “entiende” los protocolos HTTP y HTTPS y que atiende peticiones de navegadores sirviendo páginas a medida que se van solicitando. Tener un servidor no implica obligatoriamente un nombre de dominio. Es decir cuando en el navegador escribimos «\sphinxurl{http://www.google.es}» contactamos con un programa que se está ejecutando en un ordenador al que se le ha dado el nombre «www.google.es» y le pedimos establecer una conexión para intercambiar información usando el protocolo HTTP. Así, cuando un servidor recibe una petición examina sus directorios para ver si tiene ese archivo en el directorio que le han dicho, si es así, el fichero se transmite al que hizo la petición. Si no existe, el navegador devuelve un código 404 (recurso no encontrado) y nos mostrará la página web asociada a ese código 404. Si todo va bien nuestro navegador recibe un código 200

\begin{figure}[htbp]
\centering
\capstart

\noindent\sphinxincludegraphics[scale=0.5]{{http_get_y_post}.png}
\caption{Intercambio de mensajes entre cliente web (o navegador) y servidor web.}\label{\detokenize{tema1:id1}}\end{figure}

Un servidor web puede ser privado o alquilado, existiendo grandes diferencias entre la forma de gestionar ambos.
\begin{itemize}
\item {} 
Uno privado, tiene la ventaja de ofrecer un control absoluto. No siempre es fácil mantener un ordenador doméstico conectado 24x7. En especial, en España, las  conexiones ADSL no suelen ofrecer las capacidades necesarias para un sitio web de tamaño mediano.

\item {} 
Los alquilados suelen conllevar un mayor precio cuando se necesita mas espacio para alojar nuestros ficheros. Ofrecen muchas garantías como por ejemplo anchos de banda muy aceptables a precios bastante competitivos y sobre todo que permiten al diseñador web liberarse de la gestión de los recursos de red

\end{itemize}


\subsection{Como configurar un servidor en casa}
\label{\detokenize{tema1:como-configurar-un-servidor-en-casa}}\begin{enumerate}
\item {} 
Se necesita un servidor web instalado en un equipo (Apache del paquete XAMPP)

\item {} 
Apuntar la IP del equipo en el que se instala Apache.

\item {} 
Entrar en el router y \sphinxstylestrong{abrir el puerto 80} indicando que se debe reenviar el tráfico a la IP que se apuntó.

\item {} 
Se debe averiguar la IP pública del router (por ejemplo podemos visitar \sphinxhref{http://whatsmyip.org}{WhatsMyIp} ,

\item {} 
La IP que aparece es la que se puede dar a clientes o amigos para que naveguen por nuestro sitio web.

\item {} 
(Optativo) se puede alquilar un nombre de dominio y solicitar que ese nombre (loquesea.com) sea redirigido a nuestra IP pública.

\end{enumerate}


\section{Los nombres de dominio}
\label{\detokenize{tema1:los-nombres-de-dominio}}
El servicio de nombres de dominio es un sistema que convierte de direcciones tipo www.loquesea.com a direcciones IP. Esto permite que sea más fácil recordar direcciones de páginas. Sin embargo, DNS esun sistema muy complejo que funciona de forma distribuida entre distintos países.

Los nombres de dominio se resuelven de final a principio. La última parte se llama TLD o Top Level Domain o dominio de primer nivel.. Estos dominios son administrados por países ocupándose cada uno de ellos de los nombres o marcas que hay dentro de dichos países.


\section{Los sistemas de gestión de información}
\label{\detokenize{tema1:los-sistemas-de-gestion-de-informacion}}
Definamos primero algunos términos:
\begin{itemize}
\item {} 
Sistema: conjunto de elementos interrelacionados que colaboran en la consecución de un objetivo.

\item {} 
Gestión: conjunto de operaciones que resultan de relevancia para una persona o empresa.

\item {} 
Información: conjunto de datos que resultan de utilidad a las funciones de la empresa.

\end{itemize}

Un SGI no tiene por qué estar informatizado.

Son programas que se pueden adaptar las necesidades de la empresa y que a veces necesitan
importar (o exportar) datos e información.

El uso de un formato unificado (normalmente basado en marcas) facilita enormemente los
trasvases de datos.


\section{Tipos de lenguajes}
\label{\detokenize{tema1:tipos-de-lenguajes}}
Aparte de los lenguajes de marcas típicos, existen otros de uso bastante común. Siendo el caso más conocido LaTeX.

Programas como LateX tienen la desventaja de obligar a aprender “marcas”. Sin embargo, los algoritmos de colocación del texto suelen ser más sofisticados que otros programas. Los programas de redacción de documentos más utilizados suelen ser los del tipo MS-Word o LibreOffice que son del tipo WYSIWYG (What You See Is What You Get).

Dentro de los tipos de formatos, RTF es un mecanismo público que permite describir el aspecto de documentos. Al ser público, muchos procesadores pueden implementarlo si necesidad de pagar “royalties”.

Adobe lidera la especificación de documentos PDF (Portable Document Format). Antes de
PDF, existía un lenguaje abierto denominado PostScript. Por otro lado RTF o Latex, son lenguajes descriptivos, mientras que PostScript es un lenguaje completo


\section{Organismos de regulación}
\label{\detokenize{tema1:organismos-de-regulacion}}\begin{enumerate}
\item {} 
La Internation Standards Organization emite estándares para la documentación.

\item {} 
El W3C (o World Wide Web Consortium) emite “technical recommendations” (o TR’s) que los interesados en la web pueden seguir para garantizar la interoperabilidad. \sphinxhref{http://www.w3c.org}{Su web es muy útil}

\end{enumerate}


\section{Gramáticas y DTD’s}
\label{\detokenize{tema1:gramaticas-y-dtds}}
Las gramáticas HTML indican el conjunto de reglas para determinar lo que se acepta o no
se acepta. Si se elige una gramática (como por ejemplo la de HTML5) es muy recomendable
respetar las reglas de esa gramática.
A modo de ejemplo, esto es una página válida

\begin{sphinxVerbatim}[commandchars=\\\{\}]
\PYG{c+cp}{\PYGZlt{}!DOCTYPE html\PYGZgt{}}
\PYG{p}{\PYGZlt{}}\PYG{n+nt}{html}\PYG{p}{\PYGZgt{}}
        \PYG{p}{\PYGZlt{}}\PYG{n+nt}{head}\PYG{p}{\PYGZgt{}}
                \PYG{p}{\PYGZlt{}}\PYG{n+nt}{title}\PYG{p}{\PYGZgt{}}Page Title\PYG{p}{\PYGZlt{}}\PYG{p}{/}\PYG{n+nt}{title}\PYG{p}{\PYGZgt{}}
        \PYG{p}{\PYGZlt{}}\PYG{p}{/}\PYG{n+nt}{head}\PYG{p}{\PYGZgt{}}
        \PYG{p}{\PYGZlt{}}\PYG{n+nt}{body}\PYG{p}{\PYGZgt{}}
                Mi primera página
        \PYG{p}{\PYGZlt{}}\PYG{p}{/}\PYG{n+nt}{body}\PYG{p}{\PYGZgt{}}
\PYG{p}{\PYGZlt{}}\PYG{p}{/}\PYG{n+nt}{html}\PYG{p}{\PYGZgt{}}
\end{sphinxVerbatim}

y esto no lo es.

\begin{sphinxVerbatim}[commandchars=\\\{\}]
\PYG{c+cp}{\PYGZlt{}!DOCTYPE html\PYGZgt{}}
        \PYG{p}{\PYGZlt{}}\PYG{n+nt}{title}\PYG{p}{\PYGZgt{}}Esto es el título de la página
\PYG{p}{\PYGZlt{}}\PYG{n+nt}{body}\PYG{p}{\PYGZgt{}}
\end{sphinxVerbatim}

A lo largo del tiempo ha habido diversas versiones de HTML (con sus correspondientes gramáticas)
y tales documentos deben llevar en la cabecera algo que diga a qué estándar se ciñen.

Las tres últimas familias de estándares han sido
\begin{itemize}
\item {} 
HTML4: muy permisivo, lo que dificulta a los navegadores el procesar el HTML dando lugar a que fuera bastante difícil para ellos el mostrar correctamente y de igual forma todos los HTML

\item {} 
XHTML: es HTML con las estrictas reglas que impuso XML. Esto simplificó el desarrollo de navegadores y se avanzó en facilidad para mostrar páginas en distintos navegadores.

\item {} 
HTML5: es una nueva revisión de XHTML en el que se han incluido nuevas posibilidades como etiquetas \textless{}audio\textgreater{} y \textless{}video\textgreater{} así como posibilidad de hacer muchas cosas desde JavaScript.

\end{itemize}

Un ejemplo de DTD, sería esto:

\begin{sphinxVerbatim}[commandchars=\\\{\}]
\PYG{k}{\PYGZlt{}!ELEMENT} \PYG{n+nt}{lista\PYGZus{}de\PYGZus{}personas} \PYG{o}{(}\PYG{n+nt}{persona}\PYG{o}{*}\PYG{o}{)}\PYG{k}{\PYGZgt{}}
\PYG{k}{\PYGZlt{}!ELEMENT} \PYG{n+nt}{persona} \PYG{o}{(}\PYG{n+nt}{nombre}\PYG{o}{,} \PYG{n+nt}{fechanacimiento}\PYG{o}{?}\PYG{o}{,} \PYG{n+nt}{sexo}\PYG{o}{?}\PYG{o}{,} \PYG{n+nt}{numeroseguridadsocial}\PYG{o}{?}\PYG{o}{)}\PYG{k}{\PYGZgt{}}
\PYG{k}{\PYGZlt{}!ELEMENT} \PYG{n+nt}{nombre} \PYG{o}{(}\PYG{k+kc}{\PYGZsh{}PCDATA}\PYG{o}{)} \PYG{k}{\PYGZgt{}}
\PYG{k}{\PYGZlt{}!ELEMENT} \PYG{n+nt}{fechanacimiento} \PYG{o}{(}\PYG{k+kc}{\PYGZsh{}PCDATA}\PYG{o}{)} \PYG{k}{\PYGZgt{}}
\PYG{k}{\PYGZlt{}!ELEMENT} \PYG{n+nt}{sexo} \PYG{o}{(}\PYG{k+kc}{\PYGZsh{}PCDATA}\PYG{o}{)} \PYG{k}{\PYGZgt{}}
\PYG{k}{\PYGZlt{}!ELEMENT} \PYG{n+nt}{numeroseguridadsocial} \PYG{o}{(}\PYG{k+kc}{\PYGZsh{}PCDATA}\PYG{o}{)}\PYG{k}{\PYGZgt{}}
\end{sphinxVerbatim}

Y un ejemplo de archivo aceptado por esa DTD sería este

\begin{sphinxVerbatim}[commandchars=\\\{\}]
\PYG{n+nt}{\PYGZlt{}lista\PYGZus{}de\PYGZus{}personas}\PYG{n+nt}{\PYGZgt{}}
        \PYG{n+nt}{\PYGZlt{}persona}\PYG{n+nt}{\PYGZgt{}}
                \PYG{n+nt}{\PYGZlt{}nombre}\PYG{n+nt}{\PYGZgt{}} Pepe Pérez \PYG{n+nt}{\PYGZlt{}/nombre\PYGZgt{}}
                \PYG{n+nt}{\PYGZlt{}sexo}\PYG{n+nt}{\PYGZgt{}} Varón \PYG{n+nt}{\PYGZlt{}/sexo\PYGZgt{}}
                \PYG{n+nt}{\PYGZlt{}numeroseguridadsocial}\PYG{n+nt}{\PYGZgt{}}555\PYG{n+nt}{\PYGZlt{}/numeroseguridadsocial\PYGZgt{}}
        \PYG{n+nt}{\PYGZlt{}/persona\PYGZgt{}}
        \PYG{n+nt}{\PYGZlt{}persona}\PYG{n+nt}{\PYGZgt{}}
                \PYG{n+nt}{\PYGZlt{}nombre}\PYG{n+nt}{\PYGZgt{}} Angela Lopez \PYG{n+nt}{\PYGZlt{}/nombre\PYGZgt{}}
                \PYG{n+nt}{\PYGZlt{}fechanacimiento}\PYG{n+nt}{\PYGZgt{}}13\PYGZhy{}2\PYGZhy{}1995\PYG{n+nt}{\PYGZlt{}/fechanacimiento\PYGZgt{}}
                \PYG{n+nt}{\PYGZlt{}sexo}\PYG{n+nt}{\PYGZgt{}} Mujer \PYG{n+nt}{\PYGZlt{}/sexo\PYGZgt{}}
                \PYG{n+nt}{\PYGZlt{}numeroseguridadsocial}\PYG{n+nt}{\PYGZgt{}}355\PYG{n+nt}{\PYGZlt{}/numeroseguridadsocial\PYGZgt{}}
        \PYG{n+nt}{\PYGZlt{}/persona\PYGZgt{}}
\PYG{n+nt}{\PYGZlt{}/lista\PYGZus{}de\PYGZus{}personas\PYGZgt{}}
\end{sphinxVerbatim}


\section{XML Schemas}
\label{\detokenize{tema1:xml-schemas}}
Los XML Schemas surgen para mejorar las faltas de precisión que tenían las DTD. Sin embargo,
la mejora en la precisión de la definición ha implicado que escribir XML Schemas sea
mucho más complicado.

Un ejemplo de XML Schema (tomado de Wikipedia):

\begin{sphinxVerbatim}[commandchars=\\\{\}]
\PYG{c+cp}{\PYGZlt{}?xml version=\PYGZdq{}1.0\PYGZdq{} encoding=\PYGZdq{}UTF\PYGZhy{}8\PYGZdq{}?\PYGZgt{}}
        \PYG{n+nt}{\PYGZlt{}xsd:schema} \PYG{n+na}{xmlns:xsd=}\PYG{l+s}{\PYGZdq{}http://www.w3.org/2001/XMLSchema\PYGZdq{}}\PYG{n+nt}{\PYGZgt{}}
                \PYG{n+nt}{\PYGZlt{}xsd:element} \PYG{n+na}{name=}\PYG{l+s}{\PYGZdq{}Libro\PYGZdq{}}\PYG{n+nt}{\PYGZgt{}}
                        \PYG{n+nt}{\PYGZlt{}xsd:complexType}\PYG{n+nt}{\PYGZgt{}}
                                \PYG{n+nt}{\PYGZlt{}xsd:sequence}\PYG{n+nt}{\PYGZgt{}}
                                        \PYG{n+nt}{\PYGZlt{}xsd:element} \PYG{n+na}{name=}\PYG{l+s}{\PYGZdq{}Título\PYGZdq{}}
                                        \PYG{n+na}{type=}\PYG{l+s}{\PYGZdq{}xsd:string\PYGZdq{}}\PYG{n+nt}{/\PYGZgt{}}
                                        \PYG{n+nt}{\PYGZlt{}xsd:element} \PYG{n+na}{name=}\PYG{l+s}{\PYGZdq{}Autores\PYGZdq{}}
                                        \PYG{n+na}{type=}\PYG{l+s}{\PYGZdq{}xsd:string\PYGZdq{}}
                                        \PYG{n+na}{maxOccurs=}\PYG{l+s}{\PYGZdq{}10\PYGZdq{}}\PYG{n+nt}{/\PYGZgt{}}
                                        \PYG{n+nt}{\PYGZlt{}xsd:element}    \PYG{n+na}{name=}\PYG{l+s}{\PYGZdq{}Editorial\PYGZdq{}}
                                        \PYG{n+na}{type=}\PYG{l+s}{\PYGZdq{}xsd:string\PYGZdq{}}\PYG{n+nt}{/\PYGZgt{}}
                                \PYG{n+nt}{\PYGZlt{}/xsd:sequence\PYGZgt{}}
                                \PYG{n+nt}{\PYGZlt{}xsd:attribute} \PYG{n+na}{name=}\PYG{l+s}{\PYGZdq{}precio\PYGZdq{}}
                                \PYG{n+na}{type=}\PYG{l+s}{\PYGZdq{}xsd:double\PYGZdq{}}\PYG{n+nt}{/\PYGZgt{}}
                \PYG{n+nt}{\PYGZlt{}/xsd:complexType\PYGZgt{}}
                \PYG{n+nt}{\PYGZlt{}/xsd:element\PYGZgt{}}
\PYG{n+nt}{\PYGZlt{}/xsd:schema\PYGZgt{}}
\end{sphinxVerbatim}


\section{Definiciones}
\label{\detokenize{tema1:definiciones}}
\sphinxstylestrong{Etiqueta:} Es una secuencia de texto encerrada entre \textless{} y \textgreater{}

\sphinxstylestrong{Elemento:} Es todo lo que va entre una cierta etiqueta de apertura y cierre. En el ejemplo siguiente
si nos hablan de la etiqueta libro se refieren simplemente a la etiqueta entre los
paréntesis angulares. Si hay que procesar el elemento libro esto significa procesar los
sub-elementos o “elementos hijo”.

\sphinxstylestrong{Atributo:} Es un texto junto a la etiqueta que amplía información sobre la misma. En el ejemplo
anterior podemos ver un atributo precio en la etiqueta titulo

\sphinxstylestrong{Árbol del documento:} Todo documento XML y HTML5 puede representarse como un árbol
que se puede recorrer desde distintos lenguajes. Este árbol a veces se llama el árbol DOM
o simplemente el DOM (Document Object Model).

\sphinxstylestrong{Relaciones de parentesco:} En un árbol DOM, los distintos elementos (o nodos). Se dice que
un nodo es hijo de otro si aparece más abajo en el árbol DOM. Se dice que dos nodos
son hermanos si están en el mismo nivel del árbol DOM. Se dice que un nodo es padre
de otro si está en un nivel más arriba en el árbol DOM.


\chapter{HTML5}
\label{\detokenize{tema2::doc}}\label{\detokenize{tema2:html5}}

\section{Introducción}
\label{\detokenize{tema2:introduccion}}
Es la última revisión del estándar HTML. Se incluyen algunas etiquetas nuevas cuyo significado
se comentará a continuación.


\section{Etiquetas estructurales}
\label{\detokenize{tema2:etiquetas-estructurales}}
Crean la estructura para el resto de la página:,
\begin{itemize}
\item {} 
doctype: identifica el estándar.

\item {} 
todo documento debe ir entre las marcas \textless{}html\textgreater{} y \textless{}/html\textgreater{}.

\item {} 
Todo html tiene dos partes: head y body. El primero incluye otros elementos estructurales como \sphinxcode{\textless{}title\textgreater{}} que indica el título de dicha página. Dentro del body se incluye el contenido real de la página.

\item {} 
Existe una etiqueta vital para el correcto visionado de los símbolos de nuestra página. Esta etiqueta se denomina \textless{}meta\textgreater{} y lleva el atributo \sphinxcode{charset="...."}

\item {} 
Dentro del body pueden incluirse otras etiquetas que estructuran el contenido de la página:
\begin{itemize}
\item {} 
La etiqueta \sphinxcode{\textless{}section\textgreater{}} permite marcar contenido de una página relacionado con un tema concreto.

\item {} 
\sphinxcode{\textless{}article\textgreater{}} es una unidad de contenido sobre un tema específico el cual puede ser independiente de otros «artículos».

\item {} 
\sphinxcode{\textless{}header\textgreater{}} se utiliza para indicar cual es la cabecera de un artículo o sección.

\item {} 
\sphinxcode{\textless{}footer\textgreater{}} permite definir un «pie de página», normalmente con indicación de derechos de autor, fecha o datos similares.

\item {} 
\sphinxcode{\textless{}address\textgreater{}} se usa para marcar información de contacto.

\item {} 
\sphinxcode{\textless{}aside\textgreater{}} se usa para definir contenido con «una relación vaga con el resto de la página» (definición tomada del estándar).

\item {} 
\sphinxcode{\textless{}hgroup\textgreater{}} permite agrupar un conjunto de encabezados y marcarlos como pertenecientes al mismo contenido.

\end{itemize}

\item {} 
\sphinxcode{\textless{}h1\textgreater{}}, \sphinxcode{\textless{}h2\textgreater{}}, \sphinxcode{\textless{}h3\textgreater{}}, \sphinxcode{\textless{}h4\textgreater{}}, \sphinxcode{\textless{}h5\textgreater{}} y \sphinxcode{\textless{}h6\textgreater{}} establecen encabezados: trozos de texto que identifican la importancia del siguiente trozo de texto.

\item {} 
Cualquier etiqueta puede ir comentada. Los comentarios no se muestran, son solo de interés para el programador en un futuro. Un comentario se abre con \sphinxcode{\textless{}!-{-}} y se cierra con \sphinxcode{-{-}\textgreater{}}

\item {} 
La etiqueta \sphinxcode{\textless{}nav\textgreater{}} se utilizará para crear barras de navegación.

\item {} 
La etiqueta \sphinxcode{\textless{}aside\textgreater{}} se utiliza para indicar información relacionada con el artículo o texto pero que no tiene porque ser parte del mismo. El ejemplo más común es utilizarlo para publicidad relacionada o texto del tipo «artículos relacionados con este».

\item {} 
La etiqueta \sphinxcode{\textless{}base\textgreater{}} define la URL raíz de toda la página. Permite cambiar fácilmente las URL de los enlaces de una página.

\item {} 
Las etiquetas \sphinxcode{\textless{}script\textgreater{}} y \sphinxcode{\textless{}noscript\textgreater{}} se utilizan para marcar pequeños programas o la ausencia de ellos.

\item {} 
El elemento \sphinxcode{\textless{}main\textgreater{}} indica \sphinxstylestrong{el contenido principal de una página}

\end{itemize}


\section{Etiquetas de formato}
\label{\detokenize{tema2:etiquetas-de-formato}}
Para el formateo elemental de textos se utilizan varias etiquetas:
\begin{itemize}
\item {} 
\sphinxcode{\textless{}b\textgreater{}} Formatea el texto en “negrita”.

\item {} 
\sphinxcode{\textless{}i\textgreater{}} Lo pone en “itálica” (cursiva).

\item {} 
\sphinxcode{\textless{}u\textgreater{}} Subraya el texto.

\item {} 
Las diversas etiquetas se pueden meter unas dentro de otras para obtener efectos como “cursiva, y negrita” o “subrayado y cursiva”, sin embargo las etiquetas deben cerrarse en el orden inverso al que se abrieron.

\item {} 
\sphinxcode{\textless{}sup\textgreater{}} y \sphinxcode{\textless{}sub\textgreater{}} fabrican respectivamente superíndices y subíndices.

\item {} 
\sphinxcode{\textless{}em\textgreater{}} se utiliza para enfatizar un texto.

\item {} 
\sphinxcode{\textless{}p\textgreater{}} Se utiliza para marcar el comienzo y el fin de un párrafo.

\item {} 
La etiqueta \sphinxcode{br} se utiliza para hacer una ruptura en el flujo del texto. Se escribe en forma abreviada \sphinxcode{\textless{}br/\textgreater{}}

\end{itemize}


\section{Gestión de espacios}
\label{\detokenize{tema2:gestion-de-espacios}}
Los navegadores web manejan el espacio de una forma un poco especial:
\begin{itemize}
\item {} 
Si se pone uno o varios espacios en blanco o si se pulsa la tecla ENTER muchas veces el navegador mostrará \sphinxstyleemphasis{un solo espacio en blanco}

\item {} 
Para poner un espacio en blanco horizontal se puede usar la entidad \sphinxcode{\&nbsp;}.

\item {} 
Para hacer un salto de línea se puede usar la etiqueta \sphinxcode{\textless{}br/\textgreater{}} (esta etiqueta no lleva asociada una etiqueta de cierra, es \sphinxstyleemphasis{autocerrada})

\item {} 
Se puede indicar el comienzo y el final de un párrafo con \sphinxcode{\textless{}p\textgreater{}} y \sphinxcode{\textless{}/p\textgreater{}}.

\end{itemize}

Una pregunta habitual es «¿Cuando se debe usar \sphinxcode{\textless{}p\textgreater{}} y cuando \sphinxcode{\textless{}br/\textgreater{}?}. La respuesta es «depende». Una posible respuesta es que si se escriben varios párrafos relacionados es bastante habitual separarlos con \sphinxcode{\textless{}br/\textgreater{}} mientras que si se ponen varios párrafos que hablan de distintas cosas es habitual usar \sphinxcode{\textless{}p\textgreater{}} con cada uno de ellos, sin embargo no hay una respuesta universal.,


\section{Entidades}
\label{\detokenize{tema2:entidades}}
Las entidades HTML permiten escribir determinados símbolos especiales que podrían confundir al navegador, así como otros símbolos que no aparecen directamente en los teclados:
\begin{itemize}
\item {} 
\&lt; y \&gt; representan los símbolos \textless{} y \textgreater{}.

\item {} 
\&copy;

\item {} 
\&trade;

\item {} 
\&reg;

\item {} 
\&euro; y \&yen;

\item {} 
\&amp;

\end{itemize}


\section{Texto preformateado}
\label{\detokenize{tema2:texto-preformateado}}
Algunas marcas, como \sphinxcode{\textless{}pre\textgreater{}} permiten obligar al navegador a que respete los espacios en blanco tal y como aparecen en la página original.

Si se desea indicar que algo debe ser teclado por el usuario se usa la marca \sphinxcode{\textless{}kbd\textgreater{}}.

Si se desea indicar que algo es una variable se puede usar la marca \sphinxcode{\textless{}var\textgreater{}}.

La etiqueta \sphinxcode{\textless{}code\textgreater{}} permite indicar que un determinado es código en un lenguaje de programación.


\section{Listas}
\label{\detokenize{tema2:listas}}
Es una secuencia de elementos relacionados en torno a un mismo concepto Para abrir una lista de elementos se utilizan dos posibles marcas:
\begin{itemize}
\item {} 
\sphinxcode{\textless{}ol\textgreater{}} Para crear una lista ordenada (numerada)

\item {} 
\sphinxcode{\textless{}ul\textgreater{}} Para crear una lista desordenada (no numerada)

\end{itemize}

Una vez creadas hay que etiquetar cada elemento de la lista con la etiqueta \sphinxcode{\textless{}li\textgreater{}}.

En un plano distinto se pueden encontrar las \sphinxstylestrong{listas de definiciones}. Con estos elementos se puede especificar una secuencia de términos para los cuales proporcionamos una definición. Su estructura es la siguiente:
\begin{itemize}
\item {} 
\sphinxcode{\textless{}dl\textgreater{}} y \sphinxcode{\textless{}/dl\textgreater{}} marcan el inicio y el final de la lista de definiciones. Dentro de estas etiquetas pondremos las dos siguientes.

\item {} 
\sphinxcode{\textless{}dt\textgreater{}} y \sphinxcode{\textless{}/dt\textgreater{}} especifican el \sphinxstyleemphasis{término} que vamos a definir.

\item {} 
\sphinxcode{\textless{}dd\textgreater{}} y \sphinxcode{\textless{}/dd\textgreater{}} indican la definición asociada al término anterior.

\end{itemize}

Ejemplo:

\begin{sphinxVerbatim}[commandchars=\\\{\}]
\PYG{p}{\PYGZlt{}}\PYG{n+nt}{dl}\PYG{p}{\PYGZgt{}}
        \PYG{p}{\PYGZlt{}}\PYG{n+nt}{dt}\PYG{p}{\PYGZgt{}}Etiqueta\PYG{p}{\PYGZlt{}}\PYG{p}{/}\PYG{n+nt}{dt}\PYG{p}{\PYGZgt{}}
        \PYG{p}{\PYGZlt{}}\PYG{n+nt}{dd}\PYG{p}{\PYGZgt{}}Todo lo contenido...\PYG{p}{\PYGZlt{}}\PYG{p}{/}\PYG{n+nt}{dd}\PYG{p}{\PYGZgt{}}
        \PYG{p}{\PYGZlt{}}\PYG{n+nt}{dt}\PYG{p}{\PYGZgt{}}Elemento\PYG{p}{\PYGZlt{}}\PYG{p}{/}\PYG{n+nt}{dt}\PYG{p}{\PYGZgt{}}
        \PYG{p}{\PYGZlt{}}\PYG{n+nt}{dd}\PYG{p}{\PYGZgt{}}
                Se define así a todo el árbol
                de nodos comprendido
                entre dos etiquetas
                de apertura y cierre.
        \PYG{p}{\PYGZlt{}}\PYG{p}{/}\PYG{n+nt}{dd}\PYG{p}{\PYGZgt{}}
\PYG{p}{\PYGZlt{}}\PYG{p}{/}\PYG{n+nt}{dl}\PYG{p}{\PYGZgt{}}
\end{sphinxVerbatim}


\subsection{Ejercicio}
\label{\detokenize{tema2:ejercicio}}
Comprueba que el siguiente código HTML crea unas listas dentro de otras. Prueba a crear listas desordenadas dentro de listas desordenadas.

\begin{sphinxVerbatim}[commandchars=\\\{\}]
    \PYG{p}{\PYGZlt{}}\PYG{n+nt}{body}\PYG{p}{\PYGZgt{}}
    Antes de programar
    \PYG{p}{\PYGZlt{}}\PYG{n+nt}{ol}\PYG{p}{\PYGZgt{}}
        \PYG{p}{\PYGZlt{}}\PYG{n+nt}{li}\PYG{p}{\PYGZgt{}}
            Instalar JDK
            \PYG{p}{\PYGZlt{}}\PYG{n+nt}{ol}\PYG{p}{\PYGZgt{}}
                \PYG{p}{\PYGZlt{}}\PYG{n+nt}{li}\PYG{p}{\PYGZgt{}}Ir a oracle.com\PYG{p}{\PYGZlt{}}\PYG{p}{/}\PYG{n+nt}{li}\PYG{p}{\PYGZgt{}}
                \PYG{p}{\PYGZlt{}}\PYG{n+nt}{li}\PYG{p}{\PYGZgt{}}Buscar JDK\PYG{p}{\PYGZlt{}}\PYG{p}{/}\PYG{n+nt}{li}\PYG{p}{\PYGZgt{}}
                \PYG{p}{\PYGZlt{}}\PYG{n+nt}{li}\PYG{p}{\PYGZgt{}}Aceptar licencia\PYG{p}{\PYGZlt{}}\PYG{p}{/}\PYG{n+nt}{li}\PYG{p}{\PYGZgt{}}
                \PYG{p}{\PYGZlt{}}\PYG{n+nt}{li}\PYG{p}{\PYGZgt{}}Descargar\PYG{p}{\PYGZlt{}}\PYG{p}{/}\PYG{n+nt}{li}\PYG{p}{\PYGZgt{}}
                \PYG{p}{\PYGZlt{}}\PYG{n+nt}{li}\PYG{p}{\PYGZgt{}}
                    Ejecutar setup.exe
                    \PYG{p}{\PYGZlt{}}\PYG{n+nt}{ol}\PYG{p}{\PYGZgt{}}
                        \PYG{p}{\PYGZlt{}}\PYG{n+nt}{li}\PYG{p}{\PYGZgt{}}
                            Ejecutar como
                            admin
                        \PYG{p}{\PYGZlt{}}\PYG{p}{/}\PYG{n+nt}{li}\PYG{p}{\PYGZgt{}}
                        \PYG{p}{\PYGZlt{}}\PYG{n+nt}{li}\PYG{p}{\PYGZgt{}}Comprobar\PYG{p}{\PYGZlt{}}\PYG{p}{/}\PYG{n+nt}{li}\PYG{p}{\PYGZgt{}}
                    \PYG{p}{\PYGZlt{}}\PYG{p}{/}\PYG{n+nt}{ol}\PYG{p}{\PYGZgt{}}
                \PYG{p}{\PYGZlt{}}\PYG{p}{/}\PYG{n+nt}{li}\PYG{p}{\PYGZgt{}}
            \PYG{p}{\PYGZlt{}}\PYG{p}{/}\PYG{n+nt}{ol}\PYG{p}{\PYGZgt{}}
        \PYG{p}{\PYGZlt{}}\PYG{p}{/}\PYG{n+nt}{li}\PYG{p}{\PYGZgt{}}
        \PYG{p}{\PYGZlt{}}\PYG{n+nt}{li}\PYG{p}{\PYGZgt{}}Modificar variables de entorno\PYG{p}{\PYGZlt{}}\PYG{p}{/}\PYG{n+nt}{li}\PYG{p}{\PYGZgt{}}
        \PYG{p}{\PYGZlt{}}\PYG{n+nt}{li}\PYG{p}{\PYGZgt{}}Asignar más memoria\PYG{p}{\PYGZlt{}}\PYG{p}{/}\PYG{n+nt}{li}\PYG{p}{\PYGZgt{}}
        \PYG{p}{\PYGZlt{}}\PYG{n+nt}{li}\PYG{p}{\PYGZgt{}}Reiniciar\PYG{p}{\PYGZlt{}}\PYG{p}{/}\PYG{n+nt}{li}\PYG{p}{\PYGZgt{}}
    \PYG{p}{\PYGZlt{}}\PYG{p}{/}\PYG{n+nt}{ol}\PYG{p}{\PYGZgt{}}
    Prerrequisitos
    \PYG{p}{\PYGZlt{}}\PYG{n+nt}{ul}\PYG{p}{\PYGZgt{}}
        \PYG{p}{\PYGZlt{}}\PYG{n+nt}{li}\PYG{p}{\PYGZgt{}}Comprobar RAM\PYG{p}{\PYGZlt{}}\PYG{p}{/}\PYG{n+nt}{li}\PYG{p}{\PYGZgt{}}
        \PYG{p}{\PYGZlt{}}\PYG{n+nt}{li}\PYG{p}{\PYGZgt{}}Comprobar disco\PYG{p}{\PYGZlt{}}\PYG{p}{/}\PYG{n+nt}{li}\PYG{p}{\PYGZgt{}}
        \PYG{p}{\PYGZlt{}}\PYG{n+nt}{li}\PYG{p}{\PYGZgt{}}Comprobar arranque\PYG{p}{\PYGZlt{}}\PYG{p}{/}\PYG{n+nt}{li}\PYG{p}{\PYGZgt{}}
    \PYG{p}{\PYGZlt{}}\PYG{p}{/}\PYG{n+nt}{ul}\PYG{p}{\PYGZgt{}}
\PYG{p}{\PYGZlt{}}\PYG{p}{/}\PYG{n+nt}{body}\PYG{p}{\PYGZgt{}}
\end{sphinxVerbatim}


\section{Tablas}
\label{\detokenize{tema2:tablas}}
Una tabla muestra un conjunto de elementos relacionados en forma de matriz. No deberían usarse para maquetar la posición de los elementos. Todo el contenido de la tabla debe ir entre las etiquetas \sphinxcode{\textless{}table\textgreater{}} y \sphinxcode{\textless{}/table\textgreater{}}. Las tablas se construyen de izquierda a derecha (por columnas) y de arriba a abajo (filas).

Una tabla puede tener una cabecera, un cuerpo y un pie, especificados por \sphinxcode{\textless{}thead\textgreater{}}, \sphinxcode{\textless{}tbody\textgreater{}} y \sphinxcode{\textless{}tfoot\textgreater{}}. La primera etiqueta dentro de \sphinxcode{\textless{}tbody\textgreater{}}, solo puede ser \textless{}tr\textgreater{}. \sphinxstylestrong{Cuidado al crear tablas, todo dato, o subtablas debe ir dentro de \textless{}td\textgreater{}, es absolutamente obligatorio}

Para ser exactos una tabla puede llevar estas tres etiquetas:
\begin{itemize}
\item {} 
\sphinxcode{thead}: dentro de ella a su vez pondremos una fila (\sphinxcode{\textless{}tr\textgreater{}}) con celdas en las que la etiqueta es \sphinxcode{\textless{}th\textgreater{}}

\item {} 
\sphinxcode{tbody}: utiliza las filas y columnas normales.

\item {} 
\sphinxcode{tfooter}: también usa \sphinxcode{\textless{}tr\textgreater{}} y \sphinxcode{\textless{}td\textgreater{}} de la forma habitual, sin embargo permite describir mejor el contenido de la tabla. Se utiliza para celdas con los valores acumulados o similares.

\end{itemize}


\subsection{Un ejemplo de tabla}
\label{\detokenize{tema2:un-ejemplo-de-tabla}}
Se desea crear una tabla que represente los datos del medallero de unas olimpiadas y que se muestre de forma parecida a lo que muestra la figura:

\noindent{\hspace*{\fill}\sphinxincludegraphics[scale=0.5]{{tablamedallero}.png}\hspace*{\fill}}

\begin{sphinxVerbatim}[commandchars=\\\{\}]
\PYG{p}{\PYGZlt{}}\PYG{n+nt}{table} \PYG{n+na}{border}\PYG{o}{=}\PYG{l+s}{\PYGZdq{}1\PYGZdq{}}\PYG{p}{\PYGZgt{}}
    \PYG{p}{\PYGZlt{}}\PYG{n+nt}{thead}\PYG{p}{\PYGZgt{}}
        \PYG{p}{\PYGZlt{}}\PYG{n+nt}{tr}\PYG{p}{\PYGZgt{}}
            \PYG{p}{\PYGZlt{}}\PYG{n+nt}{th}\PYG{p}{\PYGZgt{}}País\PYG{p}{\PYGZlt{}}\PYG{p}{/}\PYG{n+nt}{th}\PYG{p}{\PYGZgt{}}
            \PYG{p}{\PYGZlt{}}\PYG{n+nt}{th}\PYG{p}{\PYGZgt{}}Oro\PYG{p}{\PYGZlt{}}\PYG{p}{/}\PYG{n+nt}{th}\PYG{p}{\PYGZgt{}}
            \PYG{p}{\PYGZlt{}}\PYG{n+nt}{th}\PYG{p}{\PYGZgt{}}Plata\PYG{p}{\PYGZlt{}}\PYG{p}{/}\PYG{n+nt}{th}\PYG{p}{\PYGZgt{}}
            \PYG{p}{\PYGZlt{}}\PYG{n+nt}{th}\PYG{p}{\PYGZgt{}}Bronce\PYG{p}{\PYGZlt{}}\PYG{p}{/}\PYG{n+nt}{th}\PYG{p}{\PYGZgt{}}
        \PYG{p}{\PYGZlt{}}\PYG{p}{/}\PYG{n+nt}{tr}\PYG{p}{\PYGZgt{}}
    \PYG{p}{\PYGZlt{}}\PYG{p}{/}\PYG{n+nt}{thead}\PYG{p}{\PYGZgt{}}
    \PYG{p}{\PYGZlt{}}\PYG{n+nt}{tbody}\PYG{p}{\PYGZgt{}}
        \PYG{p}{\PYGZlt{}}\PYG{n+nt}{tr}\PYG{p}{\PYGZgt{}}
            \PYG{p}{\PYGZlt{}}\PYG{n+nt}{td}\PYG{p}{\PYGZgt{}}USA\PYG{p}{\PYGZlt{}}\PYG{p}{/}\PYG{n+nt}{td}\PYG{p}{\PYGZgt{}}
            \PYG{p}{\PYGZlt{}}\PYG{n+nt}{td}\PYG{p}{\PYGZgt{}}110\PYG{p}{\PYGZlt{}}\PYG{p}{/}\PYG{n+nt}{td}\PYG{p}{\PYGZgt{}}
            \PYG{p}{\PYGZlt{}}\PYG{n+nt}{td}\PYG{p}{\PYGZgt{}}115\PYG{p}{\PYGZlt{}}\PYG{p}{/}\PYG{n+nt}{td}\PYG{p}{\PYGZgt{}}
            \PYG{p}{\PYGZlt{}}\PYG{n+nt}{td}\PYG{p}{\PYGZgt{}}99\PYG{p}{\PYGZlt{}}\PYG{p}{/}\PYG{n+nt}{td}\PYG{p}{\PYGZgt{}}
        \PYG{p}{\PYGZlt{}}\PYG{p}{/}\PYG{n+nt}{tr}\PYG{p}{\PYGZgt{}}
    \PYG{p}{\PYGZlt{}}\PYG{p}{/}\PYG{n+nt}{tbody}\PYG{p}{\PYGZgt{}}
    \PYG{p}{\PYGZlt{}}\PYG{n+nt}{tfoot}\PYG{p}{\PYGZgt{}}
        \PYG{p}{\PYGZlt{}}\PYG{n+nt}{tr}\PYG{p}{\PYGZgt{}}
            \PYG{p}{\PYGZlt{}}\PYG{n+nt}{td}\PYG{p}{\PYGZgt{}}Total\PYG{p}{\PYGZlt{}}\PYG{p}{/}\PYG{n+nt}{td}\PYG{p}{\PYGZgt{}}
            \PYG{p}{\PYGZlt{}}\PYG{n+nt}{td}\PYG{p}{\PYGZgt{}}219\PYG{p}{\PYGZlt{}}\PYG{p}{/}\PYG{n+nt}{td}\PYG{p}{\PYGZgt{}}
            \PYG{p}{\PYGZlt{}}\PYG{n+nt}{td}\PYG{p}{\PYGZgt{}}247\PYG{p}{\PYGZlt{}}\PYG{p}{/}\PYG{n+nt}{td}\PYG{p}{\PYGZgt{}}
            \PYG{p}{\PYGZlt{}}\PYG{n+nt}{td}\PYG{p}{\PYGZgt{}}206\PYG{p}{\PYGZlt{}}\PYG{p}{/}\PYG{n+nt}{td}\PYG{p}{\PYGZgt{}}
        \PYG{p}{\PYGZlt{}}\PYG{p}{/}\PYG{n+nt}{tr}\PYG{p}{\PYGZgt{}}
    \PYG{p}{\PYGZlt{}}\PYG{p}{/}\PYG{n+nt}{tfoot}\PYG{p}{\PYGZgt{}}
\PYG{p}{\PYGZlt{}}\PYG{p}{/}\PYG{n+nt}{table}\PYG{p}{\PYGZgt{}}
\end{sphinxVerbatim}


\subsection{Ejercicio sobre tablas}
\label{\detokenize{tema2:ejercicio-sobre-tablas}}
Crea una tabla con la estructura siguiente:

\noindent{\hspace*{\fill}\sphinxincludegraphics[scale=0.5]{{tabla1}.png}\hspace*{\fill}}


\subsection{Solución}
\label{\detokenize{tema2:solucion}}
Un posible HTML que resuelve esto sería:

\begin{sphinxVerbatim}[commandchars=\\\{\}]
\PYG{p}{\PYGZlt{}}\PYG{n+nt}{table} \PYG{n+na}{border}\PYG{o}{=}\PYG{l+s}{\PYGZdq{}1\PYGZdq{}}\PYG{p}{\PYGZgt{}}
        \PYG{p}{\PYGZlt{}}\PYG{n+nt}{tbody}\PYG{p}{\PYGZgt{}}
                \PYG{p}{\PYGZlt{}}\PYG{n+nt}{tr}\PYG{p}{\PYGZgt{}}
                        \PYG{p}{\PYGZlt{}}\PYG{n+nt}{td}\PYG{p}{\PYGZgt{}}A\PYG{p}{\PYGZlt{}}\PYG{p}{/}\PYG{n+nt}{td}\PYG{p}{\PYGZgt{}}\PYG{p}{\PYGZlt{}}\PYG{n+nt}{td}\PYG{p}{\PYGZgt{}}B\PYG{p}{\PYGZlt{}}\PYG{p}{/}\PYG{n+nt}{td}\PYG{p}{\PYGZgt{}}
                \PYG{p}{\PYGZlt{}}\PYG{p}{/}\PYG{n+nt}{tr}\PYG{p}{\PYGZgt{}}
                \PYG{p}{\PYGZlt{}}\PYG{n+nt}{tr}\PYG{p}{\PYGZgt{}}
                        \PYG{p}{\PYGZlt{}}\PYG{n+nt}{td}\PYG{p}{\PYGZgt{}}C\PYG{p}{\PYGZlt{}}\PYG{p}{/}\PYG{n+nt}{td}\PYG{p}{\PYGZgt{}}
                        \PYG{p}{\PYGZlt{}}\PYG{n+nt}{td}\PYG{p}{\PYGZgt{}}
                                \PYG{p}{\PYGZlt{}}\PYG{n+nt}{table}\PYG{p}{\PYGZgt{}}
                                        \PYG{p}{\PYGZlt{}}\PYG{n+nt}{tbody}\PYG{p}{\PYGZgt{}}
                                                \PYG{p}{\PYGZlt{}}\PYG{n+nt}{tr}\PYG{p}{\PYGZgt{}}\PYG{p}{\PYGZlt{}}\PYG{n+nt}{td}\PYG{p}{\PYGZgt{}}D1\PYG{p}{\PYGZlt{}}\PYG{p}{/}\PYG{n+nt}{td}\PYG{p}{\PYGZgt{}}\PYG{p}{\PYGZlt{}}\PYG{p}{/}\PYG{n+nt}{tr}\PYG{p}{\PYGZgt{}}
                                                \PYG{p}{\PYGZlt{}}\PYG{n+nt}{tr}\PYG{p}{\PYGZgt{}}\PYG{p}{\PYGZlt{}}\PYG{n+nt}{td}\PYG{p}{\PYGZgt{}}D2\PYG{p}{\PYGZlt{}}\PYG{p}{/}\PYG{n+nt}{td}\PYG{p}{\PYGZgt{}}\PYG{p}{\PYGZlt{}}\PYG{p}{/}\PYG{n+nt}{tr}\PYG{p}{\PYGZgt{}}
                                                \PYG{p}{\PYGZlt{}}\PYG{n+nt}{tr}\PYG{p}{\PYGZgt{}}\PYG{p}{\PYGZlt{}}\PYG{n+nt}{td}\PYG{p}{\PYGZgt{}}D3\PYG{p}{\PYGZlt{}}\PYG{p}{/}\PYG{n+nt}{td}\PYG{p}{\PYGZgt{}}\PYG{p}{\PYGZlt{}}\PYG{p}{/}\PYG{n+nt}{tr}\PYG{p}{\PYGZgt{}}
                                                \PYG{p}{\PYGZlt{}}\PYG{n+nt}{tr}\PYG{p}{\PYGZgt{}}\PYG{p}{\PYGZlt{}}\PYG{n+nt}{td}\PYG{p}{\PYGZgt{}}D4\PYG{p}{\PYGZlt{}}\PYG{p}{/}\PYG{n+nt}{td}\PYG{p}{\PYGZgt{}}\PYG{p}{\PYGZlt{}}\PYG{p}{/}\PYG{n+nt}{tr}\PYG{p}{\PYGZgt{}}
                                        \PYG{p}{\PYGZlt{}}\PYG{p}{/}\PYG{n+nt}{tbody}\PYG{p}{\PYGZgt{}}
                                \PYG{p}{\PYGZlt{}}\PYG{p}{/}\PYG{n+nt}{table}\PYG{p}{\PYGZgt{}}
                        \PYG{p}{\PYGZlt{}}\PYG{p}{/}\PYG{n+nt}{td}\PYG{p}{\PYGZgt{}}
                \PYG{p}{\PYGZlt{}}\PYG{p}{/}\PYG{n+nt}{tr}\PYG{p}{\PYGZgt{}}
        \PYG{p}{\PYGZlt{}}\PYG{p}{/}\PYG{n+nt}{tbody}\PYG{p}{\PYGZgt{}}
\PYG{p}{\PYGZlt{}}\PYG{p}{/}\PYG{n+nt}{table}\PYG{p}{\PYGZgt{}}
\end{sphinxVerbatim}


\subsection{Ejercicio sobre tablas (II)}
\label{\detokenize{tema2:ejercicio-sobre-tablas-ii}}
Crea una tabla con la estructura siguiente:

\noindent{\hspace*{\fill}\sphinxincludegraphics[scale=0.5]{{tabla2}.png}\hspace*{\fill}}


\subsection{Solución}
\label{\detokenize{tema2:id1}}
Un posible HTML que resuelve esto sería:

\begin{sphinxVerbatim}[commandchars=\\\{\}]
\PYG{p}{\PYGZlt{}}\PYG{n+nt}{table} \PYG{n+na}{border}\PYG{o}{=}\PYG{l+s}{\PYGZdq{}1\PYGZdq{}}\PYG{p}{\PYGZgt{}}
        \PYG{p}{\PYGZlt{}}\PYG{n+nt}{tbody}\PYG{p}{\PYGZgt{}}
                \PYG{p}{\PYGZlt{}}\PYG{n+nt}{tr}\PYG{p}{\PYGZgt{}}\PYG{p}{\PYGZlt{}}\PYG{n+nt}{td}\PYG{p}{\PYGZgt{}}A\PYG{p}{\PYGZlt{}}\PYG{p}{/}\PYG{n+nt}{td}\PYG{p}{\PYGZgt{}}\PYG{p}{\PYGZlt{}}\PYG{p}{/}\PYG{n+nt}{tr}\PYG{p}{\PYGZgt{}}
                \PYG{p}{\PYGZlt{}}\PYG{n+nt}{tr}\PYG{p}{\PYGZgt{}}
                        \PYG{p}{\PYGZlt{}}\PYG{n+nt}{td}\PYG{p}{\PYGZgt{}}
                                \PYG{p}{\PYGZlt{}}\PYG{n+nt}{table}\PYG{p}{\PYGZgt{}}
                                        \PYG{p}{\PYGZlt{}}\PYG{n+nt}{tr}\PYG{p}{\PYGZgt{}}
                                                \PYG{p}{\PYGZlt{}}\PYG{n+nt}{td}\PYG{p}{\PYGZgt{}}
                                                        \PYG{p}{\PYGZlt{}}\PYG{n+nt}{table} \PYG{n+na}{border}\PYG{o}{=}\PYG{l+s}{\PYGZdq{}1\PYGZdq{}}\PYG{p}{\PYGZgt{}}
                                                                \PYG{p}{\PYGZlt{}}\PYG{n+nt}{tbody}\PYG{p}{\PYGZgt{}}
                                                                        \PYG{p}{\PYGZlt{}}\PYG{n+nt}{tr}\PYG{p}{\PYGZgt{}}
                                                                                \PYG{p}{\PYGZlt{}}\PYG{n+nt}{td}\PYG{p}{\PYGZgt{}}B1\PYG{p}{\PYGZlt{}}\PYG{p}{/}\PYG{n+nt}{td}\PYG{p}{\PYGZgt{}}
                                                                        \PYG{p}{\PYGZlt{}}\PYG{p}{/}\PYG{n+nt}{tr}\PYG{p}{\PYGZgt{}}
                                                                        \PYG{p}{\PYGZlt{}}\PYG{n+nt}{tr}\PYG{p}{\PYGZgt{}}
                                                                                \PYG{p}{\PYGZlt{}}\PYG{n+nt}{td}\PYG{p}{\PYGZgt{}}B2\PYG{p}{\PYGZlt{}}\PYG{p}{/}\PYG{n+nt}{td}\PYG{p}{\PYGZgt{}}
                                                                        \PYG{p}{\PYGZlt{}}\PYG{p}{/}\PYG{n+nt}{tr}\PYG{p}{\PYGZgt{}}
                                                                        \PYG{p}{\PYGZlt{}}\PYG{n+nt}{tr}\PYG{p}{\PYGZgt{}}
                                                                                \PYG{p}{\PYGZlt{}}\PYG{n+nt}{td}\PYG{p}{\PYGZgt{}}B3\PYG{p}{\PYGZlt{}}\PYG{p}{/}\PYG{n+nt}{td}\PYG{p}{\PYGZgt{}}
                                                                        \PYG{p}{\PYGZlt{}}\PYG{p}{/}\PYG{n+nt}{tr}\PYG{p}{\PYGZgt{}}
                                                                \PYG{p}{\PYGZlt{}}\PYG{p}{/}\PYG{n+nt}{tbody}\PYG{p}{\PYGZgt{}}
                                                        \PYG{p}{\PYGZlt{}}\PYG{p}{/}\PYG{n+nt}{table}\PYG{p}{\PYGZgt{}}
                                                \PYG{p}{\PYGZlt{}}\PYG{p}{/}\PYG{n+nt}{td}\PYG{p}{\PYGZgt{}}
                                                \PYG{p}{\PYGZlt{}}\PYG{n+nt}{td}\PYG{p}{\PYGZgt{}}
                                                        \PYG{p}{\PYGZlt{}}\PYG{n+nt}{table} \PYG{n+na}{border}\PYG{o}{=}\PYG{l+s}{\PYGZdq{}1\PYGZdq{}}\PYG{p}{\PYGZgt{}}
                                                                \PYG{p}{\PYGZlt{}}\PYG{n+nt}{tbody}\PYG{p}{\PYGZgt{}}
                                                                        \PYG{p}{\PYGZlt{}}\PYG{n+nt}{tr}\PYG{p}{\PYGZgt{}}
                                                                                \PYG{p}{\PYGZlt{}}\PYG{n+nt}{td}\PYG{p}{\PYGZgt{}}C1\PYG{p}{\PYGZlt{}}\PYG{p}{/}\PYG{n+nt}{td}\PYG{p}{\PYGZgt{}}
                                                                                \PYG{p}{\PYGZlt{}}\PYG{n+nt}{td}\PYG{p}{\PYGZgt{}}C2\PYG{p}{\PYGZlt{}}\PYG{p}{/}\PYG{n+nt}{td}\PYG{p}{\PYGZgt{}}
                                                                        \PYG{p}{\PYGZlt{}}\PYG{p}{/}\PYG{n+nt}{tr}\PYG{p}{\PYGZgt{}}
                                                                \PYG{p}{\PYGZlt{}}\PYG{p}{/}\PYG{n+nt}{tbody}\PYG{p}{\PYGZgt{}}
                                                        \PYG{p}{\PYGZlt{}}\PYG{p}{/}\PYG{n+nt}{table}\PYG{p}{\PYGZgt{}}
                                                \PYG{p}{\PYGZlt{}}\PYG{p}{/}\PYG{n+nt}{td}\PYG{p}{\PYGZgt{}}
                                        \PYG{p}{\PYGZlt{}}\PYG{p}{/}\PYG{n+nt}{tr}\PYG{p}{\PYGZgt{}}
                                \PYG{p}{\PYGZlt{}}\PYG{p}{/}\PYG{n+nt}{table}\PYG{p}{\PYGZgt{}}
                        \PYG{p}{\PYGZlt{}}\PYG{p}{/}\PYG{n+nt}{td}\PYG{p}{\PYGZgt{}}
                \PYG{p}{\PYGZlt{}}\PYG{p}{/}\PYG{n+nt}{tr}\PYG{p}{\PYGZgt{}}
        \PYG{p}{\PYGZlt{}}\PYG{p}{/}\PYG{n+nt}{tbody}\PYG{p}{\PYGZgt{}}
\PYG{p}{\PYGZlt{}}\PYG{p}{/}\PYG{n+nt}{table}\PYG{p}{\PYGZgt{}}
\end{sphinxVerbatim}


\subsection{Ejercicio sobre tablas (III)}
\label{\detokenize{tema2:ejercicio-sobre-tablas-iii}}
Crea una tabla con la estructura siguiente:

\noindent{\hspace*{\fill}\sphinxincludegraphics[scale=0.5]{{tabla3}.png}\hspace*{\fill}}


\subsection{Solución}
\label{\detokenize{tema2:id2}}
Un posible HTML que resuelve esto sería:

\begin{sphinxVerbatim}[commandchars=\\\{\}]
\PYG{c+cp}{\PYGZlt{}!DOCTYPE html\PYGZgt{}}

\PYG{p}{\PYGZlt{}}\PYG{n+nt}{html}\PYG{p}{\PYGZgt{}}
\PYG{p}{\PYGZlt{}}\PYG{n+nt}{head}\PYG{p}{\PYGZgt{}}
    \PYG{p}{\PYGZlt{}}\PYG{n+nt}{title}\PYG{p}{\PYGZgt{}}Solución al ejercicio propuesto el viernes\PYG{p}{\PYGZlt{}}\PYG{p}{/}\PYG{n+nt}{title}\PYG{p}{\PYGZgt{}}
    \PYG{p}{\PYGZlt{}}\PYG{n+nt}{meta} \PYG{n+na}{charset}\PYG{o}{=}\PYG{l+s}{\PYGZdq{}utf\PYGZhy{}8\PYGZdq{}}\PYG{p}{\PYGZgt{}}
\PYG{p}{\PYGZlt{}}\PYG{p}{/}\PYG{n+nt}{head}\PYG{p}{\PYGZgt{}}

\PYG{p}{\PYGZlt{}}\PYG{n+nt}{body}\PYG{p}{\PYGZgt{}}
\PYG{p}{\PYGZlt{}}\PYG{n+nt}{table} \PYG{n+na}{border}\PYG{o}{=}\PYG{l+s}{\PYGZdq{}1\PYGZdq{}}\PYG{p}{\PYGZgt{}}
    \PYG{p}{\PYGZlt{}}\PYG{n+nt}{tbody}\PYG{p}{\PYGZgt{}}
        \PYG{p}{\PYGZlt{}}\PYG{n+nt}{tr}\PYG{p}{\PYGZgt{}}
            \PYG{p}{\PYGZlt{}}\PYG{n+nt}{td}\PYG{p}{\PYGZgt{}}
                \PYG{p}{\PYGZlt{}}\PYG{n+nt}{table} \PYG{n+na}{border}\PYG{o}{=}\PYG{l+s}{\PYGZdq{}1\PYGZdq{}}\PYG{p}{\PYGZgt{}}
                    \PYG{p}{\PYGZlt{}}\PYG{n+nt}{tbody}\PYG{p}{\PYGZgt{}}
                        \PYG{p}{\PYGZlt{}}\PYG{n+nt}{tr}\PYG{p}{\PYGZgt{}}\PYG{p}{\PYGZlt{}}\PYG{n+nt}{td}\PYG{p}{\PYGZgt{}}A1\PYG{p}{\PYGZlt{}}\PYG{p}{/}\PYG{n+nt}{td}\PYG{p}{\PYGZgt{}}\PYG{p}{\PYGZlt{}}\PYG{p}{/}\PYG{n+nt}{tr}\PYG{p}{\PYGZgt{}}
                        \PYG{p}{\PYGZlt{}}\PYG{n+nt}{tr}\PYG{p}{\PYGZgt{}}\PYG{p}{\PYGZlt{}}\PYG{n+nt}{td}\PYG{p}{\PYGZgt{}}A2\PYG{p}{\PYGZlt{}}\PYG{p}{/}\PYG{n+nt}{td}\PYG{p}{\PYGZgt{}}\PYG{p}{\PYGZlt{}}\PYG{p}{/}\PYG{n+nt}{tr}\PYG{p}{\PYGZgt{}}
                    \PYG{p}{\PYGZlt{}}\PYG{p}{/}\PYG{n+nt}{tbody}\PYG{p}{\PYGZgt{}}
                \PYG{p}{\PYGZlt{}}\PYG{p}{/}\PYG{n+nt}{table}\PYG{p}{\PYGZgt{}}
            \PYG{p}{\PYGZlt{}}\PYG{p}{/}\PYG{n+nt}{td}\PYG{p}{\PYGZgt{}}
            \PYG{p}{\PYGZlt{}}\PYG{n+nt}{td}\PYG{p}{\PYGZgt{}}
                \PYG{p}{\PYGZlt{}}\PYG{n+nt}{table} \PYG{n+na}{border}\PYG{o}{=}\PYG{l+s}{\PYGZdq{}1\PYGZdq{}}\PYG{p}{\PYGZgt{}}
                    \PYG{p}{\PYGZlt{}}\PYG{n+nt}{tbody}\PYG{p}{\PYGZgt{}}
                        \PYG{p}{\PYGZlt{}}\PYG{n+nt}{tr}\PYG{p}{\PYGZgt{}}
                            \PYG{p}{\PYGZlt{}}\PYG{n+nt}{td}\PYG{p}{\PYGZgt{}}B1\PYG{p}{\PYGZlt{}}\PYG{p}{/}\PYG{n+nt}{td}\PYG{p}{\PYGZgt{}}
                            \PYG{p}{\PYGZlt{}}\PYG{n+nt}{td}\PYG{p}{\PYGZgt{}}
                                \PYG{p}{\PYGZlt{}}\PYG{n+nt}{table} \PYG{n+na}{border}\PYG{o}{=}\PYG{l+s}{\PYGZdq{}1\PYGZdq{}}\PYG{p}{\PYGZgt{}}
                                    \PYG{p}{\PYGZlt{}}\PYG{n+nt}{tr}\PYG{p}{\PYGZgt{}}\PYG{p}{\PYGZlt{}}\PYG{n+nt}{td}\PYG{p}{\PYGZgt{}}B1\PYGZhy{}1\PYG{p}{\PYGZlt{}}\PYG{p}{/}\PYG{n+nt}{td}\PYG{p}{\PYGZgt{}}\PYG{p}{\PYGZlt{}}\PYG{p}{/}\PYG{n+nt}{tr}\PYG{p}{\PYGZgt{}}
                                    \PYG{p}{\PYGZlt{}}\PYG{n+nt}{tr}\PYG{p}{\PYGZgt{}}\PYG{p}{\PYGZlt{}}\PYG{n+nt}{td}\PYG{p}{\PYGZgt{}}B1\PYGZhy{}2\PYG{p}{\PYGZlt{}}\PYG{p}{/}\PYG{n+nt}{td}\PYG{p}{\PYGZgt{}}\PYG{p}{\PYGZlt{}}\PYG{p}{/}\PYG{n+nt}{tr}\PYG{p}{\PYGZgt{}}
                                \PYG{p}{\PYGZlt{}}\PYG{p}{/}\PYG{n+nt}{table}\PYG{p}{\PYGZgt{}}
                            \PYG{p}{\PYGZlt{}}\PYG{p}{/}\PYG{n+nt}{td}\PYG{p}{\PYGZgt{}}
                        \PYG{p}{\PYGZlt{}}\PYG{p}{/}\PYG{n+nt}{tr}\PYG{p}{\PYGZgt{}}
                    \PYG{p}{\PYGZlt{}}\PYG{p}{/}\PYG{n+nt}{tbody}\PYG{p}{\PYGZgt{}}
                \PYG{p}{\PYGZlt{}}\PYG{p}{/}\PYG{n+nt}{table}\PYG{p}{\PYGZgt{}}
            \PYG{p}{\PYGZlt{}}\PYG{p}{/}\PYG{n+nt}{td}\PYG{p}{\PYGZgt{}}
        \PYG{p}{\PYGZlt{}}\PYG{p}{/}\PYG{n+nt}{tr}\PYG{p}{\PYGZgt{}}
        \PYG{p}{\PYGZlt{}}\PYG{n+nt}{tr}\PYG{p}{\PYGZgt{}}
            \PYG{p}{\PYGZlt{}}\PYG{n+nt}{td}\PYG{p}{\PYGZgt{}}
                \PYG{p}{\PYGZlt{}}\PYG{n+nt}{table}\PYG{p}{\PYGZgt{}}
                    \PYG{p}{\PYGZlt{}}\PYG{n+nt}{tbody}\PYG{p}{\PYGZgt{}}
                        \PYG{p}{\PYGZlt{}}\PYG{n+nt}{tr}\PYG{p}{\PYGZgt{}}
                            \PYG{p}{\PYGZlt{}}\PYG{n+nt}{td}\PYG{p}{\PYGZgt{}}C1\PYG{p}{\PYGZlt{}}\PYG{p}{/}\PYG{n+nt}{td}\PYG{p}{\PYGZgt{}}
                            \PYG{p}{\PYGZlt{}}\PYG{n+nt}{td}\PYG{p}{\PYGZgt{}}
                                \PYG{p}{\PYGZlt{}}\PYG{n+nt}{table} \PYG{n+na}{border}\PYG{o}{=}\PYG{l+s}{\PYGZdq{}1\PYGZdq{}}\PYG{p}{\PYGZgt{}}
                                    \PYG{p}{\PYGZlt{}}\PYG{n+nt}{tbody}\PYG{p}{\PYGZgt{}}
                                        \PYG{p}{\PYGZlt{}}\PYG{n+nt}{tr}\PYG{p}{\PYGZgt{}}\PYG{p}{\PYGZlt{}}\PYG{n+nt}{td}\PYG{p}{\PYGZgt{}}C1\PYGZhy{}1\PYG{p}{\PYGZlt{}}\PYG{p}{/}\PYG{n+nt}{td}\PYG{p}{\PYGZgt{}}\PYG{p}{\PYGZlt{}}\PYG{p}{/}\PYG{n+nt}{tr}\PYG{p}{\PYGZgt{}}
                                        \PYG{p}{\PYGZlt{}}\PYG{n+nt}{tr}\PYG{p}{\PYGZgt{}}\PYG{p}{\PYGZlt{}}\PYG{n+nt}{td}\PYG{p}{\PYGZgt{}}C1\PYGZhy{}2\PYG{p}{\PYGZlt{}}\PYG{p}{/}\PYG{n+nt}{td}\PYG{p}{\PYGZgt{}}\PYG{p}{\PYGZlt{}}\PYG{p}{/}\PYG{n+nt}{tr}\PYG{p}{\PYGZgt{}}
                                    \PYG{p}{\PYGZlt{}}\PYG{p}{/}\PYG{n+nt}{tbody}\PYG{p}{\PYGZgt{}}
                                \PYG{p}{\PYGZlt{}}\PYG{p}{/}\PYG{n+nt}{table}\PYG{p}{\PYGZgt{}}
                            \PYG{p}{\PYGZlt{}}\PYG{p}{/}\PYG{n+nt}{td}\PYG{p}{\PYGZgt{}}
                        \PYG{p}{\PYGZlt{}}\PYG{p}{/}\PYG{n+nt}{tr}\PYG{p}{\PYGZgt{}}
                    \PYG{p}{\PYGZlt{}}\PYG{p}{/}\PYG{n+nt}{tbody}\PYG{p}{\PYGZgt{}}
                \PYG{p}{\PYGZlt{}}\PYG{p}{/}\PYG{n+nt}{table}\PYG{p}{\PYGZgt{}}
            \PYG{p}{\PYGZlt{}}\PYG{p}{/}\PYG{n+nt}{td}\PYG{p}{\PYGZgt{}}
            \PYG{p}{\PYGZlt{}}\PYG{n+nt}{td}\PYG{p}{\PYGZgt{}}
                \PYG{p}{\PYGZlt{}}\PYG{n+nt}{table} \PYG{n+na}{border}\PYG{o}{=}\PYG{l+s}{\PYGZdq{}1\PYGZdq{}}\PYG{p}{\PYGZgt{}}
                    \PYG{p}{\PYGZlt{}}\PYG{n+nt}{tbody}\PYG{p}{\PYGZgt{}}
                        \PYG{p}{\PYGZlt{}}\PYG{n+nt}{tr}\PYG{p}{\PYGZgt{}}\PYG{p}{\PYGZlt{}}\PYG{n+nt}{td}\PYG{p}{\PYGZgt{}}D1\PYG{p}{\PYGZlt{}}\PYG{p}{/}\PYG{n+nt}{td}\PYG{p}{\PYGZgt{}}\PYG{p}{\PYGZlt{}}\PYG{p}{/}\PYG{n+nt}{tr}\PYG{p}{\PYGZgt{}}
                        \PYG{p}{\PYGZlt{}}\PYG{n+nt}{tr}\PYG{p}{\PYGZgt{}}\PYG{p}{\PYGZlt{}}\PYG{n+nt}{td}\PYG{p}{\PYGZgt{}}D2\PYG{p}{\PYGZlt{}}\PYG{p}{/}\PYG{n+nt}{td}\PYG{p}{\PYGZgt{}}\PYG{p}{\PYGZlt{}}\PYG{p}{/}\PYG{n+nt}{tr}\PYG{p}{\PYGZgt{}}
                    \PYG{p}{\PYGZlt{}}\PYG{p}{/}\PYG{n+nt}{tbody}\PYG{p}{\PYGZgt{}}
                \PYG{p}{\PYGZlt{}}\PYG{p}{/}\PYG{n+nt}{table}\PYG{p}{\PYGZgt{}}
            \PYG{p}{\PYGZlt{}}\PYG{p}{/}\PYG{n+nt}{td}\PYG{p}{\PYGZgt{}}
        \PYG{p}{\PYGZlt{}}\PYG{p}{/}\PYG{n+nt}{tr}\PYG{p}{\PYGZgt{}}

    \PYG{p}{\PYGZlt{}}\PYG{p}{/}\PYG{n+nt}{tbody}\PYG{p}{\PYGZgt{}}
\PYG{p}{\PYGZlt{}}\PYG{p}{/}\PYG{n+nt}{table}\PYG{p}{\PYGZgt{}}


\PYG{p}{\PYGZlt{}}\PYG{p}{/}\PYG{n+nt}{body}\PYG{p}{\PYGZgt{}}
\PYG{p}{\PYGZlt{}}\PYG{p}{/}\PYG{n+nt}{html}\PYG{p}{\PYGZgt{}}
\end{sphinxVerbatim}


\subsection{Ejercicio sobre tablas (V)}
\label{\detokenize{tema2:ejercicio-sobre-tablas-v}}
Crea una tabla con la estructura siguiente

\begin{figure}[htbp]
\centering

\noindent\sphinxincludegraphics{{tablaeco1}.png}
\end{figure}


\subsection{Solución}
\label{\detokenize{tema2:id3}}
\begin{sphinxVerbatim}[commandchars=\\\{\}]
\PYG{p}{\PYGZlt{}}\PYG{n+nt}{table} \PYG{n+na}{border}\PYG{o}{=}\PYG{l+s}{\PYGZdq{}1\PYGZdq{}}\PYG{p}{\PYGZgt{}}
        \PYG{p}{\PYGZlt{}}\PYG{n+nt}{thead}\PYG{p}{\PYGZgt{}}
                \PYG{p}{\PYGZlt{}}\PYG{n+nt}{tr}\PYG{p}{\PYGZgt{}}
                        \PYG{p}{\PYGZlt{}}\PYG{n+nt}{th}\PYG{p}{\PYGZgt{}}País\PYG{p}{\PYGZlt{}}\PYG{p}{/}\PYG{n+nt}{th}\PYG{p}{\PYGZgt{}}
                        \PYG{p}{\PYGZlt{}}\PYG{n+nt}{th}\PYG{p}{\PYGZgt{}}Datos econ.\PYG{p}{\PYGZlt{}}\PYG{p}{/}\PYG{n+nt}{th}\PYG{p}{\PYGZgt{}}
                \PYG{p}{\PYGZlt{}}\PYG{p}{/}\PYG{n+nt}{tr}\PYG{p}{\PYGZgt{}}
        \PYG{p}{\PYGZlt{}}\PYG{p}{/}\PYG{n+nt}{thead}\PYG{p}{\PYGZgt{}}
        \PYG{p}{\PYGZlt{}}\PYG{n+nt}{tbody}\PYG{p}{\PYGZgt{}}
                \PYG{p}{\PYGZlt{}}\PYG{n+nt}{tr}\PYG{p}{\PYGZgt{}}
                        \PYG{p}{\PYGZlt{}}\PYG{n+nt}{td}\PYG{p}{\PYGZgt{}}España\PYG{p}{\PYGZlt{}}\PYG{p}{/}\PYG{n+nt}{td}\PYG{p}{\PYGZgt{}}
                        \PYG{c}{\PYGZlt{}!\PYGZhy{}\PYGZhy{}}\PYG{c}{Atención}
\PYG{c}{                        ¡Tabla dentro de celda}\PYG{c}{\PYGZhy{}\PYGZhy{}\PYGZgt{}}
                        \PYG{p}{\PYGZlt{}}\PYG{n+nt}{td}\PYG{p}{\PYGZgt{}}
                                \PYG{p}{\PYGZlt{}}\PYG{n+nt}{table} \PYG{n+na}{border}\PYG{o}{=}\PYG{l+s}{\PYGZdq{}1\PYGZdq{}}\PYG{p}{\PYGZgt{}}
                                        \PYG{p}{\PYGZlt{}}\PYG{n+nt}{tr}\PYG{p}{\PYGZgt{}}
                                                \PYG{p}{\PYGZlt{}}\PYG{n+nt}{td}\PYG{p}{\PYGZgt{}}
                                                        PIB:\PYGZhy{}0,1\PYGZpc{}
                                                \PYG{p}{\PYGZlt{}}\PYG{p}{/}\PYG{n+nt}{td}\PYG{p}{\PYGZgt{}}
                                        \PYG{p}{\PYGZlt{}}\PYG{p}{/}\PYG{n+nt}{tr}\PYG{p}{\PYGZgt{}}
                                        \PYG{p}{\PYGZlt{}}\PYG{n+nt}{tr}\PYG{p}{\PYGZgt{}}
                                                \PYG{p}{\PYGZlt{}}\PYG{n+nt}{td}\PYG{p}{\PYGZgt{}}
                                                        Déficit:5\PYGZpc{}
                                                \PYG{p}{\PYGZlt{}}\PYG{p}{/}\PYG{n+nt}{td}\PYG{p}{\PYGZgt{}}
                                        \PYG{p}{\PYGZlt{}}\PYG{p}{/}\PYG{n+nt}{tr}\PYG{p}{\PYGZgt{}}
                                        \PYG{p}{\PYGZlt{}}\PYG{n+nt}{tr}\PYG{p}{\PYGZgt{}}
                                                \PYG{p}{\PYGZlt{}}\PYG{n+nt}{td}\PYG{p}{\PYGZgt{}}
                                                        Paro 25,4\PYGZpc{}
                                                \PYG{p}{\PYGZlt{}}\PYG{p}{/}\PYG{n+nt}{td}\PYG{p}{\PYGZgt{}}
                                        \PYG{p}{\PYGZlt{}}\PYG{p}{/}\PYG{n+nt}{tr}\PYG{p}{\PYGZgt{}}
                                \PYG{p}{\PYGZlt{}}\PYG{p}{/}\PYG{n+nt}{table}\PYG{p}{\PYGZgt{}}
                        \PYG{p}{\PYGZlt{}}\PYG{p}{/}\PYG{n+nt}{td}\PYG{p}{\PYGZgt{}}
                \PYG{p}{\PYGZlt{}}\PYG{p}{/}\PYG{n+nt}{tr}\PYG{p}{\PYGZgt{}}
                \PYG{p}{\PYGZlt{}}\PYG{n+nt}{tr}\PYG{p}{\PYGZgt{}}
                        \PYG{p}{\PYGZlt{}}\PYG{n+nt}{td}\PYG{p}{\PYGZgt{}}USA\PYG{p}{\PYGZlt{}}\PYG{p}{/}\PYG{n+nt}{td}\PYG{p}{\PYGZgt{}}
                        \PYG{c}{\PYGZlt{}!\PYGZhy{}\PYGZhy{}}\PYG{c}{Otra tabla dentro de celda}\PYG{c}{\PYGZhy{}\PYGZhy{}\PYGZgt{}}
                        \PYG{p}{\PYGZlt{}}\PYG{n+nt}{td}\PYG{p}{\PYGZgt{}}
                                \PYG{p}{\PYGZlt{}}\PYG{n+nt}{table} \PYG{n+na}{border}\PYG{o}{=}\PYG{l+s}{\PYGZdq{}1\PYGZdq{}}\PYG{p}{\PYGZgt{}}
                                        \PYG{p}{\PYGZlt{}}\PYG{n+nt}{tr}\PYG{p}{\PYGZgt{}}
                                                \PYG{p}{\PYGZlt{}}\PYG{n+nt}{td}\PYG{p}{\PYGZgt{}}
                                                        PIB:0,4\PYGZpc{}
                                                \PYG{p}{\PYGZlt{}}\PYG{p}{/}\PYG{n+nt}{td}\PYG{p}{\PYGZgt{}}
                                        \PYG{p}{\PYGZlt{}}\PYG{p}{/}\PYG{n+nt}{tr}\PYG{p}{\PYGZgt{}}
                                        \PYG{p}{\PYGZlt{}}\PYG{n+nt}{tr}\PYG{p}{\PYGZgt{}}
                                                \PYG{p}{\PYGZlt{}}\PYG{n+nt}{td}\PYG{p}{\PYGZgt{}}
                                                        Déficit:3\PYGZpc{}
                                                \PYG{p}{\PYGZlt{}}\PYG{p}{/}\PYG{n+nt}{td}\PYG{p}{\PYGZgt{}}
                                        \PYG{p}{\PYGZlt{}}\PYG{p}{/}\PYG{n+nt}{tr}\PYG{p}{\PYGZgt{}}
                                        \PYG{p}{\PYGZlt{}}\PYG{n+nt}{tr}\PYG{p}{\PYGZgt{}}
                                                \PYG{p}{\PYGZlt{}}\PYG{n+nt}{td}\PYG{p}{\PYGZgt{}}
                                                        Paro:11\PYGZpc{}
                                                \PYG{p}{\PYGZlt{}}\PYG{p}{/}\PYG{n+nt}{td}\PYG{p}{\PYGZgt{}}
                                        \PYG{p}{\PYGZlt{}}\PYG{p}{/}\PYG{n+nt}{tr}\PYG{p}{\PYGZgt{}}
                                \PYG{p}{\PYGZlt{}}\PYG{p}{/}\PYG{n+nt}{table}\PYG{p}{\PYGZgt{}}
                        \PYG{p}{\PYGZlt{}}\PYG{p}{/}\PYG{n+nt}{td}\PYG{p}{\PYGZgt{}}
                \PYG{p}{\PYGZlt{}}\PYG{p}{/}\PYG{n+nt}{tr}\PYG{p}{\PYGZgt{}}
        \PYG{p}{\PYGZlt{}}\PYG{p}{/}\PYG{n+nt}{tbody}\PYG{p}{\PYGZgt{}}
\PYG{p}{\PYGZlt{}}\PYG{p}{/}\PYG{n+nt}{table}\PYG{p}{\PYGZgt{}}
\end{sphinxVerbatim}


\subsection{Ejercicio sobre tablas (VI)}
\label{\detokenize{tema2:ejercicio-sobre-tablas-vi}}
Crea una tabla con la estructura siguiente

\begin{figure}[htbp]
\centering

\noindent\sphinxincludegraphics{{tablaeco2}.png}
\end{figure}


\subsection{Solución}
\label{\detokenize{tema2:id4}}
\begin{sphinxVerbatim}[commandchars=\\\{\}]
\PYG{p}{\PYGZlt{}}\PYG{n+nt}{table} \PYG{n+na}{border}\PYG{o}{=}\PYG{l+s}{\PYGZdq{}1\PYGZdq{}}\PYG{p}{\PYGZgt{}}
        \PYG{p}{\PYGZlt{}}\PYG{n+nt}{tbody}\PYG{p}{\PYGZgt{}}
                \PYG{p}{\PYGZlt{}}\PYG{n+nt}{tr}\PYG{p}{\PYGZgt{}}
                        \PYG{p}{\PYGZlt{}}\PYG{n+nt}{td}\PYG{p}{\PYGZgt{}}España\PYG{p}{\PYGZlt{}}\PYG{p}{/}\PYG{n+nt}{td}\PYG{p}{\PYGZgt{}}
                        \PYG{p}{\PYGZlt{}}\PYG{n+nt}{td}\PYG{p}{\PYGZgt{}}
                                \PYG{c}{\PYGZlt{}!\PYGZhy{}\PYGZhy{}}\PYG{c}{Subtabla}\PYG{c}{\PYGZhy{}\PYGZhy{}\PYGZgt{}}
                                \PYG{p}{\PYGZlt{}}\PYG{n+nt}{table} \PYG{n+na}{border}\PYG{o}{=}\PYG{l+s}{\PYGZdq{}1\PYGZdq{}}\PYG{p}{\PYGZgt{}}
                                        \PYG{p}{\PYGZlt{}}\PYG{n+nt}{tr}\PYG{p}{\PYGZgt{}}
                                                \PYG{p}{\PYGZlt{}}\PYG{n+nt}{td}\PYG{p}{\PYGZgt{}}
                                                        I+D:7\PYGZpc{}
                                                \PYG{p}{\PYGZlt{}}\PYG{p}{/}\PYG{n+nt}{td}\PYG{p}{\PYGZgt{}}
                                                \PYG{p}{\PYGZlt{}}\PYG{n+nt}{td}\PYG{p}{\PYGZgt{}}
                                                        IRS:13\PYGZpc{}
                                                \PYG{p}{\PYGZlt{}}\PYG{p}{/}\PYG{n+nt}{td}\PYG{p}{\PYGZgt{}}
                                        \PYG{p}{\PYGZlt{}}\PYG{p}{/}\PYG{n+nt}{tr}\PYG{p}{\PYGZgt{}}
                                \PYG{p}{\PYGZlt{}}\PYG{p}{/}\PYG{n+nt}{table}\PYG{p}{\PYGZgt{}}
                        \PYG{p}{\PYGZlt{}}\PYG{p}{/}\PYG{n+nt}{td}\PYG{p}{\PYGZgt{}}
                \PYG{p}{\PYGZlt{}}\PYG{p}{/}\PYG{n+nt}{tr}\PYG{p}{\PYGZgt{}}
                \PYG{p}{\PYGZlt{}}\PYG{n+nt}{tr}\PYG{p}{\PYGZgt{}}
                        \PYG{p}{\PYGZlt{}}\PYG{n+nt}{td}\PYG{p}{\PYGZgt{}}ALCA\PYG{p}{\PYGZlt{}}\PYG{p}{/}\PYG{n+nt}{td}\PYG{p}{\PYGZgt{}}
                        \PYG{p}{\PYGZlt{}}\PYG{n+nt}{td}\PYG{p}{\PYGZgt{}}
                                \PYG{c}{\PYGZlt{}!\PYGZhy{}\PYGZhy{}}\PYG{c}{Subtabla}\PYG{c}{\PYGZhy{}\PYGZhy{}\PYGZgt{}}
                                \PYG{p}{\PYGZlt{}}\PYG{n+nt}{table} \PYG{n+na}{border}\PYG{o}{=}\PYG{l+s}{\PYGZdq{}1\PYGZdq{}}\PYG{p}{\PYGZgt{}}
                                        \PYG{p}{\PYGZlt{}}\PYG{n+nt}{tr}\PYG{p}{\PYGZgt{}}
                                                \PYG{p}{\PYGZlt{}}\PYG{n+nt}{td}\PYG{p}{\PYGZgt{}}
                                                        I+D:3\PYGZpc{}
                                                \PYG{p}{\PYGZlt{}}\PYG{p}{/}\PYG{n+nt}{td}\PYG{p}{\PYGZgt{}}
                                                \PYG{p}{\PYGZlt{}}\PYG{n+nt}{td}\PYG{p}{\PYGZgt{}}
                                                        Felic:38
                                                \PYG{p}{\PYGZlt{}}\PYG{p}{/}\PYG{n+nt}{td}\PYG{p}{\PYGZgt{}}
                                                \PYG{p}{\PYGZlt{}}\PYG{n+nt}{td}\PYG{p}{\PYGZgt{}}
                                                        Resto:42\PYGZpc{}
                                                \PYG{p}{\PYGZlt{}}\PYG{p}{/}\PYG{n+nt}{td}\PYG{p}{\PYGZgt{}}
                                        \PYG{p}{\PYGZlt{}}\PYG{p}{/}\PYG{n+nt}{tr}\PYG{p}{\PYGZgt{}}
                                \PYG{p}{\PYGZlt{}}\PYG{p}{/}\PYG{n+nt}{table}\PYG{p}{\PYGZgt{}}
                        \PYG{p}{\PYGZlt{}}\PYG{p}{/}\PYG{n+nt}{td}\PYG{p}{\PYGZgt{}}
                \PYG{p}{\PYGZlt{}}\PYG{p}{/}\PYG{n+nt}{tr}\PYG{p}{\PYGZgt{}}
                \PYG{p}{\PYGZlt{}}\PYG{n+nt}{tr}\PYG{p}{\PYGZgt{}}
                        \PYG{p}{\PYGZlt{}}\PYG{n+nt}{td}\PYG{p}{\PYGZgt{}}UE\PYG{p}{\PYGZlt{}}\PYG{p}{/}\PYG{n+nt}{td}\PYG{p}{\PYGZgt{}}
                        \PYG{p}{\PYGZlt{}}\PYG{n+nt}{td}\PYG{p}{\PYGZgt{}}
                                \PYG{c}{\PYGZlt{}!\PYGZhy{}\PYGZhy{}}\PYG{c}{Subtabla}\PYG{c}{\PYGZhy{}\PYGZhy{}\PYGZgt{}}
                                \PYG{p}{\PYGZlt{}}\PYG{n+nt}{table} \PYG{n+na}{border}\PYG{o}{=}\PYG{l+s}{\PYGZdq{}1\PYGZdq{}}\PYG{p}{\PYGZgt{}}
                                        \PYG{p}{\PYGZlt{}}\PYG{n+nt}{tr}\PYG{p}{\PYGZgt{}}
                                                \PYG{p}{\PYGZlt{}}\PYG{n+nt}{td}\PYG{p}{\PYGZgt{}}
                                                        España
                                                \PYG{p}{\PYGZlt{}}\PYG{p}{/}\PYG{n+nt}{td}\PYG{p}{\PYGZgt{}}
                                                \PYG{p}{\PYGZlt{}}\PYG{n+nt}{td}\PYG{p}{\PYGZgt{}}
                                                        I+D:8\PYGZpc{}
                                                \PYG{p}{\PYGZlt{}}\PYG{p}{/}\PYG{n+nt}{td}\PYG{p}{\PYGZgt{}}
                                                \PYG{p}{\PYGZlt{}}\PYG{n+nt}{td}\PYG{p}{\PYGZgt{}}
                                                        SS:25\PYGZpc{}
                                                \PYG{p}{\PYGZlt{}}\PYG{p}{/}\PYG{n+nt}{td}\PYG{p}{\PYGZgt{}}
                                        \PYG{p}{\PYGZlt{}}\PYG{p}{/}\PYG{n+nt}{tr}\PYG{p}{\PYGZgt{}}
                                        \PYG{p}{\PYGZlt{}}\PYG{n+nt}{tr}\PYG{p}{\PYGZgt{}}
                                                \PYG{p}{\PYGZlt{}}\PYG{n+nt}{td}\PYG{p}{\PYGZgt{}}
                                                        Resto UE
                                                \PYG{p}{\PYGZlt{}}\PYG{p}{/}\PYG{n+nt}{td}\PYG{p}{\PYGZgt{}}
                                                \PYG{p}{\PYGZlt{}}\PYG{n+nt}{td}\PYG{p}{\PYGZgt{}}
                                                        I+D:13\PYGZpc{}
                                                \PYG{p}{\PYGZlt{}}\PYG{p}{/}\PYG{n+nt}{td}\PYG{p}{\PYGZgt{}}
                                                \PYG{p}{\PYGZlt{}}\PYG{n+nt}{td}\PYG{p}{\PYGZgt{}}
                                                        Otros:28\PYGZpc{}
                                                \PYG{p}{\PYGZlt{}}\PYG{p}{/}\PYG{n+nt}{td}\PYG{p}{\PYGZgt{}}
                                        \PYG{p}{\PYGZlt{}}\PYG{p}{/}\PYG{n+nt}{tr}\PYG{p}{\PYGZgt{}}
                                \PYG{p}{\PYGZlt{}}\PYG{p}{/}\PYG{n+nt}{table}\PYG{p}{\PYGZgt{}}
                        \PYG{p}{\PYGZlt{}}\PYG{p}{/}\PYG{n+nt}{td}\PYG{p}{\PYGZgt{}}
                \PYG{p}{\PYGZlt{}}\PYG{p}{/}\PYG{n+nt}{tr}\PYG{p}{\PYGZgt{}}
        \PYG{p}{\PYGZlt{}}\PYG{p}{/}\PYG{n+nt}{tbody}\PYG{p}{\PYGZgt{}}
\PYG{p}{\PYGZlt{}}\PYG{p}{/}\PYG{n+nt}{table}\PYG{p}{\PYGZgt{}}
\end{sphinxVerbatim}


\subsection{Ejercicio sobre tablas (VII)}
\label{\detokenize{tema2:ejercicio-sobre-tablas-vii}}
Crea una tabla con la estructura siguiente

\begin{figure}[htbp]
\centering

\noindent\sphinxincludegraphics{{tabla7}.png}
\end{figure}


\subsection{Solución}
\label{\detokenize{tema2:id5}}
\begin{sphinxVerbatim}[commandchars=\\\{\}]
\PYG{p}{\PYGZlt{}}\PYG{n+nt}{table} \PYG{n+na}{border}\PYG{o}{=}\PYG{l+s}{\PYGZdq{}1\PYGZdq{}}\PYG{p}{\PYGZgt{}}
  \PYG{p}{\PYGZlt{}}\PYG{n+nt}{tbody}\PYG{p}{\PYGZgt{}}
        \PYG{p}{\PYGZlt{}}\PYG{n+nt}{tr}\PYG{p}{\PYGZgt{}}
          \PYG{p}{\PYGZlt{}}\PYG{n+nt}{td}\PYG{p}{\PYGZgt{}}A\PYG{p}{\PYGZlt{}}\PYG{p}{/}\PYG{n+nt}{td}\PYG{p}{\PYGZgt{}}
        \PYG{p}{\PYGZlt{}}\PYG{p}{/}\PYG{n+nt}{tr}\PYG{p}{\PYGZgt{}}
        \PYG{p}{\PYGZlt{}}\PYG{n+nt}{tr}\PYG{p}{\PYGZgt{}}
          \PYG{p}{\PYGZlt{}}\PYG{n+nt}{td}\PYG{p}{\PYGZgt{}}
                \PYG{p}{\PYGZlt{}}\PYG{n+nt}{table} \PYG{n+na}{border}\PYG{o}{=}\PYG{l+s}{\PYGZdq{}1\PYGZdq{}}\PYG{p}{\PYGZgt{}}
                  \PYG{p}{\PYGZlt{}}\PYG{n+nt}{tbody}\PYG{p}{\PYGZgt{}}
                        \PYG{p}{\PYGZlt{}}\PYG{n+nt}{tr}\PYG{p}{\PYGZgt{}}
                          \PYG{p}{\PYGZlt{}}\PYG{n+nt}{td}\PYG{p}{\PYGZgt{}}
                                \PYG{p}{\PYGZlt{}}\PYG{n+nt}{table} \PYG{n+na}{border}\PYG{o}{=}\PYG{l+s}{\PYGZdq{}1\PYGZdq{}}\PYG{p}{\PYGZgt{}}
                                  \PYG{p}{\PYGZlt{}}\PYG{n+nt}{tbody}\PYG{p}{\PYGZgt{}}
                                        \PYG{p}{\PYGZlt{}}\PYG{n+nt}{tr}\PYG{p}{\PYGZgt{}}
                                          \PYG{p}{\PYGZlt{}}\PYG{n+nt}{td}\PYG{p}{\PYGZgt{}}B1\PYG{p}{\PYGZlt{}}\PYG{p}{/}\PYG{n+nt}{td}\PYG{p}{\PYGZgt{}}
                                        \PYG{p}{\PYGZlt{}}\PYG{p}{/}\PYG{n+nt}{tr}\PYG{p}{\PYGZgt{}}
                                        \PYG{p}{\PYGZlt{}}\PYG{n+nt}{tr}\PYG{p}{\PYGZgt{}}
                                          \PYG{p}{\PYGZlt{}}\PYG{n+nt}{td}\PYG{p}{\PYGZgt{}}B2\PYG{p}{\PYGZlt{}}\PYG{p}{/}\PYG{n+nt}{td}\PYG{p}{\PYGZgt{}}
                                        \PYG{p}{\PYGZlt{}}\PYG{p}{/}\PYG{n+nt}{tr}\PYG{p}{\PYGZgt{}}
                                        \PYG{p}{\PYGZlt{}}\PYG{n+nt}{tr}\PYG{p}{\PYGZgt{}}
                                          \PYG{p}{\PYGZlt{}}\PYG{n+nt}{td}\PYG{p}{\PYGZgt{}}B3\PYG{p}{\PYGZlt{}}\PYG{p}{/}\PYG{n+nt}{td}\PYG{p}{\PYGZgt{}}
                                        \PYG{p}{\PYGZlt{}}\PYG{p}{/}\PYG{n+nt}{tr}\PYG{p}{\PYGZgt{}}
                                        \PYG{p}{\PYGZlt{}}\PYG{p}{/}\PYG{n+nt}{tbody}\PYG{p}{\PYGZgt{}}
                                \PYG{p}{\PYGZlt{}}\PYG{p}{/}\PYG{n+nt}{table}\PYG{p}{\PYGZgt{}}
                          \PYG{p}{\PYGZlt{}}\PYG{p}{/}\PYG{n+nt}{td}\PYG{p}{\PYGZgt{}}
                          \PYG{p}{\PYGZlt{}}\PYG{n+nt}{td}\PYG{p}{\PYGZgt{}}
                                \PYG{p}{\PYGZlt{}}\PYG{n+nt}{table} \PYG{n+na}{border}\PYG{o}{=}\PYG{l+s}{\PYGZdq{}1\PYGZdq{}}\PYG{p}{\PYGZgt{}}
                                  \PYG{p}{\PYGZlt{}}\PYG{n+nt}{tbody}\PYG{p}{\PYGZgt{}}
                                        \PYG{p}{\PYGZlt{}}\PYG{n+nt}{tr}\PYG{p}{\PYGZgt{}}
                                          \PYG{p}{\PYGZlt{}}\PYG{n+nt}{td}\PYG{p}{\PYGZgt{}}C1\PYG{p}{\PYGZlt{}}\PYG{p}{/}\PYG{n+nt}{td}\PYG{p}{\PYGZgt{}}
                                          \PYG{p}{\PYGZlt{}}\PYG{n+nt}{td}\PYG{p}{\PYGZgt{}}C2\PYG{p}{\PYGZlt{}}\PYG{p}{/}\PYG{n+nt}{td}\PYG{p}{\PYGZgt{}}
                                        \PYG{p}{\PYGZlt{}}\PYG{p}{/}\PYG{n+nt}{tr}\PYG{p}{\PYGZgt{}}
                                  \PYG{p}{\PYGZlt{}}\PYG{p}{/}\PYG{n+nt}{tbody}\PYG{p}{\PYGZgt{}}
                                \PYG{p}{\PYGZlt{}}\PYG{p}{/}\PYG{n+nt}{table}\PYG{p}{\PYGZgt{}}
                          \PYG{p}{\PYGZlt{}}\PYG{p}{/}\PYG{n+nt}{td}\PYG{p}{\PYGZgt{}}
                        \PYG{p}{\PYGZlt{}}\PYG{p}{/}\PYG{n+nt}{tr}\PYG{p}{\PYGZgt{}}
                  \PYG{p}{\PYGZlt{}}\PYG{p}{/}\PYG{n+nt}{tbody}\PYG{p}{\PYGZgt{}}
                \PYG{p}{\PYGZlt{}}\PYG{p}{/}\PYG{n+nt}{table}\PYG{p}{\PYGZgt{}}
          \PYG{p}{\PYGZlt{}}\PYG{p}{/}\PYG{n+nt}{td}\PYG{p}{\PYGZgt{}}
        \PYG{p}{\PYGZlt{}}\PYG{p}{/}\PYG{n+nt}{tr}\PYG{p}{\PYGZgt{}}
  \PYG{p}{\PYGZlt{}}\PYG{p}{/}\PYG{n+nt}{tbody}\PYG{p}{\PYGZgt{}}
\PYG{p}{\PYGZlt{}}\PYG{p}{/}\PYG{n+nt}{table}\PYG{p}{\PYGZgt{}}
\end{sphinxVerbatim}


\subsection{Ejercicio sobre tablas (VIII)}
\label{\detokenize{tema2:ejercicio-sobre-tablas-viii}}
Crea una tabla con la estructura siguiente

\begin{figure}[htbp]
\centering

\noindent\sphinxincludegraphics{{tabla8}.png}
\end{figure}


\subsection{Solución}
\label{\detokenize{tema2:id6}}
\begin{sphinxVerbatim}[commandchars=\\\{\}]
\PYG{p}{\PYGZlt{}}\PYG{n+nt}{table} \PYG{n+na}{border}\PYG{o}{=}\PYG{l+s}{\PYGZdq{}1\PYGZdq{}}\PYG{p}{\PYGZgt{}}
  \PYG{p}{\PYGZlt{}}\PYG{n+nt}{tr}\PYG{p}{\PYGZgt{}}
    \PYG{p}{\PYGZlt{}}\PYG{n+nt}{td}\PYG{p}{\PYGZgt{}}
      \PYG{p}{\PYGZlt{}}\PYG{n+nt}{table} \PYG{n+na}{border}\PYG{o}{=}\PYG{l+s}{\PYGZdq{}1\PYGZdq{}}\PYG{p}{\PYGZgt{}}
        \PYG{p}{\PYGZlt{}}\PYG{n+nt}{tbody}\PYG{p}{\PYGZgt{}}
          \PYG{p}{\PYGZlt{}}\PYG{n+nt}{tr}\PYG{p}{\PYGZgt{}}
              \PYG{p}{\PYGZlt{}}\PYG{n+nt}{td}\PYG{p}{\PYGZgt{}}A1\PYG{p}{\PYGZlt{}}\PYG{p}{/}\PYG{n+nt}{td}\PYG{p}{\PYGZgt{}}
              \PYG{p}{\PYGZlt{}}\PYG{n+nt}{td}\PYG{p}{\PYGZgt{}}A2\PYG{p}{\PYGZlt{}}\PYG{p}{/}\PYG{n+nt}{td}\PYG{p}{\PYGZgt{}}
              \PYG{p}{\PYGZlt{}}\PYG{n+nt}{td}\PYG{p}{\PYGZgt{}}A3\PYG{p}{\PYGZlt{}}\PYG{p}{/}\PYG{n+nt}{td}\PYG{p}{\PYGZgt{}}
          \PYG{p}{\PYGZlt{}}\PYG{p}{/}\PYG{n+nt}{tr}\PYG{p}{\PYGZgt{}}
        \PYG{p}{\PYGZlt{}}\PYG{p}{/}\PYG{n+nt}{tbody}\PYG{p}{\PYGZgt{}}
      \PYG{p}{\PYGZlt{}}\PYG{p}{/}\PYG{n+nt}{table}\PYG{p}{\PYGZgt{}}
    \PYG{p}{\PYGZlt{}}\PYG{p}{/}\PYG{n+nt}{td}\PYG{p}{\PYGZgt{}}
  \PYG{p}{\PYGZlt{}}\PYG{p}{/}\PYG{n+nt}{tr}\PYG{p}{\PYGZgt{}}
  \PYG{p}{\PYGZlt{}}\PYG{n+nt}{tr}\PYG{p}{\PYGZgt{}}
    \PYG{p}{\PYGZlt{}}\PYG{n+nt}{td}\PYG{p}{\PYGZgt{}}
      \PYG{p}{\PYGZlt{}}\PYG{n+nt}{table} \PYG{n+na}{border}\PYG{o}{=}\PYG{l+s}{\PYGZdq{}1\PYGZdq{}}\PYG{p}{\PYGZgt{}}
        \PYG{p}{\PYGZlt{}}\PYG{n+nt}{tbody}\PYG{p}{\PYGZgt{}}
          \PYG{p}{\PYGZlt{}}\PYG{n+nt}{tr}\PYG{p}{\PYGZgt{}}
            \PYG{p}{\PYGZlt{}}\PYG{n+nt}{td}\PYG{p}{\PYGZgt{}}B1\PYG{p}{\PYGZlt{}}\PYG{p}{/}\PYG{n+nt}{td}\PYG{p}{\PYGZgt{}}
          \PYG{p}{\PYGZlt{}}\PYG{p}{/}\PYG{n+nt}{tr}\PYG{p}{\PYGZgt{}}
          \PYG{p}{\PYGZlt{}}\PYG{n+nt}{tr}\PYG{p}{\PYGZgt{}}
            \PYG{p}{\PYGZlt{}}\PYG{n+nt}{td}\PYG{p}{\PYGZgt{}}B2\PYG{p}{\PYGZlt{}}\PYG{p}{/}\PYG{n+nt}{td}\PYG{p}{\PYGZgt{}}
          \PYG{p}{\PYGZlt{}}\PYG{p}{/}\PYG{n+nt}{tr}\PYG{p}{\PYGZgt{}}
        \PYG{p}{\PYGZlt{}}\PYG{p}{/}\PYG{n+nt}{tbody}\PYG{p}{\PYGZgt{}}
      \PYG{p}{\PYGZlt{}}\PYG{p}{/}\PYG{n+nt}{table}\PYG{p}{\PYGZgt{}}
    \PYG{p}{\PYGZlt{}}\PYG{p}{/}\PYG{n+nt}{td}\PYG{p}{\PYGZgt{}}
  \PYG{p}{\PYGZlt{}}\PYG{p}{/}\PYG{n+nt}{tr}\PYG{p}{\PYGZgt{}}
\PYG{p}{\PYGZlt{}}\PYG{p}{/}\PYG{n+nt}{table}\PYG{p}{\PYGZgt{}}
\end{sphinxVerbatim}


\section{Formularios}
\label{\detokenize{tema2:formularios}}
Un formulario permite que el usuario interactúe con la página por medio de una serie de controles


\subsection{Campo de texto}
\label{\detokenize{tema2:campo-de-texto}}
Permite crear una zona donde el usuario puede escribir y se muestra un ejemplo a continuación. Tiene algunos atributos que se usan muy a menudo:
\begin{itemize}
\item {} 
\sphinxcode{type} indica el tipo de control

\item {} 
\sphinxcode{name} será el nombre de la variable en JS (no lo usaremos por ahora)

\item {} 
\sphinxcode{id} permitirá procesar el control de JS (no se usará por ahora)

\item {} 
\sphinxcode{value} permite indicar un valor por defecto

\item {} 
\sphinxcode{size} indica la anchura por defecto

\end{itemize}

\begin{sphinxVerbatim}[commandchars=\\\{\}]
\PYG{p}{\PYGZlt{}}\PYG{n+nt}{input} \PYG{n+na}{type}\PYG{o}{=}\PYG{l+s}{\PYGZdq{}text\PYGZdq{}} \PYG{n+na}{name}\PYG{o}{=}\PYG{l+s}{\PYGZdq{}nombre\PYGZus{}usuario\PYGZdq{}}
        \PYG{n+na}{id}\PYG{o}{=}\PYG{l+s}{\PYGZdq{}id\PYGZus{}nombre\PYGZdq{}} \PYG{n+na}{value}\PYG{o}{=}\PYG{l+s}{\PYGZdq{}Escriba su nombre aqui\PYGZdq{}}\PYG{p}{\PYGZgt{}}
\end{sphinxVerbatim}

Un campo de texto puede llevar asociada una etiqueta \sphinxcode{label} que indique al navegador que texto va con ese campo. Esto es de utilidad para programas lectores de páginas y en general para gente con discapacidad.

\begin{sphinxVerbatim}[commandchars=\\\{\}]
\PYG{p}{\PYGZlt{}}\PYG{n+nt}{label} \PYG{n+na}{for}\PYG{o}{=}\PYG{l+s}{\PYGZsq{}\PYGZdq{}d\PYGZus{}nombre\PYGZdq{}}\PYG{p}{\PYGZgt{}}Nombre de usuario\PYG{p}{\PYGZlt{}}\PYG{p}{/}\PYG{n+nt}{label}\PYG{p}{\PYGZgt{}}
\PYG{p}{\PYGZlt{}}\PYG{n+nt}{input} \PYG{n+na}{type}\PYG{o}{=}\PYG{l+s}{\PYGZdq{}text\PYGZdq{}} \PYG{n+na}{name}\PYG{o}{=}\PYG{l+s}{\PYGZdq{}nombre\PYGZus{}usuario\PYGZdq{}}
        \PYG{n+na}{id}\PYG{o}{=}\PYG{l+s}{\PYGZdq{}id\PYGZus{}nombre\PYGZdq{}} \PYG{n+na}{value}\PYG{o}{=}\PYG{l+s}{\PYGZdq{}Escriba su nombre aqui\PYGZdq{}}\PYG{p}{\PYGZgt{}}
\end{sphinxVerbatim}

Si el type de este elemento se sustituye por \sphinxcode{password} se obtiene un control igual, pero que reemplaza el texto por símbolos que ocultan el texto.


\subsection{Selector único (radio-button)}
\label{\detokenize{tema2:selector-unico-radio-button}}
Permite elegir una sola opción de entre muchas, se necesita usar un \sphinxcode{input} de tipo \sphinxcode{radio}.

\begin{sphinxVerbatim}[commandchars=\\\{\}]
\PYG{p}{\PYGZlt{}}\PYG{n+nt}{br}\PYG{p}{/}\PYG{p}{\PYGZgt{}}
\PYG{p}{\PYGZlt{}}\PYG{n+nt}{input} \PYG{n+na}{type}\PYG{o}{=}\PYG{l+s}{\PYGZdq{}radio\PYGZdq{}} \PYG{n+na}{name}\PYG{o}{=}\PYG{l+s}{\PYGZdq{}sexo\PYGZdq{}}\PYG{p}{\PYGZgt{}}Masculino
\PYG{p}{\PYGZlt{}}\PYG{n+nt}{br}\PYG{p}{/}\PYG{p}{\PYGZgt{}}
\PYG{p}{\PYGZlt{}}\PYG{n+nt}{input} \PYG{n+na}{type}\PYG{o}{=}\PYG{l+s}{\PYGZdq{}radio\PYGZdq{}} \PYG{n+na}{name}\PYG{o}{=}\PYG{l+s}{\PYGZdq{}sexo\PYGZdq{}}\PYG{p}{\PYGZgt{}}Femenino
\end{sphinxVerbatim}

Si se desea que un control de tipo \sphinxcode{checkbox} o \sphinxcode{radio} aparezca marcado por defecto se debe añadir el atributo \sphinxcode{multiple="multiple"}.


\subsection{Selector múltiple (checkbox)}
\label{\detokenize{tema2:selector-multiple-checkbox}}
Permite elegir múltiples combinaciones de opciones. El nombre del control utilizará los corchetes para crear un vector que se procesará desde Javascript.

\begin{sphinxVerbatim}[commandchars=\\\{\}]
\PYG{p}{\PYGZlt{}}\PYG{n+nt}{br}\PYG{p}{/}\PYG{p}{\PYGZgt{}}
\PYG{p}{\PYGZlt{}}\PYG{n+nt}{input} \PYG{n+na}{type}\PYG{o}{=}\PYG{l+s}{\PYGZdq{}checkbox\PYGZdq{}} \PYG{n+na}{name}\PYG{o}{=}\PYG{l+s}{\PYGZdq{}medios[]\PYGZdq{}}\PYG{p}{\PYGZgt{}}
Coche
\PYG{p}{\PYGZlt{}}\PYG{n+nt}{br}\PYG{p}{/}\PYG{p}{\PYGZgt{}}
\PYG{p}{\PYGZlt{}}\PYG{n+nt}{input} \PYG{n+na}{type}\PYG{o}{=}\PYG{l+s}{\PYGZdq{}checkbox\PYGZdq{}} \PYG{n+na}{name}\PYG{o}{=}\PYG{l+s}{\PYGZdq{}medios[]\PYGZdq{}}\PYG{p}{\PYGZgt{}}
Moto
\PYG{p}{\PYGZlt{}}\PYG{n+nt}{br}\PYG{p}{/}\PYG{p}{\PYGZgt{}}
\PYG{p}{\PYGZlt{}}\PYG{n+nt}{input} \PYG{n+na}{type}\PYG{o}{=}\PYG{l+s}{\PYGZdq{}checkbox\PYGZdq{}} \PYG{n+na}{name}\PYG{o}{=}\PYG{l+s}{\PYGZdq{}medios[]\PYGZdq{}}\PYG{p}{\PYGZgt{}}
Bici
\end{sphinxVerbatim}


\subsection{Lista desplegable}
\label{\detokenize{tema2:lista-desplegable}}
Permite elegir valores de una lista.

\begin{sphinxVerbatim}[commandchars=\\\{\}]
\PYG{p}{\PYGZlt{}}\PYG{n+nt}{select} \PYG{n+na}{name}\PYG{o}{=}\PYG{l+s}{\PYGZdq{}provincia\PYGZdq{}}\PYG{p}{\PYGZgt{}}
        \PYG{p}{\PYGZlt{}}\PYG{n+nt}{option} \PYG{n+na}{value}\PYG{o}{=}\PYG{l+s}{\PYGZdq{}AB\PYGZdq{}}\PYG{p}{\PYGZgt{}}Albacete\PYG{p}{\PYGZlt{}}\PYG{p}{/}\PYG{n+nt}{option}\PYG{p}{\PYGZgt{}}
        \PYG{p}{\PYGZlt{}}\PYG{n+nt}{option} \PYG{n+na}{value}\PYG{o}{=}\PYG{l+s}{\PYGZdq{}CR\PYGZdq{}}\PYG{p}{\PYGZgt{}}Ciudad R.\PYG{p}{\PYGZlt{}}\PYG{p}{/}\PYG{n+nt}{option}\PYG{p}{\PYGZgt{}}
        \PYG{p}{\PYGZlt{}}\PYG{n+nt}{option} \PYG{n+na}{value}\PYG{o}{=}\PYG{l+s}{\PYGZdq{}CU\PYGZdq{}}\PYG{p}{\PYGZgt{}}Cuenca\PYG{p}{\PYGZlt{}}\PYG{p}{/}\PYG{n+nt}{option}\PYG{p}{\PYGZgt{}}
\PYG{p}{\PYGZlt{}}\PYG{p}{/}\PYG{n+nt}{select}\PYG{p}{\PYGZgt{}}
\end{sphinxVerbatim}

Se debe recordar que el texto que ven los usuarios es lo que va entre las etiquetas option. El valor que comprobarán los programadores es lo que va en value En una lista desplegable se pueden elegir muchos valores usando el atributo \sphinxcode{multiple}.

Si se desea que una opción (o varias si usamos el selector múltiple) aparezca marcada se debe usar \sphinxcode{selected="selected"}.


\subsection{Textareas}
\label{\detokenize{tema2:textareas}}
Permiten introducir textos muy largos:

\begin{sphinxVerbatim}[commandchars=\\\{\}]
\PYG{p}{\PYGZlt{}}\PYG{n+nt}{textarea} \PYG{n+na}{rows}\PYG{o}{=}\PYG{l+s}{\PYGZdq{}10\PYGZdq{}} \PYG{n+na}{cols}\PYG{o}{=}\PYG{l+s}{\PYGZdq{}15\PYGZdq{}}\PYG{p}{\PYGZgt{}}
        Valor por defecto
\PYG{p}{\PYGZlt{}}\PYG{p}{/}\PYG{n+nt}{textarea}\PYG{p}{\PYGZgt{}}
\end{sphinxVerbatim}


\section{Ejercicios tipo examen}
\label{\detokenize{tema2:ejercicios-tipo-examen}}

\subsection{Enunciado}
\label{\detokenize{tema2:enunciado}}
\sphinxstyleemphasis{Crea una tabla con la estructura siguiente}

\noindent{\hspace*{\fill}\sphinxincludegraphics[scale=0.5]{{tabla5}.png}\hspace*{\fill}}


\subsection{Solución}
\label{\detokenize{tema2:id7}}
El siguiente HTML produce algo muy parecido:

\begin{sphinxVerbatim}[commandchars=\\\{\}]
\PYG{p}{\PYGZlt{}}\PYG{n+nt}{table} \PYG{n+na}{border}\PYG{o}{=}\PYG{l+s}{\PYGZdq{}1\PYGZdq{}}\PYG{p}{\PYGZgt{}}
        \PYG{p}{\PYGZlt{}}\PYG{n+nt}{tr}\PYG{p}{\PYGZgt{}}
                \PYG{p}{\PYGZlt{}}\PYG{n+nt}{td}\PYG{p}{\PYGZgt{}}Celda 1\PYG{p}{\PYGZlt{}}\PYG{p}{/}\PYG{n+nt}{td}\PYG{p}{\PYGZgt{}}
                \PYG{p}{\PYGZlt{}}\PYG{n+nt}{td}\PYG{p}{\PYGZgt{}}Celda 2\PYG{p}{\PYGZlt{}}\PYG{p}{/}\PYG{n+nt}{td}\PYG{p}{\PYGZgt{}}
        \PYG{p}{\PYGZlt{}}\PYG{p}{/}\PYG{n+nt}{tr}\PYG{p}{\PYGZgt{}}
        \PYG{p}{\PYGZlt{}}\PYG{n+nt}{tr}\PYG{p}{\PYGZgt{}}
                \PYG{p}{\PYGZlt{}}\PYG{n+nt}{td}\PYG{p}{\PYGZgt{}}
                        \PYG{p}{\PYGZlt{}}\PYG{n+nt}{table} \PYG{n+na}{border}\PYG{o}{=}\PYG{l+s}{\PYGZdq{}1\PYGZdq{}}\PYG{p}{\PYGZgt{}}
                                \PYG{p}{\PYGZlt{}}\PYG{n+nt}{tr}\PYG{p}{\PYGZgt{}}
                                        \PYG{p}{\PYGZlt{}}\PYG{n+nt}{td}\PYG{p}{\PYGZgt{}}Celda 3a\PYG{p}{\PYGZlt{}}\PYG{p}{/}\PYG{n+nt}{td}\PYG{p}{\PYGZgt{}}
                                        \PYG{p}{\PYGZlt{}}\PYG{n+nt}{td}\PYG{p}{\PYGZgt{}}Celda 3b\PYG{p}{\PYGZlt{}}\PYG{p}{/}\PYG{n+nt}{td}\PYG{p}{\PYGZgt{}}
                                        \PYG{p}{\PYGZlt{}}\PYG{n+nt}{td}\PYG{p}{\PYGZgt{}}Celda 3c\PYG{p}{\PYGZlt{}}\PYG{p}{/}\PYG{n+nt}{td}\PYG{p}{\PYGZgt{}}
                                \PYG{p}{\PYGZlt{}}\PYG{p}{/}\PYG{n+nt}{tr}\PYG{p}{\PYGZgt{}}
                        \PYG{p}{\PYGZlt{}}\PYG{p}{/}\PYG{n+nt}{table}\PYG{p}{\PYGZgt{}}
                \PYG{p}{\PYGZlt{}}\PYG{p}{/}\PYG{n+nt}{td}\PYG{p}{\PYGZgt{}}
                \PYG{p}{\PYGZlt{}}\PYG{n+nt}{td}\PYG{p}{\PYGZgt{}}
                        \PYG{p}{\PYGZlt{}}\PYG{n+nt}{table} \PYG{n+na}{border}\PYG{o}{=}\PYG{l+s}{\PYGZdq{}1\PYGZdq{}}\PYG{p}{\PYGZgt{}}
                                \PYG{p}{\PYGZlt{}}\PYG{n+nt}{tr}\PYG{p}{\PYGZgt{}}
                                        \PYG{p}{\PYGZlt{}}\PYG{n+nt}{td}\PYG{p}{\PYGZgt{}}Celda 4a\PYG{p}{\PYGZlt{}}\PYG{p}{/}\PYG{n+nt}{td}\PYG{p}{\PYGZgt{}}
                                \PYG{p}{\PYGZlt{}}\PYG{p}{/}\PYG{n+nt}{tr}\PYG{p}{\PYGZgt{}}
                                \PYG{p}{\PYGZlt{}}\PYG{n+nt}{tr}\PYG{p}{\PYGZgt{}}
                                        \PYG{p}{\PYGZlt{}}\PYG{n+nt}{td}\PYG{p}{\PYGZgt{}}Celda 4b\PYG{p}{\PYGZlt{}}\PYG{p}{/}\PYG{n+nt}{td}\PYG{p}{\PYGZgt{}}
                                \PYG{p}{\PYGZlt{}}\PYG{p}{/}\PYG{n+nt}{tr}\PYG{p}{\PYGZgt{}}
                                \PYG{p}{\PYGZlt{}}\PYG{n+nt}{tr}\PYG{p}{\PYGZgt{}}
                                        \PYG{p}{\PYGZlt{}}\PYG{n+nt}{td}\PYG{p}{\PYGZgt{}}Celda 4c\PYG{p}{\PYGZlt{}}\PYG{p}{/}\PYG{n+nt}{td}\PYG{p}{\PYGZgt{}}
                                \PYG{p}{\PYGZlt{}}\PYG{p}{/}\PYG{n+nt}{tr}\PYG{p}{\PYGZgt{}}
                        \PYG{p}{\PYGZlt{}}\PYG{p}{/}\PYG{n+nt}{table}\PYG{p}{\PYGZgt{}}
                \PYG{p}{\PYGZlt{}}\PYG{p}{/}\PYG{n+nt}{td}\PYG{p}{\PYGZgt{}}
        \PYG{p}{\PYGZlt{}}\PYG{p}{/}\PYG{n+nt}{tr}\PYG{p}{\PYGZgt{}}
\PYG{p}{\PYGZlt{}}\PYG{p}{/}\PYG{n+nt}{table}\PYG{p}{\PYGZgt{}}
\end{sphinxVerbatim}


\subsection{Enunciado}
\label{\detokenize{tema2:id8}}
\sphinxstyleemphasis{Crea una página HTML que produzca este resultado}

\begin{figure}[htbp]
\centering
\capstart

\noindent\sphinxincludegraphics{{tabla6}.png}
\caption{Una tabla compleja}\label{\detokenize{tema2:id17}}\end{figure}


\subsection{Solución}
\label{\detokenize{tema2:id9}}
El HTML siguiente produce el resultado pedido:

\begin{sphinxVerbatim}[commandchars=\\\{\}]
\PYG{p}{\PYGZlt{}}\PYG{n+nt}{table} \PYG{n+na}{border}\PYG{o}{=}\PYG{l+s}{\PYGZdq{}1\PYGZdq{}}\PYG{p}{\PYGZgt{}}
        \PYG{p}{\PYGZlt{}}\PYG{n+nt}{tr}\PYG{p}{\PYGZgt{}}\PYG{c}{\PYGZlt{}!\PYGZhy{}\PYGZhy{}}\PYG{c}{Primera fila}\PYG{c}{\PYGZhy{}\PYGZhy{}\PYGZgt{}}
                \PYG{p}{\PYGZlt{}}\PYG{n+nt}{td}\PYG{p}{\PYGZgt{}}
                        \PYG{p}{\PYGZlt{}}\PYG{n+nt}{table} \PYG{n+na}{border}\PYG{o}{=}\PYG{l+s}{\PYGZdq{}1\PYGZdq{}}\PYG{p}{\PYGZgt{}}
                                \PYG{p}{\PYGZlt{}}\PYG{n+nt}{tr}\PYG{p}{\PYGZgt{}}
                                        \PYG{p}{\PYGZlt{}}\PYG{n+nt}{td}\PYG{p}{\PYGZgt{}}1a\PYG{p}{\PYGZlt{}}\PYG{p}{/}\PYG{n+nt}{td}\PYG{p}{\PYGZgt{}}
                                        \PYG{p}{\PYGZlt{}}\PYG{n+nt}{td}\PYG{p}{\PYGZgt{}}1b\PYG{p}{\PYGZlt{}}\PYG{p}{/}\PYG{n+nt}{td}\PYG{p}{\PYGZgt{}}
                                \PYG{p}{\PYGZlt{}}\PYG{p}{/}\PYG{n+nt}{tr}\PYG{p}{\PYGZgt{}}
                        \PYG{p}{\PYGZlt{}}\PYG{p}{/}\PYG{n+nt}{table}\PYG{p}{\PYGZgt{}}
                \PYG{p}{\PYGZlt{}}\PYG{p}{/}\PYG{n+nt}{td}\PYG{p}{\PYGZgt{}}
                \PYG{p}{\PYGZlt{}}\PYG{n+nt}{td}\PYG{p}{\PYGZgt{}}
                        \PYG{p}{\PYGZlt{}}\PYG{n+nt}{table} \PYG{n+na}{border}\PYG{o}{=}\PYG{l+s}{\PYGZdq{}1\PYGZdq{}}\PYG{p}{\PYGZgt{}}
                                \PYG{p}{\PYGZlt{}}\PYG{n+nt}{tr}\PYG{p}{\PYGZgt{}}
                                        \PYG{p}{\PYGZlt{}}\PYG{n+nt}{td}\PYG{p}{\PYGZgt{}}2a\PYG{p}{\PYGZlt{}}\PYG{p}{/}\PYG{n+nt}{td}\PYG{p}{\PYGZgt{}}
                                        \PYG{p}{\PYGZlt{}}\PYG{n+nt}{td}\PYG{p}{\PYGZgt{}}2b\PYG{p}{\PYGZlt{}}\PYG{p}{/}\PYG{n+nt}{td}\PYG{p}{\PYGZgt{}}
                                \PYG{p}{\PYGZlt{}}\PYG{p}{/}\PYG{n+nt}{tr}\PYG{p}{\PYGZgt{}}
                                \PYG{p}{\PYGZlt{}}\PYG{n+nt}{tr}\PYG{p}{\PYGZgt{}}
                                        \PYG{p}{\PYGZlt{}}\PYG{n+nt}{td}\PYG{p}{\PYGZgt{}}2c\PYG{p}{\PYGZlt{}}\PYG{p}{/}\PYG{n+nt}{td}\PYG{p}{\PYGZgt{}}
                                        \PYG{p}{\PYGZlt{}}\PYG{n+nt}{td}\PYG{p}{\PYGZgt{}}2c\PYG{p}{\PYGZlt{}}\PYG{p}{/}\PYG{n+nt}{td}\PYG{p}{\PYGZgt{}}
                                \PYG{p}{\PYGZlt{}}\PYG{p}{/}\PYG{n+nt}{tr}\PYG{p}{\PYGZgt{}}
                        \PYG{p}{\PYGZlt{}}\PYG{p}{/}\PYG{n+nt}{table}\PYG{p}{\PYGZgt{}}
                \PYG{p}{\PYGZlt{}}\PYG{p}{/}\PYG{n+nt}{td}\PYG{p}{\PYGZgt{}}
        \PYG{p}{\PYGZlt{}}\PYG{p}{/}\PYG{n+nt}{tr}\PYG{p}{\PYGZgt{}}
        \PYG{p}{\PYGZlt{}}\PYG{n+nt}{tr}\PYG{p}{\PYGZgt{}}\PYG{c}{\PYGZlt{}!\PYGZhy{}\PYGZhy{}}\PYG{c}{Segunda fila}\PYG{c}{\PYGZhy{}\PYGZhy{}\PYGZgt{}}
                \PYG{p}{\PYGZlt{}}\PYG{n+nt}{td}\PYG{p}{\PYGZgt{}}
                        3a
                \PYG{p}{\PYGZlt{}}\PYG{p}{/}\PYG{n+nt}{td}\PYG{p}{\PYGZgt{}}
                \PYG{p}{\PYGZlt{}}\PYG{n+nt}{td}\PYG{p}{\PYGZgt{}}
                        \PYG{p}{\PYGZlt{}}\PYG{n+nt}{table} \PYG{n+na}{border}\PYG{o}{=}\PYG{l+s}{\PYGZdq{}1\PYGZdq{}}\PYG{p}{\PYGZgt{}}
                                \PYG{p}{\PYGZlt{}}\PYG{n+nt}{tr}\PYG{p}{\PYGZgt{}}
                                        \PYG{p}{\PYGZlt{}}\PYG{n+nt}{td}\PYG{p}{\PYGZgt{}}4a1\PYG{p}{\PYGZlt{}}\PYG{p}{/}\PYG{n+nt}{td}\PYG{p}{\PYGZgt{}}
                                        \PYG{p}{\PYGZlt{}}\PYG{n+nt}{td}\PYG{p}{\PYGZgt{}}4a2\PYG{p}{\PYGZlt{}}\PYG{p}{/}\PYG{n+nt}{td}\PYG{p}{\PYGZgt{}}
                                        \PYG{p}{\PYGZlt{}}\PYG{n+nt}{td}\PYG{p}{\PYGZgt{}}4a3\PYG{p}{\PYGZlt{}}\PYG{p}{/}\PYG{n+nt}{td}\PYG{p}{\PYGZgt{}}
                                \PYG{p}{\PYGZlt{}}\PYG{p}{/}\PYG{n+nt}{tr}\PYG{p}{\PYGZgt{}}
                        \PYG{p}{\PYGZlt{}}\PYG{p}{/}\PYG{n+nt}{table}\PYG{p}{\PYGZgt{}}
                \PYG{p}{\PYGZlt{}}\PYG{p}{/}\PYG{n+nt}{td}\PYG{p}{\PYGZgt{}}
        \PYG{p}{\PYGZlt{}}\PYG{p}{/}\PYG{n+nt}{tr}\PYG{p}{\PYGZgt{}}
        \PYG{p}{\PYGZlt{}}\PYG{n+nt}{tr}\PYG{p}{\PYGZgt{}} \PYG{c}{\PYGZlt{}!\PYGZhy{}\PYGZhy{}}\PYG{c}{Tercera fila}\PYG{c}{\PYGZhy{}\PYGZhy{}\PYGZgt{}}
                \PYG{p}{\PYGZlt{}}\PYG{n+nt}{td}\PYG{p}{\PYGZgt{}}
                        \PYG{p}{\PYGZlt{}}\PYG{n+nt}{table} \PYG{n+na}{border}\PYG{o}{=}\PYG{l+s}{\PYGZdq{}1\PYGZdq{}}\PYG{p}{\PYGZgt{}}
                                \PYG{p}{\PYGZlt{}}\PYG{n+nt}{tr}\PYG{p}{\PYGZgt{}}
                                        \PYG{p}{\PYGZlt{}}\PYG{n+nt}{td}\PYG{p}{\PYGZgt{}}3b1\PYG{p}{\PYGZlt{}}\PYG{p}{/}\PYG{n+nt}{td}\PYG{p}{\PYGZgt{}}
                                        \PYG{p}{\PYGZlt{}}\PYG{n+nt}{td}\PYG{p}{\PYGZgt{}}3b2\PYG{p}{\PYGZlt{}}\PYG{p}{/}\PYG{n+nt}{td}\PYG{p}{\PYGZgt{}}
                                \PYG{p}{\PYGZlt{}}\PYG{p}{/}\PYG{n+nt}{tr}\PYG{p}{\PYGZgt{}}
                                \PYG{p}{\PYGZlt{}}\PYG{n+nt}{tr}\PYG{p}{\PYGZgt{}}
                                        \PYG{p}{\PYGZlt{}}\PYG{n+nt}{td}\PYG{p}{\PYGZgt{}}3b3\PYG{p}{\PYGZlt{}}\PYG{p}{/}\PYG{n+nt}{td}\PYG{p}{\PYGZgt{}}
                                        \PYG{p}{\PYGZlt{}}\PYG{n+nt}{td}\PYG{p}{\PYGZgt{}}3b4\PYG{p}{\PYGZlt{}}\PYG{p}{/}\PYG{n+nt}{td}\PYG{p}{\PYGZgt{}}
                                \PYG{p}{\PYGZlt{}}\PYG{p}{/}\PYG{n+nt}{tr}\PYG{p}{\PYGZgt{}}
                        \PYG{p}{\PYGZlt{}}\PYG{p}{/}\PYG{n+nt}{table}\PYG{p}{\PYGZgt{}}
                \PYG{p}{\PYGZlt{}}\PYG{p}{/}\PYG{n+nt}{td}\PYG{p}{\PYGZgt{}}
                \PYG{p}{\PYGZlt{}}\PYG{n+nt}{td}\PYG{p}{\PYGZgt{}}
                        4b
                \PYG{p}{\PYGZlt{}}\PYG{p}{/}\PYG{n+nt}{td}\PYG{p}{\PYGZgt{}}
        \PYG{p}{\PYGZlt{}}\PYG{p}{/}\PYG{n+nt}{tr}\PYG{p}{\PYGZgt{}}
\PYG{p}{\PYGZlt{}}\PYG{p}{/}\PYG{n+nt}{table}\PYG{p}{\PYGZgt{}}
\end{sphinxVerbatim}


\subsection{Enunciado}
\label{\detokenize{tema2:id10}}
\sphinxstyleemphasis{Crea una página HTML que produzca este resultado}

\noindent{\hspace*{\fill}\sphinxincludegraphics[scale=0.5]{{maqueta1}.png}\hspace*{\fill}}


\subsection{Solución}
\label{\detokenize{tema2:id11}}

\subsection{Enunciado}
\label{\detokenize{tema2:id12}}
\sphinxstyleemphasis{Crea un formulario como este donde haya 3 opciones en la lista desplegable: «Más de 400», «Menos de 400», «Variables»}

\noindent{\hspace*{\fill}\sphinxincludegraphics[scale=0.5]{{maqueta2}.png}\hspace*{\fill}}


\subsection{Solución}
\label{\detokenize{tema2:id13}}
El HTML siguiente produce el resultado que nos piden

\begin{sphinxVerbatim}[commandchars=\\\{\}]
\PYG{c+cp}{\PYGZlt{}!DOCTYPE html\PYGZgt{}}
\PYG{p}{\PYGZlt{}}\PYG{n+nt}{html}\PYG{p}{\PYGZgt{}}
\PYG{p}{\PYGZlt{}}\PYG{n+nt}{head}\PYG{p}{\PYGZgt{}}
    \PYG{p}{\PYGZlt{}}\PYG{n+nt}{meta} \PYG{n+na}{charset}\PYG{o}{=}\PYG{l+s}{\PYGZdq{}utf\PYGZhy{}8\PYGZdq{}}\PYG{p}{\PYGZgt{}}
    \PYG{p}{\PYGZlt{}}\PYG{n+nt}{title}\PYG{p}{\PYGZgt{}}Formulario fiscal\PYG{p}{\PYGZlt{}}\PYG{p}{/}\PYG{n+nt}{title}\PYG{p}{\PYGZgt{}}
\PYG{p}{\PYGZlt{}}\PYG{p}{/}\PYG{n+nt}{head}\PYG{p}{\PYGZgt{}}
\PYG{p}{\PYGZlt{}}\PYG{n+nt}{body}\PYG{p}{\PYGZgt{}}

\PYG{p}{\PYGZlt{}}\PYG{n+nt}{form}\PYG{p}{\PYGZgt{}}
   \PYG{p}{\PYGZlt{}}\PYG{n+nt}{fieldset}\PYG{p}{\PYGZgt{}}
    \PYG{p}{\PYGZlt{}}\PYG{n+nt}{legend}\PYG{p}{\PYGZgt{}}
        Datos fiscales
    \PYG{p}{\PYGZlt{}}\PYG{p}{/}\PYG{n+nt}{legend}\PYG{p}{\PYGZgt{}}
    \PYG{p}{\PYGZlt{}}\PYG{n+nt}{input} \PYG{n+na}{type}\PYG{o}{=}\PYG{l+s}{\PYGZdq{}checkbox\PYGZdq{}} \PYG{n+na}{id}\PYG{o}{=}\PYG{l+s}{\PYGZdq{}enparo\PYGZdq{}}
        \PYG{n+na}{name}\PYG{o}{=}\PYG{l+s}{\PYGZdq{}sit\PYGZus{}laboral\PYGZdq{}}\PYG{p}{\PYGZgt{}}
    \PYG{p}{\PYGZlt{}}\PYG{n+nt}{label} \PYG{n+na}{for}\PYG{o}{=}\PYG{l+s}{\PYGZdq{}enparo\PYGZdq{}}\PYG{p}{\PYGZgt{}}En paro\PYG{p}{\PYGZlt{}}\PYG{p}{/}\PYG{n+nt}{label}\PYG{p}{\PYGZgt{}}
    \PYG{p}{\PYGZlt{}}\PYG{n+nt}{br}\PYG{p}{/}\PYG{p}{\PYGZgt{}}
    \PYG{p}{\PYGZlt{}}\PYG{n+nt}{input} \PYG{n+na}{type}\PYG{o}{=}\PYG{l+s}{\PYGZdq{}checkbox\PYGZdq{}} \PYG{n+na}{id}\PYG{o}{=}\PYG{l+s}{\PYGZdq{}autonomo\PYGZdq{}}
        \PYG{n+na}{name}\PYG{o}{=}\PYG{l+s}{\PYGZdq{}sit\PYGZus{}laboral\PYGZdq{}}\PYG{p}{\PYGZgt{}}
    \PYG{p}{\PYGZlt{}}\PYG{n+nt}{label} \PYG{n+na}{for}\PYG{o}{=}\PYG{l+s}{\PYGZdq{}autonomo\PYGZdq{}}\PYG{p}{\PYGZgt{}}Autónomo\PYG{p}{\PYGZlt{}}\PYG{p}{/}\PYG{n+nt}{label}\PYG{p}{\PYGZgt{}}
    \PYG{p}{\PYGZlt{}}\PYG{n+nt}{br}\PYG{p}{/}\PYG{p}{\PYGZgt{}}
    \PYG{p}{\PYGZlt{}}\PYG{n+nt}{input} \PYG{n+na}{type}\PYG{o}{=}\PYG{l+s}{\PYGZdq{}checkbox\PYGZdq{}} \PYG{n+na}{id}\PYG{o}{=}\PYG{l+s}{\PYGZdq{}c\PYGZus{}ajena\PYGZdq{}}
        \PYG{n+na}{name}\PYG{o}{=}\PYG{l+s}{\PYGZdq{}sit\PYGZus{}laboral\PYGZdq{}}\PYG{p}{\PYGZgt{}}
    \PYG{p}{\PYGZlt{}}\PYG{n+nt}{label} \PYG{n+na}{for}\PYG{o}{=}\PYG{l+s}{\PYGZdq{}c\PYGZus{}ajena\PYGZdq{}}\PYG{p}{\PYGZgt{}}Por c.ajena\PYG{p}{\PYGZlt{}}\PYG{p}{/}\PYG{n+nt}{label}\PYG{p}{\PYGZgt{}}
    \PYG{p}{\PYGZlt{}}\PYG{n+nt}{br}\PYG{p}{/}\PYG{p}{\PYGZgt{}}
   \PYG{p}{\PYGZlt{}}\PYG{p}{/}\PYG{n+nt}{fieldset}\PYG{p}{\PYGZgt{}}
   \PYG{p}{\PYGZlt{}}\PYG{n+nt}{fieldset}\PYG{p}{\PYGZgt{}}
     \PYG{p}{\PYGZlt{}}\PYG{n+nt}{legend}\PYG{p}{\PYGZgt{}}Datos personales\PYG{p}{\PYGZlt{}}\PYG{p}{/}\PYG{n+nt}{legend}\PYG{p}{\PYGZgt{}}
     \PYG{p}{\PYGZlt{}}\PYG{n+nt}{label} \PYG{n+na}{for}\PYG{o}{=}\PYG{l+s}{\PYGZdq{}nombre\PYGZdq{}}\PYG{p}{\PYGZgt{}}Nombre\PYG{p}{\PYGZlt{}}\PYG{p}{/}\PYG{n+nt}{label}\PYG{p}{\PYGZgt{}}
     \PYG{p}{\PYGZlt{}}\PYG{n+nt}{input} \PYG{n+na}{type}\PYG{o}{=}\PYG{l+s}{\PYGZdq{}text\PYGZdq{}} \PYG{n+na}{id}\PYG{o}{=}\PYG{l+s}{\PYGZdq{}nombre\PYGZdq{}}\PYG{p}{\PYGZgt{}}
     \PYG{p}{\PYGZlt{}}\PYG{n+nt}{br}\PYG{p}{/}\PYG{p}{\PYGZgt{}}
     \PYG{p}{\PYGZlt{}}\PYG{n+nt}{label} \PYG{n+na}{for}\PYG{o}{=}\PYG{l+s}{\PYGZdq{}apellidos\PYGZdq{}}\PYG{p}{\PYGZgt{}}Apellidos\PYG{p}{\PYGZlt{}}\PYG{p}{/}\PYG{n+nt}{label}\PYG{p}{\PYGZgt{}}
     \PYG{p}{\PYGZlt{}}\PYG{n+nt}{input} \PYG{n+na}{type}\PYG{o}{=}\PYG{l+s}{\PYGZdq{}text\PYGZdq{}} \PYG{n+na}{id}\PYG{o}{=}\PYG{l+s}{\PYGZdq{}apellidos\PYGZdq{}}\PYG{p}{\PYGZgt{}}
     \PYG{p}{\PYGZlt{}}\PYG{n+nt}{br}\PYG{p}{/}\PYG{p}{\PYGZgt{}}
     \PYG{p}{\PYGZlt{}}\PYG{n+nt}{label} \PYG{n+na}{for}\PYG{o}{=}\PYG{l+s}{\PYGZdq{}sueldo\PYGZdq{}}\PYG{p}{\PYGZgt{}}Sueldo\PYG{p}{\PYGZlt{}}\PYG{p}{/}\PYG{n+nt}{label}\PYG{p}{\PYGZgt{}}
     \PYG{p}{\PYGZlt{}}\PYG{n+nt}{select} \PYG{n+na}{id}\PYG{o}{=}\PYG{l+s}{\PYGZdq{}sueldo\PYGZdq{}}\PYG{p}{\PYGZgt{}}\PYG{p}{\PYGZlt{}}\PYG{p}{/}\PYG{n+nt}{select}\PYG{p}{\PYGZgt{}}
       \PYG{p}{\PYGZlt{}}\PYG{n+nt}{option}\PYG{p}{\PYGZgt{}}Más de 400 euros\PYG{p}{\PYGZlt{}}\PYG{p}{/}\PYG{n+nt}{option}\PYG{p}{\PYGZgt{}}
       \PYG{p}{\PYGZlt{}}\PYG{n+nt}{option}\PYG{p}{\PYGZgt{}}Menos de 400 euros\PYG{p}{\PYGZlt{}}\PYG{p}{/}\PYG{n+nt}{option}\PYG{p}{\PYGZgt{}}
       \PYG{p}{\PYGZlt{}}\PYG{n+nt}{option}\PYG{p}{\PYGZgt{}}Variable\PYG{p}{\PYGZlt{}}\PYG{p}{/}\PYG{n+nt}{option}\PYG{p}{\PYGZgt{}}
     \PYG{p}{\PYGZlt{}}\PYG{p}{/}\PYG{n+nt}{select}\PYG{p}{\PYGZgt{}}
     \PYG{p}{\PYGZlt{}}\PYG{n+nt}{br}\PYG{p}{/}\PYG{p}{\PYGZgt{}}
     \PYG{p}{\PYGZlt{}}\PYG{n+nt}{input} \PYG{n+na}{type}\PYG{o}{=}\PYG{l+s}{\PYGZdq{}checkbox\PYGZdq{}} \PYG{n+na}{id}\PYG{o}{=}\PYG{l+s}{\PYGZdq{}con\PYGZus{}ep\PYGZdq{}}\PYG{p}{\PYGZgt{}}
     \PYG{p}{\PYGZlt{}}\PYG{n+nt}{label} \PYG{n+na}{for}\PYG{o}{=}\PYG{l+s}{\PYGZdq{}con\PYGZus{}ep\PYGZdq{}}\PYG{p}{\PYGZgt{}}
         Con enfermedad profesional
     \PYG{p}{\PYGZlt{}}\PYG{p}{/}\PYG{n+nt}{label}\PYG{p}{\PYGZgt{}} \PYG{p}{\PYGZlt{}}\PYG{n+nt}{br}\PYG{p}{/}\PYG{p}{\PYGZgt{}}
     \PYG{p}{\PYGZlt{}}\PYG{n+nt}{input} \PYG{n+na}{type}\PYG{o}{=}\PYG{l+s}{\PYGZdq{}checkbox\PYGZdq{}} \PYG{n+na}{id}\PYG{o}{=}\PYG{l+s}{\PYGZdq{}con\PYGZus{}padres\PYGZdq{}}\PYG{p}{\PYGZgt{}}
     \PYG{p}{\PYGZlt{}}\PYG{n+nt}{label} \PYG{n+na}{for}\PYG{o}{=}\PYG{l+s}{\PYGZdq{}con\PYGZus{}padres\PYGZdq{}}\PYG{p}{\PYGZgt{}}
         Con padres a cargo
     \PYG{p}{\PYGZlt{}}\PYG{p}{/}\PYG{n+nt}{label}\PYG{p}{\PYGZgt{}} \PYG{p}{\PYGZlt{}}\PYG{n+nt}{br}\PYG{p}{/}\PYG{p}{\PYGZgt{}}
     \PYG{p}{\PYGZlt{}}\PYG{n+nt}{input} \PYG{n+na}{type}\PYG{o}{=}\PYG{l+s}{\PYGZdq{}checkbox\PYGZdq{}} \PYG{n+na}{id}\PYG{o}{=}\PYG{l+s}{\PYGZdq{}con\PYGZus{}hijos\PYGZdq{}}\PYG{p}{\PYGZgt{}}
     \PYG{p}{\PYGZlt{}}\PYG{n+nt}{label} \PYG{n+na}{for}\PYG{o}{=}\PYG{l+s}{\PYGZdq{}con\PYGZus{}hijos\PYGZdq{}}\PYG{p}{\PYGZgt{}}
         Con hijos a cargo
     \PYG{p}{\PYGZlt{}}\PYG{p}{/}\PYG{n+nt}{label}\PYG{p}{\PYGZgt{}} \PYG{p}{\PYGZlt{}}\PYG{n+nt}{br}\PYG{p}{/}\PYG{p}{\PYGZgt{}}
   \PYG{p}{\PYGZlt{}}\PYG{p}{/}\PYG{n+nt}{fieldset}\PYG{p}{\PYGZgt{}}   
\PYG{p}{\PYGZlt{}}\PYG{p}{/}\PYG{n+nt}{form}\PYG{p}{\PYGZgt{}}
\PYG{p}{\PYGZlt{}}\PYG{p}{/}\PYG{n+nt}{body}\PYG{p}{\PYGZgt{}}
\PYG{p}{\PYGZlt{}}\PYG{p}{/}\PYG{n+nt}{html}\PYG{p}{\PYGZgt{}}
\end{sphinxVerbatim}


\subsection{Enunciado}
\label{\detokenize{tema2:id14}}
\sphinxstyleemphasis{Crea un formulario como este}

\noindent{\hspace*{\fill}\sphinxincludegraphics[scale=0.5]{{maqueta4}.png}\hspace*{\fill}}

\begin{sphinxVerbatim}[commandchars=\\\{\}]
\PYG{c+cp}{\PYGZlt{}!DOCTYPE html\PYGZgt{}}
\PYG{p}{\PYGZlt{}}\PYG{n+nt}{html}\PYG{p}{\PYGZgt{}}
\PYG{p}{\PYGZlt{}}\PYG{n+nt}{head}\PYG{p}{\PYGZgt{}}
    \PYG{p}{\PYGZlt{}}\PYG{n+nt}{meta} \PYG{n+na}{charset}\PYG{o}{=}\PYG{l+s}{\PYGZdq{}utf\PYGZhy{}8\PYGZdq{}}\PYG{p}{\PYGZgt{}}
    \PYG{p}{\PYGZlt{}}\PYG{n+nt}{title}\PYG{p}{\PYGZgt{}}Formulario\PYG{p}{\PYGZlt{}}\PYG{p}{/}\PYG{n+nt}{title}\PYG{p}{\PYGZgt{}}
\PYG{p}{\PYGZlt{}}\PYG{p}{/}\PYG{n+nt}{head}\PYG{p}{\PYGZgt{}}
\PYG{p}{\PYGZlt{}}\PYG{n+nt}{body}\PYG{p}{\PYGZgt{}}
\PYG{p}{\PYGZlt{}}\PYG{n+nt}{form}\PYG{p}{\PYGZgt{}}
    \PYG{p}{\PYGZlt{}}\PYG{n+nt}{fieldset}\PYG{p}{\PYGZgt{}}
        \PYG{p}{\PYGZlt{}}\PYG{n+nt}{legend}\PYG{p}{\PYGZgt{}}Datos fiscales\PYG{p}{\PYGZlt{}}\PYG{p}{/}\PYG{n+nt}{legend}\PYG{p}{\PYGZgt{}}
        \PYG{p}{\PYGZlt{}}\PYG{n+nt}{select} \PYG{n+na}{multiple}\PYG{o}{=}\PYG{l+s}{\PYGZdq{}multiple\PYGZdq{}}\PYG{p}{\PYGZgt{}}
            \PYG{p}{\PYGZlt{}}\PYG{n+nt}{option}\PYG{p}{\PYGZgt{}}Automoción\PYG{p}{\PYGZlt{}}\PYG{p}{/}\PYG{n+nt}{option}\PYG{p}{\PYGZgt{}}
            \PYG{p}{\PYGZlt{}}\PYG{n+nt}{option} \PYG{n+na}{selected}\PYG{o}{=}\PYG{l+s}{\PYGZdq{}selected\PYGZdq{}}\PYG{p}{\PYGZgt{}}
                Metal
            \PYG{p}{\PYGZlt{}}\PYG{p}{/}\PYG{n+nt}{option}\PYG{p}{\PYGZgt{}}
            \PYG{p}{\PYGZlt{}}\PYG{n+nt}{option}\PYG{p}{\PYGZgt{}}Informática\PYG{p}{\PYGZlt{}}\PYG{p}{/}\PYG{n+nt}{option}\PYG{p}{\PYGZgt{}}
            \PYG{p}{\PYGZlt{}}\PYG{n+nt}{option} \PYG{n+na}{selected}\PYG{o}{=}\PYG{l+s}{\PYGZdq{}selected\PYGZdq{}}\PYG{p}{\PYGZgt{}}
                Finanzas
            \PYG{p}{\PYGZlt{}}\PYG{p}{/}\PYG{n+nt}{option}\PYG{p}{\PYGZgt{}}
        \PYG{p}{\PYGZlt{}}\PYG{p}{/}\PYG{n+nt}{select}\PYG{p}{\PYGZgt{}}
        \PYG{p}{\PYGZlt{}}\PYG{n+nt}{br}\PYG{p}{/}\PYG{p}{\PYGZgt{}}
        \PYG{p}{\PYGZlt{}}\PYG{n+nt}{input} \PYG{n+na}{type}\PYG{o}{=}\PYG{l+s}{\PYGZdq{}checkbox\PYGZdq{}} \PYG{n+na}{id}\PYG{o}{=}\PYG{l+s}{\PYGZdq{}autonomo\PYGZdq{}}\PYG{p}{\PYGZgt{}}
        \PYG{p}{\PYGZlt{}}\PYG{n+nt}{label} \PYG{n+na}{for}\PYG{o}{=}\PYG{l+s}{\PYGZdq{}autonomo\PYGZdq{}}\PYG{p}{\PYGZgt{}}
            Autónomo
        \PYG{p}{\PYGZlt{}}\PYG{p}{/}\PYG{n+nt}{label}\PYG{p}{\PYGZgt{}}
        \PYG{p}{\PYGZlt{}}\PYG{n+nt}{input} \PYG{n+na}{type}\PYG{o}{=}\PYG{l+s}{\PYGZdq{}checkbox\PYGZdq{}} \PYG{n+na}{id}\PYG{o}{=}\PYG{l+s}{\PYGZdq{}c\PYGZus{}ajena\PYGZdq{}}\PYG{p}{\PYGZgt{}}
        \PYG{p}{\PYGZlt{}}\PYG{n+nt}{label} \PYG{n+na}{for}\PYG{o}{=}\PYG{l+s}{\PYGZdq{}c\PYGZus{}ajena\PYGZdq{}}\PYG{p}{\PYGZgt{}}
            Por c. ajena
        \PYG{p}{\PYGZlt{}}\PYG{p}{/}\PYG{n+nt}{label}\PYG{p}{\PYGZgt{}}
        \PYG{p}{\PYGZlt{}}\PYG{n+nt}{input} \PYG{n+na}{type}\PYG{o}{=}\PYG{l+s}{\PYGZdq{}checkbox\PYGZdq{}} \PYG{n+na}{id}\PYG{o}{=}\PYG{l+s}{\PYGZdq{}nosabe\PYGZdq{}}\PYG{p}{\PYGZgt{}}
        \PYG{p}{\PYGZlt{}}\PYG{n+nt}{label} \PYG{n+na}{for}\PYG{o}{=}\PYG{l+s}{\PYGZdq{}nosabe\PYGZdq{}}\PYG{p}{\PYGZgt{}}
            No sabe
        \PYG{p}{\PYGZlt{}}\PYG{p}{/}\PYG{n+nt}{label}\PYG{p}{\PYGZgt{}}
        \PYG{p}{\PYGZlt{}}\PYG{n+nt}{br}\PYG{p}{/}\PYG{p}{\PYGZgt{}}
        Describa su función:\PYG{p}{\PYGZlt{}}\PYG{n+nt}{br}\PYG{p}{/}\PYG{p}{\PYGZgt{}}
        \PYG{p}{\PYGZlt{}}\PYG{n+nt}{textarea}\PYG{p}{\PYGZgt{}}Escriba aquí\PYG{p}{\PYGZlt{}}\PYG{p}{/}\PYG{n+nt}{textarea}\PYG{p}{\PYGZgt{}}
    \PYG{p}{\PYGZlt{}}\PYG{p}{/}\PYG{n+nt}{fieldset}\PYG{p}{\PYGZgt{}}
\PYG{p}{\PYGZlt{}}\PYG{p}{/}\PYG{n+nt}{form}\PYG{p}{\PYGZgt{}}

\PYG{p}{\PYGZlt{}}\PYG{p}{/}\PYG{n+nt}{body}\PYG{p}{\PYGZgt{}}
\PYG{p}{\PYGZlt{}}\PYG{p}{/}\PYG{n+nt}{html}\PYG{p}{\PYGZgt{}}
\end{sphinxVerbatim}


\subsection{Enunciado}
\label{\detokenize{tema2:id15}}
\sphinxstyleemphasis{Crea un formulario como este}

\noindent{\hspace*{\fill}\sphinxincludegraphics[scale=0.5]{{formu3}.png}\hspace*{\fill}}


\subsection{Solución}
\label{\detokenize{tema2:id16}}
\begin{sphinxVerbatim}[commandchars=\\\{\}]
\PYG{p}{\PYGZlt{}}\PYG{n+nt}{form}\PYG{p}{\PYGZgt{}}
        \PYG{p}{\PYGZlt{}}\PYG{n+nt}{fieldset}\PYG{p}{\PYGZgt{}}
                \PYG{p}{\PYGZlt{}}\PYG{n+nt}{legend}\PYG{p}{\PYGZgt{}}
                        Laboral
                \PYG{p}{\PYGZlt{}}\PYG{p}{/}\PYG{n+nt}{legend}\PYG{p}{\PYGZgt{}}
                \PYG{p}{\PYGZlt{}}\PYG{n+nt}{input} \PYG{n+na}{type}\PYG{o}{=}\PYG{l+s}{\PYGZdq{}checkbox\PYGZdq{}}
                           \PYG{n+na}{name}\PYG{o}{=}\PYG{l+s}{\PYGZdq{}contratos[]\PYGZdq{}}
                           \PYG{n+na}{id}\PYG{o}{=}\PYG{l+s}{\PYGZdq{}ajena\PYGZdq{}}\PYG{p}{\PYGZgt{}}
                Por cuenta ajena
                \PYG{p}{\PYGZlt{}}\PYG{n+nt}{input} \PYG{n+na}{type}\PYG{o}{=}\PYG{l+s}{\PYGZdq{}checkbox\PYGZdq{}}
                           \PYG{n+na}{name}\PYG{o}{=}\PYG{l+s}{\PYGZdq{}contratos[]\PYGZdq{}}
                           \PYG{n+na}{id}\PYG{o}{=}\PYG{l+s}{\PYGZdq{}autonomo\PYGZdq{}}\PYG{p}{\PYGZgt{}}
                Autónomo
                \PYG{p}{\PYGZlt{}}\PYG{n+nt}{br}\PYG{p}{/}\PYG{p}{\PYGZgt{}}
                ¿Alguna vez en el extranjero?
                \PYG{p}{\PYGZlt{}}\PYG{n+nt}{br}\PYG{p}{/}\PYG{p}{\PYGZgt{}}
                \PYG{p}{\PYGZlt{}}\PYG{n+nt}{input} \PYG{n+na}{type}\PYG{o}{=}\PYG{l+s}{\PYGZdq{}radio\PYGZdq{}}
                           \PYG{n+na}{name}\PYG{o}{=}\PYG{l+s}{\PYGZdq{}en\PYGZus{}extranjero\PYGZdq{}}
                           \PYG{n+na}{id}\PYG{o}{=}\PYG{l+s}{\PYGZdq{}si\PYGZus{}en\PYGZus{}extranjero\PYGZdq{}}\PYG{p}{\PYGZgt{}}
                Sí

                \PYG{p}{\PYGZlt{}}\PYG{n+nt}{select} \PYG{n+na}{name}\PYG{o}{=}\PYG{l+s}{\PYGZdq{}lugar\PYGZdq{}}\PYG{p}{\PYGZgt{}}
                        \PYG{p}{\PYGZlt{}}\PYG{n+nt}{option} \PYG{n+na}{id}\PYG{o}{=}\PYG{l+s}{\PYGZdq{}en\PYGZus{}ue\PYGZdq{}}\PYG{p}{\PYGZgt{}}
                                Dentro de la UE
                        \PYG{p}{\PYGZlt{}}\PYG{p}{/}\PYG{n+nt}{option}\PYG{p}{\PYGZgt{}}
                        \PYG{p}{\PYGZlt{}}\PYG{n+nt}{option} \PYG{n+na}{id}\PYG{o}{=}\PYG{l+s}{\PYGZdq{}en\PYGZus{}asia\PYGZdq{}}\PYG{p}{\PYGZgt{}}
                                En Asia
                        \PYG{p}{\PYGZlt{}}\PYG{p}{/}\PYG{n+nt}{option}\PYG{p}{\PYGZgt{}}
                        \PYG{p}{\PYGZlt{}}\PYG{n+nt}{option} \PYG{n+na}{id}\PYG{o}{=}\PYG{l+s}{\PYGZdq{}en\PYGZus{}hispanoamerica\PYGZdq{}}\PYG{p}{\PYGZgt{}}
                                En Hispanoamérica
                        \PYG{p}{\PYGZlt{}}\PYG{p}{/}\PYG{n+nt}{option}\PYG{p}{\PYGZgt{}}
                        \PYG{p}{\PYGZlt{}}\PYG{n+nt}{option} \PYG{n+na}{id}\PYG{o}{=}\PYG{l+s}{\PYGZdq{}en\PYGZus{}eeuu\PYGZdq{}}\PYG{p}{\PYGZgt{}}
                                En EE.UU
                        \PYG{p}{\PYGZlt{}}\PYG{p}{/}\PYG{n+nt}{option}\PYG{p}{\PYGZgt{}}
                        \PYG{p}{\PYGZlt{}}\PYG{n+nt}{option} \PYG{n+na}{id}\PYG{o}{=}\PYG{l+s}{\PYGZdq{}en\PYGZus{}otro\PYGZdq{}}\PYG{p}{\PYGZgt{}}
                                En otro
                        \PYG{p}{\PYGZlt{}}\PYG{p}{/}\PYG{n+nt}{option}\PYG{p}{\PYGZgt{}}
                \PYG{p}{\PYGZlt{}}\PYG{p}{/}\PYG{n+nt}{select}\PYG{p}{\PYGZgt{}}
                \PYG{p}{\PYGZlt{}}\PYG{n+nt}{br}\PYG{p}{/}\PYG{p}{\PYGZgt{}}
                \PYG{p}{\PYGZlt{}}\PYG{n+nt}{input} \PYG{n+na}{type}\PYG{o}{=}\PYG{l+s}{\PYGZdq{}radio\PYGZdq{}}
                           \PYG{n+na}{name}\PYG{o}{=}\PYG{l+s}{\PYGZdq{}en\PYGZus{}extranjero\PYGZdq{}}
                           \PYG{n+na}{id}\PYG{o}{=}\PYG{l+s}{\PYGZdq{}no\PYGZus{}en\PYGZus{}extranjero\PYGZdq{}}\PYG{p}{\PYGZgt{}}
                No
        \PYG{p}{\PYGZlt{}}\PYG{p}{/}\PYG{n+nt}{fieldset}\PYG{p}{\PYGZgt{}}
        \PYG{p}{\PYGZlt{}}\PYG{n+nt}{fieldset}\PYG{p}{\PYGZgt{}}
                \PYG{p}{\PYGZlt{}}\PYG{n+nt}{legend}\PYG{p}{\PYGZgt{}}
                        Personal
                \PYG{p}{\PYGZlt{}}\PYG{p}{/}\PYG{n+nt}{legend}\PYG{p}{\PYGZgt{}}
                Apellidos y nombre:
                \PYG{p}{\PYGZlt{}}\PYG{n+nt}{input} \PYG{n+na}{type}\PYG{o}{=}\PYG{l+s}{\PYGZdq{}text\PYGZdq{}}
                           \PYG{n+na}{id}\PYG{o}{=}\PYG{l+s}{\PYGZdq{}ap\PYGZus{}nombre\PYGZdq{}}\PYG{p}{\PYGZgt{}}
                \PYG{p}{\PYGZlt{}}\PYG{n+nt}{br}\PYG{p}{/}\PYG{p}{\PYGZgt{}}
                Conocimientos sobre:\PYG{p}{\PYGZlt{}}\PYG{n+nt}{br}\PYG{p}{/}\PYG{p}{\PYGZgt{}}
                \PYG{p}{\PYGZlt{}}\PYG{n+nt}{select} \PYG{n+na}{name}\PYG{o}{=}\PYG{l+s}{\PYGZdq{}cono\PYGZdq{}} \PYG{n+na}{multiple}\PYG{p}{\PYGZgt{}}
                        \PYG{p}{\PYGZlt{}}\PYG{n+nt}{option} \PYG{n+na}{id}\PYG{o}{=}\PYG{l+s}{\PYGZdq{}informatica\PYGZdq{}}\PYG{p}{\PYGZgt{}}
                                Informática
                        \PYG{p}{\PYGZlt{}}\PYG{p}{/}\PYG{n+nt}{option}\PYG{p}{\PYGZgt{}}
                        \PYG{p}{\PYGZlt{}}\PYG{n+nt}{option} \PYG{n+na}{id}\PYG{o}{=}\PYG{l+s}{\PYGZdq{}conduccion\PYGZdq{}}\PYG{p}{\PYGZgt{}}
                                Conducción
                        \PYG{p}{\PYGZlt{}}\PYG{p}{/}\PYG{n+nt}{option}\PYG{p}{\PYGZgt{}}
                        \PYG{p}{\PYGZlt{}}\PYG{n+nt}{option} \PYG{n+na}{id}\PYG{o}{=}\PYG{l+s}{\PYGZdq{}finanzas\PYGZdq{}}\PYG{p}{\PYGZgt{}}
                                Finanzas
                        \PYG{p}{\PYGZlt{}}\PYG{p}{/}\PYG{n+nt}{option}\PYG{p}{\PYGZgt{}}
                        \PYG{p}{\PYGZlt{}}\PYG{n+nt}{option} \PYG{n+na}{id}\PYG{o}{=}\PYG{l+s}{\PYGZdq{}leyes\PYGZdq{}}\PYG{p}{\PYGZgt{}}
                                Leyes
                        \PYG{p}{\PYGZlt{}}\PYG{p}{/}\PYG{n+nt}{option}\PYG{p}{\PYGZgt{}}
                \PYG{p}{\PYGZlt{}}\PYG{p}{/}\PYG{n+nt}{select}\PYG{p}{\PYGZgt{}}
        \PYG{p}{\PYGZlt{}}\PYG{p}{/}\PYG{n+nt}{fieldset}\PYG{p}{\PYGZgt{}}
\PYG{p}{\PYGZlt{}}\PYG{p}{/}\PYG{n+nt}{form}\PYG{p}{\PYGZgt{}}
\end{sphinxVerbatim}


\section{Examen}
\label{\detokenize{tema2:examen}}\begin{itemize}
\item {} 
El grupo DAM-1 realizará el Jueves 10 de noviembre de 2016

\item {} 
El grupo ASIR-1 realizará el examen el Jueves 17 de noviembre de 2016.

\end{itemize}


\chapter{CSS}
\label{\detokenize{tema3::doc}}\label{\detokenize{tema3:css}}

\section{Introducción}
\label{\detokenize{tema3:introduccion}}
El lenguaje CSS permite cambiar el aspecto de páginas web utilizando enlaces a archivos de hojas de estilo. Si todos los HTML de un portal web cargan el mismo archivo CSS se puede cambiar todo un conjunto de HTML’s modificando un solo CSS.


\section{Sintaxis}
\label{\detokenize{tema3:sintaxis}}
CSS funciona mediante reglas, de las cuales se muestra un ejemplo a continuación.

\begin{sphinxVerbatim}[commandchars=\\\{\}]
\PYG{n+nt}{h1}\PYG{o}{,} \PYG{n+nt}{h3} \PYG{p}{\PYGZob{}}
    \PYG{k}{background\PYGZhy{}color}\PYG{p}{:} \PYG{k+kc}{blue}\PYG{p}{;}
    \PYG{k}{color}\PYG{p}{:} \PYG{k+kc}{white}\PYG{p}{;}
\PYG{p}{\PYGZcb{}}
\end{sphinxVerbatim}

En las reglas tenemos tres cosas:
\begin{enumerate}
\item {} 
Los selectores. En este caso queremos modificar como van a quedar todos los \sphinxcode{\textless{}h1\textgreater{}} y \sphinxcode{\textless{}h3\textgreater{}}

\item {} 
Las propiedades. En el ejemplo se pretende cambiar el color de fondo y el color de las letras.

\item {} 
Los valores. En este caso se pone el valor \sphinxcode{blue} para la propiedad \sphinxcode{background-color} y el valor \sphinxcode{white} para la propiedad \sphinxcode{color}

\end{enumerate}


\section{Los atributos \sphinxstyleliteralintitle{class} e \sphinxstyleliteralintitle{id}}
\label{\detokenize{tema3:los-atributos-class-e-id}}
A menudo tendremos que hacer en cambios en un grupo de elementos, pero a veces no serán todos los elementos de una misma clase. Por ejemplo, puede que queramos cambiar un elemento en concreto. Para poder hacer cambios \sphinxstyleemphasis{a un solo elemento} tendremos que haber puesto el atributo \sphinxcode{id} como muestra el ejemplo siguiente:

\begin{sphinxVerbatim}[commandchars=\\\{\}]
\PYG{p}{\PYGZlt{}}\PYG{n+nt}{h1}\PYG{p}{\PYGZgt{}}Encabezamiento\PYG{p}{\PYGZlt{}}\PYG{p}{/}\PYG{n+nt}{h1}\PYG{p}{\PYGZgt{}}
.... texto ...

\PYG{p}{\PYGZlt{}}\PYG{n+nt}{h1} \PYG{n+na}{id}\PYG{o}{=}\PYG{l+s}{\PYGZdq{}noticia\PYGZus{}del\PYGZus{}dia\PYGZdq{}}\PYG{p}{\PYGZgt{}}Otro encabezamiento\PYG{p}{\PYGZlt{}}\PYG{p}{/}\PYG{n+nt}{h1}\PYG{p}{\PYGZgt{}}
.... texto ...

\PYG{p}{\PYGZlt{}}\PYG{n+nt}{h1}\PYG{p}{\PYGZgt{}}Y otro más\PYG{p}{\PYGZlt{}}\PYG{p}{/}\PYG{n+nt}{h1}\PYG{p}{\PYGZgt{}}
.... texto ...
\end{sphinxVerbatim}

Si luego desde CSS queremos modificar solo el encabezamiento con la noticia del dia deberemos hacer esto

\begin{sphinxVerbatim}[commandchars=\\\{\}]
\PYG{n+nt}{h1}\PYG{p}{\PYGZsh{}}\PYG{n+nn}{noticia\PYGZus{}del\PYGZus{}dia}\PYG{p}{\PYGZob{}}
    \PYG{k}{font\PYGZhy{}weight}\PYG{p}{:} \PYG{k+kc}{bold}\PYG{p}{;}
\PYG{p}{\PYGZcb{}}
\end{sphinxVerbatim}

Obsérvese que hemos usado la almohadillas (\#) para decir «queremos seleccionar el h1 cuyo id es noticia\_del\_dia» y ponerlo en negrita. Debe señalarse que es importante que en el HTML \sphinxstylestrong{NO DEBE HABER DOS ELEMENTOS CON EL MISMO ID}

Si en vez de uno queremos aplicar un cambio a \sphinxstylestrong{un conjunto de elementos} deberemos ir al HTML y ponerles a todos
ellos un atributo \sphinxcode{class} con el mismo valor en todos ellos. Por ejemplo:

\begin{sphinxVerbatim}[commandchars=\\\{\}]
\PYG{p}{\PYGZlt{}}\PYG{n+nt}{h1} \PYG{n+na}{class}\PYG{o}{=}\PYG{l+s}{\PYGZdq{}titular\PYGZus{}economia\PYGZdq{}}\PYG{p}{\PYGZgt{}}Encabezamiento\PYG{p}{\PYGZlt{}}\PYG{p}{/}\PYG{n+nt}{h1}\PYG{p}{\PYGZgt{}}
.... texto ...

\PYG{p}{\PYGZlt{}}\PYG{n+nt}{h1}\PYG{p}{\PYGZgt{}}Otro encabezamiento\PYG{p}{\PYGZlt{}}\PYG{p}{/}\PYG{n+nt}{h1}\PYG{p}{\PYGZgt{}}
.... texto ...

\PYG{p}{\PYGZlt{}}\PYG{n+nt}{h1} \PYG{n+na}{class}\PYG{o}{=}\PYG{l+s}{\PYGZdq{}titular\PYGZus{}economia\PYGZdq{}}\PYG{p}{\PYGZgt{}}Y otro más\PYG{p}{\PYGZlt{}}\PYG{p}{/}\PYG{n+nt}{h1}\PYG{p}{\PYGZgt{}}
.... texto ...
\end{sphinxVerbatim}

Con este HTML podemos crear un CSS como este que cambia solo los \sphinxcode{h1} cuyo \sphinxcode{class} tiene el valor \sphinxcode{titular\_economia}

\begin{sphinxVerbatim}[commandchars=\\\{\}]
\PYG{n+nt}{h1}\PYG{p}{.}\PYG{n+nc}{titular\PYGZus{}economia}\PYG{p}{\PYGZob{}}
    \PYG{k}{font\PYGZhy{}weight}\PYG{p}{:} \PYG{k+kc}{bold}\PYG{p}{;}
\PYG{p}{\PYGZcb{}}
\end{sphinxVerbatim}

Se puede poner el mismo \sphinxcode{class} a distintos elementos, por ejemplo así

\begin{sphinxVerbatim}[commandchars=\\\{\}]
\PYG{p}{\PYGZlt{}}\PYG{n+nt}{h1} \PYG{n+na}{class}\PYG{o}{=}\PYG{l+s}{\PYGZdq{}titular\PYGZus{}economia\PYGZdq{}}\PYG{p}{\PYGZgt{}}Encabezamiento\PYG{p}{\PYGZlt{}}\PYG{p}{/}\PYG{n+nt}{h1}\PYG{p}{\PYGZgt{}}
.... texto ...

\PYG{p}{\PYGZlt{}}\PYG{n+nt}{h1}\PYG{p}{\PYGZgt{}}Otro encabezamiento\PYG{p}{\PYGZlt{}}\PYG{p}{/}\PYG{n+nt}{h1}\PYG{p}{\PYGZgt{}}
.... texto ...

\PYG{p}{\PYGZlt{}}\PYG{n+nt}{h2} \PYG{n+na}{class}\PYG{o}{=}\PYG{l+s}{\PYGZdq{}titular\PYGZus{}economia\PYGZdq{}}\PYG{p}{\PYGZgt{}}Y otro más\PYG{p}{\PYGZlt{}}\PYG{p}{/}\PYG{n+nt}{h2}\PYG{p}{\PYGZgt{}}
.... texto ...
\end{sphinxVerbatim}

Y luego usar un CSS como este:

\begin{sphinxVerbatim}[commandchars=\\\{\}]
\PYG{n+nt}{h1}\PYG{p}{.}\PYG{n+nc}{titular\PYGZus{}economia}\PYG{o}{,} \PYG{n+nt}{h2}\PYG{p}{.}\PYG{n+nc}{titular\PYGZus{}economia}\PYG{p}{\PYGZob{}}
    \PYG{k}{font\PYGZhy{}weight}\PYG{p}{:} \PYG{k+kc}{bold}\PYG{p}{;}
\PYG{p}{\PYGZcb{}}
\end{sphinxVerbatim}

Este último CSS \sphinxstylestrong{se puede resumir}

\begin{sphinxVerbatim}[commandchars=\\\{\}]
\PYG{p}{.}\PYG{n+nc}{titular\PYGZus{}economia}\PYG{p}{\PYGZob{}}
    \PYG{k}{font\PYGZhy{}weight}\PYG{p}{:} \PYG{k+kc}{bold}\PYG{p}{;}
\PYG{p}{\PYGZcb{}}
\end{sphinxVerbatim}

Esta regla dice «seleccionar todos los elementos cuyo class sea titular\_economia»  y ponerlos en negrita. Estos mecanismos de resumen son muy útiles y facilitan mucho la tarea del diseñador CSS.


\section{Posicionamiento}
\label{\detokenize{tema3:posicionamiento}}
Para posicionar los elementos se suelen utilizar dos etiquetas que no hacen nada especial, salvo actuar de contenedores. Las etiquetas \textless{}span\textgreater{} y \textless{}div\textgreater{}.
\begin{itemize}
\item {} 
\textless{}span\textgreater{} se usa para no romper el flujo, es decir en principio todo va en la misma línea

\item {} 
\textless{}div\textgreater{} sí rompe el flujo, por lo que va a una línea distinta

\end{itemize}

En cualquier etiqueta puede ocurrir que deseemos que el estilo no se aplique a todos los elementos o que queramos que se aplique a unos cuantos (pero no a todos). En ese caso, recordemos que se deben utilizar los atributos \sphinxcode{class} e \sphinxcode{id}
\begin{itemize}
\item {} 
El \sphinxcode{class} es un atributo que puede llevar el mismo valor en muchos elementos HTML y que nos permitirá despues seleccionarlos a todos.

\item {} 
El \sphinxcode{id} es un atributo que debe tener distinto valor en todos los casos, no se puede repetir.

\end{itemize}

Para posicionar correctamente un span o un div, se deben tener en cuenta
varias cosas:
\begin{itemize}
\item {} 
Todos deberían llevar un id o un class (o las dos cosas)

\item {} \begin{description}
\item[{El posicionamiento tiene varias posibilidades:}] \leavevmode\begin{itemize}
\item {} 
\sphinxcode{fixed}: la caja va en cierta posición y no se mueve de allí

\item {} 
\sphinxcode{absolute}: la caja va en cierta posición inicial y puede desaparecer al hacer scroll.

\item {} 
\sphinxcode{relative}: podemos indicar una posición para indicar el desplazamiento relativo con respecto a la posición que le correspondería según el navegador

\item {} 
\sphinxcode{static}: dar permiso al navegador para que coloque la caja donde corresponda

\item {} 
\sphinxcode{float}: mover la caja a cierta posición permitiendo que otras cajas floten a su alrededor

\end{itemize}

\end{description}

\end{itemize}


\subsection{Ejercicio propuesto}
\label{\detokenize{tema3:ejercicio-propuesto}}
Crea una página con la siguiente estructura.
\begin{itemize}
\item {} 
En la parte superior debe haber dos cajas. Una de ellas, a la izquierda, ocupa el 33\% y contiene el lema. La otra, a la derecha, contiene enlaces y ocupa el 66\%.

\item {} 
En la parte central 3 cajas. La de la izquierda contiene publicidad y ocupa el 25\%. La central tiene el contenido y ocupa el 50\%, la de la derecha tiene más publicidad y ocupa el 25\%.

\item {} 
En la parte de abajo hay una barra \sphinxstylestrong{que no se mueve nunca} y que ocupa el 100\%. Contiene el mensaje de copyright de la empresa.

\end{itemize}

Un posible HTML sería el siguiente:

\begin{sphinxVerbatim}[commandchars=\\\{\}]
\PYG{c+cp}{\PYGZlt{}!DOCTYPE html\PYGZgt{}}
\PYG{p}{\PYGZlt{}}\PYG{n+nt}{html}\PYG{p}{\PYGZgt{}}
\PYG{p}{\PYGZlt{}}\PYG{n+nt}{head}\PYG{p}{\PYGZgt{}}
    \PYG{p}{\PYGZlt{}}\PYG{n+nt}{link} \PYG{n+na}{rel}\PYG{o}{=}\PYG{l+s}{\PYGZdq{}stylesheet\PYGZdq{}}
          \PYG{n+na}{type}\PYG{o}{=}\PYG{l+s}{\PYGZdq{}text/css\PYGZdq{}}
          \PYG{n+na}{href}\PYG{o}{=}\PYG{l+s}{\PYGZdq{}estilo.css\PYGZdq{}}\PYG{p}{\PYGZgt{}}
    \PYG{p}{\PYGZlt{}}\PYG{n+nt}{meta} \PYG{n+na}{charset}\PYG{o}{=}\PYG{l+s}{\PYGZdq{}utf\PYGZhy{}8\PYGZdq{}}\PYG{p}{\PYGZgt{}}
    \PYG{p}{\PYGZlt{}}\PYG{n+nt}{title}\PYG{p}{\PYGZgt{}}Page Title\PYG{p}{\PYGZlt{}}\PYG{p}{/}\PYG{n+nt}{title}\PYG{p}{\PYGZgt{}}
\PYG{p}{\PYGZlt{}}\PYG{p}{/}\PYG{n+nt}{head}\PYG{p}{\PYGZgt{}}
\PYG{p}{\PYGZlt{}}\PYG{n+nt}{body}\PYG{p}{\PYGZgt{}}
\PYG{p}{\PYGZlt{}}\PYG{n+nt}{div} \PYG{n+na}{id}\PYG{o}{=}\PYG{l+s}{\PYGZdq{}lema\PYGZdq{}}\PYG{p}{\PYGZgt{}}
    Lema...
\PYG{p}{\PYGZlt{}}\PYG{p}{/}\PYG{n+nt}{div}\PYG{p}{\PYGZgt{}}
\PYG{p}{\PYGZlt{}}\PYG{n+nt}{div} \PYG{n+na}{id}\PYG{o}{=}\PYG{l+s}{\PYGZdq{}enlaces\PYGZdq{}}\PYG{p}{\PYGZgt{}}
    Enlaces....
\PYG{p}{\PYGZlt{}}\PYG{p}{/}\PYG{n+nt}{div}\PYG{p}{\PYGZgt{}}
\PYG{p}{\PYGZlt{}}\PYG{n+nt}{div} \PYG{n+na}{id}\PYG{o}{=}\PYG{l+s}{\PYGZdq{}publi1\PYGZdq{}}\PYG{p}{\PYGZgt{}}
    Publicidad
\PYG{p}{\PYGZlt{}}\PYG{p}{/}\PYG{n+nt}{div}\PYG{p}{\PYGZgt{}}
\PYG{p}{\PYGZlt{}}\PYG{n+nt}{div} \PYG{n+na}{id}\PYG{o}{=}\PYG{l+s}{\PYGZdq{}contenido\PYGZdq{}}\PYG{p}{\PYGZgt{}}
    Contenido...
\PYG{p}{\PYGZlt{}}\PYG{p}{/}\PYG{n+nt}{div}\PYG{p}{\PYGZgt{}}
\PYG{p}{\PYGZlt{}}\PYG{n+nt}{div} \PYG{n+na}{id}\PYG{o}{=}\PYG{l+s}{\PYGZdq{}publi2\PYGZdq{}}\PYG{p}{\PYGZgt{}}
    Publicidad...
\PYG{p}{\PYGZlt{}}\PYG{p}{/}\PYG{n+nt}{div}\PYG{p}{\PYGZgt{}}
\PYG{p}{\PYGZlt{}}\PYG{n+nt}{div} \PYG{n+na}{id}\PYG{o}{=}\PYG{l+s}{\PYGZdq{}copyright\PYGZdq{}}\PYG{p}{\PYGZgt{}}
    \PYG{n+ni}{\PYGZam{}copy;} IES Maestre...
\PYG{p}{\PYGZlt{}}\PYG{p}{/}\PYG{n+nt}{div}\PYG{p}{\PYGZgt{}}
\PYG{p}{\PYGZlt{}}\PYG{p}{/}\PYG{n+nt}{body}\PYG{p}{\PYGZgt{}}
\PYG{p}{\PYGZlt{}}\PYG{p}{/}\PYG{n+nt}{html}\PYG{p}{\PYGZgt{}}
\end{sphinxVerbatim}

Y un posible CSS sería este:

\begin{sphinxVerbatim}[commandchars=\\\{\}]
\PYG{n+nt}{h1}\PYG{p}{\PYGZsh{}}\PYG{n+nn}{bienvenida}\PYG{p}{\PYGZob{}}
  \PYG{k}{color}\PYG{p}{:} \PYG{n+nb}{rgb}\PYG{p}{(}\PYG{l+m+mi}{66}\PYG{p}{,} \PYG{l+m+mi}{66}\PYG{p}{,} \PYG{l+m+mi}{220}\PYG{p}{)}\PYG{p}{;}
  \PYG{k}{font\PYGZhy{}family}\PYG{p}{:} \PYG{l+s+s2}{\PYGZdq{}Comic Sans MS\PYGZdq{}}\PYG{p}{;}
\PYG{p}{\PYGZcb{}}

\PYG{n+nt}{p}\PYG{p}{\PYGZob{}}
  \PYG{k}{background\PYGZhy{}color}\PYG{p}{:} \PYG{n+nb}{rgb}\PYG{p}{(}\PYG{l+m+mi}{240}\PYG{p}{,} \PYG{l+m+mi}{250}\PYG{p}{,} \PYG{l+m+mi}{230}\PYG{p}{)}\PYG{p}{;}
  \PYG{k}{margin\PYGZhy{}left}\PYG{p}{:}\PYG{l+m+mi}{15}\PYG{k+kt}{\PYGZpc{}}\PYG{p}{;}
  \PYG{k}{margin\PYGZhy{}right}\PYG{p}{:} \PYG{l+m+mi}{15}\PYG{k+kt}{\PYGZpc{}}\PYG{p}{;}
  \PYG{k}{padding}\PYG{p}{:}\PYG{l+m+mi}{5}\PYG{k+kt}{em}\PYG{p}{;}
  \PYG{k}{font\PYGZhy{}weight}\PYG{p}{:} \PYG{k+kc}{bold}\PYG{p}{;}
  \PYG{k}{font\PYGZhy{}style}\PYG{p}{:} \PYG{k+kc}{italic}\PYG{p}{;}
  \PYG{k}{text\PYGZhy{}align}\PYG{p}{:} \PYG{k+kc}{justify}\PYG{p}{;}
\PYG{p}{\PYGZcb{}}

\PYG{n+nt}{tr}\PYG{p}{.}\PYG{n+nc}{par}\PYG{p}{\PYGZob{}}
  \PYG{k}{background\PYGZhy{}color}\PYG{p}{:} \PYG{n+nb}{rgb}\PYG{p}{(}\PYG{l+m+mi}{180}\PYG{p}{,} \PYG{l+m+mi}{180}\PYG{p}{,} \PYG{l+m+mi}{180}\PYG{p}{)}\PYG{p}{;}
\PYG{p}{\PYGZcb{}}

\PYG{n+nt}{tr}\PYG{p}{.}\PYG{n+nc}{impar}\PYG{p}{\PYGZob{}}
  \PYG{k}{background\PYGZhy{}color}\PYG{p}{:} \PYG{n+nb}{rgb}\PYG{p}{(}\PYG{l+m+mi}{220}\PYG{p}{,}\PYG{l+m+mi}{220}\PYG{p}{,} \PYG{l+m+mi}{220}\PYG{p}{)}\PYG{p}{;}
\PYG{p}{\PYGZcb{}}
\end{sphinxVerbatim}


\subsection{Ejercicio propuesto (II)}
\label{\detokenize{tema3:ejercicio-propuesto-ii}}
Crear una página con la siguiente estructura:
\begin{itemize}
\item {} 
En la parte izquierda hay una barra de enlaces. Ocupa el 25\% y está \sphinxstylestrong{fija}.

\item {} 
En la zona superior hay una capa con el lema de la empresa. Ocupa el 75\% y no se debe ver tapada por los enlaces.

\item {} 
En la zona central hay dos capas. Una de ellas es el contenido y ocupa aproximadamente el 50\%. A su lado hay una capa con publicidad que ocupa el 20\%.

\item {} 
En la zona inferior hay una capa con el copyright de la empresa. Ocupa el 50\% y se ve junto al margen derecho de la página.

\end{itemize}

Un posible HTML sería el siguiente:

\begin{sphinxVerbatim}[commandchars=\\\{\}]
\PYG{c+cp}{\PYGZlt{}!DOCTYPE html\PYGZgt{}}
\PYG{p}{\PYGZlt{}}\PYG{n+nt}{html}\PYG{p}{\PYGZgt{}}
\PYG{p}{\PYGZlt{}}\PYG{n+nt}{head}\PYG{p}{\PYGZgt{}}
  \PYG{p}{\PYGZlt{}}\PYG{n+nt}{link} \PYG{n+na}{rel}\PYG{o}{=}\PYG{l+s}{\PYGZdq{}stylesheet\PYGZdq{}}
  \PYG{n+na}{type}\PYG{o}{=}\PYG{l+s}{\PYGZdq{}text/css\PYGZdq{}} \PYG{n+na}{href}\PYG{o}{=}\PYG{l+s}{\PYGZdq{}estilo.css\PYGZdq{}}\PYG{p}{\PYGZgt{}}
    \PYG{p}{\PYGZlt{}}\PYG{n+nt}{meta} \PYG{n+na}{charset}\PYG{o}{=}\PYG{l+s}{\PYGZdq{}utf\PYGZhy{}8\PYGZdq{}}\PYG{p}{\PYGZgt{}}
    \PYG{p}{\PYGZlt{}}\PYG{n+nt}{title}\PYG{p}{\PYGZgt{}}Ejercicio\PYG{p}{\PYGZlt{}}\PYG{p}{/}\PYG{n+nt}{title}\PYG{p}{\PYGZgt{}}
\PYG{p}{\PYGZlt{}}\PYG{p}{/}\PYG{n+nt}{head}\PYG{p}{\PYGZgt{}}

\PYG{p}{\PYGZlt{}}\PYG{n+nt}{body}\PYG{p}{\PYGZgt{}}
\PYG{p}{\PYGZlt{}}\PYG{n+nt}{div} \PYG{n+na}{id}\PYG{o}{=}\PYG{l+s}{\PYGZdq{}enlaces\PYGZdq{}}\PYG{p}{\PYGZgt{}}
Enlaces enlaces 
\PYG{p}{\PYGZlt{}}\PYG{p}{/}\PYG{n+nt}{div}\PYG{p}{\PYGZgt{}}

\PYG{p}{\PYGZlt{}}\PYG{n+nt}{div} \PYG{n+na}{id}\PYG{o}{=}\PYG{l+s}{\PYGZdq{}lema\PYGZdq{}}\PYG{p}{\PYGZgt{}}
Progresando con el tiempo,
mejorando con la experiencia
\PYG{p}{\PYGZlt{}}\PYG{p}{/}\PYG{n+nt}{div}\PYG{p}{\PYGZgt{}}

\PYG{p}{\PYGZlt{}}\PYG{n+nt}{div} \PYG{n+na}{id}\PYG{o}{=}\PYG{l+s}{\PYGZdq{}contenedor\PYGZus{}central\PYGZdq{}}\PYG{p}{\PYGZgt{}}
\PYG{p}{\PYGZlt{}}\PYG{n+nt}{div} \PYG{n+na}{id}\PYG{o}{=}\PYG{l+s}{\PYGZdq{}contenido\PYGZdq{}}\PYG{p}{\PYGZgt{}}
Texto principal de la página
\PYG{p}{\PYGZlt{}}\PYG{p}{/}\PYG{n+nt}{div}\PYG{p}{\PYGZgt{}}
\PYG{p}{\PYGZlt{}}\PYG{n+nt}{div} \PYG{n+na}{id}\PYG{o}{=}\PYG{l+s}{\PYGZdq{}publi\PYGZdq{}}\PYG{p}{\PYGZgt{}}
Publicidad publicidad
\PYG{p}{\PYGZlt{}}\PYG{p}{/}\PYG{n+nt}{div}\PYG{p}{\PYGZgt{}}
\PYG{p}{\PYGZlt{}}\PYG{p}{/}\PYG{n+nt}{div}\PYG{p}{\PYGZgt{}}


\PYG{p}{\PYGZlt{}}\PYG{n+nt}{div} \PYG{n+na}{id}\PYG{o}{=}\PYG{l+s}{\PYGZdq{}copyright\PYGZdq{}}\PYG{p}{\PYGZgt{}}
\PYG{n+ni}{\PYGZam{}copy;} IES Maestre 2016
\PYG{p}{\PYGZlt{}}\PYG{p}{/}\PYG{n+nt}{div}\PYG{p}{\PYGZgt{}}
\PYG{p}{\PYGZlt{}}\PYG{p}{/}\PYG{n+nt}{body}\PYG{p}{\PYGZgt{}}
\PYG{p}{\PYGZlt{}}\PYG{p}{/}\PYG{n+nt}{html}\PYG{p}{\PYGZgt{}}
\end{sphinxVerbatim}

Y un posible CSS sería este:

\begin{sphinxVerbatim}[commandchars=\\\{\}]
\PYG{n+nt}{div}\PYG{p}{\PYGZsh{}}\PYG{n+nn}{enlaces}\PYG{p}{\PYGZob{}}
  \PYG{k}{position}\PYG{p}{:} \PYG{k+kc}{fixed}\PYG{p}{;}
  \PYG{k}{top}\PYG{p}{:}\PYG{l+m+mi}{0}\PYG{k+kt}{px}\PYG{p}{;}
  \PYG{k}{left}\PYG{p}{:}\PYG{l+m+mi}{0}\PYG{k+kt}{px}\PYG{p}{;}
  \PYG{k}{width}\PYG{p}{:}\PYG{l+m+mi}{20}\PYG{k+kt}{\PYGZpc{}}\PYG{p}{;}
\PYG{p}{\PYGZcb{}}
\PYG{n+nt}{div}\PYG{p}{\PYGZsh{}}\PYG{n+nn}{lema}\PYG{p}{\PYGZob{}}
 \PYG{k}{float}\PYG{p}{:}\PYG{k+kc}{right}\PYG{p}{;}
  \PYG{k}{width}\PYG{p}{:}\PYG{l+m+mi}{70}\PYG{k+kt}{\PYGZpc{}}\PYG{p}{;}
\PYG{p}{\PYGZcb{}}
\PYG{n+nt}{div}\PYG{p}{\PYGZsh{}}\PYG{n+nn}{contenedor\PYGZus{}central}\PYG{p}{\PYGZob{}}
  \PYG{k}{float}\PYG{p}{:}\PYG{k+kc}{right}\PYG{p}{;}
  \PYG{k}{width}\PYG{p}{:}\PYG{l+m+mi}{75}\PYG{k+kt}{\PYGZpc{}}\PYG{p}{;}
  \PYG{k}{margin\PYGZhy{}right}\PYG{p}{:} \PYG{l+m+mi}{0}\PYG{k+kt}{px}\PYG{p}{;}
\PYG{p}{\PYGZcb{}}
\PYG{n+nt}{div}\PYG{p}{\PYGZsh{}}\PYG{n+nn}{contenido}\PYG{p}{\PYGZob{}}
  \PYG{k}{font\PYGZhy{}family}\PYG{p}{:} \PYG{n}{Helvetica}\PYG{p}{;} 
  \PYG{k}{float}\PYG{p}{:}\PYG{k+kc}{left}\PYG{p}{;}
  \PYG{k}{width}\PYG{p}{:} \PYG{l+m+mi}{62}\PYG{k+kt}{\PYGZpc{}}\PYG{p}{;}
\PYG{p}{\PYGZcb{}}
\PYG{n+nt}{div}\PYG{p}{\PYGZsh{}}\PYG{n+nn}{publi}\PYG{p}{\PYGZob{}}
  \PYG{k}{margin\PYGZhy{}right}\PYG{p}{:} \PYG{l+m+mi}{0}\PYG{k+kt}{px}\PYG{p}{;}
  \PYG{k}{float}\PYG{p}{:}\PYG{k+kc}{right}\PYG{p}{;}
  \PYG{k}{width}\PYG{p}{:}\PYG{l+m+mi}{32}\PYG{k+kt}{\PYGZpc{}}\PYG{p}{;}
\PYG{p}{\PYGZcb{}}
\PYG{n+nt}{div}\PYG{p}{\PYGZsh{}}\PYG{n+nn}{copyright}\PYG{p}{\PYGZob{}}
  \PYG{k}{float}\PYG{p}{:}\PYG{k+kc}{right}\PYG{p}{;}
  \PYG{k}{width}\PYG{p}{:} \PYG{l+m+mi}{50}\PYG{k+kt}{\PYGZpc{}}\PYG{p}{;}
\PYG{p}{\PYGZcb{}}
\end{sphinxVerbatim}


\subsection{Ejercicio de maquetación}
\label{\detokenize{tema3:ejercicio-de-maquetacion}}
Crear una página con la siguiente estructura:
\begin{itemize}
\item {} 
En el margen izquierdo debe aparecer una barra de enlaces que ocupe el 20 o 25\% de la anchura de la página y no debe desaparecer aunque el usuario se mueva.

\item {} 
En el margen derecho debe aparecer una caja con el texto y resto de información de interés que debe ocupar el 80 o 75\% de la página y el texto se mueve cuando el usuario se mueve.

\item {} 
Aplicar bordes y efectos visuales a ambas cajas para intentar que el efecto final sea estéticamente aceptable.

\end{itemize}

\begin{figure}[htbp]
\centering
\capstart

\noindent\sphinxincludegraphics{{maquetacion1}.png}
\caption{Resultado final}\label{\detokenize{tema3:id6}}\end{figure}


\subsubsection{HTML}
\label{\detokenize{tema3:html}}
\begin{sphinxVerbatim}[commandchars=\\\{\}]
    \PYG{p}{\PYGZlt{}}\PYG{n+nt}{div} \PYG{n+na}{id}\PYG{o}{=}\PYG{l+s}{\PYGZdq{}enlaces\PYGZdq{}}\PYG{p}{\PYGZgt{}}
    \PYG{p}{\PYGZlt{}}\PYG{n+nt}{ul}\PYG{p}{\PYGZgt{}}
        \PYG{p}{\PYGZlt{}}\PYG{n+nt}{li}\PYG{p}{\PYGZgt{}}
            \PYG{p}{\PYGZlt{}}\PYG{n+nt}{a} \PYG{n+na}{href}\PYG{o}{=}\PYG{l+s}{\PYGZdq{}http://www.google.es\PYGZdq{}}\PYG{p}{\PYGZgt{}}
                Google
            \PYG{p}{\PYGZlt{}}\PYG{p}{/}\PYG{n+nt}{a}\PYG{p}{\PYGZgt{}}
        \PYG{p}{\PYGZlt{}}\PYG{p}{/}\PYG{n+nt}{li}\PYG{p}{\PYGZgt{}}
        \PYG{p}{\PYGZlt{}}\PYG{n+nt}{li}\PYG{p}{\PYGZgt{}}
            \PYG{p}{\PYGZlt{}}\PYG{n+nt}{a} \PYG{n+na}{href}\PYG{o}{=}\PYG{l+s}{\PYGZdq{}http://www.terra.es\PYGZdq{}}\PYG{p}{\PYGZgt{}}
                Terra
            \PYG{p}{\PYGZlt{}}\PYG{p}{/}\PYG{n+nt}{a}\PYG{p}{\PYGZgt{}}
        \PYG{p}{\PYGZlt{}}\PYG{p}{/}\PYG{n+nt}{li}\PYG{p}{\PYGZgt{}}
        \PYG{p}{\PYGZlt{}}\PYG{n+nt}{li}\PYG{p}{\PYGZgt{}}
            \PYG{p}{\PYGZlt{}}\PYG{n+nt}{a} \PYG{n+na}{href}\PYG{o}{=}\PYG{l+s}{\PYGZdq{}http://www.yahoo.es\PYGZdq{}}\PYG{p}{\PYGZgt{}}
                Yahoo
            \PYG{p}{\PYGZlt{}}\PYG{p}{/}\PYG{n+nt}{a}\PYG{p}{\PYGZgt{}}
        \PYG{p}{\PYGZlt{}}\PYG{p}{/}\PYG{n+nt}{li}\PYG{p}{\PYGZgt{}}
    \PYG{p}{\PYGZlt{}}\PYG{p}{/}\PYG{n+nt}{ul}\PYG{p}{\PYGZgt{}}
\PYG{p}{\PYGZlt{}}\PYG{p}{/}\PYG{n+nt}{div}\PYG{p}{\PYGZgt{}}
\PYG{p}{\PYGZlt{}}\PYG{n+nt}{div} \PYG{n+na}{id}\PYG{o}{=}\PYG{l+s}{\PYGZdq{}contenido\PYGZdq{}}\PYG{p}{\PYGZgt{}}

\PYG{p}{\PYGZlt{}}\PYG{n+nt}{p}\PYG{p}{\PYGZgt{}}
            Nunc tempor libero risus, et ultricies ex auctor a. Curabitur efficitur convallis justo consectetur porttitor. Suspendisse potenti. Curabitur ex felis, lacinia non varius ac, ornare eget lacus. Nunc dolor mauris, fermentum nec augue id, imperdiet eleifend tortor. Vestibulum commodo orci ut lorem suscipit, at sodales justo aliquet. Vestibulum sit amet purus eu mauris imperdiet aliquet sed vitae magna. Integer tempus elit purus ...

            \PYG{p}{\PYGZlt{}}\PYG{n+nt}{img} \PYG{n+na}{src}\PYG{o}{=}\PYG{l+s}{\PYGZdq{}gatito1.jpg\PYGZdq{}}\PYG{p}{\PYGZgt{}}
    \PYG{p}{\PYGZlt{}}\PYG{p}{/}\PYG{n+nt}{p}\PYG{p}{\PYGZgt{}}
    \PYG{p}{\PYGZlt{}}\PYG{p}{/}\PYG{n+nt}{div}\PYG{p}{\PYGZgt{}}
\end{sphinxVerbatim}


\subsubsection{CSS}
\label{\detokenize{tema3:id1}}
\begin{sphinxVerbatim}[commandchars=\\\{\}]
\PYG{n+nt}{body}\PYG{p}{\PYGZob{}}
        \PYG{k}{background\PYGZhy{}image}\PYG{p}{:}
                \PYG{n+nb}{url}\PYG{p}{(}\PYG{l+s+s2}{\PYGZdq{}textura.jpg\PYGZdq{}}\PYG{p}{)}\PYG{p}{;}
        \PYG{k}{background\PYGZhy{}attachment}\PYG{p}{:} \PYG{k+kc}{fixed}\PYG{p}{;}
\PYG{p}{\PYGZcb{}}

\PYG{n+nt}{div}\PYG{p}{\PYGZsh{}}\PYG{n+nn}{enlaces}\PYG{p}{\PYGZob{}}
        \PYG{k}{position}\PYG{p}{:} \PYG{k+kc}{fixed}\PYG{p}{;}
        \PYG{k}{top}\PYG{p}{:}\PYG{l+m+mi}{40}\PYG{k+kt}{\PYGZpc{}}\PYG{p}{;}
        \PYG{k}{left}\PYG{p}{:}\PYG{l+m+mi}{0}\PYG{k+kt}{px}\PYG{p}{;}
        \PYG{k}{width}\PYG{p}{:}\PYG{l+m+mi}{17}\PYG{k+kt}{\PYGZpc{}}\PYG{p}{;}
\PYG{p}{\PYGZcb{}}

\PYG{n+nt}{div}\PYG{p}{\PYGZsh{}}\PYG{n+nn}{contenido}\PYG{p}{\PYGZob{}}
        \PYG{k}{width}\PYG{p}{:} \PYG{l+m+mi}{80}\PYG{k+kt}{\PYGZpc{}}\PYG{p}{;}
        \PYG{k}{position}\PYG{p}{:} \PYG{k+kc}{absolute}\PYG{p}{;}
        \PYG{k}{top}\PYG{p}{:}\PYG{l+m+mi}{0}\PYG{k+kt}{px}\PYG{p}{;}
        \PYG{k}{right}\PYG{p}{:} \PYG{l+m+mi}{0}\PYG{k+kt}{px}\PYG{p}{;}
\PYG{p}{\PYGZcb{}}

\PYG{c}{/* Todos los párrafos llevan}
\PYG{c}{ * un pequeño sangrado extra}
\PYG{c}{ * de 15 px en la primera línea*/}
\PYG{n+nt}{p}\PYG{p}{\PYGZob{}}
        \PYG{k}{text\PYGZhy{}indent}\PYG{p}{:} \PYG{l+m+mi}{15}\PYG{k+kt}{px}\PYG{p}{;}
\PYG{p}{\PYGZcb{}}

\PYG{n+nt}{img}\PYG{p}{\PYGZob{}}
        \PYG{k}{width}\PYG{p}{:} \PYG{l+m+mi}{25}\PYG{k+kt}{\PYGZpc{}}\PYG{p}{;}
        \PYG{k}{border}\PYG{p}{:} \PYG{k+kc}{solid} \PYG{k+kc}{black} \PYG{l+m+mi}{1}\PYG{k+kt}{px}\PYG{p}{;}
        \PYG{k}{padding}\PYG{p}{:} \PYG{l+m+mi}{4}\PYG{k+kt}{px}\PYG{p}{;}
        \PYG{k}{display}\PYG{p}{:} \PYG{k+kc}{block}\PYG{p}{;}
        \PYG{k}{margin\PYGZhy{}left}\PYG{p}{:} \PYG{k+kc}{auto}\PYG{p}{;}
        \PYG{k}{margin\PYGZhy{}right}\PYG{p}{:} \PYG{k+kc}{auto}\PYG{p}{;}
\PYG{p}{\PYGZcb{}}
\end{sphinxVerbatim}


\subsection{Ejercicio 2 de maquetación}
\label{\detokenize{tema3:ejercicio-2-de-maquetacion}}
Conseguir una página como esta

\begin{figure}[htbp]
\centering

\noindent\sphinxincludegraphics{{maquetaacme}.png}
\end{figure}


\subsection{HTML}
\label{\detokenize{tema3:id2}}
\begin{sphinxVerbatim}[commandchars=\\\{\}]
\PYG{p}{\PYGZlt{}}\PYG{n+nt}{div} \PYG{n+na}{id}\PYG{o}{=}\PYG{l+s}{\PYGZdq{}contenedorglobal\PYGZdq{}}\PYG{p}{\PYGZgt{}}
        \PYG{p}{\PYGZlt{}}\PYG{n+nt}{div} \PYG{n+na}{id}\PYG{o}{=}\PYG{l+s}{\PYGZdq{}cabecera\PYGZdq{}}\PYG{p}{\PYGZgt{}}
                \PYG{p}{\PYGZlt{}}\PYG{n+nt}{span} \PYG{n+na}{id}\PYG{o}{=}\PYG{l+s}{\PYGZdq{}marcacabecera\PYGZdq{}}\PYG{p}{\PYGZgt{}}
                        ACME
                \PYG{p}{\PYGZlt{}}\PYG{p}{/}\PYG{n+nt}{span}\PYG{p}{\PYGZgt{}}
                \PYG{p}{\PYGZlt{}}\PYG{n+nt}{span} \PYG{n+na}{id}\PYG{o}{=}\PYG{l+s}{\PYGZdq{}lemacabecera\PYGZdq{}}\PYG{p}{\PYGZgt{}}
                        donde hay que comprar
                \PYG{p}{\PYGZlt{}}\PYG{p}{/}\PYG{n+nt}{span}\PYG{p}{\PYGZgt{}}
        \PYG{p}{\PYGZlt{}}\PYG{p}{/}\PYG{n+nt}{div}\PYG{p}{\PYGZgt{}} \PYG{c}{\PYGZlt{}!\PYGZhy{}\PYGZhy{}}\PYG{c}{Fin de la cabecera}\PYG{c}{\PYGZhy{}\PYGZhy{}\PYGZgt{}}
        \PYG{p}{\PYGZlt{}}\PYG{n+nt}{div} \PYG{n+na}{id}\PYG{o}{=}\PYG{l+s}{\PYGZdq{}cuerpo\PYGZdq{}}\PYG{p}{\PYGZgt{}}
                En un lugar de la Mancha ...
        \PYG{p}{\PYGZlt{}}\PYG{p}{/}\PYG{n+nt}{div}\PYG{p}{\PYGZgt{}}
        \PYG{p}{\PYGZlt{}}\PYG{n+nt}{div} \PYG{n+na}{id}\PYG{o}{=}\PYG{l+s}{\PYGZdq{}enlaces\PYGZdq{}}\PYG{p}{\PYGZgt{}}
                \PYG{p}{\PYGZlt{}}\PYG{n+nt}{ol}\PYG{p}{\PYGZgt{}}
                        \PYG{p}{\PYGZlt{}}\PYG{n+nt}{li}\PYG{p}{\PYGZgt{}}
                                \PYG{p}{\PYGZlt{}}\PYG{n+nt}{a} \PYG{n+na}{href}\PYG{o}{=}\PYG{l+s}{\PYGZdq{}google.es\PYGZdq{}}\PYG{p}{\PYGZgt{}}
                                        Google
                                \PYG{p}{\PYGZlt{}}\PYG{p}{/}\PYG{n+nt}{a}\PYG{p}{\PYGZgt{}}
                        \PYG{p}{\PYGZlt{}}\PYG{p}{/}\PYG{n+nt}{li}\PYG{p}{\PYGZgt{}}
                        \PYG{p}{\PYGZlt{}}\PYG{n+nt}{li}\PYG{p}{\PYGZgt{}}
                                \PYG{p}{\PYGZlt{}}\PYG{n+nt}{a} \PYG{n+na}{href}\PYG{o}{=}\PYG{l+s}{\PYGZdq{}google.es\PYGZdq{}}\PYG{p}{\PYGZgt{}}
                                        Google
                                \PYG{p}{\PYGZlt{}}\PYG{p}{/}\PYG{n+nt}{a}\PYG{p}{\PYGZgt{}}
                        \PYG{p}{\PYGZlt{}}\PYG{p}{/}\PYG{n+nt}{li}\PYG{p}{\PYGZgt{}}
                        \PYG{p}{\PYGZlt{}}\PYG{n+nt}{li}\PYG{p}{\PYGZgt{}}
                                \PYG{p}{\PYGZlt{}}\PYG{n+nt}{a} \PYG{n+na}{href}\PYG{o}{=}\PYG{l+s}{\PYGZdq{}google.es\PYGZdq{}}\PYG{p}{\PYGZgt{}}
                                        Google
                                \PYG{p}{\PYGZlt{}}\PYG{p}{/}\PYG{n+nt}{a}\PYG{p}{\PYGZgt{}}
                        \PYG{p}{\PYGZlt{}}\PYG{p}{/}\PYG{n+nt}{li}\PYG{p}{\PYGZgt{}}
                        \PYG{p}{\PYGZlt{}}\PYG{n+nt}{li}\PYG{p}{\PYGZgt{}}
                                \PYG{p}{\PYGZlt{}}\PYG{n+nt}{a} \PYG{n+na}{href}\PYG{o}{=}\PYG{l+s}{\PYGZdq{}google.es\PYGZdq{}}\PYG{p}{\PYGZgt{}}
                                        Google
                                \PYG{p}{\PYGZlt{}}\PYG{p}{/}\PYG{n+nt}{a}\PYG{p}{\PYGZgt{}}
                        \PYG{p}{\PYGZlt{}}\PYG{p}{/}\PYG{n+nt}{li}\PYG{p}{\PYGZgt{}}
                \PYG{p}{\PYGZlt{}}\PYG{p}{/}\PYG{n+nt}{ol}\PYG{p}{\PYGZgt{}}
        \PYG{p}{\PYGZlt{}}\PYG{p}{/}\PYG{n+nt}{div}\PYG{p}{\PYGZgt{}}
\PYG{p}{\PYGZlt{}}\PYG{p}{/}\PYG{n+nt}{div}\PYG{p}{\PYGZgt{}}
\end{sphinxVerbatim}


\subsection{CSS}
\label{\detokenize{tema3:id3}}
\begin{sphinxVerbatim}[commandchars=\\\{\}]
\PYG{p}{\PYGZsh{}}\PYG{n+nn}{cabecera}\PYG{p}{\PYGZob{}}
        \PYG{k}{text\PYGZhy{}align}\PYG{p}{:} \PYG{k+kc}{center}\PYG{p}{;}
        \PYG{k}{background\PYGZhy{}color}\PYG{p}{:}
                \PYG{n+nb}{rgb}\PYG{p}{(}\PYG{l+m+mi}{242}\PYG{p}{,}\PYG{l+m+mi}{227}\PYG{p}{,}\PYG{l+m+mi}{148}\PYG{p}{)}
\PYG{p}{\PYGZcb{}}
\PYG{p}{\PYGZsh{}}\PYG{n+nn}{marcacabecera}\PYG{p}{\PYGZob{}}
        \PYG{k}{font\PYGZhy{}size}\PYG{p}{:} \PYG{k+kc}{larger}\PYG{p}{;}
        \PYG{k}{font\PYGZhy{}family}\PYG{p}{:} \PYG{l+s+s2}{\PYGZdq{}Impact\PYGZdq{}}\PYG{p}{;}
\PYG{p}{\PYGZcb{}}
\PYG{p}{\PYGZsh{}}\PYG{n+nn}{lemacabecera}\PYG{p}{\PYGZob{}}
        \PYG{k}{font\PYGZhy{}style}\PYG{p}{:} \PYG{k+kc}{italic}\PYG{p}{;}
        \PYG{k}{font\PYGZhy{}size}\PYG{p}{:} \PYG{k+kc}{smaller}\PYG{p}{;}
        \PYG{k}{font\PYGZhy{}family}\PYG{p}{:} \PYG{l+s+s2}{\PYGZdq{}Lucida Handwriting\PYGZdq{}}\PYG{p}{;}
\PYG{p}{\PYGZcb{}}

\PYG{n+nt}{div}\PYG{p}{\PYGZsh{}}\PYG{n+nn}{cuerpo}\PYG{o}{,} \PYG{n+nt}{div}\PYG{p}{\PYGZsh{}}\PYG{n+nn}{enlaces}\PYG{p}{\PYGZob{}}
        \PYG{k}{background\PYGZhy{}color}\PYG{p}{:}
                \PYG{n+nb}{rgb}\PYG{p}{(}\PYG{l+m+mi}{217}\PYG{p}{,}\PYG{l+m+mi}{195}\PYG{p}{,}\PYG{l+m+mi}{89}\PYG{p}{)}\PYG{p}{;}

\PYG{p}{\PYGZcb{}}

\PYG{n+nt}{div}\PYG{p}{\PYGZsh{}}\PYG{n+nn}{cuerpo}\PYG{p}{\PYGZob{}}
        \PYG{k}{width}\PYG{p}{:}\PYG{l+m+mi}{65}\PYG{k+kt}{\PYGZpc{}}\PYG{p}{;}
        \PYG{k}{float}\PYG{p}{:}\PYG{k+kc}{left}\PYG{p}{;}
        \PYG{k}{margin\PYGZhy{}top}\PYG{p}{:} \PYG{l+m+mi}{20}\PYG{k+kt}{px}\PYG{p}{;}
        \PYG{k}{padding}\PYG{p}{:}\PYG{l+m+mi}{10}\PYG{k+kt}{px}\PYG{p}{;}
        \PYG{k}{text\PYGZhy{}align}\PYG{p}{:} \PYG{k+kc}{justify}\PYG{p}{;}

\PYG{p}{\PYGZcb{}}
\PYG{n+nt}{div}\PYG{p}{\PYGZsh{}}\PYG{n+nn}{enlaces}\PYG{p}{\PYGZob{}}
        \PYG{k}{width}\PYG{p}{:} \PYG{l+m+mi}{20}\PYG{k+kt}{\PYGZpc{}}\PYG{p}{;}
        \PYG{k}{float}\PYG{p}{:}\PYG{k+kc}{right}\PYG{p}{;}
        \PYG{k}{margin\PYGZhy{}top}\PYG{p}{:}\PYG{l+m+mi}{20}\PYG{k+kt}{px}\PYG{p}{;}
\PYG{p}{\PYGZcb{}}

\PYG{n+nt}{div}\PYG{p}{\PYGZob{}}
        \PYG{k}{border\PYGZhy{}width}\PYG{p}{:} \PYG{l+m+mi}{1}\PYG{k+kt}{px}\PYG{p}{;}
        \PYG{k}{border\PYGZhy{}style}\PYG{p}{:} \PYG{k+kc}{solid}\PYG{p}{;}
        \PYG{k}{border\PYGZhy{}color}\PYG{p}{:} \PYG{k+kc}{black}\PYG{p}{;}
        \PYG{k}{background\PYGZhy{}color}\PYG{p}{:}
                \PYG{n+nb}{rgb}\PYG{p}{(}\PYG{l+m+mi}{230}\PYG{p}{,}\PYG{l+m+mi}{230}\PYG{p}{,} \PYG{l+m+mi}{230}\PYG{p}{)}\PYG{p}{;}
\PYG{p}{\PYGZcb{}}

\PYG{n+nt}{div}\PYG{p}{\PYGZsh{}}\PYG{n+nn}{contenedorglobal}\PYG{p}{\PYGZob{}}
        \PYG{k}{background\PYGZhy{}color}\PYG{p}{:}
                \PYG{n+nb}{rgb}\PYG{p}{(}\PYG{l+m+mi}{188}\PYG{p}{,}\PYG{l+m+mi}{182}\PYG{p}{,}\PYG{l+m+mi}{175}\PYG{p}{)}\PYG{p}{;}
\PYG{p}{\PYGZcb{}}
\end{sphinxVerbatim}


\subsection{Ejercicio: barra de herramientas}
\label{\detokenize{tema3:ejercicio-barra-de-herramientas}}
Crear una página con dos cajas diferenciadas. Una de ellas, que ocupará el 30\% de la página contendrá enlaces a diferentes sitios web. La caja no se moverá aunque el usuario haga scroll. Por  otro lado, la otra caja ocupará el 70\% de la página y habrá que llenarla de texto para poder desplazarse por él y comprobar que la caja de enlaces no se mueve.

El HTML sería algo así:

\begin{sphinxVerbatim}[commandchars=\\\{\}]
\PYG{c+cp}{\PYGZlt{}!DOCTYPE html\PYGZgt{}}
\PYG{p}{\PYGZlt{}}\PYG{n+nt}{html}\PYG{p}{\PYGZgt{}}
\PYG{p}{\PYGZlt{}}\PYG{n+nt}{head}\PYG{p}{\PYGZgt{}}
        \PYG{p}{\PYGZlt{}}\PYG{n+nt}{link} \PYG{n+na}{rel}\PYG{o}{=}\PYG{l+s}{\PYGZdq{}stylesheet\PYGZdq{}} \PYG{n+na}{href}\PYG{o}{=}\PYG{l+s}{\PYGZdq{}solucion1.css\PYGZdq{}} \PYG{n+na}{type}\PYG{o}{=}\PYG{l+s}{\PYGZdq{}text/css\PYGZdq{}}\PYG{p}{/}\PYG{p}{\PYGZgt{}}
        \PYG{p}{\PYGZlt{}}\PYG{n+nt}{title}\PYG{p}{\PYGZgt{}}Ejercicio 1\PYG{p}{\PYGZlt{}}\PYG{p}{/}\PYG{n+nt}{title}\PYG{p}{\PYGZgt{}}
\PYG{p}{\PYGZlt{}}\PYG{p}{/}\PYG{n+nt}{head}\PYG{p}{\PYGZgt{}}
\PYG{p}{\PYGZlt{}}\PYG{n+nt}{body}\PYG{p}{\PYGZgt{}}
\PYG{p}{\PYGZlt{}}\PYG{n+nt}{div} \PYG{n+na}{id}\PYG{o}{=}\PYG{l+s}{\PYGZdq{}enlaces\PYGZdq{}}\PYG{p}{\PYGZgt{}}
        \PYG{p}{\PYGZlt{}}\PYG{n+nt}{ul}\PYG{p}{\PYGZgt{}}
                \PYG{p}{\PYGZlt{}}\PYG{n+nt}{li}\PYG{p}{\PYGZgt{}}
                        \PYG{p}{\PYGZlt{}}\PYG{n+nt}{a} \PYG{n+na}{href}\PYG{o}{=}\PYG{l+s}{\PYGZdq{}http://cocacola.com\PYGZdq{}}\PYG{p}{\PYGZgt{}}CocaCola\PYG{p}{\PYGZlt{}}\PYG{p}{/}\PYG{n+nt}{a}\PYG{p}{\PYGZgt{}}
                \PYG{p}{\PYGZlt{}}\PYG{p}{/}\PYG{n+nt}{li}\PYG{p}{\PYGZgt{}}
                \PYG{p}{\PYGZlt{}}\PYG{n+nt}{li}\PYG{p}{\PYGZgt{}}
                        \PYG{p}{\PYGZlt{}}\PYG{n+nt}{a} \PYG{n+na}{href}\PYG{o}{=}\PYG{l+s}{\PYGZdq{}http://google.com\PYGZdq{}}\PYG{p}{\PYGZgt{}}Google\PYG{p}{\PYGZlt{}}\PYG{p}{/}\PYG{n+nt}{a}\PYG{p}{\PYGZgt{}}
                \PYG{p}{\PYGZlt{}}\PYG{p}{/}\PYG{n+nt}{li}\PYG{p}{\PYGZgt{}}
                \PYG{p}{\PYGZlt{}}\PYG{n+nt}{li}\PYG{p}{\PYGZgt{}}
                        \PYG{p}{\PYGZlt{}}\PYG{n+nt}{a} \PYG{n+na}{href}\PYG{o}{=}\PYG{l+s}{\PYGZdq{}http://terra.es\PYGZdq{}}\PYG{p}{\PYGZgt{}}Terra\PYG{p}{\PYGZlt{}}\PYG{p}{/}\PYG{n+nt}{a}\PYG{p}{\PYGZgt{}}
                \PYG{p}{\PYGZlt{}}\PYG{p}{/}\PYG{n+nt}{li}\PYG{p}{\PYGZgt{}}

        \PYG{p}{\PYGZlt{}}\PYG{p}{/}\PYG{n+nt}{ul}\PYG{p}{\PYGZgt{}}
\PYG{p}{\PYGZlt{}}\PYG{p}{/}\PYG{n+nt}{div}\PYG{p}{\PYGZgt{}}
\PYG{p}{\PYGZlt{}}\PYG{n+nt}{div} \PYG{n+na}{id}\PYG{o}{=}\PYG{l+s}{\PYGZdq{}contenido\PYGZdq{}}\PYG{p}{\PYGZgt{}}
        En un lugar de la Mancha..
        En ...
\PYG{p}{\PYGZlt{}}\PYG{p}{/}\PYG{n+nt}{div}\PYG{p}{\PYGZgt{}}
\PYG{p}{\PYGZlt{}}\PYG{p}{/}\PYG{n+nt}{body}\PYG{p}{\PYGZgt{}}
\PYG{p}{\PYGZlt{}}\PYG{p}{/}\PYG{n+nt}{html}\PYG{p}{\PYGZgt{}}
\end{sphinxVerbatim}


\subsection{Ejercicio: ampliación}
\label{\detokenize{tema3:ejercicio-ampliacion}}
Ampliar el ejemplo anterior para hacer que el contenido solo ocupe el 50\% y añadir una barra de publicidad fija en el centro vertical que ocupe el 20\%.

El HTML sería

\begin{sphinxVerbatim}[commandchars=\\\{\}]
\PYG{c+cp}{\PYGZlt{}!DOCTYPE html\PYGZgt{}}
\PYG{p}{\PYGZlt{}}\PYG{n+nt}{html}\PYG{p}{\PYGZgt{}}
\PYG{p}{\PYGZlt{}}\PYG{n+nt}{head}\PYG{p}{\PYGZgt{}}
        \PYG{p}{\PYGZlt{}}\PYG{n+nt}{link} \PYG{n+na}{rel}\PYG{o}{=}\PYG{l+s}{\PYGZdq{}stylesheet\PYGZdq{}} \PYG{n+na}{href}\PYG{o}{=}\PYG{l+s}{\PYGZdq{}solucion2.css\PYGZdq{}} \PYG{n+na}{type}\PYG{o}{=}\PYG{l+s}{\PYGZdq{}text/css\PYGZdq{}}\PYG{p}{/}\PYG{p}{\PYGZgt{}}
        \PYG{p}{\PYGZlt{}}\PYG{n+nt}{title}\PYG{p}{\PYGZgt{}}Ejercicio 1\PYG{p}{\PYGZlt{}}\PYG{p}{/}\PYG{n+nt}{title}\PYG{p}{\PYGZgt{}}
\PYG{p}{\PYGZlt{}}\PYG{p}{/}\PYG{n+nt}{head}\PYG{p}{\PYGZgt{}}
\PYG{p}{\PYGZlt{}}\PYG{n+nt}{body}\PYG{p}{\PYGZgt{}}
\PYG{p}{\PYGZlt{}}\PYG{n+nt}{div} \PYG{n+na}{id}\PYG{o}{=}\PYG{l+s}{\PYGZdq{}enlaces\PYGZdq{}}\PYG{p}{\PYGZgt{}}
        \PYG{p}{\PYGZlt{}}\PYG{n+nt}{ul}\PYG{p}{\PYGZgt{}}
                \PYG{p}{\PYGZlt{}}\PYG{n+nt}{li}\PYG{p}{\PYGZgt{}}
                        \PYG{p}{\PYGZlt{}}\PYG{n+nt}{a} \PYG{n+na}{href}\PYG{o}{=}\PYG{l+s}{\PYGZdq{}http://cocacola.com\PYGZdq{}}\PYG{p}{\PYGZgt{}}CocaCola\PYG{p}{\PYGZlt{}}\PYG{p}{/}\PYG{n+nt}{a}\PYG{p}{\PYGZgt{}}
                \PYG{p}{\PYGZlt{}}\PYG{p}{/}\PYG{n+nt}{li}\PYG{p}{\PYGZgt{}}
                \PYG{p}{\PYGZlt{}}\PYG{n+nt}{li}\PYG{p}{\PYGZgt{}}
                        \PYG{p}{\PYGZlt{}}\PYG{n+nt}{a} \PYG{n+na}{href}\PYG{o}{=}\PYG{l+s}{\PYGZdq{}http://google.com\PYGZdq{}}\PYG{p}{\PYGZgt{}}Google\PYG{p}{\PYGZlt{}}\PYG{p}{/}\PYG{n+nt}{a}\PYG{p}{\PYGZgt{}}
                \PYG{p}{\PYGZlt{}}\PYG{p}{/}\PYG{n+nt}{li}\PYG{p}{\PYGZgt{}}
                \PYG{p}{\PYGZlt{}}\PYG{n+nt}{li}\PYG{p}{\PYGZgt{}}
                        \PYG{p}{\PYGZlt{}}\PYG{n+nt}{a} \PYG{n+na}{href}\PYG{o}{=}\PYG{l+s}{\PYGZdq{}http://terra.es\PYGZdq{}}\PYG{p}{\PYGZgt{}}Terra\PYG{p}{\PYGZlt{}}\PYG{p}{/}\PYG{n+nt}{a}\PYG{p}{\PYGZgt{}}
                \PYG{p}{\PYGZlt{}}\PYG{p}{/}\PYG{n+nt}{li}\PYG{p}{\PYGZgt{}}

        \PYG{p}{\PYGZlt{}}\PYG{p}{/}\PYG{n+nt}{ul}\PYG{p}{\PYGZgt{}}
\PYG{p}{\PYGZlt{}}\PYG{p}{/}\PYG{n+nt}{div}\PYG{p}{\PYGZgt{}}
\PYG{p}{\PYGZlt{}}\PYG{n+nt}{div} \PYG{n+na}{id}\PYG{o}{=}\PYG{l+s}{\PYGZdq{}publi\PYGZdq{}}\PYG{p}{\PYGZgt{}}
        \PYG{p}{\PYGZlt{}}\PYG{n+nt}{ul}\PYG{p}{\PYGZgt{}}
                \PYG{p}{\PYGZlt{}}\PYG{n+nt}{li}\PYG{p}{\PYGZgt{}}
                        \PYG{p}{\PYGZlt{}}\PYG{n+nt}{a} \PYG{n+na}{href}\PYG{o}{=}\PYG{l+s}{\PYGZdq{}http://iesmaestredecalatrava.es\PYGZdq{}}\PYG{p}{\PYGZgt{}}IES\PYG{p}{\PYGZlt{}}\PYG{p}{/}\PYG{n+nt}{a}\PYG{p}{\PYGZgt{}}
                \PYG{p}{\PYGZlt{}}\PYG{p}{/}\PYG{n+nt}{li}\PYG{p}{\PYGZgt{}}
        \PYG{p}{\PYGZlt{}}\PYG{p}{/}\PYG{n+nt}{ul}\PYG{p}{\PYGZgt{}}
\PYG{p}{\PYGZlt{}}\PYG{p}{/}\PYG{n+nt}{div}\PYG{p}{\PYGZgt{}}
\PYG{p}{\PYGZlt{}}\PYG{n+nt}{div} \PYG{n+na}{id}\PYG{o}{=}\PYG{l+s}{\PYGZdq{}contenido\PYGZdq{}}\PYG{p}{\PYGZgt{}}
        En un lugar de la Mancha.. (repetido)
\PYG{p}{\PYGZlt{}}\PYG{p}{/}\PYG{n+nt}{div}\PYG{p}{\PYGZgt{}}


\PYG{p}{\PYGZlt{}}\PYG{p}{/}\PYG{n+nt}{body}\PYG{p}{\PYGZgt{}}
\PYG{p}{\PYGZlt{}}\PYG{p}{/}\PYG{n+nt}{html}\PYG{p}{\PYGZgt{}}
\end{sphinxVerbatim}


\subsection{Ejercicio}
\label{\detokenize{tema3:ejercicio}}
Crear una estructura de página con una cabecera que ocupe el 100\% de la página, con texto centrado y algunos enlaces. A continuación un bloque de contenido que ocupe el 70\% y a su izquierda un bloque de publicidad que ocupe el 30\%. Debe haber un pie de página con el copyright que ocupe el 100\% de la página, con el texto centrado y que no se mueva cuando el usuario desplace el texto.

El HTML sería

\begin{sphinxVerbatim}[commandchars=\\\{\}]
\PYG{c+cp}{\PYGZlt{}!DOCTYPE html\PYGZgt{}}

\PYG{p}{\PYGZlt{}}\PYG{n+nt}{html}\PYG{p}{\PYGZgt{}}
\PYG{p}{\PYGZlt{}}\PYG{n+nt}{head}\PYG{p}{\PYGZgt{}}
        \PYG{p}{\PYGZlt{}}\PYG{n+nt}{link} \PYG{n+na}{href}\PYG{o}{=}\PYG{l+s}{\PYGZdq{}solucion4.css\PYGZdq{}} \PYG{n+na}{rel}\PYG{o}{=}\PYG{l+s}{\PYGZdq{}stylesheet\PYGZdq{}} \PYG{n+na}{type}\PYG{o}{=}\PYG{l+s}{\PYGZdq{}text/css\PYGZdq{}}\PYG{p}{\PYGZgt{}}
        \PYG{p}{\PYGZlt{}}\PYG{n+nt}{title}\PYG{p}{\PYGZgt{}}Ejercicio\PYG{p}{\PYGZlt{}}\PYG{p}{/}\PYG{n+nt}{title}\PYG{p}{\PYGZgt{}}
\PYG{p}{\PYGZlt{}}\PYG{p}{/}\PYG{n+nt}{head}\PYG{p}{\PYGZgt{}}

\PYG{p}{\PYGZlt{}}\PYG{n+nt}{body}\PYG{p}{\PYGZgt{}}

\PYG{p}{\PYGZlt{}}\PYG{n+nt}{header} \PYG{n+na}{id}\PYG{o}{=}\PYG{l+s}{\PYGZdq{}cabecera\PYGZdq{}}\PYG{p}{\PYGZgt{}}
        \PYG{p}{\PYGZlt{}}\PYG{n+nt}{a} \PYG{n+na}{href}\PYG{o}{=}\PYG{l+s}{\PYGZdq{}http://google.es\PYGZdq{}}\PYG{p}{\PYGZgt{}}Google\PYG{p}{\PYGZlt{}}\PYG{p}{/}\PYG{n+nt}{a}\PYG{p}{\PYGZgt{}}
        \PYG{p}{\PYGZlt{}}\PYG{n+nt}{a} \PYG{n+na}{href}\PYG{o}{=}\PYG{l+s}{\PYGZdq{}http://terra.es\PYGZdq{}}\PYG{p}{\PYGZgt{}}Terra\PYG{p}{\PYGZlt{}}\PYG{p}{/}\PYG{n+nt}{a}\PYG{p}{\PYGZgt{}}
\PYG{p}{\PYGZlt{}}\PYG{p}{/}\PYG{n+nt}{header}\PYG{p}{\PYGZgt{}}
\PYG{p}{\PYGZlt{}}\PYG{n+nt}{section} \PYG{n+na}{id}\PYG{o}{=}\PYG{l+s}{\PYGZdq{}contenido\PYGZdq{}}\PYG{p}{\PYGZgt{}}
        Texto texto texto ...

\PYG{p}{\PYGZlt{}}\PYG{p}{/}\PYG{n+nt}{section}\PYG{p}{\PYGZgt{}}
\PYG{p}{\PYGZlt{}}\PYG{n+nt}{aside} \PYG{n+na}{id}\PYG{o}{=}\PYG{l+s}{\PYGZdq{}publi\PYGZdq{}}\PYG{p}{\PYGZgt{}}
        \PYG{p}{\PYGZlt{}}\PYG{n+nt}{a} \PYG{n+na}{href}\PYG{o}{=}\PYG{l+s}{\PYGZdq{}http://cocacola.es\PYGZdq{}}\PYG{p}{\PYGZgt{}}Beba Coca\PYGZhy{}Cola\PYG{p}{\PYGZlt{}}\PYG{p}{/}\PYG{n+nt}{a}\PYG{p}{\PYGZgt{}}
\PYG{p}{\PYGZlt{}}\PYG{p}{/}\PYG{n+nt}{aside}\PYG{p}{\PYGZgt{}}
\PYG{p}{\PYGZlt{}}\PYG{n+nt}{footer}\PYG{p}{\PYGZgt{}}
        \PYG{n+ni}{\PYGZam{}copy;} Pepe Perez, IES Maestre 2013\PYGZhy{}2014
\PYG{p}{\PYGZlt{}}\PYG{p}{/}\PYG{n+nt}{footer}\PYG{p}{\PYGZgt{}}
\PYG{p}{\PYGZlt{}}\PYG{p}{/}\PYG{n+nt}{body}\PYG{p}{\PYGZgt{}}
\PYG{p}{\PYGZlt{}}\PYG{p}{/}\PYG{n+nt}{html}\PYG{p}{\PYGZgt{}}
\end{sphinxVerbatim}


\section{Posicionamiento float}
\label{\detokenize{tema3:posicionamiento-float}}
En el posicionamiento \sphinxcode{float} solo indicaremos la anchura de una caja. El resto de los elementos se encajará automáticamente en el espacio restante dejado por dicha caja.

Deben recordarse algunas cosas:
\begin{itemize}
\item {} 
Un elemento \sphinxcode{float} lo que hace es \sphinxstyleemphasis{dar permiso a otros elementos} para que ocupen el espacio que le ha sobrado.

\item {} 
Cuando se usan varios elementos \sphinxcode{float} es posible que otros elementos tengan que hacer \sphinxcode{clear:both;} para asegurarnos de que dejen de aprovechar el espacio sobrante.

\item {} 
Es muy frecuente que haya que «crear contenedores extra» para conseguir colocar cajas utilizando elementos \sphinxcode{float}.

\end{itemize}

Por ejemplo, supongamos el siguiente HTML:

\begin{sphinxVerbatim}[commandchars=\\\{\}]
\PYG{p}{\PYGZlt{}}\PYG{n+nt}{div} \PYG{n+na}{id}\PYG{o}{=}\PYG{l+s}{\PYGZdq{}caja1\PYGZdq{}}\PYG{p}{\PYGZgt{}}
    Texto...
\PYG{p}{\PYGZlt{}}\PYG{p}{/}\PYG{n+nt}{div}\PYG{p}{\PYGZgt{}}
\PYG{p}{\PYGZlt{}}\PYG{n+nt}{div} \PYG{n+na}{id}\PYG{o}{=}\PYG{l+s}{\PYGZdq{}caja2\PYGZdq{}}\PYG{p}{\PYGZgt{}}
    Texto...
\PYG{p}{\PYGZlt{}}\PYG{p}{/}\PYG{n+nt}{div}\PYG{p}{\PYGZgt{}}
\PYG{p}{\PYGZlt{}}\PYG{n+nt}{div} \PYG{n+na}{id}\PYG{o}{=}\PYG{l+s}{\PYGZdq{}caja3\PYGZdq{}}\PYG{p}{\PYGZgt{}}
    Texto...
\PYG{p}{\PYGZlt{}}\PYG{p}{/}\PYG{n+nt}{div}\PYG{p}{\PYGZgt{}}
\PYG{p}{\PYGZlt{}}\PYG{n+nt}{div} \PYG{n+na}{id}\PYG{o}{=}\PYG{l+s}{\PYGZdq{}caja4\PYGZdq{}}\PYG{p}{\PYGZgt{}}
    Texto...
\PYG{p}{\PYGZlt{}}\PYG{p}{/}\PYG{n+nt}{div}\PYG{p}{\PYGZgt{}}
\end{sphinxVerbatim}

Y supongamos que se necesita que las cuatro cajas queden colocadas verticalmente en pantalla y siguiendo el orden «caja4», «caja3», «caja1» y «caja2». Una posible solución sería crear dos contenedores extra de la manera siguiente:

\begin{sphinxVerbatim}[commandchars=\\\{\}]
\PYG{p}{\PYGZlt{}}\PYG{n+nt}{div} \PYG{n+na}{id}\PYG{o}{=}\PYG{l+s}{\PYGZdq{}contenedor1\PYGZdq{}}\PYG{p}{\PYGZgt{}}
    \PYG{p}{\PYGZlt{}}\PYG{n+nt}{div} \PYG{n+na}{id}\PYG{o}{=}\PYG{l+s}{\PYGZdq{}caja1\PYGZdq{}}\PYG{p}{\PYGZgt{}}
        Texto...
    \PYG{p}{\PYGZlt{}}\PYG{p}{/}\PYG{n+nt}{div}\PYG{p}{\PYGZgt{}}
    \PYG{p}{\PYGZlt{}}\PYG{n+nt}{div} \PYG{n+na}{id}\PYG{o}{=}\PYG{l+s}{\PYGZdq{}caja2\PYGZdq{}}\PYG{p}{\PYGZgt{}}
        Texto...
    \PYG{p}{\PYGZlt{}}\PYG{p}{/}\PYG{n+nt}{div}\PYG{p}{\PYGZgt{}}
\PYG{p}{\PYGZlt{}}\PYG{p}{/}\PYG{n+nt}{div}\PYG{p}{\PYGZgt{}} \PYG{c}{\PYGZlt{}!\PYGZhy{}\PYGZhy{}}\PYG{c}{Fin del contenedor1}\PYG{c}{\PYGZhy{}\PYGZhy{}\PYGZgt{}}
\PYG{p}{\PYGZlt{}}\PYG{n+nt}{div} \PYG{n+na}{id}\PYG{o}{=}\PYG{l+s}{\PYGZdq{}contenedor2\PYGZdq{}}\PYG{p}{\PYGZgt{}}
    \PYG{p}{\PYGZlt{}}\PYG{n+nt}{div} \PYG{n+na}{id}\PYG{o}{=}\PYG{l+s}{\PYGZdq{}caja3\PYGZdq{}}\PYG{p}{\PYGZgt{}}
        Texto...
    \PYG{p}{\PYGZlt{}}\PYG{p}{/}\PYG{n+nt}{div}\PYG{p}{\PYGZgt{}}
    \PYG{p}{\PYGZlt{}}\PYG{n+nt}{div} \PYG{n+na}{id}\PYG{o}{=}\PYG{l+s}{\PYGZdq{}caja4\PYGZdq{}}\PYG{p}{\PYGZgt{}}
        Texto...
    \PYG{p}{\PYGZlt{}}\PYG{p}{/}\PYG{n+nt}{div}\PYG{p}{\PYGZgt{}}
\PYG{p}{\PYGZlt{}}\PYG{p}{/}\PYG{n+nt}{div}\PYG{p}{\PYGZgt{}} \PYG{c}{\PYGZlt{}!\PYGZhy{}\PYGZhy{}}\PYG{c}{Fin del contenedor2}\PYG{c}{\PYGZhy{}\PYGZhy{}\PYGZgt{}}
\end{sphinxVerbatim}

Y ahora haríamos lo siguiente:
\begin{itemize}
\item {} 
El contenedor 1 flotará hacia la derecha. Dentro de él la caja 1 flotará a la izquierda y la caja 2 flotará a la derecha.

\item {} 
El contenedor 2 flotará hacia la izquierda. Dentro de él la caja 4 flotará hacia la izquierda y la caja 3 hacia la derecha.

\end{itemize}

Probemos el siguiente CSS:

\begin{sphinxVerbatim}[commandchars=\\\{\}]
\PYG{p}{\PYGZsh{}}\PYG{n+nn}{contenedor}\PYG{p}{\PYGZob{}}
    \PYG{k}{float}\PYG{p}{:}\PYG{k+kc}{right}\PYG{p}{;}
    \PYG{k}{width}\PYG{p}{:}\PYG{l+m+mi}{48}\PYG{k+kt}{\PYGZpc{}}\PYG{p}{;}
\PYG{p}{\PYGZcb{}}
\PYG{p}{\PYGZsh{}}\PYG{n+nn}{contenedor2}\PYG{p}{\PYGZob{}}
    \PYG{k}{float}\PYG{p}{:}\PYG{k+kc}{left}\PYG{p}{;}
    \PYG{k}{width}\PYG{p}{:}\PYG{l+m+mi}{48}\PYG{k+kt}{\PYGZpc{}}\PYG{p}{;}
\PYG{p}{\PYGZcb{}}
\PYG{p}{\PYGZsh{}}\PYG{n+nn}{caja1}\PYG{o}{,} \PYG{p}{\PYGZsh{}}\PYG{n+nn}{caja4}\PYG{p}{\PYGZob{}}
    \PYG{k}{float}\PYG{p}{:}\PYG{k+kc}{left}\PYG{p}{;}
    \PYG{k}{width}\PYG{p}{:}\PYG{l+m+mi}{48}\PYG{k+kt}{\PYGZpc{}}\PYG{p}{;}
\PYG{p}{\PYGZcb{}}
\PYG{p}{\PYGZsh{}}\PYG{n+nn}{caja2}\PYG{o}{,} \PYG{p}{\PYGZsh{}}\PYG{n+nn}{caja3}\PYG{p}{\PYGZob{}}
    \PYG{k}{float}\PYG{p}{:}\PYG{k+kc}{right}\PYG{p}{;}
    \PYG{k}{width}\PYG{p}{:}\PYG{l+m+mi}{48}\PYG{k+kt}{\PYGZpc{}}\PYG{p}{;}
\PYG{p}{\PYGZcb{}}
\end{sphinxVerbatim}

El resultado será:

\begin{figure}[htbp]
\centering
\capstart

\noindent\sphinxincludegraphics{{cuatro_cajas_con_floats}.png}
\caption{Cuatro cajas manipuladas con \sphinxcode{float}}\label{\detokenize{tema3:id7}}\end{figure}


\subsection{Ejercicio}
\label{\detokenize{tema3:id4}}
Crear una página con una cabecera que ocupe el 100\%, que tenga el texto centrado y una zona debajo que tenga 3 partes: contenido (60\%), enlaces\_relacionados (20\%) y publicidad(20\% restante). Crear un pie de página con una anchura del 100\%.


\section{Maquetación avanzada con \sphinxstyleliteralintitle{grid-layouts}}
\label{\detokenize{tema3:maquetacion-avanzada-con-grid-layouts}}
La llegada de CSS 3 ha supuesto un gran cambio en la forma de maquetar páginas, ya que esta versión ha incluido una novedad llamada «grid-layouts», que nos permiten varias características interesantes:
\begin{enumerate}
\item {} 
Un elemento (por ejemplo un \sphinxcode{div}) se puede dividir en forma de tabla

\item {} 
Dicha tabla puede tener columnas y/o filas de distinto tamaño.

\item {} 
Un elemento hijo puede ocupar la celda que le toque o llenar un área de varias celdas.

\end{enumerate}

Para poder utilizarlo necesitaremos:
\begin{itemize}
\item {} 
Un contenedor principal (en concreto un \sphinxcode{div}). Este contenedor se portará como una tabla en la que definiremos las filas y las columnas

\item {} 
Un conjunto de elementos (normalmente otros \sphinxcode{div}) que irán dentro del contenedor principal. Estos elementos se portarán como «celdas flexibles», ya que podremos colocar cada celda donde queramos y hacer que ocupe las filas y columnas que queramos.

\end{itemize}

Supongamos el siguiente HTML

\begin{sphinxVerbatim}[commandchars=\\\{\}]
\PYG{p}{\PYGZlt{}}\PYG{n+nt}{div} \PYG{n+na}{id}\PYG{o}{=}\PYG{l+s}{\PYGZdq{}contenedor\PYGZdq{}}\PYG{p}{\PYGZgt{}}
    \PYG{p}{\PYGZlt{}}\PYG{n+nt}{div} \PYG{n+na}{class}\PYG{o}{=}\PYG{l+s}{\PYGZdq{}celda\PYGZdq{}} \PYG{n+na}{id}\PYG{o}{=}\PYG{l+s}{\PYGZdq{}a\PYGZdq{}}\PYG{p}{\PYGZgt{}}Celda A\PYG{p}{\PYGZlt{}}\PYG{p}{/}\PYG{n+nt}{div}\PYG{p}{\PYGZgt{}}
    \PYG{p}{\PYGZlt{}}\PYG{n+nt}{div} \PYG{n+na}{class}\PYG{o}{=}\PYG{l+s}{\PYGZdq{}celda\PYGZdq{}} \PYG{n+na}{id}\PYG{o}{=}\PYG{l+s}{\PYGZdq{}b\PYGZdq{}}\PYG{p}{\PYGZgt{}}Celda B\PYG{p}{\PYGZlt{}}\PYG{p}{/}\PYG{n+nt}{div}\PYG{p}{\PYGZgt{}}
    \PYG{p}{\PYGZlt{}}\PYG{n+nt}{div} \PYG{n+na}{class}\PYG{o}{=}\PYG{l+s}{\PYGZdq{}celda\PYGZdq{}} \PYG{n+na}{id}\PYG{o}{=}\PYG{l+s}{\PYGZdq{}c\PYGZdq{}}\PYG{p}{\PYGZgt{}}Celda C\PYG{p}{\PYGZlt{}}\PYG{p}{/}\PYG{n+nt}{div}\PYG{p}{\PYGZgt{}}
    \PYG{p}{\PYGZlt{}}\PYG{n+nt}{div} \PYG{n+na}{class}\PYG{o}{=}\PYG{l+s}{\PYGZdq{}celda\PYGZdq{}} \PYG{n+na}{id}\PYG{o}{=}\PYG{l+s}{\PYGZdq{}d\PYGZdq{}}\PYG{p}{\PYGZgt{}}Celda D\PYG{p}{\PYGZlt{}}\PYG{p}{/}\PYG{n+nt}{div}\PYG{p}{\PYGZgt{}}
    \PYG{p}{\PYGZlt{}}\PYG{n+nt}{div} \PYG{n+na}{class}\PYG{o}{=}\PYG{l+s}{\PYGZdq{}celda\PYGZdq{}} \PYG{n+na}{id}\PYG{o}{=}\PYG{l+s}{\PYGZdq{}e\PYGZdq{}}\PYG{p}{\PYGZgt{}}Celda E\PYG{p}{\PYGZlt{}}\PYG{p}{/}\PYG{n+nt}{div}\PYG{p}{\PYGZgt{}}
    \PYG{p}{\PYGZlt{}}\PYG{n+nt}{div} \PYG{n+na}{class}\PYG{o}{=}\PYG{l+s}{\PYGZdq{}celda\PYGZdq{}} \PYG{n+na}{id}\PYG{o}{=}\PYG{l+s}{\PYGZdq{}f\PYGZdq{}}\PYG{p}{\PYGZgt{}}Celda F\PYG{p}{\PYGZlt{}}\PYG{p}{/}\PYG{n+nt}{div}\PYG{p}{\PYGZgt{}}
\PYG{p}{\PYGZlt{}}\PYG{p}{/}\PYG{n+nt}{div}\PYG{p}{\PYGZgt{}}
\end{sphinxVerbatim}

Si no le hacemos nada se verá lo siguiente:

\begin{figure}[htbp]
\centering
\capstart

\noindent\sphinxincludegraphics{{grid_sin_estilo}.png}
\caption{Grid sin maquetar}\label{\detokenize{tema3:id8}}\end{figure}

Vamos a añadir bordes a la celdas para que se vea  mejor el efecto. Usemos el siguiente CSS

\begin{sphinxVerbatim}[commandchars=\\\{\}]
\PYG{p}{.}\PYG{n+nc}{celda}\PYG{p}{\PYGZob{}}
            \PYG{k}{border}\PYG{p}{:} \PYG{k+kc}{solid} \PYG{k+kc}{black} \PYG{l+m+mi}{1}\PYG{k+kt}{px}\PYG{p}{;}
\PYG{p}{\PYGZcb{}}
\end{sphinxVerbatim}

El resultado queda así:

\begin{figure}[htbp]
\centering
\capstart

\noindent\sphinxincludegraphics{{celdas_con_borde}.png}
\caption{Celdas con borde}\label{\detokenize{tema3:id9}}\end{figure}

Supongamos que queremos que el \sphinxcode{div} cuyo class es \sphinxcode{contenedor} se porte como una tabla de 5 por 5. ¡Recuérdese que
solo tenemos 6 celdas! A pesar de eso, queremos una distribución como esta:

\begin{figure}[htbp]
\centering
\capstart

\noindent\sphinxincludegraphics{{tabla_5_por_5}.png}
\caption{Tabla de 5 x 5}\label{\detokenize{tema3:id10}}\end{figure}

Y ahora supongamos que dentro de esa tabla queremos repartir los elementos de manera que quede más o menos como lo siguiente:

\begin{figure}[htbp]
\centering
\capstart

\noindent\sphinxincludegraphics{{resultado_5_por_5}.png}
\caption{Tabla de 5 x 5}\label{\detokenize{tema3:id11}}\end{figure}

Analizando lo que se pide se observa que:
\begin{itemize}
\item {} 
La «rejilla invisible» ocupa todo el ancho de la pantalla y es más alta de lo normal (pondremos una medida vertical en pixeles que sea razonablemente grande para poder apreciar el efecto)

\item {} 
La celda A empieza en la fila 1 y llega hasta la 3. Empieza en la columna 4 y llega hasta la 6 \sphinxstylestrong{que sabemos que no existe, sino que es el límite de la tabla}.

\item {} 
La celda B empieza en la fila 3 y llega hasta la 4. En columnas va de la 2 a la 5.

\item {} 
La celda C va de la fila 1 a la 6 y está solo en la columna 1.

\item {} 
La D va de la fila 3 a la 6 y solo ocupa la 5.

\item {} 
La E va de la fila 1 a la 3 y de la columna 2 a la 4.

\item {} 
La celda F va de la fila 4 a la 6 y de la columna 2 a la 5.

\end{itemize}

Además añadiremos algún color a tales «celdas» para que podamos ver el área que ocupan.

\begin{sphinxVerbatim}[commandchars=\\\{\}]
\PYG{p}{\PYGZsh{}}\PYG{n+nn}{contenedor}\PYG{p}{\PYGZob{}}
    \PYG{k}{display}\PYG{p}{:} \PYG{k}{grid}\PYG{p}{;}
    \PYG{k}{grid\PYGZhy{}template\PYGZhy{}rows}\PYG{p}{:} \PYG{l+m+mi}{20}\PYG{k+kt}{\PYGZpc{}} \PYG{l+m+mi}{20}\PYG{k+kt}{\PYGZpc{}} \PYG{l+m+mi}{20}\PYG{k+kt}{\PYGZpc{}} \PYG{l+m+mi}{20}\PYG{k+kt}{\PYGZpc{}} \PYG{l+m+mi}{20}\PYG{k+kt}{\PYGZpc{}}\PYG{p}{;}
    \PYG{k}{grid\PYGZhy{}template\PYGZhy{}columns}\PYG{p}{:} \PYG{l+m+mi}{20}\PYG{k+kt}{\PYGZpc{}} \PYG{l+m+mi}{20}\PYG{k+kt}{\PYGZpc{}} \PYG{l+m+mi}{20}\PYG{k+kt}{\PYGZpc{}} \PYG{l+m+mi}{20}\PYG{k+kt}{\PYGZpc{}} \PYG{l+m+mi}{20}\PYG{k+kt}{\PYGZpc{}}\PYG{p}{;}
    \PYG{k}{width}\PYG{p}{:}\PYG{l+m+mi}{100}\PYG{k+kt}{\PYGZpc{}}\PYG{p}{;}
    \PYG{k}{height}\PYG{p}{:}\PYG{l+m+mi}{640}\PYG{k+kt}{px}\PYG{p}{;}
\PYG{p}{\PYGZcb{}}
\PYG{p}{.}\PYG{n+nc}{celda}\PYG{p}{\PYGZob{}}
    \PYG{k}{border}\PYG{p}{:} \PYG{k+kc}{solid} \PYG{k+kc}{black} \PYG{l+m+mi}{1}\PYG{k+kt}{px}\PYG{p}{;}
\PYG{p}{\PYGZcb{}}
\PYG{p}{\PYGZsh{}}\PYG{n+nn}{a}\PYG{p}{\PYGZob{}}
    \PYG{k}{grid\PYGZhy{}row}\PYG{p}{:} \PYG{l+m+mi}{1}\PYG{o}{/}\PYG{l+m+mi}{3}\PYG{p}{;}
    \PYG{k}{grid\PYGZhy{}column}\PYG{p}{:} \PYG{l+m+mi}{4}\PYG{o}{/}\PYG{l+m+mi}{6}\PYG{p}{;}
    \PYG{k}{background\PYGZhy{}color}\PYG{p}{:} \PYG{n+nb}{rgb}\PYG{p}{(}\PYG{l+m+mi}{200}\PYG{p}{,} \PYG{l+m+mi}{200}\PYG{p}{,} \PYG{l+m+mi}{200}\PYG{p}{)}\PYG{p}{;}
\PYG{p}{\PYGZcb{}}
\PYG{p}{\PYGZsh{}}\PYG{n+nn}{b}\PYG{p}{\PYGZob{}}
    \PYG{k}{grid\PYGZhy{}row}\PYG{p}{:}  \PYG{l+m+mi}{3}\PYG{o}{/}\PYG{l+m+mi}{4}\PYG{p}{;}
    \PYG{k}{grid\PYGZhy{}column} \PYG{p}{:} \PYG{l+m+mi}{2}\PYG{o}{/}\PYG{l+m+mi}{5}   \PYG{p}{;}
    \PYG{k}{background\PYGZhy{}color}\PYG{p}{:} \PYG{n+nb}{rgb}\PYG{p}{(}\PYG{l+m+mi}{210}\PYG{p}{,} \PYG{l+m+mi}{240}\PYG{p}{,} \PYG{l+m+mi}{200}\PYG{p}{)}\PYG{p}{;}
\PYG{p}{\PYGZcb{}}
\PYG{p}{\PYGZsh{}}\PYG{n+nn}{c}\PYG{p}{\PYGZob{}}
    \PYG{k}{grid\PYGZhy{}row}\PYG{p}{:}  \PYG{l+m+mi}{1}\PYG{o}{/}\PYG{l+m+mi}{6}\PYG{p}{;}
    \PYG{k}{grid\PYGZhy{}column} \PYG{p}{:} \PYG{l+m+mi}{1}  \PYG{p}{;}
    \PYG{k}{background\PYGZhy{}color}\PYG{p}{:} \PYG{n+nb}{rgb}\PYG{p}{(}\PYG{l+m+mi}{210}\PYG{p}{,} \PYG{l+m+mi}{220}\PYG{p}{,} \PYG{l+m+mi}{200}\PYG{p}{)}\PYG{p}{;}
\PYG{p}{\PYGZcb{}}
\PYG{p}{\PYGZsh{}}\PYG{n+nn}{d}\PYG{p}{\PYGZob{}}
    \PYG{k}{grid\PYGZhy{}row}\PYG{p}{:} \PYG{l+m+mi}{3}\PYG{o}{/}\PYG{l+m+mi}{6} \PYG{p}{;}
    \PYG{k}{grid\PYGZhy{}column} \PYG{p}{:}\PYG{l+m+mi}{5}   \PYG{p}{;}
    \PYG{k}{background\PYGZhy{}color}\PYG{p}{:} \PYG{n+nb}{rgb}\PYG{p}{(}\PYG{l+m+mi}{210}\PYG{p}{,} \PYG{l+m+mi}{230}\PYG{p}{,} \PYG{l+m+mi}{230}\PYG{p}{)}\PYG{p}{;}
\PYG{p}{\PYGZcb{}}
\PYG{p}{\PYGZsh{}}\PYG{n+nn}{e}\PYG{p}{\PYGZob{}}

    \PYG{k}{grid\PYGZhy{}row}\PYG{p}{:}  \PYG{l+m+mi}{1}\PYG{o}{/}\PYG{l+m+mi}{3}\PYG{p}{;}
    \PYG{k}{grid\PYGZhy{}column} \PYG{p}{:} \PYG{l+m+mi}{2}\PYG{o}{/}\PYG{l+m+mi}{4}  \PYG{p}{;}
    \PYG{k}{background\PYGZhy{}color}\PYG{p}{:} \PYG{n+nb}{rgb}\PYG{p}{(}\PYG{l+m+mi}{210}\PYG{p}{,} \PYG{l+m+mi}{240}\PYG{p}{,} \PYG{l+m+mi}{240}\PYG{p}{)}\PYG{p}{;}
\PYG{p}{\PYGZcb{}}
\PYG{p}{\PYGZsh{}}\PYG{n+nn}{f}\PYG{p}{\PYGZob{}}
    \PYG{k}{grid\PYGZhy{}row}\PYG{p}{:}  \PYG{l+m+mi}{4}\PYG{o}{/}\PYG{l+m+mi}{6}\PYG{p}{;}
    \PYG{k}{grid\PYGZhy{}column} \PYG{p}{:} \PYG{l+m+mi}{2}\PYG{o}{/}\PYG{l+m+mi}{5}  \PYG{p}{;}
    \PYG{k}{background\PYGZhy{}color}\PYG{p}{:} \PYG{n+nb}{rgb}\PYG{p}{(}\PYG{l+m+mi}{240}\PYG{p}{,} \PYG{l+m+mi}{240}\PYG{p}{,} \PYG{l+m+mi}{240}\PYG{p}{)}\PYG{p}{;}
\PYG{p}{\PYGZcb{}}
\end{sphinxVerbatim}


\section{Media queries}
\label{\detokenize{tema3:media-queries}}
Las \sphinxstyleemphasis{media queries} (algo así como «consultas sobre el tipo de medio en el que se va a mostrar/procesar el HTML») forman parte de CSS 3, por lo que solo deben utilizarse en navegadores relativamente modernos.

Las media queries permiten hacer diversas comprobaciones. Si se cumplen dichas comprobaciones se ejecutará un CSS u otro. Por ejemplo, supongamos que queremos tener dos estructuras diferentes de página en función de si se va a mostrar el HTML en pantalla o en papel.

Partamos del siguiente HTML:

\begin{sphinxVerbatim}[commandchars=\\\{\}]
\PYG{p}{\PYGZlt{}}\PYG{n+nt}{div} \PYG{n+na}{id}\PYG{o}{=}\PYG{l+s}{\PYGZdq{}caja1\PYGZdq{}}\PYG{p}{\PYGZgt{}}
    Caja 1
\PYG{p}{\PYGZlt{}}\PYG{p}{/}\PYG{n+nt}{div}\PYG{p}{\PYGZgt{}}

\PYG{p}{\PYGZlt{}}\PYG{n+nt}{div} \PYG{n+na}{id}\PYG{o}{=}\PYG{l+s}{\PYGZdq{}caja2\PYGZdq{}}\PYG{p}{\PYGZgt{}}
    Caja 2
\PYG{p}{\PYGZlt{}}\PYG{p}{/}\PYG{n+nt}{div}\PYG{p}{\PYGZgt{}}

\PYG{p}{\PYGZlt{}}\PYG{n+nt}{div} \PYG{n+na}{id}\PYG{o}{=}\PYG{l+s}{\PYGZdq{}caja3\PYGZdq{}}\PYG{p}{\PYGZgt{}}
    Caja 3
\PYG{p}{\PYGZlt{}}\PYG{p}{/}\PYG{n+nt}{div}\PYG{p}{\PYGZgt{}}
\end{sphinxVerbatim}

Y ahora supongamos que cuando se muestra el HTML en pantalla queremos que tenga un color de fondo, pero que si se va a imprimir no tengan ningún fondo (para ahorrar tinta, por ejemplo). Sabiendo que existen dos tipos de medios llamados \sphinxcode{screen} y \sphinxcode{print} podemos usar un CSS como este:

\begin{sphinxVerbatim}[commandchars=\\\{\}]
\PYG{p}{@}\PYG{k}{media} \PYG{n+nt}{screen} \PYG{p}{\PYGZob{}}
    \PYG{n+nt}{div}\PYG{p}{\PYGZob{}}
        \PYG{k}{font\PYGZhy{}size}\PYG{p}{:} \PYG{k+kc}{xx\PYGZhy{}large}\PYG{p}{;}
        \PYG{k}{border}\PYG{p}{:} \PYG{k+kc}{double} \PYG{k+kc}{black} \PYG{l+m+mi}{1}\PYG{k+kt}{px}\PYG{p}{;}
        \PYG{k}{background\PYGZhy{}color}\PYG{p}{:} \PYG{k+kc}{grey}\PYG{p}{;}
        \PYG{k}{margin}\PYG{p}{:} \PYG{l+m+mi}{30}\PYG{k+kt}{px}\PYG{p}{;}
    \PYG{p}{\PYGZcb{}}
\PYG{p}{\PYGZcb{}}
\PYG{p}{@}\PYG{k}{media} \PYG{n+nt}{print} \PYG{p}{\PYGZob{}}
    \PYG{n+nt}{div}\PYG{p}{\PYGZob{}}
        \PYG{k}{font\PYGZhy{}size}\PYG{p}{:} \PYG{k+kc}{small}\PYG{p}{;}
        \PYG{k}{margin}\PYG{p}{:} \PYG{l+m+mi}{15}\PYG{k+kt}{px}\PYG{p}{;}
    \PYG{p}{\PYGZcb{}}
\PYG{p}{\PYGZcb{}}
\end{sphinxVerbatim}

Este CSS puede hacer dos cosas distintas:
\begin{itemize}
\item {} 
Cuando el HTML se muestra en pantalla las cajas tendrán todas un borde, un fondo, un tipo de letra grande y mucho margen.

\item {} 
Sin embargo, cuando se va a imprimir no hay fondos, el margen es mucho más pequeño y el tipo de letra también.

\end{itemize}

A continuación se muestra una captura de lo que muestra el navegador:

\begin{figure}[htbp]
\centering
\capstart

\noindent\sphinxincludegraphics{{media_query_screen}.png}
\caption{HTML mostrado en pantalla}\label{\detokenize{tema3:id12}}\end{figure}

Y también se muestra una captura de lo que muestra el navegador cuando vamos a imprimir:

\begin{figure}[htbp]
\centering
\capstart

\noindent\sphinxincludegraphics{{media_query_print}.png}
\caption{HTML para imprimir}\label{\detokenize{tema3:id13}}\end{figure}

Uno de los usos más comunes de las \sphinxstyleemphasis{media queries} es la comprobación de la resolución en la que se está visualizando el HTML y en función de ello mostrar distintas estructuras al usuario. Por ejemplo, supongamos que deseamos mostrar nuestra página anterior de dos formas en pantalla.
\begin{itemize}
\item {} 
Cuando la resolución sea de 800 px o más haremos que \sphinxcode{caja1}  y \sphinxcode{caja2} estén una al lado de la otra ocupando cada una la mitad de la pantalla aproximadamente.

\item {} 
Cuando la resolución sea de 799px o menos \sphinxcode{caja1}, \sphinxcode{caja2}  y \sphinxcode{caja3} se mostrarán una encima de otra pero con un margen entre ellas de 40px.

\end{itemize}

Para conseguir esto hay predicados de utilidad que podemos combinar con los que acabamos de ver para conseguir lo que deseamos. Dos de los más útiles son \sphinxcode{min-width} y \sphinxcode{max-width}. Veamos como se usan para conseguir lo que nos piden:

\begin{sphinxVerbatim}[commandchars=\\\{\}]
\PYG{c}{/* Si estamos en una pantalla y la anchura mínima que podemos}
\PYG{c}{ * usar es 800px...*/}
\PYG{p}{@}\PYG{k}{media} \PYG{n+nt}{screen} \PYG{n+nt}{and} \PYG{o}{(}\PYG{n+nt}{min\PYGZhy{}width}\PYG{p}{:}\PYG{n+nd}{800px}\PYG{o}{)}\PYG{p}{\PYGZob{}}
    \PYG{n+nt}{div}\PYG{p}{\PYGZob{}}
        \PYG{k}{border}\PYG{p}{:}\PYG{k+kc}{solid} \PYG{k+kc}{black} \PYG{l+m+mi}{1}\PYG{k+kt}{px}\PYG{p}{;}
    \PYG{p}{\PYGZcb{}}
    \PYG{c}{/*... entonces dividir caja 1 y caja 2 en dos mitades*/}
    \PYG{p}{\PYGZsh{}}\PYG{n+nn}{caja1}\PYG{p}{\PYGZob{}}
        \PYG{k}{float}\PYG{p}{:}\PYG{k+kc}{left}\PYG{p}{;}
        \PYG{k}{width}\PYG{p}{:}\PYG{l+m+mi}{48}\PYG{k+kt}{\PYGZpc{}}\PYG{p}{;}
    \PYG{p}{\PYGZcb{}}
    \PYG{p}{\PYGZsh{}}\PYG{n+nn}{caja2}\PYG{p}{\PYGZob{}}
        \PYG{k}{float}\PYG{p}{:}\PYG{k+kc}{right}\PYG{p}{;}
        \PYG{k}{width}\PYG{p}{:}\PYG{l+m+mi}{48}\PYG{k+kt}{\PYGZpc{}}\PYG{p}{;}
    \PYG{p}{\PYGZcb{}}
    \PYG{p}{\PYGZsh{}}\PYG{n+nn}{caja3}\PYG{p}{\PYGZob{}}
        \PYG{k}{clear}\PYG{p}{:}\PYG{k+kc}{both}\PYG{p}{;}
    \PYG{p}{\PYGZcb{}}
\PYG{p}{\PYGZcb{}}
\PYG{c}{/* Sin embargo, si estamos en una pantalla y como máximo}
\PYG{c}{ * tenemos 799px (es decir, una pantalla más bien estrecha)...*/}
\PYG{p}{@}\PYG{k}{media} \PYG{n+nt}{screen} \PYG{n+nt}{and} \PYG{o}{(}\PYG{n+nt}{max\PYGZhy{}width}\PYG{p}{:}\PYG{n+nd}{799px}\PYG{o}{)}\PYG{p}{\PYGZob{}}
    \PYG{n+nt}{div}\PYG{p}{\PYGZob{}}
        \PYG{k}{border}\PYG{p}{:}\PYG{k+kc}{solid} \PYG{k+kc}{black} \PYG{l+m+mi}{1}\PYG{k+kt}{px}\PYG{p}{;}
        \PYG{k}{margin\PYGZhy{}top}\PYG{p}{:} \PYG{l+m+mi}{40}\PYG{k+kt}{px}\PYG{p}{;}
        \PYG{k}{text\PYGZhy{}align}\PYG{p}{:} \PYG{k+kc}{center}\PYG{p}{;}
    \PYG{p}{\PYGZcb{}}
\PYG{p}{\PYGZcb{}}
\end{sphinxVerbatim}

El resultado de este CSS en una pantalla grande es:

\begin{figure}[htbp]
\centering
\capstart

\noindent\sphinxincludegraphics{{mq_anchura_grande}.png}
\caption{Página para una resolución grande}\label{\detokenize{tema3:id14}}\end{figure}

Sin embargo, en una pantalla pequeña (se puede cambiar el tamaño de la ventana del navegador para simular el ejemplo):

\begin{figure}[htbp]
\centering
\capstart

\noindent\sphinxincludegraphics{{mq_anchura_pequena}.png}
\caption{Página para una resolución pequeña}\label{\detokenize{tema3:id15}}\end{figure}

Hay diversas cosas que podemos comprobar:
\begin{itemize}
\item {} 
\sphinxcode{min-width: 100px {}`{}` o {}`{}`max-width:900px} para ver las anchuras mínimas o máximas que nos ofrece un dispositivo.

\item {} 
\sphinxcode{@media screen} , \sphinxcode{@media print} o \sphinxcode{@media: handheld} para comprobar si el HTML se va a mostrar en una pantalla, se va a imprimir o se muestra un dispositivo portátil como móvil o tablet. Hay otros \sphinxcode{@media} como \sphinxcode{@media braille} o \sphinxcode{@media tv}, pero se usan menos.

\item {} 
\sphinxcode{orientation:  portrait} u \sphinxcode{orientation: landscape} para saber si la pantalla está en horizontal o en vertical.

\end{itemize}


\section{Gestión de espacios}
\label{\detokenize{tema3:gestion-de-espacios}}
En CSS se puede controlar el espacio interno y externo por medio de las propiedades \sphinxcode{padding-} y \sphinxcode{margin-} pudiendo usar \sphinxcode{margin-top} o \sphinxcode{padding-left}. Las cuatro posiciones son \sphinxcode{top}, \sphinxcode{bottom}, \sphinxcode{left} y \sphinxcode{right}


\section{Colores}
\label{\detokenize{tema3:colores}}
Los colores en CSS se pueden especificar de varias maneras:
\begin{itemize}
\item {} 
Por nombre: \sphinxcode{red}, \sphinxcode{yellow}, \sphinxcode{green}

\item {} 
Mediante \sphinxcode{rgb(rojo, verde, azul}, donde entre comas se pone la cantidad de cada color de 0 a 255. Así, \sphinxcode{rgb(0,0,0)} es negro y \sphinxcode{rgb(255,255,255)} es blanco.

\item {} 
Se puede usar directamente la nomenclatura hexadecimal \#ffffff. Donde cada dos letras se indica un número hexadecimal de 00 a ff, que indica respectivamente la cantidad de color rojo, verde o azul.

\item {} 
Desde hace poco se pueden indicar también con \sphinxcode{hsl(num, num, num)}

\end{itemize}

Se pueden encontrar en Internet listas de colores denominados «seguros» (buscando por «web safe colors») que indican nombres de color que se ven igual en los distintos navegadores.


\section{Tipografías}
\label{\detokenize{tema3:tipografias}}
En tipografía se habla de dos términos distintos: el «typeface» y la «font».
\begin{itemize}
\item {} 
Hay tipos «Serif», que llevan «rabito».

\item {} 
Hay tipos «Sans-serif» que no lo llevan

\item {} 
Hay tipos monoespaciados

\end{itemize}

Lo más relevante, es que cuando usamos \sphinxcode{font-family: "Arial";}, el navegador puede decidir poner otro tipo de letra de la misma familia.

Se pueden indicar varios tipos de letra por orden de preferencia.

Google Fonts permite el «embebido» de fuentes de manera muy segura.


\section{Alineación del texto}
\label{\detokenize{tema3:alineacion-del-texto}}
Se puede usar la propiedad \sphinxcode{text-align: left} para modificar la alineación del texto, usando \sphinxcode{left}, \sphinxcode{center}, \sphinxcode{right} o \sphinxcode{justify}


\section{Decoración del texto}
\label{\detokenize{tema3:decoracion-del-texto}}
Se pueden usar otras propiedades para cambiar
el aspecto del texto como estas:
\begin{itemize}
\item {} 
\sphinxcode{text-decoration: underline}

\item {} 
\sphinxcode{text-decoration: overline}

\item {} 
\sphinxcode{text-decoration: line-through}

\end{itemize}


\section{Medidas}
\label{\detokenize{tema3:medidas}}
Normalmente, lo más seguro es usar medidas en forma de porcentajes, pero hay otras
\begin{itemize}
\item {} 
\sphinxcode{margin: 1cm}

\item {} 
\sphinxcode{margin: 1in}: esta y la anterior son más útiles cuando creamos hojas de estilo enfocadas a que la página quede bien cuando se imprima.

\item {} 
\sphinxcode{margin: 1px}: muy dependiente de la resolución

\item {} 
\sphinxcode{margin: 1\%}: es la más apropiada al modificar elementos div en pantalla.

\item {} 
\sphinxcode{margin: 1em}: equivale aproximadamente a la anchura de una letra «m».

\end{itemize}


\section{Selectores}
\label{\detokenize{tema3:selectores}}
Explica qué hacen los siguientes selectores y crea un ejemplo HTML donde se pueda ver que realmente funcionan como esperas
\begin{itemize}
\item {} 
p\#destacado

\item {} 
p.destacado

\item {} 
p.destacado, span\#id1

\item {} 
p.destacado \textgreater{} li.elemento\_enumeracion

\item {} 
p.destacado \textgreater{} .elemento\_numeracion

\item {} 
.destacado \textgreater{} \#id1

\end{itemize}

Supongamos que tenemos un archivo HTML como este:

\begin{sphinxVerbatim}[commandchars=\\\{\}]
\PYG{p}{\PYGZlt{}}\PYG{n+nt}{p}\PYG{p}{\PYGZgt{}}
        Párrafo sin class ni ide
\PYG{p}{\PYGZlt{}}\PYG{p}{/}\PYG{n+nt}{p}\PYG{p}{\PYGZgt{}}
\PYG{p}{\PYGZlt{}}\PYG{n+nt}{p} \PYG{n+na}{class}\PYG{o}{=}\PYG{l+s}{\PYGZdq{}cita\PYGZdq{}} \PYG{n+na}{id}\PYG{o}{=}\PYG{l+s}{\PYGZdq{}destacado\PYGZdq{}}\PYG{p}{\PYGZgt{}}
        Párrafo con el class \PYGZsq{}cita\PYGZsq{}
        y el id \PYGZsq{}p\PYGZus{}destacado\PYGZsq{}
\PYG{p}{\PYGZlt{}}\PYG{p}{/}\PYG{n+nt}{p}\PYG{p}{\PYGZgt{}}
\PYG{p}{\PYGZlt{}}\PYG{n+nt}{p} \PYG{n+na}{class}\PYG{o}{=}\PYG{l+s}{\PYGZdq{}p\PYGZus{}destacado\PYGZdq{}}\PYG{p}{\PYGZgt{}}Párrafo con el class
        destacado que no
        contiene nada
\PYG{p}{\PYGZlt{}}\PYG{p}{/}\PYG{n+nt}{p}\PYG{p}{\PYGZgt{}}
\PYG{p}{\PYGZlt{}}\PYG{n+nt}{p} \PYG{n+na}{class}\PYG{o}{=}\PYG{l+s}{\PYGZdq{}p\PYGZus{}destacado\PYGZdq{}}\PYG{p}{\PYGZgt{}}
        Párrafo con el class destacado.
        \PYG{p}{\PYGZlt{}}\PYG{n+nt}{span} \PYG{n+na}{id}\PYG{o}{=}\PYG{l+s}{\PYGZdq{}id1\PYGZdq{}}\PYG{p}{\PYGZgt{}}
                Este texto va dentro de
                un span con el
                id id1
        \PYG{p}{\PYGZlt{}}\PYG{p}{/}\PYG{n+nt}{span}\PYG{p}{\PYGZgt{}}
\PYG{p}{\PYGZlt{}}\PYG{p}{/}\PYG{n+nt}{p}\PYG{p}{\PYGZgt{}}
\PYG{p}{\PYGZlt{}}\PYG{n+nt}{p} \PYG{n+na}{class}\PYG{o}{=}\PYG{l+s}{\PYGZdq{}destacado\PYGZdq{}}\PYG{p}{\PYGZgt{}}
        \PYG{p}{\PYGZlt{}}\PYG{n+nt}{ol}\PYG{p}{\PYGZgt{}}
                \PYG{p}{\PYGZlt{}}\PYG{n+nt}{li} \PYG{n+na}{class}\PYG{o}{=}\PYG{l+s}{\PYGZdq{}elemento\PYGZus{}numeracion\PYGZdq{}}\PYG{p}{\PYGZgt{}}
                        Esto es un li
                \PYG{p}{\PYGZlt{}}\PYG{p}{/}\PYG{n+nt}{li}\PYG{p}{\PYGZgt{}}
                \PYG{p}{\PYGZlt{}}\PYG{n+nt}{li} \PYG{n+na}{class}\PYG{o}{=}\PYG{l+s}{\PYGZdq{}elemento\PYGZus{}numeracion\PYGZdq{}}\PYG{p}{\PYGZgt{}}
                        Esto es otro li
                \PYG{p}{\PYGZlt{}}\PYG{p}{/}\PYG{n+nt}{li}\PYG{p}{\PYGZgt{}}
        \PYG{p}{\PYGZlt{}}\PYG{p}{/}\PYG{n+nt}{ol}\PYG{p}{\PYGZgt{}}
\PYG{p}{\PYGZlt{}}\PYG{p}{/}\PYG{n+nt}{p}\PYG{p}{\PYGZgt{}}
\PYG{p}{\PYGZlt{}}\PYG{n+nt}{p} \PYG{n+na}{class}\PYG{o}{=}\PYG{l+s}{\PYGZdq{}destacado\PYGZdq{}}\PYG{p}{\PYGZgt{}}
        Este párrafo tiene el class
        destacado y en él enumeramos
        cosas como
        \PYG{p}{\PYGZlt{}}\PYG{n+nt}{span} \PYG{n+na}{class}\PYG{o}{=}\PYG{l+s}{\PYGZdq{}elemento\PYGZus{}numeracion\PYGZdq{}}\PYG{p}{\PYGZgt{}}
                A
        \PYG{p}{\PYGZlt{}}\PYG{p}{/}\PYG{n+nt}{span}\PYG{p}{\PYGZgt{}},
        \PYG{p}{\PYGZlt{}}\PYG{n+nt}{span} \PYG{n+na}{class}\PYG{o}{=}\PYG{l+s}{\PYGZdq{}elemento\PYGZus{}numeracion\PYGZdq{}}\PYG{p}{\PYGZgt{}}
                B
        \PYG{p}{\PYGZlt{}}\PYG{p}{/}\PYG{n+nt}{span}\PYG{p}{\PYGZgt{}} o también
        \PYG{p}{\PYGZlt{}}\PYG{n+nt}{span} \PYG{n+na}{class}\PYG{o}{=}\PYG{l+s}{\PYGZdq{}elemento\PYGZus{}numeracion\PYGZdq{}}\PYG{p}{\PYGZgt{}}
                C
        \PYG{p}{\PYGZlt{}}\PYG{p}{/}\PYG{n+nt}{span}\PYG{p}{\PYGZgt{}}
\PYG{p}{\PYGZlt{}}\PYG{p}{/}\PYG{n+nt}{p}\PYG{p}{\PYGZgt{}}

\PYG{p}{\PYGZlt{}}\PYG{n+nt}{div} \PYG{n+na}{class}\PYG{o}{=}\PYG{l+s}{\PYGZdq{}destacado\PYGZdq{}}\PYG{p}{\PYGZgt{}}
        Aquí hay un
        \PYG{p}{\PYGZlt{}}\PYG{n+nt}{span} \PYG{n+na}{id}\PYG{o}{=}\PYG{l+s}{\PYGZdq{}id1\PYGZdq{}}\PYG{p}{\PYGZgt{}}
                span con el id id1
        \PYG{p}{\PYGZlt{}}\PYG{p}{/}\PYG{n+nt}{span}\PYG{p}{\PYGZgt{}}
\PYG{p}{\PYGZlt{}}\PYG{p}{/}\PYG{n+nt}{div}\PYG{p}{\PYGZgt{}}
\end{sphinxVerbatim}


\subsection{Solución \sphinxstyleliteralintitle{p\#destacado}}
\label{\detokenize{tema3:solucion-p-destacado}}
Si tenemos un estilo como este:

\begin{sphinxVerbatim}[commandchars=\\\{\}]
\PYG{n+nt}{p}\PYG{p}{\PYGZsh{}}\PYG{n+nn}{destacado}\PYG{p}{\PYGZob{}}
        \PYG{k}{border}\PYG{p}{:}\PYG{k+kc}{solid} \PYG{k+kc}{black} \PYG{l+m+mi}{1}\PYG{k+kt}{px}\PYG{p}{;}
\PYG{p}{\PYGZcb{}}
\end{sphinxVerbatim}

Lo que ocurrirá es que se pondrá un borde solo al párrafo cuyo \sphinxcode{id} sea \sphinxcode{destacado}

\begin{figure}[htbp]
\centering
\capstart

\noindent\sphinxincludegraphics{{selectores1}.png}
\caption{Resultado}\label{\detokenize{tema3:id16}}\end{figure}


\subsection{Solución \sphinxstyleliteralintitle{p.destacado}}
\label{\detokenize{tema3:id5}}
Los cambios se aplican a todos los párrafos con el \sphinxcode{class} \sphinxstyleemphasis{destacado}

\begin{figure}[htbp]
\centering
\capstart

\noindent\sphinxincludegraphics{{selectores2}.png}
\caption{Resultado}\label{\detokenize{tema3:id17}}\end{figure}


\subsection{Solución \sphinxstyleliteralintitle{p.destacado, span\#id1}}
\label{\detokenize{tema3:solucion-p-destacado-span-id1}}
Los cambios se aplican a todos los párrafos con el \sphinxcode{class} \sphinxstyleemphasis{destacado} y también al \sphinxcode{span} cuyo \sphinxcode{id} sea \sphinxcode{id1}

\begin{sphinxadmonition}{warning}{Advertencia:}
Obsérvese que en el HTML hay \sphinxstylestrong{dos elementos con el mismo ID}. No se debe hacer esto, ya que corremos el riesgo de que todo se vea mal.
\end{sphinxadmonition}

\begin{figure}[htbp]
\centering
\capstart

\noindent\sphinxincludegraphics{{selectores3}.png}
\caption{Resultado}\label{\detokenize{tema3:id18}}\end{figure}


\subsection{Solución \sphinxstyleliteralintitle{p.destacado \textgreater{} li.elemento\_enumeracion}}
\label{\detokenize{tema3:solucion-p-destacado-li-elemento-enumeracion}}
Los cambios solo se aplican a \sphinxstylestrong{los li cuyo class sea elemento\_numeración y que además sean hijos de un p cuyo class sea destacado}

¿Por qué los cambios no afectan a ninguno?


\subsection{Solución \sphinxstyleliteralintitle{p.destacado \textgreater{} .elemento\_numeracion}}
\label{\detokenize{tema3:solucion-p-destacado-elemento-numeracion}}
Ahora sí veremos que algo cambia, en concreto los ultimos \sphinxcode{\textless{}span\textgreater{}} que llevan el \sphinxcode{class=elemento\_numeracion}, ya que \sphinxstyleemphasis{sí son hijos directos} de un elemento que lleva un \sphinxcode{class=destacado}.


\subsection{Solución \sphinxstyleliteralintitle{.destacado \textgreater{} \#id1}}
\label{\detokenize{tema3:solucion-destacado-id1}}
Ahora el resultado es este

\begin{figure}[htbp]
\centering
\capstart

\noindent\sphinxincludegraphics{{selectores4}.png}
\caption{Resultado}\label{\detokenize{tema3:id19}}\end{figure}

¿Por qué ahora sí funciona?


\section{Bootstrap}
\label{\detokenize{tema3:bootstrap}}
Bootstrap define una estructura básica de clases CSS para facilitar el desarrollo web. En concreto CSS consigue que crear páginas que se vean igual en dispositivos muy distintos sea algo relativamente sencillo.


\subsection{Estructura básica}
\label{\detokenize{tema3:estructura-basica}}
El siguiente HTML define lo mínimo que se necesita para crear una página con Bootstrap. Dentro de \sphinxcode{\textless{}body\textgreater{}} podremos poner lo que necesitemos y el \sphinxstyleemphasis{framework} colocará todo automáticamente y le aplicará cierto estilismo.

\begin{sphinxVerbatim}[commandchars=\\\{\}]
\PYG{c+cp}{\PYGZlt{}!DOCTYPE html\PYGZgt{}}
\PYG{p}{\PYGZlt{}}\PYG{n+nt}{html} \PYG{n+na}{lang}\PYG{o}{=}\PYG{l+s}{\PYGZdq{}en\PYGZdq{}}\PYG{p}{\PYGZgt{}}
  \PYG{p}{\PYGZlt{}}\PYG{n+nt}{head}\PYG{p}{\PYGZgt{}}
        \PYG{p}{\PYGZlt{}}\PYG{n+nt}{meta} \PYG{n+na}{charset}\PYG{o}{=}\PYG{l+s}{\PYGZdq{}utf\PYGZhy{}8\PYGZdq{}}\PYG{p}{\PYGZgt{}}
        \PYG{p}{\PYGZlt{}}\PYG{n+nt}{meta} \PYG{n+na}{http\PYGZhy{}equiv}\PYG{o}{=}\PYG{l+s}{\PYGZdq{}X\PYGZhy{}UA\PYGZhy{}Compatible\PYGZdq{}} \PYG{n+na}{content}\PYG{o}{=}\PYG{l+s}{\PYGZdq{}IE=edge\PYGZdq{}}\PYG{p}{\PYGZgt{}}
        \PYG{p}{\PYGZlt{}}\PYG{n+nt}{meta} \PYG{n+na}{name}\PYG{o}{=}\PYG{l+s}{\PYGZdq{}viewport\PYGZdq{}} \PYG{n+na}{content}\PYG{o}{=}\PYG{l+s}{\PYGZdq{}width=device\PYGZhy{}width, initial\PYGZhy{}scale=1\PYGZdq{}}\PYG{p}{\PYGZgt{}}
        \PYG{p}{\PYGZlt{}}\PYG{n+nt}{title}\PYG{p}{\PYGZgt{}}Plantilla bootstrap\PYG{p}{\PYGZlt{}}\PYG{p}{/}\PYG{n+nt}{title}\PYG{p}{\PYGZgt{}}

        \PYG{p}{\PYGZlt{}}\PYG{n+nt}{link} \PYG{n+na}{href}\PYG{o}{=}\PYG{l+s}{\PYGZdq{}css/bootstrap.min.css\PYGZdq{}} \PYG{n+na}{rel}\PYG{o}{=}\PYG{l+s}{\PYGZdq{}stylesheet\PYGZdq{}}\PYG{p}{\PYGZgt{}}


        \PYG{c}{\PYGZlt{}!\PYGZhy{}\PYGZhy{}}\PYG{c}{[if lt IE 9]\PYGZgt{}}
\PYG{c}{          \PYGZlt{}script src=\PYGZdq{}js/html5shiv.min.js\PYGZdq{}\PYGZgt{}\PYGZlt{}/script\PYGZgt{}}
\PYG{c}{          \PYGZlt{}script src=\PYGZdq{}js/respond.min.js\PYGZdq{}\PYGZgt{}\PYGZlt{}/script\PYGZgt{}}
\PYG{c}{        \PYGZlt{}![endif]}\PYG{c}{\PYGZhy{}\PYGZhy{}\PYGZgt{}}
  \PYG{p}{\PYGZlt{}}\PYG{p}{/}\PYG{n+nt}{head}\PYG{p}{\PYGZgt{}}
  \PYG{p}{\PYGZlt{}}\PYG{n+nt}{body}\PYG{p}{\PYGZgt{}}
        \PYG{p}{\PYGZlt{}}\PYG{n+nt}{h1}\PYG{p}{\PYGZgt{}}Página con Bootstrap\PYG{p}{\PYGZlt{}}\PYG{p}{/}\PYG{n+nt}{h1}\PYG{p}{\PYGZgt{}}
                \PYG{p}{\PYGZlt{}}\PYG{n+nt}{div} \PYG{n+na}{class}\PYG{o}{=}\PYG{l+s}{\PYGZdq{}container\PYGZdq{}}\PYG{p}{\PYGZgt{}}
                \PYG{p}{\PYGZlt{}}\PYG{p}{/}\PYG{n+nt}{div}\PYG{p}{\PYGZgt{}}
        \PYG{p}{\PYGZlt{}}\PYG{n+nt}{script} \PYG{n+na}{src}\PYG{o}{=}\PYG{l+s}{\PYGZdq{}js/jquery.min.js\PYGZdq{}}\PYG{p}{\PYGZgt{}}\PYG{p}{\PYGZlt{}}\PYG{p}{/}\PYG{n+nt}{script}\PYG{p}{\PYGZgt{}}

        \PYG{p}{\PYGZlt{}}\PYG{n+nt}{script} \PYG{n+na}{src}\PYG{o}{=}\PYG{l+s}{\PYGZdq{}js/bootstrap.min.js\PYGZdq{}}\PYG{p}{\PYGZgt{}}\PYG{p}{\PYGZlt{}}\PYG{p}{/}\PYG{n+nt}{script}\PYG{p}{\PYGZgt{}}
  \PYG{p}{\PYGZlt{}}\PYG{p}{/}\PYG{n+nt}{body}\PYG{p}{\PYGZgt{}}
\PYG{p}{\PYGZlt{}}\PYG{p}{/}\PYG{n+nt}{html}\PYG{p}{\PYGZgt{}}
\end{sphinxVerbatim}

Lo único que se debe asumir es que debe existir un \sphinxcode{\textless{}div\textgreater{}} cuyo \sphinxcode{class} sea \sphinxcode{container}


\subsection{Rejilla o \sphinxstyleemphasis{grid}}
\label{\detokenize{tema3:rejilla-o-grid}}
Bootstrap asume que cualquier pantalla tiene una anchura básica de 12 columnas. Podremos crear una fila de elementos y hacer que cada una de ellas ocupe cierta proporción de esas columnas.

Por ejemplo, si deseamos que una fila de contenidos tenga una columna que ocupe la mitad de esas 12 columnas (6) y dos columnas que ocupen la mitad restante, podremos hacer lo siguiente.

\begin{sphinxVerbatim}[commandchars=\\\{\}]
    \PYG{p}{\PYGZlt{}}\PYG{n+nt}{div} \PYG{n+na}{class}\PYG{o}{=}\PYG{l+s}{\PYGZdq{}container\PYGZdq{}}\PYG{p}{\PYGZgt{}}
    \PYG{p}{\PYGZlt{}}\PYG{n+nt}{div} \PYG{n+na}{class}\PYG{o}{=}\PYG{l+s}{\PYGZdq{}row\PYGZdq{}}\PYG{p}{\PYGZgt{}}
        \PYG{p}{\PYGZlt{}}\PYG{n+nt}{div} \PYG{n+na}{class}\PYG{o}{=}\PYG{l+s}{\PYGZdq{}col\PYGZhy{}md\PYGZhy{}6\PYGZdq{}}\PYG{p}{\PYGZgt{}}
            Mitad del contenedor
        \PYG{p}{\PYGZlt{}}\PYG{p}{/}\PYG{n+nt}{div}\PYG{p}{\PYGZgt{}}
        \PYG{p}{\PYGZlt{}}\PYG{n+nt}{div} \PYG{n+na}{class}\PYG{o}{=}\PYG{l+s}{\PYGZdq{}col\PYGZhy{}md\PYGZhy{}3\PYGZdq{}}\PYG{p}{\PYGZgt{}}
            Esto ocupa un cuarto
        \PYG{p}{\PYGZlt{}}\PYG{p}{/}\PYG{n+nt}{div}\PYG{p}{\PYGZgt{}}
        \PYG{p}{\PYGZlt{}}\PYG{n+nt}{div} \PYG{n+na}{class}\PYG{o}{=}\PYG{l+s}{\PYGZdq{}col\PYGZhy{}md\PYGZhy{}3\PYGZdq{}}\PYG{p}{\PYGZgt{}}
            Esto ocupa otro  cuarto
        \PYG{p}{\PYGZlt{}}\PYG{p}{/}\PYG{n+nt}{div}\PYG{p}{\PYGZgt{}}
    \PYG{p}{\PYGZlt{}}\PYG{p}{/}\PYG{n+nt}{div}\PYG{p}{\PYGZgt{}}
\PYG{p}{\PYGZlt{}}\PYG{p}{/}\PYG{n+nt}{div}\PYG{p}{\PYGZgt{}}
\end{sphinxVerbatim}

\begin{figure}[htbp]
\centering
\capstart

\noindent\sphinxincludegraphics{{bs1}.png}
\caption{Ejemplo del grid}\label{\detokenize{tema3:id20}}\end{figure}

En realidad Bootstrap define muchos tipos de columna dependiendo del tipo de dispositivo al que nos hayamos enfocado más:
\begin{itemize}
\item {} 
col-xs-3: ocupa 3 de las doce columnas de un dispositivo que se ha dividido en 12 pero tiene una anchura «muy pequeña/extrasmall» (menos de 768)

\item {} 
col-sm-3: ocupa 3 de las doce columnas de un dispositivo que se ha dividido en 12 pero tiene una anchura «pequeña/small» (más de 768 y menos de 992)

\item {} 
col-md-6: ocupa 6 de las doce columnas de un dispositivo que se ha dividido en 12 pero tiene una anchura «media» (unos 992 px)

\item {} 
col-lg-9: ocupa 9 de las doce columnas de un dispositivo que se ha dividido en 12 pero tiene una anchura «grande/large» (unos 992 px)

\end{itemize}


\subsection{Tipografía}
\label{\detokenize{tema3:tipografia}}
Bootstrap modifica la tipografía por defecto e incluso permite destacar algunos elementos. Por ejemplo un párrafo con la clase \sphinxcode{lead} destacará:

\begin{sphinxVerbatim}[commandchars=\\\{\}]
    \PYG{p}{\PYGZlt{}}\PYG{n+nt}{p} \PYG{n+na}{class}\PYG{o}{=}\PYG{l+s}{\PYGZdq{}lead\PYGZdq{}}\PYG{p}{\PYGZgt{}}
            Este párrafo es muy importante
    \PYG{p}{\PYGZlt{}}\PYG{p}{/}\PYG{n+nt}{p}\PYG{p}{\PYGZgt{}}
\PYG{p}{\PYGZlt{}}\PYG{n+nt}{p}\PYG{p}{\PYGZgt{}}Este párrafo es normal\PYG{p}{\PYGZlt{}}\PYG{p}{/}\PYG{n+nt}{p}\PYG{p}{\PYGZgt{}}
\end{sphinxVerbatim}

\begin{figure}[htbp]
\centering
\capstart

\noindent\sphinxincludegraphics{{bs2}.png}
\caption{Párrafo destacado}\label{\detokenize{tema3:id21}}\end{figure}

Se puede destacar texto usando lo siguiente:

\begin{sphinxVerbatim}[commandchars=\\\{\}]
\PYG{p}{\PYGZlt{}}\PYG{n+nt}{mark}\PYG{p}{\PYGZgt{}}Texto subrayado en amarillo\PYG{p}{\PYGZlt{}}\PYG{p}{/}\PYG{n+nt}{mark}\PYG{p}{\PYGZgt{}}
\end{sphinxVerbatim}


\section{Ejercicio responsive I}
\label{\detokenize{tema3:ejercicio-responsive-i}}
Hacer una página cuyo diseño se adapte automáticamente en función de la resolución. Dicha página tendrá 3 cajas cuyos \sphinxcode{id} serán A, B y C. El comportamiento de las cajas será el siguiente:
\begin{itemize}
\item {} 
Si la página se visualiza en una pantalla de 400px o menos las 3 cajas se limitarán a mostrarse una encima de la otra.

\item {} 
Si la pantalla tiene un tamaño de entre 401px y 800px las cajas A y B se mostrarán al principio, cubriendo cada una una anchura del 50\% (quizá haya que ajustar a un porcentaje menor). La caja C se mostrará debajo de A y B.

\item {} 
Si la pantalla tiene 801px o más A, B y C se mostrarán una al lado de la otra. A cubrirá un 20\%, B un 20\% y C un 58\%.

\end{itemize}

A continuación se muestra el resultado aproximado que se debe conseguir:

\begin{figure}[htbp]
\centering
\capstart

\noindent\sphinxincludegraphics{{responsive1-1}.png}
\caption{Resultado para pantallas pequeñas}\label{\detokenize{tema3:id22}}\end{figure}

\begin{figure}[htbp]
\centering
\capstart

\noindent\sphinxincludegraphics{{responsive1-2}.png}
\caption{Resultado para pantallas medianas}\label{\detokenize{tema3:id23}}\end{figure}

\begin{figure}[htbp]
\centering
\capstart

\noindent\sphinxincludegraphics{{responsive1-3}.png}
\caption{Resultado para pantallas grandes}\label{\detokenize{tema3:id24}}\end{figure}

Un posible HTML que resolviera esto sería el siguiente:

\begin{sphinxVerbatim}[commandchars=\\\{\}]
\PYG{p}{\PYGZlt{}}\PYG{n+nt}{body}\PYG{p}{\PYGZgt{}}
    \PYG{p}{\PYGZlt{}}\PYG{n+nt}{div} \PYG{n+na}{id}\PYG{o}{=}\PYG{l+s}{\PYGZdq{}A\PYGZdq{}}\PYG{p}{\PYGZgt{}}
        Caja A caja A caja A caja A
        caja A caja A caja A caja A
        caja A caja A caja A caja A
    \PYG{p}{\PYGZlt{}}\PYG{p}{/}\PYG{n+nt}{div}\PYG{p}{\PYGZgt{}}
    \PYG{p}{\PYGZlt{}}\PYG{n+nt}{div} \PYG{n+na}{id}\PYG{o}{=}\PYG{l+s}{\PYGZdq{}B\PYGZdq{}}\PYG{p}{\PYGZgt{}}
        Caja B caja B caja B caja B
        caja B caja B caja B caja B
        caja B caja B caja B caja B
    \PYG{p}{\PYGZlt{}}\PYG{p}{/}\PYG{n+nt}{div}\PYG{p}{\PYGZgt{}}
    \PYG{p}{\PYGZlt{}}\PYG{n+nt}{div} \PYG{n+na}{id}\PYG{o}{=}\PYG{l+s}{\PYGZdq{}C\PYGZdq{}}\PYG{p}{\PYGZgt{}}
        Caja C caja C caja C caja C
        caja C caja C caja C caja C
        caja C caja C caja C caja C
    \PYG{p}{\PYGZlt{}}\PYG{p}{/}\PYG{n+nt}{div}\PYG{p}{\PYGZgt{}}
\PYG{p}{\PYGZlt{}}\PYG{p}{/}\PYG{n+nt}{body}\PYG{p}{\PYGZgt{}}
\end{sphinxVerbatim}

Y el CSS que lo acompaña sería este:

\begin{sphinxVerbatim}[commandchars=\\\{\}]
\PYG{c}{/* En pantallas pequeñas...*/}
\PYG{p}{@}\PYG{k}{media} \PYG{n+nt}{screen} \PYG{n+nt}{and} \PYG{o}{(}\PYG{n+nt}{max\PYGZhy{}width}\PYG{p}{:}\PYG{n+nd}{400px}\PYG{o}{)}\PYG{p}{\PYGZob{}}
    \PYG{c}{/* ...no hacemos nada,dejamos que el}
\PYG{c}{     * navegador \PYGZdq{}apile\PYGZdq{} las cajas. Simplemente}
\PYG{c}{     * cambiamos el margen y el color para ver}
\PYG{c}{     * que nos funciona*/}
    \PYG{n+nt}{div} \PYG{p}{\PYGZob{}}
        \PYG{k}{margin\PYGZhy{}top}\PYG{p}{:} \PYG{l+m+mi}{40}\PYG{k+kt}{px}\PYG{p}{;}
        \PYG{k}{background\PYGZhy{}color}\PYG{p}{:} \PYG{l+m+mh}{\PYGZsh{}eeeeee}\PYG{p}{;}
    \PYG{p}{\PYGZcb{}}
\PYG{p}{\PYGZcb{}}\PYG{c}{/*Fin del media para ventanas pequeñas*/}
\PYG{c}{/* En pantallas medianas...*/}
\PYG{p}{@}\PYG{k}{media} \PYG{n+nt}{screen} \PYG{n+nt}{and} \PYG{o}{(}\PYG{n+nt}{min\PYGZhy{}width}\PYG{o}{:} \PYG{n+nt}{401px}\PYG{o}{)} \PYG{n+nt}{and} \PYG{o}{(}\PYG{n+nt}{max\PYGZhy{}width}\PYG{p}{:}\PYG{n+nd}{800px}\PYG{o}{)}\PYG{p}{\PYGZob{}}
    \PYG{c}{/* ...haremos \PYGZdq{}flotar\PYGZdq{} a A y B...*/}
    \PYG{p}{\PYGZsh{}}\PYG{n+nn}{A}\PYG{p}{\PYGZob{}}
        \PYG{k}{float}\PYG{p}{:} \PYG{k+kc}{left}\PYG{p}{;}
        \PYG{k}{width}\PYG{p}{:}\PYG{l+m+mi}{48}\PYG{k+kt}{\PYGZpc{}}\PYG{p}{;}
    \PYG{p}{\PYGZcb{}}
    \PYG{p}{\PYGZsh{}}\PYG{n+nn}{B}\PYG{p}{\PYGZob{}}
        \PYG{k}{float}\PYG{p}{:}\PYG{k+kc}{right}\PYG{p}{;}
        \PYG{k}{width}\PYG{p}{:} \PYG{l+m+mi}{48}\PYG{k+kt}{\PYGZpc{}}\PYG{p}{;}
    \PYG{p}{\PYGZcb{}}
    \PYG{c}{/* ...y haremos que C \PYGZdq{}limpie\PYGZdq{} el espacio sobrante\PYGZdq{}*/}
    \PYG{p}{\PYGZsh{}}\PYG{n+nn}{C}\PYG{p}{\PYGZob{}}
        \PYG{k}{clear}\PYG{p}{:} \PYG{k+kc}{both}\PYG{p}{;}
    \PYG{p}{\PYGZcb{}}
    \PYG{c}{/* También ponemos un color distinto}
\PYG{c}{     * pero solo para ver si lo hacemos bien*/}
    \PYG{n+nt}{div}\PYG{p}{\PYGZob{}}
        \PYG{k}{background\PYGZhy{}color}\PYG{p}{:} \PYG{l+m+mh}{\PYGZsh{}cccccc}\PYG{p}{;}
    \PYG{p}{\PYGZcb{}}
\PYG{p}{\PYGZcb{}} \PYG{c}{/*Fin del media para ventanas medianas*/}
\PYG{c}{/* Si estamos en pantallas grandes...*/}
\PYG{p}{@}\PYG{k}{media} \PYG{n+nt}{screen} \PYG{n+nt}{and} \PYG{o}{(}\PYG{n+nt}{min\PYGZhy{}width}\PYG{p}{:}\PYG{n+nd}{801px}\PYG{o}{)}\PYG{p}{\PYGZob{}}
    \PYG{c}{/*...entonces A y B flotan hacia la izquierda...*/}
    \PYG{p}{\PYGZsh{}}\PYG{n+nn}{A}\PYG{o}{,} \PYG{p}{\PYGZsh{}}\PYG{n+nn}{B}\PYG{p}{\PYGZob{}}
        \PYG{k}{width}\PYG{p}{:}\PYG{l+m+mi}{20}\PYG{k+kt}{\PYGZpc{}}\PYG{p}{;}
        \PYG{k}{float}\PYG{p}{:}\PYG{k+kc}{left}\PYG{p}{;}
    \PYG{p}{\PYGZcb{}}
    \PYG{c}{/* y C flota a la derecha llevándose el espacio}
\PYG{c}{     * que sobre. No lo ajustamos al 60\PYGZpc{} para}
\PYG{c}{     * evitar desbordamientos*/}
    \PYG{p}{\PYGZsh{}}\PYG{n+nn}{C}\PYG{p}{\PYGZob{}}
        \PYG{k}{width}\PYG{p}{:}\PYG{l+m+mi}{58}\PYG{k+kt}{\PYGZpc{}}\PYG{p}{;}
        \PYG{k}{float}\PYG{p}{:} \PYG{k+kc}{right}\PYG{p}{;}
    \PYG{p}{\PYGZcb{}}
    \PYG{c}{/* También volvemos a cambiar el color para}
\PYG{c}{     * las comprobaciones*/}
    \PYG{n+nt}{div}\PYG{p}{\PYGZob{}}
        \PYG{k}{background\PYGZhy{}color}\PYG{p}{:} \PYG{l+m+mh}{\PYGZsh{}eeeeee}\PYG{p}{;}
    \PYG{p}{\PYGZcb{}}
\PYG{p}{\PYGZcb{}}
\end{sphinxVerbatim}


\section{Ejercicio responsive II}
\label{\detokenize{tema3:ejercicio-responsive-ii}}
Hacer una página cuyo diseño se adapte automáticamente en función de la resolución. Dicha página tendrá 4 cajas cuyos \sphinxcode{id} serán A, B,  C y D. El comportamiento de las cajas será el siguiente:
\begin{itemize}
\item {} 
Para todos los casos hay una rejilla contenedora de 4 filas (todas de la misma altura) y 3 columnas (de anchos 20\%, 20\% y 60\%).

\item {} 
Si la pantalla tiene menos de 800px se mostrará una distribución como la que se muestra en la figura II-1.

\item {} 
Si la pantalla tiene menos de 800px se mostrará una distribución como la que se muestra en la figura II-2.

\end{itemize}

\begin{figure}[htbp]
\centering
\capstart

\noindent\sphinxincludegraphics{{responsive2-1}.png}
\caption{Figura II-1. (Para pantallas estrechas)}\label{\detokenize{tema3:id25}}\end{figure}

\begin{figure}[htbp]
\centering
\capstart

\noindent\sphinxincludegraphics{{responsive2-2}.png}
\caption{Figura II-2 (Para pantallas anchas)}\label{\detokenize{tema3:id26}}\end{figure}

A continuación se muestra el HTML:

\begin{sphinxVerbatim}[commandchars=\\\{\}]
\PYG{p}{\PYGZlt{}}\PYG{n+nt}{div} \PYG{n+na}{id}\PYG{o}{=}\PYG{l+s}{\PYGZdq{}contenedor\PYGZdq{}}\PYG{p}{\PYGZgt{}}
    \PYG{p}{\PYGZlt{}}\PYG{n+nt}{div} \PYG{n+na}{id}\PYG{o}{=}\PYG{l+s}{\PYGZdq{}A\PYGZdq{}}\PYG{p}{\PYGZgt{}}
        Caja A
    \PYG{p}{\PYGZlt{}}\PYG{p}{/}\PYG{n+nt}{div}\PYG{p}{\PYGZgt{}}
    \PYG{p}{\PYGZlt{}}\PYG{n+nt}{div} \PYG{n+na}{id}\PYG{o}{=}\PYG{l+s}{\PYGZdq{}B\PYGZdq{}}\PYG{p}{\PYGZgt{}}
        Caja B
    \PYG{p}{\PYGZlt{}}\PYG{p}{/}\PYG{n+nt}{div}\PYG{p}{\PYGZgt{}}
    \PYG{p}{\PYGZlt{}}\PYG{n+nt}{div} \PYG{n+na}{id}\PYG{o}{=}\PYG{l+s}{\PYGZdq{}C\PYGZdq{}}\PYG{p}{\PYGZgt{}}
        Caja C
    \PYG{p}{\PYGZlt{}}\PYG{p}{/}\PYG{n+nt}{div}\PYG{p}{\PYGZgt{}}
    \PYG{p}{\PYGZlt{}}\PYG{n+nt}{div} \PYG{n+na}{id}\PYG{o}{=}\PYG{l+s}{\PYGZdq{}D\PYGZdq{}}\PYG{p}{\PYGZgt{}}
        Caja D
    \PYG{p}{\PYGZlt{}}\PYG{p}{/}\PYG{n+nt}{div}\PYG{p}{\PYGZgt{}}
\PYG{p}{\PYGZlt{}}\PYG{p}{/}\PYG{n+nt}{div}\PYG{p}{\PYGZgt{}}
\end{sphinxVerbatim}

Y un posible CSS:

\begin{sphinxVerbatim}[commandchars=\\\{\}]
\PYG{c}{/* Todas las cajas tienen borde siempre*/}
\PYG{n+nt}{div}\PYG{p}{\PYGZob{}}
    \PYG{k}{border}\PYG{p}{:} \PYG{k+kc}{solid} \PYG{l+m+mi}{1}\PYG{k+kt}{px} \PYG{k+kc}{black}\PYG{p}{;}
\PYG{p}{\PYGZcb{}}

\PYG{n+nt}{div}\PYG{p}{\PYGZsh{}}\PYG{n+nn}{contenedor}\PYG{p}{\PYGZob{}}
    \PYG{k}{display}\PYG{p}{:} \PYG{k}{grid}\PYG{p}{;}
    \PYG{k}{grid\PYGZhy{}template\PYGZhy{}rows}\PYG{p}{:} \PYG{l+m+mi}{25}\PYG{k+kt}{\PYGZpc{}} \PYG{l+m+mi}{25}\PYG{k+kt}{\PYGZpc{}} \PYG{l+m+mi}{25}\PYG{k+kt}{\PYGZpc{}} \PYG{l+m+mi}{25}\PYG{k+kt}{\PYGZpc{}}\PYG{p}{;}
    \PYG{k}{grid\PYGZhy{}template\PYGZhy{}columns}\PYG{p}{:} \PYG{l+m+mi}{20}\PYG{k+kt}{\PYGZpc{}} \PYG{l+m+mi}{20}\PYG{k+kt}{\PYGZpc{}} \PYG{l+m+mi}{60}\PYG{k+kt}{\PYGZpc{}}\PYG{p}{;}
\PYG{p}{\PYGZcb{}}
\PYG{p}{@}\PYG{k}{media} \PYG{n+nt}{screen} \PYG{n+nt}{and} \PYG{o}{(}\PYG{n+nt}{min\PYGZhy{}width}\PYG{p}{:}\PYG{n+nd}{800px}\PYG{o}{)}\PYG{p}{\PYGZob{}}

    \PYG{c}{/* Esto no hacía falta, se usa}
\PYG{c}{     * para comprobar que nos sale}
\PYG{c}{     * bien al estrechar o ensanchar}
\PYG{c}{     * la \PYGZdq{}pantalla\PYGZdq{}*/}
    \PYG{p}{\PYGZsh{}}\PYG{n+nn}{A}\PYG{o}{,} \PYG{p}{\PYGZsh{}}\PYG{n+nn}{B}\PYG{o}{,} \PYG{p}{\PYGZsh{}}\PYG{n+nn}{C}\PYG{o}{,} \PYG{p}{\PYGZsh{}}\PYG{n+nn}{D}\PYG{p}{\PYGZob{}}
        \PYG{k}{background\PYGZhy{}color}\PYG{p}{:} \PYG{n+nb}{rgb}\PYG{p}{(}\PYG{l+m+mi}{240}\PYG{p}{,} \PYG{l+m+mi}{240}\PYG{p}{,}\PYG{l+m+mi}{220}\PYG{p}{)}\PYG{p}{;}
    \PYG{p}{\PYGZcb{}}

    \PYG{p}{\PYGZsh{}}\PYG{n+nn}{A}\PYG{p}{\PYGZob{}}
        \PYG{k}{grid\PYGZhy{}row}\PYG{p}{:} \PYG{l+m+mi}{1}\PYG{p}{;}
        \PYG{k}{grid\PYGZhy{}column}\PYG{p}{:}\PYG{l+m+mi}{1}\PYG{o}{/}\PYG{l+m+mi}{4} \PYG{p}{;}
    \PYG{p}{\PYGZcb{}}
    \PYG{p}{\PYGZsh{}}\PYG{n+nn}{B}\PYG{p}{\PYGZob{}}
        \PYG{k}{grid\PYGZhy{}row}\PYG{p}{:}\PYG{l+m+mi}{2} \PYG{p}{;}
        \PYG{k}{grid\PYGZhy{}column}\PYG{p}{:}\PYG{l+m+mi}{1}\PYG{o}{/}\PYG{l+m+mi}{4} \PYG{p}{;}
    \PYG{p}{\PYGZcb{}}
    \PYG{p}{\PYGZsh{}}\PYG{n+nn}{C}\PYG{p}{\PYGZob{}}
        \PYG{k}{grid\PYGZhy{}row}\PYG{p}{:} \PYG{l+m+mi}{3}\PYG{o}{/}\PYG{l+m+mi}{5}\PYG{p}{;}
        \PYG{k}{grid\PYGZhy{}column}\PYG{p}{:} \PYG{l+m+mi}{1}\PYG{o}{/}\PYG{l+m+mi}{3}\PYG{p}{;}
    \PYG{p}{\PYGZcb{}}
    \PYG{p}{\PYGZsh{}}\PYG{n+nn}{D}\PYG{p}{\PYGZob{}}
        \PYG{k}{grid\PYGZhy{}row}\PYG{p}{:} \PYG{l+m+mi}{3}\PYG{o}{/}\PYG{l+m+mi}{5}\PYG{p}{;}
        \PYG{k}{grid\PYGZhy{}column}\PYG{p}{:}\PYG{l+m+mi}{3}\PYG{o}{/}\PYG{l+m+mi}{4} \PYG{p}{;}
    \PYG{p}{\PYGZcb{}}
\PYG{p}{\PYGZcb{}} \PYG{c}{/* Fin del media para max\PYGZhy{}width 800px*/}
\PYG{p}{@}\PYG{k}{media} \PYG{n+nt}{screen} \PYG{n+nt}{and} \PYG{o}{(}\PYG{n+nt}{max\PYGZhy{}width}\PYG{p}{:}\PYG{n+nd}{799px}\PYG{o}{)}\PYG{p}{\PYGZob{}}
    \PYG{c}{/* Esto no hacía falta, se usa}
\PYG{c}{     * para comprobar que nos sale}
\PYG{c}{     * bien al estrechar o ensanchar}
\PYG{c}{     * la \PYGZdq{}pantalla\PYGZdq{}*/}
    \PYG{p}{\PYGZsh{}}\PYG{n+nn}{A}\PYG{o}{,} \PYG{p}{\PYGZsh{}}\PYG{n+nn}{B}\PYG{o}{,} \PYG{p}{\PYGZsh{}}\PYG{n+nn}{C}\PYG{o}{,} \PYG{p}{\PYGZsh{}}\PYG{n+nn}{D}\PYG{p}{\PYGZob{}}
        \PYG{k}{background\PYGZhy{}color}\PYG{p}{:} \PYG{n+nb}{rgb}\PYG{p}{(}\PYG{l+m+mi}{220}\PYG{p}{,} \PYG{l+m+mi}{240}\PYG{p}{,} \PYG{l+m+mi}{230}\PYG{p}{)}\PYG{p}{;}
    \PYG{p}{\PYGZcb{}}
    \PYG{p}{\PYGZsh{}}\PYG{n+nn}{A}\PYG{p}{\PYGZob{}}
        \PYG{k}{grid\PYGZhy{}row}\PYG{p}{:} \PYG{l+m+mi}{1}\PYG{p}{;}
        \PYG{k}{grid\PYGZhy{}column}\PYG{p}{:}\PYG{l+m+mi}{1}\PYG{o}{/}\PYG{l+m+mi}{3} \PYG{p}{;}
    \PYG{p}{\PYGZcb{}}
    \PYG{p}{\PYGZsh{}}\PYG{n+nn}{B}\PYG{p}{\PYGZob{}}
        \PYG{k}{grid\PYGZhy{}row}\PYG{p}{:} \PYG{l+m+mi}{1}\PYG{p}{;}
        \PYG{k}{grid\PYGZhy{}column}\PYG{p}{:}\PYG{l+m+mi}{3} \PYG{p}{;}
    \PYG{p}{\PYGZcb{}}
    \PYG{p}{\PYGZsh{}}\PYG{n+nn}{C}\PYG{p}{\PYGZob{}}
        \PYG{k}{grid\PYGZhy{}row}\PYG{p}{:} \PYG{l+m+mi}{2}\PYG{p}{;}
        \PYG{k}{grid\PYGZhy{}column}\PYG{p}{:}\PYG{l+m+mi}{1}\PYG{o}{/}\PYG{l+m+mi}{5} \PYG{p}{;}
    \PYG{p}{\PYGZcb{}}
    \PYG{p}{\PYGZsh{}}\PYG{n+nn}{D}\PYG{p}{\PYGZob{}}
        \PYG{k}{grid\PYGZhy{}row}\PYG{p}{:} \PYG{l+m+mi}{3}\PYG{o}{/}\PYG{l+m+mi}{4}\PYG{p}{;}
        \PYG{k}{grid\PYGZhy{}column}\PYG{p}{:}\PYG{l+m+mi}{1}\PYG{o}{/}\PYG{l+m+mi}{5} \PYG{p}{;}
    \PYG{p}{\PYGZcb{}}
\PYG{p}{\PYGZcb{}} \PYG{c}{/* Fin del media para min\PYGZhy{}width 799px*/}
\end{sphinxVerbatim}


\chapter{Javascript}
\label{\detokenize{tema4:javascript}}\label{\detokenize{tema4::doc}}

\section{Introducción}
\label{\detokenize{tema4:introduccion}}
Surgió como una iniciativa de Netscape. \sphinxstylestrong{No hay ninguna relación entre Java y Javascript}. Algunas características de Javascript son:
\begin{itemize}
\item {} 
No se convierte en bytecodes, lo interpreta el navegador

\item {} 
Está estandarizado aunque no todos los navegadores se ciñen al 100\% al estándar. El nombre del estándar es «ECMAScript»

\item {} 
No está orientado a objetos, sino que está basado en objetos.

\item {} 
Es de tipado débil: esto significa que podemos cambiar una variable de tipo sin problemas, el intérprete intentará hacer las conversiones correctas.

\end{itemize}


\section{Tipos de datos}
\label{\detokenize{tema4:tipos-de-datos}}
Javascript acepta los siguientes tipos de datos:
\begin{itemize}
\item {} 
Números.

\item {} 
Cadenas

\item {} 
Booleanos (lógicos)

\item {} 
undefined: se utiliza cuando intentamos acceder a una variable que no contiene nada porque no se ha creado.

\item {} 
null: se utiliza habitualmente para indicar algo vacío.

\end{itemize}


\section{Incrustando Javascript}
\label{\detokenize{tema4:incrustando-javascript}}
Javascript se inserta en HTML con la etiqueta \sphinxcode{\textless{}script\textgreater{}}. Esta etiqueta puede ir en cualquier sitio del HTML, dentro de \sphinxcode{\textless{}head\textgreater{}} o dentro de \sphinxcode{\textless{}body\textgreater{}}.

Un programa muy simple sería este:

\begin{sphinxVerbatim}[commandchars=\\\{\}]
    \PYG{k+kd}{var} \PYG{n+nx}{una\PYGZus{}variable}
\PYG{n+nx}{una\PYGZus{}variable}\PYG{o}{=}\PYG{l+m+mi}{42}
\PYG{n+nb}{document}\PYG{p}{.}\PYG{n+nx}{write}\PYG{p}{(}\PYG{n+nx}{una\PYGZus{}variable}\PYG{p}{)}
\end{sphinxVerbatim}


\section{Decisiones}
\label{\detokenize{tema4:decisiones}}
Las decisiones se toman con la sentencia \sphinxcode{if} que funciona exactamente igual que en Java. Se pueden utilizar los mismos operadores \sphinxcode{\&\&} y \sphinxcode{\textbar{}\textbar{}} al igual que en Java.

\begin{sphinxVerbatim}[commandchars=\\\{\}]
    \PYG{k}{if} \PYG{p}{(}\PYG{n+nx}{una\PYGZus{}variable} \PYG{o}{\PYGZgt{}} \PYG{n+nx}{otra\PYGZus{}variable}\PYG{p}{)} \PYG{p}{\PYGZob{}}
    \PYG{n+nb}{document}\PYG{p}{.}\PYG{n+nx}{write} \PYG{p}{(}\PYG{l+s+s2}{\PYGZdq{}La primera es mayor que la segunda\PYGZdq{}}\PYG{p}{)}
\PYG{p}{\PYGZcb{}} \PYG{k}{else} \PYG{p}{\PYGZob{}}
    \PYG{n+nb}{document}\PYG{p}{.}\PYG{n+nx}{write} \PYG{p}{(}\PYG{l+s+s2}{\PYGZdq{}La segunda es mayor que la primera\PYGZdq{}}\PYG{p}{)}
\PYG{p}{\PYGZcb{}}
\end{sphinxVerbatim}


\section{Vectores o Arrays}
\label{\detokenize{tema4:vectores-o-arrays}}
En Javascript los arrays pueden almacenar elementos de distinto tipo. Al crearlos podemos indicar el tamaño o no, pero no habrá problemas si queremos almacenar más elementos de los previstos.

\begin{sphinxVerbatim}[commandchars=\\\{\}]
    \PYG{c+cm}{/* Una forma de crear un array*/}
\PYG{n+nx}{vector\PYGZus{}nombres}\PYG{o}{=}\PYG{k}{new} \PYG{n+nb}{Array}\PYG{p}{(}\PYG{p}{)}
\PYG{n+nx}{vector\PYGZus{}nombres}\PYG{p}{[}\PYG{l+m+mi}{0}\PYG{p}{]}\PYG{o}{=}\PYG{l+s+s2}{\PYGZdq{}Juan Perez\PYGZdq{}}
\PYG{n+nx}{vector\PYGZus{}nombres}\PYG{p}{[}\PYG{l+m+mi}{1}\PYG{p}{]}\PYG{o}{=}\PYG{l+s+s2}{\PYGZdq{}Pedro Diaz\PYGZdq{}}
\PYG{n+nb}{document}\PYG{p}{.}\PYG{n+nx}{write} \PYG{p}{(}\PYG{l+s+s2}{\PYGZdq{}El primer nombre es:\PYGZdq{}}\PYG{o}{+}\PYG{n+nx}{vector\PYGZus{}nombres}\PYG{p}{[}\PYG{l+m+mi}{0}\PYG{p}{]}\PYG{p}{)}

\PYG{c+cm}{/* Otra forma de crearlos*/}
\PYG{n+nx}{vector\PYGZus{}numeros}\PYG{o}{=}\PYG{k}{new} \PYG{n+nb}{Array}\PYG{p}{(}\PYG{l+m+mi}{2}\PYG{p}{)}
\PYG{n+nx}{vector\PYGZus{}numeros}\PYG{p}{[}\PYG{l+m+mi}{0}\PYG{p}{]}\PYG{o}{=}\PYG{l+m+mi}{23}
\PYG{n+nx}{vector\PYGZus{}numeros}\PYG{p}{[}\PYG{l+m+mi}{1}\PYG{p}{]}\PYG{o}{=}\PYG{o}{\PYGZhy{}}\PYG{l+m+mf}{45.23}
\PYG{n+nx}{vector\PYGZus{}numeros}\PYG{p}{[}\PYG{l+m+mi}{2}\PYG{p}{]}\PYG{o}{=}\PYG{l+m+mi}{45}\PYG{n+nx}{e2}
\end{sphinxVerbatim}


\section{Bucles}
\label{\detokenize{tema4:bucles}}

\subsection{Bucles for}
\label{\detokenize{tema4:bucles-for}}

\subsubsection{Bucles for estilo clásico}
\label{\detokenize{tema4:bucles-for-estilo-clasico}}
En estos bucles hay que poner la inicialización, la condición de final y la actualización:

\begin{sphinxVerbatim}[commandchars=\\\{\}]
\PYG{k}{for} \PYG{p}{(}\PYG{k+kd}{var} \PYG{n+nx}{i}\PYG{o}{=}\PYG{l+m+mi}{0}\PYG{p}{;} \PYG{n+nx}{i}\PYG{o}{\PYGZlt{}}\PYG{n+nx}{vector\PYGZus{}numeros}\PYG{p}{.}\PYG{n+nx}{length}\PYG{p}{;} \PYG{n+nx}{i}\PYG{o}{++}\PYG{p}{)}\PYG{p}{\PYGZob{}}
        \PYG{n+nb}{document}\PYG{p}{.}\PYG{n+nx}{write}\PYG{p}{(}\PYG{l+s+s2}{\PYGZdq{}\PYGZlt{}br/\PYGZgt{}\PYGZdq{}}\PYG{p}{)}
\PYG{n+nb}{document}\PYG{p}{.}\PYG{n+nx}{write} \PYG{p}{(}\PYG{l+s+s2}{\PYGZdq{}En la posición \PYGZdq{}}\PYG{o}{+}\PYG{n+nx}{i}\PYG{p}{)}
\PYG{n+nb}{document}\PYG{p}{.}\PYG{n+nx}{write} \PYG{p}{(}\PYG{l+s+s2}{\PYGZdq{} está el número \PYGZdq{}} \PYG{o}{+} \PYG{n+nx}{vector\PYGZus{}numeros}\PYG{p}{[}\PYG{n+nx}{i}\PYG{p}{]}\PYG{p}{)}
\PYG{p}{\PYGZcb{}}
\end{sphinxVerbatim}

Obsérvese que hemos introducido el atributo \sphinxcode{length} de la clase \sphinxcode{Array} que nos indica la longitud del vector.


\subsubsection{Ejercicio}
\label{\detokenize{tema4:ejercicio}}
Crear un vector de 6 posiciones y rellenarlo con estos números: 9.98, 7.86, 4.53, 8.91, 5.76, 2.31.

Ordenar el vector y mostrar el contenido del vector ordenado por pantalla.

\begin{sphinxVerbatim}[commandchars=\\\{\}]
\PYG{n}{var} \PYG{n}{v}\PYG{o}{=}\PYG{k}{new} \PYG{n}{Array}\PYG{o}{(}\PYG{o}{)}
\PYG{n}{v}\PYG{o}{=}\PYG{o}{[}\PYG{l+m+mf}{9.98}\PYG{o}{,} \PYG{l+m+mf}{7.86}\PYG{o}{,} \PYG{l+m+mf}{4.53}\PYG{o}{,}
   \PYG{l+m+mf}{8.91}\PYG{o}{,} \PYG{l+m+mf}{5.76}\PYG{o}{,} \PYG{l+m+mf}{2.31}\PYG{o}{]}

\PYG{c+cm}{/* Vamos cogiendo cada elemento...*/}
\PYG{k}{for} \PYG{o}{(}\PYG{n}{var} \PYG{n}{i}\PYG{o}{=}\PYG{l+m+mi}{0}\PYG{o}{;} \PYG{n}{i}\PYG{o}{\PYGZlt{}}\PYG{n}{v}\PYG{o}{.}\PYG{n+na}{length}\PYG{o}{;} \PYG{n}{i}\PYG{o}{+}\PYG{o}{+}\PYG{o}{)}\PYG{o}{\PYGZob{}}
    \PYG{c+cm}{/* Y se compara con}
\PYG{c+cm}{     * todos los demas*/}
    \PYG{k}{for} \PYG{o}{(}\PYG{n}{var} \PYG{n}{j}\PYG{o}{=}\PYG{l+m+mi}{0}\PYG{o}{;} \PYG{n}{j}\PYG{o}{\PYGZlt{}}\PYG{n}{v}\PYG{o}{.}\PYG{n+na}{length}\PYG{o}{;} \PYG{n}{j}\PYG{o}{+}\PYG{o}{+}\PYG{o}{)} \PYG{o}{\PYGZob{}}
        \PYG{k}{if} \PYG{o}{(}\PYG{n}{v}\PYG{o}{[}\PYG{n}{j}\PYG{o}{]}\PYG{o}{\PYGZgt{}}\PYG{n}{v}\PYG{o}{[}\PYG{n}{i}\PYG{o}{]}\PYG{o}{)} \PYG{o}{\PYGZob{}}
            \PYG{n}{aux}\PYG{o}{=}\PYG{n}{v}\PYG{o}{[}\PYG{n}{i}\PYG{o}{]}
            \PYG{n}{v}\PYG{o}{[}\PYG{n}{i}\PYG{o}{]}\PYG{o}{=}\PYG{n}{v}\PYG{o}{[}\PYG{n}{j}\PYG{o}{]}
            \PYG{n}{v}\PYG{o}{[}\PYG{n}{j}\PYG{o}{]}\PYG{o}{=}\PYG{n}{aux}
        \PYG{o}{\PYGZcb{}} \PYG{c+cm}{/* Fin del if*/}
    \PYG{o}{\PYGZcb{}} \PYG{c+cm}{/* Fin del for interno*/}
\PYG{o}{\PYGZcb{}} \PYG{c+cm}{/* fin del for externo*/}


\PYG{c+cm}{/* Se imprime el contenido*/}
\PYG{k}{for} \PYG{o}{(}\PYG{n}{var} \PYG{n}{i}\PYG{o}{=}\PYG{l+m+mi}{0}\PYG{o}{;} \PYG{n}{i}\PYG{o}{\PYGZlt{}}\PYG{n}{v}\PYG{o}{.}\PYG{n+na}{length}\PYG{o}{;} \PYG{n}{i}\PYG{o}{+}\PYG{o}{+}\PYG{o}{)}\PYG{o}{\PYGZob{}}
    \PYG{n}{alert} \PYG{o}{(}\PYG{l+s}{\PYGZdq{}Pos \PYGZdq{}}\PYG{o}{+}\PYG{n}{i}\PYG{o}{+} \PYG{l+s}{\PYGZdq{}:\PYGZdq{}}\PYG{o}{+}\PYG{n}{v}\PYG{o}{[}\PYG{n}{i}\PYG{o}{]}\PYG{o}{)}
\PYG{o}{\PYGZcb{}}
\end{sphinxVerbatim}


\subsubsection{Bucles foreach}
\label{\detokenize{tema4:bucles-foreach}}
Funciona igual que el anterior pero es mucho más corto.

\begin{sphinxVerbatim}[commandchars=\\\{\}]
\PYG{k}{for} \PYG{p}{(}\PYG{k+kd}{var} \PYG{n+nx}{posicion} \PYG{k}{in} \PYG{n+nx}{vector\PYGZus{}numeros}\PYG{p}{)} \PYG{p}{\PYGZob{}}
                \PYG{n+nb}{document}\PYG{p}{.}\PYG{n+nx}{write}\PYG{p}{(}\PYG{l+s+s2}{\PYGZdq{}\PYGZlt{}br/\PYGZgt{}\PYGZdq{}}\PYG{p}{)}
                \PYG{n+nb}{document}\PYG{p}{.}\PYG{n+nx}{write} \PYG{p}{(}\PYG{l+s+s2}{\PYGZdq{}En la posición \PYGZdq{}}\PYG{o}{+}\PYG{n+nx}{posicion}\PYG{p}{)}
                \PYG{n+nb}{document}\PYG{p}{.}\PYG{n+nx}{write} \PYG{p}{(}\PYG{l+s+s2}{\PYGZdq{} está el número \PYGZdq{}} \PYG{o}{+} \PYG{n+nx}{vector\PYGZus{}numeros}\PYG{p}{[}\PYG{n+nx}{posicion}\PYG{p}{]}\PYG{p}{)}
\PYG{p}{\PYGZcb{}}
\end{sphinxVerbatim}


\subsection{Bucles while}
\label{\detokenize{tema4:bucles-while}}
Los bucles \sphinxcode{while} funcionan igual que en Java

\begin{sphinxVerbatim}[commandchars=\\\{\}]
    \PYG{k+kd}{var} \PYG{n+nx}{posicion}\PYG{o}{=}\PYG{l+m+mi}{0}
    \PYG{k}{while} \PYG{p}{(}\PYG{n+nx}{posicion}\PYG{o}{\PYGZlt{}}\PYG{n+nx}{vector\PYGZus{}numeros}\PYG{p}{.}\PYG{n+nx}{length}\PYG{p}{)}\PYG{p}{\PYGZob{}}
                    \PYG{n+nb}{document}\PYG{p}{.}\PYG{n+nx}{write}\PYG{p}{(}\PYG{l+s+s2}{\PYGZdq{}\PYGZlt{}br/\PYGZgt{}\PYGZdq{}}\PYG{p}{)}
                    \PYG{n+nb}{document}\PYG{p}{.}\PYG{n+nx}{write} \PYG{p}{(}\PYG{l+s+s2}{\PYGZdq{}En la posición \PYGZdq{}}\PYG{o}{+}\PYG{n+nx}{posicion}\PYG{p}{)}
                    \PYG{n+nb}{document}\PYG{p}{.}\PYG{n+nx}{write} \PYG{p}{(}\PYG{l+s+s2}{\PYGZdq{} está el número \PYGZdq{}} \PYG{o}{+} \PYG{n+nx}{vector\PYGZus{}numeros}\PYG{p}{[}\PYG{n+nx}{posicion}\PYG{p}{]}\PYG{p}{)}
                    \PYG{n+nx}{posicion}\PYG{o}{++}
\PYG{p}{\PYGZcb{}}
\end{sphinxVerbatim}


\section{Ejercicio: media aritmética}
\label{\detokenize{tema4:ejercicio-media-aritmetica}}
Crear un programa que calcule la media aritmética del vector de números.

\begin{sphinxVerbatim}[commandchars=\\\{\}]
\PYG{k+kd}{var} \PYG{n+nx}{suma}\PYG{o}{=}\PYG{l+m+mi}{0}
\PYG{k}{for} \PYG{p}{(}\PYG{k+kd}{var} \PYG{n+nx}{pos} \PYG{k}{in} \PYG{n+nx}{vector\PYGZus{}numeros}\PYG{p}{)}\PYG{p}{\PYGZob{}}
        \PYG{n+nx}{suma}\PYG{o}{=}\PYG{n+nx}{suma}\PYG{o}{+}\PYG{n+nx}{vector\PYGZus{}numeros}\PYG{p}{[}\PYG{n+nx}{pos}\PYG{p}{]}
\PYG{p}{\PYGZcb{}}
\PYG{k+kd}{var} \PYG{n+nx}{media}\PYG{o}{=}\PYG{n+nx}{suma} \PYG{o}{/} \PYG{n+nx}{vector\PYGZus{}numeros}\PYG{p}{.}\PYG{n+nx}{length}
\PYG{n+nb}{document}\PYG{p}{.}\PYG{n+nx}{write}\PYG{p}{(}\PYG{l+s+s2}{\PYGZdq{}\PYGZlt{}br/\PYGZgt{}La media es:\PYGZdq{}} \PYG{o}{+} \PYG{n+nx}{media}\PYG{p}{)}
\end{sphinxVerbatim}


\section{Ejercicio: desviación media}
\label{\detokenize{tema4:ejercicio-desviacion-media}}
Crear un programa que calcule la desviación media del vector de números.

\begin{sphinxVerbatim}[commandchars=\\\{\}]
    \PYG{c+cm}{/* Para calcular la desviación media*/}
\PYG{n+nx}{suma}\PYG{o}{=}\PYG{l+m+mi}{0}
\PYG{k}{for} \PYG{p}{(}\PYG{k+kd}{var} \PYG{n+nx}{pos} \PYG{k}{in} \PYG{n+nx}{vector\PYGZus{}numeros}\PYG{p}{)} \PYG{p}{\PYGZob{}}
    \PYG{k+kd}{var} \PYG{n+nx}{desviacion}\PYG{o}{=} \PYG{n+nb}{Math}\PYG{p}{.}\PYG{n+nx}{abs} \PYG{p}{(} \PYG{n+nx}{vector\PYGZus{}numeros}\PYG{p}{[}\PYG{n+nx}{pos}\PYG{p}{]} \PYG{o}{\PYGZhy{}} \PYG{n+nx}{media} \PYG{p}{)}
    \PYG{n+nx}{suma} \PYG{o}{=} \PYG{n+nx}{suma} \PYG{o}{+} \PYG{n+nx}{desviacion}
\PYG{p}{\PYGZcb{}}
\PYG{c+cm}{/* En este punto la variable suma contiene la suma de las desviaciones*/}
\PYG{k+kd}{var} \PYG{n+nx}{desv\PYGZus{}media} \PYG{o}{=} \PYG{n+nx}{suma} \PYG{o}{/} \PYG{n+nx}{vector\PYGZus{}numeros}\PYG{p}{.}\PYG{n+nx}{length}
\PYG{n+nb}{document}\PYG{p}{.}\PYG{n+nx}{write}\PYG{p}{(}\PYG{l+s+s2}{\PYGZdq{}\PYGZlt{}br/\PYGZgt{}La desv media es:\PYGZdq{}}\PYG{o}{+}\PYG{n+nx}{desv\PYGZus{}media}\PYG{p}{)}
\end{sphinxVerbatim}


\section{Ejercicio: la mediana}
\label{\detokenize{tema4:ejercicio-la-mediana}}
Calcular la mediana del vector

\begin{sphinxVerbatim}[commandchars=\\\{\}]
        \PYG{k}{if} \PYG{o}{(}\PYG{n}{v}\PYG{o}{.}\PYG{n+na}{length}\PYG{o}{\PYGZpc{}}\PYG{l+m+mi}{2}\PYG{o}{=}\PYG{o}{=}\PYG{l+m+mi}{0}\PYG{o}{)} \PYG{o}{\PYGZob{}}
    \PYG{n}{var} \PYG{n}{pos1}\PYG{o}{=}\PYG{n}{v}\PYG{o}{.}\PYG{n+na}{length}\PYG{o}{/}\PYG{l+m+mi}{2}
    \PYG{n}{var} \PYG{n}{pos2}\PYG{o}{=}\PYG{n}{pos1}\PYG{o}{\PYGZhy{}}\PYG{l+m+mi}{1}
    \PYG{n}{var} \PYG{n}{elem1}\PYG{o}{=}\PYG{n}{v}\PYG{o}{[}\PYG{n}{pos1}\PYG{o}{]}
    \PYG{n}{var} \PYG{n}{elem2}\PYG{o}{=}\PYG{n}{v}\PYG{o}{[}\PYG{n}{pos2}\PYG{o}{]}
    \PYG{n}{var} \PYG{n}{mediana}\PYG{o}{=}\PYG{o}{(}\PYG{n}{elem1}\PYG{o}{+}\PYG{n}{elem2}\PYG{o}{)}\PYG{o}{/}\PYG{l+m+mi}{2}
\PYG{o}{\PYGZcb{}} \PYG{k}{else} \PYG{o}{\PYGZob{}}
    \PYG{n}{var} \PYG{n}{pos\PYGZus{}central}\PYG{o}{=}\PYG{o}{(}\PYG{n}{v}\PYG{o}{.}\PYG{n+na}{length}\PYG{o}{\PYGZhy{}}\PYG{l+m+mi}{1}\PYG{o}{)}\PYG{o}{/}\PYG{l+m+mi}{2}
    \PYG{n}{var} \PYG{n}{mediana}\PYG{o}{=}\PYG{n}{v}\PYG{o}{[}\PYG{n}{pos\PYGZus{}central}\PYG{o}{]}
\PYG{o}{\PYGZcb{}}
\PYG{n}{doc}     \PYG{n}{ument}\PYG{o}{.}\PYG{n+na}{write}\PYG{o}{(}\PYG{l+s}{\PYGZdq{}La mediana es:\PYGZdq{}}\PYG{o}{+}\PYG{n}{mediana}\PYG{o}{)}
\end{sphinxVerbatim}


\section{Funciones}
\label{\detokenize{tema4:funciones}}
Para crear una función usaremos la palabra \sphinxcode{function}, pondremos el nombre, luego los parámetros, dentro irá el código de la función, y si queremos devolver algo usaremos \sphinxcode{return}.

\begin{sphinxVerbatim}[commandchars=\\\{\}]
    \PYG{c+cm}{/* Función a la que le pasamos un vector de números y que}
\PYG{c+cm}{ * nos devuelve la media de sus valores*/}

\PYG{k+kd}{function} \PYG{n+nx}{calcularMedia}\PYG{p}{(}\PYG{n+nx}{vector\PYGZus{}valores}\PYG{p}{)}\PYG{p}{\PYGZob{}}
    \PYG{k+kd}{var} \PYG{n+nx}{suma}\PYG{o}{=}\PYG{l+m+mi}{0}
    \PYG{k}{for} \PYG{p}{(}\PYG{k+kd}{var} \PYG{n+nx}{pos} \PYG{k}{in} \PYG{n+nx}{vector\PYGZus{}valores}\PYG{p}{)}\PYG{p}{\PYGZob{}}
        \PYG{n+nx}{suma} \PYG{o}{=} \PYG{n+nx}{suma} \PYG{o}{+} \PYG{n+nx}{vector\PYGZus{}valores}\PYG{p}{[}\PYG{n+nx}{pos}\PYG{p}{]}
    \PYG{p}{\PYGZcb{}}
    \PYG{k}{return} \PYG{n+nx}{suma} \PYG{o}{/} \PYG{n+nx}{vector\PYGZus{}valores}\PYG{p}{.}\PYG{n+nx}{length}
\PYG{p}{\PYGZcb{}}

\PYG{k+kd}{var} \PYG{n+nx}{vector}\PYG{o}{=}\PYG{k}{new} \PYG{n+nb}{Array}\PYG{p}{(}\PYG{l+m+mi}{4}\PYG{p}{)}
\PYG{n+nx}{vector}\PYG{p}{[}\PYG{l+m+mi}{0}\PYG{p}{]}\PYG{o}{=}\PYG{l+m+mi}{5}
\PYG{n+nx}{vector}\PYG{p}{[}\PYG{l+m+mi}{1}\PYG{p}{]}\PYG{o}{=}\PYG{l+m+mi}{2}
\PYG{n+nx}{vector}\PYG{p}{[}\PYG{l+m+mi}{2}\PYG{p}{]}\PYG{o}{=}\PYG{l+m+mi}{7}
\PYG{n+nx}{vector}\PYG{p}{[}\PYG{l+m+mi}{3}\PYG{p}{]}\PYG{o}{=}\PYG{l+m+mi}{8}

\PYG{k+kd}{var} \PYG{n+nx}{media}\PYG{o}{=}\PYG{n+nx}{calcularMedia}\PYG{p}{(}\PYG{n+nx}{vector}\PYG{p}{)}
\PYG{n+nb}{document}\PYG{p}{.}\PYG{n+nx}{write}\PYG{p}{(}\PYG{l+s+s2}{\PYGZdq{}\PYGZlt{}br/\PYGZgt{}La media es:\PYGZdq{}}\PYG{o}{+}\PYG{n+nx}{media}\PYG{p}{)}
\end{sphinxVerbatim}

Una cuestión importante es que las funciones son valores asignables. Cuando queramos asignar una función a una variable \sphinxstylestrong{no pondremos paréntesis}. Cuando sí queramos ejecutar una función (ya sea con su nombre original o con el de la variable, sí pondremos los paréntesis con los parámetros que queramos pasar**.

\begin{sphinxVerbatim}[commandchars=\\\{\}]
    \PYG{k+kd}{function} \PYG{n+nx}{saludar}\PYG{p}{(}\PYG{n+nx}{nombre}\PYG{p}{)}\PYG{p}{\PYGZob{}}
    \PYG{n+nb}{document}\PYG{p}{.}\PYG{n+nx}{write}\PYG{p}{(}\PYG{l+s+s2}{\PYGZdq{}Hola \PYGZdq{}}
        \PYG{o}{+}\PYG{n+nx}{nombre}\PYG{o}{+}\PYG{l+s+s2}{\PYGZdq{}\PYGZlt{}br/\PYGZgt{}\PYGZdq{}}\PYG{p}{)}
\PYG{p}{\PYGZcb{}}
\PYG{k+kd}{function} \PYG{n+nx}{despedir}\PYG{p}{(}\PYG{n+nx}{nombre}\PYG{p}{)}\PYG{p}{\PYGZob{}}
    \PYG{n+nb}{document}\PYG{p}{.}\PYG{n+nx}{write}\PYG{p}{(}\PYG{l+s+s2}{\PYGZdq{}Adios \PYGZdq{}}
        \PYG{o}{+}\PYG{n+nx}{nombre}\PYG{o}{+}\PYG{l+s+s2}{\PYGZdq{}\PYGZlt{}br/\PYGZgt{}\PYGZdq{}}\PYG{p}{)}
\PYG{p}{\PYGZcb{}}
\PYG{n+nx}{saludar}\PYG{p}{(}\PYG{l+s+s2}{\PYGZdq{}Antonio\PYGZdq{}}\PYG{p}{)}
\PYG{n+nx}{despedir}\PYG{p}{(}\PYG{l+s+s2}{\PYGZdq{}Antonio\PYGZdq{}}\PYG{p}{)}
\PYG{c+cm}{/* Las funciones son valores}
\PYG{c+cm}{ * asignables*/}
\PYG{k+kd}{var} \PYG{n+nx}{f}\PYG{o}{=}\PYG{n+nx}{despedir}
\PYG{n+nx}{f}\PYG{p}{(}\PYG{l+s+s2}{\PYGZdq{}Tomas\PYGZdq{}}\PYG{p}{)}
\end{sphinxVerbatim}


\subsection{Ejercicio}
\label{\detokenize{tema4:id1}}
Crear un programa que tenga una función que calcule la desviación media de valores de un vector.

\begin{sphinxVerbatim}[commandchars=\\\{\}]
    \PYG{c+cm}{/* Función que calcula la desviacion media de}
\PYG{c+cm}{    * un vector de valores numericos*/}
\PYG{k+kd}{function} \PYG{n+nx}{calcularDesviacionMedia}\PYG{p}{(}\PYG{n+nx}{vector\PYGZus{}valores}\PYG{p}{)}\PYG{p}{\PYGZob{}}
    \PYG{k+kd}{var} \PYG{n+nx}{media}\PYG{o}{=}\PYG{n+nx}{calcularMedia}\PYG{p}{(}\PYG{n+nx}{vector\PYGZus{}valores}\PYG{p}{)}
    \PYG{k+kd}{var} \PYG{n+nx}{suma}\PYG{o}{=}\PYG{l+m+mi}{0}
    \PYG{k}{for} \PYG{p}{(}\PYG{k+kd}{var} \PYG{n+nx}{pos} \PYG{k}{in} \PYG{n+nx}{vector\PYGZus{}valores}\PYG{p}{)}\PYG{p}{\PYGZob{}}
        \PYG{n+nx}{suma}\PYG{o}{=} \PYG{n+nx}{suma} \PYG{o}{+} \PYG{n+nb}{Math}\PYG{p}{.}\PYG{n+nx}{abs} \PYG{p}{(}  \PYG{n+nx}{vector\PYGZus{}valores}\PYG{p}{[}\PYG{n+nx}{pos}\PYG{p}{]} \PYG{o}{\PYGZhy{}} \PYG{n+nx}{media}  \PYG{p}{)}
    \PYG{p}{\PYGZcb{}}
    \PYG{k}{return} \PYG{n+nx}{suma} \PYG{o}{/} \PYG{n+nx}{vector\PYGZus{}valores}\PYG{p}{.}\PYG{n+nx}{length}
\PYG{p}{\PYGZcb{}}
\end{sphinxVerbatim}


\subsection{Ejercicio}
\label{\detokenize{tema4:id2}}
Crear un programa que tenga una función que calcule la moda.

\begin{sphinxVerbatim}[commandchars=\\\{\}]
    \PYG{c+cm}{/* Este vector nos dice cuantas veces aparece un número}
\PYG{c+cm}{ * en un vector*/}
\PYG{k+kd}{function} \PYG{n+nx}{calcularFrecuencia}\PYG{p}{(}\PYG{n+nx}{numero}\PYG{p}{,} \PYG{n+nx}{vector}\PYG{p}{)}\PYG{p}{\PYGZob{}}
    \PYG{k+kd}{var} \PYG{n+nx}{num\PYGZus{}veces}\PYG{o}{=}\PYG{l+m+mi}{0}
    \PYG{k}{for} \PYG{p}{(}\PYG{k+kd}{var} \PYG{n+nx}{pos} \PYG{k}{in} \PYG{n+nx}{vector}\PYG{p}{)} \PYG{p}{\PYGZob{}}
        \PYG{k}{if} \PYG{p}{(}\PYG{n+nx}{vector}\PYG{p}{[}\PYG{n+nx}{pos}\PYG{p}{]}\PYG{o}{==}\PYG{n+nx}{numero}\PYG{p}{)} \PYG{p}{\PYGZob{}}
            \PYG{n+nx}{num\PYGZus{}veces}\PYG{o}{++}
        \PYG{p}{\PYGZcb{}}
    \PYG{p}{\PYGZcb{}}
    \PYG{k}{return} \PYG{n+nx}{num\PYGZus{}veces}
\PYG{p}{\PYGZcb{}}

\PYG{c+cm}{/* Dado un vector de números se nos devuelve la posición}
\PYG{c+cm}{ * del número mayor*/}
\PYG{k+kd}{function} \PYG{n+nx}{obtenerPosMayor}\PYG{p}{(}\PYG{n+nx}{vector\PYGZus{}valores}\PYG{p}{)}\PYG{p}{\PYGZob{}}
    \PYG{k+kd}{var} \PYG{n+nx}{posMayor}\PYG{o}{=}\PYG{l+m+mi}{0}
    \PYG{k+kd}{var} \PYG{n+nx}{numMayor}\PYG{o}{=}\PYG{n+nx}{vector\PYGZus{}valores}\PYG{p}{[}\PYG{l+m+mi}{0}\PYG{p}{]}
    \PYG{k}{for} \PYG{p}{(}\PYG{k+kd}{var} \PYG{n+nx}{pos} \PYG{k}{in} \PYG{n+nx}{vector\PYGZus{}valores}\PYG{p}{)}\PYG{p}{\PYGZob{}}
        \PYG{k}{if} \PYG{p}{(}\PYG{n+nx}{vector\PYGZus{}valores}\PYG{p}{[}\PYG{n+nx}{pos}\PYG{p}{]}\PYG{o}{\PYGZgt{}}\PYG{n+nx}{numMayor}\PYG{p}{)} \PYG{p}{\PYGZob{}}
            \PYG{n+nx}{numMayor}\PYG{o}{=}\PYG{n+nx}{vector\PYGZus{}valores}\PYG{p}{[}\PYG{n+nx}{pos}\PYG{p}{]}
            \PYG{n+nx}{posMayor}\PYG{o}{=}\PYG{n+nx}{pos}
        \PYG{p}{\PYGZcb{}}
    \PYG{p}{\PYGZcb{}}
    \PYG{k}{return} \PYG{n+nx}{posMayor}
\PYG{p}{\PYGZcb{}}
\PYG{c+cm}{/* Función que devuelve el número \PYGZdq{}moda\PYGZdq{} de un vector*/}
\PYG{k+kd}{function} \PYG{n+nx}{obtenerModa}\PYG{p}{(}\PYG{n+nx}{vector\PYGZus{}valores}\PYG{p}{)}\PYG{p}{\PYGZob{}}
    \PYG{k+kd}{var} \PYG{n+nx}{frecuencias}\PYG{o}{=}\PYG{k}{new} \PYG{n+nb}{Array}\PYG{p}{(}\PYG{n+nx}{vector\PYGZus{}valores}\PYG{p}{.}\PYG{n+nx}{length}\PYG{p}{)}
    \PYG{k}{for} \PYG{p}{(}\PYG{k+kd}{var} \PYG{n+nx}{pos} \PYG{k}{in} \PYG{n+nx}{vector\PYGZus{}valores}\PYG{p}{)}\PYG{p}{\PYGZob{}}
         \PYG{k+kd}{var} \PYG{n+nx}{numero}\PYG{o}{=}\PYG{n+nx}{vector\PYGZus{}valores}\PYG{p}{[}\PYG{n+nx}{pos}\PYG{p}{]}
         \PYG{n+nx}{frecuencias}\PYG{p}{[}\PYG{n+nx}{pos}\PYG{p}{]}\PYG{o}{=}\PYG{n+nx}{calcularFrecuencia}\PYG{p}{(}\PYG{n+nx}{numero}\PYG{p}{,} \PYG{n+nx}{vector\PYGZus{}valores}\PYG{p}{)}
    \PYG{p}{\PYGZcb{}}
    \PYG{k+kd}{var} \PYG{n+nx}{posModa}\PYG{o}{=}\PYG{n+nx}{obtenerPosMayor}\PYG{p}{(}\PYG{n+nx}{frecuencias}\PYG{p}{)}
    \PYG{k}{return} \PYG{n+nx}{vector\PYGZus{}valores}\PYG{p}{[}\PYG{n+nx}{posModa}\PYG{p}{]}

\PYG{p}{\PYGZcb{}}
\PYG{k+kd}{var} \PYG{n+nx}{vector}\PYG{o}{=}\PYG{k}{new} \PYG{n+nb}{Array}\PYG{p}{(}\PYG{l+m+mi}{4}\PYG{p}{)}
\PYG{n+nx}{vector}\PYG{p}{[}\PYG{l+m+mi}{0}\PYG{p}{]}\PYG{o}{=}\PYG{l+m+mi}{7}
\PYG{n+nx}{vector}\PYG{p}{[}\PYG{l+m+mi}{1}\PYG{p}{]}\PYG{o}{=}\PYG{l+m+mi}{7}
\PYG{n+nx}{vector}\PYG{p}{[}\PYG{l+m+mi}{2}\PYG{p}{]}\PYG{o}{=}\PYG{l+m+mi}{7}
\PYG{n+nx}{vector}\PYG{p}{[}\PYG{l+m+mi}{3}\PYG{p}{]}\PYG{o}{=}\PYG{l+m+mi}{5}
    \PYG{k+kd}{var} \PYG{n+nx}{moda}\PYG{o}{=}\PYG{n+nx}{obtenerModa}\PYG{p}{(}\PYG{n+nx}{vector}\PYG{p}{)}
    \PYG{n+nb}{document}\PYG{p}{.}\PYG{n+nx}{write}\PYG{p}{(}\PYG{l+s+s2}{\PYGZdq{}\PYGZlt{}br/\PYGZgt{}La moda es:\PYGZdq{}}\PYG{o}{+}\PYG{n+nx}{moda}\PYG{p}{)}
\end{sphinxVerbatim}


\section{Programación OO}
\label{\detokenize{tema4:programacion-oo}}
Se ha dicho anteriormente que Javascript es «basado en objetos» y no «orientado a objetos», es decir la POO es optativa. No por ello es menos potente.

En primer lugar, es posible crear objetos sin crear clases.

\begin{sphinxVerbatim}[commandchars=\\\{\}]
    \PYG{k+kd}{var} \PYG{n+nx}{empleado}\PYG{o}{=}\PYG{p}{\PYGZob{}}
    \PYG{n+nx}{nombre}\PYG{o}{:}\PYG{l+s+s2}{\PYGZdq{}Pepe Perez\PYGZdq{}}\PYG{p}{,}
    \PYG{n+nx}{edad}\PYG{o}{:}\PYG{l+m+mi}{27}\PYG{p}{,}
    \PYG{n+nx}{fijo}\PYG{o}{:}\PYG{k+kc}{true}\PYG{p}{,}
    \PYG{n+nx}{estaJubilado}\PYG{o}{:}\PYG{k+kd}{function} \PYG{p}{(}\PYG{p}{)}\PYG{p}{\PYGZob{}}
        \PYG{k}{if} \PYG{p}{(}\PYG{k}{this}\PYG{p}{.}\PYG{n+nx}{edad}\PYG{o}{\PYGZgt{}}\PYG{l+m+mi}{65}\PYG{p}{)} \PYG{p}{\PYGZob{}}
            \PYG{k}{return} \PYG{k+kc}{true}
        \PYG{p}{\PYGZcb{}} \PYG{k}{else} \PYG{p}{\PYGZob{}}
            \PYG{k}{return} \PYG{k+kc}{false}
        \PYG{p}{\PYGZcb{}}

    \PYG{p}{\PYGZcb{}}
\PYG{p}{\PYGZcb{}}
\PYG{n+nb}{document}\PYG{p}{.}\PYG{n+nx}{write}\PYG{p}{(}\PYG{l+s+s2}{\PYGZdq{}\PYGZlt{}br/\PYGZgt{}El nombre es:\PYGZdq{}}\PYG{o}{+}\PYG{n+nx}{empleado}\PYG{p}{.}\PYG{n+nx}{nombre}\PYG{p}{)}
\PYG{n+nb}{document}\PYG{p}{.}\PYG{n+nx}{write}\PYG{p}{(}\PYG{l+s+s2}{\PYGZdq{}\PYGZlt{}br/\PYGZgt{}¿Jubilado?\PYGZdq{}} \PYG{o}{+} \PYG{n+nx}{empleado}\PYG{p}{.}\PYG{n+nx}{estaJubilado}\PYG{p}{(}\PYG{p}{)} \PYG{p}{)}
\end{sphinxVerbatim}


\subsection{Ejercicio}
\label{\detokenize{tema4:id3}}
Añadir un método llamado \sphinxcode{nivelExperiencia} que nos diga una de estas cosas:
\begin{itemize}
\item {} 
Nos debe devolver «junior» si la edad está entre 18 y 25

\item {} 
Nos debe devolver «asociado» si la edad está entre 26 y 45

\item {} 
Nos debe devolver «senior» si la edad está entre 46 y 60

\item {} 
Nos debe devolver «experto» si la edad está entre 61 y 65

\item {} 
Nos debe devolver «no aplicable» si la edad es mayor de 65

\end{itemize}

\begin{sphinxVerbatim}[commandchars=\\\{\}]
    \PYG{k+kd}{var} \PYG{n+nx}{empleado}\PYG{o}{=}\PYG{p}{\PYGZob{}}
    \PYG{n+nx}{nombre}\PYG{o}{:}\PYG{l+s+s2}{\PYGZdq{}Pepe Perez\PYGZdq{}}\PYG{p}{,}
    \PYG{n+nx}{edad}\PYG{o}{:}\PYG{l+m+mi}{27}\PYG{p}{,}
    \PYG{n+nx}{fijo}\PYG{o}{:}\PYG{k+kc}{true}\PYG{p}{,}
    \PYG{n+nx}{estaJubilado}\PYG{o}{:}\PYG{k+kd}{function} \PYG{p}{(}\PYG{p}{)}\PYG{p}{\PYGZob{}}
        \PYG{k}{if} \PYG{p}{(}\PYG{k}{this}\PYG{p}{.}\PYG{n+nx}{edad}\PYG{o}{\PYGZgt{}}\PYG{l+m+mi}{65}\PYG{p}{)} \PYG{p}{\PYGZob{}}
            \PYG{k}{return} \PYG{k+kc}{true}
        \PYG{p}{\PYGZcb{}} \PYG{k}{else} \PYG{p}{\PYGZob{}}
            \PYG{k}{return} \PYG{k+kc}{false}
        \PYG{p}{\PYGZcb{}}
    \PYG{p}{\PYGZcb{}}\PYG{p}{,}
    \PYG{n+nx}{nivelExperiencia}\PYG{o}{:}\PYG{k+kd}{function}\PYG{p}{(}\PYG{p}{)}\PYG{p}{\PYGZob{}}
        \PYG{k}{if} \PYG{p}{(} \PYG{p}{(}\PYG{k}{this}\PYG{p}{.}\PYG{n+nx}{edad}\PYG{o}{\PYGZgt{}}\PYG{l+m+mi}{18}\PYG{p}{)}  \PYG{o}{\PYGZam{}\PYGZam{}} \PYG{p}{(}\PYG{k}{this}\PYG{p}{.}\PYG{n+nx}{edad}\PYG{o}{\PYGZlt{}=}\PYG{l+m+mi}{25}\PYG{p}{)} \PYG{p}{)}\PYG{p}{\PYGZob{}}
            \PYG{k}{return} \PYG{l+s+s2}{\PYGZdq{}junior\PYGZdq{}}
        \PYG{p}{\PYGZcb{}}
        \PYG{k}{if} \PYG{p}{(} \PYG{p}{(}\PYG{k}{this}\PYG{p}{.}\PYG{n+nx}{edad}\PYG{o}{\PYGZgt{}=}\PYG{l+m+mi}{26}\PYG{p}{)}  \PYG{o}{\PYGZam{}\PYGZam{}} \PYG{p}{(}\PYG{k}{this}\PYG{p}{.}\PYG{n+nx}{edad}\PYG{o}{\PYGZlt{}=}\PYG{l+m+mi}{45}\PYG{p}{)} \PYG{p}{)}\PYG{p}{\PYGZob{}}
            \PYG{k}{return} \PYG{l+s+s2}{\PYGZdq{}asociado\PYGZdq{}}
        \PYG{p}{\PYGZcb{}}
    \PYG{p}{\PYGZcb{}}
\PYG{p}{\PYGZcb{}}
\end{sphinxVerbatim}


\subsection{Ejercicio}
\label{\detokenize{tema4:id4}}
Crear una clase GestorVectores que tenga los principales métodos estadísticos vistos hasta ahora: media, desviación media, mediana y moda.

\begin{sphinxVerbatim}[commandchars=\\\{\}]
\PYG{n+nx}{gestor\PYGZus{}vectores}\PYG{o}{=}\PYG{p}{\PYGZob{}}
        \PYG{n+nx}{vector\PYGZus{}numeros}\PYG{o}{:}\PYG{k}{new} \PYG{n+nb}{Array}\PYG{p}{(}\PYG{p}{)}\PYG{p}{,}
        \PYG{n+nx}{setDatos}\PYG{o}{:}\PYG{k+kd}{function}\PYG{p}{(}\PYG{n+nx}{vector}\PYG{p}{)}\PYG{p}{\PYGZob{}}
                \PYG{k}{this}\PYG{p}{.}\PYG{n+nx}{vector\PYGZus{}numeros}\PYG{o}{=}\PYG{n+nx}{vector}

        \PYG{p}{\PYGZcb{}}
        \PYG{p}{,} \PYG{c+c1}{//Importante: separar métodos y atributos con ,}
        \PYG{n+nx}{getMedia}\PYG{o}{:}\PYG{k+kd}{function}\PYG{p}{(}\PYG{p}{)}\PYG{p}{\PYGZob{}}
                \PYG{k+kd}{var} \PYG{n+nx}{suma}\PYG{o}{=}\PYG{l+m+mi}{0}
                \PYG{k+kd}{var} \PYG{n+nx}{media}\PYG{o}{=}\PYG{l+m+mi}{0}
                \PYG{k}{for} \PYG{p}{(}\PYG{n+nx}{pos} \PYG{k}{in} \PYG{k}{this}\PYG{p}{.}\PYG{n+nx}{vector\PYGZus{}numeros}\PYG{p}{)} \PYG{p}{\PYGZob{}}
                        \PYG{n+nx}{suma}\PYG{o}{=}\PYG{n+nx}{suma} \PYG{o}{+} \PYG{k}{this}\PYG{p}{.}\PYG{n+nx}{vector\PYGZus{}numeros}\PYG{p}{[}\PYG{n+nx}{pos}\PYG{p}{]}
                \PYG{p}{\PYGZcb{}}
                \PYG{n+nx}{media}\PYG{o}{=}\PYG{n+nx}{suma} \PYG{o}{/} \PYG{k}{this}\PYG{p}{.}\PYG{n+nx}{vector\PYGZus{}numeros}\PYG{p}{.}\PYG{n+nx}{length}
                \PYG{k}{return} \PYG{n+nx}{media}
        \PYG{p}{\PYGZcb{}}
        \PYG{p}{,}
        \PYG{n+nx}{getModa}\PYG{o}{:}\PYG{k+kd}{function}\PYG{p}{(}\PYG{p}{)}\PYG{p}{\PYGZob{}}

        \PYG{p}{\PYGZcb{}}
        \PYG{p}{,}
        \PYG{n+nx}{getMediana}\PYG{o}{:}\PYG{k+kd}{function}\PYG{p}{(}\PYG{p}{)}\PYG{p}{\PYGZob{}}
                \PYG{k}{this}\PYG{p}{.}\PYG{n+nx}{vector\PYGZus{}numeros}\PYG{p}{.}\PYG{n+nx}{sort}\PYG{p}{(}\PYG{p}{)}
        \PYG{p}{\PYGZcb{}}
\PYG{p}{\PYGZcb{}}


\PYG{k+kd}{var} \PYG{n+nx}{vector\PYGZus{}prueba}\PYG{o}{=}\PYG{k}{new} \PYG{n+nb}{Array}\PYG{p}{(}\PYG{l+m+mi}{3}\PYG{p}{)}
\PYG{n+nx}{vector\PYGZus{}prueba}\PYG{p}{[}\PYG{l+m+mi}{0}\PYG{p}{]}\PYG{o}{=}\PYG{l+m+mi}{5}
\PYG{n+nx}{vector\PYGZus{}prueba}\PYG{p}{[}\PYG{l+m+mi}{1}\PYG{p}{]}\PYG{o}{=}\PYG{l+m+mi}{10}
\PYG{n+nx}{vector\PYGZus{}prueba}\PYG{p}{[}\PYG{l+m+mi}{2}\PYG{p}{]}\PYG{o}{=}\PYG{l+m+mi}{8}
\PYG{n+nx}{gestor\PYGZus{}vectores}\PYG{p}{.}\PYG{n+nx}{setDatos} \PYG{p}{(} \PYG{n+nx}{vector\PYGZus{}prueba} \PYG{p}{)}
\PYG{k+kd}{var} \PYG{n+nx}{media}\PYG{o}{=}\PYG{n+nx}{gestor\PYGZus{}vectores}\PYG{p}{.}\PYG{n+nx}{getMedia}\PYG{p}{(}\PYG{p}{)}
\PYG{n+nb}{document}\PYG{p}{.}\PYG{n+nx}{write} \PYG{p}{(}\PYG{l+s+s2}{\PYGZdq{}La media es:\PYGZdq{}}\PYG{o}{+}\PYG{n+nx}{media}\PYG{p}{)}
\end{sphinxVerbatim}


\section{Programación con JQuery}
\label{\detokenize{tema4:programacion-con-jquery}}
Existen muchos navegadores que a veces muestran pequeñas diferencias entre ellos. Para evitar problemas los programadores tenían que incluir muchos código para comprobar qué navegador ejecutaba su JS y en función de eso actuar. Para resolver estas diferencias John Resig creó JQuery.


\subsection{Inicio}
\label{\detokenize{tema4:inicio}}
A partir de ahora todos nuestros archivos HTML tendrán que cargar al comienzo la biblioteca JQuery con una etiqueta como esta:

\begin{sphinxVerbatim}[commandchars=\\\{\}]
\PYG{p}{\PYGZlt{}}\PYG{n+nt}{script} \PYG{n+na}{src}\PYG{o}{=}\PYG{l+s}{\PYGZdq{}jquery.js\PYGZdq{}} \PYG{n+na}{language}\PYG{o}{=}\PYG{l+s}{\PYGZdq{}Javascript\PYGZdq{}}\PYG{p}{\PYGZgt{}}
\PYG{p}{\PYGZlt{}}\PYG{p}{/}\PYG{n+nt}{script}\PYG{p}{\PYGZgt{}}
\PYG{p}{\PYGZlt{}}\PYG{n+nt}{script} \PYG{n+na}{src}\PYG{o}{=}\PYG{l+s}{\PYGZdq{}nuestroprograma.js\PYGZdq{}} \PYG{n+na}{language}\PYG{o}{=}\PYG{l+s}{\PYGZdq{}Javascript\PYGZdq{}}\PYG{p}{\PYGZgt{}}
\PYG{p}{\PYGZlt{}}\PYG{p}{/}\PYG{n+nt}{script}\PYG{p}{\PYGZgt{}}
\end{sphinxVerbatim}

\sphinxstylestrong{El orden es importante}


\subsection{La función \$}
\label{\detokenize{tema4:la-funcion}}
La función \$ selecciona elementos de la página para que podamos hacer cosas con ellos. Es la función más utilizada de JQuery y veremos que podemos pedir que nos seleccione grupos de cosas de forma muy sencilla.

La función \$ devuelve siempre objetos. Los atributos y métodos de esos objetos los iremos aprendiendo poco a poco.

En general, antes de poder procesar un elemento, deberemos seleccionarlo utilizando los mismos selectores que en CSS.


\subsection{Gestión de eventos}
\label{\detokenize{tema4:gestion-de-eventos}}
Utilizando \sphinxcode{click} podemos indicar a la biblioteca que queremos que cuando alguien haga click en un elemento se ejecute una cierta función. El siguiente código HTML y JS ilustra una posibilidad

\begin{sphinxVerbatim}[commandchars=\\\{\}]
\PYG{c+cp}{\PYGZlt{}!DOCTYPE html\PYGZgt{}}
\PYG{p}{\PYGZlt{}}\PYG{n+nt}{html}\PYG{p}{\PYGZgt{}}
\PYG{p}{\PYGZlt{}}\PYG{n+nt}{head}\PYG{p}{\PYGZgt{}}
        \PYG{p}{\PYGZlt{}}\PYG{n+nt}{script} \PYG{n+na}{src}\PYG{o}{=}\PYG{l+s}{\PYGZdq{}jquery.js\PYGZdq{}}\PYG{p}{\PYGZgt{}}\PYG{p}{\PYGZlt{}}\PYG{p}{/}\PYG{n+nt}{script}\PYG{p}{\PYGZgt{}}
        \PYG{p}{\PYGZlt{}}\PYG{n+nt}{script} \PYG{n+na}{src}\PYG{o}{=}\PYG{l+s}{\PYGZdq{}ejemplo.js\PYGZdq{}}\PYG{p}{\PYGZgt{}}\PYG{p}{\PYGZlt{}}\PYG{p}{/}\PYG{n+nt}{script}\PYG{p}{\PYGZgt{}}
        \PYG{p}{\PYGZlt{}}\PYG{n+nt}{title}\PYG{p}{\PYGZgt{}}Ejemplos\PYG{p}{\PYGZlt{}}\PYG{p}{/}\PYG{n+nt}{title}\PYG{p}{\PYGZgt{}}
        \PYG{p}{\PYGZlt{}}\PYG{n+nt}{style}\PYG{p}{\PYGZgt{}}
                \PYG{n+nt}{div}\PYG{p}{\PYGZsh{}}\PYG{n+nn}{texto}\PYG{p}{\PYGZob{}}
                        \PYG{k}{background\PYGZhy{}color}\PYG{p}{:}\PYG{k+kc}{yellow}\PYG{p}{;}
                \PYG{p}{\PYGZcb{}}
        \PYG{p}{\PYGZlt{}}\PYG{p}{/}\PYG{n+nt}{style}\PYG{p}{\PYGZgt{}}
\PYG{p}{\PYGZlt{}}\PYG{p}{/}\PYG{n+nt}{head}\PYG{p}{\PYGZgt{}}

\PYG{p}{\PYGZlt{}}\PYG{n+nt}{body}\PYG{p}{\PYGZgt{}}
\PYG{p}{\PYGZlt{}}\PYG{n+nt}{form}\PYG{p}{\PYGZgt{}}
        \PYG{p}{\PYGZlt{}}\PYG{n+nt}{input} \PYG{n+na}{type}\PYG{o}{=}\PYG{l+s}{\PYGZdq{}button\PYGZdq{}} \PYG{n+na}{value}\PYG{o}{=}\PYG{l+s}{\PYGZdq{}fadeOut\PYGZdq{}} \PYG{n+na}{id}\PYG{o}{=}\PYG{l+s}{\PYGZdq{}botonizq\PYGZdq{}}\PYG{p}{\PYGZgt{}}
        \PYG{p}{\PYGZlt{}}\PYG{n+nt}{input} \PYG{n+na}{type}\PYG{o}{=}\PYG{l+s}{\PYGZdq{}button\PYGZdq{}} \PYG{n+na}{value}\PYG{o}{=}\PYG{l+s}{\PYGZdq{}fadeIn\PYGZdq{}} \PYG{n+na}{id}\PYG{o}{=}\PYG{l+s}{\PYGZdq{}botonder\PYGZdq{}}\PYG{p}{\PYGZgt{}}
\PYG{p}{\PYGZlt{}}\PYG{p}{/}\PYG{n+nt}{form}\PYG{p}{\PYGZgt{}}

\PYG{p}{\PYGZlt{}}\PYG{n+nt}{div} \PYG{n+na}{id}\PYG{o}{=}\PYG{l+s}{\PYGZdq{}texto\PYGZdq{}}\PYG{p}{\PYGZgt{}}
        Texto texto texto
\PYG{p}{\PYGZlt{}}\PYG{p}{/}\PYG{n+nt}{div}\PYG{p}{\PYGZgt{}}

\PYG{p}{\PYGZlt{}}\PYG{p}{/}\PYG{n+nt}{body}\PYG{p}{\PYGZgt{}}
\PYG{p}{\PYGZlt{}}\PYG{p}{/}\PYG{n+nt}{html}\PYG{p}{\PYGZgt{}}
\end{sphinxVerbatim}

El código Javascript asociado al HTML anterior es este.

\begin{sphinxVerbatim}[commandchars=\\\{\}]
\PYG{c+cm}{/* Esperaremos hasta que el documento esté cargado y listo}
\PYG{c+cm}{ * para ser procesado por nuestro programa*/}

\PYG{k+kd}{var} \PYG{n+nx}{obj\PYGZus{}documento} \PYG{o}{=} \PYG{n+nx}{\PYGZdl{}}\PYG{p}{(}\PYG{n+nb}{document}\PYG{p}{)}

\PYG{c+cm}{/* Cuando esté cargado ejecutaremos la función cuyo nombre aparezca aquí*/}
\PYG{n+nx}{obj\PYGZus{}documento}\PYG{p}{.}\PYG{n+nx}{ready}\PYG{p}{(}\PYG{n+nx}{inicio}\PYG{p}{)}

\PYG{c+c1}{//* Error gravísimo*/}
\PYG{c+c1}{//obj\PYGZus{}documento.ready( inicio() )}

\PYG{k+kd}{function} \PYG{n+nx}{inicio}\PYG{p}{(}\PYG{p}{)}\PYG{p}{\PYGZob{}}
        \PYG{k+kd}{var} \PYG{n+nx}{obj\PYGZus{}izq}\PYG{o}{=}\PYG{n+nx}{\PYGZdl{}}\PYG{p}{(}\PYG{l+s+s2}{\PYGZdq{}\PYGZsh{}botonizq\PYGZdq{}}\PYG{p}{)}
        \PYG{n+nx}{obj\PYGZus{}izq}\PYG{p}{.}\PYG{n+nx}{click} \PYG{p}{(} \PYG{n+nx}{fn\PYGZus{}click\PYGZus{}izq} \PYG{p}{)}
        \PYG{k+kd}{var} \PYG{n+nx}{obj\PYGZus{}der}\PYG{o}{=}\PYG{n+nx}{\PYGZdl{}}\PYG{p}{(}\PYG{l+s+s2}{\PYGZdq{}\PYGZsh{}botonder\PYGZdq{}}\PYG{p}{)}
        \PYG{n+nx}{obj\PYGZus{}der}\PYG{p}{.}\PYG{n+nx}{click} \PYG{p}{(} \PYG{n+nx}{fn\PYGZus{}click\PYGZus{}der} \PYG{p}{)}
\PYG{p}{\PYGZcb{}}

\PYG{k+kd}{function} \PYG{n+nx}{fn\PYGZus{}click\PYGZus{}izq}\PYG{p}{(}\PYG{p}{)}\PYG{p}{\PYGZob{}}
        \PYG{k+kd}{var} \PYG{n+nx}{obj\PYGZus{}div}\PYG{o}{=}\PYG{n+nx}{\PYGZdl{}}\PYG{p}{(}\PYG{l+s+s2}{\PYGZdq{}\PYGZsh{}texto\PYGZdq{}}\PYG{p}{)}
        \PYG{n+nx}{obj\PYGZus{}div}\PYG{p}{.}\PYG{n+nx}{fadeOut}\PYG{p}{(}\PYG{p}{)}
\PYG{p}{\PYGZcb{}}

\PYG{k+kd}{function} \PYG{n+nx}{fn\PYGZus{}click\PYGZus{}der}\PYG{p}{(}\PYG{p}{)}\PYG{p}{\PYGZob{}}
        \PYG{k+kd}{var} \PYG{n+nx}{obj\PYGZus{}div}\PYG{o}{=}\PYG{n+nx}{\PYGZdl{}}\PYG{p}{(}\PYG{l+s+s2}{\PYGZdq{}\PYGZsh{}texto\PYGZdq{}}\PYG{p}{)}
        \PYG{n+nx}{obj\PYGZus{}div}\PYG{p}{.}\PYG{n+nx}{fadeIn}\PYG{p}{(}\PYG{p}{)}
\PYG{p}{\PYGZcb{}}
\end{sphinxVerbatim}


\subsubsection{Solución HTML (párrafos)}
\label{\detokenize{tema4:solucion-html-parrafos}}
\begin{sphinxVerbatim}[commandchars=\\\{\}]
\PYG{p}{\PYGZlt{}}\PYG{n+nt}{div} \PYG{n+na}{data\PYGZhy{}role}\PYG{o}{=}\PYG{l+s}{\PYGZdq{}content\PYGZdq{}}\PYG{p}{\PYGZgt{}}
    \PYG{p}{\PYGZlt{}}\PYG{n+nt}{div} \PYG{n+na}{class}\PYG{o}{=}\PYG{l+s}{\PYGZdq{}ui\PYGZhy{}grid\PYGZhy{}c\PYGZdq{}}\PYG{p}{\PYGZgt{}}
        \PYG{p}{\PYGZlt{}}\PYG{n+nt}{div} \PYG{n+na}{class}\PYG{o}{=}\PYG{l+s}{\PYGZdq{}ui\PYGZhy{}block\PYGZhy{}a\PYGZdq{}}\PYG{p}{\PYGZgt{}}
            \PYG{p}{\PYGZlt{}}\PYG{n+nt}{input} \PYG{n+na}{type}\PYG{o}{=}\PYG{l+s}{\PYGZdq{}submit\PYGZdq{}}
                   \PYG{n+na}{id}\PYG{o}{=}\PYG{l+s}{\PYGZdq{}mostrar\PYGZus{}pares\PYGZdq{}}
                   \PYG{n+na}{value}\PYG{o}{=}\PYG{l+s}{\PYGZdq{}Mostrar pares\PYGZdq{}}\PYG{p}{\PYGZgt{}}
        \PYG{p}{\PYGZlt{}}\PYG{p}{/}\PYG{n+nt}{div}\PYG{p}{\PYGZgt{}}
        \PYG{p}{\PYGZlt{}}\PYG{n+nt}{div} \PYG{n+na}{class}\PYG{o}{=}\PYG{l+s}{\PYGZdq{}ui\PYGZhy{}block\PYGZhy{}b\PYGZdq{}}\PYG{p}{\PYGZgt{}}
            \PYG{p}{\PYGZlt{}}\PYG{n+nt}{input} \PYG{n+na}{type}\PYG{o}{=}\PYG{l+s}{\PYGZdq{}submit\PYGZdq{}}
                   \PYG{n+na}{id}\PYG{o}{=}\PYG{l+s}{\PYGZdq{}ocultar\PYGZus{}pares\PYGZdq{}}
                   \PYG{n+na}{value}\PYG{o}{=}\PYG{l+s}{\PYGZdq{}Ocultar pares\PYGZdq{}}\PYG{p}{\PYGZgt{}}
        \PYG{p}{\PYGZlt{}}\PYG{p}{/}\PYG{n+nt}{div}\PYG{p}{\PYGZgt{}}
        \PYG{p}{\PYGZlt{}}\PYG{n+nt}{div} \PYG{n+na}{class}\PYG{o}{=}\PYG{l+s}{\PYGZdq{}ui\PYGZhy{}block\PYGZhy{}c\PYGZdq{}}\PYG{p}{\PYGZgt{}}
            \PYG{p}{\PYGZlt{}}\PYG{n+nt}{input} \PYG{n+na}{type}\PYG{o}{=}\PYG{l+s}{\PYGZdq{}submit\PYGZdq{}}
                   \PYG{n+na}{id}\PYG{o}{=}\PYG{l+s}{\PYGZdq{}mostrar\PYGZus{}impares\PYGZdq{}}
                   \PYG{n+na}{value}\PYG{o}{=}\PYG{l+s}{\PYGZdq{}Mostrar impares\PYGZdq{}}\PYG{p}{\PYGZgt{}}
        \PYG{p}{\PYGZlt{}}\PYG{p}{/}\PYG{n+nt}{div}\PYG{p}{\PYGZgt{}}
        \PYG{p}{\PYGZlt{}}\PYG{n+nt}{div} \PYG{n+na}{class}\PYG{o}{=}\PYG{l+s}{\PYGZdq{}ui\PYGZhy{}block\PYGZhy{}d\PYGZdq{}}\PYG{p}{\PYGZgt{}}
            \PYG{p}{\PYGZlt{}}\PYG{n+nt}{input} \PYG{n+na}{type}\PYG{o}{=}\PYG{l+s}{\PYGZdq{}submit\PYGZdq{}}
                   \PYG{n+na}{id}\PYG{o}{=}\PYG{l+s}{\PYGZdq{}ocultar\PYGZus{}impares\PYGZdq{}}
                   \PYG{n+na}{value}\PYG{o}{=}\PYG{l+s}{\PYGZdq{}Ocultar impares\PYGZdq{}}\PYG{p}{\PYGZgt{}}
        \PYG{p}{\PYGZlt{}}\PYG{p}{/}\PYG{n+nt}{div}\PYG{p}{\PYGZgt{}}
    \PYG{p}{\PYGZlt{}}\PYG{p}{/}\PYG{n+nt}{div}\PYG{p}{\PYGZgt{}}
    \PYG{p}{\PYGZlt{}}\PYG{n+nt}{p} \PYG{n+na}{class}\PYG{o}{=}\PYG{l+s}{\PYGZdq{}p\PYGZus{}impar\PYGZdq{}}\PYG{p}{\PYGZgt{}}
        Soy un párrafo impar
    \PYG{p}{\PYGZlt{}}\PYG{p}{/}\PYG{n+nt}{p}\PYG{p}{\PYGZgt{}}
    \PYG{p}{\PYGZlt{}}\PYG{n+nt}{p} \PYG{n+na}{class}\PYG{o}{=}\PYG{l+s}{\PYGZdq{}p\PYGZus{}par\PYGZdq{}}\PYG{p}{\PYGZgt{}}
        Soy un párrafo par
    \PYG{p}{\PYGZlt{}}\PYG{p}{/}\PYG{n+nt}{p}\PYG{p}{\PYGZgt{}}
    \PYG{p}{\PYGZlt{}}\PYG{n+nt}{p} \PYG{n+na}{class}\PYG{o}{=}\PYG{l+s}{\PYGZdq{}p\PYGZus{}impar\PYGZdq{}}\PYG{p}{\PYGZgt{}}
        Soy un párrafo impar
    \PYG{p}{\PYGZlt{}}\PYG{p}{/}\PYG{n+nt}{p}\PYG{p}{\PYGZgt{}}
    \PYG{p}{\PYGZlt{}}\PYG{n+nt}{p} \PYG{n+na}{class}\PYG{o}{=}\PYG{l+s}{\PYGZdq{}p\PYGZus{}par\PYGZdq{}}\PYG{p}{\PYGZgt{}}
        Soy un párrafo par
    \PYG{p}{\PYGZlt{}}\PYG{p}{/}\PYG{n+nt}{p}\PYG{p}{\PYGZgt{}}
    \PYG{p}{\PYGZlt{}}\PYG{n+nt}{p} \PYG{n+na}{class}\PYG{o}{=}\PYG{l+s}{\PYGZdq{}p\PYGZus{}impar\PYGZdq{}}\PYG{p}{\PYGZgt{}}
        Soy un párrafo impar
    \PYG{p}{\PYGZlt{}}\PYG{p}{/}\PYG{n+nt}{p}\PYG{p}{\PYGZgt{}}
    \PYG{p}{\PYGZlt{}}\PYG{n+nt}{p} \PYG{n+na}{class}\PYG{o}{=}\PYG{l+s}{\PYGZdq{}p\PYGZus{}par\PYGZdq{}}\PYG{p}{\PYGZgt{}}
        Soy un párrafo par
    \PYG{p}{\PYGZlt{}}\PYG{p}{/}\PYG{n+nt}{p}\PYG{p}{\PYGZgt{}}
\PYG{p}{\PYGZlt{}}\PYG{p}{/}\PYG{n+nt}{div}\PYG{p}{\PYGZgt{}}
\end{sphinxVerbatim}


\subsubsection{Solución Javascript (párrafos)}
\label{\detokenize{tema4:solucion-javascript-parrafos}}
\begin{sphinxVerbatim}[commandchars=\\\{\}]
\PYG{n+nx}{\PYGZdl{}}\PYG{p}{(}\PYG{n+nb}{document}\PYG{p}{)}\PYG{p}{.}\PYG{n+nx}{ready}\PYG{p}{(}\PYG{n+nx}{main}\PYG{p}{)}

\PYG{k+kd}{function} \PYG{n+nx}{mostrar\PYGZus{}pares}\PYG{p}{(}\PYG{p}{)}\PYG{p}{\PYGZob{}}
        \PYG{k+kd}{var} \PYG{n+nx}{objetos}\PYG{o}{=}\PYG{n+nx}{\PYGZdl{}}\PYG{p}{(}\PYG{l+s+s2}{\PYGZdq{}.p\PYGZus{}par\PYGZdq{}}\PYG{p}{)}
        \PYG{n+nx}{objetos}\PYG{p}{.}\PYG{n+nx}{slideDown}\PYG{p}{(}\PYG{p}{)}
\PYG{p}{\PYGZcb{}}
\PYG{k+kd}{function} \PYG{n+nx}{mostrar\PYGZus{}impares}\PYG{p}{(}\PYG{p}{)}\PYG{p}{\PYGZob{}}
        \PYG{k+kd}{var} \PYG{n+nx}{objetos}\PYG{o}{=}\PYG{n+nx}{\PYGZdl{}}\PYG{p}{(}\PYG{l+s+s2}{\PYGZdq{}.p\PYGZus{}impar\PYGZdq{}}\PYG{p}{)}
        \PYG{n+nx}{objetos}\PYG{p}{.}\PYG{n+nx}{slideDown}\PYG{p}{(}\PYG{p}{)}
\PYG{p}{\PYGZcb{}}
\PYG{k+kd}{function} \PYG{n+nx}{ocultar\PYGZus{}pares}\PYG{p}{(}\PYG{p}{)}\PYG{p}{\PYGZob{}}
        \PYG{k+kd}{var} \PYG{n+nx}{objetos}\PYG{o}{=}\PYG{n+nx}{\PYGZdl{}}\PYG{p}{(}\PYG{l+s+s2}{\PYGZdq{}.p\PYGZus{}par\PYGZdq{}}\PYG{p}{)}
        \PYG{n+nx}{objetos}\PYG{p}{.}\PYG{n+nx}{slideUp}\PYG{p}{(}\PYG{p}{)}
\PYG{p}{\PYGZcb{}}
\PYG{k+kd}{function} \PYG{n+nx}{ocultar\PYGZus{}impares}\PYG{p}{(}\PYG{p}{)}\PYG{p}{\PYGZob{}}
        \PYG{k+kd}{var} \PYG{n+nx}{objetos}\PYG{o}{=}\PYG{n+nx}{\PYGZdl{}}\PYG{p}{(}\PYG{l+s+s2}{\PYGZdq{}.p\PYGZus{}impar\PYGZdq{}}\PYG{p}{)}
        \PYG{n+nx}{objetos}\PYG{p}{.}\PYG{n+nx}{slideUp}\PYG{p}{(}\PYG{p}{)}
\PYG{p}{\PYGZcb{}}


\PYG{k+kd}{function} \PYG{n+nx}{main}\PYG{p}{(}\PYG{p}{)}\PYG{p}{\PYGZob{}}

        \PYG{n+nx}{\PYGZdl{}}\PYG{p}{(}\PYG{l+s+s2}{\PYGZdq{}\PYGZsh{}mostrar\PYGZus{}pares\PYGZdq{}}\PYG{p}{)}\PYG{p}{.}\PYG{n+nx}{click}\PYG{p}{(}\PYG{n+nx}{mostrar\PYGZus{}pares}\PYG{p}{)}
        \PYG{n+nx}{\PYGZdl{}}\PYG{p}{(}\PYG{l+s+s2}{\PYGZdq{}\PYGZsh{}mostrar\PYGZus{}impares\PYGZdq{}}\PYG{p}{)}\PYG{p}{.}\PYG{n+nx}{click}\PYG{p}{(}\PYG{n+nx}{mostrar\PYGZus{}impares}\PYG{p}{)}

        \PYG{n+nx}{\PYGZdl{}}\PYG{p}{(}\PYG{l+s+s2}{\PYGZdq{}\PYGZsh{}ocultar\PYGZus{}pares\PYGZdq{}}\PYG{p}{)}\PYG{p}{.}\PYG{n+nx}{click}\PYG{p}{(}\PYG{n+nx}{ocultar\PYGZus{}pares}\PYG{p}{)}
        \PYG{n+nx}{\PYGZdl{}}\PYG{p}{(}\PYG{l+s+s2}{\PYGZdq{}\PYGZsh{}ocultar\PYGZus{}impares\PYGZdq{}}\PYG{p}{)}\PYG{p}{.}\PYG{n+nx}{click}\PYG{p}{(}\PYG{n+nx}{ocultar\PYGZus{}impares}\PYG{p}{)}

\PYG{p}{\PYGZcb{}}
\end{sphinxVerbatim}

Existen diversos eventos aunque los más utilizados son:
\begin{itemize}
\item {} 
\sphinxcode{click}

\item {} 
\sphinxcode{dblclick}

\item {} 
\sphinxcode{mouseover}

\end{itemize}


\subsection{Ejercicio}
\label{\detokenize{tema4:id5}}
Crear un programa que tenga varios párrafos con 4 botones que permitan que cuando se haga click en ellos ocurran distintas cosas
\begin{itemize}
\item {} 
Habrá un botón con el texto «Ocultar pares». Cuando se hace click en él se ocultan los párrafos pares.

\item {} 
Habrá un botón con el texto «Ocultar impares». Cuando se hace click en él se ocultan los párrafos impares.

\item {} 
Habrá un botón con el texto «Mostrar pares». Cuando se hace click en él se muestran los párrafos pares (que tal vez estaban ocultos).

\item {} 
Habrá un botón con el texto «Mostrar impares». Cuando se hace click en él se muestran los párrafos impares (que tal vez estaban ocultos).

\end{itemize}


\subsection{Ejercicio}
\label{\detokenize{tema4:id6}}
Crear una página en la que hay un div con texto y al pasar el ratón por encima de ella, la caja cambia de color.

Antes de poder resolver este ejercicio, hay que echar un vistazo a varias posibilidades de JQuery.


\section{Procesado de atributos}
\label{\detokenize{tema4:procesado-de-atributos}}
En JQuery sabemos que podemos procesar elementos utilizando su \sphinxcode{id} con cosas como esta:

\begin{sphinxVerbatim}[commandchars=\\\{\}]
\PYG{k+kd}{var} \PYG{n+nx}{objeto}\PYG{o}{=}\PYG{n+nx}{\PYGZdl{}}\PYG{p}{(}\PYG{l+s+s2}{\PYGZdq{}\PYGZsh{}identificador1\PYGZdq{}}\PYG{p}{)}
\PYG{n+nx}{objeto}\PYG{p}{.}\PYG{n+nx}{metodo}\PYG{p}{(} \PYG{p}{...} \PYG{p}{)}
\end{sphinxVerbatim}

Una de las cosas que se puede hacer es leer y escribir diversos atributos de los objetos. Además, se pueden leer propiedades especiales como comprobar si un radio o un checkbox están en el estado \sphinxcode{checked}.

Supongamos este formulario:

\begin{sphinxVerbatim}[commandchars=\\\{\}]
    \PYG{p}{\PYGZlt{}}\PYG{n+nt}{form}\PYG{p}{\PYGZgt{}}
    \PYG{p}{\PYGZlt{}}\PYG{n+nt}{input} \PYG{n+na}{type}\PYG{o}{=}\PYG{l+s}{\PYGZdq{}radio\PYGZdq{}} \PYG{n+na}{name}\PYG{o}{=}\PYG{l+s}{\PYGZdq{}sexo\PYGZdq{}} \PYG{n+na}{value}\PYG{o}{=}\PYG{l+s}{\PYGZdq{}h\PYGZdq{}} \PYG{n+na}{id}\PYG{o}{=}\PYG{l+s}{\PYGZdq{}opc\PYGZus{}h\PYGZdq{}}\PYG{p}{\PYGZgt{}}Hombre
    \PYG{p}{\PYGZlt{}}\PYG{n+nt}{br}\PYG{p}{/}\PYG{p}{\PYGZgt{}}
    \PYG{p}{\PYGZlt{}}\PYG{n+nt}{input} \PYG{n+na}{type}\PYG{o}{=}\PYG{l+s}{\PYGZdq{}radio\PYGZdq{}} \PYG{n+na}{name}\PYG{o}{=}\PYG{l+s}{\PYGZdq{}sexo\PYGZdq{}} \PYG{n+na}{value}\PYG{o}{=}\PYG{l+s}{\PYGZdq{}m\PYGZdq{}} \PYG{n+na}{id}\PYG{o}{=}\PYG{l+s}{\PYGZdq{}opc\PYGZus{}m\PYGZdq{}}\PYG{p}{\PYGZgt{}}Mujer
    \PYG{p}{\PYGZlt{}}\PYG{n+nt}{br}\PYG{p}{/}\PYG{p}{\PYGZgt{}}
    \PYG{p}{\PYGZlt{}}\PYG{n+nt}{input} \PYG{n+na}{type}\PYG{o}{=}\PYG{l+s}{\PYGZdq{}text\PYGZdq{}} \PYG{n+na}{id}\PYG{o}{=}\PYG{l+s}{\PYGZdq{}informe\PYGZdq{}}\PYG{p}{\PYGZgt{}}
    \PYG{p}{\PYGZlt{}}\PYG{n+nt}{br}\PYG{p}{/}\PYG{p}{\PYGZgt{}}
    \PYG{p}{\PYGZlt{}}\PYG{n+nt}{input} \PYG{n+na}{type}\PYG{o}{=}\PYG{l+s}{\PYGZdq{}checkbox\PYGZdq{}} \PYG{n+na}{name}\PYG{o}{=}\PYG{l+s}{\PYGZdq{}medios[]\PYGZdq{}} \PYG{n+na}{id}\PYG{o}{=}\PYG{l+s}{\PYGZdq{}bus\PYGZdq{}}\PYG{p}{\PYGZgt{}}Autobús
    \PYG{p}{\PYGZlt{}}\PYG{n+nt}{br}\PYG{p}{/}\PYG{p}{\PYGZgt{}}
    \PYG{p}{\PYGZlt{}}\PYG{n+nt}{input} \PYG{n+na}{type}\PYG{o}{=}\PYG{l+s}{\PYGZdq{}checkbox\PYGZdq{}} \PYG{n+na}{name}\PYG{o}{=}\PYG{l+s}{\PYGZdq{}medios[]\PYGZdq{}} \PYG{n+na}{id}\PYG{o}{=}\PYG{l+s}{\PYGZdq{}coche\PYGZdq{}}\PYG{p}{\PYGZgt{}}Coche
    \PYG{p}{\PYGZlt{}}\PYG{n+nt}{br}\PYG{p}{/}\PYG{p}{\PYGZgt{}}
    \PYG{p}{\PYGZlt{}}\PYG{n+nt}{input} \PYG{n+na}{type}\PYG{o}{=}\PYG{l+s}{\PYGZdq{}checkbox\PYGZdq{}} \PYG{n+na}{name}\PYG{o}{=}\PYG{l+s}{\PYGZdq{}medios[]\PYGZdq{}} \PYG{n+na}{id}\PYG{o}{=}\PYG{l+s}{\PYGZdq{}moto\PYGZdq{}}\PYG{p}{\PYGZgt{}}Moto
    \PYG{p}{\PYGZlt{}}\PYG{n+nt}{br}\PYG{p}{/}\PYG{p}{\PYGZgt{}}
    \PYG{p}{\PYGZlt{}}\PYG{n+nt}{input} \PYG{n+na}{type}\PYG{o}{=}\PYG{l+s}{\PYGZdq{}checkbox\PYGZdq{}} \PYG{n+na}{name}\PYG{o}{=}\PYG{l+s}{\PYGZdq{}medios[]\PYGZdq{}} \PYG{n+na}{id}\PYG{o}{=}\PYG{l+s}{\PYGZdq{}bici\PYGZdq{}}\PYG{p}{\PYGZgt{}}Bici
    \PYG{p}{\PYGZlt{}}\PYG{n+nt}{br}\PYG{p}{/}\PYG{p}{\PYGZgt{}}
    \PYG{p}{\PYGZlt{}}\PYG{n+nt}{input} \PYG{n+na}{type}\PYG{o}{=}\PYG{l+s}{\PYGZdq{}checkbox\PYGZdq{}} \PYG{n+na}{name}\PYG{o}{=}\PYG{l+s}{\PYGZdq{}medios[]\PYGZdq{}} \PYG{n+na}{id}\PYG{o}{=}\PYG{l+s}{\PYGZdq{}tren\PYGZdq{}}\PYG{p}{\PYGZgt{}}Tren
    \PYG{p}{\PYGZlt{}}\PYG{n+nt}{br}\PYG{p}{/}\PYG{p}{\PYGZgt{}}

\PYG{p}{\PYGZlt{}}\PYG{p}{/}\PYG{n+nt}{form}\PYG{p}{\PYGZgt{}}
\end{sphinxVerbatim}

Podemos usar el método \sphinxcode{val} para cambiar el valor de un objeto cualquiera:

\begin{sphinxVerbatim}[commandchars=\\\{\}]
\PYG{k+kd}{function} \PYG{n+nx}{inicio}\PYG{p}{(}\PYG{p}{)}\PYG{p}{\PYGZob{}}
        \PYG{k+kd}{var} \PYG{n+nx}{opc\PYGZus{}h}\PYG{o}{=}\PYG{n+nx}{\PYGZdl{}}\PYG{p}{(}\PYG{l+s+s2}{\PYGZdq{}\PYGZsh{}opc\PYGZus{}h\PYGZdq{}}\PYG{p}{)}
        \PYG{n+nx}{opc\PYGZus{}h}\PYG{p}{.}\PYG{n+nx}{click} \PYG{p}{(} \PYG{n+nx}{click\PYGZus{}hombre} \PYG{p}{)}

        \PYG{k+kd}{var} \PYG{n+nx}{opc\PYGZus{}m}\PYG{o}{=}\PYG{n+nx}{\PYGZdl{}}\PYG{p}{(}\PYG{l+s+s2}{\PYGZdq{}\PYGZsh{}opc\PYGZus{}m\PYGZdq{}}\PYG{p}{)}
        \PYG{n+nx}{opc\PYGZus{}m}\PYG{p}{.}\PYG{n+nx}{click} \PYG{p}{(} \PYG{n+nx}{click\PYGZus{}mujer} \PYG{p}{)}
\PYG{p}{\PYGZcb{}}

\PYG{k+kd}{function} \PYG{n+nx}{click\PYGZus{}hombre}\PYG{p}{(}\PYG{p}{)} \PYG{p}{\PYGZob{}}
        \PYG{k+kd}{var} \PYG{n+nx}{cuadro\PYGZus{}texto}\PYG{o}{=}\PYG{n+nx}{\PYGZdl{}}\PYG{p}{(}\PYG{l+s+s2}{\PYGZdq{}\PYGZsh{}informe\PYGZdq{}}\PYG{p}{)}
        \PYG{n+nx}{cuadro\PYGZus{}texto}\PYG{p}{.}\PYG{n+nx}{val}\PYG{p}{(}\PYG{l+s+s2}{\PYGZdq{}Bienvenido Sr.\PYGZdq{}}\PYG{p}{)}
\PYG{p}{\PYGZcb{}}

\PYG{k+kd}{function} \PYG{n+nx}{click\PYGZus{}mujer}\PYG{p}{(}\PYG{p}{)}\PYG{p}{\PYGZob{}}
        \PYG{k+kd}{var} \PYG{n+nx}{cuadro\PYGZus{}texto}\PYG{o}{=}\PYG{n+nx}{\PYGZdl{}}\PYG{p}{(}\PYG{l+s+s2}{\PYGZdq{}\PYGZsh{}informe\PYGZdq{}}\PYG{p}{)}
        \PYG{n+nx}{cuadro\PYGZus{}texto}\PYG{p}{.}\PYG{n+nx}{val}\PYG{p}{(}\PYG{l+s+s2}{\PYGZdq{}Bienvenido Sra/Srta.\PYGZdq{}}\PYG{p}{)}
\PYG{p}{\PYGZcb{}}
\end{sphinxVerbatim}

Por ejemplo, en los checkboxes y en los radios, podemos comprobar si uno de ellos está marcado comprobando la propiedad «checked» con el método \sphinxcode{prop}.

Supongamos que deseamos saber cuantos checkboxes se marcan. Si se marcan cero, una o dos, mostraremos el texto «poca variedad», si se marcan tres mostraremos «cierta variedad» y si se marcan cuatro o cinco, mostraremos «mucha variedad».

Aquí hay dos posibles soluciones, siendo una de ellas  más corta y flexible que la otra.

La primera:

\begin{sphinxVerbatim}[commandchars=\\\{\}]
\PYG{k+kd}{var} \PYG{n+nx}{opc\PYGZus{}coche}\PYG{o}{=}\PYG{n+nx}{\PYGZdl{}}\PYG{p}{(}\PYG{l+s+s2}{\PYGZdq{}\PYGZsh{}coche\PYGZdq{}}\PYG{p}{)}
\PYG{n+nx}{opc\PYGZus{}coche}\PYG{p}{.}\PYG{n+nx}{click} \PYG{p}{(} \PYG{n+nx}{cuantas\PYGZus{}pulsadas} \PYG{p}{)}

\PYG{k+kd}{var} \PYG{n+nx}{opc\PYGZus{}moto}\PYG{o}{=}\PYG{n+nx}{\PYGZdl{}}\PYG{p}{(}\PYG{l+s+s2}{\PYGZdq{}\PYGZsh{}moto\PYGZdq{}}\PYG{p}{)}
\PYG{n+nx}{opc\PYGZus{}moto}\PYG{p}{.}\PYG{n+nx}{click} \PYG{p}{(} \PYG{n+nx}{cuantas\PYGZus{}pulsadas} \PYG{p}{)}

\PYG{k+kd}{var} \PYG{n+nx}{opc\PYGZus{}bici}\PYG{o}{=}\PYG{n+nx}{\PYGZdl{}}\PYG{p}{(}\PYG{l+s+s2}{\PYGZdq{}\PYGZsh{}bici\PYGZdq{}}\PYG{p}{)}
\PYG{n+nx}{opc\PYGZus{}bici}\PYG{p}{.}\PYG{n+nx}{click} \PYG{p}{(} \PYG{n+nx}{cuantas\PYGZus{}pulsadas} \PYG{p}{)}

\PYG{k+kd}{var} \PYG{n+nx}{opc\PYGZus{}bus}\PYG{o}{=}\PYG{n+nx}{\PYGZdl{}}\PYG{p}{(}\PYG{l+s+s2}{\PYGZdq{}\PYGZsh{}bus\PYGZdq{}}\PYG{p}{)}
\PYG{n+nx}{opc\PYGZus{}bus}\PYG{p}{.}\PYG{n+nx}{click} \PYG{p}{(} \PYG{n+nx}{cuantas\PYGZus{}pulsadas} \PYG{p}{)}

\PYG{k+kd}{var} \PYG{n+nx}{opc\PYGZus{}tren}\PYG{o}{=}\PYG{n+nx}{\PYGZdl{}}\PYG{p}{(}\PYG{l+s+s2}{\PYGZdq{}\PYGZsh{}tren\PYGZdq{}}\PYG{p}{)}
\PYG{n+nx}{opc\PYGZus{}tren}\PYG{p}{.}\PYG{n+nx}{click} \PYG{p}{(} \PYG{n+nx}{cuantas\PYGZus{}pulsadas} \PYG{p}{)}

    \PYG{k+kd}{function} \PYG{n+nx}{cuantas\PYGZus{}pulsadas}\PYG{p}{(}\PYG{p}{)}\PYG{p}{\PYGZob{}}
            \PYG{c+c1}{//Aquí contaríamos cuantas tienen la propiedad checked}
    \PYG{p}{\PYGZcb{}}
\end{sphinxVerbatim}

El segundo implica que todos los controles tengan el mismo atributo \sphinxcode{class}. Ahora la solución tendría un HTML como este:

\begin{sphinxVerbatim}[commandchars=\\\{\}]
        \PYG{p}{\PYGZlt{}}\PYG{n+nt}{input} \PYG{n+na}{type}\PYG{o}{=}\PYG{l+s}{\PYGZdq{}checkbox\PYGZdq{}} \PYG{n+na}{name}\PYG{o}{=}\PYG{l+s}{\PYGZdq{}medios[]\PYGZdq{}} \PYG{n+na}{id}\PYG{o}{=}\PYG{l+s}{\PYGZdq{}bus\PYGZdq{}} \PYG{n+na}{class}\PYG{o}{=}\PYG{l+s}{\PYGZdq{}medio\PYGZdq{}}\PYG{p}{\PYGZgt{}}Autobús
\PYG{p}{\PYGZlt{}}\PYG{n+nt}{br}\PYG{p}{/}\PYG{p}{\PYGZgt{}}
\PYG{p}{\PYGZlt{}}\PYG{n+nt}{input} \PYG{n+na}{type}\PYG{o}{=}\PYG{l+s}{\PYGZdq{}checkbox\PYGZdq{}} \PYG{n+na}{name}\PYG{o}{=}\PYG{l+s}{\PYGZdq{}medios[]\PYGZdq{}} \PYG{n+na}{id}\PYG{o}{=}\PYG{l+s}{\PYGZdq{}coche\PYGZdq{}} \PYG{n+na}{class}\PYG{o}{=}\PYG{l+s}{\PYGZdq{}medio\PYGZdq{}}\PYG{p}{\PYGZgt{}}Coche
\PYG{p}{\PYGZlt{}}\PYG{n+nt}{br}\PYG{p}{/}\PYG{p}{\PYGZgt{}}
\PYG{p}{\PYGZlt{}}\PYG{n+nt}{input} \PYG{n+na}{type}\PYG{o}{=}\PYG{l+s}{\PYGZdq{}checkbox\PYGZdq{}} \PYG{n+na}{name}\PYG{o}{=}\PYG{l+s}{\PYGZdq{}medios[]\PYGZdq{}} \PYG{n+na}{id}\PYG{o}{=}\PYG{l+s}{\PYGZdq{}moto\PYGZdq{}} \PYG{n+na}{class}\PYG{o}{=}\PYG{l+s}{\PYGZdq{}medio\PYGZdq{}}\PYG{p}{\PYGZgt{}}Moto
\PYG{p}{\PYGZlt{}}\PYG{n+nt}{br}\PYG{p}{/}\PYG{p}{\PYGZgt{}}
\PYG{p}{\PYGZlt{}}\PYG{n+nt}{input} \PYG{n+na}{type}\PYG{o}{=}\PYG{l+s}{\PYGZdq{}checkbox\PYGZdq{}} \PYG{n+na}{name}\PYG{o}{=}\PYG{l+s}{\PYGZdq{}medios[]\PYGZdq{}} \PYG{n+na}{id}\PYG{o}{=}\PYG{l+s}{\PYGZdq{}bici\PYGZdq{}} \PYG{n+na}{class}\PYG{o}{=}\PYG{l+s}{\PYGZdq{}medio\PYGZdq{}}\PYG{p}{\PYGZgt{}}Bici
\PYG{p}{\PYGZlt{}}\PYG{n+nt}{br}\PYG{p}{/}\PYG{p}{\PYGZgt{}}
\PYG{p}{\PYGZlt{}}\PYG{n+nt}{input} \PYG{n+na}{type}\PYG{o}{=}\PYG{l+s}{\PYGZdq{}checkbox\PYGZdq{}} \PYG{n+na}{name}\PYG{o}{=}\PYG{l+s}{\PYGZdq{}medios[]\PYGZdq{}} \PYG{n+na}{id}\PYG{o}{=}\PYG{l+s}{\PYGZdq{}tren\PYGZdq{}} \PYG{n+na}{class}\PYG{o}{=}\PYG{l+s}{\PYGZdq{}medio\PYGZdq{}}\PYG{p}{\PYGZgt{}}Tren
\PYG{p}{\PYGZlt{}}\PYG{n+nt}{br}\PYG{p}{/}\PYG{p}{\PYGZgt{}}
\end{sphinxVerbatim}

Y el JS sería así:

\begin{sphinxVerbatim}[commandchars=\\\{\}]
    \PYG{k+kd}{var} \PYG{n+nx}{medios\PYGZus{}de\PYGZus{}locomocion}\PYG{o}{=}\PYG{n+nx}{\PYGZdl{}}\PYG{p}{(}\PYG{l+s+s2}{\PYGZdq{}.medio\PYGZdq{}}\PYG{p}{)}
\PYG{n+nx}{medios\PYGZus{}de\PYGZus{}locomocion}\PYG{p}{.}\PYG{n+nx}{click} \PYG{p}{(} \PYG{n+nx}{cuantas\PYGZus{}pulsadas} \PYG{p}{)}
    \PYG{k+kd}{function} \PYG{n+nx}{cuantas\PYGZus{}pulsadas}\PYG{p}{(}\PYG{p}{)}\PYG{p}{\PYGZob{}}
            \PYG{k+kd}{var} \PYG{n+nx}{cuantas\PYGZus{}marcadas}\PYG{o}{=}\PYG{l+m+mi}{0}
            \PYG{k+kd}{var} \PYG{n+nx}{vector\PYGZus{}ids}\PYG{o}{=}\PYG{p}{[}\PYG{l+s+s2}{\PYGZdq{}\PYGZsh{}bus\PYGZdq{}}\PYG{p}{,} \PYG{l+s+s2}{\PYGZdq{}\PYGZsh{}coche\PYGZdq{}}\PYG{p}{,} \PYG{l+s+s2}{\PYGZdq{}\PYGZsh{}moto\PYGZdq{}}\PYG{p}{,} \PYG{l+s+s2}{\PYGZdq{}\PYGZsh{}bici\PYGZdq{}}\PYG{p}{,} \PYG{l+s+s2}{\PYGZdq{}\PYGZsh{}tren\PYGZdq{}}\PYG{p}{]}

            \PYG{k}{for} \PYG{p}{(}\PYG{n+nx}{pos} \PYG{k}{in} \PYG{n+nx}{vector\PYGZus{}ids}\PYG{p}{)}\PYG{p}{\PYGZob{}}
                    \PYG{k+kd}{var} \PYG{n+nx}{objeto} \PYG{o}{=} \PYG{n+nx}{\PYGZdl{}}\PYG{p}{(} \PYG{n+nx}{vector\PYGZus{}ids}\PYG{p}{[}\PYG{n+nx}{pos}\PYG{p}{]} \PYG{p}{)}
                    \PYG{k}{if} \PYG{p}{(}\PYG{n+nx}{objeto}\PYG{p}{.}\PYG{n+nx}{prop}\PYG{p}{(}\PYG{l+s+s2}{\PYGZdq{}checked\PYGZdq{}}\PYG{p}{)}\PYG{p}{)} \PYG{p}{\PYGZob{}}
                            \PYG{n+nx}{cuantas\PYGZus{}marcadas}\PYG{o}{=}\PYG{n+nx}{cuantas\PYGZus{}marcadas}\PYG{o}{+}\PYG{l+m+mi}{1}
                    \PYG{p}{\PYGZcb{}}
            \PYG{p}{\PYGZcb{}}

            \PYG{k}{if} \PYG{p}{(}\PYG{p}{(}\PYG{n+nx}{cuantas\PYGZus{}marcadas}\PYG{o}{\PYGZgt{}=}\PYG{l+m+mi}{0} \PYG{p}{)} \PYG{o}{\PYGZam{}\PYGZam{}} \PYG{p}{(}\PYG{n+nx}{cuantas\PYGZus{}marcadas}\PYG{o}{\PYGZlt{}=}\PYG{l+m+mi}{2}\PYG{p}{)}\PYG{p}{)}\PYG{p}{\PYGZob{}}
                    \PYG{n+nx}{alert} \PYG{p}{(}\PYG{l+s+s2}{\PYGZdq{}Poca variedad\PYGZdq{}}\PYG{p}{)}
            \PYG{p}{\PYGZcb{}}
            \PYG{k}{if} \PYG{p}{(}\PYG{n+nx}{cuantas\PYGZus{}marcadas}\PYG{o}{==}\PYG{l+m+mi}{3}\PYG{p}{)} \PYG{p}{\PYGZob{}}
                    \PYG{n+nx}{alert} \PYG{p}{(}\PYG{l+s+s2}{\PYGZdq{}Variedad media\PYGZdq{}}\PYG{p}{)}
            \PYG{p}{\PYGZcb{}}
            \PYG{k}{if} \PYG{p}{(}\PYG{n+nx}{cuantas\PYGZus{}marcadas}\PYG{o}{\PYGZgt{}=}\PYG{l+m+mi}{4}\PYG{p}{)}\PYG{p}{\PYGZob{}}
                    \PYG{n+nx}{alert} \PYG{p}{(}\PYG{l+s+s2}{\PYGZdq{}Mucha variedad\PYGZdq{}}\PYG{p}{)}
            \PYG{p}{\PYGZcb{}}
    \PYG{p}{\PYGZcb{}}
\end{sphinxVerbatim}


\subsection{Ejercicio: recuento de medios de locomoción}
\label{\detokenize{tema4:ejercicio-recuento-de-medios-de-locomocion}}
Crear un programa que permita al usuario indicar cinco posibles medios de locomoción (usar checkboxes), a saber:coche, moto, bus, tren y avión. El programa debe contabilizar cuantos se usan en informar del número de medios usados en un textbox.


\subsubsection{Solución: recuento de medios (HTML)}
\label{\detokenize{tema4:solucion-recuento-de-medios-html}}
\begin{sphinxVerbatim}[commandchars=\\\{\}]
        \PYG{p}{\PYGZlt{}}\PYG{n+nt}{div} \PYG{n+na}{class}\PYG{o}{=}\PYG{l+s}{\PYGZdq{}ui\PYGZhy{}grid\PYGZhy{}d\PYGZdq{}}\PYG{p}{\PYGZgt{}}
    \PYG{p}{\PYGZlt{}}\PYG{n+nt}{div} \PYG{n+na}{class}\PYG{o}{=}\PYG{l+s}{\PYGZdq{}ui\PYGZhy{}block\PYGZhy{}a\PYGZdq{}}\PYG{p}{\PYGZgt{}}
        \PYG{p}{\PYGZlt{}}\PYG{n+nt}{input} \PYG{n+na}{type}\PYG{o}{=}\PYG{l+s}{\PYGZdq{}checkbox\PYGZdq{}}
               \PYG{n+na}{name}\PYG{o}{=}\PYG{l+s}{\PYGZdq{}medio\PYGZdq{}}
               \PYG{n+na}{id}\PYG{o}{=}\PYG{l+s}{\PYGZdq{}coche\PYGZdq{}}\PYG{p}{\PYGZgt{}}
        \PYG{p}{\PYGZlt{}}\PYG{n+nt}{label} \PYG{n+na}{for}\PYG{o}{=}\PYG{l+s}{\PYGZdq{}coche\PYGZdq{}}\PYG{p}{\PYGZgt{}}Coche\PYG{p}{\PYGZlt{}}\PYG{p}{/}\PYG{n+nt}{label}\PYG{p}{\PYGZgt{}}
    \PYG{p}{\PYGZlt{}}\PYG{p}{/}\PYG{n+nt}{div}\PYG{p}{\PYGZgt{}}
    \PYG{p}{\PYGZlt{}}\PYG{n+nt}{div} \PYG{n+na}{class}\PYG{o}{=}\PYG{l+s}{\PYGZdq{}ui\PYGZhy{}block\PYGZhy{}b\PYGZdq{}}\PYG{p}{\PYGZgt{}}
        \PYG{p}{\PYGZlt{}}\PYG{n+nt}{input} \PYG{n+na}{type}\PYG{o}{=}\PYG{l+s}{\PYGZdq{}checkbox\PYGZdq{}}
               \PYG{n+na}{name}\PYG{o}{=}\PYG{l+s}{\PYGZdq{}medio\PYGZdq{}}
               \PYG{n+na}{id}\PYG{o}{=}\PYG{l+s}{\PYGZdq{}moto\PYGZdq{}}\PYG{p}{\PYGZgt{}}
        \PYG{p}{\PYGZlt{}}\PYG{n+nt}{label} \PYG{n+na}{for}\PYG{o}{=}\PYG{l+s}{\PYGZdq{}moto\PYGZdq{}}\PYG{p}{\PYGZgt{}}Moto\PYG{p}{\PYGZlt{}}\PYG{p}{/}\PYG{n+nt}{label}\PYG{p}{\PYGZgt{}}
    \PYG{p}{\PYGZlt{}}\PYG{p}{/}\PYG{n+nt}{div}\PYG{p}{\PYGZgt{}}
    \PYG{p}{\PYGZlt{}}\PYG{n+nt}{div} \PYG{n+na}{class}\PYG{o}{=}\PYG{l+s}{\PYGZdq{}ui\PYGZhy{}block\PYGZhy{}c\PYGZdq{}}\PYG{p}{\PYGZgt{}}
        \PYG{p}{\PYGZlt{}}\PYG{n+nt}{input} \PYG{n+na}{type}\PYG{o}{=}\PYG{l+s}{\PYGZdq{}checkbox\PYGZdq{}}
               \PYG{n+na}{name}\PYG{o}{=}\PYG{l+s}{\PYGZdq{}medio\PYGZdq{}}
               \PYG{n+na}{id}\PYG{o}{=}\PYG{l+s}{\PYGZdq{}bus\PYGZdq{}}\PYG{p}{\PYGZgt{}}
        \PYG{p}{\PYGZlt{}}\PYG{n+nt}{label} \PYG{n+na}{for}\PYG{o}{=}\PYG{l+s}{\PYGZdq{}bus\PYGZdq{}}\PYG{p}{\PYGZgt{}}Bus\PYG{p}{\PYGZlt{}}\PYG{p}{/}\PYG{n+nt}{label}\PYG{p}{\PYGZgt{}}
    \PYG{p}{\PYGZlt{}}\PYG{p}{/}\PYG{n+nt}{div}\PYG{p}{\PYGZgt{}}
    \PYG{p}{\PYGZlt{}}\PYG{n+nt}{div} \PYG{n+na}{class}\PYG{o}{=}\PYG{l+s}{\PYGZdq{}ui\PYGZhy{}block\PYGZhy{}d\PYGZdq{}}\PYG{p}{\PYGZgt{}}
        \PYG{p}{\PYGZlt{}}\PYG{n+nt}{input} \PYG{n+na}{type}\PYG{o}{=}\PYG{l+s}{\PYGZdq{}checkbox\PYGZdq{}}
               \PYG{n+na}{name}\PYG{o}{=}\PYG{l+s}{\PYGZdq{}medio\PYGZdq{}}
               \PYG{n+na}{id}\PYG{o}{=}\PYG{l+s}{\PYGZdq{}tren\PYGZdq{}}\PYG{p}{\PYGZgt{}}
        \PYG{p}{\PYGZlt{}}\PYG{n+nt}{label} \PYG{n+na}{for}\PYG{o}{=}\PYG{l+s}{\PYGZdq{}tren\PYGZdq{}}\PYG{p}{\PYGZgt{}}Tren\PYG{p}{\PYGZlt{}}\PYG{p}{/}\PYG{n+nt}{label}\PYG{p}{\PYGZgt{}}
    \PYG{p}{\PYGZlt{}}\PYG{p}{/}\PYG{n+nt}{div}\PYG{p}{\PYGZgt{}}
    \PYG{p}{\PYGZlt{}}\PYG{n+nt}{div} \PYG{n+na}{class}\PYG{o}{=}\PYG{l+s}{\PYGZdq{}ui\PYGZhy{}block\PYGZhy{}e\PYGZdq{}}\PYG{p}{\PYGZgt{}}
        \PYG{p}{\PYGZlt{}}\PYG{n+nt}{input} \PYG{n+na}{type}\PYG{o}{=}\PYG{l+s}{\PYGZdq{}checkbox\PYGZdq{}}
               \PYG{n+na}{name}\PYG{o}{=}\PYG{l+s}{\PYGZdq{}medio\PYGZdq{}}
               \PYG{n+na}{id}\PYG{o}{=}\PYG{l+s}{\PYGZdq{}avion\PYGZdq{}}\PYG{p}{\PYGZgt{}}
        \PYG{p}{\PYGZlt{}}\PYG{n+nt}{label} \PYG{n+na}{for}\PYG{o}{=}\PYG{l+s}{\PYGZdq{}avion\PYGZdq{}}\PYG{p}{\PYGZgt{}}Avión\PYG{p}{\PYGZlt{}}\PYG{p}{/}\PYG{n+nt}{label}\PYG{p}{\PYGZgt{}}
    \PYG{p}{\PYGZlt{}}\PYG{p}{/}\PYG{n+nt}{div}\PYG{p}{\PYGZgt{}}
\PYG{p}{\PYGZlt{}}\PYG{p}{/}\PYG{n+nt}{div}\PYG{p}{\PYGZgt{}}
\PYG{p}{\PYGZlt{}}\PYG{n+nt}{input} \PYG{n+na}{type}\PYG{o}{=}\PYG{l+s}{\PYGZdq{}text\PYGZdq{}} \PYG{n+na}{id}\PYG{o}{=}\PYG{l+s}{\PYGZdq{}informe\PYGZdq{}}\PYG{p}{\PYGZgt{}}
\end{sphinxVerbatim}


\subsubsection{Solución: recuento de medios (JS)}
\label{\detokenize{tema4:solucion-recuento-de-medios-js}}
Variante 1: Sin vectores, implica usar muchos \sphinxcode{if}. Aunque funcione supone cortar y pegar, que aunque en este caso no sea un trabajo muy grande nos obliga a adoptar malos hábitos

\begin{sphinxVerbatim}[commandchars=\\\{\}]
\PYG{n+nx}{\PYGZdl{}}\PYG{p}{(}\PYG{n+nb}{document}\PYG{p}{)}\PYG{p}{.}\PYG{n+nx}{ready}\PYG{p}{(}\PYG{n+nx}{main}\PYG{p}{)}

\PYG{k+kd}{function} \PYG{n+nx}{contar}\PYG{p}{(}\PYG{p}{)}\PYG{p}{\PYGZob{}}
        \PYG{k+kd}{var} \PYG{n+nx}{contador}\PYG{o}{=}\PYG{l+m+mi}{0}

        \PYG{k}{if} \PYG{p}{(}\PYG{n+nx}{\PYGZdl{}}\PYG{p}{(}\PYG{l+s+s2}{\PYGZdq{}\PYGZsh{}coche\PYGZdq{}}\PYG{p}{)}\PYG{p}{.}\PYG{n+nx}{prop}\PYG{p}{(}\PYG{l+s+s2}{\PYGZdq{}checked\PYGZdq{}}\PYG{p}{)}\PYG{p}{)}
        \PYG{p}{\PYGZob{}}
                \PYG{n+nx}{contador}\PYG{o}{=}\PYG{n+nx}{contador}\PYG{o}{+}\PYG{l+m+mi}{1}
        \PYG{p}{\PYGZcb{}}
        \PYG{k}{if} \PYG{p}{(}\PYG{n+nx}{\PYGZdl{}}\PYG{p}{(}\PYG{l+s+s2}{\PYGZdq{}\PYGZsh{}moto\PYGZdq{}}\PYG{p}{)}\PYG{p}{.}\PYG{n+nx}{prop}\PYG{p}{(}\PYG{l+s+s2}{\PYGZdq{}checked\PYGZdq{}}\PYG{p}{)}\PYG{p}{)}
        \PYG{p}{\PYGZob{}}
                \PYG{n+nx}{contador}\PYG{o}{=}\PYG{n+nx}{contador}\PYG{o}{+}\PYG{l+m+mi}{1}
        \PYG{p}{\PYGZcb{}}
        \PYG{k+kd}{var} \PYG{n+nx}{mensaje}\PYG{o}{=}\PYG{l+s+s2}{\PYGZdq{}Medios:\PYGZdq{}}\PYG{o}{+}\PYG{n+nx}{contador}
        \PYG{n+nx}{\PYGZdl{}}\PYG{p}{(}\PYG{l+s+s2}{\PYGZdq{}\PYGZsh{}informe\PYGZdq{}}\PYG{p}{)}\PYG{p}{.}\PYG{n+nx}{val}\PYG{p}{(}\PYG{n+nx}{mensaje}\PYG{p}{)}

\PYG{p}{\PYGZcb{}}

\PYG{k+kd}{function} \PYG{n+nx}{main}\PYG{p}{(}\PYG{p}{)}\PYG{p}{\PYGZob{}}
        \PYG{n+nx}{\PYGZdl{}}\PYG{p}{(}\PYG{l+s+s2}{\PYGZdq{}\PYGZsh{}coche\PYGZdq{}}\PYG{p}{)}\PYG{p}{.}\PYG{n+nx}{click}\PYG{p}{(}\PYG{n+nx}{contar}\PYG{p}{)}
        \PYG{n+nx}{\PYGZdl{}}\PYG{p}{(}\PYG{l+s+s2}{\PYGZdq{}\PYGZsh{}moto\PYGZdq{}}\PYG{p}{)}\PYG{p}{.}\PYG{n+nx}{click}\PYG{p}{(}\PYG{n+nx}{contar}\PYG{p}{)}
        \PYG{n+nx}{\PYGZdl{}}\PYG{p}{(}\PYG{l+s+s2}{\PYGZdq{}\PYGZsh{}bus\PYGZdq{}}\PYG{p}{)}\PYG{p}{.}\PYG{n+nx}{click}\PYG{p}{(}\PYG{n+nx}{contar}\PYG{p}{)}
        \PYG{n+nx}{\PYGZdl{}}\PYG{p}{(}\PYG{l+s+s2}{\PYGZdq{}\PYGZsh{}tren\PYGZdq{}}\PYG{p}{)}\PYG{p}{.}\PYG{n+nx}{click}\PYG{p}{(}\PYG{n+nx}{contar}\PYG{p}{)}
        \PYG{n+nx}{\PYGZdl{}}\PYG{p}{(}\PYG{l+s+s2}{\PYGZdq{}\PYGZsh{}avion\PYGZdq{}}\PYG{p}{)}\PYG{p}{.}\PYG{n+nx}{click}\PYG{p}{(}\PYG{n+nx}{contar}\PYG{p}{)}

\PYG{p}{\PYGZcb{}}
\end{sphinxVerbatim}

Variante 2: Con vectores

\begin{sphinxVerbatim}[commandchars=\\\{\}]
\PYG{n+nx}{\PYGZdl{}}\PYG{p}{(}\PYG{n+nb}{document}\PYG{p}{)}\PYG{p}{.}\PYG{n+nx}{ready}\PYG{p}{(}\PYG{n+nx}{main}\PYG{p}{)}
\PYG{k+kd}{function} \PYG{n+nx}{contar}\PYG{p}{(}\PYG{p}{)}\PYG{p}{\PYGZob{}}
        \PYG{k+kd}{var} \PYG{n+nx}{contador}\PYG{o}{=}\PYG{l+m+mi}{0}
        \PYG{k+kd}{var} \PYG{n+nx}{ids}\PYG{o}{=}\PYG{k}{new} \PYG{n+nb}{Array}\PYG{p}{(}\PYG{p}{)}
        \PYG{k+kd}{var} \PYG{n+nx}{ids}\PYG{o}{=}\PYG{p}{[}\PYG{l+s+s2}{\PYGZdq{}\PYGZsh{}coche\PYGZdq{}}\PYG{p}{,} \PYG{l+s+s2}{\PYGZdq{}\PYGZsh{}moto\PYGZdq{}}\PYG{p}{,}
                         \PYG{l+s+s2}{\PYGZdq{}\PYGZsh{}bus\PYGZdq{}}\PYG{p}{,} \PYG{l+s+s2}{\PYGZdq{}\PYGZsh{}tren\PYGZdq{}}\PYG{p}{,}
                         \PYG{l+s+s2}{\PYGZdq{}\PYGZsh{}avion\PYGZdq{}}\PYG{p}{]}

        \PYG{k}{for} \PYG{p}{(}\PYG{n+nx}{pos} \PYG{k}{in} \PYG{n+nx}{ids}\PYG{p}{)}\PYG{p}{\PYGZob{}}
                \PYG{k+kd}{var} \PYG{n+nx}{medio}\PYG{o}{=}\PYG{n+nx}{\PYGZdl{}}\PYG{p}{(}\PYG{n+nx}{ids}\PYG{p}{[}\PYG{n+nx}{pos}\PYG{p}{]}\PYG{p}{)}
                \PYG{k}{if} \PYG{p}{(}\PYG{n+nx}{medio}\PYG{p}{.}\PYG{n+nx}{prop}\PYG{p}{(}\PYG{l+s+s2}{\PYGZdq{}checked\PYGZdq{}}\PYG{p}{)}\PYG{p}{)}
                \PYG{p}{\PYGZob{}}
                        \PYG{n+nx}{contador}\PYG{o}{=}\PYG{n+nx}{contador}\PYG{o}{+}\PYG{l+m+mi}{1}
                \PYG{p}{\PYGZcb{}}
        \PYG{p}{\PYGZcb{}}
        \PYG{k+kd}{var} \PYG{n+nx}{mensaje}\PYG{o}{=}\PYG{l+s+s2}{\PYGZdq{}Medios:\PYGZdq{}}\PYG{o}{+}\PYG{n+nx}{contador}
        \PYG{n+nx}{\PYGZdl{}}\PYG{p}{(}\PYG{l+s+s2}{\PYGZdq{}\PYGZsh{}informe\PYGZdq{}}\PYG{p}{)}\PYG{p}{.}\PYG{n+nx}{val}\PYG{p}{(}\PYG{n+nx}{mensaje}\PYG{p}{)}

\PYG{p}{\PYGZcb{}}

\PYG{k+kd}{function} \PYG{n+nx}{main}\PYG{p}{(}\PYG{p}{)}\PYG{p}{\PYGZob{}}
        \PYG{n+nx}{\PYGZdl{}}\PYG{p}{(}\PYG{l+s+s2}{\PYGZdq{}\PYGZsh{}coche\PYGZdq{}}\PYG{p}{)}\PYG{p}{.}\PYG{n+nx}{click}\PYG{p}{(}\PYG{n+nx}{contar}\PYG{p}{)}
        \PYG{n+nx}{\PYGZdl{}}\PYG{p}{(}\PYG{l+s+s2}{\PYGZdq{}\PYGZsh{}moto\PYGZdq{}}\PYG{p}{)}\PYG{p}{.}\PYG{n+nx}{click}\PYG{p}{(}\PYG{n+nx}{contar}\PYG{p}{)}
        \PYG{n+nx}{\PYGZdl{}}\PYG{p}{(}\PYG{l+s+s2}{\PYGZdq{}\PYGZsh{}bus\PYGZdq{}}\PYG{p}{)}\PYG{p}{.}\PYG{n+nx}{click}\PYG{p}{(}\PYG{n+nx}{contar}\PYG{p}{)}
        \PYG{n+nx}{\PYGZdl{}}\PYG{p}{(}\PYG{l+s+s2}{\PYGZdq{}\PYGZsh{}tren\PYGZdq{}}\PYG{p}{)}\PYG{p}{.}\PYG{n+nx}{click}\PYG{p}{(}\PYG{n+nx}{contar}\PYG{p}{)}
        \PYG{n+nx}{\PYGZdl{}}\PYG{p}{(}\PYG{l+s+s2}{\PYGZdq{}\PYGZsh{}avion\PYGZdq{}}\PYG{p}{)}\PYG{p}{.}\PYG{n+nx}{click}\PYG{p}{(}\PYG{n+nx}{contar}\PYG{p}{)}
\PYG{p}{\PYGZcb{}}
\end{sphinxVerbatim}


\subsection{Ejercicio configurador}
\label{\detokenize{tema4:ejercicio-configurador}}
Se desea tener una aplicación que permita configurar un equipo al gusto del usuario:
\begin{itemize}
\item {} 
Se debe elegir entre un procesador Intel o AMD. El primero cuesta 250 euros y el segundo 230.

\item {} 
Se debe elegir entre 2, 4 y 8 GB de memoria. El coste es respectivamente 90, 145, 210

\item {} 
Hay extras que se pueden elegir o no, ya que son completamente optativos (es decir, usar checkboxes). En concreto se puede tener un grabador de Blu-ray (190 euros), tarjeta gráfica aceleradora (430 euros) y un monitor LED (185 euros).

\end{itemize}


\subsection{Solución HTML configurador}
\label{\detokenize{tema4:solucion-html-configurador}}
\begin{sphinxVerbatim}[commandchars=\\\{\}]
\PYG{p}{\PYGZlt{}}\PYG{n+nt}{label} \PYG{n+na}{for}\PYG{o}{=}\PYG{l+s}{\PYGZdq{}intel\PYGZdq{}}\PYG{p}{\PYGZgt{}}Intel i5\PYG{p}{\PYGZlt{}}\PYG{p}{/}\PYG{n+nt}{label}\PYG{p}{\PYGZgt{}}
\PYG{p}{\PYGZlt{}}\PYG{n+nt}{input} \PYG{n+na}{type}\PYG{o}{=}\PYG{l+s}{\PYGZdq{}radio\PYGZdq{}}
       \PYG{n+na}{name}\PYG{o}{=}\PYG{l+s}{\PYGZdq{}procesador\PYGZdq{}} \PYG{n+na}{id}\PYG{o}{=}\PYG{l+s}{\PYGZdq{}intel\PYGZdq{}}\PYG{p}{\PYGZgt{}}
\PYG{p}{\PYGZlt{}}\PYG{n+nt}{label} \PYG{n+na}{for}\PYG{o}{=}\PYG{l+s}{\PYGZdq{}amd\PYGZdq{}}\PYG{p}{\PYGZgt{}}AMD\PYG{p}{\PYGZlt{}}\PYG{p}{/}\PYG{n+nt}{label}\PYG{p}{\PYGZgt{}}
\PYG{p}{\PYGZlt{}}\PYG{n+nt}{input} \PYG{n+na}{type}\PYG{o}{=}\PYG{l+s}{\PYGZdq{}radio\PYGZdq{}}
       \PYG{n+na}{name}\PYG{o}{=}\PYG{l+s}{\PYGZdq{}procesador\PYGZdq{}} \PYG{n+na}{id}\PYG{o}{=}\PYG{l+s}{\PYGZdq{}amd\PYGZdq{}}\PYG{p}{\PYGZgt{}}
\PYG{p}{\PYGZlt{}}\PYG{n+nt}{label} \PYG{n+na}{for}\PYG{o}{=}\PYG{l+s}{\PYGZdq{}2gb\PYGZdq{}}\PYG{p}{\PYGZgt{}}2GB\PYG{p}{\PYGZlt{}}\PYG{p}{/}\PYG{n+nt}{label}\PYG{p}{\PYGZgt{}}
\PYG{p}{\PYGZlt{}}\PYG{n+nt}{input} \PYG{n+na}{type}\PYG{o}{=}\PYG{l+s}{\PYGZdq{}radio\PYGZdq{}}
       \PYG{n+na}{name}\PYG{o}{=}\PYG{l+s}{\PYGZdq{}memoria\PYGZdq{}} \PYG{n+na}{id}\PYG{o}{=}\PYG{l+s}{\PYGZdq{}2gb\PYGZdq{}}\PYG{p}{\PYGZgt{}}
\PYG{p}{\PYGZlt{}}\PYG{n+nt}{label} \PYG{n+na}{for}\PYG{o}{=}\PYG{l+s}{\PYGZdq{}4gb\PYGZdq{}}\PYG{p}{\PYGZgt{}}4 GB\PYG{p}{\PYGZlt{}}\PYG{p}{/}\PYG{n+nt}{label}\PYG{p}{\PYGZgt{}}
\PYG{p}{\PYGZlt{}}\PYG{n+nt}{input} \PYG{n+na}{type}\PYG{o}{=}\PYG{l+s}{\PYGZdq{}radio\PYGZdq{}}
       \PYG{n+na}{name}\PYG{o}{=}\PYG{l+s}{\PYGZdq{}memoria\PYGZdq{}} \PYG{n+na}{id}\PYG{o}{=}\PYG{l+s}{\PYGZdq{}4gb\PYGZdq{}}\PYG{p}{\PYGZgt{}}
\PYG{p}{\PYGZlt{}}\PYG{n+nt}{label} \PYG{n+na}{for}\PYG{o}{=}\PYG{l+s}{\PYGZdq{}8gb\PYGZdq{}}\PYG{p}{\PYGZgt{}}8 GB\PYG{p}{\PYGZlt{}}\PYG{p}{/}\PYG{n+nt}{label}\PYG{p}{\PYGZgt{}}
\PYG{p}{\PYGZlt{}}\PYG{n+nt}{input} \PYG{n+na}{type}\PYG{o}{=}\PYG{l+s}{\PYGZdq{}radio\PYGZdq{}}
       \PYG{n+na}{name}\PYG{o}{=}\PYG{l+s}{\PYGZdq{}memoria\PYGZdq{}} \PYG{n+na}{id}\PYG{o}{=}\PYG{l+s}{\PYGZdq{}8gb\PYGZdq{}}\PYG{p}{\PYGZgt{}}
\PYG{p}{\PYGZlt{}}\PYG{n+nt}{label} \PYG{n+na}{for}\PYG{o}{=}\PYG{l+s}{\PYGZdq{}bluray\PYGZdq{}}\PYG{p}{\PYGZgt{}}Blu\PYGZhy{}ray\PYG{p}{\PYGZlt{}}\PYG{p}{/}\PYG{n+nt}{label}\PYG{p}{\PYGZgt{}}
\PYG{p}{\PYGZlt{}}\PYG{n+nt}{input} \PYG{n+na}{type}\PYG{o}{=}\PYG{l+s}{\PYGZdq{}checkbox\PYGZdq{}} \PYG{n+na}{name}\PYG{o}{=}\PYG{l+s}{\PYGZdq{}extra[]\PYGZdq{}}
       \PYG{n+na}{id}\PYG{o}{=}\PYG{l+s}{\PYGZdq{}bluray\PYGZdq{}}\PYG{p}{\PYGZgt{}}
\PYG{p}{\PYGZlt{}}\PYG{n+nt}{label} \PYG{n+na}{for}\PYG{o}{=}\PYG{l+s}{\PYGZdq{}aceleradora\PYGZdq{}}\PYG{p}{\PYGZgt{}}Aceleradora\PYG{p}{\PYGZlt{}}\PYG{p}{/}\PYG{n+nt}{label}\PYG{p}{\PYGZgt{}}
\PYG{p}{\PYGZlt{}}\PYG{n+nt}{input} \PYG{n+na}{type}\PYG{o}{=}\PYG{l+s}{\PYGZdq{}checkbox\PYGZdq{}} \PYG{n+na}{name}\PYG{o}{=}\PYG{l+s}{\PYGZdq{}extra[]\PYGZdq{}}
       \PYG{n+na}{id}\PYG{o}{=}\PYG{l+s}{\PYGZdq{}aceleradora\PYGZdq{}}\PYG{p}{\PYGZgt{}}
\PYG{p}{\PYGZlt{}}\PYG{n+nt}{label} \PYG{n+na}{for}\PYG{o}{=}\PYG{l+s}{\PYGZdq{}monitor\PYGZdq{}}\PYG{p}{\PYGZgt{}}Monitor 25\PYG{p}{\PYGZlt{}}\PYG{p}{/}\PYG{n+nt}{label}\PYG{p}{\PYGZgt{}}
\PYG{p}{\PYGZlt{}}\PYG{n+nt}{input} \PYG{n+na}{type}\PYG{o}{=}\PYG{l+s}{\PYGZdq{}checkbox\PYGZdq{}} \PYG{n+na}{name}\PYG{o}{=}\PYG{l+s}{\PYGZdq{}extra[]\PYGZdq{}}
       \PYG{n+na}{id}\PYG{o}{=}\PYG{l+s}{\PYGZdq{}monitor\PYGZdq{}}\PYG{p}{\PYGZgt{}}
\PYG{p}{\PYGZlt{}}\PYG{n+nt}{input} \PYG{n+na}{type}\PYG{o}{=}\PYG{l+s}{\PYGZdq{}text\PYGZdq{}} \PYG{n+na}{id}\PYG{o}{=}\PYG{l+s}{\PYGZdq{}total\PYGZdq{}}\PYG{p}{\PYGZgt{}}
\end{sphinxVerbatim}


\subsection{Solución JS configurador}
\label{\detokenize{tema4:solucion-js-configurador}}
\begin{sphinxVerbatim}[commandchars=\\\{\}]
\PYG{n+nx}{\PYGZdl{}}\PYG{p}{(}\PYG{n+nb}{document}\PYG{p}{)}\PYG{p}{.}\PYG{n+nx}{ready}\PYG{p}{(}\PYG{n+nx}{main}\PYG{p}{)}

\PYG{k+kd}{function} \PYG{n+nx}{calcular\PYGZus{}precio}\PYG{p}{(}\PYG{p}{)}\PYG{p}{\PYGZob{}}
        \PYG{k+kd}{var} \PYG{n+nx}{precio}\PYG{o}{=}\PYG{l+m+mi}{0}
        \PYG{k}{if} \PYG{p}{(}\PYG{n+nx}{\PYGZdl{}}\PYG{p}{(}\PYG{l+s+s2}{\PYGZdq{}\PYGZsh{}intel\PYGZdq{}}\PYG{p}{)}\PYG{p}{.}\PYG{n+nx}{prop}\PYG{p}{(}\PYG{l+s+s2}{\PYGZdq{}checked\PYGZdq{}}\PYG{p}{)}\PYG{p}{)} \PYG{p}{\PYGZob{}}
                \PYG{n+nx}{precio}\PYG{o}{=}\PYG{n+nx}{precio}\PYG{o}{+}\PYG{l+m+mi}{250}
        \PYG{p}{\PYGZcb{}}
        \PYG{k}{if} \PYG{p}{(}\PYG{n+nx}{\PYGZdl{}}\PYG{p}{(}\PYG{l+s+s2}{\PYGZdq{}\PYGZsh{}amd\PYGZdq{}}\PYG{p}{)}\PYG{p}{.}\PYG{n+nx}{prop}\PYG{p}{(}\PYG{l+s+s2}{\PYGZdq{}checked\PYGZdq{}}\PYG{p}{)}\PYG{p}{)} \PYG{p}{\PYGZob{}}
                \PYG{n+nx}{precio}\PYG{o}{=}\PYG{n+nx}{precio}\PYG{o}{+}\PYG{l+m+mi}{210}
        \PYG{p}{\PYGZcb{}}
        \PYG{k}{if} \PYG{p}{(}\PYG{n+nx}{\PYGZdl{}}\PYG{p}{(}\PYG{l+s+s2}{\PYGZdq{}\PYGZsh{}2gb\PYGZdq{}}\PYG{p}{)}\PYG{p}{.}\PYG{n+nx}{prop}\PYG{p}{(}\PYG{l+s+s2}{\PYGZdq{}checked\PYGZdq{}}\PYG{p}{)}\PYG{p}{)} \PYG{p}{\PYGZob{}}
                \PYG{n+nx}{precio}\PYG{o}{=}\PYG{n+nx}{precio}\PYG{o}{+}\PYG{l+m+mi}{90}
        \PYG{p}{\PYGZcb{}}
        \PYG{k}{if} \PYG{p}{(}\PYG{n+nx}{\PYGZdl{}}\PYG{p}{(}\PYG{l+s+s2}{\PYGZdq{}\PYGZsh{}4gb\PYGZdq{}}\PYG{p}{)}\PYG{p}{.}\PYG{n+nx}{prop}\PYG{p}{(}\PYG{l+s+s2}{\PYGZdq{}checked\PYGZdq{}}\PYG{p}{)}\PYG{p}{)} \PYG{p}{\PYGZob{}}
                \PYG{n+nx}{precio}\PYG{o}{=}\PYG{n+nx}{precio}\PYG{o}{+}\PYG{l+m+mi}{140}
        \PYG{p}{\PYGZcb{}}
        \PYG{k}{if} \PYG{p}{(}\PYG{n+nx}{\PYGZdl{}}\PYG{p}{(}\PYG{l+s+s2}{\PYGZdq{}\PYGZsh{}8gb\PYGZdq{}}\PYG{p}{)}\PYG{p}{.}\PYG{n+nx}{prop}\PYG{p}{(}\PYG{l+s+s2}{\PYGZdq{}checked\PYGZdq{}}\PYG{p}{)}\PYG{p}{)} \PYG{p}{\PYGZob{}}
                \PYG{n+nx}{precio}\PYG{o}{=}\PYG{n+nx}{precio}\PYG{o}{+}\PYG{l+m+mi}{210}
        \PYG{p}{\PYGZcb{}}
        \PYG{k}{if} \PYG{p}{(}\PYG{n+nx}{\PYGZdl{}}\PYG{p}{(}\PYG{l+s+s2}{\PYGZdq{}\PYGZsh{}bluray\PYGZdq{}}\PYG{p}{)}\PYG{p}{.}\PYG{n+nx}{prop}\PYG{p}{(}\PYG{l+s+s2}{\PYGZdq{}checked\PYGZdq{}}\PYG{p}{)}\PYG{p}{)} \PYG{p}{\PYGZob{}}
                \PYG{n+nx}{precio}\PYG{o}{=}\PYG{n+nx}{precio}\PYG{o}{+}\PYG{l+m+mi}{190}
        \PYG{p}{\PYGZcb{}}
        \PYG{k}{if} \PYG{p}{(}\PYG{n+nx}{\PYGZdl{}}\PYG{p}{(}\PYG{l+s+s2}{\PYGZdq{}\PYGZsh{}aceleradora\PYGZdq{}}\PYG{p}{)}\PYG{p}{.}\PYG{n+nx}{prop}\PYG{p}{(}\PYG{l+s+s2}{\PYGZdq{}checked\PYGZdq{}}\PYG{p}{)}\PYG{p}{)} \PYG{p}{\PYGZob{}}
                \PYG{n+nx}{precio}\PYG{o}{=}\PYG{n+nx}{precio}\PYG{o}{+}\PYG{l+m+mi}{430}
        \PYG{p}{\PYGZcb{}}
        \PYG{k}{if} \PYG{p}{(}\PYG{n+nx}{\PYGZdl{}}\PYG{p}{(}\PYG{l+s+s2}{\PYGZdq{}\PYGZsh{}monitor\PYGZdq{}}\PYG{p}{)}\PYG{p}{.}\PYG{n+nx}{prop}\PYG{p}{(}\PYG{l+s+s2}{\PYGZdq{}checked\PYGZdq{}}\PYG{p}{)}\PYG{p}{)} \PYG{p}{\PYGZob{}}
                \PYG{n+nx}{precio}\PYG{o}{=}\PYG{n+nx}{precio}\PYG{o}{+}\PYG{l+m+mi}{185}
        \PYG{p}{\PYGZcb{}}

        \PYG{n+nx}{\PYGZdl{}}\PYG{p}{(}\PYG{l+s+s2}{\PYGZdq{}\PYGZsh{}total\PYGZdq{}}\PYG{p}{)}\PYG{p}{.}\PYG{n+nx}{val}\PYG{p}{(}\PYG{n+nx}{precio}\PYG{p}{)}
\PYG{p}{\PYGZcb{}}
\PYG{k+kd}{function} \PYG{n+nx}{main}\PYG{p}{(}\PYG{p}{)}\PYG{p}{\PYGZob{}}
        \PYG{n+nx}{\PYGZdl{}}\PYG{p}{(}\PYG{l+s+s2}{\PYGZdq{}\PYGZsh{}intel\PYGZdq{}}\PYG{p}{)}\PYG{p}{.}\PYG{n+nx}{click} \PYG{p}{(}\PYG{n+nx}{calcular\PYGZus{}precio}\PYG{p}{)}
        \PYG{n+nx}{\PYGZdl{}}\PYG{p}{(}\PYG{l+s+s2}{\PYGZdq{}\PYGZsh{}amd\PYGZdq{}}\PYG{p}{)}\PYG{p}{.}\PYG{n+nx}{click} \PYG{p}{(}\PYG{n+nx}{calcular\PYGZus{}precio}\PYG{p}{)}

        \PYG{n+nx}{\PYGZdl{}}\PYG{p}{(}\PYG{l+s+s2}{\PYGZdq{}\PYGZsh{}2gb\PYGZdq{}}\PYG{p}{)}\PYG{p}{.}\PYG{n+nx}{click} \PYG{p}{(}\PYG{n+nx}{calcular\PYGZus{}precio}\PYG{p}{)}
        \PYG{n+nx}{\PYGZdl{}}\PYG{p}{(}\PYG{l+s+s2}{\PYGZdq{}\PYGZsh{}4gb\PYGZdq{}}\PYG{p}{)}\PYG{p}{.}\PYG{n+nx}{click} \PYG{p}{(}\PYG{n+nx}{calcular\PYGZus{}precio}\PYG{p}{)}
        \PYG{n+nx}{\PYGZdl{}}\PYG{p}{(}\PYG{l+s+s2}{\PYGZdq{}\PYGZsh{}8gb\PYGZdq{}}\PYG{p}{)}\PYG{p}{.}\PYG{n+nx}{click} \PYG{p}{(}\PYG{n+nx}{calcular\PYGZus{}precio}\PYG{p}{)}

        \PYG{n+nx}{\PYGZdl{}}\PYG{p}{(}\PYG{l+s+s2}{\PYGZdq{}\PYGZsh{}bluray\PYGZdq{}}\PYG{p}{)}\PYG{p}{.}\PYG{n+nx}{click} \PYG{p}{(}\PYG{n+nx}{calcular\PYGZus{}precio}\PYG{p}{)}
        \PYG{n+nx}{\PYGZdl{}}\PYG{p}{(}\PYG{l+s+s2}{\PYGZdq{}\PYGZsh{}monitor\PYGZdq{}}\PYG{p}{)}\PYG{p}{.}\PYG{n+nx}{click} \PYG{p}{(}\PYG{n+nx}{calcular\PYGZus{}precio}\PYG{p}{)}
        \PYG{n+nx}{\PYGZdl{}}\PYG{p}{(}\PYG{l+s+s2}{\PYGZdq{}\PYGZsh{}aceleradora\PYGZdq{}}\PYG{p}{)}\PYG{p}{.}\PYG{n+nx}{click} \PYG{p}{(}\PYG{n+nx}{calcular\PYGZus{}precio}\PYG{p}{)}
\PYG{p}{\PYGZcb{}}
\end{sphinxVerbatim}


\subsection{Ejercicio configurador de PCs ampliado}
\label{\detokenize{tema4:ejercicio-configurador-de-pcs-ampliado}}
En el ejercicio anterior ocurre que por un problema hardware no es posible tener procesadores AMD con aceleradora, por lo que cuando se marque un AMD se debe desactivar el checkbox de la aceleradora y si hubiera una marca, también se debe desactivar y por supuesto recalcular el precio.


\section{Configurador de coches}
\label{\detokenize{tema4:configurador-de-coches}}
Un fabricante de automóviles desea ofrecer a sus clientes una aplicación que les permita configurar sus vehículos según sus preferencias y ver el precio final del coche. Los precios y las restricciones son los siguientes:
\begin{itemize}
\item {} 
Se pueden tener dos tipos de motor: gasolina (precio base 7000 euros) y diésel (precio base 8200).

\item {} 
Se pueden tener 3 potencias: 1100, 1800 y 2300 centímetros cúbicos. Los precios de cada uno son 800, 1900 y 2500. Sin embargo \sphinxstylestrong{no es posible fabricar motores diésel de 2300}.

\item {} 
Hay dos tipos de pintura: normal y metalizada. Los precios respectivos son 750 y 1580 euros.

\item {} 
Hay seis colores: negro, blanco, rojo, azul polar, verde turquesa y gris marengo. \sphinxstylestrong{No se pueden fabricar colores de pintura normal de ninguno de los tres últimos colores}.

\item {} 
Se dispone de diversos extras: alerón deportivo (190 euros \sphinxstylestrong{pero solo se puede elegir si se elige pintura metalizada}), radio-CD con MP3 (230 euros más), altavoces traseros (320 euros más, \sphinxstylestrong{pero solo si se elige antes el Radio-CD}), y GPS incorporado (520 euros más).

\end{itemize}

Crear la aplicación que respete las restricciones exigidas por el cliente.


\subsection{HTML del configurador}
\label{\detokenize{tema4:html-del-configurador}}

\subsection{JS del configurador (con JQuery (DAM))}
\label{\detokenize{tema4:js-del-configurador-con-jquery-dam}}

\section{Dinamismo con Google Maps}
\label{\detokenize{tema4:dinamismo-con-google-maps}}
Google Maps ofrece un servicio de mapas con una limitación de 25.000 peticiones diarias. El código básico sería así:


\subsection{HTML de GMaps}
\label{\detokenize{tema4:html-de-gmaps}}
\begin{sphinxVerbatim}[commandchars=\\\{\}]
\PYG{c+cp}{\PYGZlt{}!DOCTYPE html\PYGZgt{}}

\PYG{p}{\PYGZlt{}}\PYG{n+nt}{html}\PYG{p}{\PYGZgt{}}
\PYG{p}{\PYGZlt{}}\PYG{n+nt}{head}\PYG{p}{\PYGZgt{}}
        \PYG{c}{\PYGZlt{}!\PYGZhy{}\PYGZhy{}}\PYG{c}{En móviles poner la escala inicial a 1}\PYG{c}{\PYGZhy{}\PYGZhy{}\PYGZgt{}}
        \PYG{p}{\PYGZlt{}}\PYG{n+nt}{meta} \PYG{n+na}{name}\PYG{o}{=}\PYG{l+s}{\PYGZdq{}viewport\PYGZdq{}} \PYG{n+na}{content}\PYG{o}{=}\PYG{l+s}{\PYGZdq{}width=device\PYGZhy{}width, initial\PYGZhy{}scale=1.0\PYGZdq{}}\PYG{p}{\PYGZgt{}}
        \PYG{c}{\PYGZlt{}!\PYGZhy{}\PYGZhy{}}\PYG{c}{Cargamos los estilos y los efectos de Bootstrap}\PYG{c}{\PYGZhy{}\PYGZhy{}\PYGZgt{}}
        \PYG{p}{\PYGZlt{}}\PYG{n+nt}{link}
        \PYG{n+na}{rel}\PYG{o}{=}\PYG{l+s}{\PYGZdq{}stylesheet\PYGZdq{}}
        \PYG{n+na}{type}\PYG{o}{=}\PYG{l+s}{\PYGZdq{}text/css\PYGZdq{}} \PYG{n+na}{href}\PYG{o}{=}\PYG{l+s}{\PYGZdq{}bootstrap/css/bootstrap.css\PYGZdq{}}\PYG{p}{/}\PYG{p}{\PYGZgt{}}
        \PYG{p}{\PYGZlt{}}\PYG{n+nt}{script} \PYG{n+na}{src}\PYG{o}{=}\PYG{l+s}{\PYGZdq{}bootstrap/js/bootstrap.js\PYGZdq{}}\PYG{p}{\PYGZgt{}}\PYG{p}{\PYGZlt{}}\PYG{p}{/}\PYG{n+nt}{script}\PYG{p}{\PYGZgt{}}
        \PYG{p}{\PYGZlt{}}\PYG{n+nt}{script} \PYG{n+na}{src}\PYG{o}{=}\PYG{l+s}{\PYGZdq{}js/jquery.js\PYGZdq{}}\PYG{p}{\PYGZgt{}}\PYG{p}{\PYGZlt{}}\PYG{p}{/}\PYG{n+nt}{script}\PYG{p}{\PYGZgt{}}
        \PYG{p}{\PYGZlt{}}\PYG{n+nt}{script}
        \PYG{n+na}{src}\PYG{o}{=}\PYG{l+s}{\PYGZdq{}http://maps.googleapis.com/maps/api/js?key=AIzaSyDpv9zCj9szIIu\PYGZhy{}\PYGZhy{}LuNmDsry2fZCRrOqfY\PYGZam{}sensor=false\PYGZdq{}}\PYG{p}{\PYGZgt{}}

        \PYG{p}{\PYGZlt{}}\PYG{p}{/}\PYG{n+nt}{script}\PYG{p}{\PYGZgt{}}
        \PYG{p}{\PYGZlt{}}\PYG{n+nt}{style}\PYG{p}{\PYGZgt{}}
                \PYG{p}{\PYGZsh{}}\PYG{n+nn}{mapa}\PYG{p}{\PYGZob{}}
                        \PYG{k}{width}\PYG{p}{:} \PYG{l+m+mi}{500}\PYG{k+kt}{px}\PYG{p}{;}
                        \PYG{k}{height}\PYG{p}{:} \PYG{l+m+mi}{500}\PYG{k+kt}{px}\PYG{p}{;}
                        \PYG{k}{float}\PYG{p}{:} \PYG{k+kc}{right}\PYG{p}{;}
                        \PYG{k}{background\PYGZhy{}color}\PYG{p}{:} \PYG{n+nb}{rgb}\PYG{p}{(}\PYG{l+m+mi}{180}\PYG{p}{,}\PYG{l+m+mi}{190}\PYG{p}{,}\PYG{l+m+mi}{240}\PYG{p}{)}\PYG{p}{;}
                \PYG{p}{\PYGZcb{}}
                \PYG{p}{\PYGZsh{}}\PYG{n+nn}{controles}\PYG{p}{\PYGZob{}}
                        \PYG{k}{float}\PYG{p}{:} \PYG{k+kc}{left}\PYG{p}{;}
                \PYG{p}{\PYGZcb{}}
        \PYG{p}{\PYGZlt{}}\PYG{p}{/}\PYG{n+nt}{style}\PYG{p}{\PYGZgt{}}
        \PYG{p}{\PYGZlt{}}\PYG{n+nt}{script} \PYG{n+na}{src}\PYG{o}{=}\PYG{l+s}{\PYGZdq{}js/programa.js\PYGZdq{}}\PYG{p}{\PYGZgt{}}\PYG{p}{\PYGZlt{}}\PYG{p}{/}\PYG{n+nt}{script}\PYG{p}{\PYGZgt{}}
        \PYG{p}{\PYGZlt{}}\PYG{n+nt}{title}\PYG{p}{\PYGZgt{}}Plantilla JQuery\PYG{p}{\PYGZlt{}}\PYG{p}{/}\PYG{n+nt}{title}\PYG{p}{\PYGZgt{}}
\PYG{p}{\PYGZlt{}}\PYG{p}{/}\PYG{n+nt}{head}\PYG{p}{\PYGZgt{}}

\PYG{p}{\PYGZlt{}}\PYG{n+nt}{body}\PYG{p}{\PYGZgt{}}
\PYG{p}{\PYGZlt{}}\PYG{n+nt}{div} \PYG{n+na}{class}\PYG{o}{=}\PYG{l+s}{\PYGZdq{}container\PYGZdq{}}\PYG{p}{\PYGZgt{}}
        \PYG{p}{\PYGZlt{}}\PYG{n+nt}{h1}\PYG{p}{\PYGZgt{}}Javascript con mapas\PYG{p}{\PYGZlt{}}\PYG{p}{/}\PYG{n+nt}{h1}\PYG{p}{\PYGZgt{}}
                \PYG{p}{\PYGZlt{}}\PYG{n+nt}{div} \PYG{n+na}{id}\PYG{o}{=}\PYG{l+s}{\PYGZdq{}controles\PYGZdq{}}\PYG{p}{\PYGZgt{}}
                        Introduzca latitud:\PYG{p}{\PYGZlt{}}\PYG{n+nt}{input} \PYG{n+na}{type}\PYG{o}{=}\PYG{l+s}{\PYGZdq{}text\PYGZdq{}} \PYG{n+na}{id}\PYG{o}{=}\PYG{l+s}{\PYGZdq{}latitud\PYGZdq{}}\PYG{p}{\PYGZgt{}}
                        \PYG{p}{\PYGZlt{}}\PYG{n+nt}{br}\PYG{p}{/}\PYG{p}{\PYGZgt{}}
                        Introduzca longitud:\PYG{p}{\PYGZlt{}}\PYG{n+nt}{input} \PYG{n+na}{type}\PYG{o}{=}\PYG{l+s}{\PYGZdq{}text\PYGZdq{}} \PYG{n+na}{id}\PYG{o}{=}\PYG{l+s}{\PYGZdq{}longitud\PYGZdq{}}\PYG{p}{\PYGZgt{}}
                        \PYG{p}{\PYGZlt{}}\PYG{n+nt}{br}\PYG{p}{/}\PYG{p}{\PYGZgt{}}
                        \PYG{p}{\PYGZlt{}}\PYG{n+nt}{input} \PYG{n+na}{type}\PYG{o}{=}\PYG{l+s}{\PYGZdq{}submit\PYGZdq{}} \PYG{n+na}{id}\PYG{o}{=}\PYG{l+s}{\PYGZdq{}mover\PYGZdq{}} \PYG{n+na}{value}\PYG{o}{=}\PYG{l+s}{\PYGZdq{}¡Viajar!\PYGZdq{}}\PYG{p}{\PYGZgt{}}
                        \PYG{p}{\PYGZlt{}}\PYG{n+nt}{select} \PYG{n+na}{id}\PYG{o}{=}\PYG{l+s}{\PYGZdq{}ciudades\PYGZdq{}}\PYG{p}{\PYGZgt{}}
                                \PYG{p}{\PYGZlt{}}\PYG{n+nt}{option} \PYG{n+na}{value}\PYG{o}{=}\PYG{l+s}{\PYGZdq{}CR\PYGZdq{}}\PYG{p}{\PYGZgt{}}Ir a Ciudad Real\PYG{p}{\PYGZlt{}}\PYG{p}{/}\PYG{n+nt}{option}\PYG{p}{\PYGZgt{}}
                                \PYG{p}{\PYGZlt{}}\PYG{n+nt}{option} \PYG{n+na}{value}\PYG{o}{=}\PYG{l+s}{\PYGZdq{}BA\PYGZdq{}}\PYG{p}{\PYGZgt{}}Ir a Barcelona\PYG{p}{\PYGZlt{}}\PYG{p}{/}\PYG{n+nt}{option}\PYG{p}{\PYGZgt{}}
                                \PYG{p}{\PYGZlt{}}\PYG{n+nt}{option} \PYG{n+na}{value}\PYG{o}{=}\PYG{l+s}{\PYGZdq{}PO\PYGZdq{}}\PYG{p}{\PYGZgt{}}Ir a Pontevedra\PYG{p}{\PYGZlt{}}\PYG{p}{/}\PYG{n+nt}{option}\PYG{p}{\PYGZgt{}}
                        \PYG{p}{\PYGZlt{}}\PYG{p}{/}\PYG{n+nt}{select}\PYG{p}{\PYGZgt{}}
                        \PYG{p}{\PYGZlt{}}\PYG{n+nt}{br}\PYG{p}{/}\PYG{p}{\PYGZgt{}}
                        Calculador de distancias desde CR a otras ciudades
                        \PYG{p}{\PYGZlt{}}\PYG{n+nt}{select} \PYG{n+na}{id}\PYG{o}{=}\PYG{l+s}{\PYGZdq{}ciudades\PYGZdq{}}\PYG{p}{\PYGZgt{}}
                                \PYG{p}{\PYGZlt{}}\PYG{n+nt}{option} \PYG{n+na}{value}\PYG{o}{=}\PYG{l+s}{\PYGZdq{}CR\PYGZdq{}}\PYG{p}{\PYGZgt{}}Ir a Ciudad Real\PYG{p}{\PYGZlt{}}\PYG{p}{/}\PYG{n+nt}{option}\PYG{p}{\PYGZgt{}}
                                \PYG{p}{\PYGZlt{}}\PYG{n+nt}{option} \PYG{n+na}{value}\PYG{o}{=}\PYG{l+s}{\PYGZdq{}BA\PYGZdq{}}\PYG{p}{\PYGZgt{}}Ir a Barcelona\PYG{p}{\PYGZlt{}}\PYG{p}{/}\PYG{n+nt}{option}\PYG{p}{\PYGZgt{}}
                                \PYG{p}{\PYGZlt{}}\PYG{n+nt}{option} \PYG{n+na}{value}\PYG{o}{=}\PYG{l+s}{\PYGZdq{}PO\PYGZdq{}}\PYG{p}{\PYGZgt{}}Ir a Pontevedra\PYG{p}{\PYGZlt{}}\PYG{p}{/}\PYG{n+nt}{option}\PYG{p}{\PYGZgt{}}
                        \PYG{p}{\PYGZlt{}}\PYG{p}{/}\PYG{n+nt}{select}\PYG{p}{\PYGZgt{}}
        \PYG{p}{\PYGZlt{}}\PYG{p}{/}\PYG{n+nt}{div}\PYG{p}{\PYGZgt{}}
        \PYG{p}{\PYGZlt{}}\PYG{n+nt}{div} \PYG{n+na}{id}\PYG{o}{=}\PYG{l+s}{\PYGZdq{}mapa\PYGZdq{}}\PYG{p}{\PYGZgt{}}

        \PYG{p}{\PYGZlt{}}\PYG{p}{/}\PYG{n+nt}{div}\PYG{p}{\PYGZgt{}}
\PYG{p}{\PYGZlt{}}\PYG{p}{/}\PYG{n+nt}{div}\PYG{p}{\PYGZgt{}}

\PYG{p}{\PYGZlt{}}\PYG{p}{/}\PYG{n+nt}{body}\PYG{p}{\PYGZgt{}}
\PYG{p}{\PYGZlt{}}\PYG{p}{/}\PYG{n+nt}{html}\PYG{p}{\PYGZgt{}}
\end{sphinxVerbatim}


\subsection{Javascript de GMaps}
\label{\detokenize{tema4:javascript-de-gmaps}}
\begin{sphinxVerbatim}[commandchars=\\\{\}]
\PYG{k+kd}{var} \PYG{n+nx}{latitud}\PYG{o}{=}\PYG{l+m+mf}{38.59}
\PYG{k+kd}{var} \PYG{n+nx}{longitud}\PYG{o}{=}\PYG{o}{\PYGZhy{}}\PYG{l+m+mf}{3.55}
\PYG{k+kd}{var} \PYG{n+nx}{mi\PYGZus{}nivel\PYGZus{}de\PYGZus{}zoom}\PYG{o}{=}\PYG{l+m+mi}{8}
\PYG{k+kd}{var} \PYG{n+nx}{obj\PYGZus{}mapa}
\PYG{k+kd}{function} \PYG{n+nx}{inicio}\PYG{p}{(}\PYG{p}{)}\PYG{p}{\PYGZob{}}
        \PYG{k+kd}{var} \PYG{n+nx}{div\PYGZus{}mapa}\PYG{o}{=}\PYG{n+nb}{document}\PYG{p}{.}\PYG{n+nx}{getElementById}\PYG{p}{(}\PYG{l+s+s2}{\PYGZdq{}mapa\PYGZdq{}}\PYG{p}{)}
        \PYG{k+kd}{var} \PYG{n+nx}{obj\PYGZus{}coordenadas}\PYG{o}{=}\PYG{k}{new} \PYG{n+nx}{google}\PYG{p}{.}\PYG{n+nx}{maps}\PYG{p}{.}\PYG{n+nx}{LatLng}\PYG{p}{(}\PYG{n+nx}{latitud}\PYG{p}{,}\PYG{n+nx}{longitud}\PYG{p}{)}
        \PYG{k+kd}{var} \PYG{n+nx}{obj\PYGZus{}opciones}\PYG{o}{=}\PYG{p}{\PYGZob{}}
                \PYG{n+nx}{center}\PYG{o}{:}\PYG{n+nx}{obj\PYGZus{}coordenadas}\PYG{p}{,}
                \PYG{n+nx}{zoom}\PYG{o}{:}\PYG{n+nx}{mi\PYGZus{}nivel\PYGZus{}de\PYGZus{}zoom}
        \PYG{p}{\PYGZcb{}}
        \PYG{n+nx}{obj\PYGZus{}mapa}\PYG{o}{=}\PYG{k}{new} \PYG{n+nx}{google}\PYG{p}{.}\PYG{n+nx}{maps}\PYG{p}{.}\PYG{n+nx}{Map}\PYG{p}{(}\PYG{n+nx}{div\PYGZus{}mapa}\PYG{p}{,} \PYG{n+nx}{obj\PYGZus{}opciones}\PYG{p}{)}

        \PYG{n+nx}{\PYGZdl{}}\PYG{p}{(}\PYG{l+s+s2}{\PYGZdq{}\PYGZsh{}mover\PYGZdq{}}\PYG{p}{)}\PYG{p}{.}\PYG{n+nx}{click} \PYG{p}{(}\PYG{n+nx}{mover\PYGZus{}el\PYGZus{}mapa}\PYG{p}{)}
\PYG{p}{\PYGZcb{}}


\PYG{k+kd}{function} \PYG{n+nx}{mover\PYGZus{}el\PYGZus{}mapa}\PYG{p}{(}\PYG{p}{)} \PYG{p}{\PYGZob{}}
        \PYG{k+kd}{var} \PYG{n+nx}{obj\PYGZus{}latitud}\PYG{o}{=}\PYG{n+nx}{\PYGZdl{}}\PYG{p}{(}\PYG{l+s+s2}{\PYGZdq{}\PYGZsh{}latitud\PYGZdq{}}\PYG{p}{)}
        \PYG{k+kd}{var} \PYG{n+nx}{valor\PYGZus{}latitud}\PYG{o}{=}\PYG{n+nx}{obj\PYGZus{}latitud}\PYG{p}{.}\PYG{n+nx}{val}\PYG{p}{(}\PYG{p}{)}

        \PYG{k+kd}{var} \PYG{n+nx}{valor\PYGZus{}longitud}\PYG{o}{=}\PYG{n+nx}{\PYGZdl{}}\PYG{p}{(}\PYG{l+s+s2}{\PYGZdq{}\PYGZsh{}longitud\PYGZdq{}}\PYG{p}{)}\PYG{p}{.}\PYG{n+nx}{val}\PYG{p}{(}\PYG{p}{)}
        \PYG{k+kd}{var} \PYG{n+nx}{nuevas\PYGZus{}coordenadas}\PYG{o}{=}\PYG{k}{new} \PYG{n+nx}{google}\PYG{p}{.}\PYG{n+nx}{maps}\PYG{p}{.}\PYG{n+nx}{LatLng}\PYG{p}{(}
                                \PYG{n+nx}{valor\PYGZus{}latitud}\PYG{p}{,} \PYG{n+nx}{valor\PYGZus{}longitud}\PYG{p}{)}
        \PYG{n+nx}{obj\PYGZus{}mapa}\PYG{p}{.}\PYG{n+nx}{panTo}\PYG{p}{(}\PYG{n+nx}{nuevas\PYGZus{}coordenadas}\PYG{p}{)}
\PYG{p}{\PYGZcb{}}
\end{sphinxVerbatim}


\subsection{Ejercicio}
\label{\detokenize{tema4:id7}}
Ampliar el programa con una lista de ciudades del mundo. Cuando el usuario elija una de ellas, nuestro programa nos dirá la distancia desde Ciudad Real a dichas ciudades. Considerar las siguientes coordenadas en formato (latitud, longitud):
\begin{itemize}
\item {} 
Ciudad Real: (38.59, -3.55)

\item {} 
Nueva York: (40.73, -73.87)

\item {} 
Sidney: (-33.90, 151.13)

\item {} 
Berlin: (52.31, 13.39)

\item {} 
París: (48.85, 2.35)

\end{itemize}

Para poder conseguir esto, hay que modificar la URL de carga de GoogleMaps para solicitar que se cargue una biblioteca que nos ayudará a resolver este punto. En concreto, ahora pasaremos un parámetro \sphinxcode{libraries} con el valor \sphinxcode{geometry} que nos permitirá utilizar la biblioteca en concreto. Ahora el HTML es así:

\begin{sphinxVerbatim}[commandchars=\\\{\}]
    \PYG{p}{\PYGZlt{}}\PYG{n+nt}{script} \PYG{n+na}{src}\PYG{o}{=}\PYG{l+s}{\PYGZdq{}http://maps.googleapis.com/maps/api/js?key=AIzaSyDpv9zCj9szIIu\PYGZhy{}\PYGZhy{}LuNmDsry2fZCRrOqfY\PYGZam{}sensor=false\PYGZam{}libraries=geometry\PYGZdq{}}\PYG{p}{\PYGZgt{}}

\PYG{p}{\PYGZlt{}}\PYG{p}{/}\PYG{n+nt}{script}\PYG{p}{\PYGZgt{}}
\end{sphinxVerbatim}

Ahora una función que nos calcula la distancia sería algo como esto:

\begin{sphinxVerbatim}[commandchars=\\\{\}]
\PYG{c+cm}{/* Nos da la distancia en metros entre CR}
\PYG{c+cm}{ * y el punto (latitud\PYGZus{}destino,longitud\PYGZus{}destino)}
\PYG{c+cm}{ * (abreviados lat\PYGZus{}dest y lng\PYGZus{}dest)*/}
\PYG{k+kd}{function} \PYG{n+nx}{distancia}\PYG{p}{(}\PYG{n+nx}{lat\PYGZus{}dest}\PYG{p}{,} \PYG{n+nx}{lng\PYGZus{}dest}\PYG{p}{)}
\PYG{p}{\PYGZob{}}
        \PYG{k+kd}{var} \PYG{n+nx}{latitud\PYGZus{}cr}\PYG{o}{=}\PYG{l+m+mf}{38.59}
        \PYG{k+kd}{var} \PYG{n+nx}{longitud\PYGZus{}cr}\PYG{o}{=}\PYG{o}{\PYGZhy{}}\PYG{l+m+mf}{3.55}
        \PYG{k+kd}{var} \PYG{n+nx}{coords\PYGZus{}origen}\PYG{o}{=}\PYG{k}{new} \PYG{n+nx}{google}\PYG{p}{.}\PYG{n+nx}{maps}\PYG{p}{.}\PYG{n+nx}{LatLng}\PYG{p}{(}
                \PYG{n+nx}{latitud\PYGZus{}cr}\PYG{p}{,} \PYG{n+nx}{longitud\PYGZus{}cr}\PYG{p}{)}
        \PYG{k+kd}{var} \PYG{n+nx}{coords\PYGZus{}destino}\PYG{o}{=}\PYG{k}{new} \PYG{n+nx}{google}\PYG{p}{.}\PYG{n+nx}{maps}\PYG{p}{.}\PYG{n+nx}{LatLng}\PYG{p}{(}
                \PYG{n+nx}{lat\PYGZus{}dest}\PYG{p}{,} \PYG{n+nx}{lng\PYGZus{}dest}\PYG{p}{)}
        \PYG{k+kd}{var} \PYG{n+nx}{distancia}\PYG{o}{=}\PYG{n+nx}{google}\PYG{p}{.}\PYG{n+nx}{maps}\PYG{p}{.}\PYG{n+nx}{geometry}\PYG{p}{.}\PYG{n+nx}{spherical}\PYG{p}{.}\PYG{n+nx}{computeDistanceBetween}\PYG{p}{(}
                \PYG{n+nx}{coords\PYGZus{}origen}\PYG{p}{,} \PYG{n+nx}{coords\PYGZus{}destino}
        \PYG{p}{)}
        \PYG{k}{return} \PYG{n+nx}{distancia}
\PYG{p}{\PYGZcb{}}
\end{sphinxVerbatim}


\subsection{Ejercicio}
\label{\detokenize{tema4:id8}}
La empresa Automobile Creation for Millenium Enterprise (ACME) planea lanzar una página web en la que se permita al usuario configurar los coches a su medida, ofreciendo las distintas opciones en pantalla para que el usuario las elija. Sin embargo no todas las combinaciones se permiten en fábrica por lo que deberán tenerse en cuentas las siguientes


\subsubsection{Especificaciones}
\label{\detokenize{tema4:especificaciones}}\begin{itemize}
\item {} 
Hay dos motores: gasolina (5000) y diésel (6800)

\item {} 
Hay dos carrocerías: monovolumen (4500) y berlina (3700)

\item {} 
Hay tres accesorios: radio-cd con MP3 (180), alerones deportivos (220) y llantas de aleación (200)

\end{itemize}

Por diversos problemas, no es posible combinar las siguientes opciones:
\begin{itemize}
\item {} 
No se pueden tener berlinas de gasolina.

\item {} 
No se puede integrar el alerón en los monovolúmenes.

\item {} 
No se puede poner el radio-cd a los monovolúmenes.

\end{itemize}

Cuando se marque cualquiera de estas opciones, hay que limpiar todo el configurados y avisar de que no se puede hacer eso. No se pueden usar \sphinxcode{alerts}

\begin{sphinxVerbatim}[commandchars=\\\{\}]
\PYG{k+kd}{var} \PYG{n+nx}{vector\PYGZus{}ids}\PYG{o}{=}\PYG{p}{[}\PYG{l+s+s2}{\PYGZdq{}\PYGZsh{}gasolina\PYGZdq{}}\PYG{p}{,} \PYG{l+s+s2}{\PYGZdq{}\PYGZsh{}diesel\PYGZdq{}}\PYG{p}{,} \PYG{l+s+s2}{\PYGZdq{}\PYGZsh{}monovolumen\PYGZdq{}}\PYG{p}{,}
                \PYG{l+s+s2}{\PYGZdq{}\PYGZsh{}berlina\PYGZdq{}}\PYG{p}{,} \PYG{l+s+s2}{\PYGZdq{}\PYGZsh{}radiocd\PYGZdq{}}\PYG{p}{,} \PYG{l+s+s2}{\PYGZdq{}\PYGZsh{}alerones\PYGZdq{}}\PYG{p}{,} \PYG{l+s+s2}{\PYGZdq{}\PYGZsh{}llantas\PYGZdq{}}\PYG{p}{]}
\PYG{k+kd}{var} \PYG{n+nx}{precios}\PYG{o}{=}\PYG{p}{[}\PYG{l+m+mi}{5000}\PYG{p}{,} \PYG{l+m+mi}{6800}\PYG{p}{,} \PYG{l+m+mi}{4500}\PYG{p}{,}
                \PYG{l+m+mi}{3700}\PYG{p}{,} \PYG{l+m+mi}{180}\PYG{p}{,} \PYG{l+m+mi}{220}\PYG{p}{,} \PYG{l+m+mi}{200}\PYG{p}{]}

\PYG{k+kd}{function} \PYG{n+nx}{inicio}\PYG{p}{(}\PYG{p}{)}\PYG{p}{\PYGZob{}}

        \PYG{k}{for} \PYG{p}{(}\PYG{k+kd}{var} \PYG{n+nx}{pos} \PYG{k}{in} \PYG{n+nx}{vector\PYGZus{}ids}\PYG{p}{)}\PYG{p}{\PYGZob{}}
                \PYG{k+kd}{var} \PYG{n+nx}{el\PYGZus{}id}\PYG{o}{=}\PYG{n+nx}{vector\PYGZus{}ids}\PYG{p}{[}\PYG{n+nx}{pos}\PYG{p}{]}
                \PYG{n+nx}{\PYGZdl{}}\PYG{p}{(}\PYG{n+nx}{el\PYGZus{}id}\PYG{p}{)}\PYG{p}{.}\PYG{n+nx}{click} \PYG{p}{(} \PYG{n+nx}{calcularPrecio} \PYG{p}{)}
        \PYG{p}{\PYGZcb{}}
\PYG{p}{\PYGZcb{}}

\PYG{k+kd}{function} \PYG{n+nx}{cocheEsFabricable}\PYG{p}{(}\PYG{p}{)} \PYG{p}{\PYGZob{}}
        \PYG{c+c1}{//Caso 1: nada de berlinas de gasolina}
        \PYG{k+kd}{var} \PYG{n+nx}{marcada\PYGZus{}la\PYGZus{}berlina}\PYG{o}{=}\PYG{n+nx}{\PYGZdl{}}\PYG{p}{(}\PYG{l+s+s2}{\PYGZdq{}\PYGZsh{}berlina\PYGZdq{}}\PYG{p}{)}\PYG{p}{.}\PYG{n+nx}{prop}\PYG{p}{(}\PYG{l+s+s2}{\PYGZdq{}checked\PYGZdq{}}\PYG{p}{)}
        \PYG{k+kd}{var} \PYG{n+nx}{marcada\PYGZus{}la\PYGZus{}gasolina}\PYG{o}{=}\PYG{n+nx}{\PYGZdl{}}\PYG{p}{(}\PYG{l+s+s2}{\PYGZdq{}\PYGZsh{}gasolina\PYGZdq{}}\PYG{p}{)}\PYG{p}{.}\PYG{n+nx}{prop}\PYG{p}{(}\PYG{l+s+s2}{\PYGZdq{}checked\PYGZdq{}}\PYG{p}{)}
        \PYG{k}{if} \PYG{p}{(}\PYG{n+nx}{marcada\PYGZus{}la\PYGZus{}berlina} \PYG{o}{\PYGZam{}\PYGZam{}} \PYG{n+nx}{marcada\PYGZus{}la\PYGZus{}gasolina}\PYG{p}{)} \PYG{p}{\PYGZob{}}
                \PYG{n+nx}{alert} \PYG{p}{(}\PYG{l+s+s2}{\PYGZdq{}No se pueden fabricar berlinas de gasolina\PYGZdq{}}\PYG{p}{)}
                \PYG{k}{return} \PYG{k+kc}{false}
        \PYG{p}{\PYGZcb{}}

        \PYG{k+kd}{var} \PYG{n+nx}{marcado\PYGZus{}aleron}\PYG{o}{=}\PYG{n+nx}{\PYGZdl{}}\PYG{p}{(}\PYG{l+s+s2}{\PYGZdq{}\PYGZsh{}alerones\PYGZdq{}}\PYG{p}{)}\PYG{p}{.}\PYG{n+nx}{prop}\PYG{p}{(}\PYG{l+s+s2}{\PYGZdq{}checked\PYGZdq{}}\PYG{p}{)}
        \PYG{k+kd}{var} \PYG{n+nx}{marcado\PYGZus{}monovolumen}\PYG{o}{=}\PYG{n+nx}{\PYGZdl{}}\PYG{p}{(}\PYG{l+s+s2}{\PYGZdq{}\PYGZsh{}monovolumen\PYGZdq{}}\PYG{p}{)}\PYG{p}{.}\PYG{n+nx}{prop}\PYG{p}{(}\PYG{l+s+s2}{\PYGZdq{}checked\PYGZdq{}}\PYG{p}{)}
        \PYG{k}{if} \PYG{p}{(}\PYG{n+nx}{marcado\PYGZus{}aleron} \PYG{o}{\PYGZam{}\PYGZam{}} \PYG{n+nx}{marcado\PYGZus{}monovolumen} \PYG{p}{)} \PYG{p}{\PYGZob{}}
                \PYG{n+nx}{alert} \PYG{p}{(}\PYG{l+s+s2}{\PYGZdq{}No podemos integrar los alerones en monovolúmenes\PYGZdq{}}\PYG{p}{)}
                \PYG{k}{return} \PYG{k+kc}{false}
        \PYG{p}{\PYGZcb{}}

        \PYG{k+kd}{var} \PYG{n+nx}{marcado\PYGZus{}radiocd}\PYG{o}{=}\PYG{n+nx}{\PYGZdl{}}\PYG{p}{(}\PYG{l+s+s2}{\PYGZdq{}\PYGZsh{}radiocd\PYGZdq{}}\PYG{p}{)}\PYG{p}{.}\PYG{n+nx}{prop}\PYG{p}{(}\PYG{l+s+s2}{\PYGZdq{}checked\PYGZdq{}}\PYG{p}{)}
        \PYG{k}{if} \PYG{p}{(}\PYG{n+nx}{marcado\PYGZus{}radiocd} \PYG{o}{\PYGZam{}\PYGZam{}} \PYG{n+nx}{marcado\PYGZus{}monovolumen}\PYG{p}{)} \PYG{p}{\PYGZob{}}
                \PYG{n+nx}{alert} \PYG{p}{(}\PYG{l+s+s2}{\PYGZdq{}No podemos fabricar un monovol. con radio\PYGZhy{}cd\PYGZdq{}}\PYG{p}{)}
                \PYG{k}{return} \PYG{k+kc}{false}
        \PYG{p}{\PYGZcb{}}
        \PYG{k}{return} \PYG{k+kc}{true}
\PYG{p}{\PYGZcb{}}
\PYG{c+cm}{/* Calcula el precio del coche en función de lo que esté marcado}
\PYG{c+cm}{ * y lo que no.*/}
\PYG{k+kd}{function} \PYG{n+nx}{calcularPrecio}\PYG{p}{(}\PYG{p}{)} \PYG{p}{\PYGZob{}}
        \PYG{k+kd}{var} \PYG{n+nx}{todo\PYGZus{}bien}\PYG{o}{=}\PYG{n+nx}{cocheEsFabricable}\PYG{p}{(}\PYG{p}{)}
        \PYG{k}{if} \PYG{p}{(}\PYG{n+nx}{todo\PYGZus{}bien}\PYG{o}{!=}\PYG{k+kc}{true}\PYG{p}{)} \PYG{p}{\PYGZob{}}
                \PYG{k}{return}
        \PYG{p}{\PYGZcb{}}
        \PYG{k+kd}{var} \PYG{n+nx}{precioCoche}\PYG{o}{=}\PYG{l+m+mi}{0}
        \PYG{k}{for} \PYG{p}{(}\PYG{k+kd}{var} \PYG{n+nx}{pos} \PYG{k}{in} \PYG{n+nx}{vector\PYGZus{}ids}\PYG{p}{)}\PYG{p}{\PYGZob{}}
                \PYG{k+kd}{var} \PYG{n+nx}{el\PYGZus{}id}\PYG{o}{=}\PYG{n+nx}{vector\PYGZus{}ids}\PYG{p}{[}\PYG{n+nx}{pos}\PYG{p}{]}
                \PYG{k}{if} \PYG{p}{(}\PYG{n+nx}{\PYGZdl{}}\PYG{p}{(}\PYG{n+nx}{el\PYGZus{}id}\PYG{p}{)}\PYG{p}{.}\PYG{n+nx}{prop}\PYG{p}{(}\PYG{l+s+s2}{\PYGZdq{}checked\PYGZdq{}}\PYG{p}{)}\PYG{p}{)} \PYG{p}{\PYGZob{}}
                        \PYG{k+kd}{var} \PYG{n+nx}{precio\PYGZus{}accesorio}\PYG{o}{=}\PYG{n+nx}{precios}\PYG{p}{[}\PYG{n+nx}{pos}\PYG{p}{]}
                        \PYG{n+nx}{precioCoche}\PYG{o}{=}\PYG{n+nx}{precioCoche}\PYG{o}{+}\PYG{n+nx}{precio\PYGZus{}accesorio}
                \PYG{p}{\PYGZcb{}}
        \PYG{p}{\PYGZcb{}}
        \PYG{n+nx}{alert} \PYG{p}{(}\PYG{l+s+s2}{\PYGZdq{}El precio es:\PYGZdq{}}\PYG{o}{+}\PYG{n+nx}{precioCoche}\PYG{p}{)}
\PYG{p}{\PYGZcb{}}
\end{sphinxVerbatim}


\subsection{Ampliación}
\label{\detokenize{tema4:ampliacion}}
Se desea que el usuario puede elegir entre los siguientes colores con los siguientes precios:
\begin{itemize}
\item {} 
Blanco: 700 euros

\item {} 
Rojo, Verde y Azul básicos: 950

\item {} 
Gris, Negro y Naranja: 1400 euros por ser colores metalizados

\end{itemize}

Además, se desea ver una muestra de color en algún punto de la página. Para lograrlo se necesitará utilizar un método que proporciona JQuery y que se llama \sphinxcode{addClass}


\subsubsection{Solución HTML}
\label{\detokenize{tema4:solucion-html}}
\begin{sphinxVerbatim}[commandchars=\\\{\}]
\PYG{p}{\PYGZlt{}}\PYG{n+nt}{head}\PYG{p}{\PYGZgt{}}
        \PYG{p}{\PYGZlt{}}\PYG{n+nt}{style}\PYG{p}{\PYGZgt{}}
                \PYG{p}{.}\PYG{n+nc}{muestrarojo}\PYG{p}{\PYGZob{}}
                        \PYG{k}{background\PYGZhy{}color}\PYG{p}{:}\PYG{k+kc}{red}\PYG{p}{;}
                \PYG{p}{\PYGZcb{}}
                \PYG{p}{.}\PYG{n+nc}{muestraverde}\PYG{p}{\PYGZob{}}
                        \PYG{k}{background\PYGZhy{}color}\PYG{p}{:} \PYG{k+kc}{green}\PYG{p}{;}
                \PYG{p}{\PYGZcb{}}
                \PYG{p}{.}\PYG{n+nc}{muestraazul}\PYG{p}{\PYGZob{}}
                        \PYG{k}{background\PYGZhy{}color}\PYG{p}{:} \PYG{k+kc}{blue}\PYG{p}{;}
                \PYG{p}{\PYGZcb{}}
                \PYG{p}{.}\PYG{n+nc}{muestragris}\PYG{p}{\PYGZob{}}
                        \PYG{k}{background\PYGZhy{}color}\PYG{p}{:} \PYG{k+kc}{grey}\PYG{p}{;}
                \PYG{p}{\PYGZcb{}}
                \PYG{p}{.}\PYG{n+nc}{muestrablanco}\PYG{p}{\PYGZob{}}
                        \PYG{k}{background\PYGZhy{}color}\PYG{p}{:} \PYG{k+kc}{white}\PYG{p}{;}
                \PYG{p}{\PYGZcb{}}
        \PYG{p}{\PYGZlt{}}\PYG{p}{/}\PYG{n+nt}{style}\PYG{p}{\PYGZgt{}}
\PYG{p}{\PYGZlt{}}\PYG{p}{/}\PYG{n+nt}{head}\PYG{p}{\PYGZgt{}}
\PYG{p}{\PYGZlt{}}\PYG{n+nt}{body}\PYG{p}{\PYGZgt{}}
        \PYG{p}{\PYGZlt{}}\PYG{n+nt}{div} \PYG{n+na}{class}\PYG{o}{=}\PYG{l+s}{\PYGZdq{}container\PYGZdq{}}\PYG{p}{\PYGZgt{}}
                \PYG{p}{\PYGZlt{}}\PYG{n+nt}{h1}\PYG{p}{\PYGZgt{}}Motores\PYG{p}{\PYGZlt{}}\PYG{p}{/}\PYG{n+nt}{h1}\PYG{p}{\PYGZgt{}}
                \PYG{p}{\PYGZlt{}}\PYG{n+nt}{input} \PYG{n+na}{type}\PYG{o}{=}\PYG{l+s}{\PYGZdq{}radio\PYGZdq{}} \PYG{n+na}{id}\PYG{o}{=}\PYG{l+s}{\PYGZdq{}gasolina\PYGZdq{}} \PYG{n+na}{name}\PYG{o}{=}\PYG{l+s}{\PYGZdq{}motor\PYGZdq{}}\PYG{p}{\PYGZgt{}}Motor Gasolina
                \PYG{p}{\PYGZlt{}}\PYG{n+nt}{br}\PYG{p}{/}\PYG{p}{\PYGZgt{}}
                \PYG{p}{\PYGZlt{}}\PYG{n+nt}{input} \PYG{n+na}{type}\PYG{o}{=}\PYG{l+s}{\PYGZdq{}radio\PYGZdq{}} \PYG{n+na}{id}\PYG{o}{=}\PYG{l+s}{\PYGZdq{}diesel\PYGZdq{}} \PYG{n+na}{name}\PYG{o}{=}\PYG{l+s}{\PYGZdq{}motor\PYGZdq{}}\PYG{p}{\PYGZgt{}}Motor Diésel
                \PYG{p}{\PYGZlt{}}\PYG{n+nt}{br}\PYG{p}{/}\PYG{p}{\PYGZgt{}}
                \PYG{p}{\PYGZlt{}}\PYG{n+nt}{h1}\PYG{p}{\PYGZgt{}}Carrocerías\PYG{p}{\PYGZlt{}}\PYG{p}{/}\PYG{n+nt}{h1}\PYG{p}{\PYGZgt{}}
                \PYG{p}{\PYGZlt{}}\PYG{n+nt}{input} \PYG{n+na}{type}\PYG{o}{=}\PYG{l+s}{\PYGZdq{}radio\PYGZdq{}} \PYG{n+na}{id}\PYG{o}{=}\PYG{l+s}{\PYGZdq{}monovolumen\PYGZdq{}}
                \PYG{n+na}{name}\PYG{o}{=}\PYG{l+s}{\PYGZdq{}carroceria\PYGZdq{}}\PYG{p}{\PYGZgt{}}Monovolumen
                \PYG{p}{\PYGZlt{}}\PYG{n+nt}{br}\PYG{p}{/}\PYG{p}{\PYGZgt{}}
                \PYG{p}{\PYGZlt{}}\PYG{n+nt}{input} \PYG{n+na}{type}\PYG{o}{=}\PYG{l+s}{\PYGZdq{}radio\PYGZdq{}} \PYG{n+na}{id}\PYG{o}{=}\PYG{l+s}{\PYGZdq{}berlina\PYGZdq{}} \PYG{n+na}{name}\PYG{o}{=}\PYG{l+s}{\PYGZdq{}carroceria\PYGZdq{}}\PYG{p}{\PYGZgt{}}Berlina
                \PYG{p}{\PYGZlt{}}\PYG{n+nt}{br}\PYG{p}{/}\PYG{p}{\PYGZgt{}}
                \PYG{p}{\PYGZlt{}}\PYG{n+nt}{h1}\PYG{p}{\PYGZgt{}}Accesorios\PYG{p}{\PYGZlt{}}\PYG{p}{/}\PYG{n+nt}{h1}\PYG{p}{\PYGZgt{}}
                \PYG{p}{\PYGZlt{}}\PYG{n+nt}{input} \PYG{n+na}{type}\PYG{o}{=}\PYG{l+s}{\PYGZdq{}checkbox\PYGZdq{}} \PYG{n+na}{name}\PYG{o}{=}\PYG{l+s}{\PYGZdq{}accesorios[]\PYGZdq{}}
                \PYG{n+na}{id}\PYG{o}{=}\PYG{l+s}{\PYGZdq{}radiocd\PYGZdq{}}\PYG{p}{\PYGZgt{}}Radio\PYGZhy{}CD
                \PYG{p}{\PYGZlt{}}\PYG{n+nt}{br}\PYG{p}{/}\PYG{p}{\PYGZgt{}}
                \PYG{p}{\PYGZlt{}}\PYG{n+nt}{input} \PYG{n+na}{type}\PYG{o}{=}\PYG{l+s}{\PYGZdq{}checkbox\PYGZdq{}} \PYG{n+na}{name}\PYG{o}{=}\PYG{l+s}{\PYGZdq{}accesorios[]\PYGZdq{}}
                \PYG{n+na}{id}\PYG{o}{=}\PYG{l+s}{\PYGZdq{}alerones\PYGZdq{}}\PYG{p}{\PYGZgt{}}Alerones
                \PYG{p}{\PYGZlt{}}\PYG{n+nt}{br}\PYG{p}{/}\PYG{p}{\PYGZgt{}}
                \PYG{p}{\PYGZlt{}}\PYG{n+nt}{input} \PYG{n+na}{type}\PYG{o}{=}\PYG{l+s}{\PYGZdq{}checkbox\PYGZdq{}} \PYG{n+na}{name}\PYG{o}{=}\PYG{l+s}{\PYGZdq{}accesorios[]\PYGZdq{}}
                \PYG{n+na}{id}\PYG{o}{=}\PYG{l+s}{\PYGZdq{}llantas\PYGZdq{}}\PYG{p}{\PYGZgt{}}Llantas
                \PYG{p}{\PYGZlt{}}\PYG{n+nt}{br}\PYG{p}{/}\PYG{p}{\PYGZgt{}}
                \PYG{p}{\PYGZlt{}}\PYG{n+nt}{h1}\PYG{p}{\PYGZgt{}}Colores\PYG{p}{\PYGZlt{}}\PYG{p}{/}\PYG{n+nt}{h1}\PYG{p}{\PYGZgt{}}
                \PYG{c}{\PYGZlt{}!\PYGZhy{}\PYGZhy{}}\PYG{c}{Los colores irán}
\PYG{c}{                en una columna y la muestra en otra}\PYG{c}{\PYGZhy{}\PYGZhy{}\PYGZgt{}}
                \PYG{p}{\PYGZlt{}}\PYG{n+nt}{div} \PYG{n+na}{class}\PYG{o}{=}\PYG{l+s}{\PYGZdq{}row\PYGZdq{}}\PYG{p}{\PYGZgt{}}
                        \PYG{p}{\PYGZlt{}}\PYG{n+nt}{div} \PYG{n+na}{class}\PYG{o}{=}\PYG{l+s}{\PYGZdq{}col\PYGZhy{}md\PYGZhy{}3\PYGZdq{}}\PYG{p}{\PYGZgt{}}
                                \PYG{p}{\PYGZlt{}}\PYG{n+nt}{input} \PYG{n+na}{type}\PYG{o}{=}\PYG{l+s}{\PYGZdq{}radio\PYGZdq{}}
                                \PYG{n+na}{name}\PYG{o}{=}\PYG{l+s}{\PYGZdq{}colores\PYGZdq{}}
                                \PYG{n+na}{id}\PYG{o}{=}\PYG{l+s}{\PYGZdq{}blanco\PYGZdq{}}\PYG{p}{\PYGZgt{}}Blanco
                                \PYG{p}{\PYGZlt{}}\PYG{n+nt}{br}\PYG{p}{/}\PYG{p}{\PYGZgt{}}
                                \PYG{p}{\PYGZlt{}}\PYG{n+nt}{input} \PYG{n+na}{type}\PYG{o}{=}\PYG{l+s}{\PYGZdq{}radio\PYGZdq{}}
                                \PYG{n+na}{name}\PYG{o}{=}\PYG{l+s}{\PYGZdq{}colores\PYGZdq{}}
                                \PYG{n+na}{id}\PYG{o}{=}\PYG{l+s}{\PYGZdq{}rojo\PYGZdq{}}\PYG{p}{\PYGZgt{}}Rojo
                                \PYG{p}{\PYGZlt{}}\PYG{n+nt}{br}\PYG{p}{/}\PYG{p}{\PYGZgt{}}
                                \PYG{p}{\PYGZlt{}}\PYG{n+nt}{input} \PYG{n+na}{type}\PYG{o}{=}\PYG{l+s}{\PYGZdq{}radio\PYGZdq{}}
                                \PYG{n+na}{name}\PYG{o}{=}\PYG{l+s}{\PYGZdq{}colores\PYGZdq{}} \PYG{n+na}{id}\PYG{o}{=}\PYG{l+s}{\PYGZdq{}gris\PYGZdq{}}\PYG{p}{\PYGZgt{}}Gris
                        \PYG{p}{\PYGZlt{}}\PYG{p}{/}\PYG{n+nt}{div}\PYG{p}{\PYGZgt{}}
                        \PYG{p}{\PYGZlt{}}\PYG{n+nt}{div} \PYG{n+na}{class}\PYG{o}{=}\PYG{l+s}{\PYGZdq{}col\PYGZhy{}md\PYGZhy{}9 center\PYGZhy{}block\PYGZdq{}}\PYG{p}{\PYGZgt{}}
                                \PYG{p}{\PYGZlt{}}\PYG{n+nt}{h2}\PYG{p}{\PYGZgt{}}Muestra de color
                                \PYG{p}{\PYGZlt{}}\PYG{n+nt}{small}\PYG{p}{\PYGZgt{}}Observe y compare\PYG{p}{\PYGZlt{}}\PYG{p}{/}\PYG{n+nt}{small}\PYG{p}{\PYGZgt{}}\PYG{p}{\PYGZlt{}}\PYG{p}{/}\PYG{n+nt}{h2}\PYG{p}{\PYGZgt{}}
                        \PYG{p}{\PYGZlt{}}\PYG{p}{/}\PYG{n+nt}{div}\PYG{p}{\PYGZgt{}}
                \PYG{p}{\PYGZlt{}}\PYG{p}{/}\PYG{n+nt}{div}\PYG{p}{\PYGZgt{}}\PYG{c}{\PYGZlt{}!\PYGZhy{}\PYGZhy{}}\PYG{c}{Fin de la fila}\PYG{c}{\PYGZhy{}\PYGZhy{}\PYGZgt{}}
        \PYG{p}{\PYGZlt{}}\PYG{p}{/}\PYG{n+nt}{div}\PYG{p}{\PYGZgt{}}
\PYG{p}{\PYGZlt{}}\PYG{p}{/}\PYG{n+nt}{body}\PYG{p}{\PYGZgt{}}
\end{sphinxVerbatim}


\subsubsection{Solución JS}
\label{\detokenize{tema4:solucion-js}}
Añadiremos este código a nuestro programa anterior.

\begin{sphinxVerbatim}[commandchars=\\\{\}]
\PYG{k+kd}{var} \PYG{n+nx}{obj\PYGZus{}documento} \PYG{o}{=} \PYG{n+nx}{\PYGZdl{}}\PYG{p}{(}\PYG{n+nb}{document}\PYG{p}{)}
\PYG{n+nx}{obj\PYGZus{}documento}\PYG{p}{.}\PYG{n+nx}{ready}\PYG{p}{(}\PYG{n+nx}{inicio}\PYG{p}{)}

\PYG{k+kd}{var} \PYG{n+nx}{vector\PYGZus{}ids}\PYG{o}{=}\PYG{p}{[}\PYG{l+s+s2}{\PYGZdq{}\PYGZsh{}gasolina\PYGZdq{}}\PYG{p}{,} \PYG{l+s+s2}{\PYGZdq{}\PYGZsh{}diesel\PYGZdq{}}\PYG{p}{,} \PYG{l+s+s2}{\PYGZdq{}\PYGZsh{}monovolumen\PYGZdq{}}\PYG{p}{,}
                \PYG{l+s+s2}{\PYGZdq{}\PYGZsh{}berlina\PYGZdq{}}\PYG{p}{,} \PYG{l+s+s2}{\PYGZdq{}\PYGZsh{}radiocd\PYGZdq{}}\PYG{p}{,} \PYG{l+s+s2}{\PYGZdq{}\PYGZsh{}alerones\PYGZdq{}}\PYG{p}{,} \PYG{l+s+s2}{\PYGZdq{}\PYGZsh{}llantas\PYGZdq{}}\PYG{p}{]}
\PYG{k+kd}{var} \PYG{n+nx}{precios}\PYG{o}{=}\PYG{p}{[}\PYG{l+m+mi}{5000}\PYG{p}{,} \PYG{l+m+mi}{6800}\PYG{p}{,} \PYG{l+m+mi}{4500}\PYG{p}{,}
                \PYG{l+m+mi}{3700}\PYG{p}{,} \PYG{l+m+mi}{180}\PYG{p}{,} \PYG{l+m+mi}{220}\PYG{p}{,} \PYG{l+m+mi}{200}\PYG{p}{]}

\PYG{k+kd}{function} \PYG{n+nx}{inicio}\PYG{p}{(}\PYG{p}{)}\PYG{p}{\PYGZob{}}

        \PYG{k}{for} \PYG{p}{(}\PYG{k+kd}{var} \PYG{n+nx}{pos} \PYG{k}{in} \PYG{n+nx}{vector\PYGZus{}ids}\PYG{p}{)}\PYG{p}{\PYGZob{}}
                \PYG{k+kd}{var} \PYG{n+nx}{el\PYGZus{}id}\PYG{o}{=}\PYG{n+nx}{vector\PYGZus{}ids}\PYG{p}{[}\PYG{n+nx}{pos}\PYG{p}{]}
                \PYG{n+nx}{\PYGZdl{}}\PYG{p}{(}\PYG{n+nx}{el\PYGZus{}id}\PYG{p}{)}\PYG{p}{.}\PYG{n+nx}{click} \PYG{p}{(} \PYG{n+nx}{calcularPrecio} \PYG{p}{)}
        \PYG{p}{\PYGZcb{}}
        \PYG{n+nx}{\PYGZdl{}}\PYG{p}{(}\PYG{l+s+s2}{\PYGZdq{}\PYGZsh{}blanco\PYGZdq{}}\PYG{p}{)}\PYG{p}{.}\PYG{n+nx}{click} \PYG{p}{(} \PYG{n+nx}{ponerColorBlanco} \PYG{p}{)}
        \PYG{n+nx}{\PYGZdl{}}\PYG{p}{(}\PYG{l+s+s2}{\PYGZdq{}\PYGZsh{}rojo\PYGZdq{}}\PYG{p}{)}\PYG{p}{.}\PYG{n+nx}{click} \PYG{p}{(} \PYG{n+nx}{ponerColorRojo} \PYG{p}{)}
        \PYG{n+nx}{\PYGZdl{}}\PYG{p}{(}\PYG{l+s+s2}{\PYGZdq{}\PYGZsh{}gris\PYGZdq{}}\PYG{p}{)}\PYG{p}{.}\PYG{n+nx}{click} \PYG{p}{(}\PYG{n+nx}{ponerGris} \PYG{p}{)}
\PYG{p}{\PYGZcb{}}
\PYG{k+kd}{function} \PYG{n+nx}{limpiarColores}\PYG{p}{(}\PYG{p}{)}\PYG{p}{\PYGZob{}}
        \PYG{k+kd}{var} \PYG{n+nx}{clases}\PYG{o}{=}\PYG{p}{[}\PYG{l+s+s2}{\PYGZdq{}muestrarojo\PYGZdq{}}\PYG{p}{,} \PYG{l+s+s2}{\PYGZdq{}muestrablanco\PYGZdq{}}\PYG{p}{,}
                                \PYG{l+s+s2}{\PYGZdq{}muestragris\PYGZdq{}}\PYG{p}{]}
        \PYG{k}{for} \PYG{p}{(}\PYG{k+kd}{var} \PYG{n+nx}{pos} \PYG{k}{in} \PYG{n+nx}{clases}\PYG{p}{)} \PYG{p}{\PYGZob{}}
                \PYG{n+nx}{\PYGZdl{}}\PYG{p}{(}\PYG{l+s+s2}{\PYGZdq{}\PYGZsh{}muestra\PYGZdq{}}\PYG{p}{)}\PYG{p}{.}\PYG{n+nx}{removeClass}\PYG{p}{(} \PYG{n+nx}{clases}\PYG{p}{[}\PYG{n+nx}{pos}\PYG{p}{]} \PYG{p}{)}
        \PYG{p}{\PYGZcb{}}
\PYG{p}{\PYGZcb{}}
\PYG{k+kd}{function} \PYG{n+nx}{ponerGris}\PYG{p}{(}\PYG{p}{)} \PYG{p}{\PYGZob{}}
        \PYG{n+nx}{limpiarColores}\PYG{p}{(}\PYG{p}{)}
        \PYG{n+nx}{\PYGZdl{}}\PYG{p}{(}\PYG{l+s+s2}{\PYGZdq{}\PYGZsh{}muestra\PYGZdq{}}\PYG{p}{)}\PYG{p}{.}\PYG{n+nx}{addClass}\PYG{p}{(}\PYG{l+s+s2}{\PYGZdq{}muestragris\PYGZdq{}}\PYG{p}{)}
\PYG{p}{\PYGZcb{}}
\PYG{k+kd}{function} \PYG{n+nx}{ponerColorRojo}\PYG{p}{(}\PYG{p}{)}\PYG{p}{\PYGZob{}}
        \PYG{n+nx}{limpiarColores}\PYG{p}{(}\PYG{p}{)}
        \PYG{n+nx}{\PYGZdl{}}\PYG{p}{(}\PYG{l+s+s2}{\PYGZdq{}\PYGZsh{}muestra\PYGZdq{}}\PYG{p}{)}\PYG{p}{.}\PYG{n+nx}{addClass} \PYG{p}{(}\PYG{l+s+s2}{\PYGZdq{}muestrarojo\PYGZdq{}}\PYG{p}{)}
\PYG{p}{\PYGZcb{}}
\PYG{k+kd}{function} \PYG{n+nx}{ponerColorBlanco}\PYG{p}{(}\PYG{p}{)}\PYG{p}{\PYGZob{}}
        \PYG{n+nx}{limpiarColores}\PYG{p}{(}\PYG{p}{)}
        \PYG{n+nx}{\PYGZdl{}}\PYG{p}{(}\PYG{l+s+s2}{\PYGZdq{}\PYGZsh{}muestra\PYGZdq{}}\PYG{p}{)}\PYG{p}{.}\PYG{n+nx}{addClass} \PYG{p}{(}\PYG{l+s+s2}{\PYGZdq{}muestrablanco\PYGZdq{}}\PYG{p}{)}
\PYG{p}{\PYGZcb{}}
\end{sphinxVerbatim}


\section{Otro configurador de coches}
\label{\detokenize{tema4:otro-configurador-de-coches}}
Sin utilizar JQuery se desea crear un configurador de coches en JS que responda a las siguientes premisas:
\begin{itemize}
\item {} 
Hay dos modelos a elegir: el modelo A cuesta 7000 euros y el modelo B cuesta 9000.

\item {} 
Se pueden elegir dos tipos de motor. El motor de gasolina cuesta 2000 y el diésel 5000 euros.

\item {} 
Se pueden elegir 0, 1, muchos o todos los extras siguientes: Pintura metalizada por 1000 euros más, pack de sonido por 500 euros más y pack de seguridad por 1000 euros más

\end{itemize}


\subsection{Configurador en HTML}
\label{\detokenize{tema4:configurador-en-html}}
\begin{sphinxVerbatim}[commandchars=\\\{\}]
\PYG{p}{\PYGZlt{}}\PYG{n+nt}{form}\PYG{p}{\PYGZgt{}}
        \PYG{p}{\PYGZlt{}}\PYG{n+nt}{h3}\PYG{p}{\PYGZgt{}}Elija un modelo\PYG{p}{\PYGZlt{}}\PYG{p}{/}\PYG{n+nt}{h3}\PYG{p}{\PYGZgt{}}
        \PYG{p}{\PYGZlt{}}\PYG{n+nt}{input} \PYG{n+na}{type}\PYG{o}{=}\PYG{l+s}{\PYGZdq{}radio\PYGZdq{}} \PYG{n+na}{name}\PYG{o}{=}\PYG{l+s}{\PYGZdq{}modelo\PYGZdq{}}
                   \PYG{n+na}{id}\PYG{o}{=}\PYG{l+s}{\PYGZdq{}modelo\PYGZus{}a\PYGZdq{}}\PYG{p}{\PYGZgt{}}Modelo A\PYG{p}{\PYGZlt{}}\PYG{n+nt}{br}\PYG{p}{/}\PYG{p}{\PYGZgt{}}
        \PYG{p}{\PYGZlt{}}\PYG{n+nt}{input} \PYG{n+na}{type}\PYG{o}{=}\PYG{l+s}{\PYGZdq{}radio\PYGZdq{}} \PYG{n+na}{name}\PYG{o}{=}\PYG{l+s}{\PYGZdq{}modelo\PYGZdq{}}
                   \PYG{n+na}{id}\PYG{o}{=}\PYG{l+s}{\PYGZdq{}modelo\PYGZus{}b\PYGZdq{}}\PYG{p}{\PYGZgt{}}Modelo B\PYG{p}{\PYGZlt{}}\PYG{n+nt}{br}\PYG{p}{/}\PYG{p}{\PYGZgt{}}
        \PYG{p}{\PYGZlt{}}\PYG{n+nt}{h3}\PYG{p}{\PYGZgt{}}Elija un tipo de motor\PYG{p}{\PYGZlt{}}\PYG{p}{/}\PYG{n+nt}{h3}\PYG{p}{\PYGZgt{}}
        \PYG{p}{\PYGZlt{}}\PYG{n+nt}{input} \PYG{n+na}{type}\PYG{o}{=}\PYG{l+s}{\PYGZdq{}radio\PYGZdq{}} \PYG{n+na}{name}\PYG{o}{=}\PYG{l+s}{\PYGZdq{}motor\PYGZdq{}}
                   \PYG{n+na}{id}\PYG{o}{=}\PYG{l+s}{\PYGZdq{}gasolina\PYGZdq{}}\PYG{p}{\PYGZgt{}}Gasolina\PYG{p}{\PYGZlt{}}\PYG{n+nt}{br}\PYG{p}{/}\PYG{p}{\PYGZgt{}}
        \PYG{p}{\PYGZlt{}}\PYG{n+nt}{input} \PYG{n+na}{type}\PYG{o}{=}\PYG{l+s}{\PYGZdq{}radio\PYGZdq{}} \PYG{n+na}{name}\PYG{o}{=}\PYG{l+s}{\PYGZdq{}motor\PYGZdq{}}
                   \PYG{n+na}{id}\PYG{o}{=}\PYG{l+s}{\PYGZdq{}diesel\PYGZdq{}}\PYG{p}{\PYGZgt{}}Diésel\PYG{p}{\PYGZlt{}}\PYG{n+nt}{br}\PYG{p}{/}\PYG{p}{\PYGZgt{}}
        \PYG{p}{\PYGZlt{}}\PYG{n+nt}{h3}\PYG{p}{\PYGZgt{}}Elija extras\PYG{p}{\PYGZlt{}}\PYG{p}{/}\PYG{n+nt}{h3}\PYG{p}{\PYGZgt{}}
        \PYG{p}{\PYGZlt{}}\PYG{n+nt}{input} \PYG{n+na}{type}\PYG{o}{=}\PYG{l+s}{\PYGZdq{}checkbox\PYGZdq{}} \PYG{n+na}{name}\PYG{o}{=}\PYG{l+s}{\PYGZdq{}extras\PYGZdq{}}
                   \PYG{n+na}{id}\PYG{o}{=}\PYG{l+s}{\PYGZdq{}metalizada\PYGZdq{}}\PYG{p}{\PYGZgt{}}Pintura metalizada\PYG{p}{\PYGZlt{}}\PYG{n+nt}{br}\PYG{p}{/}\PYG{p}{\PYGZgt{}}
        \PYG{p}{\PYGZlt{}}\PYG{n+nt}{input} \PYG{n+na}{type}\PYG{o}{=}\PYG{l+s}{\PYGZdq{}checkbox\PYGZdq{}} \PYG{n+na}{name}\PYG{o}{=}\PYG{l+s}{\PYGZdq{}extras\PYGZdq{}}
                   \PYG{n+na}{id}\PYG{o}{=}\PYG{l+s}{\PYGZdq{}sonido\PYGZdq{}}\PYG{p}{\PYGZgt{}}Extra sonido\PYG{p}{\PYGZlt{}}\PYG{n+nt}{br}\PYG{p}{/}\PYG{p}{\PYGZgt{}}
        \PYG{p}{\PYGZlt{}}\PYG{n+nt}{input} \PYG{n+na}{type}\PYG{o}{=}\PYG{l+s}{\PYGZdq{}checkbox\PYGZdq{}} \PYG{n+na}{name}\PYG{o}{=}\PYG{l+s}{\PYGZdq{}extras\PYGZdq{}}
                   \PYG{n+na}{id}\PYG{o}{=}\PYG{l+s}{\PYGZdq{}seguridad\PYGZdq{}}\PYG{p}{\PYGZgt{}}Extra seguridad\PYG{p}{\PYGZlt{}}\PYG{n+nt}{br}\PYG{p}{/}\PYG{p}{\PYGZgt{}}
        \PYG{p}{\PYGZlt{}}\PYG{n+nt}{input} \PYG{n+na}{type}\PYG{o}{=}\PYG{l+s}{\PYGZdq{}submit\PYGZdq{}} \PYG{n+na}{value}\PYG{o}{=}\PYG{l+s}{\PYGZdq{}Calcular\PYGZdq{}}
                   \PYG{n+na}{onclick}\PYG{o}{=}\PYG{l+s}{\PYGZdq{}calcular();return false;\PYGZdq{}}\PYG{p}{\PYGZgt{}}
\PYG{p}{\PYGZlt{}}\PYG{p}{/}\PYG{n+nt}{form}\PYG{p}{\PYGZgt{}}
\PYG{p}{\PYGZlt{}}\PYG{n+nt}{div} \PYG{n+na}{id}\PYG{o}{=}\PYG{l+s}{\PYGZdq{}zonaresultados\PYGZdq{}}\PYG{p}{\PYGZgt{}}\PYG{p}{\PYGZlt{}}\PYG{p}{/}\PYG{n+nt}{div}\PYG{p}{\PYGZgt{}}
\end{sphinxVerbatim}


\subsection{Configurador en JS}
\label{\detokenize{tema4:configurador-en-js}}
\begin{sphinxVerbatim}[commandchars=\\\{\}]
\PYG{k+kd}{function} \PYG{n+nx}{esta\PYGZus{}checked}\PYG{p}{(}\PYG{n+nx}{id}\PYG{p}{)}\PYG{p}{\PYGZob{}}
        \PYG{k+kd}{var} \PYG{n+nx}{control}\PYG{p}{;}
        \PYG{n+nx}{control}\PYG{o}{=}\PYG{n+nb}{document}\PYG{p}{.}\PYG{n+nx}{getElementById}\PYG{p}{(}\PYG{n+nx}{id}\PYG{p}{)}\PYG{p}{;}
        \PYG{k}{if} \PYG{p}{(}\PYG{n+nx}{control}\PYG{p}{.}\PYG{n+nx}{checked}\PYG{p}{)}\PYG{p}{\PYGZob{}}
                \PYG{k}{return} \PYG{k+kc}{true}\PYG{p}{;}
        \PYG{p}{\PYGZcb{}} \PYG{k}{else} \PYG{p}{\PYGZob{}}
                \PYG{k}{return} \PYG{k+kc}{false}\PYG{p}{;}
        \PYG{p}{\PYGZcb{}}
\PYG{p}{\PYGZcb{}}

\PYG{k+kd}{function} \PYG{n+nx}{calcular}\PYG{p}{(}\PYG{p}{)}\PYG{p}{\PYGZob{}}
        \PYG{k+kd}{var} \PYG{n+nx}{preciototal}   \PYG{o}{=}\PYG{l+m+mi}{0}\PYG{p}{;}
        \PYG{k+kd}{var} \PYG{n+nx}{preciomodelo}  \PYG{o}{=}\PYG{l+m+mi}{0}\PYG{p}{;}
        \PYG{k+kd}{var} \PYG{n+nx}{preciomotor}   \PYG{o}{=}\PYG{l+m+mi}{0}\PYG{p}{;}
        \PYG{k+kd}{var} \PYG{n+nx}{precioextras}  \PYG{o}{=} \PYG{l+m+mi}{0}\PYG{p}{;}
        \PYG{k}{if} \PYG{p}{(} \PYG{p}{(}\PYG{o}{!}\PYG{n+nx}{esta\PYGZus{}checked}\PYG{p}{(}\PYG{l+s+s2}{\PYGZdq{}modelo\PYGZus{}a\PYGZdq{}}\PYG{p}{)}\PYG{p}{)}
                \PYG{o}{\PYGZam{}\PYGZam{}} \PYG{p}{(}\PYG{o}{!}\PYG{n+nx}{esta\PYGZus{}checked}\PYG{p}{(}\PYG{l+s+s2}{\PYGZdq{}modelo\PYGZus{}b\PYGZdq{}}\PYG{p}{)} \PYG{p}{)} \PYG{p}{)}\PYG{p}{\PYGZob{}}
                \PYG{n+nx}{alert}\PYG{p}{(}\PYG{l+s+s2}{\PYGZdq{}Debe marcar un modelo\PYGZdq{}}\PYG{p}{)}\PYG{p}{;}
                \PYG{k}{return} \PYG{p}{;}
        \PYG{p}{\PYGZcb{}}
        \PYG{k}{if} \PYG{p}{(}\PYG{n+nx}{esta\PYGZus{}checked}\PYG{p}{(}\PYG{l+s+s2}{\PYGZdq{}modelo\PYGZus{}a\PYGZdq{}}\PYG{p}{)}\PYG{p}{)}\PYG{p}{\PYGZob{}}
                \PYG{n+nx}{preciomodelo}\PYG{o}{=}\PYG{l+m+mi}{7000}\PYG{p}{;}
        \PYG{p}{\PYGZcb{}}
        \PYG{k}{if} \PYG{p}{(}\PYG{n+nx}{esta\PYGZus{}checked}\PYG{p}{(}\PYG{l+s+s2}{\PYGZdq{}modelo\PYGZus{}b\PYGZdq{}}\PYG{p}{)}\PYG{p}{)}\PYG{p}{\PYGZob{}}
                \PYG{n+nx}{preciomodelo}\PYG{o}{=}\PYG{l+m+mi}{9000}\PYG{p}{;}
        \PYG{p}{\PYGZcb{}}
        \PYG{k}{if} \PYG{p}{(}\PYG{n+nx}{esta\PYGZus{}checked}\PYG{p}{(}\PYG{l+s+s2}{\PYGZdq{}gasolina\PYGZdq{}}\PYG{p}{)}\PYG{p}{)}\PYG{p}{\PYGZob{}}
                \PYG{n+nx}{preciomotor}\PYG{o}{=}\PYG{l+m+mi}{2000}\PYG{p}{;}
        \PYG{p}{\PYGZcb{}}
        \PYG{k}{if} \PYG{p}{(}\PYG{n+nx}{esta\PYGZus{}checked}\PYG{p}{(}\PYG{l+s+s2}{\PYGZdq{}diesel\PYGZdq{}}\PYG{p}{)}\PYG{p}{)}\PYG{p}{\PYGZob{}}
                \PYG{n+nx}{preciomotor}\PYG{o}{=}\PYG{l+m+mi}{4000}\PYG{p}{;}
        \PYG{p}{\PYGZcb{}}
        \PYG{k}{if} \PYG{p}{(}\PYG{n+nx}{esta\PYGZus{}checked}\PYG{p}{(}\PYG{l+s+s2}{\PYGZdq{}metalizada\PYGZdq{}}\PYG{p}{)}\PYG{p}{)}\PYG{p}{\PYGZob{}}
                \PYG{n+nx}{precioextras}\PYG{o}{=}\PYG{l+m+mi}{1000}\PYG{p}{;}
        \PYG{p}{\PYGZcb{}}
        \PYG{k}{if} \PYG{p}{(}\PYG{n+nx}{esta\PYGZus{}checked}\PYG{p}{(}\PYG{l+s+s2}{\PYGZdq{}sonido\PYGZdq{}}\PYG{p}{)}\PYG{p}{)}\PYG{p}{\PYGZob{}}
                \PYG{n+nx}{precioextras}\PYG{o}{=}\PYG{n+nx}{precioextras}\PYG{o}{+}\PYG{l+m+mi}{500}\PYG{p}{;}
        \PYG{p}{\PYGZcb{}}
        \PYG{k}{if} \PYG{p}{(}\PYG{n+nx}{esta\PYGZus{}checked}\PYG{p}{(}\PYG{l+s+s2}{\PYGZdq{}seguridad\PYGZdq{}}\PYG{p}{)}\PYG{p}{)}\PYG{p}{\PYGZob{}}
                \PYG{n+nx}{precioextras}\PYG{o}{=}\PYG{n+nx}{precioextras}\PYG{o}{+}\PYG{l+m+mi}{1000}\PYG{p}{;}
        \PYG{p}{\PYGZcb{}}
        \PYG{n+nx}{preciototal}\PYG{o}{=}\PYG{n+nx}{preciomodelo}\PYG{o}{+}\PYG{n+nx}{preciomotor}\PYG{o}{+}\PYG{n+nx}{precioextras}\PYG{p}{;}

        \PYG{k+kd}{var} \PYG{n+nx}{zonaresultados}\PYG{p}{;}
        \PYG{n+nx}{zonaresultados}\PYG{o}{=}\PYG{n+nb}{document}\PYG{p}{.}\PYG{n+nx}{getElementById}\PYG{p}{(}
                \PYG{l+s+s2}{\PYGZdq{}zonaresultados\PYGZdq{}}\PYG{p}{)}\PYG{p}{;}
        \PYG{n+nx}{zonaresultados}\PYG{p}{.}\PYG{n+nx}{innerHTML}\PYG{o}{=}\PYG{l+s+s2}{\PYGZdq{}Precio:\PYGZdq{}}\PYG{o}{+}\PYG{n+nx}{preciototal}\PYG{p}{;}
\PYG{p}{\PYGZcb{}}
\end{sphinxVerbatim}


\section{Calculo de impuestos}
\label{\detokenize{tema4:calculo-de-impuestos}}
Se desea crear una pequeña aplicación web que permita calcular los impuestos que se deben pagar a partir de unos datos, en concreto:
\begin{itemize}
\item {} 
Se debe introducir el salario anual en un campo de tipo \sphinxcode{number}.

\item {} 
Hay dos radios que permiten indicar si el contribuyente tiene hijos o no.

\item {} 
Hay dos checkboxes que permiten saber si el contribuyente tiene derecho a algunas modificaciones llamadas B1 y B2.

\end{itemize}

Las reglas para el calculo son:
\begin{itemize}
\item {} 
Si el salario es menor de 20000 los impuestos son 0.

\item {} 
Si el salario está entre 20001 y 30000 los impuestos son el 10\%.

\item {} 
Si el salario está entre 30001 y 50000 los impuestos son el 20\%.

\item {} 
Si el salario es mayor de 50000 se paga el 38\%

\item {} 
Si se tienen hijos los impuestos se reducen en 180 euros.

\item {} 
Si se tiene la bonificación B1 los impuestos se reducen en 355 euros.

\item {} 
Si se tiene la bonificación B2 los impuestos se reducen en 560 euros.

\item {} 
Se recuerda que las bonificaciones B1 y B2 son compatibles (de hecho hemos dicho que están en checkboxes).

\item {} 
La cantidad de impuestos puede ser negativa (que significa que le sale «a devolver»)

\end{itemize}


\subsection{HTML calculador}
\label{\detokenize{tema4:html-calculador}}
\begin{sphinxVerbatim}[commandchars=\\\{\}]
\PYG{c+cp}{\PYGZlt{}!DOCTYPE html\PYGZgt{}}

\PYG{p}{\PYGZlt{}}\PYG{n+nt}{html}\PYG{p}{\PYGZgt{}}
\PYG{p}{\PYGZlt{}}\PYG{n+nt}{head}\PYG{p}{\PYGZgt{}}
    \PYG{p}{\PYGZlt{}}\PYG{n+nt}{meta} \PYG{n+na}{charset}\PYG{o}{=}\PYG{l+s}{\PYGZdq{}utf\PYGZhy{}8\PYGZdq{}}\PYG{p}{\PYGZgt{}}
    \PYG{p}{\PYGZlt{}}\PYG{n+nt}{link} \PYG{n+na}{type}\PYG{o}{=}\PYG{l+s}{\PYGZdq{}text/css\PYGZdq{}} \PYG{n+na}{href}\PYG{o}{=}\PYG{l+s}{\PYGZdq{}css/estilo.css\PYGZdq{}}
          \PYG{n+na}{rel}\PYG{o}{=}\PYG{l+s}{\PYGZdq{}stylesheet\PYGZdq{}}\PYG{p}{\PYGZgt{}}
    \PYG{p}{\PYGZlt{}}\PYG{n+nt}{title}\PYG{p}{\PYGZgt{}}Plantilla de web\PYG{p}{\PYGZlt{}}\PYG{p}{/}\PYG{n+nt}{title}\PYG{p}{\PYGZgt{}}
    \PYG{p}{\PYGZlt{}}\PYG{n+nt}{script} \PYG{n+na}{type}\PYG{o}{=}\PYG{l+s}{\PYGZdq{}text/javascript\PYGZdq{}}
            \PYG{n+na}{src}\PYG{o}{=}\PYG{l+s}{\PYGZdq{}js/jquery.js\PYGZdq{}}\PYG{p}{\PYGZgt{}}\PYG{p}{\PYGZlt{}}\PYG{p}{/}\PYG{n+nt}{script}\PYG{p}{\PYGZgt{}}
    \PYG{p}{\PYGZlt{}}\PYG{n+nt}{script} \PYG{n+na}{type}\PYG{o}{=}\PYG{l+s}{\PYGZdq{}text/javascript\PYGZdq{}} \PYG{n+na}{src}\PYG{o}{=}\PYG{l+s}{\PYGZdq{}js/programa.js\PYGZdq{}}\PYG{p}{\PYGZgt{}}\PYG{p}{\PYGZlt{}}\PYG{p}{/}\PYG{n+nt}{script}\PYG{p}{\PYGZgt{}}
\PYG{p}{\PYGZlt{}}\PYG{p}{/}\PYG{n+nt}{head}\PYG{p}{\PYGZgt{}}

\PYG{p}{\PYGZlt{}}\PYG{n+nt}{body}\PYG{p}{\PYGZgt{}}
\PYG{c}{\PYGZlt{}!\PYGZhy{}\PYGZhy{}}\PYG{c}{Toda nuestra página irá aquí dentro}\PYG{c}{\PYGZhy{}\PYGZhy{}\PYGZgt{}}    
\PYG{p}{\PYGZlt{}}\PYG{n+nt}{div} \PYG{n+na}{id}\PYG{o}{=}\PYG{l+s}{\PYGZdq{}contenido\PYGZdq{}}\PYG{p}{\PYGZgt{}}
    
\PYG{p}{\PYGZlt{}}\PYG{n+nt}{h1}\PYG{p}{\PYGZgt{}}Plantilla de proyecto web\PYG{p}{\PYGZlt{}}\PYG{p}{/}\PYG{n+nt}{h1}\PYG{p}{\PYGZgt{}}
\PYG{p}{\PYGZlt{}}\PYG{n+nt}{form}\PYG{p}{\PYGZgt{}}
    Salario:
    \PYG{p}{\PYGZlt{}}\PYG{n+nt}{input} \PYG{n+na}{type}\PYG{o}{=}\PYG{l+s}{\PYGZdq{}number\PYGZdq{}} \PYG{n+na}{id}\PYG{o}{=}\PYG{l+s}{\PYGZdq{}salario\PYGZdq{}} \PYG{n+na}{min}\PYG{o}{=}\PYG{l+s}{\PYGZdq{}0\PYGZdq{}} \PYG{n+na}{max}\PYG{o}{=}\PYG{l+s}{\PYGZdq{}1000000\PYGZdq{}} \PYG{n+na}{value}\PYG{o}{=}\PYG{l+s}{\PYGZdq{}0\PYGZdq{}}\PYG{p}{\PYGZgt{}} \PYG{p}{\PYGZlt{}}\PYG{n+nt}{br}\PYG{p}{/}\PYG{p}{\PYGZgt{}}
    \PYG{p}{\PYGZlt{}}\PYG{n+nt}{input} \PYG{n+na}{type}\PYG{o}{=}\PYG{l+s}{\PYGZdq{}radio\PYGZdq{}} \PYG{n+na}{id}\PYG{o}{=}\PYG{l+s}{\PYGZdq{}con\PYGZus{}hijos\PYGZdq{}} \PYG{n+na}{name}\PYG{o}{=}\PYG{l+s}{\PYGZdq{}hijos\PYGZdq{}}\PYG{p}{\PYGZgt{}}Con hijos a cargo \PYG{p}{\PYGZlt{}}\PYG{n+nt}{br}\PYG{p}{/}\PYG{p}{\PYGZgt{}}
    \PYG{p}{\PYGZlt{}}\PYG{n+nt}{input} \PYG{n+na}{type}\PYG{o}{=}\PYG{l+s}{\PYGZdq{}radio\PYGZdq{}} \PYG{n+na}{id}\PYG{o}{=}\PYG{l+s}{\PYGZdq{}sin\PYGZus{}hijos\PYGZdq{}} \PYG{n+na}{name}\PYG{o}{=}\PYG{l+s}{\PYGZdq{}hijos\PYGZdq{}} \PYG{n+na}{checked}\PYG{o}{=}\PYG{l+s}{\PYGZdq{}checked\PYGZdq{}}\PYG{p}{\PYGZgt{}}
            Sin hijos a cargo \PYG{p}{\PYGZlt{}}\PYG{n+nt}{br}\PYG{p}{/}\PYG{p}{\PYGZgt{}}
    \PYG{p}{\PYGZlt{}}\PYG{n+nt}{input} \PYG{n+na}{type}\PYG{o}{=}\PYG{l+s}{\PYGZdq{}checkbox\PYGZdq{}} \PYG{n+na}{id}\PYG{o}{=}\PYG{l+s}{\PYGZdq{}bonificacion\PYGZus{}b1\PYGZdq{}}\PYG{p}{\PYGZgt{}} Con derecho a bonif. B1 \PYG{p}{\PYGZlt{}}\PYG{n+nt}{br}\PYG{p}{/}\PYG{p}{\PYGZgt{}}
    \PYG{p}{\PYGZlt{}}\PYG{n+nt}{input} \PYG{n+na}{type}\PYG{o}{=}\PYG{l+s}{\PYGZdq{}checkbox\PYGZdq{}} \PYG{n+na}{id}\PYG{o}{=}\PYG{l+s}{\PYGZdq{}bonificacion\PYGZus{}b2\PYGZdq{}}\PYG{p}{\PYGZgt{}} Con derecho a bonif. B2 \PYG{p}{\PYGZlt{}}\PYG{n+nt}{br}\PYG{p}{/}\PYG{p}{\PYGZgt{}}
    \PYG{p}{\PYGZlt{}}\PYG{n+nt}{button} \PYG{n+na}{id}\PYG{o}{=}\PYG{l+s}{\PYGZdq{}calcular\PYGZdq{}}\PYG{p}{\PYGZgt{}}Calcular impuestos\PYG{p}{\PYGZlt{}}\PYG{p}{/}\PYG{n+nt}{button}\PYG{p}{\PYGZgt{}}
    
\PYG{p}{\PYGZlt{}}\PYG{p}{/}\PYG{n+nt}{form}\PYG{p}{\PYGZgt{}}
\PYG{p}{\PYGZlt{}}\PYG{n+nt}{div} \PYG{n+na}{id}\PYG{o}{=}\PYG{l+s}{\PYGZdq{}mensajes\PYGZdq{}}\PYG{p}{\PYGZgt{}}Aquí aparecerán sus resultados\PYG{p}{\PYGZlt{}}\PYG{p}{/}\PYG{n+nt}{div}\PYG{p}{\PYGZgt{}}
\PYG{p}{\PYGZlt{}}\PYG{p}{/}\PYG{n+nt}{body}\PYG{p}{\PYGZgt{}}
\PYG{p}{\PYGZlt{}}\PYG{p}{/}\PYG{n+nt}{html}\PYG{p}{\PYGZgt{}}
\end{sphinxVerbatim}


\subsection{JS Calculador (con JQuery (DAM))}
\label{\detokenize{tema4:js-calculador-con-jquery-dam}}
\begin{sphinxVerbatim}[commandchars=\\\{\}]
\PYG{k+kd}{function} \PYG{n+nx}{getFloat} \PYG{p}{(} \PYG{n+nx}{identificador} \PYG{p}{)}\PYG{p}{\PYGZob{}}
    \PYG{k+kd}{let} \PYG{n+nx}{objeto\PYGZus{}control}\PYG{o}{=}\PYG{n+nx}{\PYGZdl{}}\PYG{p}{(}\PYG{n+nx}{identificador}\PYG{p}{)}\PYG{p}{;}
    \PYG{k+kd}{let} \PYG{n+nx}{cadena\PYGZus{}control}\PYG{o}{=}\PYG{n+nx}{objeto\PYGZus{}control}\PYG{p}{.}\PYG{n+nx}{val}\PYG{p}{(}\PYG{p}{)}\PYG{p}{;}
    \PYG{k+kd}{let} \PYG{n+nx}{cantidad}\PYG{o}{=}\PYG{n+nb}{parseFloat}\PYG{p}{(}\PYG{n+nx}{cadena\PYGZus{}control}\PYG{p}{)}\PYG{p}{;}
    \PYG{k}{return} \PYG{n+nx}{cantidad}\PYG{p}{;}
\PYG{p}{\PYGZcb{}}
\PYG{k+kd}{function} \PYG{n+nx}{isChecked} \PYG{p}{(} \PYG{n+nx}{identificador} \PYG{p}{)}\PYG{p}{\PYGZob{}}
    \PYG{k+kd}{let} \PYG{n+nx}{objeto\PYGZus{}control}\PYG{o}{=}\PYG{n+nx}{\PYGZdl{}}\PYG{p}{(}\PYG{n+nx}{identificador}\PYG{p}{)}\PYG{p}{;}
    \PYG{c+cm}{/*Comprobamos si el control}
\PYG{c+cm}{    tiene la propiedad checked*/}
    \PYG{k}{if} \PYG{p}{(}\PYG{n+nx}{objeto\PYGZus{}control}\PYG{p}{.}\PYG{n+nx}{prop}\PYG{p}{(}\PYG{l+s+s2}{\PYGZdq{}checked\PYGZdq{}}\PYG{p}{)}\PYG{p}{)}\PYG{p}{\PYGZob{}}
        \PYG{k}{return} \PYG{k+kc}{true}\PYG{p}{;}
    \PYG{p}{\PYGZcb{}}
    \PYG{c+cm}{/* Si no tiene la propiedad checked}
\PYG{c+cm}{    es que no está marcado*/}
    \PYG{k}{return} \PYG{k+kc}{false}\PYG{p}{;}
\PYG{p}{\PYGZcb{}}
\PYG{k+kd}{function} \PYG{n+nx}{calcular}\PYG{p}{(}\PYG{p}{)}\PYG{p}{\PYGZob{}}
    \PYG{k+kd}{let} \PYG{n+nx}{salario\PYGZus{}anual}\PYG{o}{=}\PYG{n+nx}{getFloat}\PYG{p}{(}\PYG{l+s+s2}{\PYGZdq{}\PYGZsh{}salario\PYGZdq{}}\PYG{p}{)}\PYG{p}{;}    
    \PYG{k+kd}{let} \PYG{n+nx}{impuestos}\PYG{o}{=}\PYG{l+m+mi}{0}\PYG{p}{;}
    
    \PYG{k}{if} \PYG{p}{(} \PYG{p}{(}\PYG{n+nx}{salario\PYGZus{}anual}\PYG{o}{\PYGZgt{}=}\PYG{l+m+mi}{20000}\PYG{p}{)} \PYG{o}{\PYGZam{}\PYGZam{}} \PYG{p}{(}\PYG{n+nx}{salario\PYGZus{}anual}\PYG{o}{\PYGZlt{}=}\PYG{l+m+mi}{30000}\PYG{p}{)} \PYG{p}{)}\PYG{p}{\PYGZob{}}
        \PYG{n+nx}{impuestos}\PYG{o}{=}\PYG{n+nx}{salario\PYGZus{}anual} \PYG{o}{*} \PYG{l+m+mf}{0.1}\PYG{p}{;}
    \PYG{p}{\PYGZcb{}}
    
    \PYG{k}{if} \PYG{p}{(} \PYG{p}{(}\PYG{n+nx}{salario\PYGZus{}anual}\PYG{o}{\PYGZgt{}=}\PYG{l+m+mi}{30001}\PYG{p}{)} \PYG{o}{\PYGZam{}\PYGZam{}} \PYG{p}{(}\PYG{n+nx}{salario\PYGZus{}anual}\PYG{o}{\PYGZlt{}=}\PYG{l+m+mi}{50000}\PYG{p}{)} \PYG{p}{)}\PYG{p}{\PYGZob{}}
        \PYG{n+nx}{impuestos}\PYG{o}{=}\PYG{n+nx}{salario\PYGZus{}anual} \PYG{o}{*} \PYG{l+m+mf}{0.2}\PYG{p}{;}
    \PYG{p}{\PYGZcb{}}
    \PYG{k}{if} \PYG{p}{(}\PYG{n+nx}{salario\PYGZus{}anual}\PYG{o}{\PYGZgt{}=}\PYG{l+m+mi}{50001}\PYG{p}{)}\PYG{p}{\PYGZob{}}
        \PYG{n+nx}{impuestos}\PYG{o}{=}\PYG{n+nx}{salario\PYGZus{}anual} \PYG{o}{*} \PYG{l+m+mf}{0.38}\PYG{p}{;}
    \PYG{p}{\PYGZcb{}}
    \PYG{k}{if} \PYG{p}{(}\PYG{n+nx}{isChecked}\PYG{p}{(}\PYG{l+s+s2}{\PYGZdq{}\PYGZsh{}con\PYGZus{}hijos\PYGZdq{}}\PYG{p}{)}\PYG{p}{)}\PYG{p}{\PYGZob{}}
        \PYG{n+nx}{impuestos} \PYG{o}{=} \PYG{n+nx}{impuestos} \PYG{o}{\PYGZhy{}} \PYG{l+m+mi}{180}\PYG{p}{;}
    \PYG{p}{\PYGZcb{}}
    \PYG{k}{if} \PYG{p}{(}\PYG{n+nx}{isChecked}\PYG{p}{(}\PYG{l+s+s2}{\PYGZdq{}\PYGZsh{}bonificacion\PYGZus{}b1\PYGZdq{}}\PYG{p}{)}\PYG{p}{)}\PYG{p}{\PYGZob{}}
        \PYG{n+nx}{impuestos} \PYG{o}{=} \PYG{n+nx}{impuestos} \PYG{o}{\PYGZhy{}} \PYG{l+m+mi}{245}\PYG{p}{;}
    \PYG{p}{\PYGZcb{}}
    \PYG{k}{if} \PYG{p}{(}\PYG{n+nx}{isChecked}\PYG{p}{(}\PYG{l+s+s2}{\PYGZdq{}\PYGZsh{}bonificacion\PYGZus{}b2\PYGZdq{}}\PYG{p}{)}\PYG{p}{)}\PYG{p}{\PYGZob{}}
        \PYG{n+nx}{impuestos} \PYG{o}{=} \PYG{n+nx}{impuestos} \PYG{o}{\PYGZhy{}} \PYG{l+m+mi}{535}\PYG{p}{;}
    \PYG{p}{\PYGZcb{}}
    
    \PYG{k+kd}{let} \PYG{n+nx}{div\PYGZus{}mensajes}\PYG{o}{=}\PYG{n+nx}{\PYGZdl{}}\PYG{p}{(}\PYG{l+s+s2}{\PYGZdq{}\PYGZsh{}mensajes\PYGZdq{}}\PYG{p}{)}\PYG{p}{;}
    \PYG{n+nx}{div\PYGZus{}mensajes}\PYG{p}{.}\PYG{n+nx}{html}\PYG{p}{(}\PYG{l+s+s2}{\PYGZdq{}Sus \PYGZlt{}b\PYGZgt{}impuestos\PYGZlt{}/b\PYGZgt{} son:\PYGZdq{}}\PYG{o}{+}\PYG{n+nx}{impuestos}\PYG{p}{)}\PYG{p}{;}
    \PYG{k}{return} \PYG{k+kc}{false}\PYG{p}{;}
    
\PYG{p}{\PYGZcb{}}
\PYG{k+kd}{function} \PYG{n+nx}{main}\PYG{p}{(}\PYG{p}{)}\PYG{p}{\PYGZob{}}
    \PYG{k+kd}{let} \PYG{n+nx}{boton}\PYG{o}{=}\PYG{n+nx}{\PYGZdl{}}\PYG{p}{(}\PYG{l+s+s2}{\PYGZdq{}\PYGZsh{}calcular\PYGZdq{}}\PYG{p}{)}\PYG{p}{;}
    \PYG{n+nx}{boton}\PYG{p}{.}\PYG{n+nx}{click} \PYG{p}{(} \PYG{n+nx}{calcular} \PYG{p}{)}\PYG{p}{;}
    
\PYG{p}{\PYGZcb{}}
\PYG{k+kd}{let} \PYG{n+nx}{objeto\PYGZus{}documento}\PYG{o}{=}\PYG{n+nx}{\PYGZdl{}}\PYG{p}{(}\PYG{n+nb}{document}\PYG{p}{)}\PYG{p}{;}
\PYG{n+nx}{objeto\PYGZus{}documento}\PYG{p}{.}\PYG{n+nx}{ready}\PYG{p}{(}\PYG{n+nx}{main}\PYG{p}{)}\PYG{p}{;}
\end{sphinxVerbatim}


\subsection{JS Calculador (sin JQuery (ASIR))}
\label{\detokenize{tema4:js-calculador-sin-jquery-asir}}
\begin{sphinxVerbatim}[commandchars=\\\{\}]
\PYG{k+kd}{function} \PYG{n+nx}{limpiar\PYGZus{}errores}\PYG{p}{(}\PYG{p}{)} \PYG{p}{\PYGZob{}}
    \PYG{k+kd}{let} \PYG{n+nx}{objeto\PYGZus{}diesel}\PYG{o}{=}\PYG{n+nb}{document}\PYG{p}{.}\PYG{n+nx}{getElementById}\PYG{p}{(}\PYG{l+s+s2}{\PYGZdq{}diesel\PYGZdq{}}\PYG{p}{)}\PYG{p}{;}
    \PYG{k+kd}{let} \PYG{n+nx}{objeto\PYGZus{}pot2300}\PYG{o}{=}\PYG{n+nb}{document}\PYG{p}{.}\PYG{n+nx}{getElementById}\PYG{p}{(}\PYG{l+s+s2}{\PYGZdq{}pot2300\PYGZdq{}}\PYG{p}{)}\PYG{p}{;}
    \PYG{k}{if} \PYG{p}{(}\PYG{p}{(}\PYG{n+nx}{objeto\PYGZus{}diesel}\PYG{p}{.}\PYG{n+nx}{checked}\PYG{p}{)} \PYG{o}{\PYGZam{}\PYGZam{}} \PYG{p}{(}\PYG{n+nx}{objeto\PYGZus{}pot2300}\PYG{p}{.}\PYG{n+nx}{checked}\PYG{p}{)}\PYG{p}{)} \PYG{p}{\PYGZob{}}
        \PYG{n+nx}{alert}\PYG{p}{(}\PYG{l+s+s2}{\PYGZdq{}Incompatibilidad diesel+2300\PYGZdq{}}\PYG{p}{)}\PYG{p}{;}
        \PYG{n+nx}{objeto\PYGZus{}pot2300}\PYG{p}{.}\PYG{n+nx}{checked}\PYG{o}{=}\PYG{k+kc}{false}\PYG{p}{;}
    \PYG{p}{\PYGZcb{}}
    
    \PYG{k+kd}{let} \PYG{n+nx}{objeto\PYGZus{}pintura\PYGZus{}normal}\PYG{o}{=}\PYG{n+nb}{document}\PYG{p}{.}\PYG{n+nx}{getElementById}\PYG{p}{(}\PYG{l+s+s2}{\PYGZdq{}normal\PYGZdq{}}\PYG{p}{)}\PYG{p}{;}
    \PYG{k+kd}{let} \PYG{n+nx}{objeto\PYGZus{}color\PYGZus{}gris} \PYG{o}{=}\PYG{n+nb}{document}\PYG{p}{.}\PYG{n+nx}{getElementById}\PYG{p}{(}\PYG{l+s+s2}{\PYGZdq{}gris\PYGZdq{}}\PYG{p}{)}\PYG{p}{;}
    \PYG{k+kd}{let} \PYG{n+nx}{objeto\PYGZus{}color\PYGZus{}azul} \PYG{o}{=}\PYG{n+nb}{document}\PYG{p}{.}\PYG{n+nx}{getElementById}\PYG{p}{(}\PYG{l+s+s2}{\PYGZdq{}azul\PYGZdq{}}\PYG{p}{)}\PYG{p}{;}
    \PYG{k+kd}{let} \PYG{n+nx}{objeto\PYGZus{}color\PYGZus{}verde}\PYG{o}{=}\PYG{n+nb}{document}\PYG{p}{.}\PYG{n+nx}{getElementById}\PYG{p}{(}\PYG{l+s+s2}{\PYGZdq{}verde\PYGZdq{}}\PYG{p}{)}\PYG{p}{;}
    
    \PYG{k}{if} \PYG{p}{(} \PYG{p}{(}\PYG{n+nx}{objeto\PYGZus{}pintura\PYGZus{}normal}\PYG{p}{.}\PYG{n+nx}{checked}\PYG{p}{)}
        \PYG{o}{\PYGZam{}\PYGZam{}} \PYG{p}{(}\PYG{n+nx}{objeto\PYGZus{}color\PYGZus{}gris}\PYG{p}{.}\PYG{n+nx}{checked}\PYG{p}{)} \PYG{p}{)}\PYG{p}{\PYGZob{}}
        \PYG{n+nx}{alert}\PYG{p}{(}\PYG{l+s+s2}{\PYGZdq{}Incompatibilidad normal+gris\PYGZdq{}}\PYG{p}{)}\PYG{p}{;}
        \PYG{n+nx}{objeto\PYGZus{}color\PYGZus{}gris}\PYG{p}{.}\PYG{n+nx}{checked}\PYG{o}{=}\PYG{k+kc}{false}\PYG{p}{;}
    \PYG{p}{\PYGZcb{}}
    \PYG{k}{if} \PYG{p}{(} \PYG{p}{(}\PYG{n+nx}{objeto\PYGZus{}pintura\PYGZus{}normal}\PYG{p}{.}\PYG{n+nx}{checked}\PYG{p}{)}
        \PYG{o}{\PYGZam{}\PYGZam{}} \PYG{p}{(}\PYG{n+nx}{objeto\PYGZus{}color\PYGZus{}azul}\PYG{p}{.}\PYG{n+nx}{checked}\PYG{p}{)} \PYG{p}{)}\PYG{p}{\PYGZob{}}
        \PYG{n+nx}{alert}\PYG{p}{(}\PYG{l+s+s2}{\PYGZdq{}Incompatibilidad normal+azul\PYGZdq{}}\PYG{p}{)}\PYG{p}{;}
        \PYG{n+nx}{objeto\PYGZus{}color\PYGZus{}azul}\PYG{p}{.}\PYG{n+nx}{checked}\PYG{o}{=}\PYG{k+kc}{false}\PYG{p}{;}
    \PYG{p}{\PYGZcb{}}
    \PYG{k}{if} \PYG{p}{(} \PYG{p}{(}\PYG{n+nx}{objeto\PYGZus{}pintura\PYGZus{}normal}\PYG{p}{.}\PYG{n+nx}{checked}\PYG{p}{)}
        \PYG{o}{\PYGZam{}\PYGZam{}} \PYG{p}{(}\PYG{n+nx}{objeto\PYGZus{}color\PYGZus{}verde}\PYG{p}{.}\PYG{n+nx}{checked}\PYG{p}{)} \PYG{p}{)}\PYG{p}{\PYGZob{}}
        \PYG{n+nx}{alert}\PYG{p}{(}\PYG{l+s+s2}{\PYGZdq{}Incompatibilidad normal+verde\PYGZdq{}}\PYG{p}{)}\PYG{p}{;}
        \PYG{n+nx}{objeto\PYGZus{}color\PYGZus{}verde}\PYG{p}{.}\PYG{n+nx}{checked}\PYG{o}{=}\PYG{k+kc}{false}\PYG{p}{;}
    \PYG{p}{\PYGZcb{}}
    
\PYG{p}{\PYGZcb{}}
\PYG{k+kd}{function} \PYG{n+nx}{calcular}\PYG{p}{(}\PYG{p}{)} \PYG{p}{\PYGZob{}}
    
    \PYG{n+nx}{limpiar\PYGZus{}errores}\PYG{p}{(}\PYG{p}{)}\PYG{p}{;}
    
    \PYG{k+kd}{let} \PYG{n+nx}{precio}\PYG{o}{=}\PYG{l+m+mi}{0}\PYG{p}{;}
    
    \PYG{k+kd}{let} \PYG{n+nx}{objeto\PYGZus{}gasolina}\PYG{o}{=}\PYG{n+nb}{document}\PYG{p}{.}\PYG{n+nx}{getElementById}\PYG{p}{(}\PYG{l+s+s2}{\PYGZdq{}gasolina\PYGZdq{}}\PYG{p}{)}\PYG{p}{;}
    \PYG{k}{if} \PYG{p}{(}\PYG{n+nx}{objeto\PYGZus{}gasolina}\PYG{p}{.}\PYG{n+nx}{checked}\PYG{p}{)} \PYG{p}{\PYGZob{}}
        \PYG{n+nx}{precio}\PYG{o}{=}\PYG{n+nx}{precio} \PYG{o}{+} \PYG{l+m+mi}{7000}\PYG{p}{;}
    \PYG{p}{\PYGZcb{}}
    \PYG{k+kd}{let} \PYG{n+nx}{objeto\PYGZus{}diesel}\PYG{o}{=}\PYG{n+nb}{document}\PYG{p}{.}\PYG{n+nx}{getElementById}\PYG{p}{(}\PYG{l+s+s2}{\PYGZdq{}diesel\PYGZdq{}}\PYG{p}{)}\PYG{p}{;}
    \PYG{k}{if} \PYG{p}{(}\PYG{n+nx}{objeto\PYGZus{}diesel}\PYG{p}{.}\PYG{n+nx}{checked}\PYG{p}{)} \PYG{p}{\PYGZob{}}
        \PYG{n+nx}{precio}\PYG{o}{=}\PYG{n+nx}{precio} \PYG{o}{+} \PYG{l+m+mi}{8200}\PYG{p}{;}
    \PYG{p}{\PYGZcb{}}
    
    \PYG{k+kd}{let} \PYG{n+nx}{objeto\PYGZus{}pot1100}\PYG{o}{=}\PYG{n+nb}{document}\PYG{p}{.}\PYG{n+nx}{getElementById}\PYG{p}{(}\PYG{l+s+s2}{\PYGZdq{}pot1100\PYGZdq{}}\PYG{p}{)}\PYG{p}{;}
    \PYG{k}{if} \PYG{p}{(}\PYG{n+nx}{objeto\PYGZus{}pot1100}\PYG{p}{.}\PYG{n+nx}{checked}\PYG{p}{)} \PYG{p}{\PYGZob{}}
        \PYG{n+nx}{precio}\PYG{o}{=}\PYG{n+nx}{precio} \PYG{o}{+} \PYG{l+m+mi}{800}\PYG{p}{;}
    \PYG{p}{\PYGZcb{}}
    \PYG{k+kd}{let} \PYG{n+nx}{objeto\PYGZus{}pot1800}\PYG{o}{=}\PYG{n+nb}{document}\PYG{p}{.}\PYG{n+nx}{getElementById}\PYG{p}{(}\PYG{l+s+s2}{\PYGZdq{}pot1800\PYGZdq{}}\PYG{p}{)}\PYG{p}{;}
    \PYG{k}{if} \PYG{p}{(}\PYG{n+nx}{objeto\PYGZus{}pot1800}\PYG{p}{.}\PYG{n+nx}{checked}\PYG{p}{)} \PYG{p}{\PYGZob{}}
        \PYG{n+nx}{precio}\PYG{o}{=}\PYG{n+nx}{precio} \PYG{o}{+} \PYG{l+m+mi}{1900}\PYG{p}{;}
    \PYG{p}{\PYGZcb{}}
    \PYG{k+kd}{let} \PYG{n+nx}{objeto\PYGZus{}pot2300}\PYG{o}{=}\PYG{n+nb}{document}\PYG{p}{.}\PYG{n+nx}{getElementById}\PYG{p}{(}\PYG{l+s+s2}{\PYGZdq{}pot2300\PYGZdq{}}\PYG{p}{)}\PYG{p}{;}
    \PYG{k}{if} \PYG{p}{(}\PYG{n+nx}{objeto\PYGZus{}pot2300}\PYG{p}{.}\PYG{n+nx}{checked}\PYG{p}{)} \PYG{p}{\PYGZob{}}
        \PYG{n+nx}{precio}\PYG{o}{=}\PYG{n+nx}{precio} \PYG{o}{+} \PYG{l+m+mi}{2500}\PYG{p}{;}
    \PYG{p}{\PYGZcb{}}
    
    
    \PYG{c+c1}{//Mostramos el precio final}
    \PYG{k+kd}{let} \PYG{n+nx}{objeto\PYGZus{}mensajes}\PYG{o}{=}\PYG{n+nb}{document}\PYG{p}{.}\PYG{n+nx}{getElementById}\PYG{p}{(}\PYG{l+s+s2}{\PYGZdq{}mensajes\PYGZdq{}}\PYG{p}{)}\PYG{p}{;}
    \PYG{n+nx}{objeto\PYGZus{}mensajes}\PYG{p}{.}\PYG{n+nx}{innerHTML}\PYG{o}{=}\PYG{l+s+s2}{\PYGZdq{}Precio \PYGZlt{}u\PYGZgt{}final\PYGZlt{}/u\PYGZgt{}\PYGZdq{}}\PYG{o}{+}\PYG{n+nx}{precio}\PYG{p}{;}
\PYG{p}{\PYGZcb{}}
\end{sphinxVerbatim}


\section{Comparador de telefonía}
\label{\detokenize{tema4:comparador-de-telefonia}}
Se desea crear una aplicación que permita al usuario saber qué compañía de telefonía le conviene más partiendo de los siguientes datos
\begin{itemize}
\item {} 
La empresa A ofrece una tarifa que cuesta 20 euros al mes con los 1000 primeros minutos gratis y despues cada minuto cuesta 8 céntimos.

\item {} 
La empresa B ofrece una tarifa que cuesta 10 euros al mes con los 500 primeros minutos gratis y despues cada llamada cuesta 12 céntimos.

\end{itemize}

Hacer una aplicación que permita al usuario introducir la cantidad de minutos que llamará para tres meses distintos que llamaremos «Mes 1», «Mes 2» y «Mes 3»  y que le diga qué compañía le interesa más.


\subsection{HTML del comparador}
\label{\detokenize{tema4:html-del-comparador}}
\begin{sphinxVerbatim}[commandchars=\\\{\}]
\PYG{p}{\PYGZlt{}}\PYG{n+nt}{form}\PYG{p}{\PYGZgt{}}
Mes 1\PYG{p}{\PYGZlt{}}\PYG{n+nt}{input} \PYG{n+na}{type}\PYG{o}{=}\PYG{l+s}{\PYGZdq{}number\PYGZdq{}} \PYG{n+na}{value}\PYG{o}{=}\PYG{l+s}{\PYGZdq{}800\PYGZdq{}}
                         \PYG{n+na}{min}\PYG{o}{=}\PYG{l+s}{\PYGZdq{}0\PYGZdq{}} \PYG{n+na}{max}\PYG{o}{=}\PYG{l+s}{\PYGZdq{}5000\PYGZdq{}} \PYG{n+na}{id}\PYG{o}{=}\PYG{l+s}{\PYGZdq{}mes1\PYGZdq{}}\PYG{p}{\PYGZgt{}} \PYG{p}{\PYGZlt{}}\PYG{n+nt}{br}\PYG{p}{/}\PYG{p}{\PYGZgt{}}
Mes 2\PYG{p}{\PYGZlt{}}\PYG{n+nt}{input} \PYG{n+na}{type}\PYG{o}{=}\PYG{l+s}{\PYGZdq{}number\PYGZdq{}} \PYG{n+na}{value}\PYG{o}{=}\PYG{l+s}{\PYGZdq{}2000\PYGZdq{}}
                         \PYG{n+na}{min}\PYG{o}{=}\PYG{l+s}{\PYGZdq{}0\PYGZdq{}} \PYG{n+na}{max}\PYG{o}{=}\PYG{l+s}{\PYGZdq{}5000\PYGZdq{}} \PYG{n+na}{id}\PYG{o}{=}\PYG{l+s}{\PYGZdq{}mes2\PYGZdq{}}\PYG{p}{\PYGZgt{}} \PYG{p}{\PYGZlt{}}\PYG{n+nt}{br}\PYG{p}{/}\PYG{p}{\PYGZgt{}}
Mes 3\PYG{p}{\PYGZlt{}}\PYG{n+nt}{input} \PYG{n+na}{type}\PYG{o}{=}\PYG{l+s}{\PYGZdq{}number\PYGZdq{}} \PYG{n+na}{value}\PYG{o}{=}\PYG{l+s}{\PYGZdq{}600\PYGZdq{}}
                         \PYG{n+na}{min}\PYG{o}{=}\PYG{l+s}{\PYGZdq{}0\PYGZdq{}} \PYG{n+na}{max}\PYG{o}{=}\PYG{l+s}{\PYGZdq{}5000\PYGZdq{}} \PYG{n+na}{id}\PYG{o}{=}\PYG{l+s}{\PYGZdq{}mes3\PYGZdq{}}\PYG{p}{\PYGZgt{}} \PYG{p}{\PYGZlt{}}\PYG{n+nt}{br}\PYG{p}{/}\PYG{p}{\PYGZgt{}}
\PYG{p}{\PYGZlt{}}\PYG{n+nt}{input} \PYG{n+na}{type}\PYG{o}{=}\PYG{l+s}{\PYGZdq{}submit\PYGZdq{}} \PYG{n+na}{value}\PYG{o}{=}\PYG{l+s}{\PYGZdq{}¿Qué me conviene?\PYGZdq{}}
                         \PYG{n+na}{onclick}\PYG{o}{=}\PYG{l+s}{\PYGZdq{}calcular();return false\PYGZdq{}}\PYG{p}{\PYGZgt{}}
\PYG{p}{\PYGZlt{}}\PYG{p}{/}\PYG{n+nt}{form}\PYG{p}{\PYGZgt{}}
\end{sphinxVerbatim}


\subsection{JS del comparador}
\label{\detokenize{tema4:js-del-comparador}}
\begin{sphinxVerbatim}[commandchars=\\\{\}]
\PYG{k+kd}{function} \PYG{n+nx}{extraer\PYGZus{}numero}\PYG{p}{(}\PYG{n+nx}{id}\PYG{p}{)}\PYG{p}{\PYGZob{}}
        \PYG{k+kd}{var} \PYG{n+nx}{control}\PYG{p}{;}
        \PYG{n+nx}{control}\PYG{o}{=}\PYG{n+nb}{document}\PYG{p}{.}\PYG{n+nx}{getElementById}\PYG{p}{(}\PYG{n+nx}{id}\PYG{p}{)}\PYG{p}{;}
        \PYG{k+kd}{var} \PYG{n+nx}{valor}\PYG{o}{=}\PYG{n+nx}{control}\PYG{p}{.}\PYG{n+nx}{value}\PYG{p}{;}
        \PYG{k+kd}{var} \PYG{n+nx}{numero}\PYG{o}{=}\PYG{n+nb}{parseInt}\PYG{p}{(}\PYG{n+nx}{valor}\PYG{p}{)}\PYG{p}{;}
        \PYG{k}{return} \PYG{n+nx}{numero}\PYG{p}{;}
\PYG{p}{\PYGZcb{}}

\PYG{k+kd}{function} \PYG{n+nx}{precio\PYGZus{}mes\PYGZus{}empresa\PYGZus{}a}\PYG{p}{(}\PYG{n+nx}{minutos}\PYG{p}{)}\PYG{p}{\PYGZob{}}
        \PYG{c+c1}{//Primero vemos si se pasa}
        \PYG{k}{if} \PYG{p}{(}\PYG{n+nx}{minutos}\PYG{o}{\PYGZlt{}}\PYG{l+m+mi}{1000}\PYG{p}{)}\PYG{p}{\PYGZob{}}
                \PYG{c+c1}{//Si no se pasa, el precio es 20}
                \PYG{k}{return} \PYG{l+m+mi}{20}\PYG{p}{;}
        \PYG{p}{\PYGZcb{}}
        \PYG{c+c1}{//Si llegamos a este punto, es porque se pasa}

        \PYG{c+c1}{//Calculamos cuantos minutos se pasa}
        \PYG{k+kd}{var} \PYG{n+nx}{exceso}\PYG{o}{=}\PYG{n+nx}{minutos}\PYG{o}{\PYGZhy{}}\PYG{l+m+mi}{1000}\PYG{p}{;}
        \PYG{c+c1}{//Y ese exceso es a 8 centimos/minutos}
        \PYG{k+kd}{var} \PYG{n+nx}{precio\PYGZus{}exceso}\PYG{o}{=}\PYG{n+nx}{exceso}\PYG{o}{*}\PYG{l+m+mf}{0.08}\PYG{p}{;}
        \PYG{c+c1}{//El precio de ese mes es}
        \PYG{c+c1}{//20 euros más lo que cueste el exceso}
        \PYG{k+kd}{var} \PYG{n+nx}{precio\PYGZus{}mes}\PYG{o}{=}\PYG{l+m+mi}{20}\PYG{o}{+}\PYG{n+nx}{precio\PYGZus{}exceso}\PYG{p}{;}
        \PYG{c+c1}{//Devolvemos el calculo a quien lo pidiese}
        \PYG{k}{return} \PYG{n+nx}{precio\PYGZus{}mes}\PYG{p}{;}
\PYG{p}{\PYGZcb{}}

\PYG{k+kd}{function} \PYG{n+nx}{precio\PYGZus{}mes\PYGZus{}empresa\PYGZus{}b}\PYG{p}{(}\PYG{n+nx}{minutos}\PYG{p}{)}\PYG{p}{\PYGZob{}}
        \PYG{c+c1}{//Primero vemos si se pasa}
        \PYG{k}{if} \PYG{p}{(}\PYG{n+nx}{minutos}\PYG{o}{\PYGZlt{}}\PYG{l+m+mi}{500}\PYG{p}{)}\PYG{p}{\PYGZob{}}
                \PYG{c+c1}{//Si no se pasa, el precio es 10}
                \PYG{k}{return} \PYG{l+m+mi}{10}\PYG{p}{;}
        \PYG{p}{\PYGZcb{}}
        \PYG{c+c1}{//Si llegamos a este punto, es porque se pasa}

        \PYG{c+c1}{//Calculamos cuantos minutos se pasa}
        \PYG{k+kd}{var} \PYG{n+nx}{exceso}\PYG{o}{=}\PYG{n+nx}{minutos}\PYG{o}{\PYGZhy{}}\PYG{l+m+mi}{500}\PYG{p}{;}
        \PYG{c+c1}{//Y ese exceso es a 12 centimos/minutos}
        \PYG{k+kd}{var} \PYG{n+nx}{precio\PYGZus{}exceso}\PYG{o}{=}\PYG{n+nx}{exceso}\PYG{o}{*}\PYG{l+m+mf}{0.12}\PYG{p}{;}
        \PYG{c+c1}{//El precio de ese mes es}
        \PYG{c+c1}{//10 euros más lo que cueste el exceso}
        \PYG{k+kd}{var} \PYG{n+nx}{precio\PYGZus{}mes}\PYG{o}{=}\PYG{l+m+mi}{10}\PYG{o}{+}\PYG{n+nx}{precio\PYGZus{}exceso}\PYG{p}{;}
        \PYG{c+c1}{//Devolvemos el calculo a quien lo pidiese}
        \PYG{k}{return} \PYG{n+nx}{precio\PYGZus{}mes}\PYG{p}{;}
\PYG{p}{\PYGZcb{}}

\PYG{k+kd}{function} \PYG{n+nx}{calcular}\PYG{p}{(}\PYG{p}{)}\PYG{p}{\PYGZob{}}
        \PYG{k+kd}{var} \PYG{n+nx}{minutos\PYGZus{}mes1}\PYG{p}{;}
        \PYG{n+nx}{minutos\PYGZus{}mes1}\PYG{o}{=}\PYG{n+nx}{extraer\PYGZus{}numero}\PYG{p}{(}\PYG{l+s+s2}{\PYGZdq{}mes1\PYGZdq{}}\PYG{p}{)}\PYG{p}{;}
        \PYG{k+kd}{var} \PYG{n+nx}{minutos\PYGZus{}mes2}\PYG{p}{;}
        \PYG{n+nx}{minutos\PYGZus{}mes2}\PYG{o}{=}\PYG{n+nx}{extraer\PYGZus{}numero}\PYG{p}{(}\PYG{l+s+s2}{\PYGZdq{}mes2\PYGZdq{}}\PYG{p}{)}\PYG{p}{;}
        \PYG{k+kd}{var} \PYG{n+nx}{minutos\PYGZus{}mes3}\PYG{p}{;}
        \PYG{n+nx}{minutos\PYGZus{}mes3}\PYG{o}{=}\PYG{n+nx}{extraer\PYGZus{}numero}\PYG{p}{(}\PYG{l+s+s2}{\PYGZdq{}mes3\PYGZdq{}}\PYG{p}{)}\PYG{p}{;}

        \PYG{k+kd}{var} \PYG{n+nx}{coste\PYGZus{}mes1\PYGZus{}empresa\PYGZus{}a}\PYG{p}{;}
        \PYG{n+nx}{coste\PYGZus{}mes1\PYGZus{}empresa\PYGZus{}a}\PYG{o}{=}\PYG{n+nx}{precio\PYGZus{}mes\PYGZus{}empresa\PYGZus{}a}\PYG{p}{(}
                \PYG{n+nx}{minutos\PYGZus{}mes1}\PYG{p}{)}\PYG{p}{;}

        \PYG{n+nx}{coste\PYGZus{}mes2\PYGZus{}empresa\PYGZus{}a}\PYG{o}{=}\PYG{n+nx}{precio\PYGZus{}mes\PYGZus{}empresa\PYGZus{}a}\PYG{p}{(}
                \PYG{n+nx}{minutos\PYGZus{}mes2}\PYG{p}{)}\PYG{p}{;}
        \PYG{n+nx}{coste\PYGZus{}mes3\PYGZus{}empresa\PYGZus{}a}\PYG{o}{=}\PYG{n+nx}{precio\PYGZus{}mes\PYGZus{}empresa\PYGZus{}a}\PYG{p}{(}
                \PYG{n+nx}{minutos\PYGZus{}mes3}\PYG{p}{)}\PYG{p}{;}

        \PYG{c+c1}{//Calculamos el precio que costaria}
        \PYG{c+c1}{//esos minutos en la empresa A}
        \PYG{k+kd}{var} \PYG{n+nx}{coste\PYGZus{}total\PYGZus{}a}\PYG{p}{;}
        \PYG{n+nx}{coste\PYGZus{}total\PYGZus{}a}\PYG{o}{=}\PYG{n+nx}{coste\PYGZus{}mes1\PYGZus{}empresa\PYGZus{}a}\PYG{o}{+}\PYG{n+nx}{coste\PYGZus{}mes2\PYGZus{}empresa\PYGZus{}a}\PYG{o}{+}\PYG{n+nx}{coste\PYGZus{}mes3\PYGZus{}empresa\PYGZus{}a}\PYG{p}{;}

        \PYG{c+c1}{//Repetimos el proceso para la empresa B}
        \PYG{k+kd}{var} \PYG{n+nx}{coste\PYGZus{}mes1\PYGZus{}empresa\PYGZus{}b}\PYG{p}{;}
        \PYG{n+nx}{coste\PYGZus{}mes1\PYGZus{}empresa\PYGZus{}b}\PYG{o}{=}\PYG{n+nx}{precio\PYGZus{}mes\PYGZus{}empresa\PYGZus{}b}\PYG{p}{(}
                \PYG{n+nx}{minutos\PYGZus{}mes1}\PYG{p}{)}\PYG{p}{;}
        \PYG{n+nx}{coste\PYGZus{}mes2\PYGZus{}empresa\PYGZus{}b}\PYG{o}{=}\PYG{n+nx}{precio\PYGZus{}mes\PYGZus{}empresa\PYGZus{}b}\PYG{p}{(}
                \PYG{n+nx}{minutos\PYGZus{}mes2}\PYG{p}{)}\PYG{p}{;}
        \PYG{n+nx}{coste\PYGZus{}mes3\PYGZus{}empresa\PYGZus{}b}\PYG{o}{=}\PYG{n+nx}{precio\PYGZus{}mes\PYGZus{}empresa\PYGZus{}b}\PYG{p}{(}
                \PYG{n+nx}{minutos\PYGZus{}mes3}\PYG{p}{)}\PYG{p}{;}
        \PYG{c+c1}{//Calculamos ahora el precio que costaria}
        \PYG{c+c1}{//esos minutos en la empresa B}
        \PYG{k+kd}{var} \PYG{n+nx}{coste\PYGZus{}total\PYGZus{}b}\PYG{p}{;}
        \PYG{n+nx}{coste\PYGZus{}total\PYGZus{}b}\PYG{o}{=}\PYG{n+nx}{coste\PYGZus{}mes1\PYGZus{}empresa\PYGZus{}b}\PYG{o}{+}\PYG{n+nx}{coste\PYGZus{}mes2\PYGZus{}empresa\PYGZus{}b}\PYG{o}{+}\PYG{n+nx}{coste\PYGZus{}mes3\PYGZus{}empresa\PYGZus{}b}\PYG{p}{;}
        \PYG{k+kd}{var} \PYG{n+nx}{zonaresultados}\PYG{p}{;}
        \PYG{n+nx}{zonaresultados}\PYG{o}{=}\PYG{n+nb}{document}\PYG{p}{.}\PYG{n+nx}{getElementById}\PYG{p}{(}\PYG{l+s+s2}{\PYGZdq{}zonaresultados\PYGZdq{}}\PYG{p}{)}\PYG{p}{;}
        \PYG{k+kd}{var} \PYG{n+nx}{texto}\PYG{p}{;}
        \PYG{n+nx}{texto}\PYG{o}{=}\PYG{l+s+s2}{\PYGZdq{}Con la A el precio es:\PYGZdq{}} \PYG{o}{+} \PYG{n+nx}{coste\PYGZus{}total\PYGZus{}a}\PYG{p}{;}
        \PYG{n+nx}{texto}\PYG{o}{=}\PYG{n+nx}{texto}\PYG{o}{+}\PYG{l+s+s2}{\PYGZdq{}\PYGZlt{}br/\PYGZgt{}\PYGZdq{}}\PYG{p}{;}
        \PYG{n+nx}{texto}\PYG{o}{=}\PYG{n+nx}{texto}\PYG{o}{+}\PYG{l+s+s2}{\PYGZdq{}Con la B el precio es:\PYGZdq{}} \PYG{o}{+} \PYG{n+nx}{coste\PYGZus{}total\PYGZus{}b}\PYG{p}{;}
        \PYG{n+nx}{zonaresultados}\PYG{p}{.}\PYG{n+nx}{innerHTML}\PYG{o}{=}\PYG{n+nx}{texto}
\PYG{p}{\PYGZcb{}}
\end{sphinxVerbatim}


\chapter{XML}
\label{\detokenize{tema5::doc}}\label{\detokenize{tema5:xml}}

\section{Introducción}
\label{\detokenize{tema5:introduccion}}
Los lenguajes de marcas como HTML tienen una orientación muy clara: describir páginas web.

En un contexto distinto, muy a menudo ocurre que es muy difícil intercambiar datos entre programas.

XML es un conjunto de tecnologías orientadas a crear nuestros propios lenguajes de marcas. A estos lenguajes de marcas «propios» se les denomina «vocabularios».


\section{Un ejemplo sencillo}
\label{\detokenize{tema5:un-ejemplo-sencillo}}
\begin{sphinxVerbatim}[commandchars=\\\{\}]
\PYG{n+nt}{\PYGZlt{}clientes}\PYG{n+nt}{\PYGZgt{}}
        \PYG{n+nt}{\PYGZlt{}cliente}\PYG{n+nt}{\PYGZgt{}}
                \PYG{n+nt}{\PYGZlt{}nombre}\PYG{n+nt}{\PYGZgt{}}AcerSA\PYG{n+nt}{\PYGZlt{}/nombre\PYGZgt{}}
                \PYG{n+nt}{\PYGZlt{}cif}\PYG{n+nt}{\PYGZgt{}}5664332\PYG{n+nt}{\PYGZlt{}/cif\PYGZgt{}}
        \PYG{n+nt}{\PYGZlt{}/cliente\PYGZgt{}}
        \PYG{n+nt}{\PYGZlt{}cliente}\PYG{n+nt}{\PYGZgt{}}
                \PYG{n+nt}{\PYGZlt{}nombre}\PYG{n+nt}{\PYGZgt{}}Mer SL\PYG{n+nt}{\PYGZlt{}/nombre\PYGZgt{}}
                \PYG{n+nt}{\PYGZlt{}cif}\PYG{n+nt}{\PYGZgt{}}5111444\PYG{n+nt}{\PYGZlt{}/cif\PYGZgt{}}
        \PYG{n+nt}{\PYGZlt{}/cliente\PYGZgt{}}
\PYG{n+nt}{\PYGZlt{}/clientes\PYGZgt{}}
\end{sphinxVerbatim}

Lo fundamental es que podemos crear nuestros propios «vocabularios» XML.


\section{Construcción de XML}
\label{\detokenize{tema5:construccion-de-xml}}
Para crear XML es importante recordar una serie de reglas:
\begin{itemize}
\item {} 
XML es «case-sensitive», es decir que no es lo mismo mayúsculas que minúsculas y que por tanto no es lo mismo \sphinxcode{\textless{}cliente\textgreater{}}, que \sphinxcode{\textless{}Cliente\textgreater{}} que \sphinxcode{\textless{}CLIENTE\textgreater{}}.

\item {} 
Obligatorio: solo un elemento raíz.

\item {} 
En general, la costumbre es poner todo en minúsculas.

\item {} 
Solo se puede poner una etiqueta que empiece por letra o \_. Es decir, esta etiqueta no funcionará en los programas \sphinxcode{\textless{}12Cliente\textgreater{}}.

\item {} 
Aparte de eso, una etiqueta sí puede contener números, por lo que esta etiqueta sí es válida \sphinxcode{\textless{}Cliente12\textgreater{}}.

\end{itemize}


\section{Validez}
\label{\detokenize{tema5:validez}}
Un documento XML puede «estar bien formado» o «ser válido». Se dice que un documento «está bien formado» cuando respeta las reglas XML básicas. Si alguien ha definido las reglas XML para un vocabulario, podremos además decir si el documento es válido o no, lo cual es mejor que simplemente estar bien formado.

Por ejemplo, los siguientes archivos ni siquiera están bien formados.

\begin{sphinxVerbatim}[commandchars=\\\{\}]
\PYG{n+nt}{\PYGZlt{}clientes}\PYG{n+nt}{\PYGZgt{}}
        \PYG{n+nt}{\PYGZlt{}cliente}\PYG{n+nt}{\PYGZgt{}}
                \PYG{n+nt}{\PYGZlt{}nombre}\PYG{n+nt}{\PYGZgt{}}AcerSA
                \PYG{n+nt}{\PYGZlt{}CIF}\PYG{n+nt}{\PYGZgt{}}5666333\PYG{n+nt}{\PYGZlt{}/CIF\PYGZgt{}}
        \PYG{n+nt}{\PYGZlt{}/cliente\PYGZgt{}}
\PYG{n+nt}{\PYGZlt{}/clientes\PYGZgt{}}
\end{sphinxVerbatim}

En este caso la etiqueta \sphinxcode{\textless{}nombre\textgreater{}} no está cerrada.

\begin{sphinxVerbatim}[commandchars=\\\{\}]
\PYG{n+nt}{\PYGZlt{}clientes}\PYG{n+nt}{\PYGZgt{}}
        \PYG{n+nt}{\PYGZlt{}cliente}\PYG{n+nt}{\PYGZgt{}}
                \PYG{n+nt}{\PYGZlt{}nombre}\PYG{n+nt}{\PYGZgt{}}AcerSA\PYG{n+nt}{\PYGZlt{}/nombre\PYGZgt{}}
                \PYG{n+nt}{\PYGZlt{}cif}\PYG{n+nt}{\PYGZgt{}}5666333\PYG{n+nt}{\PYGZlt{}/CIF\PYGZgt{}}
        \PYG{n+nt}{\PYGZlt{}/cliente\PYGZgt{}}
\PYG{n+nt}{\PYGZlt{}/clientes\PYGZgt{}}
\end{sphinxVerbatim}

En este caso, se ha puesto \sphinxcode{\textless{}cif\textgreater{}} cerrado con \sphinxcode{\textless{}/CIF\textgreater{}} (mayúsculas).

\begin{sphinxVerbatim}[commandchars=\\\{\}]
\PYGZlt{}clientes\PYGZgt{}
        \PYGZlt{}cliente\PYGZgt{}
                \PYGZlt{}nombre!\PYGZgt{}AcerSA\PYGZlt{}/nombre!\PYGZgt{}
                \PYGZlt{}CIF\PYGZgt{}5666333\PYGZlt{}/CIF\PYGZgt{}
        \PYGZlt{}/cliente\PYGZgt{}
\PYGZlt{}/clientes\PYGZgt{}
\end{sphinxVerbatim}

Se ha utilizado la admiración, que no es válida (de hecho, el coloreador de sintaxis automático descubre
que no es XML y el fichero se muestra de manera literal)

Atención a este ejemplo:

\begin{sphinxVerbatim}[commandchars=\\\{\}]
\PYG{n+nt}{\PYGZlt{}cliente}\PYG{n+nt}{\PYGZgt{}}
        \PYG{n+nt}{\PYGZlt{}nombre}\PYG{n+nt}{\PYGZgt{}}AcerSA\PYG{n+nt}{\PYGZlt{}/nombre\PYGZgt{}}
        \PYG{n+nt}{\PYGZlt{}CIF}\PYG{n+nt}{\PYGZgt{}}5666333\PYG{n+nt}{\PYGZlt{}/CIF\PYGZgt{}}
\PYG{n+nt}{\PYGZlt{}/cliente\PYGZgt{}}
\PYG{n+nt}{\PYGZlt{}cliente}\PYG{n+nt}{\PYGZgt{}}
        \PYG{n+nt}{\PYGZlt{}nombre}\PYG{n+nt}{\PYGZgt{}}ACME\PYG{n+nt}{\PYGZlt{}/nombre\PYGZgt{}}
        \PYG{n+nt}{\PYGZlt{}CIF}\PYG{n+nt}{\PYGZgt{}}455321\PYG{n+nt}{\PYGZlt{}/CIF\PYGZgt{}}
\PYG{n+nt}{\PYGZlt{}/cliente\PYGZgt{}}
\end{sphinxVerbatim}

En este caso, el problema es que hay más de un elemento raíz.

En general, podemos asumir que un documento puede estar en uno de estos estados que de peor a mejor podríamos indicar así:
\begin{enumerate}
\item {} 
Mal formado (lo peor)

\item {} 
Bien formado.

\item {} 
Válido: está bien formado y además nos han dado las reglas para determinar si algo está bien o mal y el documento XML cumple dichas reglas. Este es el mejor caso.

\end{enumerate}

Para determinar si un documento es válido o no, se puede usar el validador del W3C situado en \sphinxurl{http://validator.w3c.org}


\section{Gramáticas}
\label{\detokenize{tema5:gramaticas}}
Pensemos en el siguiente problema, un programador crea aplicaciones con documentos que se almacenan así:

\begin{sphinxVerbatim}[commandchars=\\\{\}]
\PYG{n+nt}{\PYGZlt{}clientes}\PYG{n+nt}{\PYGZgt{}}
        \PYG{n+nt}{\PYGZlt{}cliente}\PYG{n+nt}{\PYGZgt{}}
                \PYG{n+nt}{\PYGZlt{}nombre}\PYG{n+nt}{\PYGZgt{}}AcerSA\PYG{n+nt}{\PYGZlt{}/nombre\PYGZgt{}}
                \PYG{n+nt}{\PYGZlt{}cif}\PYG{n+nt}{\PYGZgt{}}455321\PYG{n+nt}{\PYGZlt{}/cif\PYGZgt{}}
        \PYG{n+nt}{\PYGZlt{}/cliente\PYGZgt{}}
        \PYG{n+nt}{\PYGZlt{}cliente}\PYG{n+nt}{\PYGZgt{}}
                \PYG{n+nt}{\PYGZlt{}nombre}\PYG{n+nt}{\PYGZgt{}}ACME\PYG{n+nt}{\PYGZlt{}/nombre\PYGZgt{}}
                \PYG{n+nt}{\PYGZlt{}cif}\PYG{n+nt}{\PYGZgt{}}455321\PYG{n+nt}{\PYGZlt{}/cif\PYGZgt{}}
        \PYG{n+nt}{\PYGZlt{}/cliente\PYGZgt{}}
\PYG{n+nt}{\PYGZlt{}/clientes\PYGZgt{}}
\end{sphinxVerbatim}

Sin embargo, otro programador de la misma empresa lo hace así:

\begin{sphinxVerbatim}[commandchars=\\\{\}]
\PYG{n+nt}{\PYGZlt{}clientes}\PYG{n+nt}{\PYGZgt{}}
        \PYG{n+nt}{\PYGZlt{}cliente}\PYG{n+nt}{\PYGZgt{}}
                \PYG{n+nt}{\PYGZlt{}cif}\PYG{n+nt}{\PYGZgt{}}455321\PYG{n+nt}{\PYGZlt{}/cif\PYGZgt{}}
                \PYG{n+nt}{\PYGZlt{}nombre}\PYG{n+nt}{\PYGZgt{}}AcerSA\PYG{n+nt}{\PYGZlt{}/nombre\PYGZgt{}}
        \PYG{n+nt}{\PYGZlt{}/cliente\PYGZgt{}}
        \PYG{n+nt}{\PYGZlt{}cliente}\PYG{n+nt}{\PYGZgt{}}
                \PYG{n+nt}{\PYGZlt{}cif}\PYG{n+nt}{\PYGZgt{}}455321\PYG{n+nt}{\PYGZlt{}/cif\PYGZgt{}}
                \PYG{n+nt}{\PYGZlt{}nombre}\PYG{n+nt}{\PYGZgt{}}ACME\PYG{n+nt}{\PYGZlt{}/nombre\PYGZgt{}}
        \PYG{n+nt}{\PYGZlt{}/cliente\PYGZgt{}}
\PYG{n+nt}{\PYGZlt{}/clientes\PYGZgt{}}
\end{sphinxVerbatim}

Está claro, que ninguno de los dos puede leer los archivos del otro, sería crítico ponerse de acuerdo en lo que se puede hacer, lo que puede aparecer y en qué orden debe hacerlo. Esto se hará mediante las DTD.

DTD significa Declaración de Tipo de Documento, y es un mecanismo para expresar las reglas sobre lo que se va a permitir y lo que no en archivos XML.

Por ejemplo, supongamos el mismo ejemplo ejemplo anterior en el que queremos formalizar lo que puede aparecer en un fichero de clientes. Se debe tener en cuenta que en un DTD se pueden indicar reglas para lo siguiente:
\begin{itemize}
\item {} 
Se puede indicar si un elemento aparece o no de forma opcional (usando \sphinxcode{?})

\item {} 
Se puede indicar si un elemento debe aparecer de forma obligatoria.

\item {} 
Se puede indicar si algo aparecer una o muchas veces (usando \sphinxcode{+}).

\item {} 
Se puede indicar si algo aparece cero o muchas veces (usando \sphinxcode{*}).

\end{itemize}

Supongamos que en nuestros ficheros deseamos indicar que el elemento raíz es \sphinxcode{\textless{}listaclientes\textgreater{}}. Dentro de \sphinxcode{\textless{}listaclientes\textgreater{}} deseamos permitir uno o más elementos \sphinxcode{\textless{}cliente\textgreater{}}. Dentro de \sphinxcode{\textless{}cliente\textgreater{}} todos deberán tener \sphinxcode{\textless{}cif\textgreater{}} y \sphinxcode{\textless{}nombre\textgreater{}} y en ese orden. Dentro de \sphinxcode{\textless{}cliente\textgreater{}} puede aparecer o no un elemento \sphinxcode{\textless{}diasentrega\textgreater{}} para indicar que ese cliente exige un máximo de plazos. Como no todo el mundo usa plazos el \sphinxcode{\textless{}diasentrega\textgreater{}} es optativo.

Por ejemplo, este XML sí es válido:

\begin{sphinxVerbatim}[commandchars=\\\{\}]
\PYG{n+nt}{\PYGZlt{}listaclientes}\PYG{n+nt}{\PYGZgt{}}
        \PYG{n+nt}{\PYGZlt{}cliente}\PYG{n+nt}{\PYGZgt{}}
                \PYG{n+nt}{\PYGZlt{}cif}\PYG{n+nt}{\PYGZgt{}}5676443\PYG{n+nt}{\PYGZlt{}/cif\PYGZgt{}}
                \PYG{n+nt}{\PYGZlt{}nombre}\PYG{n+nt}{\PYGZgt{}}Mercasa\PYG{n+nt}{\PYGZlt{}/nombre\PYGZgt{}}
        \PYG{n+nt}{\PYGZlt{}/cliente\PYGZgt{}}
\PYG{n+nt}{\PYGZlt{}/listaclientes\PYGZgt{}}
\end{sphinxVerbatim}

Este también lo es:

\begin{sphinxVerbatim}[commandchars=\\\{\}]
\PYG{n+nt}{\PYGZlt{}listaclientes}\PYG{n+nt}{\PYGZgt{}}
        \PYG{n+nt}{\PYGZlt{}cliente}\PYG{n+nt}{\PYGZgt{}}
                \PYG{n+nt}{\PYGZlt{}cif}\PYG{n+nt}{\PYGZgt{}}5676443\PYG{n+nt}{\PYGZlt{}/cif\PYGZgt{}}
                \PYG{n+nt}{\PYGZlt{}nombre}\PYG{n+nt}{\PYGZgt{}}Mercasa\PYG{n+nt}{\PYGZlt{}/nombre\PYGZgt{}}
                \PYG{n+nt}{\PYGZlt{}diasentrega}\PYG{n+nt}{\PYGZgt{}}30\PYG{n+nt}{\PYGZlt{}/diasentrega\PYGZgt{}}
        \PYG{n+nt}{\PYGZlt{}/cliente\PYGZgt{}}
\PYG{n+nt}{\PYGZlt{}/listaclientes\PYGZgt{}}
\end{sphinxVerbatim}

Este también:

\begin{sphinxVerbatim}[commandchars=\\\{\}]
\PYG{n+nt}{\PYGZlt{}listaclientes}\PYG{n+nt}{\PYGZgt{}}
        \PYG{n+nt}{\PYGZlt{}cliente}\PYG{n+nt}{\PYGZgt{}}
                \PYG{n+nt}{\PYGZlt{}cif}\PYG{n+nt}{\PYGZgt{}}5676443\PYG{n+nt}{\PYGZlt{}/cif\PYGZgt{}}
                \PYG{n+nt}{\PYGZlt{}nombre}\PYG{n+nt}{\PYGZgt{}}Mercasa\PYG{n+nt}{\PYGZlt{}/nombre\PYGZgt{}}
                \PYG{n+nt}{\PYGZlt{}diasentrega}\PYG{n+nt}{\PYGZgt{}}30\PYG{n+nt}{\PYGZlt{}/diasentrega\PYGZgt{}}
        \PYG{n+nt}{\PYGZlt{}/cliente\PYGZgt{}}
        \PYG{n+nt}{\PYGZlt{}cliente}\PYG{n+nt}{\PYGZgt{}}
                \PYG{n+nt}{\PYGZlt{}cif}\PYG{n+nt}{\PYGZgt{}}5121554\PYG{n+nt}{\PYGZlt{}/cif\PYGZgt{}}
                \PYG{n+nt}{\PYGZlt{}nombre}\PYG{n+nt}{\PYGZgt{}}Acer SL\PYG{n+nt}{\PYGZlt{}/nombre\PYGZgt{}}
        \PYG{n+nt}{\PYGZlt{}/cliente\PYGZgt{}}
\PYG{n+nt}{\PYGZlt{}/listaclientes\PYGZgt{}}
\end{sphinxVerbatim}

Sin embargo, estos no lo son:

\begin{sphinxVerbatim}[commandchars=\\\{\}]
\PYG{n+nt}{\PYGZlt{}listaclientes}\PYG{n+nt}{\PYGZgt{}}
\PYG{n+nt}{\PYGZlt{}/listaclientes\PYGZgt{}}
\end{sphinxVerbatim}

Este archivo no tenía clientes (y era obligatorio al menos uno)

\begin{sphinxVerbatim}[commandchars=\\\{\}]
\PYG{n+nt}{\PYGZlt{}listaclientes}\PYG{n+nt}{\PYGZgt{}}
        \PYG{n+nt}{\PYGZlt{}cliente}\PYG{n+nt}{\PYGZgt{}}
                \PYG{n+nt}{\PYGZlt{}cif}\PYG{n+nt}{\PYGZgt{}}5676443\PYG{n+nt}{\PYGZlt{}/cif\PYGZgt{}}
                \PYG{n+nt}{\PYGZlt{}diasentrega}\PYG{n+nt}{\PYGZgt{}}30\PYG{n+nt}{\PYGZlt{}/diasentrega\PYGZgt{}}
        \PYG{n+nt}{\PYGZlt{}/cliente\PYGZgt{}}
\PYG{n+nt}{\PYGZlt{}/listaclientes\PYGZgt{}}
\end{sphinxVerbatim}

Este archivo no tiene nombre de cliente.

\begin{sphinxVerbatim}[commandchars=\\\{\}]
\PYG{n+nt}{\PYGZlt{}listaclientes}\PYG{n+nt}{\PYGZgt{}}
        \PYG{n+nt}{\PYGZlt{}cliente}\PYG{n+nt}{\PYGZgt{}}
                \PYG{n+nt}{\PYGZlt{}nombre}\PYG{n+nt}{\PYGZgt{}}Mercasa\PYG{n+nt}{\PYGZlt{}/nombre\PYGZgt{}}
                \PYG{n+nt}{\PYGZlt{}cif}\PYG{n+nt}{\PYGZgt{}}5676443\PYG{n+nt}{\PYGZlt{}/cif\PYGZgt{}}
        \PYG{n+nt}{\PYGZlt{}/cliente\PYGZgt{}}
        \PYG{n+nt}{\PYGZlt{}cliente}\PYG{n+nt}{\PYGZgt{}}
                \PYG{n+nt}{\PYGZlt{}cif}\PYG{n+nt}{\PYGZgt{}}5121554\PYG{n+nt}{\PYGZlt{}/cif\PYGZgt{}}
                \PYG{n+nt}{\PYGZlt{}nombre}\PYG{n+nt}{\PYGZgt{}}Acer SL\PYG{n+nt}{\PYGZlt{}/nombre\PYGZgt{}}
        \PYG{n+nt}{\PYGZlt{}/cliente\PYGZgt{}}
\PYG{n+nt}{\PYGZlt{}/listaclientes\PYGZgt{}}
\end{sphinxVerbatim}

En este archivo no se respeta el orden cif, nombre.


\subsection{Sintaxis DTD}
\label{\detokenize{tema5:sintaxis-dtd}}
Una DTD es como un CSS, puede ir en el mismo archivo XML o puede ir en uno separado. Para poder subirlos al validador, meteremos la DTD junto con el XML.

La primera línea de todo XML debe ser esta:

\begin{sphinxVerbatim}[commandchars=\\\{\}]
\PYG{c+cp}{\PYGZlt{}?xml version=\PYGZdq{}1.0\PYGZdq{}?\PYGZgt{}}
\end{sphinxVerbatim}

Al final del XML pondremos los datos propiamente dichos

\begin{sphinxVerbatim}[commandchars=\\\{\}]
\PYG{n+nt}{\PYGZlt{}listaclientes}\PYG{n+nt}{\PYGZgt{}}
        \PYG{n+nt}{\PYGZlt{}cliente}\PYG{n+nt}{\PYGZgt{}}
                \PYG{n+nt}{\PYGZlt{}nombre}\PYG{n+nt}{\PYGZgt{}}Mercasa\PYG{n+nt}{\PYGZlt{}/nombre\PYGZgt{}}
                \PYG{n+nt}{\PYGZlt{}cif}\PYG{n+nt}{\PYGZgt{}}5676443\PYG{n+nt}{\PYGZlt{}/cif\PYGZgt{}}
        \PYG{n+nt}{\PYGZlt{}/cliente\PYGZgt{}}
        \PYG{n+nt}{\PYGZlt{}cliente}\PYG{n+nt}{\PYGZgt{}}
                \PYG{n+nt}{\PYGZlt{}cif}\PYG{n+nt}{\PYGZgt{}}5121554\PYG{n+nt}{\PYGZlt{}/cif\PYGZgt{}}
                \PYG{n+nt}{\PYGZlt{}nombre}\PYG{n+nt}{\PYGZgt{}}Acer SL\PYG{n+nt}{\PYGZlt{}/nombre\PYGZgt{}}
        \PYG{n+nt}{\PYGZlt{}/cliente\PYGZgt{}}
\PYG{n+nt}{\PYGZlt{}/listaclientes\PYGZgt{}}
\end{sphinxVerbatim}

La DTD tiene esta estructura

\begin{sphinxVerbatim}[commandchars=\\\{\}]
\PYG{k}{\PYGZlt{}!DOCTYPE} \PYG{n+nt}{listaclientes} \PYG{k}{[}
                \PYG{k}{\PYGZlt{}!ELEMENT} \PYG{n+nt}{listaclientes} \PYG{o}{(}\PYG{n+nt}{cliente}\PYG{o}{+}\PYG{o}{)}\PYG{k}{\PYGZgt{}}
                \PYG{k}{\PYGZlt{}!ELEMENT} \PYG{n+nt}{cliente} \PYG{o}{(}\PYG{n+nt}{nombre}\PYG{o}{,} \PYG{n+nt}{cif}\PYG{o}{,} \PYG{n+nt}{diasentrega}\PYG{o}{?}\PYG{o}{)}\PYG{k}{\PYGZgt{}}
                \PYG{k}{\PYGZlt{}!ELEMENT} \PYG{n+nt}{nombre} \PYG{o}{(}\PYG{k+kc}{\PYGZsh{}PCDATA}\PYG{o}{)}\PYG{k}{\PYGZgt{}}
                \PYG{k}{\PYGZlt{}!ELEMENT} \PYG{n+nt}{cif} \PYG{o}{(}\PYG{k+kc}{\PYGZsh{}PCDATA}\PYG{o}{)}\PYG{k}{\PYGZgt{}}
                \PYG{k}{\PYGZlt{}!ELEMENT} \PYG{n+nt}{diasentrega} \PYG{o}{(}\PYG{k+kc}{\PYGZsh{}PCDATA}\PYG{o}{)}\PYG{k}{\PYGZgt{}}
                \PYG{k}{]}
        \PYG{k}{\PYGZgt{}}
\end{sphinxVerbatim}

Esto significa lo siguiente:
\begin{itemize}
\item {} 
Se establece el tipo de documento \sphinxcode{listaclientes} que consta de una serie de elementos (dentro del corchete)

\item {} 
Un elemento \sphinxcode{listaclientes   {}`{}` consta de uno o más clientes. El signo {}`{}`+} significa «uno o más».

\item {} 
Un cliente tiene un nombre y un cif. También puede tener un elemento \sphinxcode{diasentrega} que puede o no aparecer (el signo \sphinxcode{?} significa «0 o 1 veces»).

\item {} 
Un \sphinxcode{nombre} no tiene más elementos dentro, solo caracteres (\sphinxcode{\#PCDATA})

\item {} 
Un \sphinxcode{CIF} solo consta de caracteres.

\item {} 
Un elemento \sphinxcode{diasentrega} consta solo de caracteres.

\end{itemize}

La solución completa sería así:

\begin{sphinxVerbatim}[commandchars=\\\{\}]
\PYG{c+cp}{\PYGZlt{}?xml version=\PYGZdq{}1.0\PYGZdq{} encoding=\PYGZdq{}utf\PYGZhy{}8\PYGZdq{}?\PYGZgt{}}
\PYG{c+cp}{\PYGZlt{}!DOCTYPE listaclientes [}
\PYG{c+cp}{                \PYGZlt{}!ELEMENT listaclientes (cliente+)\PYGZgt{}}
                \PYG{c+cp}{\PYGZlt{}!ELEMENT cliente (nombre, cif, diasentrega?)\PYGZgt{}}
                \PYG{c+cp}{\PYGZlt{}!ELEMENT nombre (\PYGZsh{}PCDATA)\PYGZgt{}}
                \PYG{c+cp}{\PYGZlt{}!ELEMENT cif (\PYGZsh{}PCDATA)\PYGZgt{}}
                \PYG{c+cp}{\PYGZlt{}!ELEMENT diasentrega (\PYGZsh{}PCDATA)\PYGZgt{}}
        ]\PYGZgt{}
\PYG{n+nt}{\PYGZlt{}listaclientes}\PYG{n+nt}{\PYGZgt{}}
        \PYG{n+nt}{\PYGZlt{}cliente}\PYG{n+nt}{\PYGZgt{}}
                \PYG{n+nt}{\PYGZlt{}nombre}\PYG{n+nt}{\PYGZgt{}}Mercasa\PYG{n+nt}{\PYGZlt{}/nombre\PYGZgt{}}
                \PYG{n+nt}{\PYGZlt{}cif}\PYG{n+nt}{\PYGZgt{}}5676443\PYG{n+nt}{\PYGZlt{}/cif\PYGZgt{}}
        \PYG{n+nt}{\PYGZlt{}/cliente\PYGZgt{}}
        \PYG{n+nt}{\PYGZlt{}cliente}\PYG{n+nt}{\PYGZgt{}}
                \PYG{n+nt}{\PYGZlt{}nombre}\PYG{n+nt}{\PYGZgt{}}Acer SL\PYG{n+nt}{\PYGZlt{}/nombre\PYGZgt{}}
                \PYG{n+nt}{\PYGZlt{}cif}\PYG{n+nt}{\PYGZgt{}}5121554\PYG{n+nt}{\PYGZlt{}/cif\PYGZgt{}}
        \PYG{n+nt}{\PYGZlt{}/cliente\PYGZgt{}}
\PYG{n+nt}{\PYGZlt{}/listaclientes\PYGZgt{}}
\end{sphinxVerbatim}


\subsection{Ejemplo de DTD (productos)}
\label{\detokenize{tema5:ejemplo-de-dtd-productos}}
Se pide un conjunto de reglas en forma de DTD para definir qué se permitirá en los archivos XML de datos de una empresa de fabricación:
\begin{itemize}
\item {} 
La raíz es \textless{}productos\textgreater{}

\item {} 
Dentro de productos puede haber \textless{}raton\textgreater{} o \textless{}teclado\textgreater{} que pueden repetirse e ir en cualquier orden (RRTT, T, TR, TTRR)

\item {} 
Todo \textless{}raton\textgreater{} tiene siempre un \textless{}codigo\textgreater{} y puede o no tener una \textless{}descripción\textgreater{}.

\item {} 
Todo \textless{}teclado\textgreater{} tiene siempre un \textless{}codigo\textgreater{}, debe llevar siempre una \textless{}descripcion\textgreater{} y puede o no tener un \textless{}peso\textgreater{}

\end{itemize}

Elaborar la DTD que formaliza estas reglas.

\begin{sphinxVerbatim}[commandchars=\\\{\}]
\PYG{k}{\PYGZlt{}!ELEMENT} \PYG{n+nt}{productos}   \PYG{o}{(}\PYG{n+nt}{raton}\PYG{o}{\textbar{}}\PYG{n+nt}{teclado}\PYG{o}{)}\PYG{o}{*} \PYG{k}{\PYGZgt{}}
\PYG{k}{\PYGZlt{}!ELEMENT} \PYG{n+nt}{raton}       \PYG{o}{(}\PYG{n+nt}{codigo}\PYG{o}{,} \PYG{n+nt}{descripcion}\PYG{o}{?}\PYG{o}{)} \PYG{k}{\PYGZgt{}}
\PYG{k}{\PYGZlt{}!ELEMENT} \PYG{n+nt}{codigo}      \PYG{o}{(}\PYG{k+kc}{\PYGZsh{}PCDATA}\PYG{o}{)}\PYG{k}{\PYGZgt{}}
\PYG{k}{\PYGZlt{}!ELEMENT} \PYG{n+nt}{descripcion} \PYG{o}{(}\PYG{k+kc}{\PYGZsh{}PCDATA}\PYG{o}{)}\PYG{k}{\PYGZgt{}}
\PYG{k}{\PYGZlt{}!ELEMENT} \PYG{n+nt}{teclado}     \PYG{o}{(}\PYG{n+nt}{codigo}\PYG{o}{,}\PYG{n+nt}{descripcion}\PYG{o}{,}\PYG{n+nt}{peso}\PYG{o}{?}\PYG{o}{)}\PYG{k}{\PYGZgt{}}
\PYG{k}{\PYGZlt{}!ELEMENT} \PYG{n+nt}{peso}        \PYG{o}{(}\PYG{k+kc}{\PYGZsh{}PCDATA}\PYG{o}{)}\PYG{k}{\PYGZgt{}}
\end{sphinxVerbatim}

El siguiente fichero debe validarse correctamente:

\begin{sphinxVerbatim}[commandchars=\\\{\}]
\PYG{n+nt}{\PYGZlt{}productos}\PYG{n+nt}{\PYGZgt{}}
\PYG{n+nt}{\PYGZlt{}/productos\PYGZgt{}}
\end{sphinxVerbatim}

Y el siguiente también

\begin{sphinxVerbatim}[commandchars=\\\{\}]
\PYG{n+nt}{\PYGZlt{}productos}\PYG{n+nt}{\PYGZgt{}}
    \PYG{n+nt}{\PYGZlt{}teclado}\PYG{n+nt}{\PYGZgt{}}
        \PYG{n+nt}{\PYGZlt{}codigo}\PYG{n+nt}{\PYGZgt{}}T1\PYG{n+nt}{\PYGZlt{}/codigo\PYGZgt{}}
        \PYG{n+nt}{\PYGZlt{}descripcion}\PYG{n+nt}{\PYGZgt{}}Teclado inalamb.\PYG{n+nt}{\PYGZlt{}/descripcion\PYGZgt{}}
    \PYG{n+nt}{\PYGZlt{}/teclado\PYGZgt{}}
\PYG{n+nt}{\PYGZlt{}/productos\PYGZgt{}}
\end{sphinxVerbatim}

Y este también (a pesar del flagrante error en el peso)

\begin{sphinxVerbatim}[commandchars=\\\{\}]
\PYG{n+nt}{\PYGZlt{}productos}\PYG{n+nt}{\PYGZgt{}}
    \PYG{n+nt}{\PYGZlt{}raton}\PYG{n+nt}{\PYGZgt{}}
        \PYG{n+nt}{\PYGZlt{}codigo}\PYG{n+nt}{\PYGZgt{}}R1\PYG{n+nt}{\PYGZlt{}/codigo\PYGZgt{}}
    \PYG{n+nt}{\PYGZlt{}/raton\PYGZgt{}}
    \PYG{n+nt}{\PYGZlt{}teclado}\PYG{n+nt}{\PYGZgt{}}
        \PYG{n+nt}{\PYGZlt{}codigo}\PYG{n+nt}{\PYGZgt{}}T1\PYG{n+nt}{\PYGZlt{}/codigo\PYGZgt{}}
        \PYG{n+nt}{\PYGZlt{}descripcion}\PYG{n+nt}{\PYGZgt{}}Teclado inalamb.\PYG{n+nt}{\PYGZlt{}/descripcion\PYGZgt{}}
        \PYG{n+nt}{\PYGZlt{}peso}\PYG{n+nt}{\PYGZgt{}}\textbar{}@¬\textbar{}@\PYGZti{}\textbar{}\textbar{}@\PYGZti{}\PYG{n+nt}{\PYGZlt{}/peso\PYGZgt{}}
    \PYG{n+nt}{\PYGZlt{}/teclado\PYGZgt{}}
\PYG{n+nt}{\PYGZlt{}/productos\PYGZgt{}}
\end{sphinxVerbatim}


\section{Ejercicio I (DTD)}
\label{\detokenize{tema5:ejercicio-i-dtd}}
Unos programadores necesitan un formato de fichero para que sus distintos programas intercambien información sobre ventas. El acuerdo al que han llegado es que su XML debería tener esta estructura:
\begin{itemize}
\item {} 
El elemento raíz será \sphinxcode{\textless{}listaventas\textgreater{}}

\item {} 
Toda \sphinxcode{\textless{}listaventas\textgreater{}} tiene una o más ventas.

\item {} 
Toda \sphinxcode{\textless{}venta\textgreater{}} tiene los siguientes datos:
\begin{itemize}
\item {} 
Importe.

\item {} 
Comprador.

\item {} 
Vendedor.

\item {} 
Fecha (optativa).

\item {} 
Un codigo de factura.

\end{itemize}

\end{itemize}

\begin{sphinxVerbatim}[commandchars=\\\{\}]
\PYG{c+cp}{\PYGZlt{}?xml version=\PYGZdq{}1.0\PYGZdq{} encoding=\PYGZdq{}UTF\PYGZhy{}8\PYGZdq{}?\PYGZgt{}}
\PYG{c+cp}{\PYGZlt{}!DOCTYPE listaventas[}
\PYG{c+cp}{  \PYGZlt{}!ELEMENT listaventas (venta+)\PYGZgt{}}
  \PYG{c+cp}{\PYGZlt{}!ELEMENT venta (importe, comprador, vendedor, fecha?, codigofactura)\PYGZgt{}}
  \PYG{c+cp}{\PYGZlt{}!ELEMENT importe (\PYGZsh{}PCDATA)\PYGZgt{}}
  \PYG{c+cp}{\PYGZlt{}!ELEMENT comprador (\PYGZsh{}PCDATA)\PYGZgt{}}
  \PYG{c+cp}{\PYGZlt{}!ELEMENT vendedor (\PYGZsh{}PCDATA)\PYGZgt{}}
  \PYG{c+cp}{\PYGZlt{}!ELEMENT fecha (\PYGZsh{}PCDATA)\PYGZgt{}}
  \PYG{c+cp}{\PYGZlt{}!ELEMENT codigofactura (\PYGZsh{}PCDATA)\PYGZgt{}}

]\PYGZgt{}

\PYG{n+nt}{\PYGZlt{}listaventas}\PYG{n+nt}{\PYGZgt{}}
  \PYG{n+nt}{\PYGZlt{}venta}\PYG{n+nt}{\PYGZgt{}}
        \PYG{n+nt}{\PYGZlt{}importe}\PYG{n+nt}{\PYGZgt{}}1500\PYG{n+nt}{\PYGZlt{}/importe\PYGZgt{}}
        \PYG{n+nt}{\PYGZlt{}comprador}\PYG{n+nt}{\PYGZgt{}}Wile E.Coyote\PYG{n+nt}{\PYGZlt{}/comprador\PYGZgt{}}
        \PYG{n+nt}{\PYGZlt{}vendedor}\PYG{n+nt}{\PYGZgt{}}ACME\PYG{n+nt}{\PYGZlt{}/vendedor\PYGZgt{}}
        \PYG{n+nt}{\PYGZlt{}codigofactura}\PYG{n+nt}{\PYGZgt{}}E17\PYG{n+nt}{\PYGZlt{}/codigofactura\PYGZgt{}}
  \PYG{n+nt}{\PYGZlt{}/venta\PYGZgt{}}
  \PYG{n+nt}{\PYGZlt{}venta}\PYG{n+nt}{\PYGZgt{}}
        \PYG{n+nt}{\PYGZlt{}importe}\PYG{n+nt}{\PYGZgt{}}750\PYG{n+nt}{\PYGZlt{}/importe\PYGZgt{}}
        \PYG{n+nt}{\PYGZlt{}comprador}\PYG{n+nt}{\PYGZgt{}}Elmer Fudd\PYG{n+nt}{\PYGZlt{}/comprador\PYGZgt{}}
        \PYG{n+nt}{\PYGZlt{}vendedor}\PYG{n+nt}{\PYGZgt{}}ACME\PYG{n+nt}{\PYGZlt{}/vendedor\PYGZgt{}}
        \PYG{n+nt}{\PYGZlt{}fecha}\PYG{n+nt}{\PYGZgt{}}27\PYGZhy{}2\PYGZhy{}2015\PYG{n+nt}{\PYGZlt{}/fecha\PYGZgt{}}
        \PYG{n+nt}{\PYGZlt{}codigofactura}\PYG{n+nt}{\PYGZgt{}}E18\PYG{n+nt}{\PYGZlt{}/codigofactura\PYGZgt{}}
  \PYG{n+nt}{\PYGZlt{}/venta\PYGZgt{}}
\PYG{n+nt}{\PYGZlt{}/listaventas\PYGZgt{}}
\end{sphinxVerbatim}


\section{Ejercicio II (DTD)}
\label{\detokenize{tema5:ejercicio-ii-dtd}}
Crear un XML de ejemplo y la DTD asociada para unos programadores que programan una aplicación de pedidos donde hay una lista de pedidos con 0 o más pedidos. Cada pedido tiene un número de serie, una cantidad y un peso que puede ser opcional.


\subsection{Solución}
\label{\detokenize{tema5:solucion}}
Este ejemplo es un documento XML válido.

\begin{sphinxVerbatim}[commandchars=\\\{\}]
\PYG{c+cp}{\PYGZlt{}?xml version=\PYGZdq{}1.0\PYGZdq{} encoding=\PYGZdq{}utf\PYGZhy{}8\PYGZdq{}?\PYGZgt{}}

\PYG{c+cp}{\PYGZlt{}!DOCTYPE listapedidos [}
\PYG{c+cp}{        \PYGZlt{}!ELEMENT listapedidos (pedido*)\PYGZgt{}}
        \PYG{c+cp}{\PYGZlt{}!ELEMENT pedido (numeroserie, cantidad, peso?)\PYGZgt{}}
        \PYG{c+cp}{\PYGZlt{}!ELEMENT numeroserie (\PYGZsh{}PCDATA)\PYGZgt{}}
        \PYG{c+cp}{\PYGZlt{}!ELEMENT cantidad (\PYGZsh{}PCDATA)\PYGZgt{}}
        \PYG{c+cp}{\PYGZlt{}!ELEMENT peso (\PYGZsh{}PCDATA)\PYGZgt{}}
]\PYGZgt{}

\PYG{n+nt}{\PYGZlt{}listapedidos}\PYG{n+nt}{\PYGZgt{}}
\PYG{n+nt}{\PYGZlt{}/listapedidos\PYGZgt{}}
\end{sphinxVerbatim}

Este documento \sphinxstylestrong{no es válido}

\begin{sphinxVerbatim}[commandchars=\\\{\}]
\PYG{c+cp}{\PYGZlt{}?xml version=\PYGZdq{}1.0\PYGZdq{} encoding=\PYGZdq{}utf\PYGZhy{}8\PYGZdq{}?\PYGZgt{}}

\PYG{c+cp}{\PYGZlt{}!DOCTYPE listapedidos [}
\PYG{c+cp}{        \PYGZlt{}!ELEMENT listapedidos (pedido*)\PYGZgt{}}
        \PYG{c+cp}{\PYGZlt{}!ELEMENT pedido (numeroserie, cantidad, peso?)\PYGZgt{}}
        \PYG{c+cp}{\PYGZlt{}!ELEMENT numeroserie (\PYGZsh{}PCDATA)\PYGZgt{}}
        \PYG{c+cp}{\PYGZlt{}!ELEMENT cantidad (\PYGZsh{}PCDATA)\PYGZgt{}}
        \PYG{c+cp}{\PYGZlt{}!ELEMENT peso (\PYGZsh{}PCDATA)\PYGZgt{}}
]\PYGZgt{}

\PYG{n+nt}{\PYGZlt{}listapedidos}\PYG{n+nt}{\PYGZgt{}}
        \PYG{n+nt}{\PYGZlt{}pedido}\PYG{n+nt}{\PYGZgt{}}
                \PYG{n+nt}{\PYGZlt{}numeroserie}\PYG{n+nt}{\PYGZgt{}}23332244\PYG{n+nt}{\PYGZlt{}/numeroserie\PYGZgt{}}
        \PYG{n+nt}{\PYGZlt{}/pedido\PYGZgt{}}
\PYG{n+nt}{\PYGZlt{}/listapedidos\PYGZgt{}}
\end{sphinxVerbatim}

Este documento \sphinxstylestrong{sí es válido}. Las DTD solo se ocupan de determinar qué elementos hay y en qué orden, pero no se ocupan de lo que hay dentro de los elementos.

\begin{sphinxVerbatim}[commandchars=\\\{\}]
\PYG{c+cp}{\PYGZlt{}?xml version=\PYGZdq{}1.0\PYGZdq{} encoding=\PYGZdq{}utf\PYGZhy{}8\PYGZdq{}?\PYGZgt{}}

\PYG{c+cp}{\PYGZlt{}!DOCTYPE listapedidos [}
\PYG{c+cp}{        \PYGZlt{}!ELEMENT listapedidos (pedido*)\PYGZgt{}}
        \PYG{c+cp}{\PYGZlt{}!ELEMENT pedido (numeroserie, cantidad, peso?)\PYGZgt{}}
        \PYG{c+cp}{\PYGZlt{}!ELEMENT numeroserie (\PYGZsh{}PCDATA)\PYGZgt{}}
        \PYG{c+cp}{\PYGZlt{}!ELEMENT cantidad (\PYGZsh{}PCDATA)\PYGZgt{}}
        \PYG{c+cp}{\PYGZlt{}!ELEMENT peso (\PYGZsh{}PCDATA)\PYGZgt{}}
]\PYGZgt{}

\PYG{n+nt}{\PYGZlt{}listapedidos}\PYG{n+nt}{\PYGZgt{}}
        \PYG{n+nt}{\PYGZlt{}pedido}\PYG{n+nt}{\PYGZgt{}}
                \PYG{n+nt}{\PYGZlt{}numeroserie}\PYG{n+nt}{\PYGZgt{}}23332244\PYG{n+nt}{\PYGZlt{}/numeroserie\PYGZgt{}}
                \PYG{n+nt}{\PYGZlt{}cantidad}\PYG{n+nt}{\PYGZgt{}}ññlñ\PYG{n+nt}{\PYGZlt{}/cantidad\PYGZgt{}}
        \PYG{n+nt}{\PYGZlt{}/pedido\PYGZgt{}}
\PYG{n+nt}{\PYGZlt{}/listapedidos\PYGZgt{}}
\end{sphinxVerbatim}


\section{Ejercicio (con atributos)}
\label{\detokenize{tema5:ejercicio-con-atributos}}
Unos programadores necesitan estructurar la información que intercambiarán los ficheros de sus aplicaciones para lo cual han determinado los requisitos siguientes:
\begin{itemize}
\item {} 
Los ficheros deben tener un elemento \sphinxcode{\textless{}listafacturas\textgreater{}}

\item {} 
Dentro de la lista debe haber una o más facturas.

\item {} 
Las facturas tienen un atributo \sphinxcode{fecha} que es optativo.

\item {} 
Toda factura tiene un \sphinxcode{emisor}, que es un elemento obligatorio y que debe tener un atributo \sphinxcode{cif} que es obligatorio. Dentro de \sphinxcode{emisor} debe haber un elemento \sphinxcode{nombre}, que es obligatorio y puede o no haber un elemento \sphinxcode{volumenventas}.

\item {} 
Toda factura debe tener un elemento \sphinxcode{pagador}, el cual tiene exactamente la misma estructura que \sphinxcode{emisor}.

\item {} 
Toda factura tiene un elemento \sphinxcode{importe}.

\end{itemize}


\subsection{Solución ejercicio con atributos}
\label{\detokenize{tema5:solucion-ejercicio-con-atributos}}
La siguiente DTD refleja los requisitos indicados en el enunciado.

\begin{sphinxVerbatim}[commandchars=\\\{\}]
\PYG{k}{\PYGZlt{}!ELEMENT} \PYG{n+nt}{listafacturas} \PYG{o}{(}\PYG{n+nt}{factura}\PYG{o}{+}\PYG{o}{)}\PYG{k}{\PYGZgt{}}
\PYG{k}{\PYGZlt{}!ELEMENT} \PYG{n+nt}{factura} \PYG{o}{(}\PYG{n+nt}{emisor}\PYG{o}{,} \PYG{n+nt}{pagador}\PYG{o}{,} \PYG{n+nt}{importe}\PYG{o}{)}\PYG{k}{\PYGZgt{}}
\PYG{k}{\PYGZlt{}!ATTLIST} \PYG{n+nt}{factura} \PYG{n+na}{fecha} \PYG{k+kc}{CDATA} \PYG{k+kc}{\PYGZsh{}IMPLIED}\PYG{k}{\PYGZgt{}}
\PYG{k}{\PYGZlt{}!ELEMENT} \PYG{n+nt}{emisor} \PYG{o}{(}\PYG{n+nt}{nombre}\PYG{o}{,} \PYG{n+nt}{volumenventas}\PYG{o}{?}\PYG{o}{)}\PYG{k}{\PYGZgt{}}
\PYG{k}{\PYGZlt{}!ELEMENT} \PYG{n+nt}{nombre} \PYG{o}{(}\PYG{k+kc}{\PYGZsh{}PCDATA}\PYG{o}{)}\PYG{k}{\PYGZgt{}}
\PYG{k}{\PYGZlt{}!ATTLIST} \PYG{n+nt}{emisor} \PYG{n+na}{cif} \PYG{k+kc}{CDATA} \PYG{k+kc}{\PYGZsh{}REQUIRED}\PYG{k}{\PYGZgt{}}
\PYG{k}{\PYGZlt{}!ELEMENT} \PYG{n+nt}{volumenventas} \PYG{o}{(}\PYG{k+kc}{\PYGZsh{}PCDATA}\PYG{o}{)}\PYG{k}{\PYGZgt{}}
\PYG{k}{\PYGZlt{}!ELEMENT} \PYG{n+nt}{pagador} \PYG{o}{(}\PYG{n+nt}{nombre}\PYG{o}{,} \PYG{n+nt}{volumenventas}\PYG{o}{?}\PYG{o}{)}\PYG{k}{\PYGZgt{}}
\PYG{k}{\PYGZlt{}!ATTLIST} \PYG{n+nt}{pagador} \PYG{n+na}{cif} \PYG{k+kc}{CDATA} \PYG{k+kc}{\PYGZsh{}REQUIRED}\PYG{k}{\PYGZgt{}}
\PYG{k}{\PYGZlt{}!ELEMENT} \PYG{n+nt}{importe} \PYG{o}{(}\PYG{k+kc}{\PYGZsh{}PCDATA}\PYG{o}{)}\PYG{k}{\PYGZgt{}}
\end{sphinxVerbatim}

Y el XML siguiente refleja un posible documento. Puede comprobarse que es válido con respecto a la DTD.

\begin{sphinxVerbatim}[commandchars=\\\{\}]
\PYG{c+cp}{\PYGZlt{}?xml version=\PYGZdq{}1.0\PYGZdq{} encoding=\PYGZdq{}UTF\PYGZhy{}8\PYGZdq{}?\PYGZgt{}}
\PYG{c+cp}{\PYGZlt{}!DOCTYPE listafacturas SYSTEM \PYGZdq{}ListaFacturas.dtd\PYGZdq{}\PYGZgt{}}
\PYG{n+nt}{\PYGZlt{}listafacturas}\PYG{n+nt}{\PYGZgt{}}
  \PYG{n+nt}{\PYGZlt{}factura} \PYG{n+na}{fecha=}\PYG{l+s}{\PYGZdq{}11\PYGZhy{}2\PYGZhy{}2015\PYGZdq{}}\PYG{n+nt}{\PYGZgt{}}
        \PYG{n+nt}{\PYGZlt{}emisor} \PYG{n+na}{cif=}\PYG{l+s}{\PYGZdq{}123\PYGZdq{}}\PYG{n+nt}{\PYGZgt{}}
          \PYG{n+nt}{\PYGZlt{}nombre}\PYG{n+nt}{\PYGZgt{}}ACME\PYG{n+nt}{\PYGZlt{}/nombre\PYGZgt{}}
        \PYG{n+nt}{\PYGZlt{}/emisor\PYGZgt{}}
        \PYG{n+nt}{\PYGZlt{}pagador} \PYG{n+na}{cif=}\PYG{l+s}{\PYGZdq{}234\PYGZdq{}}\PYG{n+nt}{\PYGZgt{}}
          \PYG{n+nt}{\PYGZlt{}nombre}\PYG{n+nt}{\PYGZgt{}}ACME Inc\PYG{n+nt}{\PYGZlt{}/nombre\PYGZgt{}}
          \PYG{n+nt}{\PYGZlt{}volumenventas}\PYG{n+nt}{\PYGZgt{}}2000\PYG{n+nt}{\PYGZlt{}/volumenventas\PYGZgt{}}
        \PYG{n+nt}{\PYGZlt{}/pagador\PYGZgt{}}
        \PYG{n+nt}{\PYGZlt{}importe}\PYG{n+nt}{\PYGZgt{}}2500\PYG{n+nt}{\PYGZlt{}/importe\PYGZgt{}}
  \PYG{n+nt}{\PYGZlt{}/factura\PYGZgt{}}
\PYG{n+nt}{\PYGZlt{}/listafacturas\PYGZgt{}}
\end{sphinxVerbatim}


\section{Ejercicio}
\label{\detokenize{tema5:ejercicio}}
Un instituto necesita registrar los cursos y alumnos que estudian en él y necesita una DTD para comprobar los documentos XML de los programas que utiliza:
\begin{itemize}
\item {} 
Tiene que haber un elemento raíz \sphinxcode{listacursos}. Tiene que haber uno o más cursos.

\item {} 
Un curso tiene uno o más alumnos

\item {} 
Todo alumno tiene un DNI, un nombre y un apellido, puede que tenga segundo apellido o no.

\item {} 
Un alumno escoge una lista de asignaturas donde habrá una o más asignaturas. Toda asignatura tiene un nombre, un atributo código y un profesor.

\item {} 
Un profesor tiene un NRP (Número de Registro Personal), un nombre y un apellido (también puede tener o no un segundo apellido).

\end{itemize}


\subsection{Solución completa}
\label{\detokenize{tema5:solucion-completa}}
\begin{sphinxVerbatim}[commandchars=\\\{\}]
\PYG{c+cp}{\PYGZlt{}!ELEMENT listacursos (curso)+\PYGZgt{}}
\PYG{c+cp}{\PYGZlt{}!ELEMENT curso (alumno)+\PYGZgt{}}
\PYG{c+cp}{\PYGZlt{}!ELEMENT alumno (dni, nombre,}
\PYG{c+cp}{                    ap1, ap2?, asignatura+)\PYGZgt{}}

\PYG{c+cp}{\PYGZlt{}!ELEMENT asignatura (nombre, profesor)\PYGZgt{}}
\PYG{c+cp}{\PYGZlt{}!ATTLIST asignatura codigo CDATA \PYGZsh{}REQUIRED\PYGZgt{}}

\PYG{c+cp}{\PYGZlt{}!ELEMENT profesor (nrp, nombre, ap1, ap2?)\PYGZgt{}}

\PYG{c+cp}{\PYGZlt{}!ELEMENT dni    (\PYGZsh{}PCDATA)\PYGZgt{}}
\PYG{c+cp}{\PYGZlt{}!ELEMENT nombre (\PYGZsh{}PCDATA)\PYGZgt{}}
\PYG{c+cp}{\PYGZlt{}!ELEMENT ap1    (\PYGZsh{}PCDATA)\PYGZgt{}}
\PYG{c+cp}{\PYGZlt{}!ELEMENT ap2    (\PYGZsh{}PCDATA)\PYGZgt{}}
\PYG{c+cp}{\PYGZlt{}!ELEMENT nrp    (\PYGZsh{}PCDATA)\PYGZgt{}}
\end{sphinxVerbatim}

Un ejemplo de fichero válido:

\begin{sphinxVerbatim}[commandchars=\\\{\}]
\PYG{n+nt}{\PYGZlt{}listacursos}\PYG{n+nt}{\PYGZgt{}}
    \PYG{n+nt}{\PYGZlt{}curso}\PYG{n+nt}{\PYGZgt{}}
        \PYG{n+nt}{\PYGZlt{}alumno}\PYG{n+nt}{\PYGZgt{}}
            \PYG{n+nt}{\PYGZlt{}dni}\PYG{n+nt}{\PYGZgt{}}44e\PYG{n+nt}{\PYGZlt{}/dni\PYGZgt{}}
            \PYG{n+nt}{\PYGZlt{}nombre}\PYG{n+nt}{\PYGZgt{}}Juan\PYG{n+nt}{\PYGZlt{}/nombre\PYGZgt{}}
            \PYG{n+nt}{\PYGZlt{}ap1}\PYG{n+nt}{\PYGZgt{}}Sanchez\PYG{n+nt}{\PYGZlt{}/ap1\PYGZgt{}}
            \PYG{n+nt}{\PYGZlt{}asignatura} \PYG{n+na}{codigo=}\PYG{l+s}{\PYGZdq{}LM1\PYGZdq{}}\PYG{n+nt}{\PYGZgt{}}
                \PYG{n+nt}{\PYGZlt{}nombre}\PYG{n+nt}{\PYGZgt{}}Leng marcas\PYG{n+nt}{\PYGZlt{}/nombre\PYGZgt{}}
                \PYG{n+nt}{\PYGZlt{}profesor}\PYG{n+nt}{\PYGZgt{}}
                    \PYG{n+nt}{\PYGZlt{}nrp}\PYG{n+nt}{\PYGZgt{}}8\PYG{n+nt}{\PYGZlt{}/nrp\PYGZgt{}}
                    \PYG{n+nt}{\PYGZlt{}nombre}\PYG{n+nt}{\PYGZgt{}}Oscar\PYG{n+nt}{\PYGZlt{}/nombre\PYGZgt{}}
                    \PYG{n+nt}{\PYGZlt{}ap1}\PYG{n+nt}{\PYGZgt{}}Gomez\PYG{n+nt}{\PYGZlt{}/ap1\PYGZgt{}}
                \PYG{n+nt}{\PYGZlt{}/profesor\PYGZgt{}}
            \PYG{n+nt}{\PYGZlt{}/asignatura\PYGZgt{}}
        \PYG{n+nt}{\PYGZlt{}/alumno\PYGZgt{}}
    \PYG{n+nt}{\PYGZlt{}/curso\PYGZgt{}}
\PYG{n+nt}{\PYGZlt{}/listacursos\PYGZgt{}}
\end{sphinxVerbatim}


\section{Otras características de XML}
\label{\detokenize{tema5:otras-caracteristicas-de-xml}}

\subsection{Atributos}
\label{\detokenize{tema5:atributos}}
Un atributo XML funciona exactamente igual que un atributo HTML, en concreto un atributo es un trozo de información que acompaña a la etiqueta, en lugar de ir dentro del elemento.

\begin{sphinxVerbatim}[commandchars=\\\{\}]
\PYG{n+nt}{\PYGZlt{}pedido} \PYG{n+na}{codigo=}\PYG{l+s}{\PYGZdq{}20C\PYGZdq{}}\PYG{n+nt}{\PYGZgt{}}
        \PYG{n+nt}{\PYGZlt{}contenido}\PYG{n+nt}{\PYGZgt{}}
                ...
\PYG{n+nt}{\PYGZlt{}/pedido\PYGZgt{}}
\end{sphinxVerbatim}

En este caso, la etiqueta \sphinxcode{pedido} tiene un atributo \sphinxcode{codigo}.

¿Cuando debemos usar atributos y cuando debemos usar elementos? Resulta que el ejemplo anterior también se podría haber permitido hacerlo así:

\begin{sphinxVerbatim}[commandchars=\\\{\}]
\PYG{n+nt}{\PYGZlt{}pedido}\PYG{n+nt}{\PYGZgt{}}
        \PYG{n+nt}{\PYGZlt{}codigo}\PYG{n+nt}{\PYGZgt{}}20C\PYG{n+nt}{\PYGZlt{}/codigo\PYGZgt{}}
        \PYG{n+nt}{\PYGZlt{}contenido}\PYG{n+nt}{\PYGZgt{}}
                ...
\PYG{n+nt}{\PYGZlt{}/pedido\PYGZgt{}}
\end{sphinxVerbatim}

Hay muchas discusiones sobre qué meter dentro de elemento o atributo. Sin embargo, los expertos coinciden en señalar que en caso de duda es mejor el segundo.

La definición de atributos se hace por medio de una directiva llamada \sphinxcode{ATTLIST}. En concreto si quisieramos permitir un atributo \sphinxcode{código} en el elemento \sphinxcode{pedido} se haría algo así.

\begin{sphinxVerbatim}[commandchars=\\\{\}]
\PYG{c+cp}{\PYGZlt{}?xml version=\PYGZdq{}1.0\PYGZdq{} encoding=\PYGZdq{}utf\PYGZhy{}8\PYGZdq{}?\PYGZgt{}}
\PYG{c+cp}{\PYGZlt{}!DOCTYPE pedido[}
\PYG{c+cp}{        \PYGZlt{}!ELEMENT pedido (contenido)\PYGZgt{}}
        \PYG{c+cp}{\PYGZlt{}!ELEMENT contenido (\PYGZsh{}PCDATA)\PYGZgt{}}
        \PYG{c+cp}{\PYGZlt{}!ATTLIST pedido codigo CDATA \PYGZsh{}REQUIRED\PYGZgt{}}
]\PYGZgt{}

\PYG{n+nt}{\PYGZlt{}pedido} \PYG{n+na}{codigo=}\PYG{l+s}{\PYGZdq{}20C\PYGZdq{}}\PYG{n+nt}{\PYGZgt{}}
        \PYG{n+nt}{\PYGZlt{}contenido}\PYG{n+nt}{\PYGZgt{}}Pedido de cosas\PYG{n+nt}{\PYGZlt{}/contenido\PYGZgt{}}
\PYG{n+nt}{\PYGZlt{}/pedido\PYGZgt{}}
\end{sphinxVerbatim}

En concreto este código pone que el elemento \sphinxcode{pedido} tiene un atributo \sphinxcode{código} con datos carácter dentro y que es obligatorio que esté presente (un atributo optativo en vez de \sphinxcode{\#REQUIRED} usará \sphinxcode{\#IMPLIED})

Si probamos esto, también validará porque el atributo es \sphinxstyleemphasis{optativo}

\begin{sphinxVerbatim}[commandchars=\\\{\}]
\PYG{c+cp}{\PYGZlt{}?xml version=\PYGZdq{}1.0\PYGZdq{} encoding=\PYGZdq{}utf\PYGZhy{}8\PYGZdq{}?\PYGZgt{}}
\PYG{c+cp}{\PYGZlt{}!DOCTYPE pedido[}
\PYG{c+cp}{        \PYGZlt{}!ELEMENT pedido (contenido)\PYGZgt{}}
        \PYG{c+cp}{\PYGZlt{}!ELEMENT contenido (\PYGZsh{}PCDATA)\PYGZgt{}}
        \PYG{c+cp}{\PYGZlt{}!ATTLIST pedido codigo CDATA \PYGZsh{}IMPLIED\PYGZgt{}}
]\PYGZgt{}

\PYG{n+nt}{\PYGZlt{}pedido}\PYG{n+nt}{\PYGZgt{}}
        \PYG{n+nt}{\PYGZlt{}contenido}\PYG{n+nt}{\PYGZgt{}}Pedido de cosas\PYG{n+nt}{\PYGZlt{}/contenido\PYGZgt{}}
\PYG{n+nt}{\PYGZlt{}/pedido\PYGZgt{}}
\end{sphinxVerbatim}


\subsection{Elementos vacíos}
\label{\detokenize{tema5:elementos-vacios}}
En ocasiones, un elemento en especial puede interesarnos que vaya vacío porque simplemente no contiene mucha información de relevancia. Por ejemplo en HTML podemos encontrarnos esto:

\begin{sphinxVerbatim}[commandchars=\\\{\}]
\PYG{p}{\PYGZlt{}}\PYG{n+nt}{b}\PYG{p}{\PYGZgt{}}Texto texto...\PYG{p}{\PYGZlt{}}\PYG{p}{/}\PYG{n+nt}{b}\PYG{p}{\PYGZgt{}}
\PYG{p}{\PYGZlt{}}\PYG{n+nt}{br}\PYG{p}{/}\PYG{p}{\PYGZgt{}}
\end{sphinxVerbatim}

Los elementos vacíos suelen utilizar para indicar pequeñas informaciones que no deseamos meter en atributos y que de todas formas tampoco son de demasiada relevancia.

Un elemento vacío se indica poniendo \sphinxcode{EMPTY} en lugar de \sphinxcode{\#PCDATA}

Por supuesto, estas dos formas de usar un atributo son válidas:

\begin{sphinxVerbatim}[commandchars=\\\{\}]
\PYG{n+nt}{\PYGZlt{}pedido}\PYG{n+nt}{\PYGZgt{}}
        \PYG{n+nt}{\PYGZlt{}pagado}\PYG{n+nt}{\PYGZgt{}}\PYG{n+nt}{\PYGZlt{}/pagado\PYGZgt{}}
        \PYG{n+nt}{\PYGZlt{}contenido}\PYG{n+nt}{\PYGZgt{}}...\PYG{n+nt}{\PYGZlt{}/contenido\PYGZgt{}}
\PYG{n+nt}{\PYGZlt{}/pedido\PYGZgt{}}
\end{sphinxVerbatim}

\begin{sphinxVerbatim}[commandchars=\\\{\}]
\PYG{n+nt}{\PYGZlt{}pedido}\PYG{n+nt}{\PYGZgt{}}
        \PYG{n+nt}{\PYGZlt{}pagado}\PYG{n+nt}{/\PYGZgt{}}
        \PYG{n+nt}{\PYGZlt{}contenido}\PYG{n+nt}{\PYGZgt{}}...\PYG{n+nt}{\PYGZlt{}/contenido\PYGZgt{}}
\PYG{n+nt}{\PYGZlt{}/pedido\PYGZgt{}}
\end{sphinxVerbatim}

La definición completa sería así:

\begin{sphinxVerbatim}[commandchars=\\\{\}]
\PYG{c+cp}{\PYGZlt{}?xml version=\PYGZdq{}1.0\PYGZdq{} encoding=\PYGZdq{}utf\PYGZhy{}8\PYGZdq{}?\PYGZgt{}}
\PYG{c+cp}{\PYGZlt{}!DOCTYPE pedido[}
\PYG{c+cp}{        \PYGZlt{}!ELEMENT pedido (pagado?,contenido)\PYGZgt{}}
        \PYG{c+cp}{\PYGZlt{}!ELEMENT pagado EMPTY\PYGZgt{}}
        \PYG{c+cp}{\PYGZlt{}!ELEMENT contenido (\PYGZsh{}PCDATA)\PYGZgt{}}
        \PYG{c+cp}{\PYGZlt{}!ATTLIST pedido codigo CDATA \PYGZsh{}IMPLIED\PYGZgt{}}
]\PYGZgt{}

\PYG{n+nt}{\PYGZlt{}pedido}\PYG{n+nt}{\PYGZgt{}}
        \PYG{n+nt}{\PYGZlt{}pagado}\PYG{n+nt}{/\PYGZgt{}}
        \PYG{n+nt}{\PYGZlt{}contenido}\PYG{n+nt}{\PYGZgt{}}Pedido de cosas\PYG{n+nt}{\PYGZlt{}/contenido\PYGZgt{}}
\PYG{n+nt}{\PYGZlt{}/pedido\PYGZgt{}}
\end{sphinxVerbatim}


\subsection{Alternativas}
\label{\detokenize{tema5:alternativas}}
Hasta ahora hemos indicado elementos donde un elemento puede aparecer o puede no aparecer, pero ¿qué ocurre si deseamos obligar a que aparezca una posibilidad entre varias?

Por ejemplo, supongamos que en un nuestro ejemplo de pedidos deseamos indicar si el pedido se entregó en almacén o a domicilio. A la fuerza todo pedido se entrega de alguna manera, sin embargo queremos exigir que en los XML aparezca una de esas dos alternativas. Los elementos alternativos se indican con la barra vertical \sphinxcode{almacen\textbar{}domicilio}

Una tentación sería hacer esto (que está \sphinxstylestrong{mal}):

\begin{sphinxVerbatim}[commandchars=\\\{\}]
\PYG{c+cp}{\PYGZlt{}!DOCTYPE pedido[}
\PYG{c+cp}{        \PYGZlt{}!ELEMENT pedido (pagado?, contenido, almacen?,domicilio?)\PYGZgt{}}
        \PYG{c+cp}{\PYGZlt{}!ELEMENT pagado EMPTY\PYGZgt{}}
        \PYG{c+cp}{\PYGZlt{}!ELEMENT contenido (\PYGZsh{}PCDATA)\PYGZgt{}}
        \PYG{c+cp}{\PYGZlt{}!ELEMENT almacen (\PYGZsh{}PCDATA)\PYGZgt{}}
        \PYG{c+cp}{\PYGZlt{}!ELEMENT domicilio (\PYGZsh{}PCDATA\PYGZgt{}}
]\PYGZgt{}
\end{sphinxVerbatim}

Está mal porque se permite esto:

\begin{sphinxVerbatim}[commandchars=\\\{\}]
\PYG{n+nt}{\PYGZlt{}pedido}\PYG{n+nt}{\PYGZgt{}}
        \PYG{n+nt}{\PYGZlt{}pagado}\PYG{n+nt}{/\PYGZgt{}}
        \PYG{n+nt}{\PYGZlt{}contenido}\PYG{n+nt}{\PYGZgt{}}Ordenadores\PYG{n+nt}{\PYGZlt{}/contenido\PYGZgt{}}
        \PYG{n+nt}{\PYGZlt{}almacen}\PYG{n+nt}{\PYGZgt{}}Entregado el 20\PYGZhy{}2\PYGZhy{}2011\PYG{n+nt}{\PYGZlt{}/almacen\PYGZgt{}}
        \PYG{n+nt}{\PYGZlt{}domicilio}\PYG{n+nt}{\PYGZgt{}}Entregado el 20\PYGZhy{}2011\PYG{n+nt}{\PYGZlt{}/domicilio\PYGZgt{}}
\PYG{n+nt}{\PYGZlt{}/pedido\PYGZgt{}}
\end{sphinxVerbatim}

La forma \sphinxstylestrong{correcta} es esta:

\begin{sphinxVerbatim}[commandchars=\\\{\}]
\PYG{c+cp}{\PYGZlt{}!DOCTYPE pedido[}
\PYG{c+cp}{        \PYGZlt{}!ELEMENT pedido (pagado?, contenido, (almacen\textbar{}domicilio)?\PYGZgt{}}
        \PYG{c+cp}{\PYGZlt{}!ELEMENT pagado EMPTY\PYGZgt{}}
        \PYG{c+cp}{\PYGZlt{}!ELEMENT contenido (\PYGZsh{}PCDATA)\PYGZgt{}}
        \PYG{c+cp}{\PYGZlt{}!ELEMENT almacen (\PYGZsh{}PCDATA)\PYGZgt{}}
        \PYG{c+cp}{\PYGZlt{}!ELEMENT domicilio (\PYGZsh{}PCDATA\PYGZgt{}}
]\PYGZgt{}
\PYG{n+nt}{\PYGZlt{}pedido}\PYG{n+nt}{\PYGZgt{}}
        \PYG{n+nt}{\PYGZlt{}contenido}\PYG{n+nt}{\PYGZgt{}}Ordenadores\PYG{n+nt}{\PYGZlt{}/contenido\PYGZgt{}}
\PYG{n+nt}{\PYGZlt{}/pedido\PYGZgt{}}
\end{sphinxVerbatim}


\section{Ejercicio}
\label{\detokenize{tema5:id1}}
Un mayorista informático necesita especificar las reglas de los elementos permitidos en las aplicaciones que utiliza en sus empresas, para ello ha indicado los siguientes requisitos:
\begin{itemize}
\item {} 
Una entrega consta de uno o más lotes.

\item {} 
Un lote tiene uno o más palés

\item {} 
Todo palé tiene una serie de elementos: número de cajas, contenido y peso y forma de manipulación.

\item {} 
El contenido consta de una serie de elementos: nombre del componente, procedencia (puede aparecer 0, 1 o más países), número de serie del componente, peso del componente individual y unidad de peso que puede aparecer o no.

\end{itemize}


\subsection{Solución}
\label{\detokenize{tema5:id2}}
Observa como en la siguiente DTD se pone \sphinxcode{procedencia?} y dentro de ella \sphinxcode{pais+}. Esto nos permite que si aparece la procedencia se debe especificar uno o más países. Sin embargo si no queremos que aparezca ningun pais, el XML \sphinxstylestrong{no necesita contener un elemento vacío}.

\begin{sphinxVerbatim}[commandchars=\\\{\}]
\PYG{k}{\PYGZlt{}!ELEMENT} \PYG{n+nt}{entrega} \PYG{o}{(}\PYG{n+nt}{lote}\PYG{o}{+}\PYG{o}{)}\PYG{k}{\PYGZgt{}}
\PYG{k}{\PYGZlt{}!ELEMENT} \PYG{n+nt}{lote} \PYG{o}{(}\PYG{n+nt}{pale}\PYG{o}{+}\PYG{o}{)}\PYG{k}{\PYGZgt{}}
\PYG{k}{\PYGZlt{}!ELEMENT} \PYG{n+nt}{pale} \PYG{o}{(}\PYG{n+nt}{numcajas}\PYG{o}{,} \PYG{n+nt}{contenido}\PYG{o}{,} \PYG{n+nt}{peso}\PYG{o}{,} \PYG{n+nt}{formamanipulacion}\PYG{o}{?}\PYG{o}{)}\PYG{k}{\PYGZgt{}}
\PYG{k}{\PYGZlt{}!ELEMENT} \PYG{n+nt}{numcajas} \PYG{o}{(}\PYG{k+kc}{\PYGZsh{}PCDATA}\PYG{o}{)}\PYG{k}{\PYGZgt{}}
\PYG{k}{\PYGZlt{}!ELEMENT} \PYG{n+nt}{peso} \PYG{o}{(}\PYG{k+kc}{\PYGZsh{}PCDATA}\PYG{o}{)}\PYG{k}{\PYGZgt{}}
\PYG{k}{\PYGZlt{}!ELEMENT} \PYG{n+nt}{formamanipulacion} \PYG{o}{(}\PYG{k+kc}{\PYGZsh{}PCDATA}\PYG{o}{)}\PYG{k}{\PYGZgt{}}
\PYG{k}{\PYGZlt{}!ELEMENT} \PYG{n+nt}{contenido} \PYG{o}{(}\PYG{n+nt}{nombrecomponente}\PYG{o}{,} \PYG{n+nt}{procedencia}\PYG{o}{?}\PYG{o}{,}
                        \PYG{n+nt}{numserie}\PYG{o}{,} \PYG{n+nt}{peso}\PYG{o}{,} \PYG{n+nt}{unidades}\PYG{o}{)}\PYG{k}{\PYGZgt{}}
\PYG{k}{\PYGZlt{}!ELEMENT} \PYG{n+nt}{nombrecomponente} \PYG{o}{(}\PYG{k+kc}{\PYGZsh{}PCDATA}\PYG{o}{)}\PYG{k}{\PYGZgt{}}
\PYG{k}{\PYGZlt{}!ELEMENT} \PYG{n+nt}{procedencia} \PYG{o}{(}\PYG{n+nt}{pais}\PYG{o}{+}\PYG{o}{)}\PYG{k}{\PYGZgt{}}
\PYG{k}{\PYGZlt{}!ELEMENT} \PYG{n+nt}{pais} \PYG{o}{(}\PYG{k+kc}{\PYGZsh{}PCDATA}\PYG{o}{)}\PYG{k}{\PYGZgt{}}
\PYG{k}{\PYGZlt{}!ELEMENT} \PYG{n+nt}{numserie} \PYG{o}{(}\PYG{k+kc}{\PYGZsh{}PCDATA}\PYG{o}{)}\PYG{k}{\PYGZgt{}}
\PYG{k}{\PYGZlt{}!ELEMENT} \PYG{n+nt}{unidades} \PYG{o}{(}\PYG{k+kc}{\PYGZsh{}PCDATA}\PYG{o}{)}\PYG{k}{\PYGZgt{}}
\end{sphinxVerbatim}

\begin{sphinxVerbatim}[commandchars=\\\{\}]
\PYG{c+cp}{\PYGZlt{}?xml version=\PYGZdq{}1.0\PYGZdq{} encoding=\PYGZdq{}UTF\PYGZhy{}8\PYGZdq{}?\PYGZgt{}}
\PYG{c+cp}{\PYGZlt{}!DOCTYPE entrega SYSTEM \PYGZdq{}mayorista.dtd\PYGZdq{}\PYGZgt{}}
\PYG{n+nt}{\PYGZlt{}entrega}\PYG{n+nt}{\PYGZgt{}}
  \PYG{n+nt}{\PYGZlt{}lote}\PYG{n+nt}{\PYGZgt{}}
        \PYG{n+nt}{\PYGZlt{}pale}\PYG{n+nt}{\PYGZgt{}}
          \PYG{n+nt}{\PYGZlt{}numcajas}\PYG{n+nt}{\PYGZgt{}}3\PYG{n+nt}{\PYGZlt{}/numcajas\PYGZgt{}}
          \PYG{n+nt}{\PYGZlt{}contenido}\PYG{n+nt}{\PYGZgt{}}
                \PYG{n+nt}{\PYGZlt{}nombrecomponente}\PYG{n+nt}{\PYGZgt{}}Fuentes\PYG{n+nt}{\PYGZlt{}/nombrecomponente\PYGZgt{}}
                \PYG{n+nt}{\PYGZlt{}numserie}\PYG{n+nt}{\PYGZgt{}}3A\PYG{n+nt}{\PYGZlt{}/numserie\PYGZgt{}}
                \PYG{n+nt}{\PYGZlt{}peso}\PYG{n+nt}{\PYGZgt{}}2kg\PYG{n+nt}{\PYGZlt{}/peso\PYGZgt{}}
                \PYG{n+nt}{\PYGZlt{}unidades}\PYG{n+nt}{\PYGZgt{}}50\PYG{n+nt}{\PYGZlt{}/unidades\PYGZgt{}}
          \PYG{n+nt}{\PYGZlt{}/contenido\PYGZgt{}}
          \PYG{n+nt}{\PYGZlt{}peso}\PYG{n+nt}{\PYGZgt{}}100kg\PYG{n+nt}{\PYGZlt{}/peso\PYGZgt{}}
          \PYG{n+nt}{\PYGZlt{}formamanipulacion}\PYG{n+nt}{\PYGZgt{}}Manual\PYG{n+nt}{\PYGZlt{}/formamanipulacion\PYGZgt{}}
        \PYG{n+nt}{\PYGZlt{}/pale\PYGZgt{}}
  \PYG{n+nt}{\PYGZlt{}/lote\PYGZgt{}}
  \PYG{n+nt}{\PYGZlt{}lote}\PYG{n+nt}{\PYGZgt{}}
        \PYG{n+nt}{\PYGZlt{}pale}\PYG{n+nt}{\PYGZgt{}}
          \PYG{n+nt}{\PYGZlt{}numcajas}\PYG{n+nt}{\PYGZgt{}}2\PYG{n+nt}{\PYGZlt{}/numcajas\PYGZgt{}}
          \PYG{n+nt}{\PYGZlt{}contenido}\PYG{n+nt}{\PYGZgt{}}
                \PYG{n+nt}{\PYGZlt{}nombrecomponente}\PYG{n+nt}{\PYGZgt{}}CPUs\PYG{n+nt}{\PYGZlt{}/nombrecomponente\PYGZgt{}}
                \PYG{n+nt}{\PYGZlt{}procedencia}\PYG{n+nt}{\PYGZgt{}}
                  \PYG{n+nt}{\PYGZlt{}pais}\PYG{n+nt}{\PYGZgt{}}China\PYG{n+nt}{\PYGZlt{}/pais\PYGZgt{}}
                  \PYG{n+nt}{\PYGZlt{}pais}\PYG{n+nt}{\PYGZgt{}}Corea del Sur\PYG{n+nt}{\PYGZlt{}/pais\PYGZgt{}}
                \PYG{n+nt}{\PYGZlt{}/procedencia\PYGZgt{}}
                \PYG{n+nt}{\PYGZlt{}numserie}\PYG{n+nt}{\PYGZgt{}}5B\PYG{n+nt}{\PYGZlt{}/numserie\PYGZgt{}}
                \PYG{n+nt}{\PYGZlt{}peso}\PYG{n+nt}{\PYGZgt{}}100g\PYG{n+nt}{\PYGZlt{}/peso\PYGZgt{}}
                \PYG{n+nt}{\PYGZlt{}unidades}\PYG{n+nt}{\PYGZgt{}}1000\PYG{n+nt}{\PYGZlt{}/unidades\PYGZgt{}}
          \PYG{n+nt}{\PYGZlt{}/contenido\PYGZgt{}}
          \PYG{n+nt}{\PYGZlt{}peso}\PYG{n+nt}{\PYGZgt{}}100kg\PYG{n+nt}{\PYGZlt{}/peso\PYGZgt{}}
          \PYG{n+nt}{\PYGZlt{}formamanipulacion}\PYG{n+nt}{\PYGZgt{}}Manual\PYG{n+nt}{\PYGZlt{}/formamanipulacion\PYGZgt{}}
        \PYG{n+nt}{\PYGZlt{}/pale\PYGZgt{}}
  \PYG{n+nt}{\PYGZlt{}/lote\PYGZgt{}}
\PYG{n+nt}{\PYGZlt{}/entrega\PYGZgt{}}
\end{sphinxVerbatim}


\section{Ejercicio: mayorista de libros}
\label{\detokenize{tema5:ejercicio-mayorista-de-libros}}
Se desea crear un formato de intercambio de datos para una empresa mayorista de libros con el fin de que sus distintos programas puedan manejar la información interna. El formato de archivo debe tener la siguiente estructura:
\begin{itemize}
\item {} 
Un archivo tiene una serie de operaciones dentro.

\item {} 
Las operaciones pueden ser «venta», «compra», o cualquier combinación y secuencia de ellas, pero debe haber al menos una.

\item {} 
Una venta tiene:
\begin{itemize}
\item {} 
Uno o más títulos vendidos.

\item {} 
La cantidad total de libros vendidos.

\item {} 
Puede haber un elemento «entregado» que indique si la entrega se ha realizado.

\item {} 
Debe haber un elemento importe con un atributo obligatorio llamado «moneda».

\end{itemize}

\item {} 
Una compra tiene:
\begin{itemize}
\item {} 
Uno o más títulos comprados.

\item {} 
Nombre de proveedor.

\item {} 
Una fecha de compra, que debe desglosarse en elementos día, mes y año

\end{itemize}

\end{itemize}

El objetivo final debe ser validar un fichero como este:

\begin{sphinxVerbatim}[commandchars=\\\{\}]
\PYG{n+nt}{\PYGZlt{}operaciones}\PYG{n+nt}{\PYGZgt{}}
    \PYG{n+nt}{\PYGZlt{}operacion}\PYG{n+nt}{\PYGZgt{}}
        \PYG{n+nt}{\PYGZlt{}venta}\PYG{n+nt}{\PYGZgt{}}
            \PYG{n+nt}{\PYGZlt{}titulosvendidos}\PYG{n+nt}{\PYGZgt{}}
                \PYG{n+nt}{\PYGZlt{}titulo}\PYG{n+nt}{\PYGZgt{}}Don Quijote\PYG{n+nt}{\PYGZlt{}/titulo\PYGZgt{}}
                \PYG{n+nt}{\PYGZlt{}titulo}\PYG{n+nt}{\PYGZgt{}}Rimas y leyendas\PYG{n+nt}{\PYGZlt{}/titulo\PYGZgt{}}
                \PYG{n+nt}{\PYGZlt{}cantidadtotal}\PYG{n+nt}{\PYGZgt{}}2000\PYG{n+nt}{\PYGZlt{}/cantidadtotal\PYGZgt{}}
                \PYG{n+nt}{\PYGZlt{}importe} \PYG{n+na}{moneda=}\PYG{l+s}{\PYGZdq{}euros\PYGZdq{}}\PYG{n+nt}{\PYGZgt{}}4400\PYG{n+nt}{\PYGZlt{}/importe\PYGZgt{}}
            \PYG{n+nt}{\PYGZlt{}/titulosvendidos\PYGZgt{}}
        \PYG{n+nt}{\PYGZlt{}/venta\PYGZgt{}}
        \PYG{n+nt}{\PYGZlt{}venta}\PYG{n+nt}{\PYGZgt{}}
            \PYG{n+nt}{\PYGZlt{}titulosvendidos}\PYG{n+nt}{\PYGZgt{}}
                \PYG{n+nt}{\PYGZlt{}titulo}\PYG{n+nt}{\PYGZgt{}}Rinconete y Cortadillo\PYG{n+nt}{\PYGZlt{}/titulo\PYGZgt{}}
                \PYG{n+nt}{\PYGZlt{}titulo}\PYG{n+nt}{\PYGZgt{}}Sainetes\PYG{n+nt}{\PYGZlt{}/titulo\PYGZgt{}}
                \PYG{n+nt}{\PYGZlt{}cantidadtotal}\PYG{n+nt}{\PYGZgt{}}1000\PYG{n+nt}{\PYGZlt{}/cantidadtotal\PYGZgt{}}
                \PYG{n+nt}{\PYGZlt{}entregado}\PYG{n+nt}{/\PYGZgt{}}
                \PYG{n+nt}{\PYGZlt{}importe} \PYG{n+na}{moneda=}\PYG{l+s}{\PYGZdq{}libras\PYGZdq{}}\PYG{n+nt}{\PYGZgt{}}290\PYG{n+nt}{\PYGZlt{}/importe\PYGZgt{}}
            \PYG{n+nt}{\PYGZlt{}/titulosvendidos\PYGZgt{}}
        \PYG{n+nt}{\PYGZlt{}/venta\PYGZgt{}}
    \PYG{n+nt}{\PYGZlt{}/operacion\PYGZgt{}}
    \PYG{n+nt}{\PYGZlt{}operacion}\PYG{n+nt}{\PYGZgt{}}
        \PYG{n+nt}{\PYGZlt{}compra}\PYG{n+nt}{\PYGZgt{}}
            \PYG{n+nt}{\PYGZlt{}tituloscomprados}\PYG{n+nt}{\PYGZgt{}}
                \PYG{n+nt}{\PYGZlt{}titulo}\PYG{n+nt}{\PYGZgt{}}De la Tierra a la Luna\PYG{n+nt}{\PYGZlt{}/titulo\PYGZgt{}}
                \PYG{n+nt}{\PYGZlt{}titulo}\PYG{n+nt}{\PYGZgt{}}Barbarroja\PYG{n+nt}{\PYGZlt{}/titulo\PYGZgt{}}
                \PYG{n+nt}{\PYGZlt{}proveedor}\PYG{n+nt}{\PYGZgt{}}Editorial EDSA\PYG{n+nt}{\PYGZlt{}/proveedor\PYGZgt{}}
                \PYG{n+nt}{\PYGZlt{}fechacompra}\PYG{n+nt}{\PYGZgt{}}
                    \PYG{n+nt}{\PYGZlt{}dia}\PYG{n+nt}{\PYGZgt{}}10\PYG{n+nt}{\PYGZlt{}/dia\PYGZgt{}}
                    \PYG{n+nt}{\PYGZlt{}mes}\PYG{n+nt}{\PYGZgt{}}6\PYG{n+nt}{\PYGZlt{}/mes\PYGZgt{}}
                    \PYG{n+nt}{\PYGZlt{}anio}\PYG{n+nt}{\PYGZgt{}}2018\PYG{n+nt}{\PYGZlt{}/anio\PYGZgt{}}
                \PYG{n+nt}{\PYGZlt{}/fechacompra\PYGZgt{}}
            \PYG{n+nt}{\PYGZlt{}/tituloscomprados\PYGZgt{}}
        \PYG{n+nt}{\PYGZlt{}/compra\PYGZgt{}}
        \PYG{n+nt}{\PYGZlt{}venta}\PYG{n+nt}{\PYGZgt{}}
            \PYG{n+nt}{\PYGZlt{}titulosvendidos}\PYG{n+nt}{\PYGZgt{}}
                \PYG{n+nt}{\PYGZlt{}titulo}\PYG{n+nt}{\PYGZgt{}}Cinco semanas en globo\PYG{n+nt}{\PYGZlt{}/titulo\PYGZgt{}}
                \PYG{n+nt}{\PYGZlt{}titulo}\PYG{n+nt}{\PYGZgt{}}Sainetes\PYG{n+nt}{\PYGZlt{}/titulo\PYGZgt{}}
                \PYG{n+nt}{\PYGZlt{}cantidadtotal}\PYG{n+nt}{\PYGZgt{}}700\PYG{n+nt}{\PYGZlt{}/cantidadtotal\PYGZgt{}}
                \PYG{n+nt}{\PYGZlt{}entregado}\PYG{n+nt}{/\PYGZgt{}}
                \PYG{n+nt}{\PYGZlt{}importe} \PYG{n+na}{moneda=}\PYG{l+s}{\PYGZdq{}euros\PYGZdq{}}\PYG{n+nt}{\PYGZgt{}}1490\PYG{n+nt}{\PYGZlt{}/importe\PYGZgt{}}
            \PYG{n+nt}{\PYGZlt{}/titulosvendidos\PYGZgt{}}
        \PYG{n+nt}{\PYGZlt{}/venta\PYGZgt{}}
        \PYG{n+nt}{\PYGZlt{}compra}\PYG{n+nt}{\PYGZgt{}}
            \PYG{n+nt}{\PYGZlt{}tituloscomprados}\PYG{n+nt}{\PYGZgt{}}
                \PYG{n+nt}{\PYGZlt{}titulo}\PYG{n+nt}{\PYGZgt{}}De la Tierra a la Luna\PYG{n+nt}{\PYGZlt{}/titulo\PYGZgt{}}
                \PYG{n+nt}{\PYGZlt{}titulo}\PYG{n+nt}{\PYGZgt{}}Barbarroja\PYG{n+nt}{\PYGZlt{}/titulo\PYGZgt{}}
                \PYG{n+nt}{\PYGZlt{}proveedor}\PYG{n+nt}{\PYGZgt{}}Editorial Recopila\PYG{n+nt}{\PYGZlt{}/proveedor\PYGZgt{}}
                \PYG{n+nt}{\PYGZlt{}fechacompra}\PYG{n+nt}{\PYGZgt{}}
                    \PYG{n+nt}{\PYGZlt{}dia}\PYG{n+nt}{\PYGZgt{}}2\PYG{n+nt}{\PYGZlt{}/dia\PYGZgt{}}
                    \PYG{n+nt}{\PYGZlt{}mes}\PYG{n+nt}{\PYGZgt{}}12\PYG{n+nt}{\PYGZlt{}/mes\PYGZgt{}}
                    \PYG{n+nt}{\PYGZlt{}anio}\PYG{n+nt}{\PYGZgt{}}2017\PYG{n+nt}{\PYGZlt{}/anio\PYGZgt{}}
                \PYG{n+nt}{\PYGZlt{}/fechacompra\PYGZgt{}}
            \PYG{n+nt}{\PYGZlt{}/tituloscomprados\PYGZgt{}}
        \PYG{n+nt}{\PYGZlt{}/compra\PYGZgt{}}
    \PYG{n+nt}{\PYGZlt{}/operacion\PYGZgt{}}
\PYG{n+nt}{\PYGZlt{}/operaciones\PYGZgt{}}
\end{sphinxVerbatim}


\subsection{Solución al mayorista de libros}
\label{\detokenize{tema5:solucion-al-mayorista-de-libros}}
La siguiente DTD valida el fichero arriba mostrado:

\begin{sphinxVerbatim}[commandchars=\\\{\}]
\PYG{c}{\PYGZlt{}!\PYGZhy{}\PYGZhy{}}\PYG{c}{El elemento raíz es operaciones y dentro de él hay uno o más elementos operación}\PYG{c}{\PYGZhy{}\PYGZhy{}\PYGZgt{}}
\PYG{k}{\PYGZlt{}!ELEMENT} \PYG{n+nt}{operaciones} \PYG{o}{(}\PYG{n+nt}{operacion}\PYG{o}{+}\PYG{o}{)}\PYG{k}{\PYGZgt{}}
\PYG{c}{\PYGZlt{}!\PYGZhy{}\PYGZhy{}}\PYG{c}{Una operación puede ser ventas o compras, en cualquier orden y repetidas las veces que sea necesario}\PYG{c}{\PYGZhy{}\PYGZhy{}\PYGZgt{}}
\PYG{k}{\PYGZlt{}!ELEMENT} \PYG{n+nt}{operacion} \PYG{o}{(}\PYG{n+nt}{venta}\PYG{o}{\textbar{}}\PYG{n+nt}{compra}\PYG{o}{)}\PYG{o}{+}\PYG{k}{\PYGZgt{}}
\PYG{k}{\PYGZlt{}!ELEMENT} \PYG{n+nt}{venta} \PYG{o}{(}\PYG{n+nt}{titulosvendidos}\PYG{o}{)}\PYG{k}{\PYGZgt{}}
\PYG{c}{\PYGZlt{}!\PYGZhy{}\PYGZhy{}}\PYG{c}{Una venta tiene uno o más titulos, la cantidad de libros vendidos, puede haber un elemento entregado que indique si la entrega se ha realizado, y debe haber un elemento importe con un atributo obligatorio llamado moneda. }\PYG{c}{\PYGZhy{}\PYGZhy{}\PYGZgt{}}
\PYG{k}{\PYGZlt{}!ELEMENT} \PYG{n+nt}{titulosvendidos} \PYG{o}{(}\PYG{n+nt}{titulo}\PYG{o}{+}\PYG{o}{,} \PYG{n+nt}{cantidadtotal}\PYG{o}{,} \PYG{n+nt}{entregado}\PYG{o}{?}\PYG{o}{,} \PYG{n+nt}{importe}\PYG{o}{)}\PYG{k}{\PYGZgt{}}
\PYG{c}{\PYGZlt{}!\PYGZhy{}\PYGZhy{}}\PYG{c}{Antes de que se nos olvide, fabricamos el elemento importe y su atributo moneda}\PYG{c}{\PYGZhy{}\PYGZhy{}\PYGZgt{}}
\PYG{k}{\PYGZlt{}!ELEMENT} \PYG{n+nt}{importe} \PYG{o}{(}\PYG{k+kc}{\PYGZsh{}PCDATA}\PYG{o}{)}\PYG{k}{\PYGZgt{}}
\PYG{k}{\PYGZlt{}!ATTLIST} \PYG{n+nt}{importe} \PYG{n+na}{moneda} \PYG{k+kc}{CDATA} \PYG{k+kc}{\PYGZsh{}REQUIRED}\PYG{k}{\PYGZgt{}}
\PYG{c}{\PYGZlt{}!\PYGZhy{}\PYGZhy{}}\PYG{c}{Fabricamos el titulo y la cantidad total}\PYG{c}{\PYGZhy{}\PYGZhy{}\PYGZgt{}}
\PYG{k}{\PYGZlt{}!ELEMENT} \PYG{n+nt}{titulo} \PYG{o}{(}\PYG{k+kc}{\PYGZsh{}PCDATA}\PYG{o}{)}\PYG{k}{\PYGZgt{}}
\PYG{k}{\PYGZlt{}!ELEMENT} \PYG{n+nt}{cantidadtotal} \PYG{o}{(}\PYG{k+kc}{\PYGZsh{}PCDATA}\PYG{o}{)}\PYG{k}{\PYGZgt{}}
\PYG{c}{\PYGZlt{}!\PYGZhy{}\PYGZhy{}}\PYG{c}{El elemento entregado parece que es un vacío}\PYG{c}{\PYGZhy{}\PYGZhy{}\PYGZgt{}}
\PYG{k}{\PYGZlt{}!ELEMENT} \PYG{n+nt}{entregado} \PYG{k+kc}{EMPTY}\PYG{k}{\PYGZgt{}}
\PYG{c}{\PYGZlt{}!\PYGZhy{}\PYGZhy{}}\PYG{c}{Una compra tiene:}

\PYG{c}{\PYGZhy{}}\PYG{c}{Uno o más títulos comprados.}
\PYG{c}{\PYGZhy{}}\PYG{c}{Nombre de proveedor.}
\PYG{c}{\PYGZhy{}}\PYG{c}{Una fecha de compra, que debe desglosarse en elementos día, mes y año }\PYG{c}{\PYGZhy{}\PYGZhy{}\PYGZgt{}}
\PYG{k}{\PYGZlt{}!ELEMENT} \PYG{n+nt}{compra} \PYG{o}{(}\PYG{n+nt}{tituloscomprados}\PYG{o}{)}\PYG{k}{\PYGZgt{}}
\PYG{k}{\PYGZlt{}!ELEMENT} \PYG{n+nt}{tituloscomprados} \PYG{o}{(}\PYG{n+nt}{titulo}\PYG{o}{+}\PYG{o}{,} \PYG{n+nt}{proveedor}\PYG{o}{,} \PYG{n+nt}{fechacompra}\PYG{o}{)}\PYG{k}{\PYGZgt{}}
\PYG{k}{\PYGZlt{}!ELEMENT} \PYG{n+nt}{proveedor} \PYG{o}{(}\PYG{k+kc}{\PYGZsh{}PCDATA}\PYG{o}{)}\PYG{k}{\PYGZgt{}}
\PYG{c}{\PYGZlt{}!\PYGZhy{}\PYGZhy{}}\PYG{c}{Desglosamos la fecha}\PYG{c}{\PYGZhy{}\PYGZhy{}\PYGZgt{}}
\PYG{k}{\PYGZlt{}!ELEMENT} \PYG{n+nt}{fechacompra} \PYG{o}{(}\PYG{n+nt}{dia}\PYG{o}{,} \PYG{n+nt}{mes}\PYG{o}{,} \PYG{n+nt}{anio}\PYG{o}{)}\PYG{k}{\PYGZgt{}}
\PYG{k}{\PYGZlt{}!ELEMENT} \PYG{n+nt}{dia}  \PYG{o}{(}\PYG{k+kc}{\PYGZsh{}PCDATA}\PYG{o}{)}\PYG{k}{\PYGZgt{}}
\PYG{k}{\PYGZlt{}!ELEMENT} \PYG{n+nt}{mes}  \PYG{o}{(}\PYG{k+kc}{\PYGZsh{}PCDATA}\PYG{o}{)}\PYG{k}{\PYGZgt{}}
\PYG{k}{\PYGZlt{}!ELEMENT} \PYG{n+nt}{anio} \PYG{o}{(}\PYG{k+kc}{\PYGZsh{}PCDATA}\PYG{o}{)}\PYG{k}{\PYGZgt{}}
\end{sphinxVerbatim}


\section{Ejercicio: fabricante de tractores}
\label{\detokenize{tema5:ejercicio-fabricante-de-tractores}}
Un fabricante de tractores desea unificar el formato XML de sus proveedores y para ello ha indicado que necesita que los archivos XML cumplan las siguientes restricciones:
\begin{itemize}
\item {} 
Un pedido consta de uno o más tractores.

\item {} 
Un tractor consta de uno o más componentes.

\item {} 
Un componente tiene los siguientes elementos: nombre del fabricante (atributo obligatorio), fecha de entrega  (si es posible, aunque puede que no aparezca, si aparece el dia es optativo, pero el mes y el año son obligatorios). También se necesita saber del componente si es frágil o no. También debe aparecer un elemento peso del componente y dicho elemento peso tiene un atributo unidad del peso (kilos o gramos), un elemento número de serie y puede que aparezca o no un elemento kmmaximos indicando que el componente debe sustituirse tras un cierto número de kilómetros.

\end{itemize}

Un posible fichero de ejemplo que podría validar sería este:

\begin{sphinxVerbatim}[commandchars=\\\{\}]
\PYG{n+nt}{\PYGZlt{}pedido}\PYG{n+nt}{\PYGZgt{}}
    \PYG{n+nt}{\PYGZlt{}tractor}\PYG{n+nt}{\PYGZgt{}}
        \PYG{n+nt}{\PYGZlt{}componente} \PYG{n+na}{nombrefabricante=}\PYG{l+s}{\PYGZdq{}Ebro\PYGZdq{}}\PYG{n+nt}{\PYGZgt{}}
            \PYG{n+nt}{\PYGZlt{}fechaentrega}\PYG{n+nt}{\PYGZgt{}}
                \PYG{n+nt}{\PYGZlt{}mes}\PYG{n+nt}{\PYGZgt{}}2018\PYG{n+nt}{\PYGZlt{}/mes\PYGZgt{}} \PYG{n+nt}{\PYGZlt{}anio}\PYG{n+nt}{\PYGZgt{}}2018\PYG{n+nt}{\PYGZlt{}/anio\PYGZgt{}}
            \PYG{n+nt}{\PYGZlt{}/fechaentrega\PYGZgt{}}
            \PYG{n+nt}{\PYGZlt{}fragil}\PYG{n+nt}{/\PYGZgt{}}
            \PYG{n+nt}{\PYGZlt{}peso} \PYG{n+na}{unidad=}\PYG{l+s}{\PYGZdq{}kg\PYGZdq{}}\PYG{n+nt}{\PYGZgt{}}12\PYG{n+nt}{\PYGZlt{}/peso\PYGZgt{}}
            \PYG{n+nt}{\PYGZlt{}numserie}\PYG{n+nt}{\PYGZgt{}}00A\PYG{n+nt}{\PYGZlt{}/numserie\PYGZgt{}}
        \PYG{n+nt}{\PYGZlt{}/componente\PYGZgt{}}
        \PYG{n+nt}{\PYGZlt{}componente} \PYG{n+na}{nombrefabricante=}\PYG{l+s}{\PYGZdq{}Avia\PYGZdq{}}\PYG{n+nt}{\PYGZgt{}}
            \PYG{n+nt}{\PYGZlt{}fechaentrega}\PYG{n+nt}{\PYGZgt{}}
                \PYG{n+nt}{\PYGZlt{}dia}\PYG{n+nt}{\PYGZgt{}}12\PYG{n+nt}{\PYGZlt{}/dia\PYGZgt{}}\PYG{n+nt}{\PYGZlt{}mes}\PYG{n+nt}{\PYGZgt{}}1\PYG{n+nt}{\PYGZlt{}/mes\PYGZgt{}}\PYG{n+nt}{\PYGZlt{}anio}\PYG{n+nt}{\PYGZgt{}}2019\PYG{n+nt}{\PYGZlt{}/anio\PYGZgt{}}
            \PYG{n+nt}{\PYGZlt{}/fechaentrega\PYGZgt{}}
            \PYG{n+nt}{\PYGZlt{}nofragil}\PYG{n+nt}{/\PYGZgt{}}
            \PYG{n+nt}{\PYGZlt{}peso} \PYG{n+na}{unidad=}\PYG{l+s}{\PYGZdq{}g\PYGZdq{}}\PYG{n+nt}{\PYGZgt{}}1450\PYG{n+nt}{\PYGZlt{}/peso\PYGZgt{}}
            \PYG{n+nt}{\PYGZlt{}numserie}\PYG{n+nt}{\PYGZgt{}}00D\PYG{n+nt}{\PYGZlt{}/numserie\PYGZgt{}}
            \PYG{n+nt}{\PYGZlt{}kmmaximos}\PYG{n+nt}{\PYGZgt{}}25000\PYG{n+nt}{\PYGZlt{}/kmmaximos\PYGZgt{}}
        \PYG{n+nt}{\PYGZlt{}/componente\PYGZgt{}}
    \PYG{n+nt}{\PYGZlt{}/tractor\PYGZgt{}}
    \PYG{n+nt}{\PYGZlt{}tractor}\PYG{n+nt}{\PYGZgt{}}
        \PYG{n+nt}{\PYGZlt{}componente} \PYG{n+na}{nombrefabricante=}\PYG{l+s}{\PYGZdq{}John Deere\PYGZdq{}}\PYG{n+nt}{\PYGZgt{}}
            \PYG{n+nt}{\PYGZlt{}fragil}\PYG{n+nt}{/\PYGZgt{}}
            \PYG{n+nt}{\PYGZlt{}peso} \PYG{n+na}{unidad=}\PYG{l+s}{\PYGZdq{}g\PYGZdq{}}\PYG{n+nt}{\PYGZgt{}}770\PYG{n+nt}{\PYGZlt{}/peso\PYGZgt{}}
            \PYG{n+nt}{\PYGZlt{}numserie}\PYG{n+nt}{\PYGZgt{}}43Z\PYG{n+nt}{\PYGZlt{}/numserie\PYGZgt{}}
        \PYG{n+nt}{\PYGZlt{}/componente\PYGZgt{}}
    \PYG{n+nt}{\PYGZlt{}/tractor\PYGZgt{}}
\PYG{n+nt}{\PYGZlt{}/pedido\PYGZgt{}}
\end{sphinxVerbatim}


\subsection{Solución: DTD fabricante tractores}
\label{\detokenize{tema5:solucion-dtd-fabricante-tractores}}
\begin{sphinxVerbatim}[commandchars=\\\{\}]
\PYG{k}{\PYGZlt{}!ELEMENT} \PYG{n+nt}{pedido}     \PYG{o}{(}\PYG{n+nt}{tractor}\PYG{o}{)}\PYG{o}{+}\PYG{k}{\PYGZgt{}}
\PYG{k}{\PYGZlt{}!ELEMENT} \PYG{n+nt}{tractor}    \PYG{o}{(}\PYG{n+nt}{componente}\PYG{o}{)}\PYG{o}{+}\PYG{k}{\PYGZgt{}}
\PYG{k}{\PYGZlt{}!ELEMENT} \PYG{n+nt}{componente} \PYG{o}{(}\PYG{n+nt}{fechaentrega}\PYG{o}{?}\PYG{o}{,} \PYG{o}{(}\PYG{n+nt}{fragil}\PYG{o}{\textbar{}}\PYG{n+nt}{nofragil}\PYG{o}{)}\PYG{o}{,}
                      \PYG{n+nt}{peso}\PYG{o}{,} \PYG{n+nt}{numserie}\PYG{o}{,} \PYG{n+nt}{kmmaximos}\PYG{o}{?}\PYG{o}{)}\PYG{k}{\PYGZgt{}}

\PYG{k}{\PYGZlt{}!ELEMENT} \PYG{n+nt}{fechaentrega} \PYG{o}{(}\PYG{n+nt}{dia}\PYG{o}{?}\PYG{o}{,} \PYG{n+nt}{mes}\PYG{o}{,} \PYG{n+nt}{anio}\PYG{o}{)}\PYG{k}{\PYGZgt{}}
\PYG{k}{\PYGZlt{}!ELEMENT} \PYG{n+nt}{dia}      \PYG{o}{(}\PYG{k+kc}{\PYGZsh{}PCDATA}\PYG{o}{)}\PYG{k}{\PYGZgt{}}
\PYG{k}{\PYGZlt{}!ELEMENT} \PYG{n+nt}{mes}      \PYG{o}{(}\PYG{k+kc}{\PYGZsh{}PCDATA}\PYG{o}{)}\PYG{k}{\PYGZgt{}}
\PYG{k}{\PYGZlt{}!ELEMENT} \PYG{n+nt}{anio}     \PYG{o}{(}\PYG{k+kc}{\PYGZsh{}PCDATA}\PYG{o}{)}\PYG{k}{\PYGZgt{}}
\PYG{k}{\PYGZlt{}!ELEMENT} \PYG{n+nt}{fragil}   \PYG{k+kc}{EMPTY}\PYG{k}{\PYGZgt{}}
\PYG{k}{\PYGZlt{}!ELEMENT} \PYG{n+nt}{nofragil} \PYG{k+kc}{EMPTY} \PYG{k}{\PYGZgt{}}
\PYG{k}{\PYGZlt{}!ELEMENT} \PYG{n+nt}{peso}     \PYG{o}{(}\PYG{k+kc}{\PYGZsh{}PCDATA}\PYG{o}{)}\PYG{k}{\PYGZgt{}}
\PYG{k}{\PYGZlt{}!ATTLIST} \PYG{n+nt}{peso} \PYG{n+na}{unidad} \PYG{k+kc}{CDATA} \PYG{k+kc}{\PYGZsh{}REQUIRED}\PYG{k}{\PYGZgt{}}
\PYG{k}{\PYGZlt{}!ELEMENT} \PYG{n+nt}{numserie}  \PYG{o}{(}\PYG{k+kc}{\PYGZsh{}PCDATA}\PYG{o}{)}\PYG{k}{\PYGZgt{}}
\PYG{k}{\PYGZlt{}!ELEMENT} \PYG{n+nt}{kmmaximos} \PYG{o}{(}\PYG{k+kc}{\PYGZsh{}PCDATA}\PYG{o}{)}\PYG{k}{\PYGZgt{}}
\PYG{k}{\PYGZlt{}!ATTLIST} \PYG{n+nt}{componente} \PYG{n+na}{nombrefabricante} \PYG{k+kc}{CDATA} \PYG{k+kc}{\PYGZsh{}REQUIRED}\PYG{k}{\PYGZgt{}}
\end{sphinxVerbatim}


\section{Ejercicio: repeticiones de opciones}
\label{\detokenize{tema5:ejercicio-repeticiones-de-opciones}}
Se necesita un formato de archivo para intercambiar productos entre almacenes de productos de librería y se desea una DTD que incluya estas restricciones:
\begin{itemize}
\item {} 
Debe haber un elemento raíz pedido que puede constar de libros, cuadernos y/o lápices. Los tres elementos pueden aparecer repetidos y en cualquier orden. Tambien pueden aparecer por ejemplo 4 libros, 2 lapices y luego 4 lapices de nuevo.

\item {} 
Todo libro tiene un atributo obligatorio titulo.

\item {} 
Los elementos cuaderno tiene un atributo optativo num\_hojas.

\item {} 
Todo elemento lápiz debe tener dentro un  elemento obligatorio número.

\end{itemize}

La solución a la DTD:

\begin{sphinxVerbatim}[commandchars=\\\{\}]
\PYG{k}{\PYGZlt{}!ELEMENT} \PYG{n+nt}{pedido} \PYG{o}{(}\PYG{n+nt}{libro}\PYG{o}{\textbar{}}\PYG{n+nt}{cuaderno}\PYG{o}{\textbar{}}\PYG{n+nt}{lapiz}\PYG{o}{)}\PYG{o}{+}\PYG{k}{\PYGZgt{}}
\PYG{k}{\PYGZlt{}!ELEMENT} \PYG{n+nt}{libro} \PYG{o}{(}\PYG{k+kc}{\PYGZsh{}PCDATA}\PYG{o}{)}\PYG{k}{\PYGZgt{}}
\PYG{k}{\PYGZlt{}!ATTLIST} \PYG{n+nt}{libro} \PYG{n+na}{titulo} \PYG{k+kc}{CDATA} \PYG{k+kc}{\PYGZsh{}REQUIRED}\PYG{k}{\PYGZgt{}}
\PYG{k}{\PYGZlt{}!ELEMENT} \PYG{n+nt}{cuaderno} \PYG{o}{(}\PYG{k+kc}{\PYGZsh{}PCDATA}\PYG{o}{)}\PYG{k}{\PYGZgt{}}
\PYG{k}{\PYGZlt{}!ATTLIST} \PYG{n+nt}{cuaderno} \PYG{n+na}{num\PYGZus{}hojas} \PYG{k+kc}{CDATA} \PYG{k+kc}{\PYGZsh{}IMPLIED}\PYG{k}{\PYGZgt{}}
\PYG{k}{\PYGZlt{}!ELEMENT} \PYG{n+nt}{lapiz} \PYG{o}{(}\PYG{n+nt}{numero}\PYG{o}{)}\PYG{k}{\PYGZgt{}}
\PYG{k}{\PYGZlt{}!ELEMENT} \PYG{n+nt}{numero} \PYG{o}{(}\PYG{k+kc}{\PYGZsh{}PCDATA}\PYG{o}{)}\PYG{k}{\PYGZgt{}}
\end{sphinxVerbatim}

\begin{sphinxVerbatim}[commandchars=\\\{\}]
\PYG{c+cp}{\PYGZlt{}?xml version=\PYGZdq{}1.0\PYGZdq{} encoding=\PYGZdq{}UTF\PYGZhy{}8\PYGZdq{}?\PYGZgt{}}
\PYG{c+cp}{\PYGZlt{}!DOCTYPE pedido SYSTEM \PYGZdq{}libreria.dtd\PYGZdq{}\PYGZgt{}}
\PYG{n+nt}{\PYGZlt{}pedido}\PYG{n+nt}{\PYGZgt{}}
  \PYG{n+nt}{\PYGZlt{}libro} \PYG{n+na}{titulo=}\PYG{l+s}{\PYGZdq{}Java 8\PYGZdq{}}\PYG{n+nt}{\PYGZgt{}}\PYG{n+nt}{\PYGZlt{}/libro\PYGZgt{}}
  \PYG{n+nt}{\PYGZlt{}cuaderno}\PYG{n+nt}{\PYGZgt{}}\PYG{n+nt}{\PYGZlt{}/cuaderno\PYGZgt{}}
  \PYG{n+nt}{\PYGZlt{}libro} \PYG{n+na}{titulo=}\PYG{l+s}{\PYGZdq{}HTML y CSS\PYGZdq{}}\PYG{n+nt}{/\PYGZgt{}}
  \PYG{n+nt}{\PYGZlt{}libro} \PYG{n+na}{titulo=}\PYG{l+s}{\PYGZdq{}SQL para Dummies\PYGZdq{}}\PYG{n+nt}{/\PYGZgt{}}
  \PYG{n+nt}{\PYGZlt{}cuaderno} \PYG{n+na}{num\PYGZus{}hojas=}\PYG{l+s}{\PYGZdq{}150\PYGZdq{}}\PYG{n+nt}{/\PYGZgt{}}
  \PYG{n+nt}{\PYGZlt{}lapiz}\PYG{n+nt}{\PYGZgt{}}
        \PYG{n+nt}{\PYGZlt{}numero}\PYG{n+nt}{\PYGZgt{}}2H\PYG{n+nt}{\PYGZlt{}/numero\PYGZgt{}}
  \PYG{n+nt}{\PYGZlt{}/lapiz\PYGZgt{}}
  \PYG{n+nt}{\PYGZlt{}cuaderno} \PYG{n+na}{num\PYGZus{}hojas=}\PYG{l+s}{\PYGZdq{}250\PYGZdq{}}\PYG{n+nt}{/\PYGZgt{}}
  \PYG{n+nt}{\PYGZlt{}cuaderno} \PYG{n+na}{num\PYGZus{}hojas=}\PYG{l+s}{\PYGZdq{}100\PYGZdq{}}\PYG{n+nt}{/\PYGZgt{}}
  \PYG{n+nt}{\PYGZlt{}lapiz}\PYG{n+nt}{\PYGZgt{}}
        \PYG{n+nt}{\PYGZlt{}numero}\PYG{n+nt}{\PYGZgt{}}2B\PYG{n+nt}{\PYGZlt{}/numero\PYGZgt{}}
  \PYG{n+nt}{\PYGZlt{}/lapiz\PYGZgt{}}
  \PYG{n+nt}{\PYGZlt{}lapiz}\PYG{n+nt}{\PYGZgt{}}
        \PYG{n+nt}{\PYGZlt{}numero}\PYG{n+nt}{\PYGZgt{}}1HB\PYG{n+nt}{\PYGZlt{}/numero\PYGZgt{}}
  \PYG{n+nt}{\PYGZlt{}/lapiz\PYGZgt{}}
\PYG{n+nt}{\PYGZlt{}/pedido\PYGZgt{}}
\end{sphinxVerbatim}


\section{Ejercicio: multinacional}
\label{\detokenize{tema5:ejercicio-multinacional}}
Una multinacional que opera en bolsa necesita un formato de intercambio de datos para que sus programas intercambien información sobre los mercados de acciones.

En general todo archivo constará de un listado de cosas como se detalla a continuación
\begin{itemize}
\item {} 
En el listado aparecen siempre uno o varios futuros, despues una o varias divisas, despues uno o varios bonos y una o varias letras.

\item {} 
Todos ellos tienen un atributo precio que es \sphinxstylestrong{obligatorio}

\item {} 
Todos ellos tienen un elemento vacío que indica  de donde es el producto anterior: «Madrid», «Nueva York», «Frankfurt» o «Tokio».

\item {} 
Las divisas y los bonos tienen un atributo optativo que se usa para indicar si el producto ha sido estable en el pasado o no.

\item {} 
Un futuro es un valor esperado que tendrá un cierto producto en el futuro. Se debe incluir este producto en forma de elemento. También puede aparecer un elemento mercado que indique el país de procedencia del producto.

\item {} 
Todo bono tiene un elemento país\_de\_procedencia para saber a qué estado pertenece. Debe tener tres elementos extra llamados «valor\_deseado», «valor\_mínimo» y «valor\_máximo» para saber los posibles precios.

\item {} 
Las divisas tienen siempre un nombre pueden incluir uno o más tipos de cambio para otras monedas.

\item {} 
Las letras tienen siempre un tipo de interés pagadero por un país emisor. El país emisor también debe existir y debe ser siempre de uno de los países cuyas capitales aparecen arriba (es decir «España», «EEUU», «Alemania» y «Japón»

\end{itemize}

\begin{sphinxVerbatim}[commandchars=\\\{\}]
\PYG{c+cp}{\PYGZlt{}?xml version=\PYGZdq{}1.0\PYGZdq{} encoding=\PYGZdq{}utf\PYGZhy{}8\PYGZdq{}?\PYGZgt{}}
\PYG{c+cp}{\PYGZlt{}!DOCTYPE listado [}
\PYG{c+cp}{        \PYGZlt{}!ELEMENT listado (futuro+, divisa+, bono+, letra+)\PYGZgt{}}
        \PYG{c+cp}{\PYGZlt{}!ATTLIST futuro precio CDATA \PYGZsh{}REQUIRED\PYGZgt{}}
        \PYG{c+cp}{\PYGZlt{}!ATTLIST divisa precio CDATA \PYGZsh{}REQUIRED\PYGZgt{}}
        \PYG{c+cp}{\PYGZlt{}!ATTLIST bono precio CDATA \PYGZsh{}REQUIRED\PYGZgt{}}
        \PYG{c+cp}{\PYGZlt{}!ATTLIST letra precio CDATA \PYGZsh{}REQUIRED\PYGZgt{}}
        \PYG{c+cp}{\PYGZlt{}!ELEMENT ciudad\PYGZus{}procedencia (madrid\textbar{}nyork\textbar{}frankfurt\textbar{}tokio)\PYGZgt{}}
        \PYG{c+cp}{\PYGZlt{}!ELEMENT madrid EMPTY\PYGZgt{}}
        \PYG{c+cp}{\PYGZlt{}!ELEMENT nyork EMPTY\PYGZgt{}}
        \PYG{c+cp}{\PYGZlt{}!ELEMENT frankfurt EMPTY\PYGZgt{}}
        \PYG{c+cp}{\PYGZlt{}!ELEMENT tokio EMPTY\PYGZgt{}}
        \PYG{c+cp}{\PYGZlt{}!ATTLIST divisa estable CDATA \PYGZsh{}IMPLIED\PYGZgt{}}
        \PYG{c+cp}{\PYGZlt{}!ATTLIST bono estable CDATA \PYGZsh{}IMPLIED\PYGZgt{}}
        \PYG{c+cp}{\PYGZlt{}!ELEMENT futuro (producto, mercado?, ciudad\PYGZus{}procedencia)\PYGZgt{}}
        \PYG{c+cp}{\PYGZlt{}!ELEMENT producto (\PYGZsh{}PCDATA)\PYGZgt{}}
        \PYG{c+cp}{\PYGZlt{}!ELEMENT mercado (\PYGZsh{}PCDATA)\PYGZgt{}}
        \PYG{c+cp}{\PYGZlt{}!ELEMENT bono (pais\PYGZus{}de\PYGZus{}procedencia,valor\PYGZus{}deseado,}
\PYG{c+cp}{                        valor\PYGZus{}minimo, valor\PYGZus{}maximo, ciudad\PYGZus{}procedencia)\PYGZgt{}}
        \PYG{c+cp}{\PYGZlt{}!ELEMENT valor\PYGZus{}deseado (\PYGZsh{}PCDATA)\PYGZgt{}}
        \PYG{c+cp}{\PYGZlt{}!ELEMENT valor\PYGZus{}minimo (\PYGZsh{}PCDATA)\PYGZgt{}}
        \PYG{c+cp}{\PYGZlt{}!ELEMENT valor\PYGZus{}maximo (\PYGZsh{}PCDATA)\PYGZgt{}}
        \PYG{c+cp}{\PYGZlt{}!ELEMENT pais\PYGZus{}de\PYGZus{}procedencia (\PYGZsh{}PCDATA)\PYGZgt{}}
        \PYG{c+cp}{\PYGZlt{}!ELEMENT divisa (nombre\PYGZus{}divisa,}
\PYG{c+cp}{                        tipo\PYGZus{}de\PYGZus{}cambio+, ciudad\PYGZus{}procedencia)\PYGZgt{}}
        \PYG{c+cp}{\PYGZlt{}!ELEMENT nombre\PYGZus{}divisa (\PYGZsh{}PCDATA)\PYGZgt{}}
        \PYG{c+cp}{\PYGZlt{}!ELEMENT tipo\PYGZus{}de\PYGZus{}cambio (\PYGZsh{}PCDATA)\PYGZgt{}}
        \PYG{c+cp}{\PYGZlt{}!ELEMENT letra (tipo\PYGZus{}de\PYGZus{}interes, pais\PYGZus{}emisor,ciudad\PYGZus{}procedencia)\PYGZgt{}}
        \PYG{c+cp}{\PYGZlt{}!ELEMENT tipo\PYGZus{}de\PYGZus{}interes (\PYGZsh{}PCDATA)\PYGZgt{}}
        \PYG{c+cp}{\PYGZlt{}!ELEMENT pais\PYGZus{}emisor (espania\textbar{}eeuu\textbar{}alemania\textbar{}japon)\PYGZgt{}}
        \PYG{c+cp}{\PYGZlt{}!ELEMENT espania     EMPTY\PYGZgt{}}
        \PYG{c+cp}{\PYGZlt{}!ELEMENT eeuu        EMPTY\PYGZgt{}}
        \PYG{c+cp}{\PYGZlt{}!ELEMENT alemania    EMPTY\PYGZgt{}}
        \PYG{c+cp}{\PYGZlt{}!ELEMENT japon       EMPTY\PYGZgt{}}
]\PYGZgt{}


\PYG{n+nt}{\PYGZlt{}listado}\PYG{n+nt}{\PYGZgt{}}
        \PYG{n+nt}{\PYGZlt{}futuro} \PYG{n+na}{precio=}\PYG{l+s}{\PYGZdq{}11.28\PYGZdq{}}\PYG{n+nt}{\PYGZgt{}}
                \PYG{n+nt}{\PYGZlt{}producto}\PYG{n+nt}{\PYGZgt{}}Cafe\PYG{n+nt}{\PYGZlt{}/producto\PYGZgt{}}
                \PYG{n+nt}{\PYGZlt{}mercado}\PYG{n+nt}{\PYGZgt{}}América Latina\PYG{n+nt}{\PYGZlt{}/mercado\PYGZgt{}}
                \PYG{n+nt}{\PYGZlt{}ciudad\PYGZus{}procedencia}\PYG{n+nt}{\PYGZgt{}}
                        \PYG{n+nt}{\PYGZlt{}frankfurt}\PYG{n+nt}{/\PYGZgt{}}
                \PYG{n+nt}{\PYGZlt{}/ciudad\PYGZus{}procedencia\PYGZgt{}}
        \PYG{n+nt}{\PYGZlt{}/futuro\PYGZgt{}}
        \PYG{n+nt}{\PYGZlt{}divisa} \PYG{n+na}{precio=}\PYG{l+s}{\PYGZdq{}183\PYGZdq{}}\PYG{n+nt}{\PYGZgt{}}
                \PYG{n+nt}{\PYGZlt{}nombre\PYGZus{}divisa}\PYG{n+nt}{\PYGZgt{}}Libra esterlina\PYG{n+nt}{\PYGZlt{}/nombre\PYGZus{}divisa\PYGZgt{}}
                \PYG{n+nt}{\PYGZlt{}tipo\PYGZus{}de\PYGZus{}cambio}\PYG{n+nt}{\PYGZgt{}}2.7:1 euros\PYG{n+nt}{\PYGZlt{}/tipo\PYGZus{}de\PYGZus{}cambio\PYGZgt{}}
                \PYG{n+nt}{\PYGZlt{}tipo\PYGZus{}de\PYGZus{}cambio}\PYG{n+nt}{\PYGZgt{}}1:0.87 dólares\PYG{n+nt}{\PYGZlt{}/tipo\PYGZus{}de\PYGZus{}cambio\PYGZgt{}}
                \PYG{n+nt}{\PYGZlt{}ciudad\PYGZus{}procedencia}\PYG{n+nt}{\PYGZgt{}}
                        \PYG{n+nt}{\PYGZlt{}madrid}\PYG{n+nt}{/\PYGZgt{}}
                \PYG{n+nt}{\PYGZlt{}/ciudad\PYGZus{}procedencia\PYGZgt{}}
        \PYG{n+nt}{\PYGZlt{}/divisa\PYGZgt{}}
        \PYG{n+nt}{\PYGZlt{}bono} \PYG{n+na}{precio=}\PYG{l+s}{\PYGZdq{}10000\PYGZdq{}} \PYG{n+na}{estable=}\PYG{l+s}{\PYGZdq{}si\PYGZdq{}}\PYG{n+nt}{\PYGZgt{}}
                \PYG{n+nt}{\PYGZlt{}pais\PYGZus{}de\PYGZus{}procedencia}\PYG{n+nt}{\PYGZgt{}}
                        Islandia
                \PYG{n+nt}{\PYGZlt{}/pais\PYGZus{}de\PYGZus{}procedencia\PYGZgt{}}
                \PYG{n+nt}{\PYGZlt{}valor\PYGZus{}deseado}\PYG{n+nt}{\PYGZgt{}}9980\PYG{n+nt}{\PYGZlt{}/valor\PYGZus{}deseado\PYGZgt{}}
                \PYG{n+nt}{\PYGZlt{}valor\PYGZus{}minimo}\PYG{n+nt}{\PYGZgt{}}9950\PYG{n+nt}{\PYGZlt{}/valor\PYGZus{}minimo\PYGZgt{}}
                \PYG{n+nt}{\PYGZlt{}valor\PYGZus{}maximo}\PYG{n+nt}{\PYGZgt{}}10020\PYG{n+nt}{\PYGZlt{}/valor\PYGZus{}maximo\PYGZgt{}}
                \PYG{n+nt}{\PYGZlt{}ciudad\PYGZus{}procedencia}\PYG{n+nt}{\PYGZgt{}}
                        \PYG{n+nt}{\PYGZlt{}tokio}\PYG{n+nt}{/\PYGZgt{}}
                \PYG{n+nt}{\PYGZlt{}/ciudad\PYGZus{}procedencia\PYGZgt{}}
        \PYG{n+nt}{\PYGZlt{}/bono\PYGZgt{}}
        \PYG{n+nt}{\PYGZlt{}letra} \PYG{n+na}{precio=}\PYG{l+s}{\PYGZdq{}45020\PYGZdq{}}\PYG{n+nt}{\PYGZgt{}}
                \PYG{n+nt}{\PYGZlt{}tipo\PYGZus{}de\PYGZus{}interes}\PYG{n+nt}{\PYGZgt{}}4.54\PYGZpc{}\PYG{n+nt}{\PYGZlt{}/tipo\PYGZus{}de\PYGZus{}interes\PYGZgt{}}
                \PYG{n+nt}{\PYGZlt{}pais\PYGZus{}emisor}\PYG{n+nt}{\PYGZgt{}}
                        \PYG{n+nt}{\PYGZlt{}espania}\PYG{n+nt}{/\PYGZgt{}}
                \PYG{n+nt}{\PYGZlt{}/pais\PYGZus{}emisor\PYGZgt{}}
                \PYG{n+nt}{\PYGZlt{}ciudad\PYGZus{}procedencia}\PYG{n+nt}{\PYGZgt{}}
                        \PYG{n+nt}{\PYGZlt{}madrid}\PYG{n+nt}{/\PYGZgt{}}
                \PYG{n+nt}{\PYGZlt{}/ciudad\PYGZus{}procedencia\PYGZgt{}}
        \PYG{n+nt}{\PYGZlt{}/letra\PYGZgt{}}
\PYG{n+nt}{\PYGZlt{}/listado\PYGZgt{}}
\end{sphinxVerbatim}


\section{Ejercicio}
\label{\detokenize{tema5:id3}}
La Seguridad Social necesita un formato de intercambio unificado para distribuir la información personal de los afiliados.
\begin{itemize}
\item {} 
Todo archivo XML contiene un listado de uno o mas afiliados

\item {} 
Todo afiliado tiene los siguientes elementos:
\begin{itemize}
\item {} 
DNI o NIE

\item {} 
Nombre

\item {} 
Apellidos

\item {} 
Situación laboral: que tiene que ser una y solo una de entre estas posibilidades: «en\_paro», «en\_activo», «jubilado», «edad\_no\_laboral»

\item {} 
Fecha de nacimiento: que se desglosa en los elementos obligatorios día, mes y anio.

\item {} 
Listado de bajas: que indica las situaciones de baja laboral del empleado. Dicho listado consta de una repetición de 0 o más bajas:
\begin{itemize}
\item {} 
Una baja consta de tres elementos: causa (obligatoria), fecha de inicio (obligatorio) y fecha de final (optativa),

\end{itemize}

\item {} 
Listado de prestaciones cobradas: consta de 0 o más elementos prestación, donde se indicará la cantidad percibida (obligatorio), la fecha de inicio (obligatorio) y la fecha de final (obligatorio)

\end{itemize}

\end{itemize}


\section{Esquemas XML}
\label{\detokenize{tema5:esquemas-xml}}
Los esquemas XML son un mecanismo radicalmente distinto de crear reglas para validar ficheros XML. Se caracterizan por:
\begin{itemize}
\item {} 
Estar escritos en XML. Por lo tanto, las mismas bibliotecas que permiten procesar ficheros XML de datos permitirían procesar ficheros XML de reglas.

\item {} 
Son mucho más potentes: ofrecen soporte a tipos de datos con comprobación de si el contenido de una etiqueta es de tipo \sphinxcode{integer}, \sphinxcode{date} o de otros tipos. También se permite añadir restricciones como indicar valores mínimo y máximo para un número o determinar el patrón que debe seguir una cadena válida

\item {} 
Ofrecen la posibilidad de usar \sphinxstyleemphasis{espacios de nombres}. Los espacios de nombres son similares a los paquetes Java: permiten a personas distintas el definir etiquetas con el mismo nombre pudiendo luego distinguir etiquetas iguales en función del espacio de nombres que importemos.

\end{itemize}


\subsection{Un ejemplo}
\label{\detokenize{tema5:un-ejemplo}}
Supongamos que deseamos tener ficheros XML con un solo elemento llamado \sphinxcode{\textless{}cantidad\textgreater{}} que debe tener dentro un número.

\begin{sphinxVerbatim}[commandchars=\\\{\}]
\PYG{n+nt}{\PYGZlt{}cantidad}\PYG{n+nt}{\PYGZgt{}}20\PYG{n+nt}{\PYGZlt{}/cantidad\PYGZgt{}}
\end{sphinxVerbatim}

Un posible esquema sería el siguiente:

\begin{sphinxVerbatim}[commandchars=\\\{\}]
\PYG{n+nt}{\PYGZlt{}xsd:schema} \PYG{n+na}{xmlns:xsd=}\PYG{l+s}{\PYGZdq{}http://www.w3.org/2001/XMLSchema\PYGZdq{}}\PYG{n+nt}{\PYGZgt{}}
   \PYG{n+nt}{\PYGZlt{}xsd:element} \PYG{n+na}{name=}\PYG{l+s}{\PYGZdq{}cantidad\PYGZdq{}} \PYG{n+na}{type=}\PYG{l+s}{\PYGZdq{}xsd:integer\PYGZdq{}}\PYG{n+nt}{/\PYGZgt{}}
\PYG{n+nt}{\PYGZlt{}/xsd:schema\PYGZgt{}}
\end{sphinxVerbatim}

¿Qué contiene este fichero?
\begin{enumerate}
\item {} 
En primer lugar se indica que este fichero va a usar unas etiquetas ya definidas en un espacio de nombres (o XML Namespace, de ahí \sphinxcode{xmlns}). Esa definición se hace en el espacio de nombres que aparece en la URL. Nuestro validador no descargará nada, esa URL es oficial y todos los validadores la conocen. Las etiquetas de ese espacio de nombres van a usar un prefijo que en este caso será \sphinxcode{xsd}. Nótese que el prefijo puede ser como queramos (podría ser «abcd» o «zztop»), pero la costumbre es usar \sphinxcode{xsd}.

\item {} 
Se indica que habrá un solo elemento y que el tipo de ese elemento es \sphinxcode{\textless{}xsd:integer\textgreater{}}. Es decir, un entero básico.

\end{enumerate}

Si probamos el fichero de esquema con el fichero de datos que hemos indicado veremos que efectivamente el fichero XML de datos es válido. Sin embargo, si en lugar de una cantidad incluyésemos una cadena, veríamos que el fichero \sphinxstylestrong{no se validaría}


\subsection{Tipos de datos básicos}
\label{\detokenize{tema5:tipos-de-datos-basicos}}
Podemos usar los siguientes tipos de datos:
\begin{itemize}
\item {} 
\sphinxcode{xsd:byte}: entero de 8 bits.

\item {} 
\sphinxcode{xsd:short}: entero de 16 bits

\item {} 
\sphinxcode{xsd:int}: número entero de 32 bits.

\item {} 
\sphinxcode{xsd:long}: entero de 64 bits.

\item {} 
\sphinxcode{xsd:integer}: número entero sin límite de capacidad.

\item {} 
\sphinxcode{xsd:unsignedByte}: entero de 8 bits sin signo.

\item {} 
\sphinxcode{xsd:unsignedShort}: entero de 16 bits sin signo.

\item {} 
\sphinxcode{xsd:unsignedInt}: entero de 32 bits sin signo.

\item {} 
\sphinxcode{xsd:unsignedLong}: entero de 64 bits sin signo.

\item {} 
\sphinxcode{xsd:string}: cadena de caracteres en la que los espacios en blanco se respetan.

\item {} 
\sphinxcode{xsd:normalizedString}: cadena de caracteres en la que los espacios en blanco no se respetan y se reemplazarán secuencias largas de espacios o fines de línea por un solo espacio.

\item {} 
\sphinxcode{xsd:date}: permite almacenar fechas que deben ir \sphinxstylestrong{obligatoriamente} en formato AAAA-MM-DD (4 digitos para el año, seguidos de un guión, seguido de dos dígitos para el mes, seguidos de un guión, seguidos de dos dígitos para el día del mes)

\item {} 
\sphinxcode{xsd:time}: para almacenar horas en formato HH:MM:SS.C

\item {} 
\sphinxcode{xsd:datetime}: mezcla la fecha y la hora separando ambos campos con una T mayúscula. Esto permitiría almacenar \sphinxcode{2020-09-22T10:40:22.6}.

\item {} 
\sphinxcode{xsd:duration}. Para indicar períodos. Se debe empezar con «P» y luego indicar el número de años, meses, días, minutos o segundos. Por ejemplo «P1Y4M21DT8H» indica un período de 1 año, 4 meses, 21 días y 8 horas. Se aceptan períodos negativos poniendo -P en lugar de P.

\item {} 
\sphinxcode{xsd:boolean}: acepta solo valores «true» y «false».

\item {} 
\sphinxcode{xsd:anyURI}: acepta URIs.

\item {} 
\sphinxcode{xsd:anyType}: es como la clase \sphinxcode{Object} en Java. Será el tipo del cual heredaremos cuando no vayamos a usar ningún tipo especial como tipo padre.

\end{itemize}

La figura siguiente (tomada de la web del W3C) ilustra todos los tipos así como sus relaciones de herencia:

\begin{figure}[htbp]
\centering
\capstart

\noindent\sphinxincludegraphics{{tipos_xml_schema}.png}
\caption{Tipos en los XML Schemas}\label{\detokenize{tema5:id5}}\end{figure}


\subsection{Derivaciones}
\label{\detokenize{tema5:derivaciones}}
Prácticamente en cualquier esquema XML crearemos tipos nuevos (por establecer un símil es como si programásemos clases Java). Todos nuestros tipos tienen que heredar de otros tipos pero a la hora de «heredar» tenemos más posibilidades que en Java (dondo solo tenemos el «extends»). En concreto podemos heredar de 4 formas:
\begin{enumerate}
\item {} 
Poniendo restricciones (\sphinxcode{restriction}). Consiste en tomar un tipo y crear otro nuevo en el que no se puede poner cualquier valor.

\item {} 
Extendiendo un tipo (\sphinxcode{extension}). Se toma un tipo y se crea uno nuevo añadiendo cosas a los posibles valores que pueda tomar el tipo inicial.

\item {} 
Haciendo listas (\sphinxcode{lists}). Es como crear vectores en Java.

\item {} 
Juntando otros tipos para crear tipos complejos (\sphinxcode{union}). Es como crear clases Java en las que añadimos atributos de tipo \sphinxcode{int}, \sphinxcode{String}, etc…

\end{enumerate}

En general, las dos derivaciones más usadas con diferencia son las restricciones y las extensiones, que se comentan por separado en los puntos siguientes.


\subsection{Tipos simples y complejos}
\label{\detokenize{tema5:tipos-simples-y-complejos}}
Todo elemento de un esquema debe ser de uno de estos dos tipos.
\begin{itemize}
\item {} 
Un elemento es de tipo simple si no permite dentro ni elementos hijo ni atributos.

\item {} 
Un elemento es tipo complejo si permite tener dentro otras cosas (que veremos en seguida). Un tipo complejo puede a su vez tener contenido simple o contenido complejo:
\begin{itemize}
\item {} 
Los que son de contenido simple no permiten tener dentro elementos hijo pero sí permiten atributos.

\item {} 
Los que son de contenido complejo sí permiten tener dentro elementos hijo y atributos.

\end{itemize}

\end{itemize}

Así, por ejemplo un tipo simple que no lleve ninguna restricción se puede indicar con el campo \sphinxcode{type} de un \sphinxcode{element} como hacíamos antes:

\begin{sphinxVerbatim}[commandchars=\\\{\}]
\PYG{n+nt}{\PYGZlt{}xsd:schema} \PYG{n+na}{xmlns:xsd=}\PYG{l+s}{\PYGZdq{}http://www.w3.org/2001/XMLSchema\PYGZdq{}}\PYG{n+nt}{\PYGZgt{}}
   \PYG{n+nt}{\PYGZlt{}xsd:element} \PYG{n+na}{name=}\PYG{l+s}{\PYGZdq{}cantidad\PYGZdq{}} \PYG{n+na}{type=}\PYG{l+s}{\PYGZdq{}xsd:integer\PYGZdq{}}\PYG{n+nt}{/\PYGZgt{}}
\PYG{n+nt}{\PYGZlt{}/xsd:schema\PYGZgt{}}
\end{sphinxVerbatim}

Sin embargo, si queremos indicar alguna restricción adicional ya no podremos usar el atributo \sphinxcode{type}. Deberemos reescribir nuestro esquema así:

\begin{sphinxVerbatim}[commandchars=\\\{\}]
\PYG{n+nt}{\PYGZlt{}xsd:schema} \PYG{n+na}{xmlns:xsd=}\PYG{l+s}{\PYGZdq{}http://www.w3.org/2001/XMLSchema\PYGZdq{}}\PYG{n+nt}{\PYGZgt{}}
   \PYG{n+nt}{\PYGZlt{}xsd:simpleType}\PYG{n+nt}{\PYGZgt{}}
    Aquí irán las restricciones, que hemos omitido por ahora.
   \PYG{n+nt}{\PYGZlt{}/xsd:simpleType\PYGZgt{}}
\PYG{n+nt}{\PYGZlt{}/xsd:schema\PYGZgt{}}
\end{sphinxVerbatim}


\subsection{Ejercicio:edad de los trabajadores}
\label{\detokenize{tema5:ejercicio-edad-de-los-trabajadores}}
Se desea crear un esquema que permita validar la edad de un trabajador, que debe tener un valor entero de entre 16 y 65.

Por ejemplo, este XML debería validarse:

\begin{sphinxVerbatim}[commandchars=\\\{\}]
\PYG{n+nt}{\PYGZlt{}edad}\PYG{n+nt}{\PYGZgt{}}28\PYG{n+nt}{\PYGZlt{}/edad\PYGZgt{}}
\end{sphinxVerbatim}

Pero este no debería validarse:

\begin{sphinxVerbatim}[commandchars=\\\{\}]
\PYG{n+nt}{\PYGZlt{}edad}\PYG{n+nt}{\PYGZgt{}}\PYGZhy{}3\PYG{n+nt}{\PYGZlt{}/edad\PYGZgt{}}
\end{sphinxVerbatim}

La solución podría ser algo así:

\begin{sphinxVerbatim}[commandchars=\\\{\}]
\PYG{n+nt}{\PYGZlt{}xsd:schema}
 \PYG{n+na}{xmlns:xsd=}\PYG{l+s}{\PYGZdq{}http://www.w3.org/2001/XMLSchema\PYGZdq{}}\PYG{n+nt}{\PYGZgt{}}
    \PYG{n+nt}{\PYGZlt{}xsd:element} \PYG{n+na}{name=}\PYG{l+s}{\PYGZdq{}edad\PYGZdq{}}
                 \PYG{n+na}{type=}\PYG{l+s}{\PYGZdq{}tipoEdad\PYGZdq{}}\PYG{n+nt}{/\PYGZgt{}}
    \PYG{n+nt}{\PYGZlt{}xsd:simpleType} \PYG{n+na}{name=}\PYG{l+s}{\PYGZdq{}tipoEdad\PYGZdq{}}\PYG{n+nt}{\PYGZgt{}}
        \PYG{n+nt}{\PYGZlt{}xsd:restriction} \PYG{n+na}{base=}\PYG{l+s}{\PYGZdq{}xsd:integer\PYGZdq{}}\PYG{n+nt}{\PYGZgt{}}
            \PYG{n+nt}{\PYGZlt{}xsd:minInclusive} \PYG{n+na}{value=}\PYG{l+s}{\PYGZdq{}16\PYGZdq{}}\PYG{n+nt}{/\PYGZgt{}}
            \PYG{n+nt}{\PYGZlt{}xsd:maxInclusive} \PYG{n+na}{value=}\PYG{l+s}{\PYGZdq{}65\PYGZdq{}}\PYG{n+nt}{/\PYGZgt{}}
        \PYG{n+nt}{\PYGZlt{}/xsd:restriction\PYGZgt{}}
    \PYG{n+nt}{\PYGZlt{}/xsd:simpleType\PYGZgt{}}
\PYG{n+nt}{\PYGZlt{}/xsd:schema\PYGZgt{}}
\end{sphinxVerbatim}


\subsection{Ejercicio: peso de productos}
\label{\detokenize{tema5:ejercicio-peso-de-productos}}
Se desea crear un esquema que permita validar un elemento peso, que puede tener un valor de entre 0 y 1000 pero aceptando valores con decimales, como por ejemplo 28.88

Una posible solución sería:

\begin{sphinxVerbatim}[commandchars=\\\{\}]
\PYG{n+nt}{\PYGZlt{}xsd:schema} \PYG{n+na}{xmlns:xsd=}\PYG{l+s}{\PYGZdq{}http://www.w3.org/2001/XMLSchema\PYGZdq{}}\PYG{n+nt}{\PYGZgt{}}
  \PYG{n+nt}{\PYGZlt{}xsd:element} \PYG{n+na}{name=}\PYG{l+s}{\PYGZdq{}peso\PYGZdq{}} \PYG{n+na}{type=}\PYG{l+s}{\PYGZdq{}tipoPeso\PYGZdq{}}\PYG{n+nt}{/\PYGZgt{}}
  \PYG{n+nt}{\PYGZlt{}xsd:simpleType} \PYG{n+na}{name=}\PYG{l+s}{\PYGZdq{}tipoPeso\PYGZdq{}}\PYG{n+nt}{\PYGZgt{}}
    \PYG{n+nt}{\PYGZlt{}xsd:restriction} \PYG{n+na}{base=}\PYG{l+s}{\PYGZdq{}xsd:decimal\PYGZdq{}}\PYG{n+nt}{\PYGZgt{}}
      \PYG{n+nt}{\PYGZlt{}xsd:minInclusive} \PYG{n+na}{value=}\PYG{l+s}{\PYGZdq{}0\PYGZdq{}}\PYG{n+nt}{/\PYGZgt{}}
      \PYG{n+nt}{\PYGZlt{}xsd:maxInclusive} \PYG{n+na}{value=}\PYG{l+s}{\PYGZdq{}1000\PYGZdq{}}\PYG{n+nt}{/\PYGZgt{}}
    \PYG{n+nt}{\PYGZlt{}/xsd:restriction\PYGZgt{}}
  \PYG{n+nt}{\PYGZlt{}/xsd:simpleType\PYGZgt{}}
\PYG{n+nt}{\PYGZlt{}/xsd:schema\PYGZgt{}}
\end{sphinxVerbatim}


\subsection{Ejercicio: pagos validados}
\label{\detokenize{tema5:ejercicio-pagos-validados}}
Crear un esquema que permita validar un elemento \sphinxcode{pago} en el cual puede haber cantidades enteras de entre 0 y 3000 euros.

\begin{sphinxVerbatim}[commandchars=\\\{\}]
\PYG{n+nt}{\PYGZlt{}xsd:schema}
    \PYG{n+na}{xmlns:xsd=}\PYG{l+s}{\PYGZdq{}http://www.w3.org/2001/XMLSchema\PYGZdq{}}\PYG{n+nt}{\PYGZgt{}}
  \PYG{n+nt}{\PYGZlt{}xsd:element} \PYG{n+na}{name=}\PYG{l+s}{\PYGZdq{}pago\PYGZdq{}} \PYG{n+na}{type=}\PYG{l+s}{\PYGZdq{}tipoPago\PYGZdq{}}\PYG{n+nt}{/\PYGZgt{}}
  \PYG{n+nt}{\PYGZlt{}xsd:simpleType} \PYG{n+na}{name=}\PYG{l+s}{\PYGZdq{}tipoPago\PYGZdq{}}\PYG{n+nt}{\PYGZgt{}}
    \PYG{n+nt}{\PYGZlt{}xsd:restriction} \PYG{n+na}{base=}\PYG{l+s}{\PYGZdq{}xsd:integer\PYGZdq{}}\PYG{n+nt}{\PYGZgt{}}
      \PYG{n+nt}{\PYGZlt{}xsd:minInclusive} \PYG{n+na}{value=}\PYG{l+s}{\PYGZdq{}0\PYGZdq{}}\PYG{n+nt}{/\PYGZgt{}}
      \PYG{n+nt}{\PYGZlt{}xsd:maxInclusive} \PYG{n+na}{value=}\PYG{l+s}{\PYGZdq{}3000\PYGZdq{}}\PYG{n+nt}{/\PYGZgt{}}
    \PYG{n+nt}{\PYGZlt{}/xsd:restriction\PYGZgt{}}
  \PYG{n+nt}{\PYGZlt{}/xsd:simpleType\PYGZgt{}}
\PYG{n+nt}{\PYGZlt{}/xsd:schema\PYGZgt{}}
\end{sphinxVerbatim}


\subsection{Ejercicio: validación de DNIs}
\label{\detokenize{tema5:ejercicio-validacion-de-dnis}}
Crear un esquema que permita validar un único elemento \sphinxcode{dni} que valide el patrón de 7-8 cifras + letra que suelen tener los DNI en España:

\begin{sphinxVerbatim}[commandchars=\\\{\}]
\PYG{n+nt}{\PYGZlt{}xsd:schema}
    \PYG{n+na}{xmlns:xsd=}\PYG{l+s}{\PYGZdq{}http://www.w3.org/2001/XMLSchema\PYGZdq{}}\PYG{n+nt}{\PYGZgt{}}
  \PYG{n+nt}{\PYGZlt{}xsd:element} \PYG{n+na}{name=}\PYG{l+s}{\PYGZdq{}dni\PYGZdq{}} \PYG{n+na}{type=}\PYG{l+s}{\PYGZdq{}tipoDNI\PYGZdq{}}\PYG{n+nt}{/\PYGZgt{}}
  \PYG{n+nt}{\PYGZlt{}xsd:simpleType} \PYG{n+na}{name=}\PYG{l+s}{\PYGZdq{}tipoDNI\PYGZdq{}}\PYG{n+nt}{\PYGZgt{}}
    \PYG{n+nt}{\PYGZlt{}xsd:restriction} \PYG{n+na}{base=}\PYG{l+s}{\PYGZdq{}xsd:string\PYGZdq{}}\PYG{n+nt}{\PYGZgt{}}
      \PYG{n+nt}{\PYGZlt{}xsd:pattern} \PYG{n+na}{value=}\PYG{l+s}{\PYGZdq{}[0\PYGZhy{}9]\PYGZob{}7,8\PYGZcb{}[A\PYGZhy{}Z]\PYGZdq{}}\PYG{n+nt}{/\PYGZgt{}}
    \PYG{n+nt}{\PYGZlt{}/xsd:restriction\PYGZgt{}}
  \PYG{n+nt}{\PYGZlt{}/xsd:simpleType\PYGZgt{}}
\PYG{n+nt}{\PYGZlt{}/xsd:schema\PYGZgt{}}
\end{sphinxVerbatim}


\subsection{Uniendo la herencia y el sistema de tipos}
\label{\detokenize{tema5:uniendo-la-herencia-y-el-sistema-de-tipos}}
Llegados a este punto ocurre lo siguiente:
\begin{itemize}
\item {} 
Por un lado tenemos que especificar si nuestros tipos serán simples o complejos (los cuales a su vez pueden ser complejos con contenido simple o complejos con contenido complejo).

\item {} 
Por otro lado se puede hacer herencia ampliando cosas (extensión) o reduciendo cosas (restricciones a los valores).

\end{itemize}

Se deduce por tanto que no podemos aplicar todas las «herencias» a todos los tipos:
\begin{enumerate}
\item {} 
Los tipos simples no pueden tener atributos ni subelementos, por lo tanto \sphinxstylestrong{les podremos aplicar restricciones pero nunca la extensión}.

\item {} 
Los tipos complejos (independientemente del tipo de contenido) sí pueden tener otras cosas dentro por lo que \sphinxstylestrong{les podremos aplicar tanto restricciones como extensiones}.

\end{enumerate}


\subsection{Restricciones}
\label{\detokenize{tema5:restricciones}}
Como se ha dicho anteriormente la forma más común de trabajar es crear tipos que en unos casos aplicarán modificaciones en los tipos ya sea añadiendo cosas o restringiendo posibilidades. En este apartado se verá como aplicar restricciones.

\sphinxstylestrong{Si queremos aplicar restricciones para un tipo simple las posibles restricciones son:}
\begin{itemize}
\item {} 
\sphinxcode{minInclusive} para indicar el menor valor numérico permitido.

\item {} 
\sphinxcode{maxInclusive} para indicar el mayor valor numérico permitido.

\item {} 
\sphinxcode{minExclusive} para indicar el menor valor numérico que ya no estaría permitido.

\item {} 
\sphinxcode{maxExclusive} para indicar el mayor valor numérico que ya no estaría permitido.

\item {} 
\sphinxcode{totalDigits} para indicar cuantas posibles cifras se permiten.

\item {} 
\sphinxcode{fractionDigits} para indicar cuantas posibles cifras decimales se permiten.

\item {} 
\sphinxcode{length} para indicar la longitud exacta de una cadena.

\item {} 
\sphinxcode{minLength} para indicar la longitud mínima de una cadena.

\item {} 
\sphinxcode{maxLength} para indicar la longitud máxima de una cadena.

\item {} 
\sphinxcode{enumeration} para indicar los valores aceptados por una cadena.

\item {} 
\sphinxcode{pattern} para indicar la estructura aceptada por una cadena.

\end{itemize}

\sphinxstylestrong{Si queremos aplicar restricciones para un tipo complejo con contenido las posibles restricciones son las mismas de antes, pero además podemos añadir el elemento \textless{}attribute\textgreater{} así como las siguientes.}
\begin{itemize}
\item {} 
\sphinxcode{sequence} para indicar una secuencia de elementos

\item {} 
\sphinxcode{choice} para indicar que se debe elegir un elemento de entre los que aparecen.

\end{itemize}


\subsection{Atributos}
\label{\detokenize{tema5:id4}}
En primer lugar es muy importante recordar que \sphinxstylestrong{si queremos que un elemento tenga atributos entonces ya no
se puede considerar que sea de tipo simple. Se debe usar FORZOSAMENTE un complexType}. Por otro lado en los XML Schema todos los atributos \sphinxstylestrong{son siempre opcionales, si queremos hacerlos obligatorios habrá que añadir un «required».}

Un atributo se define de la siguiente manera:

\begin{sphinxVerbatim}[commandchars=\\\{\}]
\PYG{n+nt}{\PYGZlt{}xsd:attribute} \PYG{n+na}{name=}\PYG{l+s}{\PYGZdq{}fechanacimiento\PYGZdq{}} \PYG{n+na}{type=}\PYG{l+s}{\PYGZdq{}xsd:date\PYGZdq{}} \PYG{n+na}{use=}\PYG{l+s}{\PYGZdq{}required\PYGZdq{}}\PYG{n+nt}{/\PYGZgt{}}
\end{sphinxVerbatim}

Esto define un atributo llamado \sphinxcode{nombre} que aceptará solo fechas como valores válidos y que además es obligatorio poner siempre.


\section{Ejercicios de XML Schemas}
\label{\detokenize{tema5:ejercicios-de-xml-schemas}}

\subsection{Cantidades limitades}
\label{\detokenize{tema5:cantidades-limitades}}
Crear un esquema que permita verificar algo como lo siguiente:

\begin{sphinxVerbatim}[commandchars=\\\{\}]
\PYG{n+nt}{\PYGZlt{}cantidad}\PYG{n+nt}{\PYGZgt{}}20\PYG{n+nt}{\PYGZlt{}/cantidad\PYGZgt{}}
\end{sphinxVerbatim}

Se necesita que la cantidad tenga solo valores aceptables entre -30 y +30.


\subsection{Solución a las cantidades limitadas}
\label{\detokenize{tema5:solucion-a-las-cantidades-limitadas}}
La primera pregunta que debemos hacernos es ¿necesitamos crear un tipo simple o uno complejo?. Dado que nuestro único elemento no tiene subelementos ni atributos dentro podemos afirmar que solo necesitamos un tipo simple.

Como aparentemente nuestro tipo necesita usar solo valores numéricos y además son muy pequeños nos vamos a limitar a usar un \sphinxcode{short}. Sobre ese \sphinxcode{short} pondremos una restriccion que permita indicar los valores mínimo y máximo.

\begin{sphinxVerbatim}[commandchars=\\\{\}]
\PYG{n+nt}{\PYGZlt{}xs:schema} \PYG{n+na}{xmlns:xs=}\PYG{l+s}{\PYGZdq{}http://www.w3.org/2001/XMLSchema\PYGZdq{}}\PYG{n+nt}{\PYGZgt{}}
    \PYG{n+nt}{\PYGZlt{}xs:element} \PYG{n+na}{name=}\PYG{l+s}{\PYGZdq{}cantidad\PYGZdq{}}\PYG{n+nt}{\PYGZgt{}}
        \PYG{n+nt}{\PYGZlt{}xs:simpleType}\PYG{n+nt}{\PYGZgt{}}
            \PYG{n+nt}{\PYGZlt{}xs:restriction} \PYG{n+na}{base=}\PYG{l+s}{\PYGZdq{}xs:short\PYGZdq{}}\PYG{n+nt}{\PYGZgt{}}
                \PYG{n+nt}{\PYGZlt{}xs:minInclusive} \PYG{n+na}{value=}\PYG{l+s}{\PYGZdq{}\PYGZhy{}30\PYGZdq{}}\PYG{n+nt}{/\PYGZgt{}}
                \PYG{n+nt}{\PYGZlt{}xs:maxInclusive} \PYG{n+na}{value=}\PYG{l+s}{\PYGZdq{}30\PYGZdq{}}\PYG{n+nt}{/\PYGZgt{}}
            \PYG{n+nt}{\PYGZlt{}/xs:restriction\PYGZgt{}}
        \PYG{n+nt}{\PYGZlt{}/xs:simpleType\PYGZgt{}}
    \PYG{n+nt}{\PYGZlt{}/xs:element\PYGZgt{}}
\PYG{n+nt}{\PYGZlt{}/xs:schema\PYGZgt{}}
\end{sphinxVerbatim}

Este esquema dice que el elemento raíz debe ser \sphinxcode{cantidad}. Luego indica que es un tipo simple y dentro de él indica que se va a establecer una restricción teniendo en mente que se va a «heredar» del tipo \sphinxcode{short}. En concreto se van a poner dos restricciones, una que el valor mínimo debe ser -30 y otra que el valor máximo debe ser 30.

Existe una alternativa más recomendable, que es separar los elementos de los tipos. De esa manera, se pueden «reutilizar» las definiciones de tipos.

\begin{sphinxVerbatim}[commandchars=\\\{\}]
\PYG{n+nt}{\PYGZlt{}xs:schema} \PYG{n+na}{xmlns:xs=}\PYG{l+s}{\PYGZdq{}http://www.w3.org/2001/XMLSchema\PYGZdq{}}\PYG{n+nt}{\PYGZgt{}}
    \PYG{n+nt}{\PYGZlt{}xs:element} \PYG{n+na}{name=}\PYG{l+s}{\PYGZdq{}cantidad\PYGZdq{}} \PYG{n+na}{type=}\PYG{l+s}{\PYGZdq{}tipoCantidades\PYGZdq{}}\PYG{n+nt}{\PYGZgt{}}
    \PYG{n+nt}{\PYGZlt{}/xs:element\PYGZgt{}}
    \PYG{n+nt}{\PYGZlt{}xs:simpleType} \PYG{n+na}{name=}\PYG{l+s}{\PYGZdq{}tipoCantidades\PYGZdq{}}\PYG{n+nt}{\PYGZgt{}}
            \PYG{n+nt}{\PYGZlt{}xs:restriction} \PYG{n+na}{base=}\PYG{l+s}{\PYGZdq{}xs:short\PYGZdq{}}\PYG{n+nt}{\PYGZgt{}}
                \PYG{n+nt}{\PYGZlt{}xs:minInclusive} \PYG{n+na}{value=}\PYG{l+s}{\PYGZdq{}\PYGZhy{}30\PYGZdq{}}\PYG{n+nt}{/\PYGZgt{}}
                \PYG{n+nt}{\PYGZlt{}xs:maxInclusive} \PYG{n+na}{value=}\PYG{l+s}{\PYGZdq{}30\PYGZdq{}}\PYG{n+nt}{/\PYGZgt{}}
            \PYG{n+nt}{\PYGZlt{}/xs:restriction\PYGZgt{}}
        \PYG{n+nt}{\PYGZlt{}/xs:simpleType\PYGZgt{}}
\PYG{n+nt}{\PYGZlt{}/xs:schema\PYGZgt{}}
\end{sphinxVerbatim}

Obsérvese que hemos puesto el tipo por separado y le hemos dado el nombre \sphinxcode{tipoCantidades}. El elemento raíz tiene su nombre y su tipo en la misma línea.


\subsection{Cantidades limitadas con atributo divisa}
\label{\detokenize{tema5:cantidades-limitadas-con-atributo-divisa}}
Se desea crear un esquema para validar XML en los que haya un solo elemento raíz llamado cantidad en el que se debe poner siempre un atributo «divisa» que indique en qué moneda está una cierta cantidad. El atributo divisa siempre será una cadena y la cantidad siempre será un tipo numérico que acepte decimales (por ejemplo \sphinxcode{float}). El esquema debe validar los archivos siguientes:

\begin{sphinxVerbatim}[commandchars=\\\{\}]
\PYG{n+nt}{\PYGZlt{}cantidad} \PYG{n+na}{divisa=}\PYG{l+s}{\PYGZdq{}euro\PYGZdq{}}\PYG{n+nt}{\PYGZgt{}}20\PYG{n+nt}{\PYGZlt{}/cantidad\PYGZgt{}}
\end{sphinxVerbatim}

\begin{sphinxVerbatim}[commandchars=\\\{\}]
\PYG{n+nt}{\PYGZlt{}cantidad} \PYG{n+na}{divisa=}\PYG{l+s}{\PYGZdq{}dolar\PYGZdq{}}\PYG{n+nt}{\PYGZgt{}}18.32\PYG{n+nt}{\PYGZlt{}/cantidad\PYGZgt{}}
\end{sphinxVerbatim}

Pero no debe validar ninguno de los siguientes:

\begin{sphinxVerbatim}[commandchars=\\\{\}]
\PYG{n+nt}{\PYGZlt{}cantidad}\PYG{n+nt}{\PYGZgt{}}20\PYG{n+nt}{\PYGZlt{}/cantidad\PYGZgt{}}
\end{sphinxVerbatim}

\begin{sphinxVerbatim}[commandchars=\\\{\}]
\PYG{n+nt}{\PYGZlt{}cantidad} \PYG{n+na}{divisa=}\PYG{l+s}{\PYGZdq{}dolar\PYGZdq{}}\PYG{n+nt}{\PYGZgt{}}abc\PYG{n+nt}{\PYGZlt{}/cantidad\PYGZgt{}}
\end{sphinxVerbatim}


\subsection{Solución a las cantidades limitadas con atributo divisa}
\label{\detokenize{tema5:solucion-a-las-cantidades-limitadas-con-atributo-divisa}}
Crearemos un tipo llamado «tipoCantidad». Dicho tipo \sphinxstyleemphasis{ya no puede ser un simpleType ya que necesitamos que haya atributos}. Como no necesitamos que tenga dentro subelementos entonces este \sphinxcode{complexType} llevará dentro un \sphinxcode{simpleContent} (y no un \sphinxcode{complexContent}).

Aparte de eso, como queremos «ampliar» un elemento para que acepte tener dentro un atributo obligatorio «cantidad» usaremos una \sphinxcode{\textless{}extension\textgreater{}}. Así, el posible esquema sería este:

\begin{sphinxVerbatim}[commandchars=\\\{\}]
\PYG{n+nt}{\PYGZlt{}xsd:schema} \PYG{n+na}{xmlns:xsd=}\PYG{l+s}{\PYGZdq{}http://www.w3.org/2001/XMLSchema\PYGZdq{}}\PYG{n+nt}{\PYGZgt{}}
    \PYG{n+nt}{\PYGZlt{}xsd:element} \PYG{n+na}{name=}\PYG{l+s}{\PYGZdq{}cantidad\PYGZdq{}} \PYG{n+na}{type=}\PYG{l+s}{\PYGZdq{}tipoCantidad\PYGZdq{}}\PYG{n+nt}{/\PYGZgt{}}
    \PYG{n+nt}{\PYGZlt{}xsd:complexType} \PYG{n+na}{name=}\PYG{l+s}{\PYGZdq{}tipoCantidad\PYGZdq{}}\PYG{n+nt}{\PYGZgt{}}
        \PYG{n+nt}{\PYGZlt{}xsd:simpleContent}\PYG{n+nt}{\PYGZgt{}}
            \PYG{n+nt}{\PYGZlt{}xsd:extension} \PYG{n+na}{base=}\PYG{l+s}{\PYGZdq{}xsd:float\PYGZdq{}}\PYG{n+nt}{\PYGZgt{}}
                \PYG{n+nt}{\PYGZlt{}xsd:attribute} \PYG{n+na}{name=}\PYG{l+s}{\PYGZdq{}divisa\PYGZdq{}} \PYG{n+na}{type=}\PYG{l+s}{\PYGZdq{}xsd:string\PYGZdq{}} \PYG{n+na}{use=}\PYG{l+s}{\PYGZdq{}required\PYGZdq{}}\PYG{n+nt}{/\PYGZgt{}}
            \PYG{n+nt}{\PYGZlt{}/xsd:extension\PYGZgt{}}
        \PYG{n+nt}{\PYGZlt{}/xsd:simpleContent\PYGZgt{}}
    \PYG{n+nt}{\PYGZlt{}/xsd:complexType\PYGZgt{}}
\PYG{n+nt}{\PYGZlt{}/xsd:schema\PYGZgt{}}
\end{sphinxVerbatim}


\subsection{Cantidades limitadas con atributo divisa con solo ciertos valores}
\label{\detokenize{tema5:cantidades-limitadas-con-atributo-divisa-con-solo-ciertos-valores}}
Queremos ampliar el ejercicio anterior para evitar que ocurran errores como el siguiente:

\begin{sphinxVerbatim}[commandchars=\\\{\}]
\PYG{n+nt}{\PYGZlt{}cantidad} \PYG{n+na}{divisa=}\PYG{l+s}{\PYGZdq{}aaaa\PYGZdq{}}\PYG{n+nt}{\PYGZgt{}}18.32\PYG{n+nt}{\PYGZlt{}/cantidad\PYGZgt{}}
\end{sphinxVerbatim}

Vamos a indicar que el atributo solo puede tomar tres posibles valores: «euros», «dolares» y «yenes».


\subsection{Solución al atributo con solo ciertos valores}
\label{\detokenize{tema5:solucion-al-atributo-con-solo-ciertos-valores}}
Ahora tendremos que crear dos tipos. Uno para el elemento \sphinxcode{cantidad} y otro para el atributo \sphinxcode{divisa}. Llamaremos a estos tipos \sphinxcode{tipoCantidad} y \sphinxcode{tipoDivisa}.

La solución comentada puede encontrarse a continuación. Como puede verse, hemos includo comentarios. Pueden insertarse etiquetas \sphinxcode{annotation} que permiten incluir anotaciones de diversos tipos, siendo la más interesante la etiqueta \sphinxcode{documentation} que nos permite incluir comentarios.

\begin{sphinxVerbatim}[commandchars=\\\{\}]
\PYG{n+nt}{\PYGZlt{}xsd:schema} \PYG{n+na}{xmlns:xsd=}\PYG{l+s}{\PYGZdq{}http://www.w3.org/2001/XMLSchema\PYGZdq{}}\PYG{n+nt}{\PYGZgt{}}
    \PYG{n+nt}{\PYGZlt{}xsd:element} \PYG{n+na}{name=}\PYG{l+s}{\PYGZdq{}cantidad\PYGZdq{}} \PYG{n+na}{type=}\PYG{l+s}{\PYGZdq{}tipoCantidad\PYGZdq{}}\PYG{n+nt}{/\PYGZgt{}}
    \PYG{n+nt}{\PYGZlt{}xsd:annotation}\PYG{n+nt}{\PYGZgt{}}
        \PYG{n+nt}{\PYGZlt{}xsd:documentation}\PYG{n+nt}{\PYGZgt{}}
        A continuación creamos el tipo cantidad
        \PYG{n+nt}{\PYGZlt{}/xsd:documentation\PYGZgt{}}
    \PYG{n+nt}{\PYGZlt{}/xsd:annotation\PYGZgt{}}
    \PYG{n+nt}{\PYGZlt{}xsd:complexType} \PYG{n+na}{name=}\PYG{l+s}{\PYGZdq{}tipoCantidad\PYGZdq{}}\PYG{n+nt}{\PYGZgt{}}
        \PYG{n+nt}{\PYGZlt{}xsd:annotation}\PYG{n+nt}{\PYGZgt{}}
            \PYG{n+nt}{\PYGZlt{}xsd:documentation}\PYG{n+nt}{\PYGZgt{}}
            Como solo va a llevar atributos debemos
            usar un simpleContent
            \PYG{n+nt}{\PYGZlt{}/xsd:documentation\PYGZgt{}}
        \PYG{n+nt}{\PYGZlt{}/xsd:annotation\PYGZgt{}}
        \PYG{n+nt}{\PYGZlt{}xsd:simpleContent}\PYG{n+nt}{\PYGZgt{}}
            \PYG{n+nt}{\PYGZlt{}xsd:annotation}\PYG{n+nt}{\PYGZgt{}}
                \PYG{n+nt}{\PYGZlt{}xsd:documentation}\PYG{n+nt}{\PYGZgt{}}
                Como queremos \PYGZdq{}ampliar\PYGZdq{} un tipo/clase
                para que lleve atributos usaremos
                una extension
                \PYG{n+nt}{\PYGZlt{}/xsd:documentation\PYGZgt{}}
            \PYG{n+nt}{\PYGZlt{}/xsd:annotation\PYGZgt{}}
            \PYG{n+nt}{\PYGZlt{}xsd:extension} \PYG{n+na}{base=}\PYG{l+s}{\PYGZdq{}xsd:float\PYGZdq{}}\PYG{n+nt}{\PYGZgt{}}
                \PYG{n+nt}{\PYGZlt{}xsd:attribute} \PYG{n+na}{name=}\PYG{l+s}{\PYGZdq{}divisa\PYGZdq{}} \PYG{n+na}{type=}\PYG{l+s}{\PYGZdq{}tipoDivisa\PYGZdq{}}\PYG{n+nt}{/\PYGZgt{}}
            \PYG{n+nt}{\PYGZlt{}/xsd:extension\PYGZgt{}}
        \PYG{n+nt}{\PYGZlt{}/xsd:simpleContent\PYGZgt{}}
    \PYG{n+nt}{\PYGZlt{}/xsd:complexType\PYGZgt{}}
    \PYG{n+nt}{\PYGZlt{}xsd:annotation}\PYG{n+nt}{\PYGZgt{}}
        \PYG{n+nt}{\PYGZlt{}xsd:documentation}\PYG{n+nt}{\PYGZgt{}}
        Ahora tenemos que fabricar el \PYGZdq{}tipoDivisa\PYGZdq{} que indica
        los posibles valores válidos para una divisa. Estas
        posibilidades se crean con una \PYGZdq{}enumeration\PYGZdq{}. Nuestro
        tipo es un \PYGZdq{}string\PYGZdq{} y como vamos a restringir los posibles
        valores usaremos \PYGZdq{}restriction\PYGZdq{}
        \PYG{n+nt}{\PYGZlt{}/xsd:documentation\PYGZgt{}}
    \PYG{n+nt}{\PYGZlt{}/xsd:annotation\PYGZgt{}}
    \PYG{n+nt}{\PYGZlt{}xsd:simpleType} \PYG{n+na}{name=}\PYG{l+s}{\PYGZdq{}tipoDivisa\PYGZdq{}}\PYG{n+nt}{\PYGZgt{}}
        \PYG{n+nt}{\PYGZlt{}xsd:restriction} \PYG{n+na}{base=}\PYG{l+s}{\PYGZdq{}xsd:string\PYGZdq{}}\PYG{n+nt}{\PYGZgt{}}
            \PYG{n+nt}{\PYGZlt{}xsd:enumeration} \PYG{n+na}{value=}\PYG{l+s}{\PYGZdq{}euros\PYGZdq{}}\PYG{n+nt}{/\PYGZgt{}}
            \PYG{n+nt}{\PYGZlt{}xsd:enumeration} \PYG{n+na}{value=}\PYG{l+s}{\PYGZdq{}dolares\PYGZdq{}}\PYG{n+nt}{/\PYGZgt{}}
            \PYG{n+nt}{\PYGZlt{}xsd:enumeration} \PYG{n+na}{value=}\PYG{l+s}{\PYGZdq{}yenes\PYGZdq{}}\PYG{n+nt}{/\PYGZgt{}}
        \PYG{n+nt}{\PYGZlt{}/xsd:restriction\PYGZgt{}}
    \PYG{n+nt}{\PYGZlt{}/xsd:simpleType\PYGZgt{}}
\PYG{n+nt}{\PYGZlt{}/xsd:schema\PYGZgt{}}
\end{sphinxVerbatim}


\subsection{Ejercicio: codigos y sedes}
\label{\detokenize{tema5:ejercicio-codigos-y-sedes}}
Se necesita tener un esquema que valide un fichero en el que hay un solo elemento llamado \sphinxcode{codigo}
\begin{itemize}
\item {} 
Dentro de código hay una cadena con una estructura rígida: 2 letras mayúsculas, seguidas de 2 cifras, seguidas a su vez de 3 letras.

\item {} 
El elemento \sphinxcode{código} debe llevar un atributo \sphinxcode{sede} que será de tipo cadena.

\end{itemize}


\subsection{Solución a los códigos y sedes}
\label{\detokenize{tema5:solucion-a-los-codigos-y-sedes}}
Se nos piden dos cosas:
\begin{enumerate}
\item {} 
Restringir un tipo básico, en este caso el \sphinxcode{string}

\item {} 
Extender una etiqueta para que tenga un atributo.

\end{enumerate}

Como no se puede hacer a la vez, deberemos dar dos pasos. Primero crearemos un tipo con la restricción y despues crearemos un segundo tipo con la extensión.

\sphinxstylestrong{Cuando haya conflictos, siempre debemos crear primero la restricción y luego la extensión}

Así, creamos primero esto:

\begin{sphinxVerbatim}[commandchars=\\\{\}]
\PYG{n+nt}{\PYGZlt{}xsd:schema} \PYG{n+na}{xmlns:xsd=}\PYG{l+s}{\PYGZdq{}http://www.w3.org/2001/XMLSchema\PYGZdq{}}\PYG{n+nt}{\PYGZgt{}}
    \PYG{n+nt}{\PYGZlt{}xsd:element} \PYG{n+na}{name=}\PYG{l+s}{\PYGZdq{}codigo\PYGZdq{}} \PYG{n+na}{type=}\PYG{l+s}{\PYGZdq{}tipoCodigoRestringido\PYGZdq{}}\PYG{n+nt}{/\PYGZgt{}}

    \PYG{n+nt}{\PYGZlt{}xsd:simpleType} \PYG{n+na}{name=}\PYG{l+s}{\PYGZdq{}tipoCodigoRestringido\PYGZdq{}}\PYG{n+nt}{\PYGZgt{}}
        \PYG{n+nt}{\PYGZlt{}xsd:restriction} \PYG{n+na}{base=}\PYG{l+s}{\PYGZdq{}xsd:string\PYGZdq{}}\PYG{n+nt}{\PYGZgt{}}
           \PYG{n+nt}{\PYGZlt{}xsd:pattern} \PYG{n+na}{value=}\PYG{l+s}{\PYGZdq{}[A\PYGZhy{}Z]\PYGZob{}2\PYGZcb{}[0\PYGZhy{}9]\PYGZob{}2\PYGZcb{}[A\PYGZhy{}Z]\PYGZob{}3\PYGZcb{}\PYGZdq{}}\PYG{n+nt}{/\PYGZgt{}}
        \PYG{n+nt}{\PYGZlt{}/xsd:restriction\PYGZgt{}}
    \PYG{n+nt}{\PYGZlt{}/xsd:simpleType\PYGZgt{}}
\PYG{n+nt}{\PYGZlt{}/xsd:schema\PYGZgt{}}
\end{sphinxVerbatim}

Y despues lo ampliamos para que se convierta en esto:

\begin{sphinxVerbatim}[commandchars=\\\{\}]
\PYG{n+nt}{\PYGZlt{}xsd:schema} \PYG{n+na}{xmlns:xsd=}\PYG{l+s}{\PYGZdq{}http://www.w3.org/2001/XMLSchema\PYGZdq{}}\PYG{n+nt}{\PYGZgt{}}
    \PYG{n+nt}{\PYGZlt{}xsd:element} \PYG{n+na}{name=}\PYG{l+s}{\PYGZdq{}codigo\PYGZdq{}} \PYG{n+na}{type=}\PYG{l+s}{\PYGZdq{}tipoCodigo\PYGZdq{}}\PYG{n+nt}{/\PYGZgt{}}

    \PYG{n+nt}{\PYGZlt{}xsd:simpleType} \PYG{n+na}{name=}\PYG{l+s}{\PYGZdq{}tipoCodigoRestringido\PYGZdq{}}\PYG{n+nt}{\PYGZgt{}}
        \PYG{n+nt}{\PYGZlt{}xsd:restriction} \PYG{n+na}{base=}\PYG{l+s}{\PYGZdq{}xsd:string\PYGZdq{}}\PYG{n+nt}{\PYGZgt{}}
            \PYG{n+nt}{\PYGZlt{}xsd:pattern} \PYG{n+na}{value=}\PYG{l+s}{\PYGZdq{}[A\PYGZhy{}Z]\PYGZob{}2\PYGZcb{}[0\PYGZhy{}9]\PYGZob{}2\PYGZcb{}[A\PYGZhy{}Z]\PYGZob{}3\PYGZcb{}\PYGZdq{}}\PYG{n+nt}{/\PYGZgt{}}
        \PYG{n+nt}{\PYGZlt{}/xsd:restriction\PYGZgt{}}
    \PYG{n+nt}{\PYGZlt{}/xsd:simpleType\PYGZgt{}}

    \PYG{n+nt}{\PYGZlt{}xsd:complexType} \PYG{n+na}{name=}\PYG{l+s}{\PYGZdq{}tipoCodigo\PYGZdq{}}\PYG{n+nt}{\PYGZgt{}}
        \PYG{n+nt}{\PYGZlt{}xsd:simpleContent}\PYG{n+nt}{\PYGZgt{}}
            \PYG{n+nt}{\PYGZlt{}xsd:extension} \PYG{n+na}{base=}\PYG{l+s}{\PYGZdq{}tipoCodigoRestringido\PYGZdq{}}\PYG{n+nt}{\PYGZgt{}}
                \PYG{n+nt}{\PYGZlt{}xsd:attribute} \PYG{n+na}{name=}\PYG{l+s}{\PYGZdq{}sede\PYGZdq{}}
                               \PYG{n+na}{type=}\PYG{l+s}{\PYGZdq{}xsd:string\PYGZdq{}}
                               \PYG{n+na}{use=}\PYG{l+s}{\PYGZdq{}required\PYGZdq{}}\PYG{n+nt}{/\PYGZgt{}}
            \PYG{n+nt}{\PYGZlt{}/xsd:extension\PYGZgt{}}
        \PYG{n+nt}{\PYGZlt{}/xsd:simpleContent\PYGZgt{}}
    \PYG{n+nt}{\PYGZlt{}/xsd:complexType\PYGZgt{}}
\PYG{n+nt}{\PYGZlt{}/xsd:schema\PYGZgt{}}
\end{sphinxVerbatim}


\subsection{Ejercicio: productos con atributos}
\label{\detokenize{tema5:ejercicio-productos-con-atributos}}
Se desea crear un esquema que permita validar un elemento raíz llamado \sphinxcode{producto} de tipo \sphinxcode{xsd:string}. El producto tiene dos atributos:
\begin{itemize}
\item {} 
Un atributo se llamará \sphinxcode{cantidad} y es obligatorio. Debe aceptar solo enteros positivos.

\item {} 
También habrá un atributo llamado \sphinxcode{unidad} que solo acepta los \sphinxcode{xsd:string} «cajas» y «pales».

\end{itemize}

\begin{sphinxVerbatim}[commandchars=\\\{\}]
\PYG{n+nt}{\PYGZlt{}xsd:schema}
    \PYG{n+na}{xmlns:xsd=}\PYG{l+s}{\PYGZdq{}http://www.w3.org/2001/XMLSchema\PYGZdq{}}\PYG{n+nt}{\PYGZgt{}}
  \PYG{n+nt}{\PYGZlt{}xsd:element} \PYG{n+na}{name=}\PYG{l+s}{\PYGZdq{}producto\PYGZdq{}} \PYG{n+na}{type=}\PYG{l+s}{\PYGZdq{}tipoProducto\PYGZdq{}}\PYG{n+nt}{/\PYGZgt{}}
  \PYG{n+nt}{\PYGZlt{}xsd:complexType} \PYG{n+na}{name=}\PYG{l+s}{\PYGZdq{}tipoProducto\PYGZdq{}}\PYG{n+nt}{\PYGZgt{}}
    \PYG{n+nt}{\PYGZlt{}xsd:simpleContent}\PYG{n+nt}{\PYGZgt{}}
      \PYG{n+nt}{\PYGZlt{}xsd:extension} \PYG{n+na}{base=}\PYG{l+s}{\PYGZdq{}xsd:string\PYGZdq{}}\PYG{n+nt}{\PYGZgt{}}
        \PYG{n+nt}{\PYGZlt{}xsd:attribute} \PYG{n+na}{name=}\PYG{l+s}{\PYGZdq{}cantidad\PYGZdq{}}
              \PYG{n+na}{type=}\PYG{l+s}{\PYGZdq{}xsd:unsignedInt\PYGZdq{}} \PYG{n+na}{use=}\PYG{l+s}{\PYGZdq{}required\PYGZdq{}}\PYG{n+nt}{/\PYGZgt{}}
        \PYG{n+nt}{\PYGZlt{}xsd:attribute} \PYG{n+na}{name=}\PYG{l+s}{\PYGZdq{}unidad\PYGZdq{}}
              \PYG{n+na}{type=}\PYG{l+s}{\PYGZdq{}tipoUnidad\PYGZdq{}}\PYG{n+nt}{/\PYGZgt{}}
      \PYG{n+nt}{\PYGZlt{}/xsd:extension\PYGZgt{}}
    \PYG{n+nt}{\PYGZlt{}/xsd:simpleContent\PYGZgt{}}
  \PYG{n+nt}{\PYGZlt{}/xsd:complexType\PYGZgt{}}
  \PYG{n+nt}{\PYGZlt{}xsd:simpleType} \PYG{n+na}{name=}\PYG{l+s}{\PYGZdq{}tipoUnidad\PYGZdq{}}\PYG{n+nt}{\PYGZgt{}}
    \PYG{n+nt}{\PYGZlt{}xsd:restriction} \PYG{n+na}{base=}\PYG{l+s}{\PYGZdq{}xsd:string\PYGZdq{}}\PYG{n+nt}{\PYGZgt{}}
      \PYG{n+nt}{\PYGZlt{}xsd:enumeration} \PYG{n+na}{value=}\PYG{l+s}{\PYGZdq{}caja\PYGZdq{}}\PYG{n+nt}{/\PYGZgt{}}
      \PYG{n+nt}{\PYGZlt{}xsd:enumeration} \PYG{n+na}{value=}\PYG{l+s}{\PYGZdq{}pale\PYGZdq{}}\PYG{n+nt}{/\PYGZgt{}}
    \PYG{n+nt}{\PYGZlt{}/xsd:restriction\PYGZgt{}}
  \PYG{n+nt}{\PYGZlt{}/xsd:simpleType\PYGZgt{}}
\PYG{n+nt}{\PYGZlt{}/xsd:schema\PYGZgt{}}
\end{sphinxVerbatim}


\subsection{Ejercicio: clientes con información adicional}
\label{\detokenize{tema5:ejercicio-clientes-con-informacion-adicional}}
Se desea crear un esquema XML que permita validar un elemento llamado \sphinxcode{cliente} que puede almacenar un \sphinxcode{xsd:string}. El cliente contiene:
\begin{itemize}
\item {} 
Un atributo obligatorio llamado \sphinxcode{codigo} que contiene el código del cliente, que siempre consta de tres letras mayúsculas de tres números.

\item {} 
Un atributo optativo llamado \sphinxcode{habitual} que se usará para saber si es un cliente habitual o no. Acepta valores «true» y «false».

\item {} 
Un atributo optativo llamado \sphinxcode{cantidad} que indica su compra. Es un entero con valores de entre 0 y 1000.

\end{itemize}


\subsection{Lista de clientes como XML Schemas}
\label{\detokenize{tema5:lista-de-clientes-como-xml-schemas}}
En este apartado volveremos a ver un problema que ya resolvíamos con DTD: supongamos que en nuestros ficheros deseamos indicar que el elemento raíz es \sphinxcode{\textless{}listaclientes\textgreater{}}. Dentro de \sphinxcode{\textless{}listaclientes\textgreater{}} deseamos permitir uno o más elementos \sphinxcode{\textless{}cliente\textgreater{}}. Dentro de \sphinxcode{\textless{}cliente\textgreater{}} todos deberán tener \sphinxcode{\textless{}cif\textgreater{}} y \sphinxcode{\textless{}nombre\textgreater{}} y en ese orden. Dentro de \sphinxcode{\textless{}cliente\textgreater{}} puede aparecer o no un elemento \sphinxcode{\textless{}diasentrega\textgreater{}} para indicar que ese cliente exige un máximo de plazo. Como no todo el mundo usa plazos el \sphinxcode{\textless{}diasentrega\textgreater{}} es optativo.

Vayamos paso a paso. Primero decimos como se llama el elemento raíz y de qué tipo es:

\begin{sphinxVerbatim}[commandchars=\\\{\}]
\PYG{n+nt}{\PYGZlt{}xsd:schema} \PYG{n+na}{xmlns:xsd=}\PYG{l+s}{\PYGZdq{}http://www.w3.org/2001/XMLSchema\PYGZdq{}}\PYG{n+nt}{\PYGZgt{}}
    \PYG{n+nt}{\PYGZlt{}xsd:element} \PYG{n+na}{name=}\PYG{l+s}{\PYGZdq{}listaclientes\PYGZdq{}} \PYG{n+na}{type=}\PYG{l+s}{\PYGZdq{}tipoListaClientes\PYGZdq{}}\PYG{n+nt}{/\PYGZgt{}}
\PYG{n+nt}{\PYGZlt{}/xsd:schema\PYGZgt{}}
\end{sphinxVerbatim}

Ahora queda definir el tipo \sphinxcode{tipoListaClientes}. Este tipo va a contener un elemento (por lo que ya sabemos que es un \sphinxcode{complexType} con \sphinxcode{complexContent} dentro), y en concreto queremos que sea un solo elemento llamado \sphinxcode{cliente}, es decir \sphinxstylestrong{queremos imponer una restricción}. Aunque queramos un solo elemento tendremos que indicar una restricción. Como queremos permitir que el elemento pueda aparecer muchas veces utilizaremos un \sphinxcode{maxOccurs} con el valor \sphinxcode{unbounded}.

\begin{sphinxVerbatim}[commandchars=\\\{\}]
\PYG{n+nt}{\PYGZlt{}xsd:schema} \PYG{n+na}{xmlns:xsd=}\PYG{l+s}{\PYGZdq{}http://www.w3.org/2001/XMLSchema\PYGZdq{}}\PYG{n+nt}{\PYGZgt{}}
    \PYG{n+nt}{\PYGZlt{}xsd:element} \PYG{n+na}{name=}\PYG{l+s}{\PYGZdq{}listaclientes\PYGZdq{}} \PYG{n+na}{type=}\PYG{l+s}{\PYGZdq{}tipoListaClientes\PYGZdq{}}\PYG{n+nt}{/\PYGZgt{}}
    \PYG{n+nt}{\PYGZlt{}xsd:complexType} \PYG{n+na}{name=}\PYG{l+s}{\PYGZdq{}tipoListaClientes\PYGZdq{}}\PYG{n+nt}{\PYGZgt{}}
        \PYG{n+nt}{\PYGZlt{}xsd:complexContent}\PYG{n+nt}{\PYGZgt{}}
            \PYG{n+nt}{\PYGZlt{}xsd:restriction}\PYG{n+nt}{\PYGZgt{}}
                \PYG{n+nt}{\PYGZlt{}xsd:element} \PYG{n+na}{name=}\PYG{l+s}{\PYGZdq{}cliente\PYGZdq{}} \PYG{n+na}{type=}\PYG{l+s}{\PYGZdq{}tipoCliente\PYGZdq{}}
                \PYG{n+na}{maxOccurs=}\PYG{l+s}{\PYGZdq{}unbounded\PYGZdq{}}\PYG{n+nt}{/\PYGZgt{}}
            \PYG{n+nt}{\PYGZlt{}/xsd:restriction\PYGZgt{}}
        \PYG{n+nt}{\PYGZlt{}/xsd:complexContent\PYGZgt{}}
    \PYG{n+nt}{\PYGZlt{}/xsd:complexType\PYGZgt{}}
\PYG{n+nt}{\PYGZlt{}/xsd:schema\PYGZgt{}}
\end{sphinxVerbatim}

Definamos ahora el tipo \sphinxcode{tipoCliente}. Dicho tipo \sphinxstylestrong{necesita tener subelementos dentro} así que evidentemente va a ser de tipo complejo. La pregunta es ¿es «tipo complejo con contenido simple» o «tipo complejo con contenido complejo»?. Si lo hiciéramos de «tipo complejo con contenido simple» podríamos tener atributos pero no subelementos, así que forzosamente tendrá que ser de un «tipo complejo con contenido complejo». Igual que antes impondremos una restricciones que es permitir solo que aparezcan ciertos elementos en cierto orden. El elemento \sphinxcode{plazo} lo haremos optativo.

\begin{sphinxVerbatim}[commandchars=\\\{\}]
\PYG{n+nt}{\PYGZlt{}xsd:schema} \PYG{n+na}{xmlns:xsd=}\PYG{l+s}{\PYGZdq{}http://www.w3.org/2001/XMLSchema\PYGZdq{}}\PYG{n+nt}{\PYGZgt{}}
    \PYG{n+nt}{\PYGZlt{}xsd:element} \PYG{n+na}{name=}\PYG{l+s}{\PYGZdq{}listaclientes\PYGZdq{}} \PYG{n+na}{type=}\PYG{l+s}{\PYGZdq{}tipoListaClientes\PYGZdq{}}\PYG{n+nt}{/\PYGZgt{}}
    \PYG{n+nt}{\PYGZlt{}xsd:complexType} \PYG{n+na}{name=}\PYG{l+s}{\PYGZdq{}tipoListaClientes\PYGZdq{}}\PYG{n+nt}{\PYGZgt{}}
        \PYG{n+nt}{\PYGZlt{}xsd:complexContent}\PYG{n+nt}{\PYGZgt{}}
            \PYG{n+nt}{\PYGZlt{}xsd:restriction} \PYG{n+na}{base=}\PYG{l+s}{\PYGZdq{}xsd:anyType\PYGZdq{}}\PYG{n+nt}{\PYGZgt{}}
                \PYG{n+nt}{\PYGZlt{}xsd:sequence}\PYG{n+nt}{\PYGZgt{}}
                    \PYG{n+nt}{\PYGZlt{}xsd:element} \PYG{n+na}{name=}\PYG{l+s}{\PYGZdq{}cliente\PYGZdq{}} \PYG{n+na}{type=}\PYG{l+s}{\PYGZdq{}tipoCliente\PYGZdq{}}
                    \PYG{n+na}{maxOccurs=}\PYG{l+s}{\PYGZdq{}unbounded\PYGZdq{}}\PYG{n+nt}{/\PYGZgt{}}
                \PYG{n+nt}{\PYGZlt{}/xsd:sequence\PYGZgt{}}
            \PYG{n+nt}{\PYGZlt{}/xsd:restriction\PYGZgt{}}
        \PYG{n+nt}{\PYGZlt{}/xsd:complexContent\PYGZgt{}}
    \PYG{n+nt}{\PYGZlt{}/xsd:complexType\PYGZgt{}}
    \PYG{n+nt}{\PYGZlt{}xsd:complexType} \PYG{n+na}{name=}\PYG{l+s}{\PYGZdq{}tipoCliente\PYGZdq{}}\PYG{n+nt}{\PYGZgt{}}
        \PYG{n+nt}{\PYGZlt{}xsd:complexContent}\PYG{n+nt}{\PYGZgt{}}
            \PYG{n+nt}{\PYGZlt{}xsd:restriction} \PYG{n+na}{base=}\PYG{l+s}{\PYGZdq{}xsd:anyType\PYGZdq{}}\PYG{n+nt}{\PYGZgt{}}
                \PYG{n+nt}{\PYGZlt{}xsd:sequence}\PYG{n+nt}{\PYGZgt{}}
                    \PYG{n+nt}{\PYGZlt{}xsd:element} \PYG{n+na}{name=}\PYG{l+s}{\PYGZdq{}cif\PYGZdq{}} \PYG{n+na}{type=}\PYG{l+s}{\PYGZdq{}xsd:string\PYGZdq{}}\PYG{n+nt}{/\PYGZgt{}}
                    \PYG{n+nt}{\PYGZlt{}xsd:element} \PYG{n+na}{name=}\PYG{l+s}{\PYGZdq{}nombre\PYGZdq{}} \PYG{n+na}{type=}\PYG{l+s}{\PYGZdq{}xsd:string\PYGZdq{}}\PYG{n+nt}{/\PYGZgt{}}
                    \PYG{n+nt}{\PYGZlt{}xsd:element} \PYG{n+na}{name=}\PYG{l+s}{\PYGZdq{}plazo\PYGZdq{}} \PYG{n+na}{type=}\PYG{l+s}{\PYGZdq{}xsd:string\PYGZdq{}}
                    \PYG{n+na}{minOccurs=}\PYG{l+s}{\PYGZdq{}0\PYGZdq{}}\PYG{n+nt}{/\PYGZgt{}}
                \PYG{n+nt}{\PYGZlt{}/xsd:sequence\PYGZgt{}}
            \PYG{n+nt}{\PYGZlt{}/xsd:restriction\PYGZgt{}}
        \PYG{n+nt}{\PYGZlt{}/xsd:complexContent\PYGZgt{}}
    \PYG{n+nt}{\PYGZlt{}/xsd:complexType\PYGZgt{}}
\PYG{n+nt}{\PYGZlt{}/xsd:schema\PYGZgt{}}
\end{sphinxVerbatim}

Si ahora probamos este XML veremos que el fichero se valida perfectamente a pesar de que es evidente que tiene errores. Es lógico, dado que no hemos aprovechado a fondo el sistema de tipos de XML para evitar que nadie suministre datos incorrectos en un XML. Dicha mejora la dejaremos para el siguiente ejercicio.

\begin{sphinxVerbatim}[commandchars=\\\{\}]
\PYG{n+nt}{\PYGZlt{}listaclientes}\PYG{n+nt}{\PYGZgt{}}
    \PYG{n+nt}{\PYGZlt{}cliente}\PYG{n+nt}{\PYGZgt{}}
        \PYG{n+nt}{\PYGZlt{}cif}\PYG{n+nt}{\PYGZgt{}}dd\PYG{n+nt}{\PYGZlt{}/cif\PYGZgt{}}
        \PYG{n+nt}{\PYGZlt{}nombre}\PYG{n+nt}{\PYGZgt{}}20\PYG{n+nt}{\PYGZlt{}/nombre\PYGZgt{}}
    \PYG{n+nt}{\PYGZlt{}/cliente\PYGZgt{}}
    \PYG{n+nt}{\PYGZlt{}cliente}\PYG{n+nt}{\PYGZgt{}}
        \PYG{n+nt}{\PYGZlt{}cif}\PYG{n+nt}{\PYGZgt{}}dd\PYG{n+nt}{\PYGZlt{}/cif\PYGZgt{}}
        \PYG{n+nt}{\PYGZlt{}nombre}\PYG{n+nt}{\PYGZgt{}}20\PYG{n+nt}{\PYGZlt{}/nombre\PYGZgt{}}
        \PYG{n+nt}{\PYGZlt{}plazo}\PYG{n+nt}{\PYGZgt{}}ABCD\PYG{n+nt}{\PYGZlt{}/plazo\PYGZgt{}}
    \PYG{n+nt}{\PYGZlt{}/cliente\PYGZgt{}}
\PYG{n+nt}{\PYGZlt{}/listaclientes\PYGZgt{}}
\end{sphinxVerbatim}


\subsection{Ampliación del esquema para clientes}
\label{\detokenize{tema5:ampliacion-del-esquema-para-clientes}}
Ahora ampliaremos el XML Schema del fichero anterior para que nadie suministre información incorrecta.

En concreto tenemos tres datos:
\begin{enumerate}
\item {} 
El CIF, que vamos a presuponer que siempre tiene 8 cifras y al final una letra mayúsculas. Si alguna empresa tiene 7 cifras deberá incluir un 0 extra.

\item {} 
El nombre, que puede ser una cadena cualquiera.

\item {} 
El plazo, que debería ser un número positivo válido.

\end{enumerate}

Ahora, el fichero anterior no debería ser validado por el validador, pero sí debería serlo un fichero como este.

\begin{sphinxVerbatim}[commandchars=\\\{\}]
\PYG{n+nt}{\PYGZlt{}listaclientes}\PYG{n+nt}{\PYGZgt{}}
    \PYG{n+nt}{\PYGZlt{}cliente}\PYG{n+nt}{\PYGZgt{}}
        \PYG{n+nt}{\PYGZlt{}cif}\PYG{n+nt}{\PYGZgt{}}01234567D\PYG{n+nt}{\PYGZlt{}/cif\PYGZgt{}}
        \PYG{n+nt}{\PYGZlt{}nombre}\PYG{n+nt}{\PYGZgt{}}Juan Sanchez\PYG{n+nt}{\PYGZlt{}/nombre\PYGZgt{}}
    \PYG{n+nt}{\PYGZlt{}/cliente\PYGZgt{}}
    \PYG{n+nt}{\PYGZlt{}cliente}\PYG{n+nt}{\PYGZgt{}}
        \PYG{n+nt}{\PYGZlt{}cif}\PYG{n+nt}{\PYGZgt{}}05676554A\PYG{n+nt}{\PYGZlt{}/cif\PYGZgt{}}
        \PYG{n+nt}{\PYGZlt{}nombre}\PYG{n+nt}{\PYGZgt{}}Pedro Diaz\PYG{n+nt}{\PYGZlt{}/nombre\PYGZgt{}}
        \PYG{n+nt}{\PYGZlt{}plazo}\PYG{n+nt}{\PYGZgt{}}45\PYG{n+nt}{\PYGZlt{}/plazo\PYGZgt{}}
    \PYG{n+nt}{\PYGZlt{}/cliente\PYGZgt{}}
\PYG{n+nt}{\PYGZlt{}/listaclientes\PYGZgt{}}
\end{sphinxVerbatim}

La solución a los tres problemas indicados antes sería la siguiente:
\begin{enumerate}
\item {} 
El nombre puede ser una cadena cualquiera, por lo que tendrá que seguir siendo de tipo \sphinxcode{xsd:string}. Eso significa que si alguien introdujese un número en el nombre el fichero seguiría validándose. Por desgracia dicho problema no se puede resolver.

\item {} 
El plazo debería ser un número. Le asignaremos un tipo \sphinxcode{xsd:unsignedInt}.

\item {} 
El CIF es más complejo. Deberemos crear un tipo nuevo y establecer una restricción a los posibles valores que puede tomar.

\end{enumerate}

Así, una posible solución sería esta:

\begin{sphinxVerbatim}[commandchars=\\\{\}]
\PYG{n+nt}{\PYGZlt{}xsd:schema} \PYG{n+na}{xmlns:xsd=}\PYG{l+s}{\PYGZdq{}http://www.w3.org/2001/XMLSchema\PYGZdq{}}\PYG{n+nt}{\PYGZgt{}}
    \PYG{n+nt}{\PYGZlt{}xsd:element} \PYG{n+na}{name=}\PYG{l+s}{\PYGZdq{}listaclientes\PYGZdq{}} \PYG{n+na}{type=}\PYG{l+s}{\PYGZdq{}tipoListaClientes\PYGZdq{}}\PYG{n+nt}{/\PYGZgt{}}
    \PYG{n+nt}{\PYGZlt{}xsd:complexType} \PYG{n+na}{name=}\PYG{l+s}{\PYGZdq{}tipoListaClientes\PYGZdq{}}\PYG{n+nt}{\PYGZgt{}}
        \PYG{n+nt}{\PYGZlt{}xsd:complexContent}\PYG{n+nt}{\PYGZgt{}}
            \PYG{n+nt}{\PYGZlt{}xsd:restriction} \PYG{n+na}{base=}\PYG{l+s}{\PYGZdq{}xsd:anyType\PYGZdq{}}\PYG{n+nt}{\PYGZgt{}}
                \PYG{n+nt}{\PYGZlt{}xsd:sequence}\PYG{n+nt}{\PYGZgt{}}
                    \PYG{n+nt}{\PYGZlt{}xsd:element} \PYG{n+na}{name=}\PYG{l+s}{\PYGZdq{}cliente\PYGZdq{}} \PYG{n+na}{type=}\PYG{l+s}{\PYGZdq{}tipoCliente\PYGZdq{}}
                        \PYG{n+na}{maxOccurs=}\PYG{l+s}{\PYGZdq{}unbounded\PYGZdq{}}\PYG{n+nt}{/\PYGZgt{}}
                \PYG{n+nt}{\PYGZlt{}/xsd:sequence\PYGZgt{}}
            \PYG{n+nt}{\PYGZlt{}/xsd:restriction\PYGZgt{}}
        \PYG{n+nt}{\PYGZlt{}/xsd:complexContent\PYGZgt{}}
    \PYG{n+nt}{\PYGZlt{}/xsd:complexType\PYGZgt{}}

    \PYG{n+nt}{\PYGZlt{}xsd:complexType} \PYG{n+na}{name=}\PYG{l+s}{\PYGZdq{}tipoCliente\PYGZdq{}}\PYG{n+nt}{\PYGZgt{}}
        \PYG{n+nt}{\PYGZlt{}xsd:complexContent}\PYG{n+nt}{\PYGZgt{}}
            \PYG{n+nt}{\PYGZlt{}xsd:restriction} \PYG{n+na}{base=}\PYG{l+s}{\PYGZdq{}xsd:anyType\PYGZdq{}}\PYG{n+nt}{\PYGZgt{}}
            \PYG{n+nt}{\PYGZlt{}xsd:sequence}\PYG{n+nt}{\PYGZgt{}}
                \PYG{n+nt}{\PYGZlt{}xsd:element} \PYG{n+na}{name=}\PYG{l+s}{\PYGZdq{}cif\PYGZdq{}} \PYG{n+na}{type=}\PYG{l+s}{\PYGZdq{}tipoCif\PYGZdq{}}\PYG{n+nt}{/\PYGZgt{}}
                \PYG{n+nt}{\PYGZlt{}xsd:element} \PYG{n+na}{name=}\PYG{l+s}{\PYGZdq{}nombre\PYGZdq{}} \PYG{n+na}{type=}\PYG{l+s}{\PYGZdq{}xsd:string\PYGZdq{}}\PYG{n+nt}{/\PYGZgt{}}
                \PYG{n+nt}{\PYGZlt{}xsd:element} \PYG{n+na}{name=}\PYG{l+s}{\PYGZdq{}plazo\PYGZdq{}} \PYG{n+na}{type=}\PYG{l+s}{\PYGZdq{}xsd:unsignedInt\PYGZdq{}} \PYG{n+na}{minOccurs=}\PYG{l+s}{\PYGZdq{}0\PYGZdq{}}\PYG{n+nt}{/\PYGZgt{}}
            \PYG{n+nt}{\PYGZlt{}/xsd:sequence\PYGZgt{}}
            \PYG{n+nt}{\PYGZlt{}/xsd:restriction\PYGZgt{}}
        \PYG{n+nt}{\PYGZlt{}/xsd:complexContent\PYGZgt{}}
    \PYG{n+nt}{\PYGZlt{}/xsd:complexType\PYGZgt{}}
    \PYG{n+nt}{\PYGZlt{}xsd:simpleType} \PYG{n+na}{name=}\PYG{l+s}{\PYGZdq{}tipoCif\PYGZdq{}}\PYG{n+nt}{\PYGZgt{}}
        \PYG{n+nt}{\PYGZlt{}xsd:restriction} \PYG{n+na}{base=}\PYG{l+s}{\PYGZdq{}xsd:string\PYGZdq{}}\PYG{n+nt}{\PYGZgt{}}
            \PYG{n+nt}{\PYGZlt{}xsd:pattern} \PYG{n+na}{value=}\PYG{l+s}{\PYGZdq{}[0\PYGZhy{}9]\PYGZob{}8\PYGZcb{}[A\PYGZhy{}Z]\PYGZdq{}}\PYG{n+nt}{/\PYGZgt{}}
        \PYG{n+nt}{\PYGZlt{}/xsd:restriction\PYGZgt{}}
    \PYG{n+nt}{\PYGZlt{}/xsd:simpleType\PYGZgt{}}
    \PYG{n+nt}{\PYGZlt{}xsd:simpleType} \PYG{n+na}{name=}\PYG{l+s}{\PYGZdq{}tipoPlazo\PYGZdq{}}\PYG{n+nt}{\PYGZgt{}}
        \PYG{n+nt}{\PYGZlt{}xsd:restriction} \PYG{n+na}{base=}\PYG{l+s}{\PYGZdq{}xsd:unsignedInt\PYGZdq{}}\PYG{n+nt}{/\PYGZgt{}}
    \PYG{n+nt}{\PYGZlt{}/xsd:simpleType\PYGZgt{}}
\PYG{n+nt}{\PYGZlt{}/xsd:schema\PYGZgt{}}
\end{sphinxVerbatim}


\subsection{Ejercicio: lista de códigos}
\label{\detokenize{tema5:ejercicio-lista-de-codigos}}
Se nos pide crear un esquema que permita validar un fichero como el siguiente:

\begin{sphinxVerbatim}[commandchars=\\\{\}]
\PYG{n+nt}{\PYGZlt{}listacodigos}\PYG{n+nt}{\PYGZgt{}}
  \PYG{n+nt}{\PYGZlt{}codigo}\PYG{n+nt}{\PYGZgt{}}AAA2DD\PYG{n+nt}{\PYGZlt{}/codigo\PYGZgt{}}
  \PYG{n+nt}{\PYGZlt{}codigo}\PYG{n+nt}{\PYGZgt{}}BBB2EE\PYG{n+nt}{\PYGZlt{}/codigo\PYGZgt{}}
  \PYG{n+nt}{\PYGZlt{}codigo}\PYG{n+nt}{\PYGZgt{}}BBB2EE\PYG{n+nt}{\PYGZlt{}/codigo\PYGZgt{}}
\PYG{n+nt}{\PYGZlt{}/listacodigos\PYGZgt{}}
\end{sphinxVerbatim}

En concreto, todo código tiene la estructura siguiente:
\begin{enumerate}
\item {} 
Primero van tres mayúsculas

\item {} 
Despues va exactamente un digito.

\item {} 
Por último hay exactamente dos mayúsculas.

\end{enumerate}

Un posible esquema XML sería el siguiente (obsérvese como usamos \sphinxcode{maxOccurs} para indicar que el elemento puede repetirse un máximo de «infitas veces»:

\begin{sphinxVerbatim}[commandchars=\\\{\}]
\PYG{n+nt}{\PYGZlt{}xsd:schema}
    \PYG{n+na}{xmlns:xsd=}\PYG{l+s}{\PYGZdq{}http://www.w3.org/2001/XMLSchema\PYGZdq{}}\PYG{n+nt}{\PYGZgt{}}
  \PYG{n+nt}{\PYGZlt{}xsd:element} \PYG{n+na}{name=}\PYG{l+s}{\PYGZdq{}listacodigos\PYGZdq{}}
               \PYG{n+na}{type=}\PYG{l+s}{\PYGZdq{}tipoLista\PYGZdq{}}\PYG{n+nt}{/\PYGZgt{}}
  \PYG{n+nt}{\PYGZlt{}xsd:complexType} \PYG{n+na}{name=}\PYG{l+s}{\PYGZdq{}tipoLista\PYGZdq{}}\PYG{n+nt}{\PYGZgt{}}
    \PYG{n+nt}{\PYGZlt{}xsd:complexContent}\PYG{n+nt}{\PYGZgt{}}
      \PYG{n+nt}{\PYGZlt{}xsd:restriction} \PYG{n+na}{base=}\PYG{l+s}{\PYGZdq{}xsd:anyType\PYGZdq{}}\PYG{n+nt}{\PYGZgt{}}
        \PYG{n+nt}{\PYGZlt{}xsd:sequence}\PYG{n+nt}{\PYGZgt{}}
          \PYG{n+nt}{\PYGZlt{}xsd:element} \PYG{n+na}{name=}\PYG{l+s}{\PYGZdq{}codigo\PYGZdq{}}
                       \PYG{n+na}{type=}\PYG{l+s}{\PYGZdq{}tipoCodigo\PYGZdq{}}
                       \PYG{n+na}{maxOccurs=}\PYG{l+s}{\PYGZdq{}unbounded\PYGZdq{}}\PYG{n+nt}{/\PYGZgt{}}
        \PYG{n+nt}{\PYGZlt{}/xsd:sequence\PYGZgt{}}
      \PYG{n+nt}{\PYGZlt{}/xsd:restriction\PYGZgt{}}
    \PYG{n+nt}{\PYGZlt{}/xsd:complexContent\PYGZgt{}}
  \PYG{n+nt}{\PYGZlt{}/xsd:complexType\PYGZgt{}}
  \PYG{n+nt}{\PYGZlt{}xsd:simpleType} \PYG{n+na}{name=}\PYG{l+s}{\PYGZdq{}tipoCodigo\PYGZdq{}}\PYG{n+nt}{\PYGZgt{}}
    \PYG{n+nt}{\PYGZlt{}xsd:restriction} \PYG{n+na}{base=}\PYG{l+s}{\PYGZdq{}xsd:string\PYGZdq{}}\PYG{n+nt}{\PYGZgt{}}
      \PYG{n+nt}{\PYGZlt{}xsd:pattern} \PYG{n+na}{value=}\PYG{l+s}{\PYGZdq{}[A\PYGZhy{}Z]\PYGZob{}3\PYGZcb{}[0\PYGZhy{}9][A\PYGZhy{}Z]\PYGZob{}2\PYGZcb{}\PYGZdq{}}\PYG{n+nt}{/\PYGZgt{}}
    \PYG{n+nt}{\PYGZlt{}/xsd:restriction\PYGZgt{}}
  \PYG{n+nt}{\PYGZlt{}/xsd:simpleType\PYGZgt{}}
\PYG{n+nt}{\PYGZlt{}/xsd:schema\PYGZgt{}}
\end{sphinxVerbatim}


\subsection{Ejercicio: otra lista de clientes}
\label{\detokenize{tema5:ejercicio-otra-lista-de-clientes}}
Ahora se nos pide crear un esquema que permita validar un fichero como el siguiente, en el que hay una lista de clientes y el nombre es optativo, aunque los apellidos son obligatorios:

\begin{sphinxVerbatim}[commandchars=\\\{\}]
\PYG{n+nt}{\PYGZlt{}listaclientes}\PYG{n+nt}{\PYGZgt{}}
  \PYG{n+nt}{\PYGZlt{}cliente}\PYG{n+nt}{\PYGZgt{}}
    \PYG{n+nt}{\PYGZlt{}nombre}\PYG{n+nt}{\PYGZgt{}}Juan\PYG{n+nt}{\PYGZlt{}/nombre\PYGZgt{}}
    \PYG{n+nt}{\PYGZlt{}apellidos}\PYG{n+nt}{\PYGZgt{}}Sanchez\PYG{n+nt}{\PYGZlt{}/apellidos\PYGZgt{}}
  \PYG{n+nt}{\PYGZlt{}/cliente\PYGZgt{}}
  \PYG{n+nt}{\PYGZlt{}cliente}\PYG{n+nt}{\PYGZgt{}}
    \PYG{n+nt}{\PYGZlt{}nombre}\PYG{n+nt}{\PYGZgt{}}Jose\PYG{n+nt}{\PYGZlt{}/nombre\PYGZgt{}}
    \PYG{n+nt}{\PYGZlt{}apellidos}\PYG{n+nt}{\PYGZgt{}}Diaz\PYG{n+nt}{\PYGZlt{}/apellidos\PYGZgt{}}
  \PYG{n+nt}{\PYGZlt{}/cliente\PYGZgt{}}
\PYG{n+nt}{\PYGZlt{}/listaclientes\PYGZgt{}}
\end{sphinxVerbatim}

La solución puede ser algo así:

\begin{sphinxVerbatim}[commandchars=\\\{\}]
\PYG{n+nt}{\PYGZlt{}xsd:schema}
  \PYG{n+na}{xmlns:xsd=}\PYG{l+s}{\PYGZdq{}http://www.w3.org/2001/XMLSchema\PYGZdq{}}\PYG{n+nt}{\PYGZgt{}}
  \PYG{n+nt}{\PYGZlt{}xsd:element} \PYG{n+na}{name=}\PYG{l+s}{\PYGZdq{}listaclientes\PYGZdq{}}
               \PYG{n+na}{type=}\PYG{l+s}{\PYGZdq{}tipoLista\PYGZdq{}}\PYG{n+nt}{/\PYGZgt{}}
  \PYG{n+nt}{\PYGZlt{}xsd:complexType} \PYG{n+na}{name=}\PYG{l+s}{\PYGZdq{}tipoLista\PYGZdq{}}\PYG{n+nt}{\PYGZgt{}}
    \PYG{n+nt}{\PYGZlt{}xsd:complexContent}\PYG{n+nt}{\PYGZgt{}}
      \PYG{n+nt}{\PYGZlt{}xsd:restriction} \PYG{n+na}{base=}\PYG{l+s}{\PYGZdq{}xsd:anyType\PYGZdq{}}\PYG{n+nt}{\PYGZgt{}}
        \PYG{n+nt}{\PYGZlt{}xsd:sequence}\PYG{n+nt}{\PYGZgt{}}
          \PYG{n+nt}{\PYGZlt{}xsd:element} \PYG{n+na}{name=}\PYG{l+s}{\PYGZdq{}cliente\PYGZdq{}}
                       \PYG{n+na}{type=}\PYG{l+s}{\PYGZdq{}tipoCliente\PYGZdq{}}
                       \PYG{n+na}{maxOccurs=}\PYG{l+s}{\PYGZdq{}unbounded\PYGZdq{}}\PYG{n+nt}{/\PYGZgt{}}
        \PYG{n+nt}{\PYGZlt{}/xsd:sequence\PYGZgt{}}
      \PYG{n+nt}{\PYGZlt{}/xsd:restriction\PYGZgt{}}
    \PYG{n+nt}{\PYGZlt{}/xsd:complexContent\PYGZgt{}}
  \PYG{n+nt}{\PYGZlt{}/xsd:complexType\PYGZgt{}}

  \PYG{n+nt}{\PYGZlt{}xsd:complexType} \PYG{n+na}{name=}\PYG{l+s}{\PYGZdq{}tipoCliente\PYGZdq{}}\PYG{n+nt}{\PYGZgt{}}
    \PYG{n+nt}{\PYGZlt{}xsd:complexContent}\PYG{n+nt}{\PYGZgt{}}
      \PYG{n+nt}{\PYGZlt{}xsd:restriction} \PYG{n+na}{base=}\PYG{l+s}{\PYGZdq{}xsd:anyType\PYGZdq{}}\PYG{n+nt}{\PYGZgt{}}
        \PYG{n+nt}{\PYGZlt{}xsd:sequence}\PYG{n+nt}{\PYGZgt{}}
          \PYG{n+nt}{\PYGZlt{}xsd:element} \PYG{n+na}{name=}\PYG{l+s}{\PYGZdq{}nombre\PYGZdq{}}
                       \PYG{n+na}{type=}\PYG{l+s}{\PYGZdq{}xsd:string\PYGZdq{}}
                       \PYG{n+na}{minOccurs=}\PYG{l+s}{\PYGZdq{}0\PYGZdq{}}\PYG{n+nt}{/\PYGZgt{}}
          \PYG{n+nt}{\PYGZlt{}xsd:element} \PYG{n+na}{name=}\PYG{l+s}{\PYGZdq{}apellidos\PYGZdq{}}
                       \PYG{n+na}{type=}\PYG{l+s}{\PYGZdq{}xsd:string\PYGZdq{}}\PYG{n+nt}{/\PYGZgt{}}
        \PYG{n+nt}{\PYGZlt{}/xsd:sequence\PYGZgt{}}
      \PYG{n+nt}{\PYGZlt{}/xsd:restriction\PYGZgt{}}
  \PYG{n+nt}{\PYGZlt{}/xsd:complexContent\PYGZgt{}}
  \PYG{n+nt}{\PYGZlt{}/xsd:complexType\PYGZgt{}}
\PYG{n+nt}{\PYGZlt{}/xsd:schema\PYGZgt{}}
\end{sphinxVerbatim}


\subsection{Ejercicio: lista de alumnos}
\label{\detokenize{tema5:ejercicio-lista-de-alumnos}}
Se desea construir un esquema para validar listas de alumnos en las que:
\begin{itemize}
\item {} 
La raíz es \sphinxcode{listaalumnos}.

\item {} 
Dentro de ella hay uno o más \sphinxcode{alumno}. Todo \sphinxcode{alumno} tiene siempre un DNI que es obligatorio y que tiene una estructura formada por 7 u 8 cifras seguidas de una mayúscula.

\item {} 
Todo \sphinxcode{alumno} tiene un elemento \sphinxcode{nombre} y un \sphinxcode{ap1} obligatorios.

\item {} 
Todo \sphinxcode{alumno} puede tener despues del \sphinxcode{ap1} un elemento \sphinxcode{ap2} y uno \sphinxcode{edad}, ambos son optativos.

\item {} 
El elemento \sphinxcode{edad} debe ser entero y positivo.

\end{itemize}

Un ejemplo de fichero:

\begin{sphinxVerbatim}[commandchars=\\\{\}]
\PYG{n+nt}{\PYGZlt{}listaalumnos}\PYG{n+nt}{\PYGZgt{}}
    \PYG{c}{\PYGZlt{}!\PYGZhy{}\PYGZhy{}}\PYG{c}{DNI atributo obligatorio}\PYG{c}{\PYGZhy{}\PYGZhy{}\PYGZgt{}}
    \PYG{n+nt}{\PYGZlt{}alumno} \PYG{n+na}{dni=}\PYG{l+s}{\PYGZdq{}5667545Z\PYGZdq{}}\PYG{n+nt}{\PYGZgt{}}
        \PYG{c}{\PYGZlt{}!\PYGZhy{}\PYGZhy{}}\PYG{c}{Nombre y ap1 obligatorios}\PYG{c}{\PYGZhy{}\PYGZhy{}\PYGZgt{}}
        \PYG{n+nt}{\PYGZlt{}nombre}\PYG{n+nt}{\PYGZgt{}}Jose\PYG{n+nt}{\PYGZlt{}/nombre\PYGZgt{}}
        \PYG{n+nt}{\PYGZlt{}ap1}\PYG{n+nt}{\PYGZgt{}}Sanchez\PYG{n+nt}{\PYGZlt{}/ap1\PYGZgt{}}
    \PYG{n+nt}{\PYGZlt{}/alumno\PYGZgt{}}
    \PYG{n+nt}{\PYGZlt{}alumno} \PYG{n+na}{dni=}\PYG{l+s}{\PYGZdq{}5778221D\PYGZdq{}}\PYG{n+nt}{\PYGZgt{}}
        \PYG{n+nt}{\PYGZlt{}nombre}\PYG{n+nt}{\PYGZgt{}}Andres\PYG{n+nt}{\PYGZlt{}/nombre\PYGZgt{}}
        \PYG{n+nt}{\PYGZlt{}ap1}\PYG{n+nt}{\PYGZgt{}}Ruiz\PYG{n+nt}{\PYGZlt{}/ap1\PYGZgt{}}
        \PYG{c}{\PYGZlt{}!\PYGZhy{}\PYGZhy{}}\PYG{c}{Ap2 y edad son optativos}\PYG{c}{\PYGZhy{}\PYGZhy{}\PYGZgt{}}
        \PYG{n+nt}{\PYGZlt{}ap2}\PYG{n+nt}{\PYGZgt{}}Ruiz\PYG{n+nt}{\PYGZlt{}/ap2\PYGZgt{}}
        \PYG{c}{\PYGZlt{}!\PYGZhy{}\PYGZhy{}}\PYG{c}{La edad debe ser positiva}\PYG{c}{\PYGZhy{}\PYGZhy{}\PYGZgt{}}
        \PYG{n+nt}{\PYGZlt{}edad}\PYG{n+nt}{\PYGZgt{}}25\PYG{n+nt}{\PYGZlt{}/edad\PYGZgt{}}
    \PYG{n+nt}{\PYGZlt{}/alumno\PYGZgt{}}
\PYG{n+nt}{\PYGZlt{}/listaalumnos\PYGZgt{}}
\end{sphinxVerbatim}

Y a continuación una posible solución:

\begin{sphinxVerbatim}[commandchars=\\\{\}]
\PYG{n+nt}{\PYGZlt{}xsd:schema} \PYG{n+na}{xmlns:xsd=}\PYG{l+s}{\PYGZdq{}http://www.w3.org/2001/XMLSchema\PYGZdq{}}\PYG{n+nt}{\PYGZgt{}}
    \PYG{n+nt}{\PYGZlt{}xsd:element} \PYG{n+na}{name=}\PYG{l+s}{\PYGZdq{}listaalumnos\PYGZdq{}} \PYG{n+na}{type=}\PYG{l+s}{\PYGZdq{}tipoListaAlumnos\PYGZdq{}}\PYG{n+nt}{/\PYGZgt{}}
    \PYG{n+nt}{\PYGZlt{}xsd:complexType} \PYG{n+na}{name=}\PYG{l+s}{\PYGZdq{}tipoListaAlumnos\PYGZdq{}}\PYG{n+nt}{\PYGZgt{}}
        \PYG{n+nt}{\PYGZlt{}xsd:complexContent}\PYG{n+nt}{\PYGZgt{}}
            \PYG{n+nt}{\PYGZlt{}xsd:restriction} \PYG{n+na}{base=}\PYG{l+s}{\PYGZdq{}xsd:anyType\PYGZdq{}}\PYG{n+nt}{\PYGZgt{}}
                \PYG{n+nt}{\PYGZlt{}xsd:sequence}\PYG{n+nt}{\PYGZgt{}}
                    \PYG{n+nt}{\PYGZlt{}xsd:element} \PYG{n+na}{name=}\PYG{l+s}{\PYGZdq{}alumno\PYGZdq{}}
                                 \PYG{n+na}{type=}\PYG{l+s}{\PYGZdq{}tipoAlumno\PYGZdq{}}
                                 \PYG{n+na}{maxOccurs=}\PYG{l+s}{\PYGZdq{}unbounded\PYGZdq{}}\PYG{n+nt}{/\PYGZgt{}}
                \PYG{n+nt}{\PYGZlt{}/xsd:sequence\PYGZgt{}}
            \PYG{n+nt}{\PYGZlt{}/xsd:restriction\PYGZgt{}}
        \PYG{n+nt}{\PYGZlt{}/xsd:complexContent\PYGZgt{}}
    \PYG{n+nt}{\PYGZlt{}/xsd:complexType\PYGZgt{}}
    \PYG{n+nt}{\PYGZlt{}xsd:complexType} \PYG{n+na}{name=}\PYG{l+s}{\PYGZdq{}tipoAlumno\PYGZdq{}}\PYG{n+nt}{\PYGZgt{}}
        \PYG{n+nt}{\PYGZlt{}xsd:complexContent}\PYG{n+nt}{\PYGZgt{}}
            \PYG{n+nt}{\PYGZlt{}xsd:restriction} \PYG{n+na}{base=}\PYG{l+s}{\PYGZdq{}xsd:anyType\PYGZdq{}}\PYG{n+nt}{\PYGZgt{}}
                \PYG{n+nt}{\PYGZlt{}xsd:sequence}\PYG{n+nt}{\PYGZgt{}}
                    \PYG{n+nt}{\PYGZlt{}xsd:element} \PYG{n+na}{name=}\PYG{l+s}{\PYGZdq{}nombre\PYGZdq{}}
                                 \PYG{n+na}{type=}\PYG{l+s}{\PYGZdq{}xsd:string\PYGZdq{}}\PYG{n+nt}{/\PYGZgt{}}
                    \PYG{n+nt}{\PYGZlt{}xsd:element} \PYG{n+na}{name=}\PYG{l+s}{\PYGZdq{}ap1\PYGZdq{}}
                                \PYG{n+na}{type=}\PYG{l+s}{\PYGZdq{}xsd:string\PYGZdq{}}\PYG{n+nt}{/\PYGZgt{}}
                    \PYG{n+nt}{\PYGZlt{}xsd:element} \PYG{n+na}{name=}\PYG{l+s}{\PYGZdq{}ap2\PYGZdq{}}
                                \PYG{n+na}{type=}\PYG{l+s}{\PYGZdq{}xsd:string\PYGZdq{}}
                                \PYG{n+na}{minOccurs=}\PYG{l+s}{\PYGZdq{}0\PYGZdq{}}\PYG{n+nt}{/\PYGZgt{}}
                    \PYG{n+nt}{\PYGZlt{}xsd:element} \PYG{n+na}{name=}\PYG{l+s}{\PYGZdq{}edad\PYGZdq{}}
                                 \PYG{n+na}{type=}\PYG{l+s}{\PYGZdq{}xsd:positiveInteger\PYGZdq{}}
                                 \PYG{n+na}{minOccurs=}\PYG{l+s}{\PYGZdq{}0\PYGZdq{}}\PYG{n+nt}{/\PYGZgt{}}
                \PYG{n+nt}{\PYGZlt{}/xsd:sequence\PYGZgt{}}
                \PYG{n+nt}{\PYGZlt{}xsd:attribute} \PYG{n+na}{name=}\PYG{l+s}{\PYGZdq{}dni\PYGZdq{}} \PYG{n+na}{type=}\PYG{l+s}{\PYGZdq{}tipoDNI\PYGZdq{}}\PYG{n+nt}{/\PYGZgt{}}
            \PYG{n+nt}{\PYGZlt{}/xsd:restriction\PYGZgt{}}
        \PYG{n+nt}{\PYGZlt{}/xsd:complexContent\PYGZgt{}}
    \PYG{n+nt}{\PYGZlt{}/xsd:complexType\PYGZgt{}}
    \PYG{n+nt}{\PYGZlt{}xsd:simpleType} \PYG{n+na}{name=}\PYG{l+s}{\PYGZdq{}tipoDNI\PYGZdq{}}\PYG{n+nt}{\PYGZgt{}}
        \PYG{n+nt}{\PYGZlt{}xsd:restriction} \PYG{n+na}{base=}\PYG{l+s}{\PYGZdq{}xsd:string\PYGZdq{}}\PYG{n+nt}{\PYGZgt{}}
            \PYG{n+nt}{\PYGZlt{}xsd:pattern} \PYG{n+na}{value=}\PYG{l+s}{\PYGZdq{}[0\PYGZhy{}9]\PYGZob{}7,8\PYGZcb{}[A\PYGZhy{}Z]\PYGZdq{}}\PYG{n+nt}{/\PYGZgt{}}
        \PYG{n+nt}{\PYGZlt{}/xsd:restriction\PYGZgt{}}
    \PYG{n+nt}{\PYGZlt{}/xsd:simpleType\PYGZgt{}}

\PYG{n+nt}{\PYGZlt{}/xsd:schema\PYGZgt{}}
\end{sphinxVerbatim}


\subsection{Ejercicio: lista de articulos (con atributos optativos)}
\label{\detokenize{tema5:ejercicio-lista-de-articulos-con-atributos-optativos}}
Supongamos el fichero siguiente con las reglas que se explicitan en los comentarios:

\begin{sphinxVerbatim}[commandchars=\\\{\}]
\PYG{n+nt}{\PYGZlt{}listaproductos}\PYG{n+nt}{\PYGZgt{}}
    \PYG{n+nt}{\PYGZlt{}articulo}\PYG{n+nt}{\PYGZgt{}}
        \PYG{c}{\PYGZlt{}!\PYGZhy{}\PYGZhy{}}\PYG{c}{Estructura 2 letras,2 cifras}\PYG{c}{\PYGZhy{}\PYGZhy{}\PYGZgt{}}
        \PYG{n+nt}{\PYGZlt{}codigo}\PYG{n+nt}{\PYGZgt{}}CD12\PYG{n+nt}{\PYGZlt{}/codigo\PYGZgt{}}
        \PYG{c}{\PYGZlt{}!\PYGZhy{}\PYGZhy{}}\PYG{c}{Descripcion es optativo y su atributo autor tb}\PYG{c}{\PYGZhy{}\PYGZhy{}\PYGZgt{}}
        \PYG{n+nt}{\PYGZlt{}descripcion} \PYG{n+na}{autor=}\PYG{l+s}{\PYGZdq{}Pepe\PYGZdq{}}\PYG{n+nt}{\PYGZgt{}}Monitor\PYG{n+nt}{\PYGZlt{}/descripcion\PYGZgt{}}
    \PYG{n+nt}{\PYGZlt{}/articulo\PYGZgt{}}
    \PYG{n+nt}{\PYGZlt{}articulo}\PYG{n+nt}{\PYGZgt{}}
        \PYG{n+nt}{\PYGZlt{}codigo}\PYG{n+nt}{\PYGZgt{}}CA12\PYG{n+nt}{\PYGZlt{}/codigo\PYGZgt{}}
    \PYG{n+nt}{\PYGZlt{}/articulo\PYGZgt{}}
    \PYG{n+nt}{\PYGZlt{}articulo}\PYG{n+nt}{\PYGZgt{}}
        \PYG{n+nt}{\PYGZlt{}codigo}\PYG{n+nt}{\PYGZgt{}}AA99\PYG{n+nt}{\PYGZlt{}/codigo\PYGZgt{}}
        \PYG{n+nt}{\PYGZlt{}descripcion}\PYG{n+nt}{\PYGZgt{}}Teclado\PYG{n+nt}{\PYGZlt{}/descripcion\PYGZgt{}}
    \PYG{n+nt}{\PYGZlt{}/articulo\PYGZgt{}}
\PYG{n+nt}{\PYGZlt{}/listaproductos\PYGZgt{}}
\end{sphinxVerbatim}

A continuación se muestra una solución con un esquema que valida ficheros como el indicado:

\begin{sphinxVerbatim}[commandchars=\\\{\}]
\PYG{n+nt}{\PYGZlt{}xsd:schema} \PYG{n+na}{xmlns:xsd=}\PYG{l+s}{\PYGZdq{}http://www.w3.org/2001/XMLSchema\PYGZdq{}}\PYG{n+nt}{\PYGZgt{}}
    \PYG{n+nt}{\PYGZlt{}xsd:element} \PYG{n+na}{name=}\PYG{l+s}{\PYGZdq{}listaproductos\PYGZdq{}} \PYG{n+na}{type=}\PYG{l+s}{\PYGZdq{}tipoListaProductos\PYGZdq{}}\PYG{n+nt}{/\PYGZgt{}}
    \PYG{n+nt}{\PYGZlt{}xsd:complexType} \PYG{n+na}{name=}\PYG{l+s}{\PYGZdq{}tipoListaProductos\PYGZdq{}}\PYG{n+nt}{\PYGZgt{}}
        \PYG{n+nt}{\PYGZlt{}xsd:complexContent}\PYG{n+nt}{\PYGZgt{}}
            \PYG{n+nt}{\PYGZlt{}xsd:restriction} \PYG{n+na}{base=}\PYG{l+s}{\PYGZdq{}xsd:anyType\PYGZdq{}}\PYG{n+nt}{\PYGZgt{}}
                \PYG{n+nt}{\PYGZlt{}xsd:sequence}\PYG{n+nt}{\PYGZgt{}}
                    \PYG{n+nt}{\PYGZlt{}xsd:element} \PYG{n+na}{name=}\PYG{l+s}{\PYGZdq{}articulo\PYGZdq{}}
                                 \PYG{n+na}{type=}\PYG{l+s}{\PYGZdq{}tipoArticulo\PYGZdq{}}
                                 \PYG{n+na}{maxOccurs=}\PYG{l+s}{\PYGZdq{}unbounded\PYGZdq{}}\PYG{n+nt}{/\PYGZgt{}}
                \PYG{n+nt}{\PYGZlt{}/xsd:sequence\PYGZgt{}}
            \PYG{n+nt}{\PYGZlt{}/xsd:restriction\PYGZgt{}}
        \PYG{n+nt}{\PYGZlt{}/xsd:complexContent\PYGZgt{}}
    \PYG{n+nt}{\PYGZlt{}/xsd:complexType\PYGZgt{}} \PYG{c}{\PYGZlt{}!\PYGZhy{}\PYGZhy{}}\PYG{c}{Fin de listaarticulos}\PYG{c}{\PYGZhy{}\PYGZhy{}\PYGZgt{}}
    \PYG{n+nt}{\PYGZlt{}xsd:complexType} \PYG{n+na}{name=}\PYG{l+s}{\PYGZdq{}tipoArticulo\PYGZdq{}}\PYG{n+nt}{\PYGZgt{}}
        \PYG{n+nt}{\PYGZlt{}xsd:complexContent}\PYG{n+nt}{\PYGZgt{}}
            \PYG{n+nt}{\PYGZlt{}xsd:restriction} \PYG{n+na}{base=}\PYG{l+s}{\PYGZdq{}xsd:anyType\PYGZdq{}}\PYG{n+nt}{\PYGZgt{}}
                \PYG{n+nt}{\PYGZlt{}xsd:sequence}\PYG{n+nt}{\PYGZgt{}}
                    \PYG{n+nt}{\PYGZlt{}xsd:element} \PYG{n+na}{name=}\PYG{l+s}{\PYGZdq{}codigo\PYGZdq{}} \PYG{n+na}{type=}\PYG{l+s}{\PYGZdq{}tipoCodigo\PYGZdq{}}\PYG{n+nt}{/\PYGZgt{}}
                    \PYG{n+nt}{\PYGZlt{}xsd:element} \PYG{n+na}{name=}\PYG{l+s}{\PYGZdq{}descripcion\PYGZdq{}}
                                 \PYG{n+na}{type=}\PYG{l+s}{\PYGZdq{}tipoDescripcion\PYGZdq{}}
                                 \PYG{n+na}{minOccurs=}\PYG{l+s}{\PYGZdq{}0\PYGZdq{}} \PYG{n+na}{maxOccurs=}\PYG{l+s}{\PYGZdq{}1\PYGZdq{}}\PYG{n+nt}{/\PYGZgt{}}
                \PYG{n+nt}{\PYGZlt{}/xsd:sequence\PYGZgt{}}
            \PYG{n+nt}{\PYGZlt{}/xsd:restriction\PYGZgt{}}
        \PYG{n+nt}{\PYGZlt{}/xsd:complexContent\PYGZgt{}}
    \PYG{n+nt}{\PYGZlt{}/xsd:complexType\PYGZgt{}} \PYG{c}{\PYGZlt{}!\PYGZhy{}\PYGZhy{}}\PYG{c}{Fin de  articulo}\PYG{c}{\PYGZhy{}\PYGZhy{}\PYGZgt{}}

    \PYG{n+nt}{\PYGZlt{}xsd:simpleType} \PYG{n+na}{name=}\PYG{l+s}{\PYGZdq{}tipoCodigo\PYGZdq{}}\PYG{n+nt}{\PYGZgt{}}
        \PYG{n+nt}{\PYGZlt{}xsd:restriction} \PYG{n+na}{base=}\PYG{l+s}{\PYGZdq{}xsd:string\PYGZdq{}}\PYG{n+nt}{\PYGZgt{}}
            \PYG{n+nt}{\PYGZlt{}xsd:pattern} \PYG{n+na}{value=}\PYG{l+s}{\PYGZdq{}[A\PYGZhy{}Z]\PYGZob{}2\PYGZcb{}[0\PYGZhy{}9]\PYGZob{}2\PYGZcb{}\PYGZdq{}}\PYG{n+nt}{/\PYGZgt{}}
        \PYG{n+nt}{\PYGZlt{}/xsd:restriction\PYGZgt{}}
    \PYG{n+nt}{\PYGZlt{}/xsd:simpleType\PYGZgt{}} \PYG{c}{\PYGZlt{}!\PYGZhy{}\PYGZhy{}}\PYG{c}{Fin de codigo}\PYG{c}{\PYGZhy{}\PYGZhy{}\PYGZgt{}}

    \PYG{n+nt}{\PYGZlt{}xsd:complexType} \PYG{n+na}{name=}\PYG{l+s}{\PYGZdq{}tipoDescripcion\PYGZdq{}}\PYG{n+nt}{\PYGZgt{}}
        \PYG{n+nt}{\PYGZlt{}xsd:simpleContent}\PYG{n+nt}{\PYGZgt{}}
            \PYG{n+nt}{\PYGZlt{}xsd:extension} \PYG{n+na}{base=}\PYG{l+s}{\PYGZdq{}xsd:string\PYGZdq{}}\PYG{n+nt}{\PYGZgt{}}
                \PYG{n+nt}{\PYGZlt{}xsd:attribute} \PYG{n+na}{name=}\PYG{l+s}{\PYGZdq{}autor\PYGZdq{}} \PYG{n+na}{type=}\PYG{l+s}{\PYGZdq{}xsd:string\PYGZdq{}}\PYG{n+nt}{/\PYGZgt{}}
            \PYG{n+nt}{\PYGZlt{}/xsd:extension\PYGZgt{}}
        \PYG{n+nt}{\PYGZlt{}/xsd:simpleContent\PYGZgt{}}
    \PYG{n+nt}{\PYGZlt{}/xsd:complexType\PYGZgt{}}
\PYG{n+nt}{\PYGZlt{}/xsd:schema\PYGZgt{}}
\end{sphinxVerbatim}


\subsection{Ejercicio: lista de componentes (un enfoque distinto)}
\label{\detokenize{tema5:ejercicio-lista-de-componentes-un-enfoque-distinto}}
Dado un archivo como el siguiente en el cual aparecen
las reglas incluidas como comentarios, crear el esquema que
valide la estructura de tales ficheros:

\begin{sphinxVerbatim}[commandchars=\\\{\}]
\PYG{n+nt}{\PYGZlt{}listacomponentes}\PYG{n+nt}{\PYGZgt{}}
    \PYG{c}{\PYGZlt{}!\PYGZhy{}\PYGZhy{}}\PYG{c}{Obligatoria fecha entrega}\PYG{c}{\PYGZhy{}\PYGZhy{}\PYGZgt{}}
    \PYG{n+nt}{\PYGZlt{}componente} \PYG{n+na}{entrega=}\PYG{l+s}{\PYGZdq{}2018\PYGZhy{}03\PYGZhy{}15\PYGZdq{}}\PYG{n+nt}{\PYGZgt{}}
        \PYG{n+nt}{\PYGZlt{}fabricante}\PYG{n+nt}{\PYGZgt{}}
            \PYG{c}{\PYGZlt{}!\PYGZhy{}\PYGZhy{}}\PYG{c}{Posibles fabricantes FAB1, FAB2 y FAB3}\PYG{c}{\PYGZhy{}\PYGZhy{}\PYGZgt{}}
            \PYG{n+nt}{\PYGZlt{}nombre}\PYG{n+nt}{\PYGZgt{}}FAB1\PYG{n+nt}{\PYGZlt{}/nombre\PYGZgt{}}
            \PYG{c}{\PYGZlt{}!\PYGZhy{}\PYGZhy{}}\PYG{c}{Calificacion es un string y es optativa}\PYG{c}{\PYGZhy{}\PYGZhy{}\PYGZgt{}}
            \PYG{n+nt}{\PYGZlt{}calificacion}\PYG{n+nt}{\PYGZgt{}}Positiva\PYG{n+nt}{\PYGZlt{}/calificacion\PYGZgt{}}
        \PYG{n+nt}{\PYGZlt{}/fabricante\PYGZgt{}}
        \PYG{c}{\PYGZlt{}!\PYGZhy{}\PYGZhy{}}\PYG{c}{Atributo unidad es cadena. Dentro de peso}
\PYG{c}{        solo puede haber numeros con decimales y mayores de 0}\PYG{c}{\PYGZhy{}\PYGZhy{}\PYGZgt{}}
        \PYG{n+nt}{\PYGZlt{}peso} \PYG{n+na}{unidad=}\PYG{l+s}{\PYGZdq{}kg\PYGZdq{}}\PYG{n+nt}{\PYGZgt{}}40.5\PYG{n+nt}{\PYGZlt{}/peso\PYGZgt{}}
    \PYG{n+nt}{\PYGZlt{}/componente\PYGZgt{}}
    \PYG{n+nt}{\PYGZlt{}componente} \PYG{n+na}{entrega=}\PYG{l+s}{\PYGZdq{}2018\PYGZhy{}12\PYGZhy{}31\PYGZdq{}}\PYG{n+nt}{\PYGZgt{}}
        \PYG{n+nt}{\PYGZlt{}fabricante}\PYG{n+nt}{\PYGZgt{}}
            \PYG{n+nt}{\PYGZlt{}nombre}\PYG{n+nt}{\PYGZgt{}}FAB2\PYG{n+nt}{\PYGZlt{}/nombre\PYGZgt{}}
        \PYG{n+nt}{\PYGZlt{}/fabricante\PYGZgt{}}
        \PYG{n+nt}{\PYGZlt{}peso} \PYG{n+na}{unidad=}\PYG{l+s}{\PYGZdq{}miligramos\PYGZdq{}}\PYG{n+nt}{\PYGZgt{}}260.5\PYG{n+nt}{\PYGZlt{}/peso\PYGZgt{}}
    \PYG{n+nt}{\PYGZlt{}/componente\PYGZgt{}}
\PYG{n+nt}{\PYGZlt{}/listacomponentes\PYGZgt{}}
\end{sphinxVerbatim}

Ahora en lugar de ir definiendo tipos empezando por el elemento raíz vamos a ir definiendo primero los tipos de los elementos más básicos que encontremos, e iremos construyendo los tipos más complejos a partir de los tipos fáciles que ya hayamos construido. Como veremos despues, el resultado va a ser el mismo.

La solución:

\begin{sphinxVerbatim}[commandchars=\\\{\}]
n\PYG{n+nt}{\PYGZlt{}xsd:schema} \PYG{n+na}{xmlns:xsd=}\PYG{l+s}{\PYGZdq{}http://www.w3.org/2001/XMLSchema\PYGZdq{}}\PYG{n+nt}{\PYGZgt{}}
    \PYG{n+nt}{\PYGZlt{}xsd:simpleType} \PYG{n+na}{name=}\PYG{l+s}{\PYGZdq{}tipoNombre\PYGZdq{}}\PYG{n+nt}{\PYGZgt{}}
        \PYG{n+nt}{\PYGZlt{}xsd:restriction} \PYG{n+na}{base=}\PYG{l+s}{\PYGZdq{}xsd:string\PYGZdq{}}\PYG{n+nt}{\PYGZgt{}}
            \PYG{n+nt}{\PYGZlt{}xsd:enumeration} \PYG{n+na}{value=}\PYG{l+s}{\PYGZdq{}FAB1\PYGZdq{}}\PYG{n+nt}{/\PYGZgt{}}
            \PYG{n+nt}{\PYGZlt{}xsd:enumeration} \PYG{n+na}{value=}\PYG{l+s}{\PYGZdq{}FAB2\PYGZdq{}}\PYG{n+nt}{/\PYGZgt{}}
            \PYG{n+nt}{\PYGZlt{}xsd:enumeration} \PYG{n+na}{value=}\PYG{l+s}{\PYGZdq{}FAB3\PYGZdq{}}\PYG{n+nt}{/\PYGZgt{}}
        \PYG{n+nt}{\PYGZlt{}/xsd:restriction\PYGZgt{}}
    \PYG{n+nt}{\PYGZlt{}/xsd:simpleType\PYGZgt{}}
    \PYG{c}{\PYGZlt{}!\PYGZhy{}\PYGZhy{}}\PYG{c}{Tipo auxiliar, el atributo lo incluimos despues}\PYG{c}{\PYGZhy{}\PYGZhy{}\PYGZgt{}}
    \PYG{n+nt}{\PYGZlt{}xsd:simpleType} \PYG{n+na}{name=}\PYG{l+s}{\PYGZdq{}tipoPesoRestringido\PYGZdq{}}\PYG{n+nt}{\PYGZgt{}}
        \PYG{n+nt}{\PYGZlt{}xsd:restriction} \PYG{n+na}{base=}\PYG{l+s}{\PYGZdq{}xsd:float\PYGZdq{}}\PYG{n+nt}{\PYGZgt{}}
            \PYG{n+nt}{\PYGZlt{}xsd:minExclusive} \PYG{n+na}{value=}\PYG{l+s}{\PYGZdq{}0\PYGZdq{}}\PYG{n+nt}{/\PYGZgt{}}
        \PYG{n+nt}{\PYGZlt{}/xsd:restriction\PYGZgt{}}
    \PYG{n+nt}{\PYGZlt{}/xsd:simpleType\PYGZgt{}}

    \PYG{c}{\PYGZlt{}!\PYGZhy{}\PYGZhy{}}\PYG{c}{En este tipo peso incluimos ya el atributo}\PYG{c}{\PYGZhy{}\PYGZhy{}\PYGZgt{}}
    \PYG{n+nt}{\PYGZlt{}xsd:complexType} \PYG{n+na}{name=}\PYG{l+s}{\PYGZdq{}tipoPeso\PYGZdq{}}\PYG{n+nt}{\PYGZgt{}}
        \PYG{n+nt}{\PYGZlt{}xsd:simpleContent}\PYG{n+nt}{\PYGZgt{}}
            \PYG{n+nt}{\PYGZlt{}xsd:extension} \PYG{n+na}{base=}\PYG{l+s}{\PYGZdq{}tipoPesoRestringido\PYGZdq{}}\PYG{n+nt}{\PYGZgt{}}
                \PYG{n+nt}{\PYGZlt{}xsd:attribute} \PYG{n+na}{name=}\PYG{l+s}{\PYGZdq{}unidad\PYGZdq{}}
                               \PYG{n+na}{type=}\PYG{l+s}{\PYGZdq{}xsd:string\PYGZdq{}}\PYG{n+nt}{/\PYGZgt{}}
            \PYG{n+nt}{\PYGZlt{}/xsd:extension\PYGZgt{}}
        \PYG{n+nt}{\PYGZlt{}/xsd:simpleContent\PYGZgt{}}
    \PYG{n+nt}{\PYGZlt{}/xsd:complexType\PYGZgt{}}

    \PYG{n+nt}{\PYGZlt{}xsd:complexType} \PYG{n+na}{name=}\PYG{l+s}{\PYGZdq{}tipoFabricante\PYGZdq{}}\PYG{n+nt}{\PYGZgt{}}
        \PYG{n+nt}{\PYGZlt{}xsd:complexContent}\PYG{n+nt}{\PYGZgt{}}
            \PYG{n+nt}{\PYGZlt{}xsd:restriction} \PYG{n+na}{base=}\PYG{l+s}{\PYGZdq{}xsd:anyType\PYGZdq{}}\PYG{n+nt}{\PYGZgt{}}
                \PYG{n+nt}{\PYGZlt{}xsd:sequence}\PYG{n+nt}{\PYGZgt{}}
                    \PYG{n+nt}{\PYGZlt{}xsd:element} \PYG{n+na}{name=}\PYG{l+s}{\PYGZdq{}nombre\PYGZdq{}}
                                 \PYG{n+na}{type=}\PYG{l+s}{\PYGZdq{}tipoNombre\PYGZdq{}}\PYG{n+nt}{/\PYGZgt{}}
                    \PYG{n+nt}{\PYGZlt{}xsd:element} \PYG{n+na}{name=}\PYG{l+s}{\PYGZdq{}calificacion\PYGZdq{}}
                                 \PYG{n+na}{type=}\PYG{l+s}{\PYGZdq{}xsd:string\PYGZdq{}}
                                 \PYG{n+na}{minOccurs=}\PYG{l+s}{\PYGZdq{}0\PYGZdq{}}\PYG{n+nt}{/\PYGZgt{}}
                \PYG{n+nt}{\PYGZlt{}/xsd:sequence\PYGZgt{}}
            \PYG{n+nt}{\PYGZlt{}/xsd:restriction\PYGZgt{}}
        \PYG{n+nt}{\PYGZlt{}/xsd:complexContent\PYGZgt{}}
    \PYG{n+nt}{\PYGZlt{}/xsd:complexType\PYGZgt{}}
    \PYG{n+nt}{\PYGZlt{}xsd:complexType} \PYG{n+na}{name=}\PYG{l+s}{\PYGZdq{}tipoComponente\PYGZdq{}}\PYG{n+nt}{\PYGZgt{}}
        \PYG{n+nt}{\PYGZlt{}xsd:sequence}\PYG{n+nt}{\PYGZgt{}}
            \PYG{n+nt}{\PYGZlt{}xsd:element} \PYG{n+na}{name=}\PYG{l+s}{\PYGZdq{}fabricante\PYGZdq{}}
                         \PYG{n+na}{type=}\PYG{l+s}{\PYGZdq{}tipoFabricante\PYGZdq{}}\PYG{n+nt}{/\PYGZgt{}}
            \PYG{n+nt}{\PYGZlt{}xsd:element} \PYG{n+na}{name=}\PYG{l+s}{\PYGZdq{}peso\PYGZdq{}} \PYG{n+na}{type=}\PYG{l+s}{\PYGZdq{}tipoPeso\PYGZdq{}}\PYG{n+nt}{/\PYGZgt{}}
        \PYG{n+nt}{\PYGZlt{}/xsd:sequence\PYGZgt{}}
        \PYG{n+nt}{\PYGZlt{}xsd:attribute} \PYG{n+na}{name=}\PYG{l+s}{\PYGZdq{}entrega\PYGZdq{}} \PYG{n+na}{type=}\PYG{l+s}{\PYGZdq{}xsd:date\PYGZdq{}}\PYG{n+nt}{/\PYGZgt{}}

    \PYG{n+nt}{\PYGZlt{}/xsd:complexType\PYGZgt{}}

    \PYG{n+nt}{\PYGZlt{}xsd:complexType} \PYG{n+na}{name=}\PYG{l+s}{\PYGZdq{}tipoListaComponentes\PYGZdq{}}\PYG{n+nt}{\PYGZgt{}}
        \PYG{n+nt}{\PYGZlt{}xsd:sequence}\PYG{n+nt}{\PYGZgt{}}
            \PYG{n+nt}{\PYGZlt{}xsd:element} \PYG{n+na}{name=}\PYG{l+s}{\PYGZdq{}componente\PYGZdq{}}
                         \PYG{n+na}{type=}\PYG{l+s}{\PYGZdq{}tipoComponente\PYGZdq{}}
                         \PYG{n+na}{maxOccurs=}\PYG{l+s}{\PYGZdq{}unbounded\PYGZdq{}}\PYG{n+nt}{/\PYGZgt{}}
        \PYG{n+nt}{\PYGZlt{}/xsd:sequence\PYGZgt{}}
    \PYG{n+nt}{\PYGZlt{}/xsd:complexType\PYGZgt{}}

    \PYG{c}{\PYGZlt{}!\PYGZhy{}\PYGZhy{}}\PYG{c}{Aunque el elemento aparezca al final,}
\PYG{c}{    no pasa nada}\PYG{c}{\PYGZhy{}\PYGZhy{}\PYGZgt{}}
    \PYG{n+nt}{\PYGZlt{}xsd:element} \PYG{n+na}{name=}\PYG{l+s}{\PYGZdq{}listacomponentes\PYGZdq{}}
                 \PYG{n+na}{type=}\PYG{l+s}{\PYGZdq{}tipoListaComponentes\PYGZdq{}}\PYG{n+nt}{/\PYGZgt{}}

\PYG{n+nt}{\PYGZlt{}/xsd:schema\PYGZgt{}}
\end{sphinxVerbatim}


\subsection{Ejercicio: listas con choice}
\label{\detokenize{tema5:ejercicio-listas-con-choice}}
Se pide elaborar un esquema que valide un fichero con las restricciones siguientes:
\begin{itemize}
\item {} 
El elemento raíz es \sphinxcode{articulos}. Dicho elemento raíz debe llevar siempre un atributo \sphinxcode{fechaGeneración}.

\item {} 
Dentro de la raíz puede haber uno o varios de cualquiera de los siguientes elementos: \sphinxcode{monitor}, \sphinxcode{teclado} o \sphinxcode{raton}. Cualquiera de los tres elementos puede llevar un atributo \sphinxcode{codigo} que tiene siempre la estructura «tres letras, guión, tres letras, guión, tres cifras». Además, cualquiera de los tres debe llevar dentro y en primer lugar un elemento \sphinxcode{descripción} que contiene texto.

\item {} 
Un monitor debe llevar (aparte de la descripción que va en primer lugar) un elemento \sphinxcode{resolución} que a su vez debe llevar dentro dos elementos y en este orden \sphinxcode{ancho} y \sphinxcode{alto}. Tanto \sphinxcode{ancho}, como \sphinxcode{alto} deben llevar siempre dentro un entero positivo.

\item {} 
Un \sphinxcode{ratón} debe llevar (aparte de la descripción que va en primer lugar) un elemento \sphinxcode{peso} que siempre lleva dentro un entero positivo. Además, el \sphinxcode{peso} lleva siempre dentro un atributo \sphinxcode{unidad} que solo puede valer «g» o «cg».

\end{itemize}

En el fichero siguiente se muestra un ejemplo

\begin{sphinxVerbatim}[commandchars=\\\{\}]
\PYG{c}{\PYGZlt{}!\PYGZhy{}\PYGZhy{}}\PYG{c}{Obligatorio el tener fechaGeneracion}\PYG{c}{\PYGZhy{}\PYGZhy{}\PYGZgt{}}
\PYG{n+nt}{\PYGZlt{}articulos} \PYG{n+na}{fechageneracion=}\PYG{l+s}{\PYGZdq{}2018\PYGZhy{}03\PYGZhy{}01\PYGZdq{}}\PYG{n+nt}{\PYGZgt{}}
    \PYG{c}{\PYGZlt{}!\PYGZhy{}\PYGZhy{}}\PYG{c}{El atributo codigo es optativo siempre}\PYG{c}{\PYGZhy{}\PYGZhy{}\PYGZgt{}}
    \PYG{n+nt}{\PYGZlt{}monitor} \PYG{n+na}{codigo=}\PYG{l+s}{\PYGZdq{}AAA\PYGZhy{}DDD\PYGZhy{}222\PYGZdq{}}\PYG{n+nt}{\PYGZgt{}}
        \PYG{c}{\PYGZlt{}!\PYGZhy{}\PYGZhy{}}\PYG{c}{Descripcion obligatoria}\PYG{c}{\PYGZhy{}\PYGZhy{}\PYGZgt{}}
        \PYG{n+nt}{\PYGZlt{}descripcion}\PYG{n+nt}{\PYGZgt{}}Monitor de x pulgadas...\PYG{n+nt}{\PYGZlt{}/descripcion\PYGZgt{}}
        \PYG{n+nt}{\PYGZlt{}resolucion}\PYG{n+nt}{\PYGZgt{}}
            \PYG{n+nt}{\PYGZlt{}ancho}\PYG{n+nt}{\PYGZgt{}}1920\PYG{n+nt}{\PYGZlt{}/ancho\PYGZgt{}}
            \PYG{n+nt}{\PYGZlt{}alto}\PYG{n+nt}{\PYGZgt{}}1400\PYG{n+nt}{\PYGZlt{}/alto\PYGZgt{}}
        \PYG{n+nt}{\PYGZlt{}/resolucion\PYGZgt{}}
    \PYG{n+nt}{\PYGZlt{}/monitor\PYGZgt{}}
    \PYG{n+nt}{\PYGZlt{}raton}\PYG{n+nt}{\PYGZgt{}}
        \PYG{c}{\PYGZlt{}!\PYGZhy{}\PYGZhy{}}\PYG{c}{Descripcion obligatoria}\PYG{c}{\PYGZhy{}\PYGZhy{}\PYGZgt{}}
        \PYG{n+nt}{\PYGZlt{}descripcion}\PYG{n+nt}{\PYGZgt{}}Raton ergonomico...\PYG{n+nt}{\PYGZlt{}/descripcion\PYGZgt{}}
        \PYG{c}{\PYGZlt{}!\PYGZhy{}\PYGZhy{}}\PYG{c}{La unidad es g o cg}\PYG{c}{\PYGZhy{}\PYGZhy{}\PYGZgt{}}
        \PYG{n+nt}{\PYGZlt{}peso} \PYG{n+na}{unidad=}\PYG{l+s}{\PYGZdq{}g\PYGZdq{}}\PYG{n+nt}{\PYGZgt{}}100\PYG{n+nt}{\PYGZlt{}/peso\PYGZgt{}}
    \PYG{n+nt}{\PYGZlt{}/raton\PYGZgt{}}
    \PYG{n+nt}{\PYGZlt{}teclado} \PYG{n+na}{codigo=}\PYG{l+s}{\PYGZdq{}DDD\PYGZhy{}XXX\PYGZhy{}111\PYGZdq{}}\PYG{n+nt}{\PYGZgt{}}
        \PYG{c}{\PYGZlt{}!\PYGZhy{}\PYGZhy{}}\PYG{c}{\PYGZhy{}}\PYG{c}{Descripcion obligatoria}\PYG{c}{\PYGZhy{}\PYGZhy{}\PYGZgt{}}
        \PYG{n+nt}{\PYGZlt{}descripcion}\PYG{n+nt}{\PYGZgt{}}Teclado estándar\PYG{n+nt}{\PYGZlt{}/descripcion\PYGZgt{}}
    \PYG{n+nt}{\PYGZlt{}/teclado\PYGZgt{}}
    \PYG{n+nt}{\PYGZlt{}monitor} \PYG{n+na}{codigo=}\PYG{l+s}{\PYGZdq{}CCC\PYGZhy{}GGG\PYGZhy{}666\PYGZdq{}}\PYG{n+nt}{\PYGZgt{}}
        \PYG{c}{\PYGZlt{}!\PYGZhy{}\PYGZhy{}}\PYG{c}{Descripcion obligatoria}\PYG{c}{\PYGZhy{}\PYGZhy{}\PYGZgt{}}
        \PYG{n+nt}{\PYGZlt{}descripcion}\PYG{n+nt}{\PYGZgt{}}Monitor de x pulgadas...\PYG{n+nt}{\PYGZlt{}/descripcion\PYGZgt{}}
        \PYG{n+nt}{\PYGZlt{}resolucion}\PYG{n+nt}{\PYGZgt{}}
            \PYG{n+nt}{\PYGZlt{}ancho}\PYG{n+nt}{\PYGZgt{}}1400\PYG{n+nt}{\PYGZlt{}/ancho\PYGZgt{}}
            \PYG{n+nt}{\PYGZlt{}alto}\PYG{n+nt}{\PYGZgt{}}1000\PYG{n+nt}{\PYGZlt{}/alto\PYGZgt{}}
        \PYG{n+nt}{\PYGZlt{}/resolucion\PYGZgt{}}
    \PYG{n+nt}{\PYGZlt{}/monitor\PYGZgt{}}
\PYG{n+nt}{\PYGZlt{}/articulos\PYGZgt{}}
\end{sphinxVerbatim}


\subsection{Ejercicio: listas de productos}
\label{\detokenize{tema5:ejercicio-listas-de-productos}}
Se pide validar correctamente algo como esto:

\begin{sphinxVerbatim}[commandchars=\\\{\}]
\PYG{n+nt}{\PYGZlt{}listaproductos}\PYG{n+nt}{\PYGZgt{}}
  \PYG{n+nt}{\PYGZlt{}producto} \PYG{n+na}{codigo=}\PYG{l+s}{\PYGZdq{}DX\PYGZhy{}22\PYGZdq{}}\PYG{n+nt}{\PYGZgt{}}\PYG{c}{\PYGZlt{}!\PYGZhy{}\PYGZhy{}}\PYG{c}{Codigo obligatorio}\PYG{c}{\PYGZhy{}\PYGZhy{}\PYGZgt{}}
    \PYG{n+nt}{\PYGZlt{}descripcion}\PYG{n+nt}{\PYGZgt{}}Ordenador\PYG{n+nt}{\PYGZlt{}/descripcion\PYGZgt{}}\PYG{c}{\PYGZlt{}!\PYGZhy{}\PYGZhy{}}\PYG{c}{Optativa}\PYG{c}{\PYGZhy{}\PYGZhy{}\PYGZgt{}}
    \PYG{n+nt}{\PYGZlt{}peso}\PYG{n+nt}{\PYGZgt{}}23.44\PYG{n+nt}{\PYGZlt{}/peso\PYGZgt{}}\PYG{c}{\PYGZlt{}!\PYGZhy{}\PYGZhy{}}\PYG{c}{Positivo con decimales}\PYG{c}{\PYGZhy{}\PYGZhy{}\PYGZgt{}}
  \PYG{n+nt}{\PYGZlt{}/producto\PYGZgt{}}
  \PYG{n+nt}{\PYGZlt{}producto} \PYG{n+na}{codigo=}\PYG{l+s}{\PYGZdq{}CX\PYGZhy{}124\PYGZdq{}}\PYG{n+nt}{\PYGZgt{}}
    \PYG{n+nt}{\PYGZlt{}peso}\PYG{n+nt}{\PYGZgt{}}17.50\PYG{n+nt}{\PYGZlt{}/peso\PYGZgt{}}
  \PYG{n+nt}{\PYGZlt{}/producto\PYGZgt{}}
  \PYG{n+nt}{\PYGZlt{}producto} \PYG{n+na}{codigo=}\PYG{l+s}{\PYGZdq{}CX\PYGZhy{}124\PYGZdq{}}\PYG{n+nt}{\PYGZgt{}}
    \PYG{n+nt}{\PYGZlt{}peso}\PYG{n+nt}{\PYGZgt{}}17.50\PYG{n+nt}{\PYGZlt{}/peso\PYGZgt{}}
  \PYG{n+nt}{\PYGZlt{}/producto\PYGZgt{}}
\PYG{n+nt}{\PYGZlt{}/listaproductos\PYGZgt{}}
\end{sphinxVerbatim}

Las reglas son:
\begin{enumerate}
\item {} 
Una lista de productos puede tener dentro muchos productos.

\item {} 
Todo producto tiene un «codigo» cuya estructura \sphinxstyleemphasis{dos mayúsculas seguidas de un guión seguido de dos o tres cifras}

\item {} 
Todo producto \sphinxstyleemphasis{puede tener (optativo)} un elemento descripción que es de tipo texto.

\item {} 
Todo producto \sphinxstylestrong{debe tener} un elemento peso que debe aceptar decimales pero que nunca puede ser negativo, es decir su valor mínimo es 0

\end{enumerate}

La solución se muestra a continuación:

\begin{sphinxVerbatim}[commandchars=\\\{\}]
\PYG{n+nt}{\PYGZlt{}xsd:schema}
  \PYG{n+na}{xmlns:xsd=}\PYG{l+s}{\PYGZdq{}http://www.w3.org/2001/XMLSchema\PYGZdq{}}\PYG{n+nt}{\PYGZgt{}}
  \PYG{n+nt}{\PYGZlt{}xsd:element} \PYG{n+na}{name=}\PYG{l+s}{\PYGZdq{}listaproductos\PYGZdq{}} \PYG{n+na}{type=}\PYG{l+s}{\PYGZdq{}tipoLista\PYGZdq{}}\PYG{n+nt}{/\PYGZgt{}}
  \PYG{n+nt}{\PYGZlt{}xsd:complexType} \PYG{n+na}{name=}\PYG{l+s}{\PYGZdq{}tipoLista\PYGZdq{}}\PYG{n+nt}{\PYGZgt{}}
    \PYG{n+nt}{\PYGZlt{}xsd:complexContent}\PYG{n+nt}{\PYGZgt{}}
      \PYG{n+nt}{\PYGZlt{}xsd:restriction} \PYG{n+na}{base=}\PYG{l+s}{\PYGZdq{}xsd:anyType\PYGZdq{}}\PYG{n+nt}{\PYGZgt{}}
        \PYG{n+nt}{\PYGZlt{}xsd:sequence}\PYG{n+nt}{\PYGZgt{}}
          \PYG{n+nt}{\PYGZlt{}xsd:element} \PYG{n+na}{name=}\PYG{l+s}{\PYGZdq{}producto\PYGZdq{}}
                       \PYG{n+na}{type=}\PYG{l+s}{\PYGZdq{}tipoProducto\PYGZdq{}}
                       \PYG{n+na}{maxOccurs=}\PYG{l+s}{\PYGZdq{}unbounded\PYGZdq{}}\PYG{n+nt}{/\PYGZgt{}}
        \PYG{n+nt}{\PYGZlt{}/xsd:sequence\PYGZgt{}}
      \PYG{n+nt}{\PYGZlt{}/xsd:restriction\PYGZgt{}}
    \PYG{n+nt}{\PYGZlt{}/xsd:complexContent\PYGZgt{}}
  \PYG{n+nt}{\PYGZlt{}/xsd:complexType\PYGZgt{}}
  \PYG{n+nt}{\PYGZlt{}xsd:complexType} \PYG{n+na}{name=}\PYG{l+s}{\PYGZdq{}tipoProducto\PYGZdq{}}\PYG{n+nt}{\PYGZgt{}}
    \PYG{n+nt}{\PYGZlt{}xsd:complexContent}\PYG{n+nt}{\PYGZgt{}}
      \PYG{n+nt}{\PYGZlt{}xsd:restriction} \PYG{n+na}{base=}\PYG{l+s}{\PYGZdq{}xsd:anyType\PYGZdq{}}\PYG{n+nt}{\PYGZgt{}}
        \PYG{n+nt}{\PYGZlt{}xsd:sequence}\PYG{n+nt}{\PYGZgt{}}
          \PYG{n+nt}{\PYGZlt{}xsd:element} \PYG{n+na}{name=}\PYG{l+s}{\PYGZdq{}descripcion\PYGZdq{}}
                       \PYG{n+na}{type=}\PYG{l+s}{\PYGZdq{}xsd:string\PYGZdq{}}
                       \PYG{n+na}{minOccurs=}\PYG{l+s}{\PYGZdq{}0\PYGZdq{}}\PYG{n+nt}{/\PYGZgt{}}
          \PYG{n+nt}{\PYGZlt{}xsd:element} \PYG{n+na}{name=}\PYG{l+s}{\PYGZdq{}peso\PYGZdq{}}
                       \PYG{n+na}{type=}\PYG{l+s}{\PYGZdq{}tipoPeso\PYGZdq{}}\PYG{n+nt}{/\PYGZgt{}}

        \PYG{n+nt}{\PYGZlt{}/xsd:sequence\PYGZgt{}}
        \PYG{n+nt}{\PYGZlt{}xsd:attribute} \PYG{n+na}{name=}\PYG{l+s}{\PYGZdq{}codigo\PYGZdq{}}
                       \PYG{n+na}{type=}\PYG{l+s}{\PYGZdq{}tipoCodigo\PYGZdq{}}
                       \PYG{n+na}{use=}\PYG{l+s}{\PYGZdq{}required\PYGZdq{}}\PYG{n+nt}{/\PYGZgt{}}
      \PYG{n+nt}{\PYGZlt{}/xsd:restriction\PYGZgt{}}
    \PYG{n+nt}{\PYGZlt{}/xsd:complexContent\PYGZgt{}}
  \PYG{n+nt}{\PYGZlt{}/xsd:complexType\PYGZgt{}}
  \PYG{n+nt}{\PYGZlt{}xsd:simpleType} \PYG{n+na}{name=}\PYG{l+s}{\PYGZdq{}tipoPeso\PYGZdq{}}\PYG{n+nt}{\PYGZgt{}}
    \PYG{n+nt}{\PYGZlt{}xsd:restriction} \PYG{n+na}{base=}\PYG{l+s}{\PYGZdq{}xsd:decimal\PYGZdq{}}\PYG{n+nt}{\PYGZgt{}}
      \PYG{n+nt}{\PYGZlt{}xsd:minInclusive} \PYG{n+na}{value=}\PYG{l+s}{\PYGZdq{}0\PYGZdq{}}\PYG{n+nt}{/\PYGZgt{}}
    \PYG{n+nt}{\PYGZlt{}/xsd:restriction\PYGZgt{}}
  \PYG{n+nt}{\PYGZlt{}/xsd:simpleType\PYGZgt{}}
  \PYG{n+nt}{\PYGZlt{}xsd:simpleType} \PYG{n+na}{name=}\PYG{l+s}{\PYGZdq{}tipoCodigo\PYGZdq{}}\PYG{n+nt}{\PYGZgt{}}
    \PYG{n+nt}{\PYGZlt{}xsd:restriction} \PYG{n+na}{base=}\PYG{l+s}{\PYGZdq{}xsd:string\PYGZdq{}}\PYG{n+nt}{\PYGZgt{}}
      \PYG{n+nt}{\PYGZlt{}xsd:pattern} \PYG{n+na}{value=}\PYG{l+s}{\PYGZdq{}[A\PYGZhy{}Z]\PYGZob{}2\PYGZcb{}\PYGZhy{}[0\PYGZhy{}9]\PYGZob{}2,3\PYGZcb{}\PYGZdq{}}\PYG{n+nt}{/\PYGZgt{}}
    \PYG{n+nt}{\PYGZlt{}/xsd:restriction\PYGZgt{}}
  \PYG{n+nt}{\PYGZlt{}/xsd:simpleType\PYGZgt{}}
\PYG{n+nt}{\PYGZlt{}/xsd:schema\PYGZgt{}}
\end{sphinxVerbatim}


\subsection{Ejercicio: validación de componentes}
\label{\detokenize{tema5:ejercicio-validacion-de-componentes}}
Validar un fichero como este:

\begin{sphinxVerbatim}[commandchars=\\\{\}]
\PYG{n+nt}{\PYGZlt{}listacomponentes}\PYG{n+nt}{\PYGZgt{}}
  \PYG{n+nt}{\PYGZlt{}componente}\PYG{n+nt}{\PYGZgt{}}
    \PYG{n+nt}{\PYGZlt{}tarjetagrafica}\PYG{n+nt}{\PYGZgt{}}
      \PYG{n+nt}{\PYGZlt{}memoria}\PYG{n+nt}{\PYGZgt{}}2GB\PYG{n+nt}{\PYGZlt{}/memoria\PYGZgt{}}
      \PYG{n+nt}{\PYGZlt{}precio} \PYG{n+na}{moneda=}\PYG{l+s}{\PYGZdq{}euros\PYGZdq{}}\PYG{n+nt}{\PYGZgt{}}190\PYG{n+nt}{\PYGZlt{}/precio\PYGZgt{}}
    \PYG{n+nt}{\PYGZlt{}/tarjetagrafica\PYGZgt{}}
  \PYG{n+nt}{\PYGZlt{}/componente\PYGZgt{}}
  \PYG{n+nt}{\PYGZlt{}componente} \PYG{n+na}{codigo=}\PYG{l+s}{\PYGZdq{}123456\PYGZdq{}}\PYG{n+nt}{\PYGZgt{}}
    \PYG{n+nt}{\PYGZlt{}monitor}\PYG{n+nt}{\PYGZgt{}}
      \PYG{n+nt}{\PYGZlt{}tamanio}\PYG{n+nt}{\PYGZgt{}}14\PYG{n+nt}{\PYGZlt{}/tamanio\PYGZgt{}}
      \PYG{n+nt}{\PYGZlt{}precio} \PYG{n+na}{moneda=}\PYG{l+s}{\PYGZdq{}euros\PYGZdq{}}\PYG{n+nt}{\PYGZgt{}}99.49\PYG{n+nt}{\PYGZlt{}/precio\PYGZgt{}}
    \PYG{n+nt}{\PYGZlt{}/monitor\PYGZgt{}}
  \PYG{n+nt}{\PYGZlt{}/componente\PYGZgt{}}
\PYG{n+nt}{\PYGZlt{}/listacomponentes\PYGZgt{}}
\end{sphinxVerbatim}

Las reglas son las siguientes:
\begin{enumerate}
\item {} 
El elemento raíz se llama \sphinxcode{listacomponentes}.

\item {} 
Dentro de él puede haber uno o más elementos \sphinxcode{componente}

\item {} 
Un componente puede ser una \sphinxcode{tarjetagrafica} o un \sphinxcode{monitor}.

\item {} 
Un componente puede tener un atributo llamado \sphinxcode{codigo} cuya estructura es siempre un dígito de 6 cifras.

\item {} 
Una tarjeta gráfica siempre tiene dos elementos llamados \sphinxcode{memoria} y \sphinxcode{precio}.

\item {} 
La memoria siempre es una cifra seguido de GB o TB.

\item {} 
El tamaño del monitor siempre es un entero positivo.

\item {} 
El precio siempre es una cantidad positiva con decimales. El precio siempre lleva un atributo \sphinxcode{moneda} que solo puede valer «euros» o «dolares» y que se utiliza para saber en qué moneda está el precio.

\end{enumerate}

La solución se muestra a continuación:

\begin{sphinxVerbatim}[commandchars=\\\{\}]
\PYG{n+nt}{\PYGZlt{}xsd:schema} \PYG{n+na}{xmlns:xsd=}\PYG{l+s}{\PYGZdq{}http://www.w3.org/2001/XMLSchema\PYGZdq{}}\PYG{n+nt}{\PYGZgt{}}
    \PYG{n+nt}{\PYGZlt{}xsd:element} \PYG{n+na}{name=}\PYG{l+s}{\PYGZdq{}listacomponentes\PYGZdq{}} \PYG{n+na}{type=}\PYG{l+s}{\PYGZdq{}tipoLista\PYGZdq{}}\PYG{n+nt}{/\PYGZgt{}}
    \PYG{n+nt}{\PYGZlt{}xsd:complexType} \PYG{n+na}{name=}\PYG{l+s}{\PYGZdq{}tipoLista\PYGZdq{}}\PYG{n+nt}{\PYGZgt{}}
        \PYG{n+nt}{\PYGZlt{}xsd:complexContent}\PYG{n+nt}{\PYGZgt{}}
            \PYG{n+nt}{\PYGZlt{}xsd:restriction} \PYG{n+na}{base=}\PYG{l+s}{\PYGZdq{}xsd:anyType\PYGZdq{}}\PYG{n+nt}{\PYGZgt{}}
                \PYG{n+nt}{\PYGZlt{}xsd:sequence}\PYG{n+nt}{\PYGZgt{}}
                    \PYG{n+nt}{\PYGZlt{}xsd:element} \PYG{n+na}{name=}\PYG{l+s}{\PYGZdq{}componente\PYGZdq{}}
                                 \PYG{n+na}{type=}\PYG{l+s}{\PYGZdq{}tipoComponente\PYGZdq{}}
                                 \PYG{n+na}{maxOccurs=}\PYG{l+s}{\PYGZdq{}unbounded\PYGZdq{}}\PYG{n+nt}{/\PYGZgt{}}
                \PYG{n+nt}{\PYGZlt{}/xsd:sequence\PYGZgt{}}
            \PYG{n+nt}{\PYGZlt{}/xsd:restriction\PYGZgt{}}
        \PYG{n+nt}{\PYGZlt{}/xsd:complexContent\PYGZgt{}}
    \PYG{n+nt}{\PYGZlt{}/xsd:complexType\PYGZgt{}}
    \PYG{n+nt}{\PYGZlt{}xsd:complexType} \PYG{n+na}{name=}\PYG{l+s}{\PYGZdq{}tipoComponente\PYGZdq{}}\PYG{n+nt}{\PYGZgt{}}
        \PYG{n+nt}{\PYGZlt{}xsd:complexContent}\PYG{n+nt}{\PYGZgt{}}
            \PYG{n+nt}{\PYGZlt{}xsd:restriction} \PYG{n+na}{base=}\PYG{l+s}{\PYGZdq{}xsd:anyType\PYGZdq{}}\PYG{n+nt}{\PYGZgt{}}
                \PYG{n+nt}{\PYGZlt{}xsd:choice}\PYG{n+nt}{\PYGZgt{}}
                    \PYG{n+nt}{\PYGZlt{}xsd:element} \PYG{n+na}{name=}\PYG{l+s}{\PYGZdq{}tarjetagrafica\PYGZdq{}} \PYG{n+na}{type=}\PYG{l+s}{\PYGZdq{}tipoTarjeta\PYGZdq{}}\PYG{n+nt}{/\PYGZgt{}}
                    \PYG{n+nt}{\PYGZlt{}xsd:element} \PYG{n+na}{name=}\PYG{l+s}{\PYGZdq{}monitor\PYGZdq{}} \PYG{n+na}{type=}\PYG{l+s}{\PYGZdq{}tipoMonitor\PYGZdq{}}\PYG{n+nt}{/\PYGZgt{}}
                \PYG{n+nt}{\PYGZlt{}/xsd:choice\PYGZgt{}}
                \PYG{n+nt}{\PYGZlt{}xsd:attribute} \PYG{n+na}{name=}\PYG{l+s}{\PYGZdq{}codigo\PYGZdq{}} \PYG{n+na}{type=}\PYG{l+s}{\PYGZdq{}tipoCodigo\PYGZdq{}}\PYG{n+nt}{/\PYGZgt{}}
            \PYG{n+nt}{\PYGZlt{}/xsd:restriction\PYGZgt{}}
        \PYG{n+nt}{\PYGZlt{}/xsd:complexContent\PYGZgt{}}
    \PYG{n+nt}{\PYGZlt{}/xsd:complexType\PYGZgt{}}
    \PYG{n+nt}{\PYGZlt{}xsd:simpleType} \PYG{n+na}{name=}\PYG{l+s}{\PYGZdq{}tipoCodigo\PYGZdq{}}\PYG{n+nt}{\PYGZgt{}}
        \PYG{n+nt}{\PYGZlt{}xsd:restriction} \PYG{n+na}{base=}\PYG{l+s}{\PYGZdq{}xsd:string\PYGZdq{}}\PYG{n+nt}{\PYGZgt{}}
            \PYG{n+nt}{\PYGZlt{}xsd:pattern} \PYG{n+na}{value=}\PYG{l+s}{\PYGZdq{}[1\PYGZhy{}9][0\PYGZhy{}9]\PYGZob{}5\PYGZcb{}\PYGZdq{}}\PYG{n+nt}{/\PYGZgt{}}
        \PYG{n+nt}{\PYGZlt{}/xsd:restriction\PYGZgt{}}
    \PYG{n+nt}{\PYGZlt{}/xsd:simpleType\PYGZgt{}}
    \PYG{n+nt}{\PYGZlt{}xsd:complexType} \PYG{n+na}{name=}\PYG{l+s}{\PYGZdq{}tipoTarjeta\PYGZdq{}}\PYG{n+nt}{\PYGZgt{}}
        \PYG{n+nt}{\PYGZlt{}xsd:complexContent}\PYG{n+nt}{\PYGZgt{}}
            \PYG{n+nt}{\PYGZlt{}xsd:restriction} \PYG{n+na}{base=}\PYG{l+s}{\PYGZdq{}xsd:anyType\PYGZdq{}}\PYG{n+nt}{\PYGZgt{}}
                \PYG{n+nt}{\PYGZlt{}xsd:sequence}\PYG{n+nt}{\PYGZgt{}}
                    \PYG{n+nt}{\PYGZlt{}xsd:element} \PYG{n+na}{name=}\PYG{l+s}{\PYGZdq{}memoria\PYGZdq{}} \PYG{n+na}{type=}\PYG{l+s}{\PYGZdq{}tipoMemoria\PYGZdq{}}\PYG{n+nt}{/\PYGZgt{}}
                    \PYG{n+nt}{\PYGZlt{}xsd:element} \PYG{n+na}{name=}\PYG{l+s}{\PYGZdq{}precio\PYGZdq{}} \PYG{n+na}{type=}\PYG{l+s}{\PYGZdq{}tipoPrecio\PYGZdq{}}\PYG{n+nt}{/\PYGZgt{}}
                \PYG{n+nt}{\PYGZlt{}/xsd:sequence\PYGZgt{}}
            \PYG{n+nt}{\PYGZlt{}/xsd:restriction\PYGZgt{}}
        \PYG{n+nt}{\PYGZlt{}/xsd:complexContent\PYGZgt{}}
    \PYG{n+nt}{\PYGZlt{}/xsd:complexType\PYGZgt{}}
    \PYG{n+nt}{\PYGZlt{}xsd:simpleType} \PYG{n+na}{name=}\PYG{l+s}{\PYGZdq{}tipoMemoria\PYGZdq{}}\PYG{n+nt}{\PYGZgt{}}
        \PYG{n+nt}{\PYGZlt{}xsd:restriction} \PYG{n+na}{base=}\PYG{l+s}{\PYGZdq{}xsd:string\PYGZdq{}}\PYG{n+nt}{\PYGZgt{}}
            \PYG{n+nt}{\PYGZlt{}xsd:pattern} \PYG{n+na}{value=}\PYG{l+s}{\PYGZdq{}[0\PYGZhy{}9]+[GT]B\PYGZdq{}}\PYG{n+nt}{/\PYGZgt{}}
        \PYG{n+nt}{\PYGZlt{}/xsd:restriction\PYGZgt{}}
    \PYG{n+nt}{\PYGZlt{}/xsd:simpleType\PYGZgt{}}
    \PYG{c}{\PYGZlt{}!\PYGZhy{}\PYGZhy{}}\PYG{c}{Aqui definimos un precio con restriccion del cual}
\PYG{c}{    heredaremos despues para añadir el atributo a}
\PYG{c}{    la cantidad}\PYG{c}{\PYGZhy{}\PYGZhy{}\PYGZgt{}}
    \PYG{n+nt}{\PYGZlt{}xsd:simpleType} \PYG{n+na}{name=}\PYG{l+s}{\PYGZdq{}tipoPrecioRestringido\PYGZdq{}}\PYG{n+nt}{\PYGZgt{}}
        \PYG{n+nt}{\PYGZlt{}xsd:restriction} \PYG{n+na}{base=}\PYG{l+s}{\PYGZdq{}xsd:decimal\PYGZdq{}}\PYG{n+nt}{\PYGZgt{}}
            \PYG{n+nt}{\PYGZlt{}xsd:minInclusive} \PYG{n+na}{value=}\PYG{l+s}{\PYGZdq{}0\PYGZdq{}}\PYG{n+nt}{/\PYGZgt{}}
        \PYG{n+nt}{\PYGZlt{}/xsd:restriction\PYGZgt{}}
    \PYG{n+nt}{\PYGZlt{}/xsd:simpleType\PYGZgt{}}
    \PYG{c}{\PYGZlt{}!\PYGZhy{}\PYGZhy{}}\PYG{c}{Aqui heredamos del tipo anterior y añadimos}
\PYG{c}{    el atributo}\PYG{c}{\PYGZhy{}\PYGZhy{}\PYGZgt{}}
    \PYG{n+nt}{\PYGZlt{}xsd:complexType} \PYG{n+na}{name=}\PYG{l+s}{\PYGZdq{}tipoPrecio\PYGZdq{}}\PYG{n+nt}{\PYGZgt{}}
        \PYG{n+nt}{\PYGZlt{}xsd:simpleContent}\PYG{n+nt}{\PYGZgt{}}
            \PYG{n+nt}{\PYGZlt{}xsd:extension} \PYG{n+na}{base=}\PYG{l+s}{\PYGZdq{}tipoPrecioRestringido\PYGZdq{}}\PYG{n+nt}{\PYGZgt{}}
                \PYG{n+nt}{\PYGZlt{}xsd:attribute} \PYG{n+na}{name=}\PYG{l+s}{\PYGZdq{}moneda\PYGZdq{}} \PYG{n+na}{type=}\PYG{l+s}{\PYGZdq{}tipoMoneda\PYGZdq{}}\PYG{n+nt}{/\PYGZgt{}}
            \PYG{n+nt}{\PYGZlt{}/xsd:extension\PYGZgt{}}
        \PYG{n+nt}{\PYGZlt{}/xsd:simpleContent\PYGZgt{}}
    \PYG{n+nt}{\PYGZlt{}/xsd:complexType\PYGZgt{}}


    \PYG{n+nt}{\PYGZlt{}xsd:simpleType} \PYG{n+na}{name=}\PYG{l+s}{\PYGZdq{}tipoMoneda\PYGZdq{}}\PYG{n+nt}{\PYGZgt{}}
        \PYG{n+nt}{\PYGZlt{}xsd:restriction} \PYG{n+na}{base=}\PYG{l+s}{\PYGZdq{}xsd:string\PYGZdq{}}\PYG{n+nt}{\PYGZgt{}}
            \PYG{n+nt}{\PYGZlt{}xsd:enumeration} \PYG{n+na}{value=}\PYG{l+s}{\PYGZdq{}euros\PYGZdq{}}\PYG{n+nt}{/\PYGZgt{}}
            \PYG{n+nt}{\PYGZlt{}xsd:enumeration} \PYG{n+na}{value=}\PYG{l+s}{\PYGZdq{}dolares\PYGZdq{}}\PYG{n+nt}{/\PYGZgt{}}
        \PYG{n+nt}{\PYGZlt{}/xsd:restriction\PYGZgt{}}
    \PYG{n+nt}{\PYGZlt{}/xsd:simpleType\PYGZgt{}}
    \PYG{n+nt}{\PYGZlt{}xsd:complexType} \PYG{n+na}{name=}\PYG{l+s}{\PYGZdq{}tipoMonitor\PYGZdq{}}\PYG{n+nt}{\PYGZgt{}}
        \PYG{n+nt}{\PYGZlt{}xsd:complexContent}\PYG{n+nt}{\PYGZgt{}}
            \PYG{n+nt}{\PYGZlt{}xsd:restriction} \PYG{n+na}{base=}\PYG{l+s}{\PYGZdq{}xsd:anyType\PYGZdq{}}\PYG{n+nt}{\PYGZgt{}}
                \PYG{n+nt}{\PYGZlt{}xsd:sequence}\PYG{n+nt}{\PYGZgt{}}
                    \PYG{n+nt}{\PYGZlt{}xsd:element} \PYG{n+na}{name=}\PYG{l+s}{\PYGZdq{}tamanio\PYGZdq{}} \PYG{n+na}{type=}\PYG{l+s}{\PYGZdq{}xsd:integer\PYGZdq{}}\PYG{n+nt}{/\PYGZgt{}}
                    \PYG{n+nt}{\PYGZlt{}xsd:element} \PYG{n+na}{name=}\PYG{l+s}{\PYGZdq{}precio\PYGZdq{}}  \PYG{n+na}{type=}\PYG{l+s}{\PYGZdq{}tipoPrecio\PYGZdq{}}\PYG{n+nt}{/\PYGZgt{}}
                \PYG{n+nt}{\PYGZlt{}/xsd:sequence\PYGZgt{}}
            \PYG{n+nt}{\PYGZlt{}/xsd:restriction\PYGZgt{}}
        \PYG{n+nt}{\PYGZlt{}/xsd:complexContent\PYGZgt{}}
    \PYG{n+nt}{\PYGZlt{}/xsd:complexType\PYGZgt{}}
\PYG{n+nt}{\PYGZlt{}/xsd:schema\PYGZgt{}}
\end{sphinxVerbatim}


\subsection{Ejercicio: inventario}
\label{\detokenize{tema5:ejercicio-inventario}}
Varios administradores necesitan intercambiar información sobre inventario de material de oficina. Para ello, han llegado a un acuerdo sobre lo que se permite en un fichero XML de inventario. La idea básica es permitir ficheros como este:

\begin{sphinxVerbatim}[commandchars=\\\{\}]
\PYG{n+nt}{\PYGZlt{}inventario}\PYG{n+nt}{\PYGZgt{}}
    \PYG{n+nt}{\PYGZlt{}objeto}\PYG{n+nt}{\PYGZgt{}}
        \PYG{n+nt}{\PYGZlt{}mesa}\PYG{n+nt}{\PYGZgt{}}
            \PYG{n+nt}{\PYGZlt{}peso}\PYG{n+nt}{\PYGZgt{}}4.55\PYG{n+nt}{\PYGZlt{}/peso\PYGZgt{}}
            \PYG{n+nt}{\PYGZlt{}superficie} \PYG{n+na}{unidad=}\PYG{l+s}{\PYGZdq{}cm2\PYGZdq{}}\PYG{n+nt}{\PYGZgt{}}100\PYG{n+nt}{\PYGZlt{}/superficie\PYGZgt{}}
        \PYG{n+nt}{\PYGZlt{}/mesa\PYGZgt{}}
    \PYG{n+nt}{\PYGZlt{}/objeto\PYGZgt{}}
    \PYG{n+nt}{\PYGZlt{}objeto}\PYG{n+nt}{\PYGZgt{}}
        \PYG{n+nt}{\PYGZlt{}silla}\PYG{n+nt}{\PYGZgt{}}
            \PYG{n+nt}{\PYGZlt{}peso}\PYG{n+nt}{\PYGZgt{}}3.50\PYG{n+nt}{\PYGZlt{}/peso\PYGZgt{}}
        \PYG{n+nt}{\PYGZlt{}/silla\PYGZgt{}}
    \PYG{n+nt}{\PYGZlt{}/objeto\PYGZgt{}}
\PYG{n+nt}{\PYGZlt{}/inventario\PYGZgt{}}
\end{sphinxVerbatim}

Las reglas concretas son estas:
\begin{enumerate}
\item {} 
Dentro de \sphinxcode{\textless{}objeto\textgreater{}} puede haber uno de estos dos elementos hijo: un elemento \sphinxcode{\textless{}mesa\textgreater{}} o un elemento \sphinxcode{\textless{}silla\textgreater{}}.

\item {} 
Toda mesa tiene un elemento hijo \sphinxcode{\textless{}peso\textgreater{}}. El peso siempre es un decimal positivo con dos cifras decimales.

\item {} 
Toda mesa tiene una \sphinxcode{\textless{}superficie\textgreater{}}. La superficie es un \sphinxcode{unsignedInt}. La superficie siempre tiene un atributo que puede ser solo una de estas dos cadenas: \sphinxcode{m2} o \sphinxcode{cm2}.

\item {} 
Toda silla tiene siempre un \sphinxcode{\textless{}peso\textgreater{}} y las reglas de ese peso son exactamente las mismas que las reglas de \sphinxcode{\textless{}peso\textgreater{}} del elemento \sphinxcode{\textless{}mesa\textgreater{}}

\end{enumerate}

La solución podría descomponerse de la forma siguiente:

Resolvamos primero el problema de crear un tipo para el elemento \sphinxcode{\textless{}peso\textgreater{}}.

\begin{sphinxVerbatim}[commandchars=\\\{\}]
\PYG{n+nt}{\PYGZlt{}xsd:schema} \PYG{n+na}{xmlns:xsd=}\PYG{l+s}{\PYGZdq{}http://www.w3.org/2001/XMLSchema\PYGZdq{}}\PYG{n+nt}{\PYGZgt{}}
    \PYG{n+nt}{\PYGZlt{}xsd:element} \PYG{n+na}{name=}\PYG{l+s}{\PYGZdq{}peso\PYGZdq{}}
                 \PYG{n+na}{type=}\PYG{l+s}{\PYGZdq{}tipoPeso\PYGZdq{}}\PYG{n+nt}{\PYGZgt{}}\PYG{n+nt}{\PYGZlt{}/xsd:element\PYGZgt{}}
    \PYG{n+nt}{\PYGZlt{}xsd:simpleType} \PYG{n+na}{name=}\PYG{l+s}{\PYGZdq{}tipoPeso\PYGZdq{}}\PYG{n+nt}{\PYGZgt{}}
        \PYG{n+nt}{\PYGZlt{}xsd:restriction} \PYG{n+na}{base=}\PYG{l+s}{\PYGZdq{}xsd:decimal\PYGZdq{}}\PYG{n+nt}{\PYGZgt{}}
            \PYG{n+nt}{\PYGZlt{}xsd:minInclusive}  \PYG{n+na}{value=}\PYG{l+s}{\PYGZdq{}0\PYGZdq{}}\PYG{n+nt}{/\PYGZgt{}}
            \PYG{n+nt}{\PYGZlt{}xsd:fractionDigits} \PYG{n+na}{value=}\PYG{l+s}{\PYGZdq{}2\PYGZdq{}}\PYG{n+nt}{/\PYGZgt{}}
        \PYG{n+nt}{\PYGZlt{}/xsd:restriction\PYGZgt{}}
    \PYG{n+nt}{\PYGZlt{}/xsd:simpleType\PYGZgt{}}
\PYG{n+nt}{\PYGZlt{}/xsd:schema\PYGZgt{}}
\end{sphinxVerbatim}

Ahora resolvamos el problema del elemento \sphinxcode{\textless{}silla\textgreater{}}. Para resolverlo, podemos aprovechar el tipo \sphinxcode{tipoPeso} que acabamos de crear:

\begin{sphinxVerbatim}[commandchars=\\\{\}]
\PYG{n+nt}{\PYGZlt{}xsd:schema} \PYG{n+na}{xmlns:xsd=}\PYG{l+s}{\PYGZdq{}http://www.w3.org/2001/XMLSchema\PYGZdq{}}\PYG{n+nt}{\PYGZgt{}}
    \PYG{n+nt}{\PYGZlt{}xsd:element} \PYG{n+na}{name=}\PYG{l+s}{\PYGZdq{}silla\PYGZdq{}}
                 \PYG{n+na}{type=}\PYG{l+s}{\PYGZdq{}tipoSilla\PYGZdq{}}\PYG{n+nt}{\PYGZgt{}}\PYG{n+nt}{\PYGZlt{}/xsd:element\PYGZgt{}}
    \PYG{n+nt}{\PYGZlt{}xsd:simpleType} \PYG{n+na}{name=}\PYG{l+s}{\PYGZdq{}tipoPeso\PYGZdq{}}\PYG{n+nt}{\PYGZgt{}}
        \PYG{n+nt}{\PYGZlt{}xsd:restriction} \PYG{n+na}{base=}\PYG{l+s}{\PYGZdq{}xsd:decimal\PYGZdq{}}\PYG{n+nt}{\PYGZgt{}}
            \PYG{n+nt}{\PYGZlt{}xsd:minInclusive}  \PYG{n+na}{value=}\PYG{l+s}{\PYGZdq{}0\PYGZdq{}}\PYG{n+nt}{/\PYGZgt{}}
            \PYG{n+nt}{\PYGZlt{}xsd:fractionDigits} \PYG{n+na}{value=}\PYG{l+s}{\PYGZdq{}2\PYGZdq{}}\PYG{n+nt}{/\PYGZgt{}}
        \PYG{n+nt}{\PYGZlt{}/xsd:restriction\PYGZgt{}}
    \PYG{n+nt}{\PYGZlt{}/xsd:simpleType\PYGZgt{}}
    \PYG{n+nt}{\PYGZlt{}xsd:complexType} \PYG{n+na}{name=}\PYG{l+s}{\PYGZdq{}tipoSilla\PYGZdq{}}\PYG{n+nt}{\PYGZgt{}}
        \PYG{n+nt}{\PYGZlt{}xsd:complexContent}\PYG{n+nt}{\PYGZgt{}}
            \PYG{n+nt}{\PYGZlt{}xsd:restriction} \PYG{n+na}{base=}\PYG{l+s}{\PYGZdq{}xsd:anyType\PYGZdq{}}\PYG{n+nt}{\PYGZgt{}}
                \PYG{n+nt}{\PYGZlt{}xsd:sequence}\PYG{n+nt}{\PYGZgt{}}
                    \PYG{n+nt}{\PYGZlt{}xsd:element} \PYG{n+na}{name=}\PYG{l+s}{\PYGZdq{}peso\PYGZdq{}}
                             \PYG{n+na}{type=}\PYG{l+s}{\PYGZdq{}tipoPeso\PYGZdq{}}\PYG{n+nt}{/\PYGZgt{}}
                \PYG{n+nt}{\PYGZlt{}/xsd:sequence\PYGZgt{}}
            \PYG{n+nt}{\PYGZlt{}/xsd:restriction\PYGZgt{}}
        \PYG{n+nt}{\PYGZlt{}/xsd:complexContent\PYGZgt{}}
    \PYG{n+nt}{\PYGZlt{}/xsd:complexType\PYGZgt{}}
\PYG{n+nt}{\PYGZlt{}/xsd:schema\PYGZgt{}}
\end{sphinxVerbatim}

Ahora resolveremos el problema de la superficie:

\begin{sphinxVerbatim}[commandchars=\\\{\}]
\PYG{n+nt}{\PYGZlt{}xsd:schema} \PYG{n+na}{xmlns:xsd=}\PYG{l+s}{\PYGZdq{}http://www.w3.org/2001/XMLSchema\PYGZdq{}}\PYG{n+nt}{\PYGZgt{}}
    \PYG{n+nt}{\PYGZlt{}xsd:element} \PYG{n+na}{name=}\PYG{l+s}{\PYGZdq{}superficie\PYGZdq{}}
                 \PYG{n+na}{type=}\PYG{l+s}{\PYGZdq{}tipoSuperficie\PYGZdq{}}\PYG{n+nt}{\PYGZgt{}}\PYG{n+nt}{\PYGZlt{}/xsd:element\PYGZgt{}}
    \PYG{n+nt}{\PYGZlt{}xsd:simpleType} \PYG{n+na}{name=}\PYG{l+s}{\PYGZdq{}tipoPeso\PYGZdq{}}\PYG{n+nt}{\PYGZgt{}}
        \PYG{n+nt}{\PYGZlt{}xsd:restriction} \PYG{n+na}{base=}\PYG{l+s}{\PYGZdq{}xsd:decimal\PYGZdq{}}\PYG{n+nt}{\PYGZgt{}}
            \PYG{n+nt}{\PYGZlt{}xsd:minInclusive}  \PYG{n+na}{value=}\PYG{l+s}{\PYGZdq{}0\PYGZdq{}}\PYG{n+nt}{/\PYGZgt{}}
            \PYG{n+nt}{\PYGZlt{}xsd:fractionDigits} \PYG{n+na}{value=}\PYG{l+s}{\PYGZdq{}2\PYGZdq{}}\PYG{n+nt}{/\PYGZgt{}}
        \PYG{n+nt}{\PYGZlt{}/xsd:restriction\PYGZgt{}}
    \PYG{n+nt}{\PYGZlt{}/xsd:simpleType\PYGZgt{}}
    \PYG{n+nt}{\PYGZlt{}xsd:complexType} \PYG{n+na}{name=}\PYG{l+s}{\PYGZdq{}tipoSilla\PYGZdq{}}\PYG{n+nt}{\PYGZgt{}}
        \PYG{n+nt}{\PYGZlt{}xsd:complexContent}\PYG{n+nt}{\PYGZgt{}}
            \PYG{n+nt}{\PYGZlt{}xsd:restriction} \PYG{n+na}{base=}\PYG{l+s}{\PYGZdq{}xsd:anyType\PYGZdq{}}\PYG{n+nt}{\PYGZgt{}}
                \PYG{n+nt}{\PYGZlt{}xsd:sequence}\PYG{n+nt}{\PYGZgt{}}
                    \PYG{n+nt}{\PYGZlt{}xsd:element} \PYG{n+na}{name=}\PYG{l+s}{\PYGZdq{}peso\PYGZdq{}}
                             \PYG{n+na}{type=}\PYG{l+s}{\PYGZdq{}tipoPeso\PYGZdq{}}\PYG{n+nt}{/\PYGZgt{}}
                \PYG{n+nt}{\PYGZlt{}/xsd:sequence\PYGZgt{}}
            \PYG{n+nt}{\PYGZlt{}/xsd:restriction\PYGZgt{}}
        \PYG{n+nt}{\PYGZlt{}/xsd:complexContent\PYGZgt{}}
    \PYG{n+nt}{\PYGZlt{}/xsd:complexType\PYGZgt{}}
    \PYG{n+nt}{\PYGZlt{}xsd:complexType} \PYG{n+na}{name=}\PYG{l+s}{\PYGZdq{}tipoSuperficie\PYGZdq{}}\PYG{n+nt}{\PYGZgt{}}
        \PYG{n+nt}{\PYGZlt{}xsd:simpleContent}\PYG{n+nt}{\PYGZgt{}}
            \PYG{n+nt}{\PYGZlt{}xsd:extension} \PYG{n+na}{base=}\PYG{l+s}{\PYGZdq{}xsd:unsignedInt\PYGZdq{}}\PYG{n+nt}{\PYGZgt{}}
                \PYG{n+nt}{\PYGZlt{}xsd:attribute} \PYG{n+na}{name=}\PYG{l+s}{\PYGZdq{}unidad\PYGZdq{}}
                           \PYG{n+na}{type=}\PYG{l+s}{\PYGZdq{}tipoUnidad\PYGZdq{}}
                           \PYG{n+na}{use=}\PYG{l+s}{\PYGZdq{}required\PYGZdq{}}\PYG{n+nt}{/\PYGZgt{}}
            \PYG{n+nt}{\PYGZlt{}/xsd:extension\PYGZgt{}}
        \PYG{n+nt}{\PYGZlt{}/xsd:simpleContent\PYGZgt{}}
    \PYG{n+nt}{\PYGZlt{}/xsd:complexType\PYGZgt{}}
    \PYG{n+nt}{\PYGZlt{}xsd:simpleType} \PYG{n+na}{name=}\PYG{l+s}{\PYGZdq{}tipoUnidad\PYGZdq{}}\PYG{n+nt}{\PYGZgt{}}
        \PYG{n+nt}{\PYGZlt{}xsd:restriction} \PYG{n+na}{base=}\PYG{l+s}{\PYGZdq{}xsd:string\PYGZdq{}}\PYG{n+nt}{\PYGZgt{}}
            \PYG{n+nt}{\PYGZlt{}xsd:enumeration} \PYG{n+na}{value=}\PYG{l+s}{\PYGZdq{}m2\PYGZdq{}}\PYG{n+nt}{/\PYGZgt{}}
            \PYG{n+nt}{\PYGZlt{}xsd:enumeration} \PYG{n+na}{value=}\PYG{l+s}{\PYGZdq{}cm2\PYGZdq{}}\PYG{n+nt}{/\PYGZgt{}}
        \PYG{n+nt}{\PYGZlt{}/xsd:restriction\PYGZgt{}}
    \PYG{n+nt}{\PYGZlt{}/xsd:simpleType\PYGZgt{}}
\PYG{n+nt}{\PYGZlt{}/xsd:schema\PYGZgt{}}
\end{sphinxVerbatim}

Y apoyándonos en eso haremos la mesa:

\begin{sphinxVerbatim}[commandchars=\\\{\}]
\PYG{n+nt}{\PYGZlt{}xsd:schema} \PYG{n+na}{xmlns:xsd=}\PYG{l+s}{\PYGZdq{}http://www.w3.org/2001/XMLSchema\PYGZdq{}}\PYG{n+nt}{\PYGZgt{}}
    \PYG{n+nt}{\PYGZlt{}xsd:element} \PYG{n+na}{name=}\PYG{l+s}{\PYGZdq{}mesa\PYGZdq{}}
                 \PYG{n+na}{type=}\PYG{l+s}{\PYGZdq{}tipoMesa\PYGZdq{}}\PYG{n+nt}{\PYGZgt{}}\PYG{n+nt}{\PYGZlt{}/xsd:element\PYGZgt{}}
    \PYG{n+nt}{\PYGZlt{}xsd:simpleType} \PYG{n+na}{name=}\PYG{l+s}{\PYGZdq{}tipoPeso\PYGZdq{}}\PYG{n+nt}{\PYGZgt{}}
        \PYG{n+nt}{\PYGZlt{}xsd:restriction} \PYG{n+na}{base=}\PYG{l+s}{\PYGZdq{}xsd:decimal\PYGZdq{}}\PYG{n+nt}{\PYGZgt{}}
            \PYG{n+nt}{\PYGZlt{}xsd:minInclusive}  \PYG{n+na}{value=}\PYG{l+s}{\PYGZdq{}0\PYGZdq{}}\PYG{n+nt}{/\PYGZgt{}}
            \PYG{n+nt}{\PYGZlt{}xsd:fractionDigits} \PYG{n+na}{value=}\PYG{l+s}{\PYGZdq{}2\PYGZdq{}}\PYG{n+nt}{/\PYGZgt{}}
        \PYG{n+nt}{\PYGZlt{}/xsd:restriction\PYGZgt{}}
    \PYG{n+nt}{\PYGZlt{}/xsd:simpleType\PYGZgt{}}
    \PYG{n+nt}{\PYGZlt{}xsd:complexType} \PYG{n+na}{name=}\PYG{l+s}{\PYGZdq{}tipoSilla\PYGZdq{}}\PYG{n+nt}{\PYGZgt{}}
        \PYG{n+nt}{\PYGZlt{}xsd:complexContent}\PYG{n+nt}{\PYGZgt{}}
            \PYG{n+nt}{\PYGZlt{}xsd:restriction} \PYG{n+na}{base=}\PYG{l+s}{\PYGZdq{}xsd:anyType\PYGZdq{}}\PYG{n+nt}{\PYGZgt{}}
                \PYG{n+nt}{\PYGZlt{}xsd:sequence}\PYG{n+nt}{\PYGZgt{}}
                    \PYG{n+nt}{\PYGZlt{}xsd:element} \PYG{n+na}{name=}\PYG{l+s}{\PYGZdq{}peso\PYGZdq{}}
                             \PYG{n+na}{type=}\PYG{l+s}{\PYGZdq{}tipoPeso\PYGZdq{}}\PYG{n+nt}{/\PYGZgt{}}
                \PYG{n+nt}{\PYGZlt{}/xsd:sequence\PYGZgt{}}
            \PYG{n+nt}{\PYGZlt{}/xsd:restriction\PYGZgt{}}
        \PYG{n+nt}{\PYGZlt{}/xsd:complexContent\PYGZgt{}}
    \PYG{n+nt}{\PYGZlt{}/xsd:complexType\PYGZgt{}}
    \PYG{n+nt}{\PYGZlt{}xsd:complexType} \PYG{n+na}{name=}\PYG{l+s}{\PYGZdq{}tipoSuperficie\PYGZdq{}}\PYG{n+nt}{\PYGZgt{}}
        \PYG{n+nt}{\PYGZlt{}xsd:simpleContent}\PYG{n+nt}{\PYGZgt{}}
            \PYG{n+nt}{\PYGZlt{}xsd:extension} \PYG{n+na}{base=}\PYG{l+s}{\PYGZdq{}xsd:unsignedInt\PYGZdq{}}\PYG{n+nt}{\PYGZgt{}}
                \PYG{n+nt}{\PYGZlt{}xsd:attribute} \PYG{n+na}{name=}\PYG{l+s}{\PYGZdq{}unidad\PYGZdq{}}
                           \PYG{n+na}{type=}\PYG{l+s}{\PYGZdq{}tipoUnidad\PYGZdq{}}
                           \PYG{n+na}{use=}\PYG{l+s}{\PYGZdq{}required\PYGZdq{}}\PYG{n+nt}{/\PYGZgt{}}
            \PYG{n+nt}{\PYGZlt{}/xsd:extension\PYGZgt{}}
        \PYG{n+nt}{\PYGZlt{}/xsd:simpleContent\PYGZgt{}}
    \PYG{n+nt}{\PYGZlt{}/xsd:complexType\PYGZgt{}}
    \PYG{n+nt}{\PYGZlt{}xsd:simpleType} \PYG{n+na}{name=}\PYG{l+s}{\PYGZdq{}tipoUnidad\PYGZdq{}}\PYG{n+nt}{\PYGZgt{}}
        \PYG{n+nt}{\PYGZlt{}xsd:restriction} \PYG{n+na}{base=}\PYG{l+s}{\PYGZdq{}xsd:string\PYGZdq{}}\PYG{n+nt}{\PYGZgt{}}
            \PYG{n+nt}{\PYGZlt{}xsd:enumeration} \PYG{n+na}{value=}\PYG{l+s}{\PYGZdq{}m2\PYGZdq{}}\PYG{n+nt}{/\PYGZgt{}}
            \PYG{n+nt}{\PYGZlt{}xsd:enumeration} \PYG{n+na}{value=}\PYG{l+s}{\PYGZdq{}cm2\PYGZdq{}}\PYG{n+nt}{/\PYGZgt{}}
        \PYG{n+nt}{\PYGZlt{}/xsd:restriction\PYGZgt{}}
    \PYG{n+nt}{\PYGZlt{}/xsd:simpleType\PYGZgt{}}
    \PYG{n+nt}{\PYGZlt{}xsd:complexType} \PYG{n+na}{name=}\PYG{l+s}{\PYGZdq{}tipoMesa\PYGZdq{}}\PYG{n+nt}{\PYGZgt{}}
        \PYG{n+nt}{\PYGZlt{}xsd:complexContent}\PYG{n+nt}{\PYGZgt{}}
            \PYG{n+nt}{\PYGZlt{}xsd:restriction} \PYG{n+na}{base=}\PYG{l+s}{\PYGZdq{}xsd:anyType\PYGZdq{}}\PYG{n+nt}{\PYGZgt{}}
                \PYG{n+nt}{\PYGZlt{}xsd:sequence}\PYG{n+nt}{\PYGZgt{}}
                    \PYG{n+nt}{\PYGZlt{}xsd:element} \PYG{n+na}{name=}\PYG{l+s}{\PYGZdq{}peso\PYGZdq{}}
                                 \PYG{n+na}{type=}\PYG{l+s}{\PYGZdq{}tipoPeso\PYGZdq{}}\PYG{n+nt}{/\PYGZgt{}}
                    \PYG{n+nt}{\PYGZlt{}xsd:element} \PYG{n+na}{name=}\PYG{l+s}{\PYGZdq{}superficie\PYGZdq{}}
                                 \PYG{n+na}{type=}\PYG{l+s}{\PYGZdq{}tipoSuperficie\PYGZdq{}}\PYG{n+nt}{/\PYGZgt{}}
                \PYG{n+nt}{\PYGZlt{}/xsd:sequence\PYGZgt{}}
            \PYG{n+nt}{\PYGZlt{}/xsd:restriction\PYGZgt{}}
        \PYG{n+nt}{\PYGZlt{}/xsd:complexContent\PYGZgt{}}
    \PYG{n+nt}{\PYGZlt{}/xsd:complexType\PYGZgt{}}
\PYG{n+nt}{\PYGZlt{}/xsd:schema\PYGZgt{}}
\end{sphinxVerbatim}

Y ya solo queda indicar que un inventario es una lista de objetos (pondremos el \sphinxcode{maxOccurs} a \sphinxcode{unbounded}) e indicaremos que un objeto puede ser una elección (\sphinxcode{\textless{}xsd:choice\textgreater{}}) entre dos tipos de objetos.

\begin{sphinxVerbatim}[commandchars=\\\{\}]
\PYG{n+nt}{\PYGZlt{}xsd:schema} \PYG{n+na}{xmlns:xsd=}\PYG{l+s}{\PYGZdq{}http://www.w3.org/2001/XMLSchema\PYGZdq{}}\PYG{n+nt}{\PYGZgt{}}
\PYG{n+nt}{\PYGZlt{}xsd:element} \PYG{n+na}{name=}\PYG{l+s}{\PYGZdq{}inventario\PYGZdq{}}
             \PYG{n+na}{type=}\PYG{l+s}{\PYGZdq{}tipoInventario\PYGZdq{}}\PYG{n+nt}{/\PYGZgt{}}
\PYG{n+nt}{\PYGZlt{}xsd:simpleType} \PYG{n+na}{name=}\PYG{l+s}{\PYGZdq{}tipoPeso\PYGZdq{}}\PYG{n+nt}{\PYGZgt{}}
    \PYG{n+nt}{\PYGZlt{}xsd:restriction} \PYG{n+na}{base=}\PYG{l+s}{\PYGZdq{}xsd:decimal\PYGZdq{}}\PYG{n+nt}{\PYGZgt{}}
        \PYG{n+nt}{\PYGZlt{}xsd:minInclusive}  \PYG{n+na}{value=}\PYG{l+s}{\PYGZdq{}0\PYGZdq{}}\PYG{n+nt}{/\PYGZgt{}}
        \PYG{n+nt}{\PYGZlt{}xsd:fractionDigits} \PYG{n+na}{value=}\PYG{l+s}{\PYGZdq{}2\PYGZdq{}}\PYG{n+nt}{/\PYGZgt{}}
    \PYG{n+nt}{\PYGZlt{}/xsd:restriction\PYGZgt{}}
\PYG{n+nt}{\PYGZlt{}/xsd:simpleType\PYGZgt{}}
\PYG{n+nt}{\PYGZlt{}xsd:complexType} \PYG{n+na}{name=}\PYG{l+s}{\PYGZdq{}tipoSilla\PYGZdq{}}\PYG{n+nt}{\PYGZgt{}}
    \PYG{n+nt}{\PYGZlt{}xsd:complexContent}\PYG{n+nt}{\PYGZgt{}}
        \PYG{n+nt}{\PYGZlt{}xsd:restriction} \PYG{n+na}{base=}\PYG{l+s}{\PYGZdq{}xsd:anyType\PYGZdq{}}\PYG{n+nt}{\PYGZgt{}}
            \PYG{n+nt}{\PYGZlt{}xsd:sequence}\PYG{n+nt}{\PYGZgt{}}
                \PYG{n+nt}{\PYGZlt{}xsd:element} \PYG{n+na}{name=}\PYG{l+s}{\PYGZdq{}peso\PYGZdq{}}
                         \PYG{n+na}{type=}\PYG{l+s}{\PYGZdq{}tipoPeso\PYGZdq{}}\PYG{n+nt}{/\PYGZgt{}}
            \PYG{n+nt}{\PYGZlt{}/xsd:sequence\PYGZgt{}}
        \PYG{n+nt}{\PYGZlt{}/xsd:restriction\PYGZgt{}}
    \PYG{n+nt}{\PYGZlt{}/xsd:complexContent\PYGZgt{}}
\PYG{n+nt}{\PYGZlt{}/xsd:complexType\PYGZgt{}}
\PYG{n+nt}{\PYGZlt{}xsd:complexType} \PYG{n+na}{name=}\PYG{l+s}{\PYGZdq{}tipoSuperficie\PYGZdq{}}\PYG{n+nt}{\PYGZgt{}}
    \PYG{n+nt}{\PYGZlt{}xsd:simpleContent}\PYG{n+nt}{\PYGZgt{}}
        \PYG{n+nt}{\PYGZlt{}xsd:extension} \PYG{n+na}{base=}\PYG{l+s}{\PYGZdq{}xsd:unsignedInt\PYGZdq{}}\PYG{n+nt}{\PYGZgt{}}
            \PYG{n+nt}{\PYGZlt{}xsd:attribute} \PYG{n+na}{name=}\PYG{l+s}{\PYGZdq{}unidad\PYGZdq{}}
                       \PYG{n+na}{type=}\PYG{l+s}{\PYGZdq{}tipoUnidad\PYGZdq{}}
                       \PYG{n+na}{use=}\PYG{l+s}{\PYGZdq{}required\PYGZdq{}}\PYG{n+nt}{/\PYGZgt{}}
        \PYG{n+nt}{\PYGZlt{}/xsd:extension\PYGZgt{}}
    \PYG{n+nt}{\PYGZlt{}/xsd:simpleContent\PYGZgt{}}
\PYG{n+nt}{\PYGZlt{}/xsd:complexType\PYGZgt{}}
\PYG{n+nt}{\PYGZlt{}xsd:simpleType} \PYG{n+na}{name=}\PYG{l+s}{\PYGZdq{}tipoUnidad\PYGZdq{}}\PYG{n+nt}{\PYGZgt{}}
    \PYG{n+nt}{\PYGZlt{}xsd:restriction} \PYG{n+na}{base=}\PYG{l+s}{\PYGZdq{}xsd:string\PYGZdq{}}\PYG{n+nt}{\PYGZgt{}}
        \PYG{n+nt}{\PYGZlt{}xsd:enumeration} \PYG{n+na}{value=}\PYG{l+s}{\PYGZdq{}m2\PYGZdq{}}\PYG{n+nt}{/\PYGZgt{}}
        \PYG{n+nt}{\PYGZlt{}xsd:enumeration} \PYG{n+na}{value=}\PYG{l+s}{\PYGZdq{}cm2\PYGZdq{}}\PYG{n+nt}{/\PYGZgt{}}
    \PYG{n+nt}{\PYGZlt{}/xsd:restriction\PYGZgt{}}
\PYG{n+nt}{\PYGZlt{}/xsd:simpleType\PYGZgt{}}
\PYG{n+nt}{\PYGZlt{}xsd:complexType} \PYG{n+na}{name=}\PYG{l+s}{\PYGZdq{}tipoMesa\PYGZdq{}}\PYG{n+nt}{\PYGZgt{}}
    \PYG{n+nt}{\PYGZlt{}xsd:complexContent}\PYG{n+nt}{\PYGZgt{}}
        \PYG{n+nt}{\PYGZlt{}xsd:restriction} \PYG{n+na}{base=}\PYG{l+s}{\PYGZdq{}xsd:anyType\PYGZdq{}}\PYG{n+nt}{\PYGZgt{}}
            \PYG{n+nt}{\PYGZlt{}xsd:sequence}\PYG{n+nt}{\PYGZgt{}}
                \PYG{n+nt}{\PYGZlt{}xsd:element} \PYG{n+na}{name=}\PYG{l+s}{\PYGZdq{}peso\PYGZdq{}}
                             \PYG{n+na}{type=}\PYG{l+s}{\PYGZdq{}tipoPeso\PYGZdq{}}\PYG{n+nt}{/\PYGZgt{}}
                \PYG{n+nt}{\PYGZlt{}xsd:element} \PYG{n+na}{name=}\PYG{l+s}{\PYGZdq{}superficie\PYGZdq{}}
                             \PYG{n+na}{type=}\PYG{l+s}{\PYGZdq{}tipoSuperficie\PYGZdq{}}\PYG{n+nt}{/\PYGZgt{}}
            \PYG{n+nt}{\PYGZlt{}/xsd:sequence\PYGZgt{}}
        \PYG{n+nt}{\PYGZlt{}/xsd:restriction\PYGZgt{}}
    \PYG{n+nt}{\PYGZlt{}/xsd:complexContent\PYGZgt{}}
\PYG{n+nt}{\PYGZlt{}/xsd:complexType\PYGZgt{}}
\PYG{n+nt}{\PYGZlt{}xsd:complexType} \PYG{n+na}{name=}\PYG{l+s}{\PYGZdq{}tipoInventario\PYGZdq{}}\PYG{n+nt}{\PYGZgt{}}
    \PYG{n+nt}{\PYGZlt{}xsd:complexContent}\PYG{n+nt}{\PYGZgt{}}
        \PYG{n+nt}{\PYGZlt{}xsd:restriction} \PYG{n+na}{base=}\PYG{l+s}{\PYGZdq{}xsd:anyType\PYGZdq{}}\PYG{n+nt}{\PYGZgt{}}
            \PYG{n+nt}{\PYGZlt{}xsd:sequence}\PYG{n+nt}{\PYGZgt{}}
                \PYG{n+nt}{\PYGZlt{}xsd:element} \PYG{n+na}{name=}\PYG{l+s}{\PYGZdq{}objeto\PYGZdq{}}
                             \PYG{n+na}{type=}\PYG{l+s}{\PYGZdq{}tipoObjeto\PYGZdq{}}
                             \PYG{n+na}{maxOccurs=}\PYG{l+s}{\PYGZdq{}unbounded\PYGZdq{}}\PYG{n+nt}{/\PYGZgt{}}
            \PYG{n+nt}{\PYGZlt{}/xsd:sequence\PYGZgt{}}
        \PYG{n+nt}{\PYGZlt{}/xsd:restriction\PYGZgt{}}
    \PYG{n+nt}{\PYGZlt{}/xsd:complexContent\PYGZgt{}}
\PYG{n+nt}{\PYGZlt{}/xsd:complexType\PYGZgt{}}
\PYG{n+nt}{\PYGZlt{}xsd:complexType} \PYG{n+na}{name=}\PYG{l+s}{\PYGZdq{}tipoObjeto\PYGZdq{}}\PYG{n+nt}{\PYGZgt{}}
    \PYG{n+nt}{\PYGZlt{}xsd:complexContent}\PYG{n+nt}{\PYGZgt{}}
        \PYG{n+nt}{\PYGZlt{}xsd:restriction} \PYG{n+na}{base=}\PYG{l+s}{\PYGZdq{}xsd:anyType\PYGZdq{}}\PYG{n+nt}{\PYGZgt{}}
            \PYG{n+nt}{\PYGZlt{}xsd:choice}\PYG{n+nt}{\PYGZgt{}}
                \PYG{n+nt}{\PYGZlt{}xsd:element} \PYG{n+na}{name=}\PYG{l+s}{\PYGZdq{}mesa\PYGZdq{}} \PYG{n+na}{type=}\PYG{l+s}{\PYGZdq{}tipoMesa\PYGZdq{}}\PYG{n+nt}{/\PYGZgt{}}
                \PYG{n+nt}{\PYGZlt{}xsd:element} \PYG{n+na}{name=}\PYG{l+s}{\PYGZdq{}silla\PYGZdq{}} \PYG{n+na}{type=}\PYG{l+s}{\PYGZdq{}tipoSilla\PYGZdq{}}\PYG{n+nt}{/\PYGZgt{}}
            \PYG{n+nt}{\PYGZlt{}/xsd:choice\PYGZgt{}}
        \PYG{n+nt}{\PYGZlt{}/xsd:restriction\PYGZgt{}}
    \PYG{n+nt}{\PYGZlt{}/xsd:complexContent\PYGZgt{}}
\PYG{n+nt}{\PYGZlt{}/xsd:complexType\PYGZgt{}}
\PYG{n+nt}{\PYGZlt{}/xsd:schema\PYGZgt{}}
\end{sphinxVerbatim}


\section{Ejercicio tipo examen}
\label{\detokenize{tema5:ejercicio-tipo-examen}}
Se necesita crear un esquema que controle la correcta sintaxis de ficheros con este estilo:

\begin{sphinxVerbatim}[commandchars=\\\{\}]
\PYG{n+nt}{\PYGZlt{}productosfinancieros}\PYG{n+nt}{\PYGZgt{}}
    \PYG{n+nt}{\PYGZlt{}producto}\PYG{n+nt}{\PYGZgt{}}
        \PYG{n+nt}{\PYGZlt{}bono}\PYG{n+nt}{\PYGZgt{}}
            \PYG{n+nt}{\PYGZlt{}valoractual} \PYG{n+na}{moneda=}\PYG{l+s}{\PYGZdq{}yenes\PYGZdq{}}\PYG{n+nt}{\PYGZgt{}}2.212\PYG{n+nt}{\PYGZlt{}/valoractual\PYGZgt{}}
            \PYG{n+nt}{\PYGZlt{}beneficio}\PYG{n+nt}{\PYGZgt{}}\PYGZhy{}2.83\PYG{n+nt}{\PYGZlt{}/beneficio\PYGZgt{}}
        \PYG{n+nt}{\PYGZlt{}/bono\PYGZgt{}}
    \PYG{n+nt}{\PYGZlt{}/producto\PYGZgt{}}
    \PYG{n+nt}{\PYGZlt{}producto}\PYG{n+nt}{\PYGZgt{}}
        \PYG{n+nt}{\PYGZlt{}futuro}\PYG{n+nt}{\PYGZgt{}}
            \PYG{n+nt}{\PYGZlt{}elemento} \PYG{n+na}{idioma=}\PYG{l+s}{\PYGZdq{}espanol\PYGZdq{}}\PYG{n+nt}{\PYGZgt{}}Petroleo\PYG{n+nt}{\PYGZlt{}/elemento\PYGZgt{}}
            \PYG{n+nt}{\PYGZlt{}beneficio}\PYG{n+nt}{\PYGZgt{}}\PYGZhy{}3.83\PYG{n+nt}{\PYGZlt{}/beneficio\PYGZgt{}}
        \PYG{n+nt}{\PYGZlt{}/futuro\PYGZgt{}}
    \PYG{n+nt}{\PYGZlt{}/producto\PYGZgt{}}
    \PYG{n+nt}{\PYGZlt{}producto}\PYG{n+nt}{\PYGZgt{}}
        \PYG{n+nt}{\PYGZlt{}acciones}\PYG{n+nt}{\PYGZgt{}}
            \PYG{n+nt}{\PYGZlt{}empresa} \PYG{n+na}{pais=}\PYG{l+s}{\PYGZdq{}usa\PYGZdq{}}\PYG{n+nt}{\PYGZgt{}}ENRON\PYG{n+nt}{\PYGZlt{}/empresa\PYGZgt{}}
            \PYG{n+nt}{\PYGZlt{}beneficio}\PYG{n+nt}{\PYGZgt{}}2.91\PYG{n+nt}{\PYGZlt{}/beneficio\PYGZgt{}}
        \PYG{n+nt}{\PYGZlt{}/acciones\PYGZgt{}}
    \PYG{n+nt}{\PYGZlt{}/producto\PYGZgt{}}
\PYG{n+nt}{\PYGZlt{}/productosfinancieros\PYGZgt{}}
\end{sphinxVerbatim}

Las reglas concretas son las siguientes:
\begin{enumerate}
\item {} 
El elemento raíz es \sphinxcode{\textless{}productosfinancieros\textgreater{}}. Dentro de él debe haber uno o más elementos \sphinxcode{\textless{}producto\textgreater{}}.

\item {} 
Un \sphinxcode{\textless{}producto\textgreater{}} puede ser de tres tipos: \sphinxcode{\textless{}bono\textgreater{}}, \sphinxcode{\textless{}futuro\textgreater{}} y \sphinxcode{\textless{}acciones\textgreater{}}.

\item {} 
Todos los productos tienen siempre un elemento hijo llamado \sphinxcode{\textless{}beneficio\textgreater{}} que puede ser un número con dos decimales (puede ser positivo o negativo).

\item {} 
Todo \sphinxcode{\textless{}bono\textgreater{}} puede tener dentro un elemento llamado \sphinxcode{\textless{}valoractual\textgreater{}} que contiene un valor decimal que puede ser positivo o negativo y tener o no decimales. El elemento \sphinxcode{\textless{}valoractual\textgreater{}} deberá llevar dentro un atributo llamado \sphinxcode{moneda} que solo puede tomar los valores \sphinxcode{dolares}, \sphinxcode{euros} o \sphinxcode{yenes}.

\item {} 
Todo \sphinxcode{\textless{}futuro\textgreater{}} tiene un hijo llamado \sphinxcode{\textless{}elemento\textgreater{}} que puede contener dentro cadenas de cualquier tipo. Para saber en qué idioma está la cadena se usa un atributo llamado \sphinxcode{idioma} que indica el idioma en el que está escrita la cadena.

\item {} 
Las acciones siempre tienen un elemento \sphinxcode{\textless{}empresa\textgreater{}} que indica el nombre de la empresa y un atributo llamado \sphinxcode{país} que indica de donde es la empresa. De momento queremos limitarnos a los países \sphinxcode{usa}, \sphinxcode{alemania}, \sphinxcode{japon} y \sphinxcode{espana}.

\end{enumerate}

Recuérdese que siempre que no nos digan nada, se supone que un elemento o atributo es \sphinxstylestrong{obligatorio}. Si algo es optativo nos dirán «puede tener dentro», «puede contener», «puede aparecer», etc…

Una posible solución sería esta:

\begin{sphinxVerbatim}[commandchars=\\\{\}]
\PYG{n+nt}{\PYGZlt{}xsd:schema} \PYG{n+na}{xmlns:xsd=}\PYG{l+s}{\PYGZdq{}http://www.w3.org/2001/XMLSchema\PYGZdq{}}\PYG{n+nt}{\PYGZgt{}}
    \PYG{n+nt}{\PYGZlt{}xsd:element} \PYG{n+na}{name=}\PYG{l+s}{\PYGZdq{}productosfinancieros\PYGZdq{}}
                 \PYG{n+na}{type=}\PYG{l+s}{\PYGZdq{}tipoProductosFinancieros\PYGZdq{}}\PYG{n+nt}{/\PYGZgt{}}
    \PYG{n+nt}{\PYGZlt{}xsd:complexType} \PYG{n+na}{name=}\PYG{l+s}{\PYGZdq{}tipoProductosFinancieros\PYGZdq{}}\PYG{n+nt}{\PYGZgt{}}
        \PYG{n+nt}{\PYGZlt{}xsd:complexContent}\PYG{n+nt}{\PYGZgt{}}
            \PYG{n+nt}{\PYGZlt{}xsd:restriction} \PYG{n+na}{base=}\PYG{l+s}{\PYGZdq{}xsd:anyType\PYGZdq{}}\PYG{n+nt}{\PYGZgt{}}
                \PYG{n+nt}{\PYGZlt{}xsd:sequence}\PYG{n+nt}{\PYGZgt{}}
                    \PYG{n+nt}{\PYGZlt{}xsd:element} \PYG{n+na}{name=}\PYG{l+s}{\PYGZdq{}producto\PYGZdq{}}
                                 \PYG{n+na}{type=}\PYG{l+s}{\PYGZdq{}tipoProducto\PYGZdq{}}
                                 \PYG{n+na}{maxOccurs=}\PYG{l+s}{\PYGZdq{}unbounded\PYGZdq{}}\PYG{n+nt}{/\PYGZgt{}}
                \PYG{n+nt}{\PYGZlt{}/xsd:sequence\PYGZgt{}}
            \PYG{n+nt}{\PYGZlt{}/xsd:restriction\PYGZgt{}}
        \PYG{n+nt}{\PYGZlt{}/xsd:complexContent\PYGZgt{}}
    \PYG{n+nt}{\PYGZlt{}/xsd:complexType\PYGZgt{}}
    \PYG{n+nt}{\PYGZlt{}xsd:complexType} \PYG{n+na}{name=}\PYG{l+s}{\PYGZdq{}tipoProducto\PYGZdq{}}\PYG{n+nt}{\PYGZgt{}}
        \PYG{n+nt}{\PYGZlt{}xsd:complexContent}\PYG{n+nt}{\PYGZgt{}}
            \PYG{n+nt}{\PYGZlt{}xsd:restriction} \PYG{n+na}{base=}\PYG{l+s}{\PYGZdq{}xsd:anyType\PYGZdq{}}\PYG{n+nt}{\PYGZgt{}}
                \PYG{n+nt}{\PYGZlt{}xsd:choice}\PYG{n+nt}{\PYGZgt{}}
                    \PYG{n+nt}{\PYGZlt{}xsd:element} \PYG{n+na}{name=}\PYG{l+s}{\PYGZdq{}bono\PYGZdq{}}
                                 \PYG{n+na}{type=}\PYG{l+s}{\PYGZdq{}tipoBono\PYGZdq{}}\PYG{n+nt}{/\PYGZgt{}}
                    \PYG{n+nt}{\PYGZlt{}xsd:element} \PYG{n+na}{name=}\PYG{l+s}{\PYGZdq{}futuro\PYGZdq{}}
                                 \PYG{n+na}{type=}\PYG{l+s}{\PYGZdq{}tipoFuturo\PYGZdq{}}\PYG{n+nt}{/\PYGZgt{}}
                    \PYG{n+nt}{\PYGZlt{}xsd:element} \PYG{n+na}{name=}\PYG{l+s}{\PYGZdq{}acciones\PYGZdq{}}
                                 \PYG{n+na}{type=}\PYG{l+s}{\PYGZdq{}tipoAcciones\PYGZdq{}}\PYG{n+nt}{/\PYGZgt{}}
                \PYG{n+nt}{\PYGZlt{}/xsd:choice\PYGZgt{}}
            \PYG{n+nt}{\PYGZlt{}/xsd:restriction\PYGZgt{}}
        \PYG{n+nt}{\PYGZlt{}/xsd:complexContent\PYGZgt{}}
    \PYG{n+nt}{\PYGZlt{}/xsd:complexType\PYGZgt{}}
    \PYG{n+nt}{\PYGZlt{}xsd:complexType} \PYG{n+na}{name=}\PYG{l+s}{\PYGZdq{}tipoBono\PYGZdq{}}\PYG{n+nt}{\PYGZgt{}}
        \PYG{n+nt}{\PYGZlt{}xsd:complexContent}\PYG{n+nt}{\PYGZgt{}}
            \PYG{n+nt}{\PYGZlt{}xsd:restriction} \PYG{n+na}{base=}\PYG{l+s}{\PYGZdq{}xsd:anyType\PYGZdq{}}\PYG{n+nt}{\PYGZgt{}}
                \PYG{n+nt}{\PYGZlt{}xsd:sequence}\PYG{n+nt}{\PYGZgt{}}
                    \PYG{n+nt}{\PYGZlt{}xsd:element} \PYG{n+na}{name=}\PYG{l+s}{\PYGZdq{}valoractual\PYGZdq{}}
                                 \PYG{n+na}{type=}\PYG{l+s}{\PYGZdq{}tipoValorActual\PYGZdq{}}\PYG{n+nt}{/\PYGZgt{}}
                    \PYG{n+nt}{\PYGZlt{}xsd:element} \PYG{n+na}{name=}\PYG{l+s}{\PYGZdq{}beneficio\PYGZdq{}}
                                 \PYG{n+na}{type=}\PYG{l+s}{\PYGZdq{}tipoBeneficio\PYGZdq{}}\PYG{n+nt}{/\PYGZgt{}}
                \PYG{n+nt}{\PYGZlt{}/xsd:sequence\PYGZgt{}}
            \PYG{n+nt}{\PYGZlt{}/xsd:restriction\PYGZgt{}}
        \PYG{n+nt}{\PYGZlt{}/xsd:complexContent\PYGZgt{}}
    \PYG{n+nt}{\PYGZlt{}/xsd:complexType\PYGZgt{}}
    \PYG{n+nt}{\PYGZlt{}xsd:complexType} \PYG{n+na}{name=}\PYG{l+s}{\PYGZdq{}tipoValorActual\PYGZdq{}}\PYG{n+nt}{\PYGZgt{}}
        \PYG{n+nt}{\PYGZlt{}xsd:simpleContent}\PYG{n+nt}{\PYGZgt{}}
            \PYG{n+nt}{\PYGZlt{}xsd:extension} \PYG{n+na}{base=}\PYG{l+s}{\PYGZdq{}xsd:decimal\PYGZdq{}}\PYG{n+nt}{\PYGZgt{}}
                \PYG{n+nt}{\PYGZlt{}xsd:attribute} \PYG{n+na}{name=}\PYG{l+s}{\PYGZdq{}moneda\PYGZdq{}} \PYG{n+na}{type=}\PYG{l+s}{\PYGZdq{}tipoMoneda\PYGZdq{}}\PYG{n+nt}{/\PYGZgt{}}
            \PYG{n+nt}{\PYGZlt{}/xsd:extension\PYGZgt{}}
        \PYG{n+nt}{\PYGZlt{}/xsd:simpleContent\PYGZgt{}}
    \PYG{n+nt}{\PYGZlt{}/xsd:complexType\PYGZgt{}}
    \PYG{n+nt}{\PYGZlt{}xsd:simpleType} \PYG{n+na}{name=}\PYG{l+s}{\PYGZdq{}tipoMoneda\PYGZdq{}}\PYG{n+nt}{\PYGZgt{}}
        \PYG{n+nt}{\PYGZlt{}xsd:restriction} \PYG{n+na}{base=}\PYG{l+s}{\PYGZdq{}xsd:string\PYGZdq{}}\PYG{n+nt}{\PYGZgt{}}
            \PYG{n+nt}{\PYGZlt{}xsd:enumeration} \PYG{n+na}{value=}\PYG{l+s}{\PYGZdq{}dolares\PYGZdq{}}\PYG{n+nt}{/\PYGZgt{}}
            \PYG{n+nt}{\PYGZlt{}xsd:enumeration} \PYG{n+na}{value=}\PYG{l+s}{\PYGZdq{}euros\PYGZdq{}}\PYG{n+nt}{/\PYGZgt{}}
            \PYG{n+nt}{\PYGZlt{}xsd:enumeration} \PYG{n+na}{value=}\PYG{l+s}{\PYGZdq{}yenes\PYGZdq{}}\PYG{n+nt}{/\PYGZgt{}}
        \PYG{n+nt}{\PYGZlt{}/xsd:restriction\PYGZgt{}}
    \PYG{n+nt}{\PYGZlt{}/xsd:simpleType\PYGZgt{}}
    \PYG{n+nt}{\PYGZlt{}xsd:simpleType} \PYG{n+na}{name=}\PYG{l+s}{\PYGZdq{}tipoBeneficio\PYGZdq{}}\PYG{n+nt}{\PYGZgt{}}
        \PYG{n+nt}{\PYGZlt{}xsd:restriction} \PYG{n+na}{base=}\PYG{l+s}{\PYGZdq{}xsd:decimal\PYGZdq{}}\PYG{n+nt}{\PYGZgt{}}
            \PYG{n+nt}{\PYGZlt{}xsd:fractionDigits} \PYG{n+na}{value=}\PYG{l+s}{\PYGZdq{}2\PYGZdq{}}\PYG{n+nt}{/\PYGZgt{}}
        \PYG{n+nt}{\PYGZlt{}/xsd:restriction\PYGZgt{}}
    \PYG{n+nt}{\PYGZlt{}/xsd:simpleType\PYGZgt{}}
    \PYG{n+nt}{\PYGZlt{}xsd:complexType} \PYG{n+na}{name=}\PYG{l+s}{\PYGZdq{}tipoFuturo\PYGZdq{}}\PYG{n+nt}{\PYGZgt{}}
        \PYG{n+nt}{\PYGZlt{}xsd:complexContent}\PYG{n+nt}{\PYGZgt{}}
            \PYG{n+nt}{\PYGZlt{}xsd:restriction} \PYG{n+na}{base=}\PYG{l+s}{\PYGZdq{}xsd:anyType\PYGZdq{}}\PYG{n+nt}{\PYGZgt{}}
                \PYG{n+nt}{\PYGZlt{}xsd:sequence}\PYG{n+nt}{\PYGZgt{}}
                    \PYG{n+nt}{\PYGZlt{}xsd:element} \PYG{n+na}{name=}\PYG{l+s}{\PYGZdq{}elemento\PYGZdq{}}
                                 \PYG{n+na}{type=}\PYG{l+s}{\PYGZdq{}tipoElemento\PYGZdq{}}\PYG{n+nt}{/\PYGZgt{}}
                    \PYG{n+nt}{\PYGZlt{}xsd:element} \PYG{n+na}{name=}\PYG{l+s}{\PYGZdq{}beneficio\PYGZdq{}}
                                 \PYG{n+na}{type=}\PYG{l+s}{\PYGZdq{}tipoBeneficio\PYGZdq{}}\PYG{n+nt}{/\PYGZgt{}}
                \PYG{n+nt}{\PYGZlt{}/xsd:sequence\PYGZgt{}}
            \PYG{n+nt}{\PYGZlt{}/xsd:restriction\PYGZgt{}}
        \PYG{n+nt}{\PYGZlt{}/xsd:complexContent\PYGZgt{}}
    \PYG{n+nt}{\PYGZlt{}/xsd:complexType\PYGZgt{}}
    \PYG{c}{\PYGZlt{}!\PYGZhy{}\PYGZhy{}}\PYG{c}{No nos dicen nada sobre los posibles idiomas,}
\PYG{c}{    así que podemos asumir que el idioma será una cadena cualquiera}\PYG{c}{\PYGZhy{}\PYGZhy{}\PYGZgt{}}
    \PYG{n+nt}{\PYGZlt{}xsd:complexType} \PYG{n+na}{name=}\PYG{l+s}{\PYGZdq{}tipoElemento\PYGZdq{}}\PYG{n+nt}{\PYGZgt{}}
        \PYG{n+nt}{\PYGZlt{}xsd:simpleContent}\PYG{n+nt}{\PYGZgt{}}
            \PYG{n+nt}{\PYGZlt{}xsd:extension} \PYG{n+na}{base=}\PYG{l+s}{\PYGZdq{}xsd:string\PYGZdq{}}\PYG{n+nt}{\PYGZgt{}}
                \PYG{n+nt}{\PYGZlt{}xsd:attribute} \PYG{n+na}{name=}\PYG{l+s}{\PYGZdq{}idioma\PYGZdq{}}
                               \PYG{n+na}{type=}\PYG{l+s}{\PYGZdq{}xsd:string\PYGZdq{}}\PYG{n+nt}{/\PYGZgt{}}
            \PYG{n+nt}{\PYGZlt{}/xsd:extension\PYGZgt{}}
        \PYG{n+nt}{\PYGZlt{}/xsd:simpleContent\PYGZgt{}}
    \PYG{n+nt}{\PYGZlt{}/xsd:complexType\PYGZgt{}}
    \PYG{n+nt}{\PYGZlt{}xsd:complexType} \PYG{n+na}{name=}\PYG{l+s}{\PYGZdq{}tipoAcciones\PYGZdq{}}\PYG{n+nt}{\PYGZgt{}}
        \PYG{n+nt}{\PYGZlt{}xsd:complexContent}\PYG{n+nt}{\PYGZgt{}}
            \PYG{n+nt}{\PYGZlt{}xsd:restriction} \PYG{n+na}{base=}\PYG{l+s}{\PYGZdq{}xsd:anyType\PYGZdq{}}\PYG{n+nt}{\PYGZgt{}}
                \PYG{n+nt}{\PYGZlt{}xsd:sequence}\PYG{n+nt}{\PYGZgt{}}
                    \PYG{n+nt}{\PYGZlt{}xsd:element} \PYG{n+na}{name=}\PYG{l+s}{\PYGZdq{}empresa\PYGZdq{}}
                                 \PYG{n+na}{type=}\PYG{l+s}{\PYGZdq{}tipoEmpresa\PYGZdq{}}\PYG{n+nt}{/\PYGZgt{}}
                    \PYG{n+nt}{\PYGZlt{}xsd:element} \PYG{n+na}{name=}\PYG{l+s}{\PYGZdq{}beneficio\PYGZdq{}}
                                 \PYG{n+na}{type=}\PYG{l+s}{\PYGZdq{}tipoBeneficio\PYGZdq{}}\PYG{n+nt}{/\PYGZgt{}}
                \PYG{n+nt}{\PYGZlt{}/xsd:sequence\PYGZgt{}}
            \PYG{n+nt}{\PYGZlt{}/xsd:restriction\PYGZgt{}}
        \PYG{n+nt}{\PYGZlt{}/xsd:complexContent\PYGZgt{}}
    \PYG{n+nt}{\PYGZlt{}/xsd:complexType\PYGZgt{}}
    \PYG{n+nt}{\PYGZlt{}xsd:complexType} \PYG{n+na}{name=}\PYG{l+s}{\PYGZdq{}tipoEmpresa\PYGZdq{}}\PYG{n+nt}{\PYGZgt{}}
        \PYG{n+nt}{\PYGZlt{}xsd:simpleContent}\PYG{n+nt}{\PYGZgt{}}
            \PYG{n+nt}{\PYGZlt{}xsd:extension} \PYG{n+na}{base=}\PYG{l+s}{\PYGZdq{}xsd:string\PYGZdq{}}\PYG{n+nt}{\PYGZgt{}}
                \PYG{n+nt}{\PYGZlt{}xsd:attribute} \PYG{n+na}{name=}\PYG{l+s}{\PYGZdq{}pais\PYGZdq{}}
                               \PYG{n+na}{type=}\PYG{l+s}{\PYGZdq{}tipoPais\PYGZdq{}}\PYG{n+nt}{/\PYGZgt{}}
            \PYG{n+nt}{\PYGZlt{}/xsd:extension\PYGZgt{}}
        \PYG{n+nt}{\PYGZlt{}/xsd:simpleContent\PYGZgt{}}
    \PYG{n+nt}{\PYGZlt{}/xsd:complexType\PYGZgt{}}
    \PYG{n+nt}{\PYGZlt{}xsd:simpleType} \PYG{n+na}{name=}\PYG{l+s}{\PYGZdq{}tipoPais\PYGZdq{}}\PYG{n+nt}{\PYGZgt{}}
        \PYG{n+nt}{\PYGZlt{}xsd:restriction} \PYG{n+na}{base=}\PYG{l+s}{\PYGZdq{}xsd:string\PYGZdq{}}\PYG{n+nt}{\PYGZgt{}}
            \PYG{n+nt}{\PYGZlt{}xsd:enumeration} \PYG{n+na}{value=}\PYG{l+s}{\PYGZdq{}alemania\PYGZdq{}}\PYG{n+nt}{/\PYGZgt{}}
            \PYG{n+nt}{\PYGZlt{}xsd:enumeration} \PYG{n+na}{value=}\PYG{l+s}{\PYGZdq{}japon\PYGZdq{}}\PYG{n+nt}{/\PYGZgt{}}
            \PYG{n+nt}{\PYGZlt{}xsd:enumeration} \PYG{n+na}{value=}\PYG{l+s}{\PYGZdq{}espania\PYGZdq{}}\PYG{n+nt}{/\PYGZgt{}}
            \PYG{n+nt}{\PYGZlt{}xsd:enumeration} \PYG{n+na}{value=}\PYG{l+s}{\PYGZdq{}usa\PYGZdq{}}\PYG{n+nt}{/\PYGZgt{}}
        \PYG{n+nt}{\PYGZlt{}/xsd:restriction\PYGZgt{}}
    \PYG{n+nt}{\PYGZlt{}/xsd:simpleType\PYGZgt{}}
\PYG{n+nt}{\PYGZlt{}/xsd:schema\PYGZgt{}}
\end{sphinxVerbatim}


\section{Ejercicio tipo examen (II)}
\label{\detokenize{tema5:ejercicio-tipo-examen-ii}}
Crear una DTD que permita validar un fichero como el siguiente:

\begin{sphinxVerbatim}[commandchars=\\\{\}]
\PYG{n+nt}{\PYGZlt{}inventario}\PYG{n+nt}{\PYGZgt{}}
    \PYG{n+nt}{\PYGZlt{}objeto} \PYG{n+na}{codigo=}\PYG{l+s}{\PYGZdq{}MM2809\PYGZdq{}}\PYG{n+nt}{\PYGZgt{}}
        \PYG{n+nt}{\PYGZlt{}mesa}\PYG{n+nt}{\PYGZgt{}}
            \PYG{n+nt}{\PYGZlt{}tipo}\PYG{n+nt}{\PYGZgt{}}Oficina\PYG{n+nt}{\PYGZlt{}/tipo\PYGZgt{}}
            \PYG{n+nt}{\PYGZlt{}localizacion}\PYG{n+nt}{\PYGZgt{}}B09\PYG{n+nt}{\PYGZlt{}/localizacion\PYGZgt{}}
        \PYG{n+nt}{\PYGZlt{}/mesa\PYGZgt{}}
    \PYG{n+nt}{\PYGZlt{}/objeto\PYGZgt{}}
    \PYG{n+nt}{\PYGZlt{}objeto}\PYG{n+nt}{\PYGZgt{}}
        \PYG{n+nt}{\PYGZlt{}ordenador}\PYG{n+nt}{\PYGZgt{}}
            \PYG{n+nt}{\PYGZlt{}procesador} \PYG{n+na}{fabricante=}\PYG{l+s}{\PYGZdq{}Intel\PYGZdq{}}\PYG{n+nt}{\PYGZgt{}}
                i3
            \PYG{n+nt}{\PYGZlt{}/procesador\PYGZgt{}}
            \PYG{n+nt}{\PYGZlt{}memoria} \PYG{n+na}{unidad=}\PYG{l+s}{\PYGZdq{}GB\PYGZdq{}}\PYG{n+nt}{\PYGZgt{}}2\PYG{n+nt}{\PYGZlt{}/memoria\PYGZgt{}}
            \PYG{n+nt}{\PYGZlt{}discoduro}\PYG{n+nt}{\PYGZgt{}}520\PYG{n+nt}{\PYGZlt{}/discoduro\PYGZgt{}}
        \PYG{n+nt}{\PYGZlt{}/ordenador\PYGZgt{}}
    \PYG{n+nt}{\PYGZlt{}/objeto\PYGZgt{}}
\PYG{n+nt}{\PYGZlt{}/inventario\PYGZgt{}}
\end{sphinxVerbatim}

Las reglas son las siguientes:
\begin{itemize}
\item {} 
El elemento raíz es \sphinxcode{\textless{}inventario\textgreater{}}.

\item {} 
Dentro de \sphinxcode{\textless{}inventario\textgreater{}} debe haber una \sphinxcode{\textless{}mesa\textgreater{}} o un \sphinxcode{\textless{}ordenador\textgreater{}}.

\item {} 
Dentro de mesa puede haber (o no) un primer elemento \sphinxcode{\textless{}tipo\textgreater{}}. Despues debe haber un elemento \sphinxcode{\textless{}localizacion\textgreater{}}.

\item {} 
Dentro de ordenador puede haber 3 elementos optativos pero que de aparecer lo hacen en el siguiente orden.

\item {} 
Primero un elemento \sphinxcode{\textless{}procesador\textgreater{}} que puede llevar un atributo \sphinxcode{fabricante}.

\item {} 
Despues un elemento \sphinxcode{\textless{}memoria\textgreater{}} que debe llevar obligatoriamente un atributo \sphinxcode{unidad}.

\item {} 
Despues un elemento \sphinxcode{\textless{}discoduro\textgreater{}}.

\end{itemize}


\section{Ejercicio tipo examen (III)}
\label{\detokenize{tema5:ejercicio-tipo-examen-iii}}
Un distribuidor de alimentación necesita un fichero XML que almacene la información sobre pedidos recibidos y entregados que esté regido por un esquema XML que contemple las restricciones siguientes:
\begin{itemize}
\item {} 
El elemento raíz se llama \sphinxcode{portes}.

\item {} 
Dentro de \sphinxcode{portes}, puede haber uno o más de los elementos \sphinxcode{recepcion} y \sphinxcode{entrega}. Su orden puede ser aleatorio y el número de repeticiones también.

\item {} 
Un \sphinxcode{recepcion} lleva dentro tres elementos: un elemento \sphinxcode{producto}, un elemento \sphinxcode{cantidad} y un elemento \sphinxcode{codigoreceptor}.

\item {} 
El \sphinxcode{producto} es obligatorio y lleva dentro texto.

\item {} 
El elemento \sphinxcode{cantidad} (obligatorio) lleva dentro un número con decimales pero que debe ser siempre positivo.

\item {} 
El \sphinxcode{codigoreceptor} lleva dentro un texto con la estructura: 3 cifras, guión, 3 letras (mayúsculas o minúsculas). Este elemento \sphinxcode{codigoreceptor} es optativo.

\item {} 
Una \sphinxcode{entrega} tiene siempre un atributo \sphinxcode{receptor} que lleva dentro un texto. Aparte de eso, una \sphinxcode{entrega} tiene siempre un elemento \sphinxcode{transportista} que solo puede valer \sphinxcode{T1}, \sphinxcode{T2} o \sphinxcode{T3}. Despues de el elemento \sphinxcode{transportista}  hay siempre un elemento \sphinxcode{distancia} . La \sphinxcode{distancia} es un numero mayor de 0. Es necesario que la \sphinxcode{distancia}  tenga un atributo \sphinxcode{unidad} que indica la unidad en forma de cadena. Además una \sphinxcode{entrega} lleva un atributo \sphinxcode{coste} que siempre es un entero mayor de 0.

\end{itemize}

A continuación se muestra un fichero de ejemplo

\begin{sphinxadmonition}{warning}{Advertencia:}
Por ahora NO SE DARÁ LA SOLUCIÓN DE ESTE EJERCICIO
\end{sphinxadmonition}

\begin{sphinxVerbatim}[commandchars=\\\{\}]
\PYG{n+nt}{\PYGZlt{}portes}\PYG{n+nt}{\PYGZgt{}}
    \PYG{n+nt}{\PYGZlt{}recepcion}\PYG{n+nt}{\PYGZgt{}}
        \PYG{n+nt}{\PYGZlt{}producto}\PYG{n+nt}{\PYGZgt{}}Fruta\PYG{n+nt}{\PYGZlt{}/producto\PYGZgt{}}
        \PYG{n+nt}{\PYGZlt{}cantidad}\PYG{n+nt}{\PYGZgt{}}125.5\PYG{n+nt}{\PYGZlt{}/cantidad\PYGZgt{}}
        \PYG{n+nt}{\PYGZlt{}codigoreceptor}\PYG{n+nt}{\PYGZgt{}}333\PYGZhy{}AZT\PYG{n+nt}{\PYGZlt{}/codigoreceptor\PYGZgt{}}
    \PYG{n+nt}{\PYGZlt{}/recepcion\PYGZgt{}}
    \PYG{n+nt}{\PYGZlt{}entrega} \PYG{n+na}{receptor=}\PYG{l+s}{\PYGZdq{}Mercados SL\PYGZdq{}} \PYG{n+na}{coste=}\PYG{l+s}{\PYGZdq{}1321\PYGZdq{}}\PYG{n+nt}{\PYGZgt{}}
        \PYG{n+nt}{\PYGZlt{}transportista}\PYG{n+nt}{\PYGZgt{}}T2\PYG{n+nt}{\PYGZlt{}/transportista\PYGZgt{}}
        \PYG{n+nt}{\PYGZlt{}distancia} \PYG{n+na}{unidad=}\PYG{l+s}{\PYGZdq{}millas\PYGZdq{}}\PYG{n+nt}{\PYGZgt{}}468\PYG{n+nt}{\PYGZlt{}/distancia\PYGZgt{}}
    \PYG{n+nt}{\PYGZlt{}/entrega\PYGZgt{}}
    \PYG{n+nt}{\PYGZlt{}recepcion}\PYG{n+nt}{\PYGZgt{}}
        \PYG{n+nt}{\PYGZlt{}producto}\PYG{n+nt}{\PYGZgt{}}Verdura\PYG{n+nt}{\PYGZlt{}/producto\PYGZgt{}}
        \PYG{n+nt}{\PYGZlt{}cantidad}\PYG{n+nt}{\PYGZgt{}}250\PYG{n+nt}{\PYGZlt{}/cantidad\PYGZgt{}}
        \PYG{c}{\PYGZlt{}!\PYGZhy{}\PYGZhy{}}\PYG{c}{El codigo de receptor no se usó aquí}\PYG{c}{\PYGZhy{}\PYGZhy{}\PYGZgt{}}
    \PYG{n+nt}{\PYGZlt{}/recepcion\PYGZgt{}}
\PYG{n+nt}{\PYGZlt{}/portes\PYGZgt{}}
\end{sphinxVerbatim}


\section{Examen}
\label{\detokenize{tema5:examen}}
El examen de este tema tendrá lugar
\begin{itemize}
\item {} 
Para DAM el viernes, 16 de marzo de 2018

\item {} 
Para ASIR el martes, 13 de marzo de 2018

\end{itemize}


\chapter{Recuperación de información}
\label{\detokenize{tema6::doc}}\label{\detokenize{tema6:recuperacion-de-informacion}}

\section{Introducción}
\label{\detokenize{tema6:introduccion}}
En este tema veremos que podemos recuperar información de archivos XML \sphinxstyleemphasis{igual} que si fuesen bases de datos. Para ello podemos usar dos cosas:
\begin{itemize}
\item {} 
Un lenguaje de programación de propósito general, como es Java.

\item {} 
O un lenguaje de programación especializado como es XQuery.

\end{itemize}

En líneas generales hay dos grandes formas de usar un lenguaje de programación como Java para leer o escribir archivos XML.
\begin{itemize}
\item {} 
DOM: significa Document Object Model (o Modelo del objeto documento). DOM en general almacena los archivos en memoria lo que es mucho más rápido y eficiente.

\item {} 
SAX: en algunos casos, los archivos muy grandes, pueden no caber en memoria. SAX proporciona otras clases y métodos distintos para ir procesando un archivo por partes. SAX significa Simple Access for XML, pero en general es un poco más complicado. En este módulo, no lo veremos.

\end{itemize}

DOM es un estándar y sus clases y métodos existen en muchos otros lenguajes.


\section{XQuery}
\label{\detokenize{tema6:xquery}}
Para empezar \sphinxstylestrong{XQuery es un superconjunto de XPath} por lo que las expresiones básicas de XQuery también servirán en XQuery. La primera diferencia es que en XQuery tendremos que usar la función \sphinxcode{doc} para cargar un archivo y luego navegar por él. Supongamos que tenemos un archivo de ejemplo como este:

\begin{sphinxVerbatim}[commandchars=\\\{\}]
\PYG{n+nt}{\PYGZlt{}datos}\PYG{n+nt}{\PYGZgt{}}
    \PYG{n+nt}{\PYGZlt{}proveedores}\PYG{n+nt}{\PYGZgt{}}
        \PYG{n+nt}{\PYGZlt{}proveedor} \PYG{n+na}{numprov=}\PYG{l+s}{\PYGZdq{}v1\PYGZdq{}}\PYG{n+nt}{\PYGZgt{}}
            \PYG{n+nt}{\PYGZlt{}nombreprov}\PYG{n+nt}{\PYGZgt{}}Smith\PYG{n+nt}{\PYGZlt{}/nombreprov\PYGZgt{}}
            \PYG{n+nt}{\PYGZlt{}estado}\PYG{n+nt}{\PYGZgt{}}20\PYG{n+nt}{\PYGZlt{}/estado\PYGZgt{}}
            \PYG{n+nt}{\PYGZlt{}ciudad}\PYG{n+nt}{\PYGZgt{}}Londres\PYG{n+nt}{\PYGZlt{}/ciudad\PYGZgt{}}
        \PYG{n+nt}{\PYGZlt{}/proveedor\PYGZgt{}}
        \PYG{n+nt}{\PYGZlt{}proveedor} \PYG{n+na}{numprov=}\PYG{l+s}{\PYGZdq{}v2\PYGZdq{}}\PYG{n+nt}{\PYGZgt{}}
            \PYG{n+nt}{\PYGZlt{}nombreprov}\PYG{n+nt}{\PYGZgt{}}Jones\PYG{n+nt}{\PYGZlt{}/nombreprov\PYGZgt{}}
            \PYG{n+nt}{\PYGZlt{}estado}\PYG{n+nt}{\PYGZgt{}}10\PYG{n+nt}{\PYGZlt{}/estado\PYGZgt{}}
            \PYG{n+nt}{\PYGZlt{}ciudad}\PYG{n+nt}{\PYGZgt{}}Paris\PYG{n+nt}{\PYGZlt{}/ciudad\PYGZgt{}}
        \PYG{n+nt}{\PYGZlt{}/proveedor\PYGZgt{}}
        \PYG{n+nt}{\PYGZlt{}proveedor} \PYG{n+na}{numprov=}\PYG{l+s}{\PYGZdq{}v3\PYGZdq{}}\PYG{n+nt}{\PYGZgt{}}
            \PYG{n+nt}{\PYGZlt{}nombreprov}\PYG{n+nt}{\PYGZgt{}}Blake\PYG{n+nt}{\PYGZlt{}/nombreprov\PYGZgt{}}
            \PYG{n+nt}{\PYGZlt{}estado}\PYG{n+nt}{\PYGZgt{}}30\PYG{n+nt}{\PYGZlt{}/estado\PYGZgt{}}
            \PYG{n+nt}{\PYGZlt{}ciudad}\PYG{n+nt}{\PYGZgt{}}Paris\PYG{n+nt}{\PYGZlt{}/ciudad\PYGZgt{}}
        \PYG{n+nt}{\PYGZlt{}/proveedor\PYGZgt{}}
        \PYG{n+nt}{\PYGZlt{}proveedor} \PYG{n+na}{numprov=}\PYG{l+s}{\PYGZdq{}v4\PYGZdq{}}\PYG{n+nt}{\PYGZgt{}}
            \PYG{n+nt}{\PYGZlt{}nombreprov}\PYG{n+nt}{\PYGZgt{}}Clarke\PYG{n+nt}{\PYGZlt{}/nombreprov\PYGZgt{}}
            \PYG{n+nt}{\PYGZlt{}estado}\PYG{n+nt}{\PYGZgt{}}20\PYG{n+nt}{\PYGZlt{}/estado\PYGZgt{}}
            \PYG{n+nt}{\PYGZlt{}ciudad}\PYG{n+nt}{\PYGZgt{}}Londres\PYG{n+nt}{\PYGZlt{}/ciudad\PYGZgt{}}
        \PYG{n+nt}{\PYGZlt{}/proveedor\PYGZgt{}}
        \PYG{n+nt}{\PYGZlt{}proveedor} \PYG{n+na}{numprov=}\PYG{l+s}{\PYGZdq{}v5\PYGZdq{}}\PYG{n+nt}{\PYGZgt{}}
            \PYG{n+nt}{\PYGZlt{}nombreprov}\PYG{n+nt}{\PYGZgt{}}Adams\PYG{n+nt}{\PYGZlt{}/nombreprov\PYGZgt{}}
            \PYG{n+nt}{\PYGZlt{}estado}\PYG{n+nt}{\PYGZgt{}}30\PYG{n+nt}{\PYGZlt{}/estado\PYGZgt{}}
            \PYG{n+nt}{\PYGZlt{}ciudad}\PYG{n+nt}{\PYGZgt{}}Atenas\PYG{n+nt}{\PYGZlt{}/ciudad\PYGZgt{}}
        \PYG{n+nt}{\PYGZlt{}/proveedor\PYGZgt{}}
    \PYG{n+nt}{\PYGZlt{}/proveedores\PYGZgt{}}
    \PYG{n+nt}{\PYGZlt{}partes}\PYG{n+nt}{\PYGZgt{}}
        \PYG{n+nt}{\PYGZlt{}parte} \PYG{n+na}{numparte=}\PYG{l+s}{\PYGZdq{}p1\PYGZdq{}}\PYG{n+nt}{\PYGZgt{}}
            \PYG{n+nt}{\PYGZlt{}nombreparte}\PYG{n+nt}{\PYGZgt{}}Tuerca\PYG{n+nt}{\PYGZlt{}/nombreparte\PYGZgt{}}
            \PYG{n+nt}{\PYGZlt{}color}\PYG{n+nt}{\PYGZgt{}}Rojo\PYG{n+nt}{\PYGZlt{}/color\PYGZgt{}}
            \PYG{n+nt}{\PYGZlt{}peso}\PYG{n+nt}{\PYGZgt{}}12\PYG{n+nt}{\PYGZlt{}/peso\PYGZgt{}}
            \PYG{n+nt}{\PYGZlt{}ciudad}\PYG{n+nt}{\PYGZgt{}}Londres\PYG{n+nt}{\PYGZlt{}/ciudad\PYGZgt{}}
        \PYG{n+nt}{\PYGZlt{}/parte\PYGZgt{}}
        \PYG{n+nt}{\PYGZlt{}parte} \PYG{n+na}{numparte=}\PYG{l+s}{\PYGZdq{}p2\PYGZdq{}}\PYG{n+nt}{\PYGZgt{}}
            \PYG{n+nt}{\PYGZlt{}nombreparte}\PYG{n+nt}{\PYGZgt{}}Perno\PYG{n+nt}{\PYGZlt{}/nombreparte\PYGZgt{}}
            \PYG{n+nt}{\PYGZlt{}color}\PYG{n+nt}{\PYGZgt{}}Verde\PYG{n+nt}{\PYGZlt{}/color\PYGZgt{}}
            \PYG{n+nt}{\PYGZlt{}peso}\PYG{n+nt}{\PYGZgt{}}17\PYG{n+nt}{\PYGZlt{}/peso\PYGZgt{}}
            \PYG{n+nt}{\PYGZlt{}ciudad}\PYG{n+nt}{\PYGZgt{}}Paris\PYG{n+nt}{\PYGZlt{}/ciudad\PYGZgt{}}
        \PYG{n+nt}{\PYGZlt{}/parte\PYGZgt{}}
        \PYG{n+nt}{\PYGZlt{}parte} \PYG{n+na}{numparte=}\PYG{l+s}{\PYGZdq{}p3\PYGZdq{}}\PYG{n+nt}{\PYGZgt{}}
            \PYG{n+nt}{\PYGZlt{}nombreparte}\PYG{n+nt}{\PYGZgt{}}Tornillo\PYG{n+nt}{\PYGZlt{}/nombreparte\PYGZgt{}}
            \PYG{n+nt}{\PYGZlt{}color}\PYG{n+nt}{\PYGZgt{}}Azul\PYG{n+nt}{\PYGZlt{}/color\PYGZgt{}}
            \PYG{n+nt}{\PYGZlt{}peso}\PYG{n+nt}{\PYGZgt{}}17\PYG{n+nt}{\PYGZlt{}/peso\PYGZgt{}}
            \PYG{n+nt}{\PYGZlt{}ciudad}\PYG{n+nt}{\PYGZgt{}}Roma\PYG{n+nt}{\PYGZlt{}/ciudad\PYGZgt{}}
        \PYG{n+nt}{\PYGZlt{}/parte\PYGZgt{}}
        \PYG{n+nt}{\PYGZlt{}parte} \PYG{n+na}{numparte=}\PYG{l+s}{\PYGZdq{}p4\PYGZdq{}}\PYG{n+nt}{\PYGZgt{}}
            \PYG{n+nt}{\PYGZlt{}nombreparte}\PYG{n+nt}{\PYGZgt{}}Tornillo\PYG{n+nt}{\PYGZlt{}/nombreparte\PYGZgt{}}
            \PYG{n+nt}{\PYGZlt{}color}\PYG{n+nt}{\PYGZgt{}}Rojo\PYG{n+nt}{\PYGZlt{}/color\PYGZgt{}}
            \PYG{n+nt}{\PYGZlt{}peso}\PYG{n+nt}{\PYGZgt{}}14\PYG{n+nt}{\PYGZlt{}/peso\PYGZgt{}}
            \PYG{n+nt}{\PYGZlt{}ciudad}\PYG{n+nt}{\PYGZgt{}}Londres\PYG{n+nt}{\PYGZlt{}/ciudad\PYGZgt{}}
        \PYG{n+nt}{\PYGZlt{}/parte\PYGZgt{}}
        \PYG{n+nt}{\PYGZlt{}parte} \PYG{n+na}{numparte=}\PYG{l+s}{\PYGZdq{}p5\PYGZdq{}}\PYG{n+nt}{\PYGZgt{}}
            \PYG{n+nt}{\PYGZlt{}nombreparte}\PYG{n+nt}{\PYGZgt{}}Leva\PYG{n+nt}{\PYGZlt{}/nombreparte\PYGZgt{}}
            \PYG{n+nt}{\PYGZlt{}color}\PYG{n+nt}{\PYGZgt{}}Azul\PYG{n+nt}{\PYGZlt{}/color\PYGZgt{}}
            \PYG{n+nt}{\PYGZlt{}peso}\PYG{n+nt}{\PYGZgt{}}12\PYG{n+nt}{\PYGZlt{}/peso\PYGZgt{}}
            \PYG{n+nt}{\PYGZlt{}ciudad}\PYG{n+nt}{\PYGZgt{}}Paris\PYG{n+nt}{\PYGZlt{}/ciudad\PYGZgt{}}
        \PYG{n+nt}{\PYGZlt{}/parte\PYGZgt{}}
        \PYG{n+nt}{\PYGZlt{}parte} \PYG{n+na}{numparte=}\PYG{l+s}{\PYGZdq{}p6\PYGZdq{}}\PYG{n+nt}{\PYGZgt{}}
            \PYG{n+nt}{\PYGZlt{}nombreparte}\PYG{n+nt}{\PYGZgt{}}Engranaje\PYG{n+nt}{\PYGZlt{}/nombreparte\PYGZgt{}}
            \PYG{n+nt}{\PYGZlt{}color}\PYG{n+nt}{\PYGZgt{}}Rojo\PYG{n+nt}{\PYGZlt{}/color\PYGZgt{}}
            \PYG{n+nt}{\PYGZlt{}peso}\PYG{n+nt}{\PYGZgt{}}19\PYG{n+nt}{\PYGZlt{}/peso\PYGZgt{}}
            \PYG{n+nt}{\PYGZlt{}ciudad}\PYG{n+nt}{\PYGZgt{}}Londres\PYG{n+nt}{\PYGZlt{}/ciudad\PYGZgt{}}
        \PYG{n+nt}{\PYGZlt{}/parte\PYGZgt{}}
    \PYG{n+nt}{\PYGZlt{}/partes\PYGZgt{}}
    \PYG{n+nt}{\PYGZlt{}proyectos}\PYG{n+nt}{\PYGZgt{}}
        \PYG{n+nt}{\PYGZlt{}proyecto} \PYG{n+na}{numproyecto=}\PYG{l+s}{\PYGZdq{}y1\PYGZdq{}}\PYG{n+nt}{\PYGZgt{}}
            \PYG{n+nt}{\PYGZlt{}nombreproyecto}\PYG{n+nt}{\PYGZgt{}}Clasificador\PYG{n+nt}{\PYGZlt{}/nombreproyecto\PYGZgt{}}
            \PYG{n+nt}{\PYGZlt{}ciudad}\PYG{n+nt}{\PYGZgt{}}Paris\PYG{n+nt}{\PYGZlt{}/ciudad\PYGZgt{}}
        \PYG{n+nt}{\PYGZlt{}/proyecto\PYGZgt{}}
        \PYG{n+nt}{\PYGZlt{}proyecto} \PYG{n+na}{numproyecto=}\PYG{l+s}{\PYGZdq{}y2\PYGZdq{}}\PYG{n+nt}{\PYGZgt{}}
            \PYG{n+nt}{\PYGZlt{}nombreproyecto}\PYG{n+nt}{\PYGZgt{}}Monitor\PYG{n+nt}{\PYGZlt{}/nombreproyecto\PYGZgt{}}
            \PYG{n+nt}{\PYGZlt{}ciudad}\PYG{n+nt}{\PYGZgt{}}Roma\PYG{n+nt}{\PYGZlt{}/ciudad\PYGZgt{}}
        \PYG{n+nt}{\PYGZlt{}/proyecto\PYGZgt{}}
        \PYG{n+nt}{\PYGZlt{}proyecto} \PYG{n+na}{numproyecto=}\PYG{l+s}{\PYGZdq{}y3\PYGZdq{}}\PYG{n+nt}{\PYGZgt{}}
            \PYG{n+nt}{\PYGZlt{}nombreproyecto}\PYG{n+nt}{\PYGZgt{}}OCR\PYG{n+nt}{\PYGZlt{}/nombreproyecto\PYGZgt{}}
            \PYG{n+nt}{\PYGZlt{}ciudad}\PYG{n+nt}{\PYGZgt{}}Atenas\PYG{n+nt}{\PYGZlt{}/ciudad\PYGZgt{}}
        \PYG{n+nt}{\PYGZlt{}/proyecto\PYGZgt{}}
        \PYG{n+nt}{\PYGZlt{}proyecto} \PYG{n+na}{numproyecto=}\PYG{l+s}{\PYGZdq{}y4\PYGZdq{}}\PYG{n+nt}{\PYGZgt{}}
            \PYG{n+nt}{\PYGZlt{}nombreproyecto}\PYG{n+nt}{\PYGZgt{}}Consola\PYG{n+nt}{\PYGZlt{}/nombreproyecto\PYGZgt{}}
            \PYG{n+nt}{\PYGZlt{}ciudad}\PYG{n+nt}{\PYGZgt{}}Atenas\PYG{n+nt}{\PYGZlt{}/ciudad\PYGZgt{}}
        \PYG{n+nt}{\PYGZlt{}/proyecto\PYGZgt{}}
        \PYG{n+nt}{\PYGZlt{}proyecto} \PYG{n+na}{numproyecto=}\PYG{l+s}{\PYGZdq{}y5\PYGZdq{}}\PYG{n+nt}{\PYGZgt{}}
            \PYG{n+nt}{\PYGZlt{}nombreproyecto}\PYG{n+nt}{\PYGZgt{}}RAID\PYG{n+nt}{\PYGZlt{}/nombreproyecto\PYGZgt{}}
            \PYG{n+nt}{\PYGZlt{}ciudad}\PYG{n+nt}{\PYGZgt{}}Londres\PYG{n+nt}{\PYGZlt{}/ciudad\PYGZgt{}}
        \PYG{n+nt}{\PYGZlt{}/proyecto\PYGZgt{}}
        \PYG{n+nt}{\PYGZlt{}proyecto} \PYG{n+na}{numproyecto=}\PYG{l+s}{\PYGZdq{}y6\PYGZdq{}}\PYG{n+nt}{\PYGZgt{}}
            \PYG{n+nt}{\PYGZlt{}nombreproyecto}\PYG{n+nt}{\PYGZgt{}}EDS\PYG{n+nt}{\PYGZlt{}/nombreproyecto\PYGZgt{}}
            \PYG{n+nt}{\PYGZlt{}ciudad}\PYG{n+nt}{\PYGZgt{}}Oslo\PYG{n+nt}{\PYGZlt{}/ciudad\PYGZgt{}}
        \PYG{n+nt}{\PYGZlt{}/proyecto\PYGZgt{}}
        \PYG{n+nt}{\PYGZlt{}proyecto} \PYG{n+na}{numproyecto=}\PYG{l+s}{\PYGZdq{}y7\PYGZdq{}}\PYG{n+nt}{\PYGZgt{}}
            \PYG{n+nt}{\PYGZlt{}nombreproyecto}\PYG{n+nt}{\PYGZgt{}}Cinta\PYG{n+nt}{\PYGZlt{}/nombreproyecto\PYGZgt{}}
            \PYG{n+nt}{\PYGZlt{}ciudad}\PYG{n+nt}{\PYGZgt{}}Londres\PYG{n+nt}{\PYGZlt{}/ciudad\PYGZgt{}}
        \PYG{n+nt}{\PYGZlt{}/proyecto\PYGZgt{}}
    \PYG{n+nt}{\PYGZlt{}/proyectos\PYGZgt{}}
    \PYG{n+nt}{\PYGZlt{}suministros}\PYG{n+nt}{\PYGZgt{}}
        \PYG{n+nt}{\PYGZlt{}suministra}\PYG{n+nt}{\PYGZgt{}}
            \PYG{n+nt}{\PYGZlt{}numprov}\PYG{n+nt}{\PYGZgt{}}v1\PYG{n+nt}{\PYGZlt{}/numprov\PYGZgt{}}
            \PYG{n+nt}{\PYGZlt{}numparte}\PYG{n+nt}{\PYGZgt{}}p1\PYG{n+nt}{\PYGZlt{}/numparte\PYGZgt{}}
            \PYG{n+nt}{\PYGZlt{}numproyecto}\PYG{n+nt}{\PYGZgt{}}y1\PYG{n+nt}{\PYGZlt{}/numproyecto\PYGZgt{}}
            \PYG{n+nt}{\PYGZlt{}cantidad}\PYG{n+nt}{\PYGZgt{}}200\PYG{n+nt}{\PYGZlt{}/cantidad\PYGZgt{}}
        \PYG{n+nt}{\PYGZlt{}/suministra\PYGZgt{}}
        \PYG{n+nt}{\PYGZlt{}suministra}\PYG{n+nt}{\PYGZgt{}}
            \PYG{n+nt}{\PYGZlt{}numprov}\PYG{n+nt}{\PYGZgt{}}v1\PYG{n+nt}{\PYGZlt{}/numprov\PYGZgt{}}
            \PYG{n+nt}{\PYGZlt{}numparte}\PYG{n+nt}{\PYGZgt{}}p1\PYG{n+nt}{\PYGZlt{}/numparte\PYGZgt{}}
            \PYG{n+nt}{\PYGZlt{}numproyecto}\PYG{n+nt}{\PYGZgt{}}y4\PYG{n+nt}{\PYGZlt{}/numproyecto\PYGZgt{}}
            \PYG{n+nt}{\PYGZlt{}cantidad}\PYG{n+nt}{\PYGZgt{}}700\PYG{n+nt}{\PYGZlt{}/cantidad\PYGZgt{}}
        \PYG{n+nt}{\PYGZlt{}/suministra\PYGZgt{}}
        \PYG{n+nt}{\PYGZlt{}suministra}\PYG{n+nt}{\PYGZgt{}}
            \PYG{n+nt}{\PYGZlt{}numprov}\PYG{n+nt}{\PYGZgt{}}v2\PYG{n+nt}{\PYGZlt{}/numprov\PYGZgt{}}
            \PYG{n+nt}{\PYGZlt{}numparte}\PYG{n+nt}{\PYGZgt{}}p3\PYG{n+nt}{\PYGZlt{}/numparte\PYGZgt{}}
            \PYG{n+nt}{\PYGZlt{}numproyecto}\PYG{n+nt}{\PYGZgt{}}y1\PYG{n+nt}{\PYGZlt{}/numproyecto\PYGZgt{}}
            \PYG{n+nt}{\PYGZlt{}cantidad}\PYG{n+nt}{\PYGZgt{}}400\PYG{n+nt}{\PYGZlt{}/cantidad\PYGZgt{}}
        \PYG{n+nt}{\PYGZlt{}/suministra\PYGZgt{}}
        \PYG{n+nt}{\PYGZlt{}suministra}\PYG{n+nt}{\PYGZgt{}}
            \PYG{n+nt}{\PYGZlt{}numprov}\PYG{n+nt}{\PYGZgt{}}v2\PYG{n+nt}{\PYGZlt{}/numprov\PYGZgt{}}
            \PYG{n+nt}{\PYGZlt{}numparte}\PYG{n+nt}{\PYGZgt{}}p3\PYG{n+nt}{\PYGZlt{}/numparte\PYGZgt{}}
            \PYG{n+nt}{\PYGZlt{}numproyecto}\PYG{n+nt}{\PYGZgt{}}y2\PYG{n+nt}{\PYGZlt{}/numproyecto\PYGZgt{}}
            \PYG{n+nt}{\PYGZlt{}cantidad}\PYG{n+nt}{\PYGZgt{}}200\PYG{n+nt}{\PYGZlt{}/cantidad\PYGZgt{}}
        \PYG{n+nt}{\PYGZlt{}/suministra\PYGZgt{}}
        \PYG{n+nt}{\PYGZlt{}suministra}\PYG{n+nt}{\PYGZgt{}}
            \PYG{n+nt}{\PYGZlt{}numprov}\PYG{n+nt}{\PYGZgt{}}v2\PYG{n+nt}{\PYGZlt{}/numprov\PYGZgt{}}
            \PYG{n+nt}{\PYGZlt{}numparte}\PYG{n+nt}{\PYGZgt{}}p3\PYG{n+nt}{\PYGZlt{}/numparte\PYGZgt{}}
            \PYG{n+nt}{\PYGZlt{}numproyecto}\PYG{n+nt}{\PYGZgt{}}y3\PYG{n+nt}{\PYGZlt{}/numproyecto\PYGZgt{}}
            \PYG{n+nt}{\PYGZlt{}cantidad}\PYG{n+nt}{\PYGZgt{}}300\PYG{n+nt}{\PYGZlt{}/cantidad\PYGZgt{}}
        \PYG{n+nt}{\PYGZlt{}/suministra\PYGZgt{}}
        \PYG{n+nt}{\PYGZlt{}suministra}\PYG{n+nt}{\PYGZgt{}}
            \PYG{n+nt}{\PYGZlt{}numprov}\PYG{n+nt}{\PYGZgt{}}v2\PYG{n+nt}{\PYGZlt{}/numprov\PYGZgt{}}
            \PYG{n+nt}{\PYGZlt{}numparte}\PYG{n+nt}{\PYGZgt{}}p3\PYG{n+nt}{\PYGZlt{}/numparte\PYGZgt{}}
            \PYG{n+nt}{\PYGZlt{}numproyecto}\PYG{n+nt}{\PYGZgt{}}y4\PYG{n+nt}{\PYGZlt{}/numproyecto\PYGZgt{}}
            \PYG{n+nt}{\PYGZlt{}cantidad}\PYG{n+nt}{\PYGZgt{}}500\PYG{n+nt}{\PYGZlt{}/cantidad\PYGZgt{}}
        \PYG{n+nt}{\PYGZlt{}/suministra\PYGZgt{}}
        \PYG{n+nt}{\PYGZlt{}suministra}\PYG{n+nt}{\PYGZgt{}}
            \PYG{n+nt}{\PYGZlt{}numprov}\PYG{n+nt}{\PYGZgt{}}v2\PYG{n+nt}{\PYGZlt{}/numprov\PYGZgt{}}
            \PYG{n+nt}{\PYGZlt{}numparte}\PYG{n+nt}{\PYGZgt{}}p3\PYG{n+nt}{\PYGZlt{}/numparte\PYGZgt{}}
            \PYG{n+nt}{\PYGZlt{}numproyecto}\PYG{n+nt}{\PYGZgt{}}y5\PYG{n+nt}{\PYGZlt{}/numproyecto\PYGZgt{}}
            \PYG{n+nt}{\PYGZlt{}cantidad}\PYG{n+nt}{\PYGZgt{}}600\PYG{n+nt}{\PYGZlt{}/cantidad\PYGZgt{}}
        \PYG{n+nt}{\PYGZlt{}/suministra\PYGZgt{}}
        \PYG{n+nt}{\PYGZlt{}suministra}\PYG{n+nt}{\PYGZgt{}}
            \PYG{n+nt}{\PYGZlt{}numprov}\PYG{n+nt}{\PYGZgt{}}v2\PYG{n+nt}{\PYGZlt{}/numprov\PYGZgt{}}
            \PYG{n+nt}{\PYGZlt{}numparte}\PYG{n+nt}{\PYGZgt{}}p3\PYG{n+nt}{\PYGZlt{}/numparte\PYGZgt{}}
            \PYG{n+nt}{\PYGZlt{}numproyecto}\PYG{n+nt}{\PYGZgt{}}y6\PYG{n+nt}{\PYGZlt{}/numproyecto\PYGZgt{}}
            \PYG{n+nt}{\PYGZlt{}cantidad}\PYG{n+nt}{\PYGZgt{}}400\PYG{n+nt}{\PYGZlt{}/cantidad\PYGZgt{}}
        \PYG{n+nt}{\PYGZlt{}/suministra\PYGZgt{}}
        \PYG{n+nt}{\PYGZlt{}suministra}\PYG{n+nt}{\PYGZgt{}}
            \PYG{n+nt}{\PYGZlt{}numprov}\PYG{n+nt}{\PYGZgt{}}v2\PYG{n+nt}{\PYGZlt{}/numprov\PYGZgt{}}
            \PYG{n+nt}{\PYGZlt{}numparte}\PYG{n+nt}{\PYGZgt{}}p3\PYG{n+nt}{\PYGZlt{}/numparte\PYGZgt{}}
            \PYG{n+nt}{\PYGZlt{}numproyecto}\PYG{n+nt}{\PYGZgt{}}y7\PYG{n+nt}{\PYGZlt{}/numproyecto\PYGZgt{}}
            \PYG{n+nt}{\PYGZlt{}cantidad}\PYG{n+nt}{\PYGZgt{}}600\PYG{n+nt}{\PYGZlt{}/cantidad\PYGZgt{}}
        \PYG{n+nt}{\PYGZlt{}/suministra\PYGZgt{}}
        \PYG{n+nt}{\PYGZlt{}suministra}\PYG{n+nt}{\PYGZgt{}}
            \PYG{n+nt}{\PYGZlt{}numprov}\PYG{n+nt}{\PYGZgt{}}v2\PYG{n+nt}{\PYGZlt{}/numprov\PYGZgt{}}
            \PYG{n+nt}{\PYGZlt{}numparte}\PYG{n+nt}{\PYGZgt{}}p5\PYG{n+nt}{\PYGZlt{}/numparte\PYGZgt{}}
            \PYG{n+nt}{\PYGZlt{}numproyecto}\PYG{n+nt}{\PYGZgt{}}y2\PYG{n+nt}{\PYGZlt{}/numproyecto\PYGZgt{}}
            \PYG{n+nt}{\PYGZlt{}cantidad}\PYG{n+nt}{\PYGZgt{}}100\PYG{n+nt}{\PYGZlt{}/cantidad\PYGZgt{}}
        \PYG{n+nt}{\PYGZlt{}/suministra\PYGZgt{}}
        \PYG{n+nt}{\PYGZlt{}suministra}\PYG{n+nt}{\PYGZgt{}}
            \PYG{n+nt}{\PYGZlt{}numprov}\PYG{n+nt}{\PYGZgt{}}v3\PYG{n+nt}{\PYGZlt{}/numprov\PYGZgt{}}
            \PYG{n+nt}{\PYGZlt{}numparte}\PYG{n+nt}{\PYGZgt{}}p3\PYG{n+nt}{\PYGZlt{}/numparte\PYGZgt{}}
            \PYG{n+nt}{\PYGZlt{}numproyecto}\PYG{n+nt}{\PYGZgt{}}y1\PYG{n+nt}{\PYGZlt{}/numproyecto\PYGZgt{}}
            \PYG{n+nt}{\PYGZlt{}cantidad}\PYG{n+nt}{\PYGZgt{}}200\PYG{n+nt}{\PYGZlt{}/cantidad\PYGZgt{}}
        \PYG{n+nt}{\PYGZlt{}/suministra\PYGZgt{}}
        \PYG{n+nt}{\PYGZlt{}suministra}\PYG{n+nt}{\PYGZgt{}}
            \PYG{n+nt}{\PYGZlt{}numprov}\PYG{n+nt}{\PYGZgt{}}v3\PYG{n+nt}{\PYGZlt{}/numprov\PYGZgt{}}
            \PYG{n+nt}{\PYGZlt{}numparte}\PYG{n+nt}{\PYGZgt{}}p4\PYG{n+nt}{\PYGZlt{}/numparte\PYGZgt{}}
            \PYG{n+nt}{\PYGZlt{}numproyecto}\PYG{n+nt}{\PYGZgt{}}y2\PYG{n+nt}{\PYGZlt{}/numproyecto\PYGZgt{}}
            \PYG{n+nt}{\PYGZlt{}cantidad}\PYG{n+nt}{\PYGZgt{}}500\PYG{n+nt}{\PYGZlt{}/cantidad\PYGZgt{}}
        \PYG{n+nt}{\PYGZlt{}/suministra\PYGZgt{}}
        \PYG{n+nt}{\PYGZlt{}suministra}\PYG{n+nt}{\PYGZgt{}}
            \PYG{n+nt}{\PYGZlt{}numprov}\PYG{n+nt}{\PYGZgt{}}v4\PYG{n+nt}{\PYGZlt{}/numprov\PYGZgt{}}
            \PYG{n+nt}{\PYGZlt{}numparte}\PYG{n+nt}{\PYGZgt{}}p6\PYG{n+nt}{\PYGZlt{}/numparte\PYGZgt{}}
            \PYG{n+nt}{\PYGZlt{}numproyecto}\PYG{n+nt}{\PYGZgt{}}y3\PYG{n+nt}{\PYGZlt{}/numproyecto\PYGZgt{}}
            \PYG{n+nt}{\PYGZlt{}cantidad}\PYG{n+nt}{\PYGZgt{}}300\PYG{n+nt}{\PYGZlt{}/cantidad\PYGZgt{}}
        \PYG{n+nt}{\PYGZlt{}/suministra\PYGZgt{}}
        \PYG{n+nt}{\PYGZlt{}suministra}\PYG{n+nt}{\PYGZgt{}}
            \PYG{n+nt}{\PYGZlt{}numprov}\PYG{n+nt}{\PYGZgt{}}v4\PYG{n+nt}{\PYGZlt{}/numprov\PYGZgt{}}
            \PYG{n+nt}{\PYGZlt{}numparte}\PYG{n+nt}{\PYGZgt{}}p6\PYG{n+nt}{\PYGZlt{}/numparte\PYGZgt{}}
            \PYG{n+nt}{\PYGZlt{}numproyecto}\PYG{n+nt}{\PYGZgt{}}y7\PYG{n+nt}{\PYGZlt{}/numproyecto\PYGZgt{}}
            \PYG{n+nt}{\PYGZlt{}cantidad}\PYG{n+nt}{\PYGZgt{}}300\PYG{n+nt}{\PYGZlt{}/cantidad\PYGZgt{}}
        \PYG{n+nt}{\PYGZlt{}/suministra\PYGZgt{}}
        \PYG{n+nt}{\PYGZlt{}suministra}\PYG{n+nt}{\PYGZgt{}}
            \PYG{n+nt}{\PYGZlt{}numprov}\PYG{n+nt}{\PYGZgt{}}v5\PYG{n+nt}{\PYGZlt{}/numprov\PYGZgt{}}
            \PYG{n+nt}{\PYGZlt{}numparte}\PYG{n+nt}{\PYGZgt{}}p2\PYG{n+nt}{\PYGZlt{}/numparte\PYGZgt{}}
            \PYG{n+nt}{\PYGZlt{}numproyecto}\PYG{n+nt}{\PYGZgt{}}y2\PYG{n+nt}{\PYGZlt{}/numproyecto\PYGZgt{}}
            \PYG{n+nt}{\PYGZlt{}cantidad}\PYG{n+nt}{\PYGZgt{}}200\PYG{n+nt}{\PYGZlt{}/cantidad\PYGZgt{}}
        \PYG{n+nt}{\PYGZlt{}/suministra\PYGZgt{}}
        \PYG{n+nt}{\PYGZlt{}suministra}\PYG{n+nt}{\PYGZgt{}}
            \PYG{n+nt}{\PYGZlt{}numprov}\PYG{n+nt}{\PYGZgt{}}v5\PYG{n+nt}{\PYGZlt{}/numprov\PYGZgt{}}
            \PYG{n+nt}{\PYGZlt{}numparte}\PYG{n+nt}{\PYGZgt{}}p2\PYG{n+nt}{\PYGZlt{}/numparte\PYGZgt{}}
            \PYG{n+nt}{\PYGZlt{}numproyecto}\PYG{n+nt}{\PYGZgt{}}y4\PYG{n+nt}{\PYGZlt{}/numproyecto\PYGZgt{}}
            \PYG{n+nt}{\PYGZlt{}cantidad}\PYG{n+nt}{\PYGZgt{}}100\PYG{n+nt}{\PYGZlt{}/cantidad\PYGZgt{}}
        \PYG{n+nt}{\PYGZlt{}/suministra\PYGZgt{}}
        \PYG{n+nt}{\PYGZlt{}suministra}\PYG{n+nt}{\PYGZgt{}}
            \PYG{n+nt}{\PYGZlt{}numprov}\PYG{n+nt}{\PYGZgt{}}v5\PYG{n+nt}{\PYGZlt{}/numprov\PYGZgt{}}
            \PYG{n+nt}{\PYGZlt{}numparte}\PYG{n+nt}{\PYGZgt{}}p5\PYG{n+nt}{\PYGZlt{}/numparte\PYGZgt{}}
            \PYG{n+nt}{\PYGZlt{}numproyecto}\PYG{n+nt}{\PYGZgt{}}y5\PYG{n+nt}{\PYGZlt{}/numproyecto\PYGZgt{}}
            \PYG{n+nt}{\PYGZlt{}cantidad}\PYG{n+nt}{\PYGZgt{}}500\PYG{n+nt}{\PYGZlt{}/cantidad\PYGZgt{}}
        \PYG{n+nt}{\PYGZlt{}/suministra\PYGZgt{}}
        \PYG{n+nt}{\PYGZlt{}suministra}\PYG{n+nt}{\PYGZgt{}}
            \PYG{n+nt}{\PYGZlt{}numprov}\PYG{n+nt}{\PYGZgt{}}v5\PYG{n+nt}{\PYGZlt{}/numprov\PYGZgt{}}
            \PYG{n+nt}{\PYGZlt{}numparte}\PYG{n+nt}{\PYGZgt{}}p6\PYG{n+nt}{\PYGZlt{}/numparte\PYGZgt{}}
            \PYG{n+nt}{\PYGZlt{}numproyecto}\PYG{n+nt}{\PYGZgt{}}y2\PYG{n+nt}{\PYGZlt{}/numproyecto\PYGZgt{}}
            \PYG{n+nt}{\PYGZlt{}cantidad}\PYG{n+nt}{\PYGZgt{}}200\PYG{n+nt}{\PYGZlt{}/cantidad\PYGZgt{}}
        \PYG{n+nt}{\PYGZlt{}/suministra\PYGZgt{}}
        \PYG{n+nt}{\PYGZlt{}suministra}\PYG{n+nt}{\PYGZgt{}}
            \PYG{n+nt}{\PYGZlt{}numprov}\PYG{n+nt}{\PYGZgt{}}v5\PYG{n+nt}{\PYGZlt{}/numprov\PYGZgt{}}
            \PYG{n+nt}{\PYGZlt{}numparte}\PYG{n+nt}{\PYGZgt{}}p1\PYG{n+nt}{\PYGZlt{}/numparte\PYGZgt{}}
            \PYG{n+nt}{\PYGZlt{}numproyecto}\PYG{n+nt}{\PYGZgt{}}y4\PYG{n+nt}{\PYGZlt{}/numproyecto\PYGZgt{}}
            \PYG{n+nt}{\PYGZlt{}cantidad}\PYG{n+nt}{\PYGZgt{}}100\PYG{n+nt}{\PYGZlt{}/cantidad\PYGZgt{}}
        \PYG{n+nt}{\PYGZlt{}/suministra\PYGZgt{}}
        \PYG{n+nt}{\PYGZlt{}suministra}\PYG{n+nt}{\PYGZgt{}}
            \PYG{n+nt}{\PYGZlt{}numprov}\PYG{n+nt}{\PYGZgt{}}v5\PYG{n+nt}{\PYGZlt{}/numprov\PYGZgt{}}
            \PYG{n+nt}{\PYGZlt{}numparte}\PYG{n+nt}{\PYGZgt{}}p3\PYG{n+nt}{\PYGZlt{}/numparte\PYGZgt{}}
            \PYG{n+nt}{\PYGZlt{}numproyecto}\PYG{n+nt}{\PYGZgt{}}y4\PYG{n+nt}{\PYGZlt{}/numproyecto\PYGZgt{}}
            \PYG{n+nt}{\PYGZlt{}cantidad}\PYG{n+nt}{\PYGZgt{}}200\PYG{n+nt}{\PYGZlt{}/cantidad\PYGZgt{}}
        \PYG{n+nt}{\PYGZlt{}/suministra\PYGZgt{}}
        \PYG{n+nt}{\PYGZlt{}suministra}\PYG{n+nt}{\PYGZgt{}}
            \PYG{n+nt}{\PYGZlt{}numprov}\PYG{n+nt}{\PYGZgt{}}v5\PYG{n+nt}{\PYGZlt{}/numprov\PYGZgt{}}
            \PYG{n+nt}{\PYGZlt{}numparte}\PYG{n+nt}{\PYGZgt{}}p4\PYG{n+nt}{\PYGZlt{}/numparte\PYGZgt{}}
            \PYG{n+nt}{\PYGZlt{}numproyecto}\PYG{n+nt}{\PYGZgt{}}y4\PYG{n+nt}{\PYGZlt{}/numproyecto\PYGZgt{}}
            \PYG{n+nt}{\PYGZlt{}cantidad}\PYG{n+nt}{\PYGZgt{}}800\PYG{n+nt}{\PYGZlt{}/cantidad\PYGZgt{}}
        \PYG{n+nt}{\PYGZlt{}/suministra\PYGZgt{}}
        \PYG{n+nt}{\PYGZlt{}suministra}\PYG{n+nt}{\PYGZgt{}}
            \PYG{n+nt}{\PYGZlt{}numprov}\PYG{n+nt}{\PYGZgt{}}v5\PYG{n+nt}{\PYGZlt{}/numprov\PYGZgt{}}
            \PYG{n+nt}{\PYGZlt{}numparte}\PYG{n+nt}{\PYGZgt{}}p5\PYG{n+nt}{\PYGZlt{}/numparte\PYGZgt{}}
            \PYG{n+nt}{\PYGZlt{}numproyecto}\PYG{n+nt}{\PYGZgt{}}y4\PYG{n+nt}{\PYGZlt{}/numproyecto\PYGZgt{}}
            \PYG{n+nt}{\PYGZlt{}cantidad}\PYG{n+nt}{\PYGZgt{}}400\PYG{n+nt}{\PYGZlt{}/cantidad\PYGZgt{}}
        \PYG{n+nt}{\PYGZlt{}/suministra\PYGZgt{}}
        \PYG{n+nt}{\PYGZlt{}suministra}\PYG{n+nt}{\PYGZgt{}}
            \PYG{n+nt}{\PYGZlt{}numprov}\PYG{n+nt}{\PYGZgt{}}v5\PYG{n+nt}{\PYGZlt{}/numprov\PYGZgt{}}
            \PYG{n+nt}{\PYGZlt{}numparte}\PYG{n+nt}{\PYGZgt{}}p6\PYG{n+nt}{\PYGZlt{}/numparte\PYGZgt{}}
            \PYG{n+nt}{\PYGZlt{}numproyecto}\PYG{n+nt}{\PYGZgt{}}y4\PYG{n+nt}{\PYGZlt{}/numproyecto\PYGZgt{}}
            \PYG{n+nt}{\PYGZlt{}cantidad}\PYG{n+nt}{\PYGZgt{}}500\PYG{n+nt}{\PYGZlt{}/cantidad\PYGZgt{}}
        \PYG{n+nt}{\PYGZlt{}/suministra\PYGZgt{}}
    \PYG{n+nt}{\PYGZlt{}/suministros\PYGZgt{}}
\PYG{n+nt}{\PYGZlt{}/datos\PYGZgt{}}
\end{sphinxVerbatim}

En él podríamos ejecutar consultas como estas:
* Recuperar todos los proveedores con \sphinxcode{doc("datos.xml")/datos/proveedores}
* Recuperar todos los datos con \sphinxcode{doc("datos.xml")/datos/}
* Recuperar todas las partes con \sphinxcode{doc("datos.xml")/datos/partes}.

De hecho, podemos usar las mismas consultas con predicados XPath y así por ejemplo extraer los datos del proveedor cuyo numprov es “v1” con la consulta siguiente:

\begin{sphinxVerbatim}[commandchars=\\\{\}]
\PYG{n}{doc}\PYG{p}{(}\PYG{l+s+s2}{\PYGZdq{}}\PYG{l+s+s2}{datos.xml}\PYG{l+s+s2}{\PYGZdq{}}\PYG{p}{)}\PYG{o}{/}\PYG{n}{datos}\PYG{o}{/}\PYG{n}{proveedores}\PYG{o}{/}\PYG{n}{proveedor}\PYG{p}{[}\PYG{n+nd}{@numprov}\PYG{o}{=}\PYG{l+s+s1}{\PYGZsq{}}\PYG{l+s+s1}{v1}\PYG{l+s+s1}{\PYGZsq{}}\PYG{p}{]}
\end{sphinxVerbatim}

Que devuelve este resultado:

\begin{sphinxVerbatim}[commandchars=\\\{\}]
\PYG{n+nt}{\PYGZlt{}proveedor} \PYG{n+na}{numprov=}\PYG{l+s}{\PYGZdq{}v1\PYGZdq{}}\PYG{n+nt}{\PYGZgt{}}
    \PYG{n+nt}{\PYGZlt{}nombreprov}\PYG{n+nt}{\PYGZgt{}}Smith\PYG{n+nt}{\PYGZlt{}/nombreprov\PYGZgt{}}
    \PYG{n+nt}{\PYGZlt{}estado}\PYG{n+nt}{\PYGZgt{}}20\PYG{n+nt}{\PYGZlt{}/estado\PYGZgt{}}
    \PYG{n+nt}{\PYGZlt{}ciudad}\PYG{n+nt}{\PYGZgt{}}Londres\PYG{n+nt}{\PYGZlt{}/ciudad\PYGZgt{}}
\PYG{n+nt}{\PYGZlt{}/proveedor\PYGZgt{}}
\end{sphinxVerbatim}

Extraer los datos de partes cuyo color sea “Rojo”. La consulta XQuery sería:

\begin{sphinxVerbatim}[commandchars=\\\{\}]
\PYG{n}{doc}\PYG{p}{(}\PYG{l+s+s2}{\PYGZdq{}}\PYG{l+s+s2}{datos.xml}\PYG{l+s+s2}{\PYGZdq{}}\PYG{p}{)}\PYG{o}{/}\PYG{n}{datos}\PYG{o}{/}\PYG{n}{partes}\PYG{o}{/}\PYG{n}{parte}\PYG{p}{[}\PYG{n}{color}\PYG{o}{=}\PYG{l+s+s1}{\PYGZsq{}}\PYG{l+s+s1}{Rojo}\PYG{l+s+s1}{\PYGZsq{}}\PYG{p}{]}
\end{sphinxVerbatim}

Y el resultado sería:

\begin{sphinxVerbatim}[commandchars=\\\{\}]
\PYG{n+nt}{\PYGZlt{}parte} \PYG{n+na}{numparte=}\PYG{l+s}{\PYGZdq{}p1\PYGZdq{}}\PYG{n+nt}{\PYGZgt{}}
    \PYG{n+nt}{\PYGZlt{}nombreparte}\PYG{n+nt}{\PYGZgt{}}Tuerca\PYG{n+nt}{\PYGZlt{}/nombreparte\PYGZgt{}}
    \PYG{n+nt}{\PYGZlt{}color}\PYG{n+nt}{\PYGZgt{}}Rojo\PYG{n+nt}{\PYGZlt{}/color\PYGZgt{}}
    \PYG{n+nt}{\PYGZlt{}peso}\PYG{n+nt}{\PYGZgt{}}12\PYG{n+nt}{\PYGZlt{}/peso\PYGZgt{}}
    \PYG{n+nt}{\PYGZlt{}ciudad}\PYG{n+nt}{\PYGZgt{}}Londres\PYG{n+nt}{\PYGZlt{}/ciudad\PYGZgt{}}
\PYG{n+nt}{\PYGZlt{}/parte\PYGZgt{}}
\PYG{n+nt}{\PYGZlt{}parte} \PYG{n+na}{numparte=}\PYG{l+s}{\PYGZdq{}p4\PYGZdq{}}\PYG{n+nt}{\PYGZgt{}}
    \PYG{n+nt}{\PYGZlt{}nombreparte}\PYG{n+nt}{\PYGZgt{}}Tornillo\PYG{n+nt}{\PYGZlt{}/nombreparte\PYGZgt{}}
    \PYG{n+nt}{\PYGZlt{}color}\PYG{n+nt}{\PYGZgt{}}Rojo\PYG{n+nt}{\PYGZlt{}/color\PYGZgt{}}
    \PYG{n+nt}{\PYGZlt{}peso}\PYG{n+nt}{\PYGZgt{}}14\PYG{n+nt}{\PYGZlt{}/peso\PYGZgt{}}
    \PYG{n+nt}{\PYGZlt{}ciudad}\PYG{n+nt}{\PYGZgt{}}Londres\PYG{n+nt}{\PYGZlt{}/ciudad\PYGZgt{}}
\PYG{n+nt}{\PYGZlt{}/parte\PYGZgt{}}
\PYG{n+nt}{\PYGZlt{}parte} \PYG{n+na}{numparte=}\PYG{l+s}{\PYGZdq{}p6\PYGZdq{}}\PYG{n+nt}{\PYGZgt{}}
    \PYG{n+nt}{\PYGZlt{}nombreparte}\PYG{n+nt}{\PYGZgt{}}Engranaje\PYG{n+nt}{\PYGZlt{}/nombreparte\PYGZgt{}}
    \PYG{n+nt}{\PYGZlt{}color}\PYG{n+nt}{\PYGZgt{}}Rojo\PYG{n+nt}{\PYGZlt{}/color\PYGZgt{}}
    \PYG{n+nt}{\PYGZlt{}peso}\PYG{n+nt}{\PYGZgt{}}19\PYG{n+nt}{\PYGZlt{}/peso\PYGZgt{}}
    \PYG{n+nt}{\PYGZlt{}ciudad}\PYG{n+nt}{\PYGZgt{}}Londres\PYG{n+nt}{\PYGZlt{}/ciudad\PYGZgt{}}
\PYG{n+nt}{\PYGZlt{}/parte\PYGZgt{}}
\end{sphinxVerbatim}

En XQuery se pueden mezclar las marcas con el programa. Sin embargo, para poder distinguir lo que se tiene que ejecutar de lo que no, tendremos que encerrar nuestras sentencias XQuery entre llaves y dejar las marcas fuera de las llaves.

Así, esta consulta consigue generarnos un XML valido añadiendo un elemento raíz al conjunto de partes:

\begin{sphinxVerbatim}[commandchars=\\\{\}]
\PYG{o}{\PYGZlt{}}\PYG{n}{partesrojas}\PYG{o}{\PYGZgt{}}
\PYG{p}{\PYGZob{}}
    \PYG{n}{doc}\PYG{p}{(}\PYG{l+s+s2}{\PYGZdq{}}\PYG{l+s+s2}{datos.xml}\PYG{l+s+s2}{\PYGZdq{}}\PYG{p}{)}\PYG{o}{/}\PYG{n}{datos}\PYG{o}{/}\PYG{n}{partes}\PYG{o}{/}\PYG{n}{parte}\PYG{p}{[}\PYG{n}{color}\PYG{o}{=}\PYG{l+s+s1}{\PYGZsq{}}\PYG{l+s+s1}{Rojo}\PYG{l+s+s1}{\PYGZsq{}}\PYG{p}{]}
\PYG{p}{\PYGZcb{}}
\PYG{o}{\PYGZlt{}}\PYG{o}{/}\PYG{n}{partesrojas}\PYG{o}{\PYGZgt{}}
\end{sphinxVerbatim}

El resultado devuelto es este:

\begin{sphinxVerbatim}[commandchars=\\\{\}]
\PYG{n+nt}{\PYGZlt{}partesrojas}\PYG{n+nt}{\PYGZgt{}}
    \PYG{n+nt}{\PYGZlt{}parte} \PYG{n+na}{numparte=}\PYG{l+s}{\PYGZdq{}p1\PYGZdq{}}\PYG{n+nt}{\PYGZgt{}}
        \PYG{n+nt}{\PYGZlt{}nombreparte}\PYG{n+nt}{\PYGZgt{}}Tuerca\PYG{n+nt}{\PYGZlt{}/nombreparte\PYGZgt{}}
        \PYG{n+nt}{\PYGZlt{}color}\PYG{n+nt}{\PYGZgt{}}Rojo\PYG{n+nt}{\PYGZlt{}/color\PYGZgt{}}
        \PYG{n+nt}{\PYGZlt{}peso}\PYG{n+nt}{\PYGZgt{}}12\PYG{n+nt}{\PYGZlt{}/peso\PYGZgt{}}
        \PYG{n+nt}{\PYGZlt{}ciudad}\PYG{n+nt}{\PYGZgt{}}Londres\PYG{n+nt}{\PYGZlt{}/ciudad\PYGZgt{}}
    \PYG{n+nt}{\PYGZlt{}/parte\PYGZgt{}}\PYG{n+nt}{\PYGZlt{}parte} \PYG{n+na}{numparte=}\PYG{l+s}{\PYGZdq{}p4\PYGZdq{}}\PYG{n+nt}{\PYGZgt{}}
        \PYG{n+nt}{\PYGZlt{}nombreparte}\PYG{n+nt}{\PYGZgt{}}Tornillo\PYG{n+nt}{\PYGZlt{}/nombreparte\PYGZgt{}}
        \PYG{n+nt}{\PYGZlt{}color}\PYG{n+nt}{\PYGZgt{}}Rojo\PYG{n+nt}{\PYGZlt{}/color\PYGZgt{}}
        \PYG{n+nt}{\PYGZlt{}peso}\PYG{n+nt}{\PYGZgt{}}14\PYG{n+nt}{\PYGZlt{}/peso\PYGZgt{}}
        \PYG{n+nt}{\PYGZlt{}ciudad}\PYG{n+nt}{\PYGZgt{}}Londres\PYG{n+nt}{\PYGZlt{}/ciudad\PYGZgt{}}
    \PYG{n+nt}{\PYGZlt{}/parte\PYGZgt{}}\PYG{n+nt}{\PYGZlt{}parte} \PYG{n+na}{numparte=}\PYG{l+s}{\PYGZdq{}p6\PYGZdq{}}\PYG{n+nt}{\PYGZgt{}}
        \PYG{n+nt}{\PYGZlt{}nombreparte}\PYG{n+nt}{\PYGZgt{}}Engranaje\PYG{n+nt}{\PYGZlt{}/nombreparte\PYGZgt{}}
        \PYG{n+nt}{\PYGZlt{}color}\PYG{n+nt}{\PYGZgt{}}Rojo\PYG{n+nt}{\PYGZlt{}/color\PYGZgt{}}
        \PYG{n+nt}{\PYGZlt{}peso}\PYG{n+nt}{\PYGZgt{}}19\PYG{n+nt}{\PYGZlt{}/peso\PYGZgt{}}
        \PYG{n+nt}{\PYGZlt{}ciudad}\PYG{n+nt}{\PYGZgt{}}Londres\PYG{n+nt}{\PYGZlt{}/ciudad\PYGZgt{}}
    \PYG{n+nt}{\PYGZlt{}/parte\PYGZgt{}}
\PYG{n+nt}{\PYGZlt{}/partesrojas\PYGZgt{}}
\end{sphinxVerbatim}

Antes se ha mencionado que se puede usar \sphinxcode{WHERE} para crear condiciones. ¿Como cambiar entonces la consulta anterior para poner la condición en un \sphinxcode{WHERE} y no meterla entre corchetes?

Si queremos usar un \sphinxcode{where} es porque queremos filtrar un conjunto de elementos, y si queremos un conjunto de elementos necesitaremos un bucle \sphinxcode{for}. Y a su vez, si recorremos un conjunto de elementos tendremos que hacer algún procesamiento con ellos o al menos devolverlos de la manera normal.


\subsection{Bucles \sphinxstyleliteralintitle{for} en XQuery}
\label{\detokenize{tema6:bucles-for-en-xquery}}
Los bucles de XQuery son parecidos a los de Java. Hay una variable de bucle que iremos procesando de alguna manera. Dicha variable llevará siempre el simbolo del dolar (\$). Así, un bucle que recupera todas las partes sería:

\begin{sphinxVerbatim}[commandchars=\\\{\}]
\PYG{x}{for \PYGZdl{}p in doc(\PYGZdq{}datos.xml\PYGZdq{})/datos/partes/parte}
\PYG{x}{return \PYGZdl{}p}
\end{sphinxVerbatim}

Si ahora queremos filtrar las partes rojas haremos esto:

\begin{sphinxVerbatim}[commandchars=\\\{\}]
\PYG{x}{for \PYGZdl{}p in doc(\PYGZdq{}datos.xml\PYGZdq{})/datos/partes/parte}
\PYG{x}{where \PYGZdl{}p/color=\PYGZsq{}Rojo\PYGZsq{}}
\PYG{x}{return \PYGZdl{}p}
\end{sphinxVerbatim}

Y si ahora queremos un nuevo elemento raíz podremos hacer esto:

\begin{sphinxVerbatim}[commandchars=\\\{\}]
\PYG{n+nt}{\PYGZlt{}partesrojas}\PYG{n+nt}{\PYGZgt{}}
\PYGZob{}
    for \PYGZdl{}p in doc(\PYGZdq{}datos.xml\PYGZdq{})/datos/partes/parte
    where \PYGZdl{}p/color=\PYGZsq{}Rojo\PYGZsq{}
    return \PYGZdl{}p
\PYGZcb{}
\PYG{n+nt}{\PYGZlt{}/partesrojas\PYGZgt{}}
\end{sphinxVerbatim}

\begin{sphinxVerbatim}[commandchars=\\\{\}]
\PYG{n+nt}{\PYGZlt{}suministrosgrandes}\PYG{n+nt}{\PYGZgt{}}
\PYGZob{}
    for \PYGZdl{}suministra
    in doc(\PYGZdq{}datos.xml\PYGZdq{})/datos/suministros/suministra
    where \PYGZdl{}suministra/cantidad \PYGZgt{} 450
    return \PYGZdl{}suministra
\PYGZcb{}
\PYG{n+nt}{\PYGZlt{}/suministrosgrandes\PYGZgt{}}
\end{sphinxVerbatim}


\subsection{Ordenación en XQuery}
\label{\detokenize{tema6:ordenacion-en-xquery}}
Si se desea ordenar un conjunto de datos puede usarse la clásula \sphinxcode{order by} poniendo despues uno o varios elementos o atributos y usando \sphinxcode{ascending} o \sphinxcode{descending}, de manera similar a SQL.

Así, por ejemplo, para ordenar la consulta anterior por cantidad usaríamos esto:

\begin{sphinxVerbatim}[commandchars=\\\{\}]
\PYG{n+nt}{\PYGZlt{}suministrosgrandes}\PYG{n+nt}{\PYGZgt{}}
\PYGZob{}
    for \PYGZdl{}suministra
    in doc(\PYGZdq{}datos.xml\PYGZdq{})/datos/suministros/suministra
    where \PYGZdl{}suministra/cantidad \PYGZgt{} 450
    order by \PYGZdl{}suministra/cantidad descending
    return \PYGZdl{}suministra
\PYGZcb{}
\PYG{n+nt}{\PYGZlt{}/suministrosgrandes\PYGZgt{}}
\end{sphinxVerbatim}

Igual que en SQL se pueden combinar varios campos. Si por ejemplo quisiéramos ordenar por proveedor ascendente y luego por parte descendiente haríamos esto.

\begin{sphinxVerbatim}[commandchars=\\\{\}]
\PYG{n+nt}{\PYGZlt{}suministrosgrandes}\PYG{n+nt}{\PYGZgt{}}
\PYGZob{}
    for \PYGZdl{}suministra
    in doc(\PYGZdq{}datos.xml\PYGZdq{})/datos/suministros/suministra
    where \PYGZdl{}suministra/cantidad \PYGZgt{} 450
    order by \PYGZdl{}suministra/proveedor ascending,
            \PYGZdl{}suministra/parte descending
    return \PYGZdl{}suministra
\PYGZcb{}
\PYG{n+nt}{\PYGZlt{}/suministrosgrandes\PYGZgt{}}
\end{sphinxVerbatim}


\subsection{Funciones en XQuery}
\label{\detokenize{tema6:funciones-en-xquery}}
XQuery y XPath comparten funciones que permiten aplicar procesamiento extra a los nodos de un XML. A continuación se nombran algunas muy usadas:
\begin{itemize}
\item {} 
\sphinxcode{concat(\$fila, ' ')} concatena dos elementos, en este caso pone un espacio tras los datos de \sphinxcode{fila}.

\item {} 
\sphinxcode{string-length(\$elemento)} devuelve la longitud de una cadena.

\item {} 
XQuery también tiene «funciones de agregación» al estilo de SQL como \sphinxcode{sum}, \sphinxcode{count} y \sphinxcode{avg}. Además se pueden aplicar directamente a un conjunto sin necesidad de hacer un bucle \sphinxcode{for}.

\item {} 
En XQuery se usa \sphinxcode{distinct-values} en lugar de \sphinxcode{distinct} como hace SQL.

\end{itemize}


\subsection{Consultas de ejemplo}
\label{\detokenize{tema6:consultas-de-ejemplo}}
Las siguientes consultas van referidas a la  base de datos de proveedores y partes que aparecen al principio.


\subsubsection{Suministros grandes}
\label{\detokenize{tema6:suministros-grandes}}
Usando \sphinxcode{where} recuperar de la tabla de suministros todas las cantidades que sean mayores de 450. Encerrar los resultados dentro de un elemento raíz \sphinxcode{suministrosgrandes}

\begin{sphinxVerbatim}[commandchars=\\\{\}]
\PYG{n+nt}{\PYGZlt{}suministrosgrandes}\PYG{n+nt}{\PYGZgt{}}
\PYGZob{}
for \PYGZdl{}fila in doc(\PYGZdq{}datos.xml\PYGZdq{})//suministra
where \PYGZdl{}fila/cantidad \PYGZgt{} 450
return \PYGZdl{}fila/cantidad
\PYGZcb{}
\PYG{n+nt}{\PYGZlt{}/suministrosgrandes\PYGZgt{}}
\end{sphinxVerbatim}


\subsubsection{Renombrado de etiquetas}
\label{\detokenize{tema6:renombrado-de-etiquetas}}
Usando \sphinxcode{where} recuperar de la tabla de suministros todas las cantidades que sean mayores de 450 \sphinxstyleemphasis{y haciendo que las etiquetas cantidad pasen a llamarse numpartes} . Encerrar los resultados dentro de un elemento raíz \sphinxcode{suministrosgrandes}

\begin{sphinxVerbatim}[commandchars=\\\{\}]
\PYG{n+nt}{\PYGZlt{}suministrosgrandes}\PYG{n+nt}{\PYGZgt{}}
\PYGZob{}
for \PYGZdl{}fila in doc(\PYGZdq{}datos.xml\PYGZdq{})//suministra
where \PYGZdl{}fila/cantidad \PYGZgt{} 450
return \PYG{n+nt}{\PYGZlt{}numpartes}\PYG{n+nt}{\PYGZgt{}}
         \PYGZob{}\PYGZdl{}fila/cantidad/text()\PYGZcb{}
       \PYG{n+nt}{\PYGZlt{}/numpartes\PYGZgt{}}
\PYGZcb{}
\PYG{n+nt}{\PYGZlt{}/suministrosgrandes\PYGZgt{}}
\end{sphinxVerbatim}


\subsubsection{Cruces o joins}
\label{\detokenize{tema6:cruces-o-joins}}
Recuperar la ciudad de los proveedores que suministran mas de 450 partes. Devolverlo todo dentro de un elemento raíz \sphinxcode{resultados} haciendo que en cada fila haya un elemento \sphinxcode{datos} que tenga a su vez tres hijos:
\begin{itemize}
\item {} 
Un elemento \sphinxcode{numprov} donde se vea el numero de proveedor.

\item {} 
Un elemento \sphinxcode{nombre} donde se vea el nombre del proveedor.

\item {} 
Un elemento \sphinxcode{cantidadpartes} donde vaya la cantidad de partes suministradas (que evidentemente debería ser siempre mayor de 450)

\end{itemize}

\begin{sphinxVerbatim}[commandchars=\\\{\}]
\PYG{n+nt}{\PYGZlt{}resultados}\PYG{n+nt}{\PYGZgt{}}
\PYGZob{}
for \PYGZdl{}proveedor in doc(\PYGZdq{}datos.xml\PYGZdq{})/datos/proveedores/proveedor
for \PYGZdl{}suministra in doc(\PYGZdq{}datos.xml\PYGZdq{})/datos/suministros/suministra
where \PYGZdl{}proveedor/@numprov=\PYGZdl{}suministra/numprov
and \PYGZdl{}suministra/cantidad \PYGZgt{} 450
return \PYG{n+nt}{\PYGZlt{}datos}\PYG{n+nt}{\PYGZgt{}}
        \PYG{n+nt}{\PYGZlt{}numprov}\PYG{n+nt}{\PYGZgt{}}
        \PYGZob{}string(\PYGZdl{}proveedor/@numprov)\PYGZcb{}
        \PYG{n+nt}{\PYGZlt{}/numprov\PYGZgt{}}
        \PYG{n+nt}{\PYGZlt{}nombre}\PYG{n+nt}{\PYGZgt{}}
        \PYGZob{}\PYGZdl{}proveedor/nombreprov/text()\PYGZcb{}
        \PYG{n+nt}{\PYGZlt{}/nombre\PYGZgt{}}
        \PYG{n+nt}{\PYGZlt{}cantidadpartes}\PYG{n+nt}{\PYGZgt{}}
        \PYGZob{}\PYGZdl{}suministra/cantidad/text()\PYGZcb{}
        \PYG{n+nt}{\PYGZlt{}/cantidadpartes\PYGZgt{}}
       \PYG{n+nt}{\PYGZlt{}/datos\PYGZgt{}}
\PYGZcb{}
\PYG{n+nt}{\PYGZlt{}/resultados\PYGZgt{}}
\end{sphinxVerbatim}


\section{Fundamentos de DOM con Java}
\label{\detokenize{tema6:fundamentos-de-dom-con-java}}
En primer lugar va a ser necesario importar las clases correctas para poder usar DOM. La línea correcta es

\begin{sphinxVerbatim}[commandchars=\\\{\}]
\PYG{k+kn}{import} \PYG{n+nn}{javax.xml.parsers.*}\PYG{o}{;}
\PYG{k+kn}{import} \PYG{n+nn}{org.w3c.dom.*}\PYG{o}{;}
\end{sphinxVerbatim}

Un parser es un programa que analiza la sintaxis de un fichero, en nuestro caso un fichero XML. En castellano se debería decir analizador o analizador gramatical.


\section{Ejemplo de base}
\label{\detokenize{tema6:ejemplo-de-base}}
\begin{sphinxVerbatim}[commandchars=\\\{\}]
\PYG{k+kn}{package} \PYG{n+nn}{com.ies}\PYG{o}{;}
\PYG{k+kn}{import} \PYG{n+nn}{javax.xml.parsers.*}\PYG{o}{;}
\PYG{k+kn}{import} \PYG{n+nn}{java.io.File}\PYG{o}{;}
\PYG{k+kn}{import} \PYG{n+nn}{org.w3c.dom.*}\PYG{o}{;}

\PYG{k+kd}{public} \PYG{k+kd}{class} \PYG{n+nc}{ProcesadorXML} \PYG{o}{\PYGZob{}}
        \PYG{k+kd}{public} \PYG{k+kt}{void} \PYG{n+nf}{procesarArchivo}\PYG{o}{(}\PYG{n}{String} \PYG{n}{nombreArchivo}\PYG{o}{)}\PYG{o}{\PYGZob{}}
                \PYG{n}{DocumentBuilderFactory} \PYG{n}{fabrica}\PYG{o}{;}
                \PYG{n}{DocumentBuilder} \PYG{n}{constructor}\PYG{o}{;}
                \PYG{n}{Document} \PYG{n}{documentoXML}\PYG{o}{;}
                \PYG{n}{File} \PYG{n}{fichero}\PYG{o}{=}\PYG{k}{new} \PYG{n}{File}\PYG{o}{(}\PYG{n}{nombreArchivo}\PYG{o}{)}\PYG{o}{;}
                \PYG{n}{fabrica}\PYG{o}{=}
                        \PYG{n}{DocumentBuilderFactory}\PYG{o}{.}\PYG{n+na}{newInstance}\PYG{o}{(}\PYG{o}{)}\PYG{o}{;}
                \PYG{n}{System}\PYG{o}{.}\PYG{n+na}{out}\PYG{o}{.}\PYG{n+na}{println}\PYG{o}{(}\PYG{l+s}{\PYGZdq{}Procesando \PYGZdq{}}\PYG{o}{+}\PYG{n}{nombreArchivo}\PYG{o}{)}\PYG{o}{;}
                \PYG{k}{try} \PYG{o}{\PYGZob{}}
                        \PYG{n}{constructor}\PYG{o}{=}
                          \PYG{n}{fabrica}\PYG{o}{.}\PYG{n+na}{newDocumentBuilder}\PYG{o}{(}\PYG{o}{)}\PYG{o}{;}
                        \PYG{n}{documentoXML}\PYG{o}{=}\PYG{n}{constructor}\PYG{o}{.}\PYG{n+na}{parse}\PYG{o}{(}\PYG{n}{fichero}\PYG{o}{)}\PYG{o}{;}
                \PYG{o}{\PYGZcb{}} \PYG{k}{catch} \PYG{o}{(}\PYG{n}{Exception} \PYG{n}{e}\PYG{o}{)} \PYG{o}{\PYGZob{}}
                        \PYG{c+c1}{// TODO Auto\PYGZhy{}generated catch block}
                        \PYG{n}{e}\PYG{o}{.}\PYG{n+na}{printStackTrace}\PYG{o}{(}\PYG{o}{)}\PYG{o}{;}
                \PYG{o}{\PYGZcb{}}



        \PYG{o}{\PYGZcb{}}
        \PYG{k+kd}{public} \PYG{k+kd}{static} \PYG{k+kt}{void} \PYG{n+nf}{main} \PYG{o}{(}\PYG{n}{String}\PYG{o}{[}\PYG{o}{]} \PYG{n}{argumentos}\PYG{o}{)}\PYG{o}{\PYGZob{}}
                \PYG{n}{System}\PYG{o}{.}\PYG{n+na}{out}\PYG{o}{.}\PYG{n+na}{println}\PYG{o}{(}\PYG{l+s}{\PYGZdq{}Probando...\PYGZdq{}}\PYG{o}{)}\PYG{o}{;}
                \PYG{n}{ProcesadorXML} \PYG{n}{proc}\PYG{o}{=}\PYG{k}{new} \PYG{n}{ProcesadorXML}\PYG{o}{(}\PYG{o}{)}\PYG{o}{;}
                \PYG{n}{proc}\PYG{o}{.}\PYG{n+na}{procesarArchivo}\PYG{o}{(}\PYG{l+s}{\PYGZdq{}bolsas.xml\PYGZdq{}}\PYG{o}{)}\PYG{o}{;}
        \PYG{o}{\PYGZcb{}}
\PYG{o}{\PYGZcb{}}
\end{sphinxVerbatim}


\section{La clase Document}
\label{\detokenize{tema6:la-clase-document}}
La clase Document es una representación Java que almacena en memoria un archivo XML.Mediante esta clase y otras clases compañeras podremos recorrer cualquier punto del archivo XML.

Este recorrido se basa siempre en la visita de nodos hijo o nodos hermano. No todos los nodos son iguales y se debe tener presente que en un nodo podríamos encontrar que los saltos de línea pueden ser un problema a la hora de recorrer el árbol DOM.

Por ejemplo, dado un documento, se debe empezar obteniendo la raíz. Este elemento se llama también el “elemento documento” y podemos obtenerlo así:

\begin{sphinxVerbatim}[commandchars=\\\{\}]
\PYG{n}{documento}\PYG{o}{=}\PYG{n}{constructor}\PYG{o}{.}\PYG{n+na}{parse}\PYG{o}{(}\PYG{n}{archivoXML}\PYG{o}{)}\PYG{o}{;}
\PYG{n}{Element} \PYG{n}{raiz}\PYG{o}{=}\PYG{n}{documento}\PYG{o}{.}\PYG{n+na}{getDocumentElement}\PYG{o}{(}\PYG{o}{)}\PYG{o}{;}
\PYG{n}{System}\PYG{o}{.}\PYG{n+na}{out}\PYG{o}{.}\PYG{n+na}{println}\PYG{o}{(}\PYG{n}{raiz}\PYG{o}{.}\PYG{n+na}{getNodeName}\PYG{o}{(}\PYG{o}{)}\PYG{o}{)}\PYG{o}{;}
\end{sphinxVerbatim}

La clase principal que nos interesa es la clase \sphinxcode{Node}, siendo \sphinxcode{Document} su clase Hija. Algunos métodos de interés son estos:
\begin{itemize}
\item {} 
\sphinxcode{getDocumentElement()} obtiene el elemento raíz, a partir del cual podremos empezar a «navegar» a través de los elementos.

\item {} 
\sphinxcode{getFirstChild()} obtiene el primer elemento hijo del nodo que estemos visitando.

\item {} 
\sphinxcode{getParentNode()} nos permite obtener el nodo padre de un cierto nodo.

\item {} 
\sphinxcode{getChildNodes()} obtiene todos los nodos hijo.

\item {} 
\sphinxcode{getNextSibling()} obtiene el siguiente nodo hermano.

\item {} 
\sphinxcode{getChildNodes()} devuelve un \sphinxcode{NodeList} con todos los hijos de un elemento. Esta \sphinxcode{NodeList} se puede recorrer con un \sphinxcode{for}, obteniendo el tamaño de la lista con \sphinxcode{getLength()} y extrayendo los elementos con el método \sphinxcode{item(posicion)}

\item {} 
\sphinxcode{getNodeType()} es un método que nos indica el tipo de nodo (devuelve un \sphinxcode{short}). Podemos comparar con \sphinxcode{Node.ELEMENT\_NODE} para ver si el nodo es realmente un elemento.

\item {} 
Otro método de utilidad es \sphinxcode{getElementsByTagName} que extrae todos los subelementos que tengan un cierto nombre de etiqueta.

\end{itemize}

\begin{sphinxadmonition}{hint}{Consejo:}
En general, hay muchas clases que proporcionan más métodos de utilidad, como por ejemplo la clase \sphinxcode{Element}. En muchas ocasiones, podremos hacer un \sphinxcode{cast} y aprovecharnos de ellos.
\end{sphinxadmonition}

Cuando se procesan archivos, se debe tener especial importancia a los espacios en blanco que pueda haber. Estos dos archivos no son iguales:

\begin{sphinxVerbatim}[commandchars=\\\{\}]
El primer hijo de listado es \PYG{n+nt}{\PYGZlt{}futuro}\PYG{n+nt}{\PYGZgt{}}
\PYG{n+nt}{\PYGZlt{}listado}\PYG{n+nt}{\PYGZgt{}}\PYG{n+nt}{\PYGZlt{}futuro}\PYG{n+nt}{\PYGZgt{}}...\PYG{n+nt}{\PYGZlt{}/futuro\PYGZgt{}}\PYG{n+nt}{\PYGZlt{}/listado\PYGZgt{}}
\end{sphinxVerbatim}

\begin{sphinxVerbatim}[commandchars=\\\{\}]
Aquí el primer hijo de listado es \PYGZbs{}n
\PYG{n+nt}{\PYGZlt{}listado}\PYG{n+nt}{\PYGZgt{}}
        \PYG{n+nt}{\PYGZlt{}futuro}\PYG{n+nt}{\PYGZgt{}}...\PYG{n+nt}{\PYGZlt{}/futuro\PYGZgt{}}
\PYG{n+nt}{\PYGZlt{}/listado\PYGZgt{}}
\end{sphinxVerbatim}


\section{Ejercicios}
\label{\detokenize{tema6:ejercicios}}

\subsection{Ejercicio}
\label{\detokenize{tema6:ejercicio}}
Dado el siguiente archivo XML crear un programa que muestre todos los nombres:

\begin{sphinxVerbatim}[commandchars=\\\{\}]
\PYG{n+nt}{\PYGZlt{}listaempleados}\PYG{n+nt}{\PYGZgt{}}
        \PYG{n+nt}{\PYGZlt{}empleado} \PYG{n+na}{edad=}\PYG{l+s}{\PYGZdq{}27\PYGZdq{}}\PYG{n+nt}{\PYGZgt{}}
                \PYG{n+nt}{\PYGZlt{}nombre}\PYG{n+nt}{\PYGZgt{}}Pepe Perez\PYG{n+nt}{\PYGZlt{}/nombre\PYGZgt{}}
                \PYG{n+nt}{\PYGZlt{}categoria}\PYG{n+nt}{\PYGZgt{}}Empleado\PYG{n+nt}{\PYGZlt{}/categoria\PYGZgt{}}
        \PYG{n+nt}{\PYGZlt{}/empleado\PYGZgt{}}
        \PYG{n+nt}{\PYGZlt{}empleado} \PYG{n+na}{edad=}\PYG{l+s}{\PYGZdq{}34\PYGZdq{}}\PYG{n+nt}{\PYGZgt{}}
                \PYG{n+nt}{\PYGZlt{}nombre}\PYG{n+nt}{\PYGZgt{}}Juan Sanchez\PYG{n+nt}{\PYGZlt{}/nombre\PYGZgt{}}
                \PYG{n+nt}{\PYGZlt{}categoria}\PYG{n+nt}{\PYGZgt{}}Gerente\PYG{n+nt}{\PYGZlt{}/categoria\PYGZgt{}}
        \PYG{n+nt}{\PYGZlt{}/empleado\PYGZgt{}}
\PYG{n+nt}{\PYGZlt{}/listaempleados\PYGZgt{}}
\end{sphinxVerbatim}

La solución podría ser algo así:

\begin{sphinxVerbatim}[commandchars=\\\{\}]
\PYG{k+kd}{public} \PYG{k+kd}{class} \PYG{n+nc}{ProcesadorXML} \PYG{o}{\PYGZob{}}
        \PYG{n}{String} \PYG{n}{ruta}\PYG{o}{;}
        \PYG{k+kd}{public} \PYG{n+nf}{ProcesadorXML}\PYG{o}{(}\PYG{n}{String} \PYG{n}{ruta}\PYG{o}{)}\PYG{o}{\PYGZob{}}
                \PYG{k}{this}\PYG{o}{.}\PYG{n+na}{ruta}\PYG{o}{=}\PYG{n}{ruta}\PYG{o}{;}
        \PYG{o}{\PYGZcb{}}
        \PYG{k+kd}{public} \PYG{n}{Element} \PYG{n+nf}{getRaiz}\PYG{o}{(}\PYG{o}{)}
                        \PYG{k+kd}{throws} \PYG{n}{ParserConfigurationException}\PYG{o}{,} \PYG{n}{SAXException}\PYG{o}{,} \PYG{n}{IOException}\PYG{o}{\PYGZob{}}
                \PYG{n}{DocumentBuilderFactory} \PYG{n}{fabrica}\PYG{o}{;}
                \PYG{n}{fabrica}\PYG{o}{=}
                \PYG{n}{DocumentBuilderFactory}\PYG{o}{.}\PYG{n+na}{newInstance}\PYG{o}{(}\PYG{o}{)}\PYG{o}{;}
                \PYG{c+cm}{/* A partir de un fichero XML}
\PYG{c+cm}{                 * crea el objeto documento en memoria*/}
                \PYG{n}{DocumentBuilder} \PYG{n}{creadorObjDocumento}\PYG{o}{;}
                \PYG{n}{creadorObjDocumento}\PYG{o}{=}
                        \PYG{n}{fabrica}\PYG{o}{.}\PYG{n+na}{newDocumentBuilder}\PYG{o}{(}\PYG{o}{)}\PYG{o}{;}
                \PYG{n}{FileInputStream} \PYG{n}{fich}\PYG{o}{;}
                \PYG{n}{fich}\PYG{o}{=}\PYG{k}{new} \PYG{n}{FileInputStream}\PYG{o}{(}\PYG{k}{this}\PYG{o}{.}\PYG{n+na}{ruta}\PYG{o}{)}\PYG{o}{;}

                \PYG{c+cm}{/* Analiza el XML y}
\PYG{c+cm}{                 * lo carga en memoria */}
                \PYG{n}{Document} \PYG{n}{documento}\PYG{o}{;}
                \PYG{n}{documento}\PYG{o}{=}
                        \PYG{n}{creadorObjDocumento}\PYG{o}{.}\PYG{n+na}{parse}\PYG{o}{(}\PYG{n}{fich}\PYG{o}{)}\PYG{o}{;}
                \PYG{n}{Element} \PYG{n}{raiz}\PYG{o}{;}
                \PYG{n}{raiz}\PYG{o}{=}\PYG{n}{documento}\PYG{o}{.}\PYG{n+na}{getDocumentElement}\PYG{o}{(}\PYG{o}{)}\PYG{o}{;}
                \PYG{k}{return} \PYG{n}{raiz}\PYG{o}{;}
        \PYG{o}{\PYGZcb{}}
        \PYG{c+cm}{/* Este método imprime todos los nombres*/}
        \PYG{k+kd}{public} \PYG{k+kt}{void} \PYG{n+nf}{todosNombres}\PYG{o}{(}\PYG{o}{)}
                        \PYG{k+kd}{throws} \PYG{n}{ParserConfigurationException}\PYG{o}{,}
                                \PYG{n}{SAXException}\PYG{o}{,} \PYG{n}{IOException}
        \PYG{o}{\PYGZob{}}
                \PYG{n}{Element} \PYG{n}{raiz}\PYG{o}{=}\PYG{n}{getRaiz}\PYG{o}{(}\PYG{o}{)}\PYG{o}{;}
                \PYG{n}{Node} \PYG{n}{hijo}\PYG{o}{=}\PYG{n}{raiz}\PYG{o}{.}\PYG{n+na}{getFirstChild}\PYG{o}{(}\PYG{o}{)}\PYG{o}{;}
                \PYG{k}{while} \PYG{o}{(}\PYG{n}{hijo}\PYG{o}{!}\PYG{o}{=}\PYG{k+kc}{null}\PYG{o}{)}\PYG{o}{\PYGZob{}}
                        \PYG{n}{String} \PYG{n}{nombreElemento}\PYG{o}{;}
                        \PYG{n}{nombreElemento}\PYG{o}{=}\PYG{n}{hijo}\PYG{o}{.}\PYG{n+na}{getNodeName}\PYG{o}{(}\PYG{o}{)}\PYG{o}{;}
                        \PYG{k}{if} \PYG{o}{(}\PYG{n}{nombreElemento}\PYG{o}{.}\PYG{n+na}{equals}\PYG{o}{(}\PYG{l+s}{\PYGZdq{}empleado\PYGZdq{}}\PYG{o}{)}\PYG{o}{)}\PYG{o}{\PYGZob{}}
                                \PYG{n}{Node} \PYG{n}{hijoFinLinea}\PYG{o}{=}\PYG{n}{hijo}\PYG{o}{.}\PYG{n+na}{getFirstChild}\PYG{o}{(}\PYG{o}{)}\PYG{o}{;}
                                \PYG{n}{Element} \PYG{n}{hijoNombre}\PYG{o}{=}\PYG{o}{(}\PYG{n}{Element}\PYG{o}{)} \PYG{n}{hijoFinLinea}\PYG{o}{.}\PYG{n+na}{getNextSibling}\PYG{o}{(}\PYG{o}{)}\PYG{o}{;}
                                \PYG{n}{String} \PYG{n}{contenido}\PYG{o}{=}\PYG{n}{hijoNombre}\PYG{o}{.}\PYG{n+na}{getTextContent}\PYG{o}{(}\PYG{o}{)}\PYG{o}{;}
                                \PYG{n}{System}\PYG{o}{.}\PYG{n+na}{out}\PYG{o}{.}\PYG{n+na}{println}\PYG{o}{(}\PYG{l+s}{\PYGZdq{}Empleado \PYGZdq{}}\PYG{o}{+}\PYG{n}{contenido}\PYG{o}{)}\PYG{o}{;}
                        \PYG{o}{\PYGZcb{}}
                        \PYG{n}{hijo}\PYG{o}{=}\PYG{n}{hijo}\PYG{o}{.}\PYG{n+na}{getNextSibling}\PYG{o}{(}\PYG{o}{)}\PYG{o}{;}
                \PYG{o}{\PYGZcb{}}
        \PYG{o}{\PYGZcb{}}
        \PYG{k+kd}{public} \PYG{k+kd}{static} \PYG{k+kt}{void} \PYG{n+nf}{main}\PYG{o}{(}\PYG{n}{String}\PYG{o}{[}\PYG{o}{]} \PYG{n}{args}\PYG{o}{)} \PYG{k+kd}{throws} \PYG{n}{ParserConfigurationException}\PYG{o}{,} \PYG{n}{SAXException}\PYG{o}{,} \PYG{n}{IOException} \PYG{o}{\PYGZob{}}
                \PYG{n}{ProcesadorXML} \PYG{n}{procesador}\PYG{o}{;}
                \PYG{n}{procesador}\PYG{o}{=}\PYG{k}{new} \PYG{n}{ProcesadorXML}\PYG{o}{(}
                        \PYG{l+s}{\PYGZdq{}D:/oscar/empleados.xml\PYGZdq{}}\PYG{o}{)}\PYG{o}{;}
                \PYG{n}{procesador}\PYG{o}{.}\PYG{n+na}{todosNombres}\PYG{o}{(}\PYG{o}{)}\PYG{o}{;}
        \PYG{o}{\PYGZcb{}}
\PYG{o}{\PYGZcb{}}
\end{sphinxVerbatim}

Ampliaciones:
\begin{itemize}
\item {} 
Añadir uno que devuelva todas las edades.

\item {} 
Añadir uno que devuelva los nombres de los empleados mayores de 30 (mostrando los nombres pero no las edades).

\item {} 
Añadir un método que diga cuantos empleados hay. El método debe ser capaz de tolerar que haya muchas líneas en blanco seguidas.

\end{itemize}

El siguiente código resuelve el problema de mostrar los mayores de cierta edad:

\begin{sphinxVerbatim}[commandchars=\\\{\}]
\PYG{k+kd}{public} \PYG{k+kt}{void} \PYG{n+nf}{mostrarMayoresDe}\PYG{o}{(}\PYG{k+kt}{int} \PYG{n}{edadMinima}\PYG{o}{)}
                \PYG{k+kd}{throws} \PYG{n}{ParserConfigurationException}\PYG{o}{,}
                \PYG{n}{SAXException}\PYG{o}{,} \PYG{n}{IOException}
\PYG{o}{\PYGZob{}}
        \PYG{n}{Element} \PYG{n}{raiz}\PYG{o}{=}\PYG{n}{getRaiz}\PYG{o}{(}\PYG{o}{)}\PYG{o}{;}
        \PYG{n}{Node} \PYG{n}{finLinea}\PYG{o}{=}\PYG{n}{raiz}\PYG{o}{.}\PYG{n+na}{getFirstChild}\PYG{o}{(}\PYG{o}{)}\PYG{o}{;}
        \PYG{n}{Element} \PYG{n}{empleado}\PYG{o}{=}\PYG{o}{(}\PYG{n}{Element}\PYG{o}{)} \PYG{n}{finLinea}\PYG{o}{.}\PYG{n+na}{getNextSibling}\PYG{o}{(}\PYG{o}{)}\PYG{o}{;}
        \PYG{k}{while} \PYG{o}{(}\PYG{n}{empleado}\PYG{o}{!}\PYG{o}{=}\PYG{k+kc}{null}\PYG{o}{)}\PYG{o}{\PYGZob{}}
                \PYG{n}{String} \PYG{n}{edad}\PYG{o}{=}\PYG{n}{empleado}\PYG{o}{.}\PYG{n+na}{getAttribute}\PYG{o}{(}\PYG{l+s}{\PYGZdq{}edad\PYGZdq{}}\PYG{o}{)}\PYG{o}{;}
                \PYG{k+kt}{int} \PYG{n}{iEdad}\PYG{o}{=}\PYG{n}{Integer}\PYG{o}{.}\PYG{n+na}{parseInt}\PYG{o}{(}\PYG{n}{edad}\PYG{o}{)}\PYG{o}{;}
                \PYG{k}{if} \PYG{o}{(}\PYG{n}{iEdad}\PYG{o}{\PYGZgt{}}\PYG{n}{edadMinima}\PYG{o}{)}\PYG{o}{\PYGZob{}}
                        \PYG{n}{finLinea}\PYG{o}{=}\PYG{n}{empleado}\PYG{o}{.}\PYG{n+na}{getFirstChild}\PYG{o}{(}\PYG{o}{)}\PYG{o}{;}
                        \PYG{n}{Element} \PYG{n}{elemNombre}\PYG{o}{=}\PYG{o}{(}\PYG{n}{Element}\PYG{o}{)}
                                        \PYG{n}{finLinea}\PYG{o}{.}\PYG{n+na}{getNextSibling}\PYG{o}{(}\PYG{o}{)}\PYG{o}{;}
                        \PYG{n}{String} \PYG{n}{nombreEmpleado}\PYG{o}{=}
                                        \PYG{n}{elemNombre}\PYG{o}{.}\PYG{n+na}{getTextContent}\PYG{o}{(}\PYG{o}{)}\PYG{o}{;}
                        \PYG{n}{System}\PYG{o}{.}\PYG{n+na}{out}\PYG{o}{.}\PYG{n+na}{println}\PYG{o}{(}\PYG{n}{nombreEmpleado} \PYG{o}{+}
                                        \PYG{l+s}{\PYGZdq{} es mayor de \PYGZdq{}}\PYG{o}{+}\PYG{n}{iEdad}\PYG{o}{)}\PYG{o}{;}
                \PYG{o}{\PYGZcb{}} \PYG{c+c1}{//Fin del if}
                \PYG{n}{finLinea}\PYG{o}{=}\PYG{n}{empleado}\PYG{o}{.}\PYG{n+na}{getNextSibling}\PYG{o}{(}\PYG{o}{)}\PYG{o}{;}
                \PYG{n}{empleado}\PYG{o}{=}\PYG{o}{(}\PYG{n}{Element}\PYG{o}{)} \PYG{n}{finLinea}\PYG{o}{.}\PYG{n+na}{getNextSibling}\PYG{o}{(}\PYG{o}{)}\PYG{o}{;}
        \PYG{o}{\PYGZcb{}}       \PYG{c+c1}{//Fin del while}
\PYG{o}{\PYGZcb{}} \PYG{c+c1}{//Fin del método}
\end{sphinxVerbatim}

El método \sphinxcode{getElementsByTagName} puede facilitar mucho el resolver ciertas tareas. Por ejemplo, supongamos que queremos resolver el problema de contar cuantos empleados hay:

\begin{sphinxVerbatim}[commandchars=\\\{\}]
\PYG{k+kd}{public} \PYG{k+kt}{int} \PYG{n+nf}{contarEmpleados}\PYG{o}{(}\PYG{o}{)}
                \PYG{k+kd}{throws} \PYG{n}{ParserConfigurationException}\PYG{o}{,} \PYG{n}{SAXException}\PYG{o}{,} \PYG{n}{IOException}\PYG{o}{\PYGZob{}}
        \PYG{k+kt}{int} \PYG{n}{numEmpleados}\PYG{o}{=}\PYG{l+m+mi}{0}\PYG{o}{;}
        \PYG{n}{Element} \PYG{n}{raiz}\PYG{o}{=}\PYG{n}{getRaiz}\PYG{o}{(}\PYG{o}{)}\PYG{o}{;}
        \PYG{n}{NodeList} \PYG{n}{lista}\PYG{o}{=}\PYG{n}{raiz}\PYG{o}{.}\PYG{n+na}{getElementsByTagName}\PYG{o}{(}\PYG{l+s}{\PYGZdq{}empleado\PYGZdq{}}\PYG{o}{)}\PYG{o}{;}
        \PYG{n}{numEmpleados}\PYG{o}{=}\PYG{n}{lista}\PYG{o}{.}\PYG{n+na}{getLength}\PYG{o}{(}\PYG{o}{)}\PYG{o}{;}
        \PYG{k}{return} \PYG{n}{numEmpleados}\PYG{o}{;}
\PYG{o}{\PYGZcb{}}
\end{sphinxVerbatim}


\subsection{Ejercicio}
\label{\detokenize{tema6:id1}}
Extraer la raíz de un archivo XML

\begin{sphinxVerbatim}[commandchars=\\\{\}]
\PYG{k+kd}{public} \PYG{n}{Node} \PYG{n+nf}{extraerRaiz}\PYG{o}{(}\PYG{n}{String} \PYG{n}{nombreArchivo}\PYG{o}{)}\PYG{o}{\PYGZob{}}
        \PYG{n}{DocumentBuilderFactory} \PYG{n}{fabrica}\PYG{o}{;}
        \PYG{n}{DocumentBuilder} \PYG{n}{constructor}\PYG{o}{;}
        \PYG{n}{Document} \PYG{n}{documentoXML}\PYG{o}{=}\PYG{k+kc}{null}\PYG{o}{;}
        \PYG{n}{File} \PYG{n}{fichero}\PYG{o}{=}\PYG{k}{new} \PYG{n}{File}\PYG{o}{(}\PYG{n}{nombreArchivo}\PYG{o}{)}\PYG{o}{;}
        \PYG{n}{fabrica}\PYG{o}{=}
                \PYG{n}{DocumentBuilderFactory}\PYG{o}{.}\PYG{n+na}{newInstance}\PYG{o}{(}\PYG{o}{)}\PYG{o}{;}
        \PYG{n}{System}\PYG{o}{.}\PYG{n+na}{out}\PYG{o}{.}\PYG{n+na}{println}\PYG{o}{(}\PYG{l+s}{\PYGZdq{}Procesando \PYGZdq{}}\PYG{o}{+}\PYG{n}{nombreArchivo}\PYG{o}{)}\PYG{o}{;}
        \PYG{k}{try} \PYG{o}{\PYGZob{}}
                \PYG{n}{constructor}\PYG{o}{=}
                  \PYG{n}{fabrica}\PYG{o}{.}\PYG{n+na}{newDocumentBuilder}\PYG{o}{(}\PYG{o}{)}\PYG{o}{;}
                \PYG{n}{documentoXML}\PYG{o}{=}\PYG{n}{constructor}\PYG{o}{.}\PYG{n+na}{parse}\PYG{o}{(}\PYG{n}{fichero}\PYG{o}{)}\PYG{o}{;}
        \PYG{o}{\PYGZcb{}} \PYG{k}{catch} \PYG{o}{(}\PYG{n}{Exception} \PYG{n}{e}\PYG{o}{)} \PYG{o}{\PYGZob{}}
                \PYG{c+c1}{// TODO Auto\PYGZhy{}generated catch block}
                \PYG{n}{e}\PYG{o}{.}\PYG{n+na}{printStackTrace}\PYG{o}{(}\PYG{o}{)}\PYG{o}{;}
        \PYG{o}{\PYGZcb{}}
        \PYG{k}{return} \PYG{n}{documentoXML}\PYG{o}{.}\PYG{n+na}{getDocumentElement}\PYG{o}{(}\PYG{o}{)}\PYG{o}{;}

\PYG{o}{\PYGZcb{}}
\end{sphinxVerbatim}


\subsection{Ejercicio}
\label{\detokenize{tema6:id2}}
Imprimir todos los elementos hijo del archivo XML.

Una posibilidad es la siguiente:
\begin{itemize}
\item {} 
Usar \sphinxcode{getNextSibling} para ir recorriendo hermano a hermano hasta que se encuentre un \sphinxcode{null}

\item {} 
Para evitar los nodos texto, solo imprimiremos cosas cuando el \sphinxcode{getNodeType()} nos devuelva un tipo \sphinxcode{Node.ELEMENT\_NODE}

\end{itemize}

\begin{sphinxVerbatim}[commandchars=\\\{\}]
\PYG{k+kd}{public} \PYG{k+kt}{void} \PYG{n+nf}{imprimirHijos}\PYG{o}{(}\PYG{n}{Node} \PYG{n}{nodoRaiz}\PYG{o}{)}\PYG{o}{\PYGZob{}}
                \PYG{k}{if} \PYG{o}{(}\PYG{n}{nodoRaiz}\PYG{o}{=}\PYG{o}{=}\PYG{k+kc}{null}\PYG{o}{)}\PYG{o}{\PYGZob{}}
                        \PYG{n}{System}\PYG{o}{.}\PYG{n+na}{out}\PYG{o}{.}\PYG{n+na}{println}\PYG{o}{(}\PYG{l+s}{\PYGZdq{}Imposible procesar raiz null\PYGZdq{}}\PYG{o}{)}\PYG{o}{;}
                        \PYG{k}{return} \PYG{o}{;}
                \PYG{o}{\PYGZcb{}}
                \PYG{n}{Node} \PYG{n}{nodo}\PYG{o}{=}\PYG{n}{nodoRaiz}\PYG{o}{.}\PYG{n+na}{getFirstChild}\PYG{o}{(}\PYG{o}{)}\PYG{o}{;}
                \PYG{k}{while} \PYG{o}{(}\PYG{n}{nodo}\PYG{o}{!}\PYG{o}{=}\PYG{k+kc}{null}\PYG{o}{)}\PYG{o}{\PYGZob{}}
                        \PYG{k+kt}{short} \PYG{n}{tipo}\PYG{o}{=}\PYG{n}{nodo}\PYG{o}{.}\PYG{n+na}{getNodeType}\PYG{o}{(}\PYG{o}{)}\PYG{o}{;}
                        \PYG{k}{if} \PYG{o}{(}\PYG{n}{tipo}\PYG{o}{=}\PYG{o}{=}\PYG{n}{nodo}\PYG{o}{.}\PYG{n+na}{ELEMENT\PYGZus{}NODE}\PYG{o}{)}\PYG{o}{\PYGZob{}}
                                \PYG{n}{System}\PYG{o}{.}\PYG{n+na}{out}\PYG{o}{.}\PYG{n+na}{println}\PYG{o}{(}\PYG{l+s}{\PYGZdq{}Nodo hijo:\PYGZdq{}}\PYG{o}{+}\PYG{n}{nodo}\PYG{o}{.}\PYG{n+na}{getNodeName}\PYG{o}{(}\PYG{o}{)}\PYG{o}{)}\PYG{o}{;}
                        \PYG{o}{\PYGZcb{}}
                        \PYG{n}{nodo}\PYG{o}{=}\PYG{n}{nodo}\PYG{o}{.}\PYG{n+na}{getNextSibling}\PYG{o}{(}\PYG{o}{)}\PYG{o}{;}
                \PYG{o}{\PYGZcb{}}
        \PYG{o}{\PYGZcb{}}
\end{sphinxVerbatim}

Otra posibilidad sería usar el \sphinxcode{getChildNodes} que nos devuelve un vector con todos los hijos. Sin embargo ocurrirá lo mismo, deberemos evitar el visitar nodos texto, solo nos interesan los nodos Elemento.

\begin{sphinxVerbatim}[commandchars=\\\{\}]
\PYG{k+kd}{public} \PYG{k+kt}{void} \PYG{n+nf}{imprimirHijos2}\PYG{o}{(}\PYG{n}{Node} \PYG{n}{nodoRaiz}\PYG{o}{)}\PYG{o}{\PYGZob{}}
        \PYG{k}{if} \PYG{o}{(}\PYG{n}{nodoRaiz}\PYG{o}{=}\PYG{o}{=}\PYG{k+kc}{null}\PYG{o}{)}\PYG{o}{\PYGZob{}}
                \PYG{n}{System}\PYG{o}{.}\PYG{n+na}{out}\PYG{o}{.}\PYG{n+na}{println}\PYG{o}{(}\PYG{l+s}{\PYGZdq{}Imposible procesar raíz nula\PYGZdq{}}\PYG{o}{)}\PYG{o}{;}
                \PYG{k}{return}\PYG{o}{;}
        \PYG{o}{\PYGZcb{}} \PYG{c+c1}{//Fin del if null}
        \PYG{n}{NodeList} \PYG{n}{lista}\PYG{o}{=}\PYG{n}{nodoRaiz}\PYG{o}{.}\PYG{n+na}{getChildNodes}\PYG{o}{(}\PYG{o}{)}\PYG{o}{;}
        \PYG{k}{for} \PYG{o}{(}\PYG{k+kt}{int} \PYG{n}{i}\PYG{o}{=}\PYG{l+m+mi}{0}\PYG{o}{;} \PYG{n}{i}\PYG{o}{\PYGZlt{}}\PYG{n}{lista}\PYG{o}{.}\PYG{n+na}{getLength}\PYG{o}{(}\PYG{o}{)}\PYG{o}{;}\PYG{n}{i}\PYG{o}{+}\PYG{o}{+}\PYG{o}{)}\PYG{o}{\PYGZob{}}
                \PYG{n}{Node} \PYG{n}{nodo}\PYG{o}{=}\PYG{n}{lista}\PYG{o}{.}\PYG{n+na}{item}\PYG{o}{(}\PYG{n}{i}\PYG{o}{)}\PYG{o}{;}
                \PYG{k+kt}{short} \PYG{n}{tipo}\PYG{o}{=}\PYG{n}{nodo}\PYG{o}{.}\PYG{n+na}{getNodeType}\PYG{o}{(}\PYG{o}{)}\PYG{o}{;}
                \PYG{k}{if} \PYG{o}{(}\PYG{n}{tipo}\PYG{o}{=}\PYG{o}{=}\PYG{n}{Node}\PYG{o}{.}\PYG{n+na}{ELEMENT\PYGZus{}NODE}\PYG{o}{)}\PYG{o}{\PYGZob{}}
                        \PYG{n}{System}\PYG{o}{.}\PYG{n+na}{out}\PYG{o}{.}\PYG{n+na}{println}\PYG{o}{(}\PYG{l+s}{\PYGZdq{}Hijo:\PYGZdq{}}\PYG{o}{+}\PYG{n}{nodo}\PYG{o}{.}\PYG{n+na}{getNodeName}\PYG{o}{(}\PYG{o}{)}\PYG{o}{)}\PYG{o}{;}
                \PYG{o}{\PYGZcb{}} \PYG{c+c1}{//Fin del if}
        \PYG{o}{\PYGZcb{}} \PYG{c+c1}{//Fin del for}
\PYG{o}{\PYGZcb{}} \PYG{c+c1}{//Fin del método}
\end{sphinxVerbatim}


\subsection{Ejercicio: extracción de información financiera}
\label{\detokenize{tema6:ejercicio-extraccion-de-informacion-financiera}}
Dado el archivo XML siguiente…

\begin{sphinxVerbatim}[commandchars=\\\{\}]
\PYG{n+nt}{\PYGZlt{}listado}\PYG{n+nt}{\PYGZgt{}}
        \PYG{n+nt}{\PYGZlt{}futuro} \PYG{n+na}{precio=}\PYG{l+s}{\PYGZdq{}11.28\PYGZdq{}}\PYG{n+nt}{\PYGZgt{}}
                \PYG{n+nt}{\PYGZlt{}producto}\PYG{n+nt}{\PYGZgt{}}Cafe\PYG{n+nt}{\PYGZlt{}/producto\PYGZgt{}}
                \PYG{n+nt}{\PYGZlt{}mercado}\PYG{n+nt}{\PYGZgt{}}América Latina\PYG{n+nt}{\PYGZlt{}/mercado\PYGZgt{}}
                \PYG{n+nt}{\PYGZlt{}ciudad\PYGZus{}procedencia}\PYG{n+nt}{\PYGZgt{}}
                        \PYG{n+nt}{\PYGZlt{}frankfurt}\PYG{n+nt}{/\PYGZgt{}}
                \PYG{n+nt}{\PYGZlt{}/ciudad\PYGZus{}procedencia\PYGZgt{}}
        \PYG{n+nt}{\PYGZlt{}/futuro\PYGZgt{}}
        \PYG{n+nt}{\PYGZlt{}divisa} \PYG{n+na}{precio=}\PYG{l+s}{\PYGZdq{}183\PYGZdq{}}\PYG{n+nt}{\PYGZgt{}}
                \PYG{n+nt}{\PYGZlt{}nombre\PYGZus{}divisa}\PYG{n+nt}{\PYGZgt{}}Libra esterlina\PYG{n+nt}{\PYGZlt{}/nombre\PYGZus{}divisa\PYGZgt{}}
                \PYG{n+nt}{\PYGZlt{}tipo\PYGZus{}de\PYGZus{}cambio}\PYG{n+nt}{\PYGZgt{}}2.7:1 euros\PYG{n+nt}{\PYGZlt{}/tipo\PYGZus{}de\PYGZus{}cambio\PYGZgt{}}
                \PYG{n+nt}{\PYGZlt{}tipo\PYGZus{}de\PYGZus{}cambio}\PYG{n+nt}{\PYGZgt{}}1:0.87 dólares\PYG{n+nt}{\PYGZlt{}/tipo\PYGZus{}de\PYGZus{}cambio\PYGZgt{}}
                \PYG{n+nt}{\PYGZlt{}ciudad\PYGZus{}procedencia}\PYG{n+nt}{\PYGZgt{}}
                        \PYG{n+nt}{\PYGZlt{}madrid}\PYG{n+nt}{/\PYGZgt{}}
                \PYG{n+nt}{\PYGZlt{}/ciudad\PYGZus{}procedencia\PYGZgt{}}
        \PYG{n+nt}{\PYGZlt{}/divisa\PYGZgt{}}
        \PYG{n+nt}{\PYGZlt{}bono} \PYG{n+na}{precio=}\PYG{l+s}{\PYGZdq{}10000\PYGZdq{}} \PYG{n+na}{estable=}\PYG{l+s}{\PYGZdq{}si\PYGZdq{}}\PYG{n+nt}{\PYGZgt{}}
                \PYG{n+nt}{\PYGZlt{}pais\PYGZus{}de\PYGZus{}procedencia}\PYG{n+nt}{\PYGZgt{}}
                        Islandia
                \PYG{n+nt}{\PYGZlt{}/pais\PYGZus{}de\PYGZus{}procedencia\PYGZgt{}}
                \PYG{n+nt}{\PYGZlt{}valor\PYGZus{}deseado}\PYG{n+nt}{\PYGZgt{}}9980\PYG{n+nt}{\PYGZlt{}/valor\PYGZus{}deseado\PYGZgt{}}
                \PYG{n+nt}{\PYGZlt{}valor\PYGZus{}minimo}\PYG{n+nt}{\PYGZgt{}}9950\PYG{n+nt}{\PYGZlt{}/valor\PYGZus{}minimo\PYGZgt{}}
                \PYG{n+nt}{\PYGZlt{}valor\PYGZus{}maximo}\PYG{n+nt}{\PYGZgt{}}10020\PYG{n+nt}{\PYGZlt{}/valor\PYGZus{}maximo\PYGZgt{}}
                \PYG{n+nt}{\PYGZlt{}ciudad\PYGZus{}procedencia}\PYG{n+nt}{\PYGZgt{}}
                        \PYG{n+nt}{\PYGZlt{}tokio}\PYG{n+nt}{/\PYGZgt{}}
                \PYG{n+nt}{\PYGZlt{}/ciudad\PYGZus{}procedencia\PYGZgt{}}
        \PYG{n+nt}{\PYGZlt{}/bono\PYGZgt{}}
        \PYG{n+nt}{\PYGZlt{}letra} \PYG{n+na}{precio=}\PYG{l+s}{\PYGZdq{}45020\PYGZdq{}}\PYG{n+nt}{\PYGZgt{}}
                \PYG{n+nt}{\PYGZlt{}tipo\PYGZus{}de\PYGZus{}interes}\PYG{n+nt}{\PYGZgt{}}4.54\PYGZpc{}\PYG{n+nt}{\PYGZlt{}/tipo\PYGZus{}de\PYGZus{}interes\PYGZgt{}}
                \PYG{n+nt}{\PYGZlt{}pais\PYGZus{}emisor}\PYG{n+nt}{\PYGZgt{}}
                        \PYG{n+nt}{\PYGZlt{}espania}\PYG{n+nt}{/\PYGZgt{}}
                \PYG{n+nt}{\PYGZlt{}/pais\PYGZus{}emisor\PYGZgt{}}
                \PYG{n+nt}{\PYGZlt{}ciudad\PYGZus{}procedencia}\PYG{n+nt}{\PYGZgt{}}
                        \PYG{n+nt}{\PYGZlt{}madrid}\PYG{n+nt}{/\PYGZgt{}}
                \PYG{n+nt}{\PYGZlt{}/ciudad\PYGZus{}procedencia\PYGZgt{}}
        \PYG{n+nt}{\PYGZlt{}/letra\PYGZgt{}}
\PYG{n+nt}{\PYGZlt{}/listado\PYGZgt{}}
\end{sphinxVerbatim}

…crear un programa XML que:
\begin{itemize}
\item {} 
Busque todos los elementos cuya ciudad de procedencia sea «Madrid».

\item {} 
Si el elemento es un futuro mostrará el contenido de la etiqueta «producto».

\item {} 
Si el elemento es una divisa se mostrará el contenido de la etiqueta «nombre».

\item {} 
Si el elemento es una letra  se mostrará el contenido de la etiqueta «pais\_emisor».

\item {} 
Si el elemento es un bono  se mostrará el contenido de la etiqueta «pais\_de\_procedencia».

\end{itemize}

Una posibilidad (incompleta) sería esta:

\begin{sphinxVerbatim}[commandchars=\\\{\}]
 \PYG{k+kd}{public} \PYG{k+kd}{class} \PYG{n+nc}{ProcesadorFinanzas} \PYG{o}{\PYGZob{}}

   \PYG{k+kd}{public} \PYG{k+kd}{static} \PYG{n}{Element} \PYG{n+nf}{getRaiz} \PYG{o}{(}\PYG{n}{String} \PYG{n}{rutaFichero}\PYG{o}{)}\PYG{o}{\PYGZob{}}

     \PYG{n}{Element} \PYG{n}{raiz}\PYG{o}{=}\PYG{k+kc}{null}\PYG{o}{;}
     \PYG{n}{DocumentBuilderFactory} \PYG{n}{fabrica}\PYG{o}{;}
     \PYG{n}{fabrica}\PYG{o}{=}\PYG{n}{DocumentBuilderFactory}\PYG{o}{.}\PYG{n+na}{newInstance}\PYG{o}{(}\PYG{o}{)}\PYG{o}{;}
     \PYG{n}{DocumentBuilder} \PYG{n}{constructor}\PYG{o}{;}
     \PYG{k}{try} \PYG{o}{\PYGZob{}}
       \PYG{n}{constructor}\PYG{o}{=}\PYG{n}{fabrica}\PYG{o}{.}\PYG{n+na}{newDocumentBuilder}\PYG{o}{(}\PYG{o}{)}\PYG{o}{;}
       \PYG{n}{FileInputStream} \PYG{n}{fichero}\PYG{o}{;}
       \PYG{n}{fichero}\PYG{o}{=}\PYG{k}{new} \PYG{n}{FileInputStream}\PYG{o}{(}\PYG{n}{rutaFichero}\PYG{o}{)}\PYG{o}{;}
       \PYG{n}{Document} \PYG{n}{documento}\PYG{o}{;}
       \PYG{n}{documento}\PYG{o}{=}\PYG{n}{constructor}\PYG{o}{.}\PYG{n+na}{parse}\PYG{o}{(}\PYG{n}{fichero}\PYG{o}{)}\PYG{o}{;}
       \PYG{n}{raiz}\PYG{o}{=}\PYG{n}{documento}\PYG{o}{.}\PYG{n+na}{getDocumentElement}\PYG{o}{(}\PYG{o}{)}\PYG{o}{;}
     \PYG{o}{\PYGZcb{}} \PYG{k}{catch} \PYG{o}{(}\PYG{n}{ParserConfigurationException} \PYG{n}{e}\PYG{o}{)} \PYG{o}{\PYGZob{}}
       \PYG{n}{e}\PYG{o}{.}\PYG{n+na}{printStackTrace}\PYG{o}{(}\PYG{o}{)}\PYG{o}{;}
     \PYG{o}{\PYGZcb{}} \PYG{k}{catch} \PYG{o}{(}\PYG{n}{FileNotFoundException} \PYG{n}{e}\PYG{o}{)} \PYG{o}{\PYGZob{}}
       \PYG{n}{e}\PYG{o}{.}\PYG{n+na}{printStackTrace}\PYG{o}{(}\PYG{o}{)}\PYG{o}{;}
     \PYG{o}{\PYGZcb{}} \PYG{k}{catch} \PYG{o}{(}\PYG{n}{SAXException} \PYG{n}{e}\PYG{o}{)} \PYG{o}{\PYGZob{}}
       \PYG{c+c1}{// TODO Auto\PYGZhy{}generated catch block}
     \PYG{o}{\PYGZcb{}} \PYG{k}{catch} \PYG{o}{(}\PYG{n}{IOException} \PYG{n}{e}\PYG{o}{)} \PYG{o}{\PYGZob{}}
       \PYG{n}{e}\PYG{o}{.}\PYG{n+na}{printStackTrace}\PYG{o}{(}\PYG{o}{)}\PYG{o}{;}
     \PYG{o}{\PYGZcb{}}
     \PYG{k}{return} \PYG{n}{raiz}\PYG{o}{;}
   \PYG{o}{\PYGZcb{}}
   \PYG{k+kd}{public} \PYG{k+kd}{static} \PYG{k+kt}{void} \PYG{n+nf}{mostrarMadrid}\PYG{o}{(}\PYG{n}{Element} \PYG{n}{raiz}\PYG{o}{)}\PYG{o}{\PYGZob{}}
     \PYG{n}{NodeList} \PYG{n}{hijos}\PYG{o}{=}\PYG{n}{raiz}\PYG{o}{.}\PYG{n+na}{getChildNodes}\PYG{o}{(}\PYG{o}{)}\PYG{o}{;}
     \PYG{k}{for} \PYG{o}{(}\PYG{k+kt}{int} \PYG{n}{i}\PYG{o}{=}\PYG{l+m+mi}{0}\PYG{o}{;} \PYG{n}{i}\PYG{o}{\PYGZlt{}}\PYG{n}{hijos}\PYG{o}{.}\PYG{n+na}{getLength}\PYG{o}{(}\PYG{o}{)}\PYG{o}{;} \PYG{n}{i}\PYG{o}{+}\PYG{o}{+}\PYG{o}{)}\PYG{o}{\PYGZob{}}
       \PYG{n}{Node} \PYG{n}{nodoVisitado}\PYG{o}{=}\PYG{n}{hijos}\PYG{o}{.}\PYG{n+na}{item}\PYG{o}{(}\PYG{n}{i}\PYG{o}{)}\PYG{o}{;}
       \PYG{k}{if} \PYG{o}{(}\PYG{n}{nodoVisitado}\PYG{o}{.}\PYG{n+na}{getNodeType}\PYG{o}{(}\PYG{o}{)}
           \PYG{o}{=}\PYG{o}{=} \PYG{n}{Node}\PYG{o}{.}\PYG{n+na}{ELEMENT\PYGZus{}NODE}\PYG{o}{)} \PYG{o}{\PYGZob{}}
         \PYG{c+c1}{//El nodo sí es un elemento y no}
         \PYG{c+c1}{//un nodo con texto \PYGZbs{}n}
         \PYG{n}{Element} \PYG{n}{elemVisitado}\PYG{o}{;}
         \PYG{n}{elemVisitado}\PYG{o}{=}\PYG{o}{(}\PYG{n}{Element}\PYG{o}{)} \PYG{n}{nodoVisitado}\PYG{o}{;}
         \PYG{k}{if} \PYG{o}{(}\PYG{n}{elemVisitado}\PYG{o}{.}\PYG{n+na}{getTagName}\PYG{o}{(}\PYG{o}{)}\PYG{o}{.}\PYG{n+na}{equals}\PYG{o}{(}\PYG{l+s}{\PYGZdq{}futuro\PYGZdq{}}\PYG{o}{)}\PYG{o}{)}\PYG{o}{\PYGZob{}}
           \PYG{n}{procesarFuturo}\PYG{o}{(}\PYG{n}{elemVisitado}\PYG{o}{)}\PYG{o}{;}
         \PYG{o}{\PYGZcb{}}
         \PYG{k}{if} \PYG{o}{(}\PYG{n}{elemVisitado}\PYG{o}{.}\PYG{n+na}{getTagName}\PYG{o}{(}\PYG{o}{)}\PYG{o}{.}\PYG{n+na}{equals}\PYG{o}{(}\PYG{l+s}{\PYGZdq{}bono\PYGZdq{}}\PYG{o}{)}\PYG{o}{)}\PYG{o}{\PYGZob{}}
           \PYG{n}{procesarBono}\PYG{o}{(}\PYG{n}{elemVisitado}\PYG{o}{)}\PYG{o}{;}
         \PYG{o}{\PYGZcb{}}
         \PYG{k}{if} \PYG{o}{(}\PYG{n}{elemVisitado}\PYG{o}{.}\PYG{n+na}{getTagName}\PYG{o}{(}\PYG{o}{)}\PYG{o}{.}\PYG{n+na}{equals}\PYG{o}{(}\PYG{l+s}{\PYGZdq{}letra\PYGZdq{}}\PYG{o}{)}\PYG{o}{)}\PYG{o}{\PYGZob{}}
           \PYG{n}{procesarLetra}\PYG{o}{(}\PYG{n}{elemVisitado}\PYG{o}{)}\PYG{o}{;}
         \PYG{o}{\PYGZcb{}}
         \PYG{k}{if} \PYG{o}{(}\PYG{n}{elemVisitado}\PYG{o}{.}\PYG{n+na}{getTagName}\PYG{o}{(}\PYG{o}{)}\PYG{o}{.}\PYG{n+na}{equals}\PYG{o}{(}\PYG{l+s}{\PYGZdq{}divisa\PYGZdq{}}\PYG{o}{)}\PYG{o}{)}\PYG{o}{\PYGZob{}}
           \PYG{n}{procesarDivisa}\PYG{o}{(}\PYG{n}{elemVisitado}\PYG{o}{)}\PYG{o}{;}
         \PYG{o}{\PYGZcb{}}
       \PYG{o}{\PYGZcb{}}
     \PYG{o}{\PYGZcb{}}
   \PYG{o}{\PYGZcb{}}
   \PYG{k+kd}{private} \PYG{k+kd}{static} \PYG{k+kt}{void} \PYG{n+nf}{procesarDivisa}\PYG{o}{(}\PYG{n}{Element} \PYG{n}{elemVisitado}\PYG{o}{)} \PYG{o}{\PYGZob{}}
     \PYG{c+c1}{// TODO Auto\PYGZhy{}generated method stub}

   \PYG{o}{\PYGZcb{}}
   \PYG{k+kd}{private} \PYG{k+kd}{static} \PYG{k+kt}{void} \PYG{n+nf}{procesarLetra}\PYG{o}{(}\PYG{n}{Element} \PYG{n}{elemVisitado}\PYG{o}{)} \PYG{o}{\PYGZob{}}
     \PYG{c+c1}{// TODO Auto\PYGZhy{}generated method stub}

   \PYG{o}{\PYGZcb{}}
   \PYG{k+kd}{private} \PYG{k+kd}{static} \PYG{k+kt}{void} \PYG{n+nf}{procesarBono}\PYG{o}{(}\PYG{n}{Element} \PYG{n}{elemVisitado}\PYG{o}{)} \PYG{o}{\PYGZob{}}
     \PYG{c+c1}{// TODO Auto\PYGZhy{}generated method stub}

   \PYG{o}{\PYGZcb{}}
   \PYG{k+kd}{private} \PYG{k+kd}{static} \PYG{k+kt}{void} \PYG{n+nf}{procesarFuturo}\PYG{o}{(}\PYG{n}{Element} \PYG{n}{elemVisitado}\PYG{o}{)} \PYG{o}{\PYGZob{}}
     \PYG{n}{NodeList} \PYG{n}{hijosCiudad}\PYG{o}{;}
     \PYG{n}{hijosCiudad}\PYG{o}{=}\PYG{n}{elemVisitado}\PYG{o}{.}\PYG{n+na}{getElementsByTagName}\PYG{o}{(}
         \PYG{l+s}{\PYGZdq{}ciudad\PYGZus{}procedencia\PYGZdq{}}\PYG{o}{)}\PYG{o}{;}
     \PYG{n}{Node} \PYG{n}{unicaCiudad}\PYG{o}{=}\PYG{n}{hijosCiudad}\PYG{o}{.}\PYG{n+na}{item}\PYG{o}{(}\PYG{l+m+mi}{0}\PYG{o}{)}\PYG{o}{;}
     \PYG{n}{NodeList} \PYG{n}{ciudades}\PYG{o}{=}\PYG{n}{unicaCiudad}\PYG{o}{.}\PYG{n+na}{getChildNodes}\PYG{o}{(}\PYG{o}{)}\PYG{o}{;}
     \PYG{n}{Node} \PYG{n}{ciudadProcedencia}\PYG{o}{=}\PYG{n}{ciudades}\PYG{o}{.}\PYG{n+na}{item}\PYG{o}{(}\PYG{l+m+mi}{1}\PYG{o}{)}\PYG{o}{;}
     \PYG{n}{Element} \PYG{n}{elemCiudad}\PYG{o}{=}\PYG{o}{(}\PYG{n}{Element}\PYG{o}{)} \PYG{n}{ciudadProcedencia}\PYG{o}{;}
     \PYG{k}{if} \PYG{o}{(}\PYG{n}{elemCiudad}\PYG{o}{.}\PYG{n+na}{getTagName}\PYG{o}{(}\PYG{o}{)}\PYG{o}{.}\PYG{n+na}{equals}\PYG{o}{(}\PYG{l+s}{\PYGZdq{}madrid\PYGZdq{}}\PYG{o}{)}\PYG{o}{)}\PYG{o}{\PYGZob{}}
       \PYG{n}{NodeList} \PYG{n}{productos}\PYG{o}{;}
       \PYG{n}{productos}\PYG{o}{=}\PYG{n}{elemVisitado}\PYG{o}{.}\PYG{n+na}{getElementsByTagName}\PYG{o}{(}
           \PYG{l+s}{\PYGZdq{}producto\PYGZdq{}}\PYG{o}{)}\PYG{o}{;}
       \PYG{n}{Element} \PYG{n}{elemProducto}\PYG{o}{;}
       \PYG{n}{elemProducto}\PYG{o}{=}\PYG{o}{(}\PYG{n}{Element}\PYG{o}{)} \PYG{n}{productos}\PYG{o}{.}\PYG{n+na}{item}\PYG{o}{(}\PYG{l+m+mi}{0}\PYG{o}{)}\PYG{o}{;}
       \PYG{n}{System}\PYG{o}{.}\PYG{n+na}{out}\PYG{o}{.}\PYG{n+na}{println}\PYG{o}{(}
           \PYG{n}{elemProducto}\PYG{o}{.}\PYG{n+na}{getTextContent}\PYG{o}{(}\PYG{o}{)} \PYG{o}{)}\PYG{o}{;}

     \PYG{o}{\PYGZcb{}}
   \PYG{o}{\PYGZcb{}}
   \PYG{k+kd}{public} \PYG{k+kd}{static} \PYG{k+kt}{void} \PYG{n+nf}{main}\PYG{o}{(}\PYG{n}{String}\PYG{o}{[}\PYG{o}{]} \PYG{n}{args}\PYG{o}{)} \PYG{o}{\PYGZob{}}
     \PYG{n}{Element} \PYG{n}{raiz}\PYG{o}{=}\PYG{n}{getRaiz}\PYG{o}{(}
       \PYG{l+s}{\PYGZdq{}c:/users/ogomez/documents/finanzas.xml\PYGZdq{}}\PYG{o}{)}\PYG{o}{;}
     \PYG{n}{mostrarMadrid}\PYG{o}{(}\PYG{n}{raiz}\PYG{o}{)}\PYG{o}{;}
   \PYG{o}{\PYGZcb{}}
\PYG{o}{\PYGZcb{}}
\end{sphinxVerbatim}


\subsection{Ejercicio}
\label{\detokenize{tema6:id3}}
Ampliar el programa para que nos diga cuantos elementos «divisa» hay en el archivo.

Para practicar esto y de paso practicar programación genérica, fabricaremos un método al que le pasaremos el nombre del elemento a buscar y el método nos dirá cuantos elementos con ese nombre hay.

\begin{sphinxVerbatim}[commandchars=\\\{\}]
\PYG{k+kd}{public} \PYG{k+kt}{int} \PYG{n+nf}{contadorElementos}\PYG{o}{(}\PYG{n}{Node} \PYG{n}{raiz}\PYG{o}{,}\PYG{n}{String} \PYG{n}{nombreElemento}\PYG{o}{)}\PYG{o}{\PYGZob{}}
        \PYG{k+kt}{int} \PYG{n}{contador}\PYG{o}{=}\PYG{l+m+mi}{0}\PYG{o}{;}
        \PYG{n}{NodeList} \PYG{n}{nodosHijo}\PYG{o}{=}\PYG{n}{raiz}\PYG{o}{.}\PYG{n+na}{getChildNodes}\PYG{o}{(}\PYG{o}{)}\PYG{o}{;}
        \PYG{k}{for} \PYG{o}{(}\PYG{k+kt}{int} \PYG{n}{i}\PYG{o}{=}\PYG{l+m+mi}{0}\PYG{o}{;} \PYG{n}{i}\PYG{o}{\PYGZlt{}}\PYG{n}{nodosHijo}\PYG{o}{.}\PYG{n+na}{getLength}\PYG{o}{(}\PYG{o}{)}\PYG{o}{;}\PYG{n}{i}\PYG{o}{+}\PYG{o}{+}\PYG{o}{)}\PYG{o}{\PYGZob{}}
                \PYG{n}{Node} \PYG{n}{nodo}\PYG{o}{=}\PYG{n}{nodosHijo}\PYG{o}{.}\PYG{n+na}{item}\PYG{o}{(}\PYG{n}{i}\PYG{o}{)}\PYG{o}{;}
                \PYG{k+kt}{short} \PYG{n}{tipo}\PYG{o}{=}\PYG{n}{nodo}\PYG{o}{.}\PYG{n+na}{getNodeType}\PYG{o}{(}\PYG{o}{)}\PYG{o}{;}
                \PYG{k}{if} \PYG{o}{(}\PYG{n}{tipo}\PYG{o}{=}\PYG{o}{=}\PYG{n}{Node}\PYG{o}{.}\PYG{n+na}{ELEMENT\PYGZus{}NODE}\PYG{o}{)}\PYG{o}{\PYGZob{}}
                        \PYG{n}{String} \PYG{n}{nombre}\PYG{o}{=}\PYG{n}{nodo}\PYG{o}{.}\PYG{n+na}{getNodeName}\PYG{o}{(}\PYG{o}{)}\PYG{o}{;}
                        \PYG{k}{if} \PYG{o}{(}\PYG{n}{nombre}\PYG{o}{=}\PYG{o}{=}\PYG{n}{nombreElemento}\PYG{o}{)}\PYG{o}{\PYGZob{}}
                                \PYG{n}{contador}\PYG{o}{+}\PYG{o}{+}\PYG{o}{;}
                        \PYG{o}{\PYGZcb{}} \PYG{c+c1}{//Fin del if interno}
                \PYG{o}{\PYGZcb{}} \PYG{c+c1}{//Fin del if externo}
        \PYG{o}{\PYGZcb{}} \PYG{c+c1}{//Fin del for}
        \PYG{k}{return} \PYG{n}{contador}\PYG{o}{;}
\PYG{o}{\PYGZcb{}} \PYG{c+c1}{//Fin del método}
\end{sphinxVerbatim}


\subsection{Ejercicio}
\label{\detokenize{tema6:id4}}
Nos interesa conocer el precio de todos los bonos. Crear un programa que ejecute esta tarea.

\begin{sphinxVerbatim}[commandchars=\\\{\}]
\PYG{k+kd}{private} \PYG{k+kt}{void} \PYG{n+nf}{comprobarSiEsBono}\PYG{o}{(}\PYG{n}{Node} \PYG{n}{n}\PYG{o}{)}\PYG{o}{\PYGZob{}}
        \PYG{n}{String} \PYG{n}{nombre}\PYG{o}{=}\PYG{n}{n}\PYG{o}{.}\PYG{n+na}{getNodeName}\PYG{o}{(}\PYG{o}{)}\PYG{o}{;}
        \PYG{k}{if} \PYG{o}{(}\PYG{n}{nombre}\PYG{o}{=}\PYG{o}{=}\PYG{l+s}{\PYGZdq{}bono\PYGZdq{}}\PYG{o}{)}\PYG{o}{\PYGZob{}}
                \PYG{n}{System}\PYG{o}{.}\PYG{n+na}{out}\PYG{o}{.}\PYG{n+na}{println}\PYG{o}{(}\PYG{l+s}{\PYGZdq{}Encontrado un bono\PYGZdq{}}\PYG{o}{)}\PYG{o}{;}
        \PYG{o}{\PYGZcb{}}
\PYG{o}{\PYGZcb{}}
\PYG{k+kd}{public} \PYG{k+kt}{void} \PYG{n+nf}{imprimirPrecioBonos}\PYG{o}{(}\PYG{n}{Node} \PYG{n}{raiz}\PYG{o}{)}\PYG{o}{\PYGZob{}}
        \PYG{k}{if} \PYG{o}{(}\PYG{n}{raiz}\PYG{o}{=}\PYG{o}{=}\PYG{k+kc}{null}\PYG{o}{)}\PYG{o}{\PYGZob{}}
                \PYG{n}{System}\PYG{o}{.}\PYG{n+na}{out}\PYG{o}{.}\PYG{n+na}{println}\PYG{o}{(}\PYG{l+s}{\PYGZdq{}Imposible procesar null\PYGZdq{}}\PYG{o}{)}\PYG{o}{;}
                \PYG{k}{return}\PYG{o}{;}
        \PYG{o}{\PYGZcb{}}
        \PYG{n}{NodeList} \PYG{n}{nodos}\PYG{o}{=}\PYG{n}{raiz}\PYG{o}{.}\PYG{n+na}{getChildNodes}\PYG{o}{(}\PYG{o}{)}\PYG{o}{;}
        \PYG{k}{for} \PYG{o}{(}\PYG{k+kt}{int} \PYG{n}{i}\PYG{o}{=}\PYG{l+m+mi}{0}\PYG{o}{;} \PYG{n}{i}\PYG{o}{\PYGZlt{}}\PYG{n}{nodos}\PYG{o}{.}\PYG{n+na}{getLength}\PYG{o}{(}\PYG{o}{)}\PYG{o}{;} \PYG{n}{i}\PYG{o}{+}\PYG{o}{+}\PYG{o}{)}\PYG{o}{\PYGZob{}}
                \PYG{n}{Node} \PYG{n}{nodo}\PYG{o}{=}\PYG{n}{nodos}\PYG{o}{.}\PYG{n+na}{item}\PYG{o}{(}\PYG{n}{i}\PYG{o}{)}\PYG{o}{;}
                \PYG{k+kt}{short} \PYG{n}{tipo}\PYG{o}{=}\PYG{n}{nodo}\PYG{o}{.}\PYG{n+na}{getNodeType}\PYG{o}{(}\PYG{o}{)}\PYG{o}{;}
                \PYG{k}{if} \PYG{o}{(}\PYG{n}{tipo}\PYG{o}{=}\PYG{o}{=}\PYG{n}{Node}\PYG{o}{.}\PYG{n+na}{ELEMENT\PYGZus{}NODE}\PYG{o}{)}\PYG{o}{\PYGZob{}}
                        \PYG{k}{this}\PYG{o}{.}\PYG{n+na}{comprobarSiEsBono}\PYG{o}{(}\PYG{n}{nodo}\PYG{o}{)}\PYG{o}{;}
                \PYG{o}{\PYGZcb{}}
        \PYG{o}{\PYGZcb{}}
\PYG{o}{\PYGZcb{}}
\end{sphinxVerbatim}


\subsection{Ejercicio}
\label{\detokenize{tema6:id5}}
Crear un programa que nos diga cuantos productos financieros del listado no son estables. Es decir, que tengan el atributo estable y lo tengan a \sphinxcode{false}.

En su momento, en la DTD se permitió que el atributo \sphinxcode{estable} fuera \sphinxcode{\#IMPLIED}, es decir \sphinxstylestrong{optativo}. Al ser la DTD como un contrato, esto nos obliga a preparar nuestro código para manejar la posibilidad de que el atributo no esté presente.

\begin{sphinxVerbatim}[commandchars=\\\{\}]
\PYG{k+kd}{public} \PYG{k+kt}{int} \PYG{n+nf}{cuantosInestables} \PYG{o}{(}\PYG{n}{Node} \PYG{n}{raiz}\PYG{o}{)}\PYG{o}{\PYGZob{}}
        \PYG{k+kt}{int} \PYG{n}{cuantos}\PYG{o}{=}\PYG{l+m+mi}{0}\PYG{o}{;}
        \PYG{n}{NodeList} \PYG{n}{lista}\PYG{o}{=}\PYG{n}{raiz}\PYG{o}{.}\PYG{n+na}{getChildNodes}\PYG{o}{(}\PYG{o}{)}\PYG{o}{;}
        \PYG{k}{for} \PYG{o}{(}\PYG{k+kt}{int} \PYG{n}{i}\PYG{o}{=}\PYG{l+m+mi}{0}\PYG{o}{;} \PYG{n}{i}\PYG{o}{\PYGZlt{}}\PYG{n}{lista}\PYG{o}{.}\PYG{n+na}{getLength}\PYG{o}{(}\PYG{o}{)}\PYG{o}{;} \PYG{n}{i}\PYG{o}{+}\PYG{o}{+}\PYG{o}{)}\PYG{o}{\PYGZob{}}
                \PYG{n}{Node} \PYG{n}{n}\PYG{o}{=}\PYG{n}{lista}\PYG{o}{.}\PYG{n+na}{item}\PYG{o}{(}\PYG{n}{i}\PYG{o}{)}\PYG{o}{;}
                \PYG{k}{if} \PYG{o}{(}\PYG{n}{n}\PYG{o}{.}\PYG{n+na}{getNodeType}\PYG{o}{(}\PYG{o}{)}\PYG{o}{!}\PYG{o}{=}\PYG{n}{Node}\PYG{o}{.}\PYG{n+na}{ELEMENT\PYGZus{}NODE}\PYG{o}{)} \PYG{k}{continue}\PYG{o}{;}
                \PYG{n}{Element} \PYG{n}{e}\PYG{o}{=}\PYG{o}{(}\PYG{n}{Element}\PYG{o}{)} \PYG{n}{lista}\PYG{o}{.}\PYG{n+na}{item}\PYG{o}{(}\PYG{n}{i}\PYG{o}{)}\PYG{o}{;}
                \PYG{k}{if} \PYG{o}{(}\PYG{n}{e}\PYG{o}{.}\PYG{n+na}{getNodeName}\PYG{o}{(}\PYG{o}{)}\PYG{o}{=}\PYG{o}{=}\PYG{l+s}{\PYGZdq{}divisa\PYGZdq{}} \PYG{o}{\textbar{}}\PYG{o}{\textbar{}}
                                \PYG{n}{e}\PYG{o}{.}\PYG{n+na}{getNodeName}\PYG{o}{(}\PYG{o}{)}\PYG{o}{=}\PYG{o}{=}\PYG{l+s}{\PYGZdq{}bono\PYGZdq{}}\PYG{o}{)}\PYG{o}{\PYGZob{}}
                        \PYG{n}{String} \PYG{n}{atEstable}\PYG{o}{=}\PYG{n}{e}\PYG{o}{.}\PYG{n+na}{getAttribute}\PYG{o}{(}\PYG{l+s}{\PYGZdq{}estable\PYGZdq{}}\PYG{o}{)}\PYG{o}{;}
                        \PYG{k}{if} \PYG{o}{(}\PYG{n}{atEstable}\PYG{o}{!}\PYG{o}{=}\PYG{k+kc}{null}\PYG{o}{)}\PYG{o}{\PYGZob{}}
                                \PYG{n}{System}\PYG{o}{.}\PYG{n+na}{out}\PYG{o}{.}\PYG{n+na}{println}\PYG{o}{(}\PYG{l+s}{\PYGZdq{}Atributo:\PYGZdq{}}\PYG{o}{+}\PYG{n}{atEstable}\PYG{o}{)}\PYG{o}{;}
                                \PYG{k}{if} \PYG{o}{(}\PYG{n}{atEstable}\PYG{o}{.}\PYG{n+na}{equals}\PYG{o}{(}\PYG{l+s}{\PYGZdq{}no\PYGZdq{}}\PYG{o}{)}\PYG{o}{)}\PYG{o}{\PYGZob{}}
                                        \PYG{n}{cuantos}\PYG{o}{+}\PYG{o}{=}\PYG{l+m+mi}{1}\PYG{o}{;}
                                \PYG{o}{\PYGZcb{}} \PYG{c+c1}{//Fin del if interno}
                        \PYG{o}{\PYGZcb{}} \PYG{c+c1}{//Fin del if atEstable}
                \PYG{o}{\PYGZcb{}} \PYG{c+c1}{//Fin de if nodo es divisa o bono}
        \PYG{o}{\PYGZcb{}} \PYG{c+c1}{//Fin del for que recorre los nodos}
        \PYG{k}{return} \PYG{n}{cuantos}\PYG{o}{;}
\PYG{o}{\PYGZcb{}} \PYG{c+c1}{//Fin del método cuantosInestables}
\end{sphinxVerbatim}


\subsection{Ejercicio}
\label{\detokenize{tema6:id6}}
Sumar los precios de todos los productos financieros.

\begin{sphinxVerbatim}[commandchars=\\\{\}]
\PYG{k+kd}{public} \PYG{k+kt}{float} \PYG{n+nf}{sumarAtributosPrecio}\PYG{o}{(}\PYG{n}{Node} \PYG{n}{raiz}\PYG{o}{)}\PYG{o}{\PYGZob{}}
        \PYG{k+kt}{float} \PYG{n}{precioTotal}\PYG{o}{=}\PYG{l+m+mi}{0}\PYG{o}{;}
        \PYG{n}{NodeList} \PYG{n}{hijos}\PYG{o}{=}\PYG{n}{raiz}\PYG{o}{.}\PYG{n+na}{getChildNodes}\PYG{o}{(}\PYG{o}{)}\PYG{o}{;}
        \PYG{k}{for} \PYG{o}{(}\PYG{k+kt}{int} \PYG{n}{i}\PYG{o}{=}\PYG{l+m+mi}{0}\PYG{o}{;} \PYG{n}{i}\PYG{o}{\PYGZlt{}}\PYG{n}{hijos}\PYG{o}{.}\PYG{n+na}{getLength}\PYG{o}{(}\PYG{o}{)}\PYG{o}{;} \PYG{n}{i}\PYG{o}{+}\PYG{o}{+}\PYG{o}{)}\PYG{o}{\PYGZob{}}
                \PYG{n}{Node} \PYG{n}{hijo}\PYG{o}{=}\PYG{n}{hijos}\PYG{o}{.}\PYG{n+na}{item}\PYG{o}{(}\PYG{n}{i}\PYG{o}{)}\PYG{o}{;}
                \PYG{k}{if} \PYG{o}{(}\PYG{n}{hijo}\PYG{o}{.}\PYG{n+na}{getNodeType}\PYG{o}{(}\PYG{o}{)}\PYG{o}{!}\PYG{o}{=}\PYG{n}{Node}\PYG{o}{.}\PYG{n+na}{ELEMENT\PYGZus{}NODE}\PYG{o}{)} \PYG{k}{continue}\PYG{o}{;}
                \PYG{n}{Element} \PYG{n}{e}\PYG{o}{=}\PYG{o}{(}\PYG{n}{Element}\PYG{o}{)} \PYG{n}{hijo}\PYG{o}{;}
                \PYG{n}{String} \PYG{n}{precio}\PYG{o}{=}\PYG{n}{e}\PYG{o}{.}\PYG{n+na}{getAttribute}\PYG{o}{(}\PYG{l+s}{\PYGZdq{}precio\PYGZdq{}}\PYG{o}{)}\PYG{o}{;}
                \PYG{n}{Float} \PYG{n}{f}\PYG{o}{=}\PYG{n}{Float}\PYG{o}{.}\PYG{n+na}{parseFloat}\PYG{o}{(}\PYG{n}{precio}\PYG{o}{)}\PYG{o}{;}
                \PYG{n}{precioTotal}\PYG{o}{+}\PYG{o}{=}\PYG{n}{f}\PYG{o}{;}
        \PYG{o}{\PYGZcb{}}
        \PYG{k}{return} \PYG{n}{precioTotal}\PYG{o}{;}
\PYG{o}{\PYGZcb{}} \PYG{c+c1}{//Fin del método sumarAtributosPrecio}
\end{sphinxVerbatim}


\subsection{Ejercicio}
\label{\detokenize{tema6:id7}}
Contar cuantos productos financieros tienen algo que ver con el país «Islandia»

Se deben tener presentes varias cosas:
\begin{itemize}
\item {} 
Si no se tiene claro lo que nos piden, preguntar.

\item {} 
En cualquier caso, si se tiene DTD, hay una buena pista.
\begin{itemize}
\item {} 
Aparece un elemento llamado \sphinxcode{\textless{}pais\_de\_procedencia\textgreater{}}, que puede contener cualquier cosa (incluido Islandia)

\item {} 
La ciudad de procedencia no incluye la capital o ninguna ciudad de dicho país, así que podemos ignorar eso.

\item {} 
También aparece un elemento llamado \sphinxcode{\textless{}pais\_emisor\textgreater{}}, pero tampoco incluye Islandia, en principio también podemos saltarlo.

\end{itemize}

\end{itemize}


\subsubsection{Análisis del problema}
\label{\detokenize{tema6:analisis-del-problema}}
Despues de haber examinado la DTD se llega a la conclusión de que el único elemento que puede transportar alguna clase de información relacionada con «Islandia» es el nodo \sphinxcode{país\_de\_procedencia}, que es un elemento hijo del elemento \sphinxcode{bono}.


\subsubsection{Solución}
\label{\detokenize{tema6:solucion}}\begin{itemize}
\item {} 
La clase \sphinxcode{Element} tiene un método llamado \sphinxcode{getElementsByTagName} que nos permite recuperar de una sola vez todos los elementos con el nombre \sphinxcode{bono}.

\item {} 
Se debe tener en cuenta que para llegar al elemento que nos interesa podemos seguir usando los métodos \sphinxcode{getFirstChild} o \sphinxcode{getNextSibling} para ir al primer hijo o para ir al siguiente hermano.

\item {} 
El contenido textual de un nodo se puede extraer con \sphinxcode{getTextContent}

\item {} 
Al procesar un contenido textual, podríamos encontrar muchos espacios en blanco, tabuladores u otros elementos que alteren las comparaciones entre cadenas, por lo que deberemos usar métodos como \sphinxcode{trim()} que limpian los espacios en blanco.

\end{itemize}

\begin{sphinxVerbatim}[commandchars=\\\{\}]
\PYG{k+kd}{public} \PYG{k+kt}{int} \PYG{n+nf}{algoQueVerCon}\PYG{o}{(}\PYG{n}{Node} \PYG{n}{raiz}\PYG{o}{,} \PYG{n}{String} \PYG{n}{nombrePais}\PYG{o}{)}\PYG{o}{\PYGZob{}}
        \PYG{k+kt}{int} \PYG{n}{cuantos}\PYG{o}{=}\PYG{l+m+mi}{0}\PYG{o}{;}
        \PYG{n}{Element} \PYG{n}{elementoRaiz}\PYG{o}{=}\PYG{o}{(}\PYG{n}{Element}\PYG{o}{)} \PYG{n}{raiz}\PYG{o}{;}
        \PYG{n}{NodeList} \PYG{n}{lista}\PYG{o}{=}\PYG{n}{elementoRaiz}\PYG{o}{.}\PYG{n+na}{getElementsByTagName}\PYG{o}{(}\PYG{l+s}{\PYGZdq{}bono\PYGZdq{}}\PYG{o}{)}\PYG{o}{;}
        \PYG{k}{for} \PYG{o}{(}\PYG{k+kt}{int} \PYG{n}{i}\PYG{o}{=}\PYG{l+m+mi}{0}\PYG{o}{;} \PYG{n}{i}\PYG{o}{\PYGZlt{}}\PYG{n}{lista}\PYG{o}{.}\PYG{n+na}{getLength}\PYG{o}{(}\PYG{o}{)}\PYG{o}{;} \PYG{n}{i}\PYG{o}{+}\PYG{o}{+}\PYG{o}{)}\PYG{o}{\PYGZob{}}
                \PYG{n}{Node} \PYG{n}{nodoBono}\PYG{o}{=}\PYG{n}{lista}\PYG{o}{.}\PYG{n+na}{item}\PYG{o}{(}\PYG{n}{i}\PYG{o}{)}\PYG{o}{;}
                \PYG{n}{Node} \PYG{n}{primerHijoTexto}\PYG{o}{=}\PYG{n}{nodoBono}\PYG{o}{.}\PYG{n+na}{getFirstChild}\PYG{o}{(}\PYG{o}{)}\PYG{o}{;}
                \PYG{n}{Node} \PYG{n}{segHijoPais}\PYG{o}{=}\PYG{n}{primerHijoTexto}\PYG{o}{.}\PYG{n+na}{getNextSibling}\PYG{o}{(}\PYG{o}{)}\PYG{o}{;}
                \PYG{n}{String} \PYG{n}{paisExtraido}\PYG{o}{=}\PYG{n}{segHijoPais}\PYG{o}{.}\PYG{n+na}{getTextContent}\PYG{o}{(}\PYG{o}{)}\PYG{o}{;}
                \PYG{c+c1}{//Limpiamos espacios}
                \PYG{n}{paisExtraido}\PYG{o}{=}\PYG{n}{paisExtraido}\PYG{o}{.}\PYG{n+na}{trim}\PYG{o}{(}\PYG{o}{)}\PYG{o}{;}
                \PYG{n}{System}\PYG{o}{.}\PYG{n+na}{out}\PYG{o}{.}\PYG{n+na}{println}\PYG{o}{(}\PYG{l+s}{\PYGZdq{}Pais extraido:\PYGZdq{}}\PYG{o}{+}\PYG{n}{paisExtraido}\PYG{o}{)}\PYG{o}{;}
                \PYG{k}{if} \PYG{o}{(}\PYG{n}{paisExtraido}\PYG{o}{.}\PYG{n+na}{equals}\PYG{o}{(}\PYG{n}{nombrePais}\PYG{o}{)}\PYG{o}{)}\PYG{o}{\PYGZob{}}
                        \PYG{n}{cuantos}\PYG{o}{+}\PYG{o}{+}\PYG{o}{;}
                \PYG{o}{\PYGZcb{}}
        \PYG{o}{\PYGZcb{}}
        \PYG{k}{return} \PYG{n}{cuantos}\PYG{o}{;}
\PYG{o}{\PYGZcb{}}
\end{sphinxVerbatim}


\subsection{Ejercicio}
\label{\detokenize{tema6:id8}}
Se desea crear un método que indique cuantos elementos tienen relación de alguna forma con «España».


\subsubsection{Análisis}
\label{\detokenize{tema6:analisis}}\begin{itemize}
\item {} 
Se dispone del método anterior \sphinxcode{algoQueVerCon} que nos permite contabilizar cuantos bonos tienen el país «España».

\item {} 
Al analizar la DTD, se ha encontrado que la ciudad de procedencia de un elemento \sphinxcode{futuro} puede ser Madrid.

\item {} 
Al analizar la DTD también se ha encontrado que elemento \sphinxcode{pais\_emisor} de un elemento \sphinxcode{letra} puede ser \sphinxcode{espania}

\item {} 
Al analizar las divisas se debe comprobar si el elemento \sphinxcode{ciudad\_procedencia} es el elemento \sphinxcode{madrid}

\end{itemize}


\subsubsection{Diseño}
\label{\detokenize{tema6:diseno}}
Crearemos dos métodos extra, uno para calcular la solución para el segundo punto (ver cuantos elementos \sphinxcode{futuro} tienen como \sphinxcode{ciudad\_procedencia} el valor Madrid y otro método para el tercer punto.

\begin{sphinxVerbatim}[commandchars=\\\{\}]
\PYG{k+kd}{public} \PYG{k+kt}{int} \PYG{n+nf}{cuantosFuturosTienenCiudadProcedencia}\PYG{o}{(}
                \PYG{n}{Node} \PYG{n}{raiz}\PYG{o}{,} \PYG{n}{String} \PYG{n}{ciudad}\PYG{o}{)}
\PYG{o}{\PYGZob{}}
        \PYG{k+kt}{int} \PYG{n}{cuantos}\PYG{o}{=}\PYG{l+m+mi}{0}\PYG{o}{;}
        \PYG{n}{Element} \PYG{n}{nodoRaiz}\PYG{o}{=}\PYG{o}{(}\PYG{n}{Element}\PYG{o}{)} \PYG{n}{raiz}\PYG{o}{;}
        \PYG{n}{NodeList} \PYG{n}{lista}\PYG{o}{=}\PYG{n}{nodoRaiz}\PYG{o}{.}\PYG{n+na}{getElementsByTagName}\PYG{o}{(}\PYG{l+s}{\PYGZdq{}futuro\PYGZdq{}}\PYG{o}{)}\PYG{o}{;}
        \PYG{k}{for} \PYG{o}{(}\PYG{k+kt}{int} \PYG{n}{i}\PYG{o}{=}\PYG{l+m+mi}{0}\PYG{o}{;} \PYG{n}{i}\PYG{o}{\PYGZlt{}}\PYG{n}{lista}\PYG{o}{.}\PYG{n+na}{getLength}\PYG{o}{(}\PYG{o}{)}\PYG{o}{;} \PYG{n}{i}\PYG{o}{+}\PYG{o}{+}\PYG{o}{)}\PYG{o}{\PYGZob{}}
                \PYG{n}{Element} \PYG{n}{e}\PYG{o}{=}\PYG{o}{(}\PYG{n}{Element}\PYG{o}{)}\PYG{n}{lista}\PYG{o}{.}\PYG{n+na}{item}\PYG{o}{(}\PYG{n}{i}\PYG{o}{)}\PYG{o}{;}
                \PYG{n}{NodeList} \PYG{n}{listaHijos}\PYG{o}{=}\PYG{n}{e}\PYG{o}{.}\PYG{n+na}{getChildNodes}\PYG{o}{(}\PYG{o}{)}\PYG{o}{;}
                \PYG{c+c1}{//El elemento ciudad procedencia es el quinto hijo}
                \PYG{n}{Node} \PYG{n}{nodoCiudad}\PYG{o}{=}\PYG{n}{listaHijos}\PYG{o}{.}\PYG{n+na}{item}\PYG{o}{(}\PYG{l+m+mi}{5}\PYG{o}{)}\PYG{o}{;}
                \PYG{n}{NodeList} \PYG{n}{hijosCiudad}\PYG{o}{=}\PYG{n}{nodoCiudad}\PYG{o}{.}\PYG{n+na}{getChildNodes}\PYG{o}{(}\PYG{o}{)}\PYG{o}{;}
                \PYG{n}{Node} \PYG{n}{nodoElemCiudad}\PYG{o}{=}\PYG{n}{hijosCiudad}\PYG{o}{.}\PYG{n+na}{item}\PYG{o}{(}\PYG{l+m+mi}{1}\PYG{o}{)}\PYG{o}{;}
                \PYG{n}{String} \PYG{n}{nombreCiudad}\PYG{o}{=}\PYG{n}{nodoElemCiudad}\PYG{o}{.}\PYG{n+na}{getNodeName}\PYG{o}{(}\PYG{o}{)}\PYG{o}{;}
                \PYG{k}{if} \PYG{o}{(}\PYG{n}{nombreCiudad}\PYG{o}{.}\PYG{n+na}{equals}\PYG{o}{(}\PYG{n}{ciudad}\PYG{o}{)}\PYG{o}{)}\PYG{o}{\PYGZob{}}
                        \PYG{n}{cuantos}\PYG{o}{+}\PYG{o}{+}\PYG{o}{;}
                \PYG{o}{\PYGZcb{}} \PYG{c+c1}{//Fin del if}
        \PYG{o}{\PYGZcb{}} \PYG{c+c1}{//Fin del for}
        \PYG{k}{return} \PYG{n}{cuantos}\PYG{o}{;}
\PYG{o}{\PYGZcb{}}
\end{sphinxVerbatim}

Para resolver el último punto nos bastaría un método como este:

\begin{sphinxVerbatim}[commandchars=\\\{\}]
\PYG{k+kd}{public} \PYG{k+kt}{int} \PYG{n+nf}{letrasConPaisEmisor}\PYG{o}{(}\PYG{n}{Node} \PYG{n}{raiz}\PYG{o}{,} \PYG{n}{String} \PYG{n}{nombrePais}\PYG{o}{)}\PYG{o}{\PYGZob{}}
        \PYG{k+kt}{int} \PYG{n}{cuantos}\PYG{o}{=}\PYG{l+m+mi}{0}\PYG{o}{;}
        \PYG{n}{Element} \PYG{n}{eRaiz}\PYG{o}{=}\PYG{o}{(}\PYG{n}{Element}\PYG{o}{)} \PYG{n}{raiz}\PYG{o}{;}
        \PYG{n}{NodeList} \PYG{n}{listaLetras}\PYG{o}{=}\PYG{n}{eRaiz}\PYG{o}{.}\PYG{n+na}{getElementsByTagName}\PYG{o}{(}\PYG{l+s}{\PYGZdq{}letra\PYGZdq{}}\PYG{o}{)}\PYG{o}{;}
        \PYG{k}{for} \PYG{o}{(}\PYG{k+kt}{int} \PYG{n}{i}\PYG{o}{=}\PYG{l+m+mi}{0}\PYG{o}{;} \PYG{n}{i}\PYG{o}{\PYGZlt{}}\PYG{n}{listaLetras}\PYG{o}{.}\PYG{n+na}{getLength}\PYG{o}{(}\PYG{o}{)}\PYG{o}{;}\PYG{n}{i}\PYG{o}{+}\PYG{o}{+}\PYG{o}{)}\PYG{o}{\PYGZob{}}
                \PYG{n}{Node} \PYG{n}{nodo}\PYG{o}{=}\PYG{n}{listaLetras}\PYG{o}{.}\PYG{n+na}{item}\PYG{o}{(}\PYG{n}{i}\PYG{o}{)}\PYG{o}{;}
                \PYG{n}{Element} \PYG{n}{eNodo}\PYG{o}{=}\PYG{o}{(}\PYG{n}{Element}\PYG{o}{)} \PYG{n}{nodo}\PYG{o}{;} \PYG{c+c1}{//Devuelve elemento letra}
                \PYG{n}{NodeList} \PYG{n}{hijosLetra}\PYG{o}{=}\PYG{n}{eNodo}\PYG{o}{.}\PYG{n+na}{getChildNodes}\PYG{o}{(}\PYG{o}{)}\PYG{o}{;}
                \PYG{n}{Node} \PYG{n}{nodoPaisEmisor}\PYG{o}{=}\PYG{n}{hijosLetra}\PYG{o}{.}\PYG{n+na}{item}\PYG{o}{(}\PYG{l+m+mi}{3}\PYG{o}{)}\PYG{o}{;}
                \PYG{n}{NodeList} \PYG{n}{hijosPais}\PYG{o}{=}\PYG{n}{nodoPaisEmisor}\PYG{o}{.}\PYG{n+na}{getChildNodes}\PYG{o}{(}\PYG{o}{)}\PYG{o}{;}
                \PYG{n}{Node} \PYG{n}{nodoPais}\PYG{o}{=}\PYG{n}{hijosPais}\PYG{o}{.}\PYG{n+na}{item}\PYG{o}{(}\PYG{l+m+mi}{1}\PYG{o}{)}\PYG{o}{;}
                \PYG{n}{String} \PYG{n}{nombreNodoPais}\PYG{o}{=}\PYG{n}{nodoPais}\PYG{o}{.}\PYG{n+na}{getNodeName}\PYG{o}{(}\PYG{o}{)}\PYG{o}{;}
                \PYG{k}{if} \PYG{o}{(}\PYG{n}{nombreNodoPais}\PYG{o}{.}\PYG{n+na}{equals}\PYG{o}{(}\PYG{n}{nombrePais}\PYG{o}{)}\PYG{o}{)}\PYG{o}{\PYGZob{}}
                        \PYG{n}{cuantos}\PYG{o}{+}\PYG{o}{+}\PYG{o}{;}
                \PYG{o}{\PYGZcb{}}
        \PYG{o}{\PYGZcb{}} \PYG{c+c1}{//Fin del for}
        \PYG{k}{return} \PYG{n}{cuantos}\PYG{o}{;}
\PYG{o}{\PYGZcb{}}
\end{sphinxVerbatim}

Ahora el método que resuelve este ejercicio es tan simple como esto:

\begin{sphinxVerbatim}[commandchars=\\\{\}]
\PYG{k+kd}{public} \PYG{k+kt}{int} \PYG{n+nf}{algoQueVerConEspania}\PYG{o}{(}\PYG{n}{Node} \PYG{n}{raiz}\PYG{o}{)}\PYG{o}{\PYGZob{}}
        \PYG{k+kt}{int} \PYG{n}{cuantasLetras}\PYG{o}{=}\PYG{k}{this}\PYG{o}{.}\PYG{n+na}{letrasConPaisEmisor}\PYG{o}{(}\PYG{n}{raiz}\PYG{o}{,} \PYG{l+s}{\PYGZdq{}espania\PYGZdq{}}\PYG{o}{)}\PYG{o}{;}
        \PYG{k+kt}{int} \PYG{n}{cuantosFuturos}\PYG{o}{=}\PYG{k}{this}\PYG{o}{.}\PYG{n+na}{cuantosFuturosTienenCiudadProcedencia}\PYG{o}{(}\PYG{n}{raiz}\PYG{o}{,}
                        \PYG{l+s}{\PYGZdq{}madrid\PYGZdq{}}\PYG{o}{)}\PYG{o}{;}
        \PYG{k+kt}{int} \PYG{n}{cuantosBonos}\PYG{o}{=}\PYG{k}{this}\PYG{o}{.}\PYG{n+na}{algoQueVerCon}\PYG{o}{(}\PYG{n}{raiz}\PYG{o}{,} \PYG{l+s}{\PYGZdq{}España\PYGZdq{}}\PYG{o}{)}\PYG{o}{;}
        \PYG{k}{return} \PYG{n}{cuantasLetras}\PYG{o}{+}\PYG{n}{cuantosFuturos}\PYG{o}{+}\PYG{n}{cuantosBonos}\PYG{o}{;}
\PYG{o}{\PYGZcb{}}
\end{sphinxVerbatim}


\subsection{Ejercicio}
\label{\detokenize{tema6:id9}}
Crear un programa que indique el país de procedencia de todos aquellos bonos en los que el precio deseado tenga un valor comprendido entre el precio mínimo y el máximo.

Una posible solución sería esta:

\begin{sphinxVerbatim}[commandchars=\\\{\}]
\PYG{k+kd}{public} \PYG{k+kt}{void} \PYG{n+nf}{imprimirBonos}\PYG{o}{(}\PYG{n}{Node} \PYG{n}{raiz}\PYG{o}{)}\PYG{o}{\PYGZob{}}

        \PYG{n}{Element} \PYG{n}{eRaiz}\PYG{o}{=}\PYG{o}{(}\PYG{n}{Element}\PYG{o}{)} \PYG{n}{raiz}\PYG{o}{;}
        \PYG{n}{NodeList} \PYG{n}{listaBonos}\PYG{o}{=}\PYG{n}{eRaiz}\PYG{o}{.}\PYG{n+na}{getElementsByTagName}\PYG{o}{(}\PYG{l+s}{\PYGZdq{}bono\PYGZdq{}}\PYG{o}{)}\PYG{o}{;}
        \PYG{k}{for} \PYG{o}{(}\PYG{k+kt}{int} \PYG{n}{i}\PYG{o}{=}\PYG{l+m+mi}{0}\PYG{o}{;} \PYG{n}{i}\PYG{o}{\PYGZlt{}}\PYG{n}{listaBonos}\PYG{o}{.}\PYG{n+na}{getLength}\PYG{o}{(}\PYG{o}{)}\PYG{o}{;} \PYG{n}{i}\PYG{o}{+}\PYG{o}{+}\PYG{o}{)}\PYG{o}{\PYGZob{}}
                \PYG{n}{Node} \PYG{n}{bono}\PYG{o}{=}\PYG{n}{listaBonos}\PYG{o}{.}\PYG{n+na}{item}\PYG{o}{(}\PYG{n}{i}\PYG{o}{)}\PYG{o}{;}
                \PYG{n}{Element} \PYG{n}{eBono}\PYG{o}{=}\PYG{o}{(}\PYG{n}{Element}\PYG{o}{)} \PYG{n}{bono}\PYG{o}{;}
                \PYG{c+c1}{// Element eBono=(Element) listaBonos.item(i);}
                \PYG{n}{NodeList} \PYG{n}{listaParaValorDeseado}\PYG{o}{=}
                                \PYG{n}{eBono}\PYG{o}{.}\PYG{n+na}{getElementsByTagName}\PYG{o}{(}\PYG{l+s}{\PYGZdq{}valor\PYGZus{}deseado\PYGZdq{}}\PYG{o}{)}\PYG{o}{;}
                \PYG{n}{NodeList} \PYG{n}{listaParaValorMinimo}\PYG{o}{=}
                                \PYG{n}{eBono}\PYG{o}{.}\PYG{n+na}{getElementsByTagName}\PYG{o}{(}\PYG{l+s}{\PYGZdq{}valor\PYGZus{}minimo\PYGZdq{}}\PYG{o}{)}\PYG{o}{;}
                \PYG{n}{NodeList} \PYG{n}{listaParaValorMaximo}\PYG{o}{=}
                                \PYG{n}{eBono}\PYG{o}{.}\PYG{n+na}{getElementsByTagName}\PYG{o}{(}\PYG{l+s}{\PYGZdq{}valor\PYGZus{}maximo\PYGZdq{}}\PYG{o}{)}\PYG{o}{;}
                \PYG{n}{Node} \PYG{n}{nodoValorDeseado}\PYG{o}{=}\PYG{n}{listaParaValorDeseado}\PYG{o}{.}\PYG{n+na}{item}\PYG{o}{(}\PYG{l+m+mi}{0}\PYG{o}{)}\PYG{o}{;}
                \PYG{n}{Node} \PYG{n}{nodoValorMinimo}\PYG{o}{=}\PYG{n}{listaParaValorMinimo}\PYG{o}{.}\PYG{n+na}{item}\PYG{o}{(}\PYG{l+m+mi}{0}\PYG{o}{)}\PYG{o}{;}
                \PYG{n}{Node} \PYG{n}{nodoValorMaximo}\PYG{o}{=}\PYG{n}{listaParaValorMaximo}\PYG{o}{.}\PYG{n+na}{item}\PYG{o}{(}\PYG{l+m+mi}{0}\PYG{o}{)}\PYG{o}{;}

                \PYG{n}{Element} \PYG{n}{eValorDeseado}\PYG{o}{=}\PYG{o}{(}\PYG{n}{Element}\PYG{o}{)} \PYG{n}{nodoValorDeseado}\PYG{o}{;}
                \PYG{n}{Element} \PYG{n}{eValorMinimo}\PYG{o}{=}\PYG{o}{(}\PYG{n}{Element}\PYG{o}{)} \PYG{n}{nodoValorMinimo}\PYG{o}{;}
                \PYG{n}{Element} \PYG{n}{eValorMaximo}\PYG{o}{=}\PYG{o}{(}\PYG{n}{Element}\PYG{o}{)} \PYG{n}{nodoValorMaximo}\PYG{o}{;}

                \PYG{n}{String} \PYG{n}{cadValorDeseado}\PYG{o}{=}\PYG{n}{eValorDeseado}\PYG{o}{.}\PYG{n+na}{getTextContent}\PYG{o}{(}\PYG{o}{)}\PYG{o}{;}
                \PYG{n}{String} \PYG{n}{cadValorMinimo}\PYG{o}{=}\PYG{n}{eValorMinimo}\PYG{o}{.}\PYG{n+na}{getTextContent}\PYG{o}{(}\PYG{o}{)}\PYG{o}{;}
                \PYG{n}{String} \PYG{n}{cadValorMaximo}\PYG{o}{=}\PYG{n}{eValorMaximo}\PYG{o}{.}\PYG{n+na}{getTextContent}\PYG{o}{(}\PYG{o}{)}\PYG{o}{;}

                \PYG{k+kt}{int} \PYG{n}{valorDeseado}\PYG{o}{=}\PYG{n}{Integer}\PYG{o}{.}\PYG{n+na}{parseInt}\PYG{o}{(}\PYG{n}{cadValorDeseado}\PYG{o}{)}\PYG{o}{;}
                \PYG{k+kt}{int} \PYG{n}{valorMinimo}\PYG{o}{=}\PYG{n}{Integer}\PYG{o}{.}\PYG{n+na}{parseInt}\PYG{o}{(}\PYG{n}{cadValorMinimo}\PYG{o}{)}\PYG{o}{;}
                \PYG{k+kt}{int} \PYG{n}{valorMaximo}\PYG{o}{=}\PYG{n}{Integer}\PYG{o}{.}\PYG{n+na}{parseInt}\PYG{o}{(}\PYG{n}{cadValorMaximo}\PYG{o}{)}\PYG{o}{;}

                \PYG{k}{if} \PYG{o}{(}\PYG{o}{(}\PYG{n}{valorDeseado}\PYG{o}{\PYGZgt{}}\PYG{n}{valorMinimo}\PYG{o}{)} \PYG{o}{\PYGZam{}}\PYG{o}{\PYGZam{}}
                        \PYG{o}{(}\PYG{n}{valorDeseado}\PYG{o}{\PYGZlt{}}\PYG{n}{valorMaximo}\PYG{o}{)} \PYG{o}{)}\PYG{o}{\PYGZob{}}
                        \PYG{n}{System}\PYG{o}{.}\PYG{n+na}{out}\PYG{o}{.}\PYG{n+na}{println}\PYG{o}{(}\PYG{l+s}{\PYGZdq{}Encontrado un bono!\PYGZdq{}}\PYG{o}{)}\PYG{o}{;}
                \PYG{o}{\PYGZcb{}}
                \PYG{c+c1}{//Element eValorDeseado=(Element)}
                \PYG{c+c1}{//              listaParaValorDeseado.item(0);}

        \PYG{o}{\PYGZcb{}}

\PYG{o}{\PYGZcb{}}
\end{sphinxVerbatim}

Una solución mejor sería esta:

\begin{sphinxVerbatim}[commandchars=\\\{\}]
\PYG{k+kd}{public} \PYG{k+kt}{int} \PYG{n+nf}{extraerHijoNumero} \PYG{o}{(}\PYG{n}{Element} \PYG{n}{padre}\PYG{o}{,}
                \PYG{n}{String} \PYG{n}{nombreHijo}\PYG{o}{)}\PYG{o}{\PYGZob{}}
        \PYG{k+kt}{int} \PYG{n}{valor}\PYG{o}{=}\PYG{l+m+mi}{0}\PYG{o}{;}
        \PYG{c+c1}{//Esta lista tiene solo un elemento}
        \PYG{n}{NodeList} \PYG{n}{listaHijos}\PYG{o}{=}
                        \PYG{n}{padre}\PYG{o}{.}\PYG{n+na}{getElementsByTagName}\PYG{o}{(}\PYG{n}{nombreHijo}\PYG{o}{)}\PYG{o}{;}
        \PYG{n}{Element} \PYG{n}{hijoNumerico}\PYG{o}{=}\PYG{o}{(}\PYG{n}{Element}\PYG{o}{)} \PYG{n}{listaHijos}\PYG{o}{.}\PYG{n+na}{item}\PYG{o}{(}\PYG{l+m+mi}{0}\PYG{o}{)}\PYG{o}{;}
        \PYG{n}{String} \PYG{n}{contenidoTextual}\PYG{o}{=}\PYG{n}{hijoNumerico}\PYG{o}{.}\PYG{n+na}{getTextContent}\PYG{o}{(}\PYG{o}{)}\PYG{o}{;}
        \PYG{n}{valor}\PYG{o}{=}\PYG{n}{Integer}\PYG{o}{.}\PYG{n+na}{parseInt}\PYG{o}{(}\PYG{n}{contenidoTextual}\PYG{o}{)}\PYG{o}{;}
        \PYG{k}{return} \PYG{n}{valor}\PYG{o}{;}
\PYG{o}{\PYGZcb{}}
\PYG{k+kd}{public} \PYG{k+kt}{void} \PYG{n+nf}{imprimirBonos}\PYG{o}{(}\PYG{n}{Node} \PYG{n}{raiz}\PYG{o}{)}\PYG{o}{\PYGZob{}}

        \PYG{n}{Element} \PYG{n}{eRaiz}\PYG{o}{=}\PYG{o}{(}\PYG{n}{Element}\PYG{o}{)} \PYG{n}{raiz}\PYG{o}{;}
        \PYG{n}{NodeList} \PYG{n}{listaBonos}\PYG{o}{=}\PYG{n}{eRaiz}\PYG{o}{.}\PYG{n+na}{getElementsByTagName}\PYG{o}{(}\PYG{l+s}{\PYGZdq{}bono\PYGZdq{}}\PYG{o}{)}\PYG{o}{;}
        \PYG{k}{for} \PYG{o}{(}\PYG{k+kt}{int} \PYG{n}{i}\PYG{o}{=}\PYG{l+m+mi}{0}\PYG{o}{;} \PYG{n}{i}\PYG{o}{\PYGZlt{}}\PYG{n}{listaBonos}\PYG{o}{.}\PYG{n+na}{getLength}\PYG{o}{(}\PYG{o}{)}\PYG{o}{;} \PYG{n}{i}\PYG{o}{+}\PYG{o}{+}\PYG{o}{)}\PYG{o}{\PYGZob{}}
                \PYG{n}{Node} \PYG{n}{bono}\PYG{o}{=}\PYG{n}{listaBonos}\PYG{o}{.}\PYG{n+na}{item}\PYG{o}{(}\PYG{n}{i}\PYG{o}{)}\PYG{o}{;}
                \PYG{n}{Element} \PYG{n}{eBono}\PYG{o}{=}\PYG{o}{(}\PYG{n}{Element}\PYG{o}{)} \PYG{n}{listaBonos}\PYG{o}{.}\PYG{n+na}{item}\PYG{o}{(}\PYG{n}{i}\PYG{o}{)}\PYG{o}{;}

                \PYG{k+kt}{int} \PYG{n}{valorDeseado}\PYG{o}{=}\PYG{k}{this}\PYG{o}{.}\PYG{n+na}{extraerHijoNumero}\PYG{o}{(}
                                \PYG{n}{eBono}\PYG{o}{,} \PYG{l+s}{\PYGZdq{}valor\PYGZus{}deseado\PYGZdq{}}\PYG{o}{)}\PYG{o}{;}
                \PYG{k+kt}{int} \PYG{n}{valorMinimo}\PYG{o}{=}\PYG{k}{this}\PYG{o}{.}\PYG{n+na}{extraerHijoNumero}\PYG{o}{(}
                                \PYG{n}{eBono}\PYG{o}{,} \PYG{l+s}{\PYGZdq{}valor\PYGZus{}minimo\PYGZdq{}}\PYG{o}{)}\PYG{o}{;}
                \PYG{k+kt}{int} \PYG{n}{valorMaximo}\PYG{o}{=}\PYG{k}{this}\PYG{o}{.}\PYG{n+na}{extraerHijoNumero}\PYG{o}{(}
                                \PYG{n}{eBono}\PYG{o}{,} \PYG{l+s}{\PYGZdq{}valor\PYGZus{}maximo\PYGZdq{}}\PYG{o}{)}\PYG{o}{;}

                \PYG{k}{if} \PYG{o}{(}\PYG{o}{(}\PYG{n}{valorDeseado}\PYG{o}{\PYGZgt{}}\PYG{n}{valorMinimo}\PYG{o}{)} \PYG{o}{\PYGZam{}}\PYG{o}{\PYGZam{}}
                        \PYG{o}{(}\PYG{n}{valorDeseado}\PYG{o}{\PYGZlt{}}\PYG{n}{valorMaximo}\PYG{o}{)} \PYG{o}{)}\PYG{o}{\PYGZob{}}
                        \PYG{n}{System}\PYG{o}{.}\PYG{n+na}{out}\PYG{o}{.}\PYG{n+na}{println}\PYG{o}{(}\PYG{l+s}{\PYGZdq{}Encontrado un bono!\PYGZdq{}}\PYG{o}{)}\PYG{o}{;}
                \PYG{o}{\PYGZcb{}}
                \PYG{c+c1}{//Element eValorDeseado=(Element)}
                \PYG{c+c1}{//              listaParaValorDeseado.item(0);}

        \PYG{o}{\PYGZcb{}}

\PYG{o}{\PYGZcb{}}
\end{sphinxVerbatim}


\subsection{Ejercicio}
\label{\detokenize{tema6:id10}}
Imprimir, los productos financieros con la misma ciudad de procedencia.


\subsubsection{Análisis}
\label{\detokenize{tema6:id11}}
En general, todos los problemas donde nos piden algo como \sphinxstyleemphasis{comprobar todos los elementos que tengan las mismas características} implican hacer una comprobación de \sphinxstyleemphasis{todos con todos}.


\subsubsection{Diseño}
\label{\detokenize{tema6:id12}}
Todos los elementos tienen un elemento \sphinxcode{ciudad\_de\_procedencia}, por lo cual, probablemente sea útil crear algún pequeño método de utilidad que dado un elemento nos devuelva un string con la ciudad de procedencia.

Por otro lado, comparar \sphinxstyleemphasis{todos con todos} suele implicar un doble bucle, donde el primer irá extrayendo elementos y el otro irá extrayendo todos los demás.


\subsubsection{Implementación}
\label{\detokenize{tema6:implementacion}}
\begin{sphinxVerbatim}[commandchars=\\\{\}]
\PYG{k+kd}{public} \PYG{n}{String} \PYG{n+nf}{getCiudadProcedencia}\PYG{o}{(}\PYG{n}{Element} \PYG{n}{e}\PYG{o}{)}\PYG{o}{\PYGZob{}}
        \PYG{n}{NodeList} \PYG{n}{listaHijos}\PYG{o}{=}
                        \PYG{n}{e}\PYG{o}{.}\PYG{n+na}{getElementsByTagName}\PYG{o}{(}\PYG{l+s}{\PYGZdq{}ciudad\PYGZus{}procedencia\PYGZdq{}}\PYG{o}{)}\PYG{o}{;}
        \PYG{n}{Element} \PYG{n}{eCiudad}\PYG{o}{=}\PYG{o}{(}\PYG{n}{Element}\PYG{o}{)} \PYG{n}{listaHijos}\PYG{o}{.}\PYG{n+na}{item}\PYG{o}{(}\PYG{l+m+mi}{0}\PYG{o}{)}\PYG{o}{;}
        \PYG{n}{NodeList} \PYG{n}{listaHijosCiudad}\PYG{o}{=}\PYG{n}{eCiudad}\PYG{o}{.}\PYG{n+na}{getChildNodes}\PYG{o}{(}\PYG{o}{)}\PYG{o}{;}
        \PYG{n}{Element} \PYG{n}{eCiudadConcreto}\PYG{o}{=}
                        \PYG{o}{(}\PYG{n}{Element}\PYG{o}{)} \PYG{n}{listaHijosCiudad}\PYG{o}{.}\PYG{n+na}{item}\PYG{o}{(}\PYG{l+m+mi}{1}\PYG{o}{)}\PYG{o}{;}
        \PYG{n}{String} \PYG{n}{nombre}\PYG{o}{=}\PYG{n}{eCiudadConcreto}\PYG{o}{.}\PYG{n+na}{getNodeName}\PYG{o}{(}\PYG{o}{)}\PYG{o}{;}
        \PYG{k}{return} \PYG{n}{nombre}\PYG{o}{;}
\PYG{o}{\PYGZcb{}}
\end{sphinxVerbatim}

\begin{sphinxVerbatim}[commandchars=\\\{\}]
\PYG{k+kd}{public} \PYG{k+kt}{void} \PYG{n+nf}{imprimirMismaCiudad}\PYG{o}{(}\PYG{n}{Node} \PYG{n}{raiz}\PYG{o}{)}\PYG{o}{\PYGZob{}}
        \PYG{n}{NodeList} \PYG{n}{hijos}\PYG{o}{=}\PYG{n}{raiz}\PYG{o}{.}\PYG{n+na}{getChildNodes}\PYG{o}{(}\PYG{o}{)}\PYG{o}{;}
        \PYG{k}{for} \PYG{o}{(}\PYG{k+kt}{int} \PYG{n}{i}\PYG{o}{=}\PYG{l+m+mi}{0}\PYG{o}{;} \PYG{n}{i}\PYG{o}{\PYGZlt{}}\PYG{n}{hijos}\PYG{o}{.}\PYG{n+na}{getLength}\PYG{o}{(}\PYG{o}{)}\PYG{o}{;} \PYG{n}{i}\PYG{o}{+}\PYG{o}{+}\PYG{o}{)}\PYG{o}{\PYGZob{}}
                \PYG{n}{Node} \PYG{n}{hijo}\PYG{o}{=}\PYG{n}{hijos}\PYG{o}{.}\PYG{n+na}{item}\PYG{o}{(}\PYG{n}{i}\PYG{o}{)}\PYG{o}{;}
                \PYG{k}{if} \PYG{o}{(}\PYG{n}{hijo}\PYG{o}{.}\PYG{n+na}{getNodeType}\PYG{o}{(}\PYG{o}{)}\PYG{o}{!}\PYG{o}{=}\PYG{n}{Node}\PYG{o}{.}\PYG{n+na}{ELEMENT\PYGZus{}NODE}\PYG{o}{)}\PYG{o}{\PYGZob{}}
                        \PYG{k}{continue}\PYG{o}{;}
                \PYG{o}{\PYGZcb{}}
                \PYG{k}{for} \PYG{o}{(}\PYG{k+kt}{int} \PYG{n}{j}\PYG{o}{=}\PYG{l+m+mi}{0}\PYG{o}{;} \PYG{n}{j}\PYG{o}{\PYGZlt{}}\PYG{n}{hijos}\PYG{o}{.}\PYG{n+na}{getLength}\PYG{o}{(}\PYG{o}{)}\PYG{o}{;} \PYG{n}{j}\PYG{o}{+}\PYG{o}{+}\PYG{o}{)}\PYG{o}{\PYGZob{}}
                        \PYG{n}{Node} \PYG{n}{otroHijo}\PYG{o}{=}\PYG{n}{hijos}\PYG{o}{.}\PYG{n+na}{item}\PYG{o}{(}\PYG{n}{j}\PYG{o}{)}\PYG{o}{;}
                        \PYG{k}{if} \PYG{o}{(}\PYG{n}{otroHijo}\PYG{o}{.}\PYG{n+na}{getNodeType}\PYG{o}{(}\PYG{o}{)}\PYG{o}{!}\PYG{o}{=}\PYG{n}{Node}\PYG{o}{.}\PYG{n+na}{ELEMENT\PYGZus{}NODE}\PYG{o}{)}\PYG{o}{\PYGZob{}}
                                \PYG{k}{continue}\PYG{o}{;}
                        \PYG{o}{\PYGZcb{}}
                        \PYG{n}{String} \PYG{n}{ciudadHijo}\PYG{o}{=}
                                        \PYG{k}{this}\PYG{o}{.}\PYG{n+na}{getCiudadProcedencia}\PYG{o}{(}\PYG{o}{(}\PYG{n}{Element}\PYG{o}{)}\PYG{n}{hijo}\PYG{o}{)}\PYG{o}{;}
                        \PYG{n}{String} \PYG{n}{ciudadOtro}\PYG{o}{=}
                                        \PYG{k}{this}\PYG{o}{.}\PYG{n+na}{getCiudadProcedencia}\PYG{o}{(}\PYG{o}{(}\PYG{n}{Element}\PYG{o}{)}\PYG{n}{otroHijo}\PYG{o}{)}\PYG{o}{;}
                        \PYG{k}{if} \PYG{o}{(}\PYG{n}{ciudadHijo}\PYG{o}{.}\PYG{n+na}{equals}\PYG{o}{(}\PYG{n}{ciudadOtro}\PYG{o}{)}\PYG{o}{)}\PYG{o}{\PYGZob{}}
                                \PYG{n}{System}\PYG{o}{.}\PYG{n+na}{out}\PYG{o}{.}\PYG{n+na}{println}\PYG{o}{(}
                                                \PYG{l+s}{\PYGZdq{}Encontré dos elementos con la ciudad \PYGZdq{}}\PYG{o}{+}\PYG{n}{ciudadHijo}\PYG{o}{)}\PYG{o}{;}

                        \PYG{o}{\PYGZcb{}} \PYG{c+c1}{//Fin del if ciudadHijo}
                \PYG{o}{\PYGZcb{}} \PYG{c+c1}{//Fin del for interno}
        \PYG{o}{\PYGZcb{}} \PYG{c+c1}{//Fin del for externo}
\PYG{o}{\PYGZcb{}} \PYG{c+c1}{//Fin del método}

\PYG{k+kd}{public} \PYG{k+kt}{void} \PYG{n+nf}{imprimirMismaCiudad}\PYG{o}{(}\PYG{n}{Node} \PYG{n}{raiz}\PYG{o}{)}\PYG{o}{\PYGZob{}}
        \PYG{n}{NodeList} \PYG{n}{hijos}\PYG{o}{=}\PYG{n}{raiz}\PYG{o}{.}\PYG{n+na}{getChildNodes}\PYG{o}{(}\PYG{o}{)}\PYG{o}{;}
        \PYG{k}{for} \PYG{o}{(}\PYG{k+kt}{int} \PYG{n}{i}\PYG{o}{=}\PYG{l+m+mi}{0}\PYG{o}{;} \PYG{n}{i}\PYG{o}{\PYGZlt{}}\PYG{n}{hijos}\PYG{o}{.}\PYG{n+na}{getLength}\PYG{o}{(}\PYG{o}{)}\PYG{o}{;} \PYG{n}{i}\PYG{o}{+}\PYG{o}{+}\PYG{o}{)}\PYG{o}{\PYGZob{}}
                \PYG{n}{Node} \PYG{n}{hijo}\PYG{o}{=}\PYG{n}{hijos}\PYG{o}{.}\PYG{n+na}{item}\PYG{o}{(}\PYG{n}{i}\PYG{o}{)}\PYG{o}{;}
                \PYG{k}{if} \PYG{o}{(}\PYG{n}{hijo}\PYG{o}{.}\PYG{n+na}{getNodeType}\PYG{o}{(}\PYG{o}{)}\PYG{o}{!}\PYG{o}{=}\PYG{n}{Node}\PYG{o}{.}\PYG{n+na}{ELEMENT\PYGZus{}NODE}\PYG{o}{)}\PYG{o}{\PYGZob{}}
                        \PYG{k}{continue}\PYG{o}{;}
                \PYG{o}{\PYGZcb{}}
                \PYG{n}{String} \PYG{n}{ciudadHijo}\PYG{o}{=}
                                \PYG{k}{this}\PYG{o}{.}\PYG{n+na}{getCiudadProcedencia}\PYG{o}{(}\PYG{o}{(}\PYG{n}{Element}\PYG{o}{)}\PYG{n}{hijo}\PYG{o}{)}\PYG{o}{;}
                \PYG{k}{for} \PYG{o}{(}\PYG{k+kt}{int} \PYG{n}{j}\PYG{o}{=}\PYG{n}{i}\PYG{o}{+}\PYG{l+m+mi}{1}\PYG{o}{;} \PYG{n}{j}\PYG{o}{\PYGZlt{}}\PYG{n}{hijos}\PYG{o}{.}\PYG{n+na}{getLength}\PYG{o}{(}\PYG{o}{)}\PYG{o}{;} \PYG{n}{j}\PYG{o}{+}\PYG{o}{+}\PYG{o}{)}\PYG{o}{\PYGZob{}}
                        \PYG{n}{Node} \PYG{n}{otroHijo}\PYG{o}{=}\PYG{n}{hijos}\PYG{o}{.}\PYG{n+na}{item}\PYG{o}{(}\PYG{n}{j}\PYG{o}{)}\PYG{o}{;}
                        \PYG{k}{if} \PYG{o}{(}\PYG{n}{otroHijo}\PYG{o}{.}\PYG{n+na}{getNodeType}\PYG{o}{(}\PYG{o}{)}\PYG{o}{!}\PYG{o}{=}\PYG{n}{Node}\PYG{o}{.}\PYG{n+na}{ELEMENT\PYGZus{}NODE}\PYG{o}{)}\PYG{o}{\PYGZob{}}
                                \PYG{k}{continue}\PYG{o}{;}
                        \PYG{o}{\PYGZcb{}}

                        \PYG{n}{String} \PYG{n}{ciudadOtro}\PYG{o}{=}
                                        \PYG{k}{this}\PYG{o}{.}\PYG{n+na}{getCiudadProcedencia}\PYG{o}{(}\PYG{o}{(}\PYG{n}{Element}\PYG{o}{)}\PYG{n}{otroHijo}\PYG{o}{)}\PYG{o}{;}
                        \PYG{k}{if} \PYG{o}{(}\PYG{n}{ciudadHijo}\PYG{o}{.}\PYG{n+na}{equals}\PYG{o}{(}\PYG{n}{ciudadOtro}\PYG{o}{)}\PYG{o}{)}\PYG{o}{\PYGZob{}}
                                \PYG{n}{System}\PYG{o}{.}\PYG{n+na}{out}\PYG{o}{.}\PYG{n+na}{println}\PYG{o}{(}
                                                \PYG{l+s}{\PYGZdq{}Encontré dos elementos con la ciudad \PYGZdq{}}\PYG{o}{+}\PYG{n}{ciudadHijo}\PYG{o}{)}\PYG{o}{;}

                        \PYG{o}{\PYGZcb{}} \PYG{c+c1}{//Fin del if ciudadHijo}
                \PYG{o}{\PYGZcb{}} \PYG{c+c1}{//Fin del for interno}
        \PYG{o}{\PYGZcb{}} \PYG{c+c1}{//Fin del for externo}
\PYG{o}{\PYGZcb{}} \PYG{c+c1}{//Fin del método}
\end{sphinxVerbatim}


\subsection{Ejercicio}
\label{\detokenize{tema6:id13}}
Contar cuantos productos que no sean estables tienen como ciudad de procedencia Tokio. Si hay más de 2, devolver los precios en un vector y si no devolver un vector vacío.


\subsubsection{Análisis}
\label{\detokenize{tema6:id14}}
El atributo \sphinxcode{estable}, solo lo tienen los productos \sphinxcode{bono}. Ese atributo es optativo, puede que esté o puede que no. Si existe, debemos comprobar si tiene un «no».

Por otro lado, no sabemos a priori si habrá más de 2 o no.
\begin{itemize}
\item {} 
Podríamos hacer la cuenta, y si da más de 2 repetir operaciones y meter los bonos correctos en un vector.

\item {} 
Podríamos ir haciendo las operaciones y a la vez las inserciones en un vector y ahorrarnos operaciones.

\end{itemize}

Optaremos por la segunda.


\subsubsection{Diseño}
\label{\detokenize{tema6:id15}}\begin{itemize}
\item {} 
Aprovecharemos los métodos que nos permiten extraer el elemento raíz de un fichero.

\item {} 
Necesitaremos crear vectores que tengan un cierto tamaño. Crearemos vectores muy grandes y los dejaremos con la inicialización que hace Java por defecto llenándolo con valores \sphinxcode{null}.

\item {} 
Podemos aprovechar métodos ofrecidos por Java como \sphinxcode{getElementsByTagName}.

\item {} 
Después recorreremos los elementos, comprobaremos si sus atributos cumplen las condiciones y si las cumplen almacenaremos en el vector a devolver la ciudad de ese bono.

\item {} 
Nuestro método devolverá siempre algo como String{[}{]}, ese vector puede que vaya lleno o no.

\end{itemize}


\subsubsection{Solución 1}
\label{\detokenize{tema6:solucion-1}}
\begin{sphinxVerbatim}[commandchars=\\\{\}]
\PYG{k+kd}{public} \PYG{k+kd}{class} \PYG{n+nc}{ProcesadorXML} \PYG{o}{\PYGZob{}}
        \PYG{k+kd}{public} \PYG{n}{Element} \PYG{n+nf}{getRaiz}\PYG{o}{(}\PYG{n}{String} \PYG{n}{nombreFichero}\PYG{o}{)}
                        \PYG{k+kd}{throws} \PYG{n}{ParserConfigurationException}\PYG{o}{,} \PYG{n}{SAXException}\PYG{o}{,} \PYG{n}{IOException}\PYG{o}{\PYGZob{}}
                \PYG{n}{DocumentBuilderFactory}
                        \PYG{n}{fabrica} \PYG{o}{=} \PYG{n}{DocumentBuilderFactory}\PYG{o}{.}\PYG{n+na}{newInstance}\PYG{o}{(}\PYG{o}{)}\PYG{o}{;}
                \PYG{n}{DocumentBuilder} \PYG{n}{constructor}\PYG{o}{=}
                                \PYG{n}{fabrica}\PYG{o}{.}\PYG{n+na}{newDocumentBuilder}\PYG{o}{(}\PYG{o}{)}\PYG{o}{;}
                \PYG{n}{FileInputStream} \PYG{n}{fichero}\PYG{o}{=}
                                \PYG{k}{new} \PYG{n}{FileInputStream}\PYG{o}{(}\PYG{n}{nombreFichero}\PYG{o}{)}\PYG{o}{;}
                \PYG{n}{Document} \PYG{n}{documento}\PYG{o}{=}
                                \PYG{n}{constructor}\PYG{o}{.}\PYG{n+na}{parse}\PYG{o}{(}\PYG{n}{fichero}\PYG{o}{)}\PYG{o}{;}
                \PYG{n}{Element} \PYG{n}{raiz}\PYG{o}{=}\PYG{n}{documento}\PYG{o}{.}\PYG{n+na}{getDocumentElement}\PYG{o}{(}\PYG{o}{)}\PYG{o}{;}
                \PYG{k}{return} \PYG{n}{raiz}\PYG{o}{;}
        \PYG{o}{\PYGZcb{}}
        \PYG{k+kd}{public} \PYG{k+kt}{int}\PYG{o}{[}\PYG{o}{]} \PYG{n+nf}{getPreciosInestables}\PYG{o}{(}\PYG{o}{)}
                        \PYG{k+kd}{throws} \PYG{n}{ParserConfigurationException}\PYG{o}{,} \PYG{n}{SAXException}\PYG{o}{,} \PYG{n}{IOException}\PYG{o}{\PYGZob{}}
                \PYG{k+kt}{int}\PYG{o}{[}\PYG{o}{]} \PYG{n}{vPrecios}\PYG{o}{=}\PYG{k+kc}{null}\PYG{o}{;}
                \PYG{k+kt}{int} \PYG{n}{contador}\PYG{o}{=}\PYG{l+m+mi}{0}\PYG{o}{;}
                \PYG{n}{Element} \PYG{n}{raiz}\PYG{o}{=}
                                \PYG{n}{getRaiz}\PYG{o}{(}\PYG{l+s}{\PYGZdq{}d:/oscar/productos.xml\PYGZdq{}}\PYG{o}{)}\PYG{o}{;}
                \PYG{n}{Element} \PYG{n}{hijo}\PYG{o}{=}\PYG{o}{(}\PYG{n}{Element}\PYG{o}{)}
                                \PYG{n}{raiz}\PYG{o}{.}\PYG{n+na}{getFirstChild}\PYG{o}{(}\PYG{o}{)}\PYG{o}{;}
                \PYG{c+cm}{/* Mientras le queden hijos a la raíz...*/}
                \PYG{k}{while} \PYG{o}{(}\PYG{n}{hijo}\PYG{o}{!}\PYG{o}{=}\PYG{k+kc}{null}\PYG{o}{)}\PYG{o}{\PYGZob{}}
                        \PYG{n}{String} \PYG{n}{atrEstable}\PYG{o}{=}
                                        \PYG{n}{hijo}\PYG{o}{.}\PYG{n+na}{getAttribute}\PYG{o}{(}\PYG{l+s}{\PYGZdq{}estable\PYGZdq{}}\PYG{o}{)}\PYG{o}{;}
                        \PYG{k}{if} \PYG{o}{(}\PYG{n}{atrEstable}\PYG{o}{.}\PYG{n+na}{equals}\PYG{o}{(}\PYG{l+s}{\PYGZdq{}no\PYGZdq{}}\PYG{o}{)}\PYG{o}{)}\PYG{o}{\PYGZob{}}
                                \PYG{n}{NodeList} \PYG{n}{vector}\PYG{o}{=}
                                                \PYG{n}{hijo}\PYG{o}{.}\PYG{n+na}{getElementsByTagName}\PYG{o}{(}
                                                        \PYG{l+s}{\PYGZdq{}tokio\PYGZdq{}}\PYG{o}{)}\PYG{o}{;}
                                \PYG{k}{if} \PYG{o}{(}\PYG{n}{vector}\PYG{o}{.}\PYG{n+na}{getLength}\PYG{o}{(}\PYG{o}{)}\PYG{o}{\PYGZgt{}}\PYG{l+m+mi}{0}\PYG{o}{)}\PYG{o}{\PYGZob{}}
                                        \PYG{c+c1}{//El producto sí es de Tokio}
                                        \PYG{n}{String} \PYG{n}{precio}\PYG{o}{=}
                                                        \PYG{n}{hijo}\PYG{o}{.}\PYG{n+na}{getAttribute}\PYG{o}{(}
                                                                        \PYG{l+s}{\PYGZdq{}precio\PYGZdq{}}\PYG{o}{)}\PYG{o}{;}
                                        \PYG{n}{System}\PYG{o}{.}\PYG{n+na}{out}\PYG{o}{.}\PYG{n+na}{println}\PYG{o}{(}\PYG{l+s}{\PYGZdq{}Precio:\PYGZdq{}}\PYG{o}{+}\PYG{n}{precio}\PYG{o}{)}\PYG{o}{;}
                                \PYG{o}{\PYGZcb{}}
                        \PYG{o}{\PYGZcb{}}
                        \PYG{n}{Node} \PYG{n}{finLinea}\PYG{o}{=}\PYG{n}{hijo}\PYG{o}{.}\PYG{n+na}{getNextSibling}\PYG{o}{(}\PYG{o}{)}\PYG{o}{;}
                        \PYG{n}{hijo}\PYG{o}{=}\PYG{o}{(}\PYG{n}{Element}\PYG{o}{)}
                                        \PYG{n}{finLinea}\PYG{o}{.}\PYG{n+na}{getNextSibling}\PYG{o}{(}\PYG{o}{)}\PYG{o}{;}
                \PYG{o}{\PYGZcb{}}
                \PYG{k}{return} \PYG{n}{vPrecios}\PYG{o}{;}
        \PYG{o}{\PYGZcb{}}

        \PYG{k+kd}{public} \PYG{k+kd}{static} \PYG{k+kt}{void} \PYG{n+nf}{main}\PYG{o}{(}\PYG{n}{String}\PYG{o}{[}\PYG{o}{]} \PYG{n}{args}\PYG{o}{)}
                        \PYG{k+kd}{throws} \PYG{n}{ParserConfigurationException}\PYG{o}{,} \PYG{n}{SAXException}\PYG{o}{,} \PYG{n}{IOException} \PYG{o}{\PYGZob{}}
                \PYG{n}{ProcesadorXML} \PYG{n}{procesador}\PYG{o}{=}
                                \PYG{k}{new} \PYG{n}{ProcesadorXML}\PYG{o}{(}\PYG{o}{)}\PYG{o}{;}
                \PYG{k+kt}{int}\PYG{o}{[}\PYG{o}{]} \PYG{n}{precios}\PYG{o}{=}
                                \PYG{n}{procesador}\PYG{o}{.}\PYG{n+na}{getPreciosInestables}\PYG{o}{(}\PYG{o}{)}\PYG{o}{;}
        \PYG{o}{\PYGZcb{}}
\PYG{o}{\PYGZcb{}}
\end{sphinxVerbatim}


\subsubsection{Solución 2}
\label{\detokenize{tema6:solucion-2}}
\begin{sphinxVerbatim}[commandchars=\\\{\}]
\PYG{k+kd}{public} \PYG{n}{String}\PYG{o}{[}\PYG{o}{]} \PYG{n+nf}{obtenerListaBonos}\PYG{o}{(}\PYG{n}{Node} \PYG{n}{raiz}\PYG{o}{)}\PYG{o}{\PYGZob{}}
        \PYG{k+kt}{int} \PYG{n}{tamanioMaximo}\PYG{o}{=}\PYG{l+m+mi}{1000}\PYG{o}{;}
        \PYG{n}{String}\PYG{o}{[}\PYG{o}{]} \PYG{n}{ciudades}\PYG{o}{=}\PYG{k}{new} \PYG{n}{String}\PYG{o}{[}\PYG{n}{tamanioMaximo}\PYG{o}{]}\PYG{o}{;}
        \PYG{n}{String}\PYG{o}{[}\PYG{o}{]} \PYG{n}{aux}\PYG{o}{=}\PYG{k}{new} \PYG{n}{String}\PYG{o}{[}\PYG{n}{tamanioMaximo}\PYG{o}{]}\PYG{o}{;}
        \PYG{k+kt}{int} \PYG{n}{posPrecio}\PYG{o}{=}\PYG{l+m+mi}{0}\PYG{o}{;}
        \PYG{n}{Element} \PYG{n}{eRaiz}\PYG{o}{=}\PYG{o}{(}\PYG{n}{Element}\PYG{o}{)} \PYG{n}{raiz}\PYG{o}{;}
        \PYG{n}{NodeList} \PYG{n}{listaBonos}\PYG{o}{=}\PYG{n}{eRaiz}\PYG{o}{.}\PYG{n+na}{getElementsByTagName}\PYG{o}{(}\PYG{l+s}{\PYGZdq{}bono\PYGZdq{}}\PYG{o}{)}\PYG{o}{;}
        \PYG{k}{for} \PYG{o}{(}\PYG{k+kt}{int} \PYG{n}{i}\PYG{o}{=}\PYG{l+m+mi}{0}\PYG{o}{;} \PYG{n}{i}\PYG{o}{\PYGZlt{}}\PYG{n}{listaBonos}\PYG{o}{.}\PYG{n+na}{getLength}\PYG{o}{(}\PYG{o}{)}\PYG{o}{;} \PYG{n}{i}\PYG{o}{+}\PYG{o}{+}\PYG{o}{)}\PYG{o}{\PYGZob{}}
                \PYG{n}{Element} \PYG{n}{bono}\PYG{o}{=}\PYG{o}{(}\PYG{n}{Element}\PYG{o}{)} \PYG{n}{listaBonos}\PYG{o}{.}\PYG{n+na}{item}\PYG{o}{(}\PYG{n}{i}\PYG{o}{)}\PYG{o}{;}
                \PYG{n}{String} \PYG{n}{atEstable}\PYG{o}{=}\PYG{n}{bono}\PYG{o}{.}\PYG{n+na}{getAttribute}\PYG{o}{(}\PYG{l+s}{\PYGZdq{}estable\PYGZdq{}}\PYG{o}{)}\PYG{o}{;}
                \PYG{k}{if} \PYG{o}{(}\PYG{n}{atEstable}\PYG{o}{!}\PYG{o}{=}\PYG{k+kc}{null}\PYG{o}{)}\PYG{o}{\PYGZob{}}
                        \PYG{k}{if} \PYG{o}{(}\PYG{n}{atEstable}\PYG{o}{.}\PYG{n+na}{equals}\PYG{o}{(}\PYG{l+s}{\PYGZdq{}no\PYGZdq{}}\PYG{o}{)}\PYG{o}{)}\PYG{o}{\PYGZob{}}
                                \PYG{c+c1}{//Examinamos la ciudad aprovechando}
                                \PYG{c+c1}{//un método ya construido.}
                                \PYG{n}{String} \PYG{n}{ciudad}\PYG{o}{=}\PYG{k}{this}\PYG{o}{.}\PYG{n+na}{getCiudadProcedencia}\PYG{o}{(}\PYG{n}{bono}\PYG{o}{)}\PYG{o}{;}
                                \PYG{k}{if} \PYG{o}{(}\PYG{n}{ciudad}\PYG{o}{.}\PYG{n+na}{equals}\PYG{o}{(}\PYG{l+s}{\PYGZdq{}tokio\PYGZdq{}}\PYG{o}{)}\PYG{o}{)}\PYG{o}{\PYGZob{}}
                                        \PYG{c+c1}{//Si es de tokio, copiamos el precio}
                                        \PYG{n}{String} \PYG{n}{precio}\PYG{o}{=}\PYG{n}{bono}\PYG{o}{.}\PYG{n+na}{getAttribute}\PYG{o}{(}\PYG{l+s}{\PYGZdq{}precio\PYGZdq{}}\PYG{o}{)}\PYG{o}{;}
                                        \PYG{n}{aux}\PYG{o}{[}\PYG{n}{posPrecio}\PYG{o}{]}\PYG{o}{=}\PYG{n}{precio}\PYG{o}{;}
                                        \PYG{n}{posPrecio}\PYG{o}{+}\PYG{o}{+}\PYG{o}{;}
                                \PYG{o}{\PYGZcb{}} \PYG{c+c1}{//Fin del if para tokio}
                        \PYG{o}{\PYGZcb{}} \PYG{c+c1}{//Fin del if para el \PYGZdq{}no\PYGZdq{}}
                \PYG{o}{\PYGZcb{}} \PYG{c+c1}{//Fin del if para atEstable}
        \PYG{o}{\PYGZcb{}} \PYG{c+c1}{//Fin del for que recorre los bonos}
        \PYG{k}{if} \PYG{o}{(}\PYG{n}{posPrecio}\PYG{o}{\PYGZgt{}}\PYG{l+m+mi}{2}\PYG{o}{)}\PYG{o}{\PYGZob{}}
                \PYG{k}{return} \PYG{n}{aux}\PYG{o}{;}
        \PYG{o}{\PYGZcb{}}
        \PYG{k}{return} \PYG{n}{ciudades}\PYG{o}{;}
\PYG{o}{\PYGZcb{}}
\end{sphinxVerbatim}


\section{Anexo}
\label{\detokenize{tema6:anexo}}

\subsection{Código Java}
\label{\detokenize{tema6:codigo-java}}
A continuación se muestra el código Java completo:

\begin{sphinxVerbatim}[commandchars=\\\{\}]
\PYG{k+kn}{package} \PYG{n+nn}{com.ies}\PYG{o}{;}
\PYG{k+kn}{import} \PYG{n+nn}{javax.xml.parsers.*}\PYG{o}{;}
\PYG{k+kn}{import} \PYG{n+nn}{javax.xml.transform.Transformer}\PYG{o}{;}
\PYG{k+kn}{import} \PYG{n+nn}{javax.xml.transform.TransformerFactory}\PYG{o}{;}
\PYG{k+kn}{import} \PYG{n+nn}{javax.xml.transform.dom.DOMSource}\PYG{o}{;}
\PYG{k+kn}{import} \PYG{n+nn}{javax.xml.transform.stream.StreamResult}\PYG{o}{;}

\PYG{k+kn}{import} \PYG{n+nn}{java.io.File}\PYG{o}{;}

\PYG{k+kn}{import} \PYG{n+nn}{org.w3c.dom.*}\PYG{o}{;}



\PYG{k+kd}{public} \PYG{k+kd}{class} \PYG{n+nc}{ProcesadorXML} \PYG{o}{\PYGZob{}}
        \PYG{k+kd}{public} \PYG{n}{Node} \PYG{n+nf}{extraerRaiz}\PYG{o}{(}\PYG{n}{String} \PYG{n}{nombreArchivo}\PYG{o}{)}\PYG{o}{\PYGZob{}}
                \PYG{n}{DocumentBuilderFactory} \PYG{n}{fabrica}\PYG{o}{;}
                \PYG{n}{DocumentBuilder} \PYG{n}{constructor}\PYG{o}{;}
                \PYG{n}{Document} \PYG{n}{documentoXML}\PYG{o}{=}\PYG{k+kc}{null}\PYG{o}{;}
                \PYG{n}{File} \PYG{n}{fichero}\PYG{o}{=}\PYG{k}{new} \PYG{n}{File}\PYG{o}{(}\PYG{n}{nombreArchivo}\PYG{o}{)}\PYG{o}{;}
                \PYG{n}{fabrica}\PYG{o}{=}
                        \PYG{n}{DocumentBuilderFactory}\PYG{o}{.}\PYG{n+na}{newInstance}\PYG{o}{(}\PYG{o}{)}\PYG{o}{;}
                \PYG{n}{System}\PYG{o}{.}\PYG{n+na}{out}\PYG{o}{.}\PYG{n+na}{println}\PYG{o}{(}\PYG{l+s}{\PYGZdq{}Procesando \PYGZdq{}}\PYG{o}{+}\PYG{n}{nombreArchivo}\PYG{o}{)}\PYG{o}{;}
                \PYG{k}{try} \PYG{o}{\PYGZob{}}
                        \PYG{n}{constructor}\PYG{o}{=}
                          \PYG{n}{fabrica}\PYG{o}{.}\PYG{n+na}{newDocumentBuilder}\PYG{o}{(}\PYG{o}{)}\PYG{o}{;}
                        \PYG{n}{documentoXML}\PYG{o}{=}\PYG{n}{constructor}\PYG{o}{.}\PYG{n+na}{parse}\PYG{o}{(}\PYG{n}{fichero}\PYG{o}{)}\PYG{o}{;}
                \PYG{o}{\PYGZcb{}} \PYG{k}{catch} \PYG{o}{(}\PYG{n}{Exception} \PYG{n}{e}\PYG{o}{)} \PYG{o}{\PYGZob{}}
                        \PYG{c+c1}{// TODO Auto\PYGZhy{}generated catch block}
                        \PYG{n}{e}\PYG{o}{.}\PYG{n+na}{printStackTrace}\PYG{o}{(}\PYG{o}{)}\PYG{o}{;}
                \PYG{o}{\PYGZcb{}}
                \PYG{k}{return} \PYG{n}{documentoXML}\PYG{o}{.}\PYG{n+na}{getDocumentElement}\PYG{o}{(}\PYG{o}{)}\PYG{o}{;}

        \PYG{o}{\PYGZcb{}}
        \PYG{k+kd}{public} \PYG{k+kt}{void} \PYG{n+nf}{imprimirNombreDeLaRaiz}\PYG{o}{(}\PYG{n}{Node} \PYG{n}{nodo}\PYG{o}{)}\PYG{o}{\PYGZob{}}
                \PYG{k}{if} \PYG{o}{(}\PYG{n}{nodo}\PYG{o}{!}\PYG{o}{=}\PYG{k+kc}{null}\PYG{o}{)}\PYG{o}{\PYGZob{}}
                        \PYG{n}{String} \PYG{n}{nombre}\PYG{o}{=}\PYG{n}{nodo}\PYG{o}{.}\PYG{n+na}{getNodeName}\PYG{o}{(}\PYG{o}{)}\PYG{o}{;}
                        \PYG{n}{System}\PYG{o}{.}\PYG{n+na}{out}\PYG{o}{.}\PYG{n+na}{println}\PYG{o}{(}\PYG{l+s}{\PYGZdq{}La raíz se llama:\PYGZdq{}}\PYG{o}{+}\PYG{n}{nombre}\PYG{o}{)}\PYG{o}{;}
                        \PYG{n}{Node} \PYG{n}{primerHijo}\PYG{o}{=}\PYG{n}{nodo}\PYG{o}{.}\PYG{n+na}{getFirstChild}\PYG{o}{(}\PYG{o}{)}\PYG{o}{;}
                        \PYG{n}{String} \PYG{n}{nombreHijo}\PYG{o}{=}\PYG{n}{primerHijo}\PYG{o}{.}\PYG{n+na}{getNodeName}\PYG{o}{(}\PYG{o}{)}\PYG{o}{;}
                        \PYG{n}{System}\PYG{o}{.}\PYG{n+na}{out}\PYG{o}{.}\PYG{n+na}{println}\PYG{o}{(}\PYG{l+s}{\PYGZdq{}El primer hijo se llama \PYGZlt{}\PYGZdq{}}\PYG{o}{+}\PYG{n}{nombreHijo}\PYG{o}{+}\PYG{l+s}{\PYGZdq{}\PYGZgt{}\PYGZdq{}}\PYG{o}{)}\PYG{o}{;}
                \PYG{o}{\PYGZcb{}} \PYG{k}{else} \PYG{o}{\PYGZob{}}
                        \PYG{n}{System}\PYG{o}{.}\PYG{n+na}{out}\PYG{o}{.}\PYG{n+na}{println}\PYG{o}{(}\PYG{l+s}{\PYGZdq{}No se pudo leer la raíz por ser nula\PYGZdq{}}\PYG{o}{)}\PYG{o}{;}
                \PYG{o}{\PYGZcb{}}
        \PYG{o}{\PYGZcb{}}
        \PYG{k+kd}{public} \PYG{k+kt}{void} \PYG{n+nf}{imprimirHijos}\PYG{o}{(}\PYG{n}{Node} \PYG{n}{nodoRaiz}\PYG{o}{)}\PYG{o}{\PYGZob{}}
                \PYG{k}{if} \PYG{o}{(}\PYG{n}{nodoRaiz}\PYG{o}{=}\PYG{o}{=}\PYG{k+kc}{null}\PYG{o}{)}\PYG{o}{\PYGZob{}}
                        \PYG{n}{System}\PYG{o}{.}\PYG{n+na}{out}\PYG{o}{.}\PYG{n+na}{println}\PYG{o}{(}\PYG{l+s}{\PYGZdq{}Imposible procesar raiz null\PYGZdq{}}\PYG{o}{)}\PYG{o}{;}
                        \PYG{k}{return} \PYG{o}{;}
                \PYG{o}{\PYGZcb{}}
                \PYG{n}{Node} \PYG{n}{nodo}\PYG{o}{=}\PYG{n}{nodoRaiz}\PYG{o}{.}\PYG{n+na}{getFirstChild}\PYG{o}{(}\PYG{o}{)}\PYG{o}{;}
                \PYG{k}{while} \PYG{o}{(}\PYG{n}{nodo}\PYG{o}{!}\PYG{o}{=}\PYG{k+kc}{null}\PYG{o}{)}\PYG{o}{\PYGZob{}}
                        \PYG{k+kt}{short} \PYG{n}{tipo}\PYG{o}{=}\PYG{n}{nodo}\PYG{o}{.}\PYG{n+na}{getNodeType}\PYG{o}{(}\PYG{o}{)}\PYG{o}{;}
                        \PYG{k}{if} \PYG{o}{(}\PYG{n}{tipo}\PYG{o}{=}\PYG{o}{=}\PYG{n}{nodo}\PYG{o}{.}\PYG{n+na}{ELEMENT\PYGZus{}NODE}\PYG{o}{)}\PYG{o}{\PYGZob{}}
                                \PYG{n}{System}\PYG{o}{.}\PYG{n+na}{out}\PYG{o}{.}\PYG{n+na}{println}\PYG{o}{(}\PYG{l+s}{\PYGZdq{}Nodo hijo:\PYGZdq{}}\PYG{o}{+}\PYG{n}{nodo}\PYG{o}{.}\PYG{n+na}{getNodeName}\PYG{o}{(}\PYG{o}{)}\PYG{o}{)}\PYG{o}{;}
                        \PYG{o}{\PYGZcb{}}
                        \PYG{n}{nodo}\PYG{o}{=}\PYG{n}{nodo}\PYG{o}{.}\PYG{n+na}{getNextSibling}\PYG{o}{(}\PYG{o}{)}\PYG{o}{;}
                \PYG{o}{\PYGZcb{}}
        \PYG{o}{\PYGZcb{}}

        \PYG{k+kd}{public} \PYG{k+kt}{void} \PYG{n+nf}{imprimirHijos2}\PYG{o}{(}\PYG{n}{Node} \PYG{n}{nodoRaiz}\PYG{o}{)}\PYG{o}{\PYGZob{}}
                \PYG{k}{if} \PYG{o}{(}\PYG{n}{nodoRaiz}\PYG{o}{=}\PYG{o}{=}\PYG{k+kc}{null}\PYG{o}{)}\PYG{o}{\PYGZob{}}
                        \PYG{n}{System}\PYG{o}{.}\PYG{n+na}{out}\PYG{o}{.}\PYG{n+na}{println}\PYG{o}{(}\PYG{l+s}{\PYGZdq{}Imposible procesar raíz nula\PYGZdq{}}\PYG{o}{)}\PYG{o}{;}
                        \PYG{k}{return}\PYG{o}{;}
                \PYG{o}{\PYGZcb{}} \PYG{c+c1}{//Fin del if null}
                \PYG{n}{NodeList} \PYG{n}{lista}\PYG{o}{=}\PYG{n}{nodoRaiz}\PYG{o}{.}\PYG{n+na}{getChildNodes}\PYG{o}{(}\PYG{o}{)}\PYG{o}{;}
                \PYG{k}{for} \PYG{o}{(}\PYG{k+kt}{int} \PYG{n}{i}\PYG{o}{=}\PYG{l+m+mi}{0}\PYG{o}{;} \PYG{n}{i}\PYG{o}{\PYGZlt{}}\PYG{n}{lista}\PYG{o}{.}\PYG{n+na}{getLength}\PYG{o}{(}\PYG{o}{)}\PYG{o}{;}\PYG{n}{i}\PYG{o}{+}\PYG{o}{+}\PYG{o}{)}\PYG{o}{\PYGZob{}}
                        \PYG{n}{Node} \PYG{n}{nodo}\PYG{o}{=}\PYG{n}{lista}\PYG{o}{.}\PYG{n+na}{item}\PYG{o}{(}\PYG{n}{i}\PYG{o}{)}\PYG{o}{;}
                        \PYG{k+kt}{short} \PYG{n}{tipo}\PYG{o}{=}\PYG{n}{nodo}\PYG{o}{.}\PYG{n+na}{getNodeType}\PYG{o}{(}\PYG{o}{)}\PYG{o}{;}
                        \PYG{k}{if} \PYG{o}{(}\PYG{n}{tipo}\PYG{o}{=}\PYG{o}{=}\PYG{n}{Node}\PYG{o}{.}\PYG{n+na}{ELEMENT\PYGZus{}NODE}\PYG{o}{)}\PYG{o}{\PYGZob{}}
                                \PYG{n}{System}\PYG{o}{.}\PYG{n+na}{out}\PYG{o}{.}\PYG{n+na}{println}\PYG{o}{(}\PYG{l+s}{\PYGZdq{}Hijo:\PYGZdq{}}\PYG{o}{+}\PYG{n}{nodo}\PYG{o}{.}\PYG{n+na}{getNodeName}\PYG{o}{(}\PYG{o}{)}\PYG{o}{)}\PYG{o}{;}
                        \PYG{o}{\PYGZcb{}} \PYG{c+c1}{//Fin del if}
                \PYG{o}{\PYGZcb{}} \PYG{c+c1}{//Fin del for}
        \PYG{o}{\PYGZcb{}} \PYG{c+c1}{//Fin del método}

        \PYG{k+kd}{public} \PYG{k+kt}{int} \PYG{n+nf}{contadorElementos}\PYG{o}{(}\PYG{n}{Node} \PYG{n}{raiz}\PYG{o}{,}\PYG{n}{String} \PYG{n}{nombreElemento}\PYG{o}{)}\PYG{o}{\PYGZob{}}
                \PYG{k+kt}{int} \PYG{n}{contador}\PYG{o}{=}\PYG{l+m+mi}{0}\PYG{o}{;}
                \PYG{n}{NodeList} \PYG{n}{nodosHijo}\PYG{o}{=}\PYG{n}{raiz}\PYG{o}{.}\PYG{n+na}{getChildNodes}\PYG{o}{(}\PYG{o}{)}\PYG{o}{;}
                \PYG{k}{for} \PYG{o}{(}\PYG{k+kt}{int} \PYG{n}{i}\PYG{o}{=}\PYG{l+m+mi}{0}\PYG{o}{;} \PYG{n}{i}\PYG{o}{\PYGZlt{}}\PYG{n}{nodosHijo}\PYG{o}{.}\PYG{n+na}{getLength}\PYG{o}{(}\PYG{o}{)}\PYG{o}{;}\PYG{n}{i}\PYG{o}{+}\PYG{o}{+}\PYG{o}{)}\PYG{o}{\PYGZob{}}
                        \PYG{n}{Node} \PYG{n}{nodo}\PYG{o}{=}\PYG{n}{nodosHijo}\PYG{o}{.}\PYG{n+na}{item}\PYG{o}{(}\PYG{n}{i}\PYG{o}{)}\PYG{o}{;}
                        \PYG{k+kt}{short} \PYG{n}{tipo}\PYG{o}{=}\PYG{n}{nodo}\PYG{o}{.}\PYG{n+na}{getNodeType}\PYG{o}{(}\PYG{o}{)}\PYG{o}{;}
                        \PYG{k}{if} \PYG{o}{(}\PYG{n}{tipo}\PYG{o}{=}\PYG{o}{=}\PYG{n}{Node}\PYG{o}{.}\PYG{n+na}{ELEMENT\PYGZus{}NODE}\PYG{o}{)}\PYG{o}{\PYGZob{}}
                                \PYG{n}{String} \PYG{n}{nombre}\PYG{o}{=}\PYG{n}{nodo}\PYG{o}{.}\PYG{n+na}{getNodeName}\PYG{o}{(}\PYG{o}{)}\PYG{o}{;}
                                \PYG{k}{if} \PYG{o}{(}\PYG{n}{nombre}\PYG{o}{=}\PYG{o}{=}\PYG{n}{nombreElemento}\PYG{o}{)}\PYG{o}{\PYGZob{}}
                                        \PYG{n}{contador}\PYG{o}{+}\PYG{o}{+}\PYG{o}{;}
                                \PYG{o}{\PYGZcb{}} \PYG{c+c1}{//Fin del if interno}
                        \PYG{o}{\PYGZcb{}} \PYG{c+c1}{//Fin del if externo}
                \PYG{o}{\PYGZcb{}} \PYG{c+c1}{//Fin del for}
                \PYG{k}{return} \PYG{n}{contador}\PYG{o}{;}
        \PYG{o}{\PYGZcb{}} \PYG{c+c1}{//Fin del método}
        \PYG{k+kd}{private} \PYG{k+kt}{void} \PYG{n+nf}{comprobarSiEsBono}\PYG{o}{(}\PYG{n}{Node} \PYG{n}{n}\PYG{o}{)}\PYG{o}{\PYGZob{}}
                \PYG{n}{String} \PYG{n}{nombre}\PYG{o}{=}\PYG{n}{n}\PYG{o}{.}\PYG{n+na}{getNodeName}\PYG{o}{(}\PYG{o}{)}\PYG{o}{;}
                \PYG{k}{if} \PYG{o}{(}\PYG{n}{nombre}\PYG{o}{=}\PYG{o}{=}\PYG{l+s}{\PYGZdq{}bono\PYGZdq{}}\PYG{o}{)}\PYG{o}{\PYGZob{}}
                        \PYG{n}{Element} \PYG{n}{e}\PYG{o}{=}\PYG{o}{(}\PYG{n}{Element}\PYG{o}{)} \PYG{n}{n}\PYG{o}{;}
                        \PYG{n}{String} \PYG{n}{precio}\PYG{o}{=}\PYG{n}{e}\PYG{o}{.}\PYG{n+na}{getAttribute}\PYG{o}{(}\PYG{l+s}{\PYGZdq{}precio\PYGZdq{}}\PYG{o}{)}\PYG{o}{;}
                        \PYG{n}{System}\PYG{o}{.}\PYG{n+na}{out}\PYG{o}{.}\PYG{n+na}{println}\PYG{o}{(}\PYG{l+s}{\PYGZdq{}Precio:\PYGZdq{}}\PYG{o}{+}\PYG{n}{precio}\PYG{o}{)}\PYG{o}{;}
                \PYG{o}{\PYGZcb{}}
        \PYG{o}{\PYGZcb{}}

        \PYG{k+kd}{public} \PYG{k+kt}{int} \PYG{n+nf}{cuantosInestables} \PYG{o}{(}\PYG{n}{Node} \PYG{n}{raiz}\PYG{o}{)}\PYG{o}{\PYGZob{}}
                \PYG{k+kt}{int} \PYG{n}{cuantos}\PYG{o}{=}\PYG{l+m+mi}{0}\PYG{o}{;}
                \PYG{n}{NodeList} \PYG{n}{lista}\PYG{o}{=}\PYG{n}{raiz}\PYG{o}{.}\PYG{n+na}{getChildNodes}\PYG{o}{(}\PYG{o}{)}\PYG{o}{;}
                \PYG{k}{for} \PYG{o}{(}\PYG{k+kt}{int} \PYG{n}{i}\PYG{o}{=}\PYG{l+m+mi}{0}\PYG{o}{;} \PYG{n}{i}\PYG{o}{\PYGZlt{}}\PYG{n}{lista}\PYG{o}{.}\PYG{n+na}{getLength}\PYG{o}{(}\PYG{o}{)}\PYG{o}{;} \PYG{n}{i}\PYG{o}{+}\PYG{o}{+}\PYG{o}{)}\PYG{o}{\PYGZob{}}
                        \PYG{n}{Node} \PYG{n}{n}\PYG{o}{=}\PYG{n}{lista}\PYG{o}{.}\PYG{n+na}{item}\PYG{o}{(}\PYG{n}{i}\PYG{o}{)}\PYG{o}{;}
                        \PYG{k}{if} \PYG{o}{(}\PYG{n}{n}\PYG{o}{.}\PYG{n+na}{getNodeType}\PYG{o}{(}\PYG{o}{)}\PYG{o}{!}\PYG{o}{=}\PYG{n}{Node}\PYG{o}{.}\PYG{n+na}{ELEMENT\PYGZus{}NODE}\PYG{o}{)} \PYG{k}{continue}\PYG{o}{;}
                        \PYG{n}{Element} \PYG{n}{e}\PYG{o}{=}\PYG{o}{(}\PYG{n}{Element}\PYG{o}{)} \PYG{n}{lista}\PYG{o}{.}\PYG{n+na}{item}\PYG{o}{(}\PYG{n}{i}\PYG{o}{)}\PYG{o}{;}
                        \PYG{k}{if} \PYG{o}{(}\PYG{n}{e}\PYG{o}{.}\PYG{n+na}{getNodeName}\PYG{o}{(}\PYG{o}{)}\PYG{o}{=}\PYG{o}{=}\PYG{l+s}{\PYGZdq{}divisa\PYGZdq{}} \PYG{o}{\textbar{}}\PYG{o}{\textbar{}}
                                        \PYG{n}{e}\PYG{o}{.}\PYG{n+na}{getNodeName}\PYG{o}{(}\PYG{o}{)}\PYG{o}{=}\PYG{o}{=}\PYG{l+s}{\PYGZdq{}bono\PYGZdq{}}\PYG{o}{)}\PYG{o}{\PYGZob{}}
                                \PYG{n}{String} \PYG{n}{atEstable}\PYG{o}{=}\PYG{n}{e}\PYG{o}{.}\PYG{n+na}{getAttribute}\PYG{o}{(}\PYG{l+s}{\PYGZdq{}estable\PYGZdq{}}\PYG{o}{)}\PYG{o}{;}
                                \PYG{k}{if} \PYG{o}{(}\PYG{n}{atEstable}\PYG{o}{!}\PYG{o}{=}\PYG{k+kc}{null}\PYG{o}{)}\PYG{o}{\PYGZob{}}
                                        \PYG{n}{System}\PYG{o}{.}\PYG{n+na}{out}\PYG{o}{.}\PYG{n+na}{println}\PYG{o}{(}\PYG{l+s}{\PYGZdq{}Atributo:\PYGZdq{}}\PYG{o}{+}\PYG{n}{atEstable}\PYG{o}{)}\PYG{o}{;}
                                        \PYG{k}{if} \PYG{o}{(}\PYG{n}{atEstable}\PYG{o}{.}\PYG{n+na}{equals}\PYG{o}{(}\PYG{l+s}{\PYGZdq{}no\PYGZdq{}}\PYG{o}{)}\PYG{o}{)}\PYG{o}{\PYGZob{}}
                                                \PYG{n}{cuantos}\PYG{o}{+}\PYG{o}{=}\PYG{l+m+mi}{1}\PYG{o}{;}
                                        \PYG{o}{\PYGZcb{}} \PYG{c+c1}{//Fin del if interno}
                                \PYG{o}{\PYGZcb{}} \PYG{c+c1}{//Fin del if atEstable}
                        \PYG{o}{\PYGZcb{}} \PYG{c+c1}{//Fin de if nodo es divisa o bono}
                \PYG{o}{\PYGZcb{}} \PYG{c+c1}{//Fin del for que recorre los nodos}
                \PYG{k}{return} \PYG{n}{cuantos}\PYG{o}{;}
        \PYG{o}{\PYGZcb{}} \PYG{c+c1}{//Fin del método cuantosInestables}

        \PYG{k+kd}{public} \PYG{k+kt}{float} \PYG{n+nf}{sumarAtributosPrecio}\PYG{o}{(}\PYG{n}{Node} \PYG{n}{raiz}\PYG{o}{)}\PYG{o}{\PYGZob{}}
                \PYG{k+kt}{float} \PYG{n}{precioTotal}\PYG{o}{=}\PYG{l+m+mi}{0}\PYG{o}{;}
                \PYG{n}{NodeList} \PYG{n}{hijos}\PYG{o}{=}\PYG{n}{raiz}\PYG{o}{.}\PYG{n+na}{getChildNodes}\PYG{o}{(}\PYG{o}{)}\PYG{o}{;}
                \PYG{k}{for} \PYG{o}{(}\PYG{k+kt}{int} \PYG{n}{i}\PYG{o}{=}\PYG{l+m+mi}{0}\PYG{o}{;} \PYG{n}{i}\PYG{o}{\PYGZlt{}}\PYG{n}{hijos}\PYG{o}{.}\PYG{n+na}{getLength}\PYG{o}{(}\PYG{o}{)}\PYG{o}{;} \PYG{n}{i}\PYG{o}{+}\PYG{o}{+}\PYG{o}{)}\PYG{o}{\PYGZob{}}
                        \PYG{n}{Node} \PYG{n}{hijo}\PYG{o}{=}\PYG{n}{hijos}\PYG{o}{.}\PYG{n+na}{item}\PYG{o}{(}\PYG{n}{i}\PYG{o}{)}\PYG{o}{;}
                        \PYG{k}{if} \PYG{o}{(}\PYG{n}{hijo}\PYG{o}{.}\PYG{n+na}{getNodeType}\PYG{o}{(}\PYG{o}{)}\PYG{o}{!}\PYG{o}{=}\PYG{n}{Node}\PYG{o}{.}\PYG{n+na}{ELEMENT\PYGZus{}NODE}\PYG{o}{)} \PYG{k}{continue}\PYG{o}{;}
                        \PYG{n}{Element} \PYG{n}{e}\PYG{o}{=}\PYG{o}{(}\PYG{n}{Element}\PYG{o}{)} \PYG{n}{hijo}\PYG{o}{;}
                        \PYG{n}{String} \PYG{n}{precio}\PYG{o}{=}\PYG{n}{e}\PYG{o}{.}\PYG{n+na}{getAttribute}\PYG{o}{(}\PYG{l+s}{\PYGZdq{}precio\PYGZdq{}}\PYG{o}{)}\PYG{o}{;}
                        \PYG{n}{Float} \PYG{n}{f}\PYG{o}{=}\PYG{n}{Float}\PYG{o}{.}\PYG{n+na}{parseFloat}\PYG{o}{(}\PYG{n}{precio}\PYG{o}{)}\PYG{o}{;}
                        \PYG{n}{precioTotal}\PYG{o}{+}\PYG{o}{=}\PYG{n}{f}\PYG{o}{;}
                \PYG{o}{\PYGZcb{}}
                \PYG{k}{return} \PYG{n}{precioTotal}\PYG{o}{;}
        \PYG{o}{\PYGZcb{}} \PYG{c+c1}{//Fin del método sumarAtributosPrecio}
        \PYG{k+kd}{public} \PYG{k+kt}{void} \PYG{n+nf}{imprimirPrecioBonos}\PYG{o}{(}\PYG{n}{Node} \PYG{n}{raiz}\PYG{o}{)}\PYG{o}{\PYGZob{}}
                \PYG{k}{if} \PYG{o}{(}\PYG{n}{raiz}\PYG{o}{=}\PYG{o}{=}\PYG{k+kc}{null}\PYG{o}{)}\PYG{o}{\PYGZob{}}
                        \PYG{n}{System}\PYG{o}{.}\PYG{n+na}{out}\PYG{o}{.}\PYG{n+na}{println}\PYG{o}{(}\PYG{l+s}{\PYGZdq{}Imposible procesar null\PYGZdq{}}\PYG{o}{)}\PYG{o}{;}
                        \PYG{k}{return}\PYG{o}{;}
                \PYG{o}{\PYGZcb{}}
                \PYG{n}{NodeList} \PYG{n}{nodos}\PYG{o}{=}\PYG{n}{raiz}\PYG{o}{.}\PYG{n+na}{getChildNodes}\PYG{o}{(}\PYG{o}{)}\PYG{o}{;}
                \PYG{k}{for} \PYG{o}{(}\PYG{k+kt}{int} \PYG{n}{i}\PYG{o}{=}\PYG{l+m+mi}{0}\PYG{o}{;} \PYG{n}{i}\PYG{o}{\PYGZlt{}}\PYG{n}{nodos}\PYG{o}{.}\PYG{n+na}{getLength}\PYG{o}{(}\PYG{o}{)}\PYG{o}{;} \PYG{n}{i}\PYG{o}{+}\PYG{o}{+}\PYG{o}{)}\PYG{o}{\PYGZob{}}
                        \PYG{n}{Node} \PYG{n}{nodo}\PYG{o}{=}\PYG{n}{nodos}\PYG{o}{.}\PYG{n+na}{item}\PYG{o}{(}\PYG{n}{i}\PYG{o}{)}\PYG{o}{;}
                        \PYG{k+kt}{short} \PYG{n}{tipo}\PYG{o}{=}\PYG{n}{nodo}\PYG{o}{.}\PYG{n+na}{getNodeType}\PYG{o}{(}\PYG{o}{)}\PYG{o}{;}
                        \PYG{k}{if} \PYG{o}{(}\PYG{n}{tipo}\PYG{o}{=}\PYG{o}{=}\PYG{n}{Node}\PYG{o}{.}\PYG{n+na}{ELEMENT\PYGZus{}NODE}\PYG{o}{)}\PYG{o}{\PYGZob{}}
                                \PYG{k}{this}\PYG{o}{.}\PYG{n+na}{comprobarSiEsBono}\PYG{o}{(}\PYG{n}{nodo}\PYG{o}{)}\PYG{o}{;}
                        \PYG{o}{\PYGZcb{}}
                \PYG{o}{\PYGZcb{}}
        \PYG{o}{\PYGZcb{}}
        \PYG{c+cm}{/**}
\PYG{c+cm}{         *}
\PYG{c+cm}{         * @param raiz}
\PYG{c+cm}{         * @param nombrePais}
\PYG{c+cm}{         * @return Numero de elementos dentro del nodo en}
\PYG{c+cm}{         * los cuales aparece de alguna forma el país}
\PYG{c+cm}{         */}
        \PYG{k+kd}{public} \PYG{k+kt}{int} \PYG{n+nf}{algoQueVerCon}\PYG{o}{(}\PYG{n}{Node} \PYG{n}{raiz}\PYG{o}{,} \PYG{n}{String} \PYG{n}{nombrePais}\PYG{o}{)}\PYG{o}{\PYGZob{}}
                \PYG{k+kt}{int} \PYG{n}{cuantos}\PYG{o}{=}\PYG{l+m+mi}{0}\PYG{o}{;}
                \PYG{n}{Element} \PYG{n}{elementoRaiz}\PYG{o}{=}\PYG{o}{(}\PYG{n}{Element}\PYG{o}{)} \PYG{n}{raiz}\PYG{o}{;}
                \PYG{n}{NodeList} \PYG{n}{lista}\PYG{o}{=}\PYG{n}{elementoRaiz}\PYG{o}{.}\PYG{n+na}{getElementsByTagName}\PYG{o}{(}\PYG{l+s}{\PYGZdq{}bono\PYGZdq{}}\PYG{o}{)}\PYG{o}{;}
                \PYG{k}{for} \PYG{o}{(}\PYG{k+kt}{int} \PYG{n}{i}\PYG{o}{=}\PYG{l+m+mi}{0}\PYG{o}{;} \PYG{n}{i}\PYG{o}{\PYGZlt{}}\PYG{n}{lista}\PYG{o}{.}\PYG{n+na}{getLength}\PYG{o}{(}\PYG{o}{)}\PYG{o}{;} \PYG{n}{i}\PYG{o}{+}\PYG{o}{+}\PYG{o}{)}\PYG{o}{\PYGZob{}}
                        \PYG{n}{Node} \PYG{n}{nodoBono}\PYG{o}{=}\PYG{n}{lista}\PYG{o}{.}\PYG{n+na}{item}\PYG{o}{(}\PYG{n}{i}\PYG{o}{)}\PYG{o}{;}
                        \PYG{n}{Node} \PYG{n}{primerHijoTexto}\PYG{o}{=}\PYG{n}{nodoBono}\PYG{o}{.}\PYG{n+na}{getFirstChild}\PYG{o}{(}\PYG{o}{)}\PYG{o}{;}
                        \PYG{n}{Node} \PYG{n}{segHijoPais}\PYG{o}{=}\PYG{n}{primerHijoTexto}\PYG{o}{.}\PYG{n+na}{getNextSibling}\PYG{o}{(}\PYG{o}{)}\PYG{o}{;}
                        \PYG{n}{String} \PYG{n}{paisExtraido}\PYG{o}{=}\PYG{n}{segHijoPais}\PYG{o}{.}\PYG{n+na}{getTextContent}\PYG{o}{(}\PYG{o}{)}\PYG{o}{;}
                        \PYG{c+c1}{//Limpiamos espacios}
                        \PYG{n}{paisExtraido}\PYG{o}{=}\PYG{n}{paisExtraido}\PYG{o}{.}\PYG{n+na}{trim}\PYG{o}{(}\PYG{o}{)}\PYG{o}{;}
                        \PYG{n}{System}\PYG{o}{.}\PYG{n+na}{out}\PYG{o}{.}\PYG{n+na}{println}\PYG{o}{(}\PYG{l+s}{\PYGZdq{}Pais extraido:\PYGZdq{}}\PYG{o}{+}\PYG{n}{paisExtraido}\PYG{o}{)}\PYG{o}{;}
                        \PYG{k}{if} \PYG{o}{(}\PYG{n}{paisExtraido}\PYG{o}{.}\PYG{n+na}{equals}\PYG{o}{(}\PYG{n}{nombrePais}\PYG{o}{)}\PYG{o}{)}\PYG{o}{\PYGZob{}}
                                \PYG{n}{cuantos}\PYG{o}{+}\PYG{o}{+}\PYG{o}{;}
                        \PYG{o}{\PYGZcb{}}
                \PYG{o}{\PYGZcb{}}
                \PYG{k}{return} \PYG{n}{cuantos}\PYG{o}{;}
        \PYG{o}{\PYGZcb{}}

        \PYG{c+cm}{/**}
\PYG{c+cm}{         * Este método averigua cuantos elementos futuro}
\PYG{c+cm}{         * tienen una cierta ciudad procedencia}
\PYG{c+cm}{         * @param argumentos}
\PYG{c+cm}{         */}
        \PYG{k+kd}{public} \PYG{k+kt}{int} \PYG{n+nf}{cuantosFuturosTienenCiudadProcedencia}\PYG{o}{(}
                        \PYG{n}{Node} \PYG{n}{raiz}\PYG{o}{,} \PYG{n}{String} \PYG{n}{ciudad}
                        \PYG{o}{)}
        \PYG{o}{\PYGZob{}}
                \PYG{k+kt}{int} \PYG{n}{cuantos}\PYG{o}{=}\PYG{l+m+mi}{0}\PYG{o}{;}
                \PYG{n}{Element} \PYG{n}{nodoRaiz}\PYG{o}{=}\PYG{o}{(}\PYG{n}{Element}\PYG{o}{)} \PYG{n}{raiz}\PYG{o}{;}
                \PYG{n}{NodeList} \PYG{n}{lista}\PYG{o}{=}\PYG{n}{nodoRaiz}\PYG{o}{.}\PYG{n+na}{getElementsByTagName}\PYG{o}{(}\PYG{l+s}{\PYGZdq{}futuro\PYGZdq{}}\PYG{o}{)}\PYG{o}{;}
                \PYG{k}{for} \PYG{o}{(}\PYG{k+kt}{int} \PYG{n}{i}\PYG{o}{=}\PYG{l+m+mi}{0}\PYG{o}{;} \PYG{n}{i}\PYG{o}{\PYGZlt{}}\PYG{n}{lista}\PYG{o}{.}\PYG{n+na}{getLength}\PYG{o}{(}\PYG{o}{)}\PYG{o}{;} \PYG{n}{i}\PYG{o}{+}\PYG{o}{+}\PYG{o}{)}\PYG{o}{\PYGZob{}}
                        \PYG{n}{Element} \PYG{n}{e}\PYG{o}{=}\PYG{o}{(}\PYG{n}{Element}\PYG{o}{)}\PYG{n}{lista}\PYG{o}{.}\PYG{n+na}{item}\PYG{o}{(}\PYG{n}{i}\PYG{o}{)}\PYG{o}{;}
                        \PYG{n}{NodeList} \PYG{n}{listaHijos}\PYG{o}{=}\PYG{n}{e}\PYG{o}{.}\PYG{n+na}{getChildNodes}\PYG{o}{(}\PYG{o}{)}\PYG{o}{;}
                        \PYG{c+c1}{//El elemento ciudad procedencia es el quinto hijo}
                        \PYG{n}{Node} \PYG{n}{nodoCiudad}\PYG{o}{=}\PYG{n}{listaHijos}\PYG{o}{.}\PYG{n+na}{item}\PYG{o}{(}\PYG{l+m+mi}{5}\PYG{o}{)}\PYG{o}{;}
                        \PYG{n}{NodeList} \PYG{n}{hijosCiudad}\PYG{o}{=}\PYG{n}{nodoCiudad}\PYG{o}{.}\PYG{n+na}{getChildNodes}\PYG{o}{(}\PYG{o}{)}\PYG{o}{;}
                        \PYG{n}{Node} \PYG{n}{nodoElemCiudad}\PYG{o}{=}\PYG{n}{hijosCiudad}\PYG{o}{.}\PYG{n+na}{item}\PYG{o}{(}\PYG{l+m+mi}{1}\PYG{o}{)}\PYG{o}{;}
                        \PYG{n}{String} \PYG{n}{nombreCiudad}\PYG{o}{=}\PYG{n}{nodoElemCiudad}\PYG{o}{.}\PYG{n+na}{getNodeName}\PYG{o}{(}\PYG{o}{)}\PYG{o}{;}
                        \PYG{k}{if} \PYG{o}{(}\PYG{n}{nombreCiudad}\PYG{o}{.}\PYG{n+na}{equals}\PYG{o}{(}\PYG{n}{ciudad}\PYG{o}{)}\PYG{o}{)}\PYG{o}{\PYGZob{}}
                                \PYG{n}{cuantos}\PYG{o}{+}\PYG{o}{+}\PYG{o}{;}
                        \PYG{o}{\PYGZcb{}} \PYG{c+c1}{//Fin del if}
                \PYG{o}{\PYGZcb{}} \PYG{c+c1}{//Fin del for}
                \PYG{k}{return} \PYG{n}{cuantos}\PYG{o}{;}
        \PYG{o}{\PYGZcb{}}
        \PYG{c+cm}{/**}
\PYG{c+cm}{         *}
\PYG{c+cm}{         * @param raiz Raíz del documento}
\PYG{c+cm}{         * @param nombrePais (Debe ser \PYGZdq{}espania\PYGZdq{} para España)}
\PYG{c+cm}{         * @return}
\PYG{c+cm}{         */}
        \PYG{k+kd}{public} \PYG{k+kt}{int} \PYG{n+nf}{letrasConPaisEmisor}\PYG{o}{(}\PYG{n}{Node} \PYG{n}{raiz}\PYG{o}{,} \PYG{n}{String} \PYG{n}{nombrePais}\PYG{o}{)}\PYG{o}{\PYGZob{}}
                \PYG{k+kt}{int} \PYG{n}{cuantos}\PYG{o}{=}\PYG{l+m+mi}{0}\PYG{o}{;}
                \PYG{n}{Element} \PYG{n}{eRaiz}\PYG{o}{=}\PYG{o}{(}\PYG{n}{Element}\PYG{o}{)} \PYG{n}{raiz}\PYG{o}{;}
                \PYG{n}{NodeList} \PYG{n}{listaLetras}\PYG{o}{=}\PYG{n}{eRaiz}\PYG{o}{.}\PYG{n+na}{getElementsByTagName}\PYG{o}{(}\PYG{l+s}{\PYGZdq{}letra\PYGZdq{}}\PYG{o}{)}\PYG{o}{;}
                \PYG{k}{for} \PYG{o}{(}\PYG{k+kt}{int} \PYG{n}{i}\PYG{o}{=}\PYG{l+m+mi}{0}\PYG{o}{;} \PYG{n}{i}\PYG{o}{\PYGZlt{}}\PYG{n}{listaLetras}\PYG{o}{.}\PYG{n+na}{getLength}\PYG{o}{(}\PYG{o}{)}\PYG{o}{;}\PYG{n}{i}\PYG{o}{+}\PYG{o}{+}\PYG{o}{)}\PYG{o}{\PYGZob{}}
                        \PYG{n}{Node} \PYG{n}{nodo}\PYG{o}{=}\PYG{n}{listaLetras}\PYG{o}{.}\PYG{n+na}{item}\PYG{o}{(}\PYG{n}{i}\PYG{o}{)}\PYG{o}{;}
                        \PYG{n}{Element} \PYG{n}{eNodo}\PYG{o}{=}\PYG{o}{(}\PYG{n}{Element}\PYG{o}{)} \PYG{n}{nodo}\PYG{o}{;} \PYG{c+c1}{//Devuelve elemento letra}
                        \PYG{n}{NodeList} \PYG{n}{hijosLetra}\PYG{o}{=}\PYG{n}{eNodo}\PYG{o}{.}\PYG{n+na}{getChildNodes}\PYG{o}{(}\PYG{o}{)}\PYG{o}{;}
                        \PYG{n}{Node} \PYG{n}{nodoPaisEmisor}\PYG{o}{=}\PYG{n}{hijosLetra}\PYG{o}{.}\PYG{n+na}{item}\PYG{o}{(}\PYG{l+m+mi}{3}\PYG{o}{)}\PYG{o}{;}
                        \PYG{n}{NodeList} \PYG{n}{hijosPais}\PYG{o}{=}\PYG{n}{nodoPaisEmisor}\PYG{o}{.}\PYG{n+na}{getChildNodes}\PYG{o}{(}\PYG{o}{)}\PYG{o}{;}
                        \PYG{n}{Node} \PYG{n}{nodoPais}\PYG{o}{=}\PYG{n}{hijosPais}\PYG{o}{.}\PYG{n+na}{item}\PYG{o}{(}\PYG{l+m+mi}{1}\PYG{o}{)}\PYG{o}{;}
                        \PYG{n}{String} \PYG{n}{nombreNodoPais}\PYG{o}{=}\PYG{n}{nodoPais}\PYG{o}{.}\PYG{n+na}{getNodeName}\PYG{o}{(}\PYG{o}{)}\PYG{o}{;}
                        \PYG{k}{if} \PYG{o}{(}\PYG{n}{nombreNodoPais}\PYG{o}{.}\PYG{n+na}{equals}\PYG{o}{(}\PYG{n}{nombrePais}\PYG{o}{)}\PYG{o}{)}\PYG{o}{\PYGZob{}}
                                \PYG{n}{cuantos}\PYG{o}{+}\PYG{o}{+}\PYG{o}{;}
                        \PYG{o}{\PYGZcb{}}
                \PYG{o}{\PYGZcb{}} \PYG{c+c1}{//Fin del for}
                \PYG{k}{return} \PYG{n}{cuantos}\PYG{o}{;}
        \PYG{o}{\PYGZcb{}}

        \PYG{k+kd}{public} \PYG{k+kt}{int} \PYG{n+nf}{algoQueVerConEspania}\PYG{o}{(}\PYG{n}{Node} \PYG{n}{raiz}\PYG{o}{)}\PYG{o}{\PYGZob{}}
                \PYG{k+kt}{int} \PYG{n}{cuantasLetras}\PYG{o}{=}\PYG{k}{this}\PYG{o}{.}\PYG{n+na}{letrasConPaisEmisor}\PYG{o}{(}\PYG{n}{raiz}\PYG{o}{,} \PYG{l+s}{\PYGZdq{}espania\PYGZdq{}}\PYG{o}{)}\PYG{o}{;}
                \PYG{k+kt}{int} \PYG{n}{cuantosFuturos}\PYG{o}{=}\PYG{k}{this}\PYG{o}{.}\PYG{n+na}{cuantosFuturosTienenCiudadProcedencia}\PYG{o}{(}\PYG{n}{raiz}\PYG{o}{,}
                                \PYG{l+s}{\PYGZdq{}madrid\PYGZdq{}}\PYG{o}{)}\PYG{o}{;}
                \PYG{k+kt}{int} \PYG{n}{cuantosBonos}\PYG{o}{=}\PYG{k}{this}\PYG{o}{.}\PYG{n+na}{algoQueVerCon}\PYG{o}{(}\PYG{n}{raiz}\PYG{o}{,} \PYG{l+s}{\PYGZdq{}España\PYGZdq{}}\PYG{o}{)}\PYG{o}{;}
                \PYG{k}{return} \PYG{n}{cuantasLetras}\PYG{o}{+}\PYG{n}{cuantosFuturos}\PYG{o}{+}\PYG{n}{cuantosBonos}\PYG{o}{;}
        \PYG{o}{\PYGZcb{}}
        \PYG{k+kd}{public} \PYG{k+kt}{int} \PYG{n+nf}{extraerHijoNumero} \PYG{o}{(}\PYG{n}{Element} \PYG{n}{padre}\PYG{o}{,}
                        \PYG{n}{String} \PYG{n}{nombreHijo}\PYG{o}{)}\PYG{o}{\PYGZob{}}
                \PYG{k+kt}{int} \PYG{n}{valor}\PYG{o}{=}\PYG{l+m+mi}{0}\PYG{o}{;}
                \PYG{c+c1}{//Esta lista tiene solo un elemento}
                \PYG{n}{NodeList} \PYG{n}{listaHijos}\PYG{o}{=}
                                \PYG{n}{padre}\PYG{o}{.}\PYG{n+na}{getElementsByTagName}\PYG{o}{(}\PYG{n}{nombreHijo}\PYG{o}{)}\PYG{o}{;}
                \PYG{n}{Element} \PYG{n}{hijoNumerico}\PYG{o}{=}\PYG{o}{(}\PYG{n}{Element}\PYG{o}{)} \PYG{n}{listaHijos}\PYG{o}{.}\PYG{n+na}{item}\PYG{o}{(}\PYG{l+m+mi}{0}\PYG{o}{)}\PYG{o}{;}
                \PYG{n}{String} \PYG{n}{contenidoTextual}\PYG{o}{=}\PYG{n}{hijoNumerico}\PYG{o}{.}\PYG{n+na}{getTextContent}\PYG{o}{(}\PYG{o}{)}\PYG{o}{;}
                \PYG{n}{valor}\PYG{o}{=}\PYG{n}{Integer}\PYG{o}{.}\PYG{n+na}{parseInt}\PYG{o}{(}\PYG{n}{contenidoTextual}\PYG{o}{)}\PYG{o}{;}
                \PYG{k}{return} \PYG{n}{valor}\PYG{o}{;}
        \PYG{o}{\PYGZcb{}}
        \PYG{k+kd}{public} \PYG{k+kt}{void} \PYG{n+nf}{imprimirBonos}\PYG{o}{(}\PYG{n}{Node} \PYG{n}{raiz}\PYG{o}{)}\PYG{o}{\PYGZob{}}

                \PYG{n}{Element} \PYG{n}{eRaiz}\PYG{o}{=}\PYG{o}{(}\PYG{n}{Element}\PYG{o}{)} \PYG{n}{raiz}\PYG{o}{;}
                \PYG{n}{NodeList} \PYG{n}{listaBonos}\PYG{o}{=}\PYG{n}{eRaiz}\PYG{o}{.}\PYG{n+na}{getElementsByTagName}\PYG{o}{(}\PYG{l+s}{\PYGZdq{}bono\PYGZdq{}}\PYG{o}{)}\PYG{o}{;}
                \PYG{k}{for} \PYG{o}{(}\PYG{k+kt}{int} \PYG{n}{i}\PYG{o}{=}\PYG{l+m+mi}{0}\PYG{o}{;} \PYG{n}{i}\PYG{o}{\PYGZlt{}}\PYG{n}{listaBonos}\PYG{o}{.}\PYG{n+na}{getLength}\PYG{o}{(}\PYG{o}{)}\PYG{o}{;} \PYG{n}{i}\PYG{o}{+}\PYG{o}{+}\PYG{o}{)}\PYG{o}{\PYGZob{}}
                        \PYG{n}{Node} \PYG{n}{bono}\PYG{o}{=}\PYG{n}{listaBonos}\PYG{o}{.}\PYG{n+na}{item}\PYG{o}{(}\PYG{n}{i}\PYG{o}{)}\PYG{o}{;}
                        \PYG{n}{Element} \PYG{n}{eBono}\PYG{o}{=}\PYG{o}{(}\PYG{n}{Element}\PYG{o}{)} \PYG{n}{listaBonos}\PYG{o}{.}\PYG{n+na}{item}\PYG{o}{(}\PYG{n}{i}\PYG{o}{)}\PYG{o}{;}

                        \PYG{k+kt}{int} \PYG{n}{valorDeseado}\PYG{o}{=}\PYG{k}{this}\PYG{o}{.}\PYG{n+na}{extraerHijoNumero}\PYG{o}{(}
                                        \PYG{n}{eBono}\PYG{o}{,} \PYG{l+s}{\PYGZdq{}valor\PYGZus{}deseado\PYGZdq{}}\PYG{o}{)}\PYG{o}{;}
                        \PYG{k+kt}{int} \PYG{n}{valorMinimo}\PYG{o}{=}\PYG{k}{this}\PYG{o}{.}\PYG{n+na}{extraerHijoNumero}\PYG{o}{(}
                                        \PYG{n}{eBono}\PYG{o}{,} \PYG{l+s}{\PYGZdq{}valor\PYGZus{}minimo\PYGZdq{}}\PYG{o}{)}\PYG{o}{;}
                        \PYG{k+kt}{int} \PYG{n}{valorMaximo}\PYG{o}{=}\PYG{k}{this}\PYG{o}{.}\PYG{n+na}{extraerHijoNumero}\PYG{o}{(}
                                        \PYG{n}{eBono}\PYG{o}{,} \PYG{l+s}{\PYGZdq{}valor\PYGZus{}maximo\PYGZdq{}}\PYG{o}{)}\PYG{o}{;}

                        \PYG{k}{if} \PYG{o}{(}\PYG{o}{(}\PYG{n}{valorDeseado}\PYG{o}{\PYGZgt{}}\PYG{n}{valorMinimo}\PYG{o}{)} \PYG{o}{\PYGZam{}}\PYG{o}{\PYGZam{}}
                                \PYG{o}{(}\PYG{n}{valorDeseado}\PYG{o}{\PYGZlt{}}\PYG{n}{valorMaximo}\PYG{o}{)} \PYG{o}{)}\PYG{o}{\PYGZob{}}
                                \PYG{n}{System}\PYG{o}{.}\PYG{n+na}{out}\PYG{o}{.}\PYG{n+na}{println}\PYG{o}{(}\PYG{l+s}{\PYGZdq{}Encontrado un bono!\PYGZdq{}}\PYG{o}{)}\PYG{o}{;}
                                \PYG{n}{String} \PYG{n}{ciudad}\PYG{o}{=}\PYG{k}{this}\PYG{o}{.}\PYG{n+na}{getCiudadProcedencia}\PYG{o}{(}\PYG{n}{eBono}\PYG{o}{)}\PYG{o}{;}
                                \PYG{n}{System}\PYG{o}{.}\PYG{n+na}{out}\PYG{o}{.}\PYG{n+na}{println}\PYG{o}{(}\PYG{l+s}{\PYGZdq{}Su ciudad era:\PYGZdq{}}\PYG{o}{+}\PYG{n}{ciudad}\PYG{o}{)}\PYG{o}{;}
                        \PYG{o}{\PYGZcb{}}
                        \PYG{c+c1}{//Element eValorDeseado=(Element)}
                        \PYG{c+c1}{//              listaParaValorDeseado.item(0);}

                \PYG{o}{\PYGZcb{}}

        \PYG{o}{\PYGZcb{}}
        \PYG{k+kd}{public} \PYG{n}{String} \PYG{n+nf}{getCiudadProcedencia}\PYG{o}{(}\PYG{n}{Element} \PYG{n}{e}\PYG{o}{)}\PYG{o}{\PYGZob{}}
                \PYG{n}{NodeList} \PYG{n}{listaHijos}\PYG{o}{=}
                                \PYG{n}{e}\PYG{o}{.}\PYG{n+na}{getElementsByTagName}\PYG{o}{(}\PYG{l+s}{\PYGZdq{}ciudad\PYGZus{}procedencia\PYGZdq{}}\PYG{o}{)}\PYG{o}{;}
                \PYG{n}{Element} \PYG{n}{eCiudad}\PYG{o}{=}\PYG{o}{(}\PYG{n}{Element}\PYG{o}{)} \PYG{n}{listaHijos}\PYG{o}{.}\PYG{n+na}{item}\PYG{o}{(}\PYG{l+m+mi}{0}\PYG{o}{)}\PYG{o}{;}
                \PYG{n}{NodeList} \PYG{n}{listaHijosCiudad}\PYG{o}{=}\PYG{n}{eCiudad}\PYG{o}{.}\PYG{n+na}{getChildNodes}\PYG{o}{(}\PYG{o}{)}\PYG{o}{;}
                \PYG{n}{Element} \PYG{n}{eCiudadConcreto}\PYG{o}{=}
                                \PYG{o}{(}\PYG{n}{Element}\PYG{o}{)} \PYG{n}{listaHijosCiudad}\PYG{o}{.}\PYG{n+na}{item}\PYG{o}{(}\PYG{l+m+mi}{1}\PYG{o}{)}\PYG{o}{;}
                \PYG{n}{String} \PYG{n}{nombre}\PYG{o}{=}\PYG{n}{eCiudadConcreto}\PYG{o}{.}\PYG{n+na}{getNodeName}\PYG{o}{(}\PYG{o}{)}\PYG{o}{;}
                \PYG{k}{return} \PYG{n}{nombre}\PYG{o}{;}
        \PYG{o}{\PYGZcb{}}

        \PYG{k+kd}{public} \PYG{k+kt}{void} \PYG{n+nf}{imprimirMismaCiudad}\PYG{o}{(}\PYG{n}{Node} \PYG{n}{raiz}\PYG{o}{)}\PYG{o}{\PYGZob{}}
                \PYG{n}{NodeList} \PYG{n}{hijos}\PYG{o}{=}\PYG{n}{raiz}\PYG{o}{.}\PYG{n+na}{getChildNodes}\PYG{o}{(}\PYG{o}{)}\PYG{o}{;}
                \PYG{k}{for} \PYG{o}{(}\PYG{k+kt}{int} \PYG{n}{i}\PYG{o}{=}\PYG{l+m+mi}{0}\PYG{o}{;} \PYG{n}{i}\PYG{o}{\PYGZlt{}}\PYG{n}{hijos}\PYG{o}{.}\PYG{n+na}{getLength}\PYG{o}{(}\PYG{o}{)}\PYG{o}{;} \PYG{n}{i}\PYG{o}{+}\PYG{o}{+}\PYG{o}{)}\PYG{o}{\PYGZob{}}
                        \PYG{n}{Node} \PYG{n}{hijo}\PYG{o}{=}\PYG{n}{hijos}\PYG{o}{.}\PYG{n+na}{item}\PYG{o}{(}\PYG{n}{i}\PYG{o}{)}\PYG{o}{;}
                        \PYG{k}{if} \PYG{o}{(}\PYG{n}{hijo}\PYG{o}{.}\PYG{n+na}{getNodeType}\PYG{o}{(}\PYG{o}{)}\PYG{o}{!}\PYG{o}{=}\PYG{n}{Node}\PYG{o}{.}\PYG{n+na}{ELEMENT\PYGZus{}NODE}\PYG{o}{)}\PYG{o}{\PYGZob{}}
                                \PYG{k}{continue}\PYG{o}{;}
                        \PYG{o}{\PYGZcb{}}
                        \PYG{n}{String} \PYG{n}{ciudadHijo}\PYG{o}{=}
                                        \PYG{k}{this}\PYG{o}{.}\PYG{n+na}{getCiudadProcedencia}\PYG{o}{(}\PYG{o}{(}\PYG{n}{Element}\PYG{o}{)}\PYG{n}{hijo}\PYG{o}{)}\PYG{o}{;}
                        \PYG{k}{for} \PYG{o}{(}\PYG{k+kt}{int} \PYG{n}{j}\PYG{o}{=}\PYG{n}{i}\PYG{o}{+}\PYG{l+m+mi}{1}\PYG{o}{;} \PYG{n}{j}\PYG{o}{\PYGZlt{}}\PYG{n}{hijos}\PYG{o}{.}\PYG{n+na}{getLength}\PYG{o}{(}\PYG{o}{)}\PYG{o}{;} \PYG{n}{j}\PYG{o}{+}\PYG{o}{+}\PYG{o}{)}\PYG{o}{\PYGZob{}}
                                \PYG{n}{Node} \PYG{n}{otroHijo}\PYG{o}{=}\PYG{n}{hijos}\PYG{o}{.}\PYG{n+na}{item}\PYG{o}{(}\PYG{n}{j}\PYG{o}{)}\PYG{o}{;}
                                \PYG{k}{if} \PYG{o}{(}\PYG{n}{otroHijo}\PYG{o}{.}\PYG{n+na}{getNodeType}\PYG{o}{(}\PYG{o}{)}\PYG{o}{!}\PYG{o}{=}\PYG{n}{Node}\PYG{o}{.}\PYG{n+na}{ELEMENT\PYGZus{}NODE}\PYG{o}{)}\PYG{o}{\PYGZob{}}
                                        \PYG{k}{continue}\PYG{o}{;}
                                \PYG{o}{\PYGZcb{}}

                                \PYG{n}{String} \PYG{n}{ciudadOtro}\PYG{o}{=}
                                                \PYG{k}{this}\PYG{o}{.}\PYG{n+na}{getCiudadProcedencia}\PYG{o}{(}\PYG{o}{(}\PYG{n}{Element}\PYG{o}{)}\PYG{n}{otroHijo}\PYG{o}{)}\PYG{o}{;}
                                \PYG{k}{if} \PYG{o}{(}\PYG{n}{ciudadHijo}\PYG{o}{.}\PYG{n+na}{equals}\PYG{o}{(}\PYG{n}{ciudadOtro}\PYG{o}{)}\PYG{o}{)}\PYG{o}{\PYGZob{}}
                                        \PYG{n}{System}\PYG{o}{.}\PYG{n+na}{out}\PYG{o}{.}\PYG{n+na}{println}\PYG{o}{(}
                                                        \PYG{l+s}{\PYGZdq{}Encontré dos elementos con la ciudad \PYGZdq{}}\PYG{o}{+}\PYG{n}{ciudadHijo}\PYG{o}{)}\PYG{o}{;}

                                \PYG{o}{\PYGZcb{}} \PYG{c+c1}{//Fin del if ciudadHijo}
                        \PYG{o}{\PYGZcb{}} \PYG{c+c1}{//Fin del for interno}
                \PYG{o}{\PYGZcb{}} \PYG{c+c1}{//Fin del for externo}
        \PYG{o}{\PYGZcb{}} \PYG{c+c1}{//Fin del método}

        \PYG{k+kd}{public} \PYG{n}{String}\PYG{o}{[}\PYG{o}{]} \PYG{n+nf}{obtenerListaBonos}\PYG{o}{(}\PYG{n}{Node} \PYG{n}{raiz}\PYG{o}{)}\PYG{o}{\PYGZob{}}
                \PYG{k+kt}{int} \PYG{n}{tamanioMaximo}\PYG{o}{=}\PYG{l+m+mi}{1000}\PYG{o}{;}
                \PYG{n}{String}\PYG{o}{[}\PYG{o}{]} \PYG{n}{ciudades}\PYG{o}{=}\PYG{k}{new} \PYG{n}{String}\PYG{o}{[}\PYG{n}{tamanioMaximo}\PYG{o}{]}\PYG{o}{;}
                \PYG{n}{String}\PYG{o}{[}\PYG{o}{]} \PYG{n}{aux}\PYG{o}{=}\PYG{k}{new} \PYG{n}{String}\PYG{o}{[}\PYG{n}{tamanioMaximo}\PYG{o}{]}\PYG{o}{;}
                \PYG{k+kt}{int} \PYG{n}{posPrecio}\PYG{o}{=}\PYG{l+m+mi}{0}\PYG{o}{;}
                \PYG{n}{Element} \PYG{n}{eRaiz}\PYG{o}{=}\PYG{o}{(}\PYG{n}{Element}\PYG{o}{)} \PYG{n}{raiz}\PYG{o}{;}
                \PYG{n}{NodeList} \PYG{n}{listaBonos}\PYG{o}{=}\PYG{n}{eRaiz}\PYG{o}{.}\PYG{n+na}{getElementsByTagName}\PYG{o}{(}\PYG{l+s}{\PYGZdq{}bono\PYGZdq{}}\PYG{o}{)}\PYG{o}{;}
                \PYG{k}{for} \PYG{o}{(}\PYG{k+kt}{int} \PYG{n}{i}\PYG{o}{=}\PYG{l+m+mi}{0}\PYG{o}{;} \PYG{n}{i}\PYG{o}{\PYGZlt{}}\PYG{n}{listaBonos}\PYG{o}{.}\PYG{n+na}{getLength}\PYG{o}{(}\PYG{o}{)}\PYG{o}{;} \PYG{n}{i}\PYG{o}{+}\PYG{o}{+}\PYG{o}{)}\PYG{o}{\PYGZob{}}
                        \PYG{n}{Element} \PYG{n}{bono}\PYG{o}{=}\PYG{o}{(}\PYG{n}{Element}\PYG{o}{)} \PYG{n}{listaBonos}\PYG{o}{.}\PYG{n+na}{item}\PYG{o}{(}\PYG{n}{i}\PYG{o}{)}\PYG{o}{;}
                        \PYG{n}{String} \PYG{n}{atEstable}\PYG{o}{=}\PYG{n}{bono}\PYG{o}{.}\PYG{n+na}{getAttribute}\PYG{o}{(}\PYG{l+s}{\PYGZdq{}estable\PYGZdq{}}\PYG{o}{)}\PYG{o}{;}
                        \PYG{k}{if} \PYG{o}{(}\PYG{n}{atEstable}\PYG{o}{!}\PYG{o}{=}\PYG{k+kc}{null}\PYG{o}{)}\PYG{o}{\PYGZob{}}
                                \PYG{k}{if} \PYG{o}{(}\PYG{n}{atEstable}\PYG{o}{.}\PYG{n+na}{equals}\PYG{o}{(}\PYG{l+s}{\PYGZdq{}no\PYGZdq{}}\PYG{o}{)}\PYG{o}{)}\PYG{o}{\PYGZob{}}
                                        \PYG{c+c1}{//Examinamos la ciudad aprovechando}
                                        \PYG{c+c1}{//un método ya construido.}
                                        \PYG{n}{String} \PYG{n}{ciudad}\PYG{o}{=}\PYG{k}{this}\PYG{o}{.}\PYG{n+na}{getCiudadProcedencia}\PYG{o}{(}\PYG{n}{bono}\PYG{o}{)}\PYG{o}{;}
                                        \PYG{k}{if} \PYG{o}{(}\PYG{n}{ciudad}\PYG{o}{.}\PYG{n+na}{equals}\PYG{o}{(}\PYG{l+s}{\PYGZdq{}tokio\PYGZdq{}}\PYG{o}{)}\PYG{o}{)}\PYG{o}{\PYGZob{}}
                                                \PYG{c+c1}{//Si es de tokio, copiamos el precio}
                                                \PYG{n}{String} \PYG{n}{precio}\PYG{o}{=}\PYG{n}{bono}\PYG{o}{.}\PYG{n+na}{getAttribute}\PYG{o}{(}\PYG{l+s}{\PYGZdq{}precio\PYGZdq{}}\PYG{o}{)}\PYG{o}{;}
                                                \PYG{n}{aux}\PYG{o}{[}\PYG{n}{posPrecio}\PYG{o}{]}\PYG{o}{=}\PYG{n}{precio}\PYG{o}{;}
                                                \PYG{n}{posPrecio}\PYG{o}{+}\PYG{o}{+}\PYG{o}{;}
                                        \PYG{o}{\PYGZcb{}} \PYG{c+c1}{//Fin del if para tokio}
                                \PYG{o}{\PYGZcb{}} \PYG{c+c1}{//Fin del if para el \PYGZdq{}no\PYGZdq{}}
                        \PYG{o}{\PYGZcb{}} \PYG{c+c1}{//Fin del if para atEstable}
                \PYG{o}{\PYGZcb{}} \PYG{c+c1}{//Fin del for que recorre los bonos}
                \PYG{k}{if} \PYG{o}{(}\PYG{n}{posPrecio}\PYG{o}{\PYGZgt{}}\PYG{l+m+mi}{2}\PYG{o}{)}\PYG{o}{\PYGZob{}}
                        \PYG{k}{return} \PYG{n}{aux}\PYG{o}{;}
                \PYG{o}{\PYGZcb{}}
                \PYG{k}{return} \PYG{n}{ciudades}\PYG{o}{;}
        \PYG{o}{\PYGZcb{}}

        \PYG{k+kd}{public} \PYG{k+kt}{void} \PYG{n+nf}{crearElemento}\PYG{o}{(}\PYG{n}{Document} \PYG{n}{d}\PYG{o}{,}
                        \PYG{n}{String} \PYG{n}{nombre}\PYG{o}{,}\PYG{n}{String} \PYG{n}{contenido}\PYG{o}{)}\PYG{o}{\PYGZob{}}
                \PYG{n}{Element} \PYG{n}{e}\PYG{o}{=}\PYG{n}{d}\PYG{o}{.}\PYG{n+na}{createElement}\PYG{o}{(}\PYG{n}{nombre}\PYG{o}{)}\PYG{o}{;}
                \PYG{n}{e}\PYG{o}{.}\PYG{n+na}{setTextContent}\PYG{o}{(}\PYG{n}{contenido}\PYG{o}{)}\PYG{o}{;}
        \PYG{o}{\PYGZcb{}}

        \PYG{k+kd}{public} \PYG{k+kt}{void} \PYG{n+nf}{crearRSS}\PYG{o}{(}\PYG{o}{)}\PYG{o}{\PYGZob{}}
                \PYG{n}{DocumentBuilderFactory} \PYG{n}{fabrica}\PYG{o}{;}
                \PYG{n}{DocumentBuilder} \PYG{n}{constructor}\PYG{o}{;}
                \PYG{n}{Document} \PYG{n}{documentoXML}\PYG{o}{;}
                \PYG{k}{try}\PYG{o}{\PYGZob{}}
                        \PYG{n}{fabrica}\PYG{o}{=}
                                        \PYG{n}{DocumentBuilderFactory}\PYG{o}{.}\PYG{n+na}{newInstance}\PYG{o}{(}\PYG{o}{)}\PYG{o}{;}
                        \PYG{n}{constructor}\PYG{o}{=}\PYG{n}{fabrica}\PYG{o}{.}\PYG{n+na}{newDocumentBuilder}\PYG{o}{(}\PYG{o}{)}\PYG{o}{;}
                        \PYG{n}{documentoXML}\PYG{o}{=}\PYG{n}{constructor}\PYG{o}{.}\PYG{n+na}{newDocument}\PYG{o}{(}\PYG{o}{)}\PYG{o}{;}
                        \PYG{n}{TransformerFactory} \PYG{n}{fabricaConv} \PYG{o}{=}
                                        \PYG{n}{TransformerFactory}\PYG{o}{.}\PYG{n+na}{newInstance}\PYG{o}{(}\PYG{o}{)}\PYG{o}{;}
                        \PYG{n}{Transformer} \PYG{n}{transformador} \PYG{o}{=}
                                        \PYG{n}{fabricaConv}\PYG{o}{.}\PYG{n+na}{newTransformer}\PYG{o}{(}\PYG{o}{)}\PYG{o}{;}
                        \PYG{n}{DOMSource} \PYG{n}{origenDOM} \PYG{o}{=}
                                        \PYG{k}{new} \PYG{n}{DOMSource}\PYG{o}{(}\PYG{n}{documentoXML}\PYG{o}{)}\PYG{o}{;}
                        \PYG{n}{Element} \PYG{n}{e}\PYG{o}{=}\PYG{n}{documentoXML}\PYG{o}{.}\PYG{n+na}{createElement}\PYG{o}{(}\PYG{l+s}{\PYGZdq{}rss\PYGZdq{}}\PYG{o}{)}\PYG{o}{;}
                        \PYG{n}{documentoXML}\PYG{o}{.}\PYG{n+na}{appendChild}\PYG{o}{(}\PYG{n}{e}\PYG{o}{)}\PYG{o}{;}
                        \PYG{n}{StreamResult} \PYG{n}{resultado}\PYG{o}{=}
                                        \PYG{k}{new} \PYG{n}{StreamResult}\PYG{o}{(}
                                                        \PYG{k}{new} \PYG{n}{File}\PYG{o}{(}\PYG{l+s}{\PYGZdq{}D:\PYGZbs{}\PYGZbs{}oscar\PYGZbs{}\PYGZbs{}archivo.rss\PYGZdq{}}\PYG{o}{)}\PYG{o}{)}\PYG{o}{;}
                        \PYG{n}{transformador}\PYG{o}{.}\PYG{n+na}{transform}\PYG{o}{(}\PYG{n}{origenDOM}\PYG{o}{,} \PYG{n}{resultado}\PYG{o}{)}\PYG{o}{;}
                \PYG{o}{\PYGZcb{}}
                \PYG{k}{catch} \PYG{o}{(}\PYG{n}{Exception} \PYG{n}{e}\PYG{o}{)}\PYG{o}{\PYGZob{}}
                        \PYG{n}{System}\PYG{o}{.}\PYG{n+na}{out}\PYG{o}{.}\PYG{n+na}{print}\PYG{o}{(}\PYG{l+s}{\PYGZdq{}No se han podido crear los\PYGZdq{}}\PYG{o}{)}\PYG{o}{;}
                        \PYG{n}{System}\PYG{o}{.}\PYG{n+na}{out}\PYG{o}{.}\PYG{n+na}{println}\PYG{o}{(}\PYG{l+s}{\PYGZdq{} objetos necesarios.\PYGZdq{}}\PYG{o}{)}\PYG{o}{;}
                        \PYG{n}{e}\PYG{o}{.}\PYG{n+na}{printStackTrace}\PYG{o}{(}\PYG{o}{)}\PYG{o}{;}
                        \PYG{k}{return} \PYG{o}{;}
                \PYG{o}{\PYGZcb{}}
        \PYG{o}{\PYGZcb{}}

        \PYG{k+kd}{public} \PYG{k+kd}{static} \PYG{k+kt}{void} \PYG{n+nf}{main} \PYG{o}{(}\PYG{n}{String}\PYG{o}{[}\PYG{o}{]} \PYG{n}{argumentos}\PYG{o}{)}\PYG{o}{\PYGZob{}}
                \PYG{n}{System}\PYG{o}{.}\PYG{n+na}{out}\PYG{o}{.}\PYG{n+na}{println}\PYG{o}{(}\PYG{l+s}{\PYGZdq{}Probando...\PYGZdq{}}\PYG{o}{)}\PYG{o}{;}
                \PYG{n}{ProcesadorXML} \PYG{n}{proc}\PYG{o}{=}\PYG{k}{new} \PYG{n}{ProcesadorXML}\PYG{o}{(}\PYG{o}{)}\PYG{o}{;}
                \PYG{n}{Node} \PYG{n}{nodoRaiz}\PYG{o}{=}\PYG{n}{proc}\PYG{o}{.}\PYG{n+na}{extraerRaiz}\PYG{o}{(}\PYG{l+s}{\PYGZdq{}bolsas.xml\PYGZdq{}}\PYG{o}{)}\PYG{o}{;}
\PYG{c+c1}{//              proc.imprimirNombreDeLaRaiz(nodoRaiz);}
\PYG{c+c1}{//              proc.imprimirHijos(nodoRaiz);}
\PYG{c+c1}{//              proc.imprimirHijos2(nodoRaiz);}
\PYG{c+c1}{//              int cuantosFuturos=proc.contadorElementos(nodoRaiz, \PYGZdq{}futuro\PYGZdq{});}
\PYG{c+c1}{//              System.out.println(\PYGZdq{}Hay \PYGZdq{}+cuantosFuturos);}
\PYG{c+c1}{//              proc.imprimirPrecioBonos(nodoRaiz);}
\PYG{c+c1}{//              int inestables=proc.cuantosInestables(nodoRaiz);}
\PYG{c+c1}{//              System.out.println(\PYGZdq{}Inestables hay:\PYGZdq{}+inestables);}
\PYG{c+c1}{//              float preciosTotales=proc.sumarAtributosPrecio(nodoRaiz);}
\PYG{c+c1}{//              System.out.println(\PYGZdq{}La suma total es:\PYGZdq{}+preciosTotales);}
                \PYG{k+kt}{int} \PYG{n}{cuantos}\PYG{o}{=}\PYG{n}{proc}\PYG{o}{.}\PYG{n+na}{algoQueVerCon}\PYG{o}{(}\PYG{n}{nodoRaiz}\PYG{o}{,} \PYG{l+s}{\PYGZdq{}Islandia\PYGZdq{}}\PYG{o}{)}\PYG{o}{;}
\PYG{c+c1}{//              System.out.println(\PYGZdq{}Num paises Islandia:\PYGZdq{}+cuantos);}
\PYG{c+c1}{//              cuantos=proc.cuantosFuturosTienenCiudadProcedencia(}
\PYG{c+c1}{//                              nodoRaiz, \PYGZdq{}madrid\PYGZdq{});}
\PYG{c+c1}{//              System.out.println(\PYGZdq{}Futuros de Madrid:\PYGZdq{}+cuantos);}
\PYG{c+c1}{//              cuantos=proc.letrasConPaisEmisor(nodoRaiz, \PYGZdq{}espania\PYGZdq{});}
\PYG{c+c1}{//              System.out.println(\PYGZdq{}Letras con espania:\PYGZdq{}+cuantos);}
                \PYG{n}{cuantos}\PYG{o}{=}\PYG{n}{proc}\PYG{o}{.}\PYG{n+na}{algoQueVerConEspania}\PYG{o}{(}\PYG{n}{nodoRaiz}\PYG{o}{)}\PYG{o}{;}
                \PYG{c+c1}{//System.out.println(\PYGZdq{}Productos rel. con España:\PYGZdq{}+cuantos);}
                \PYG{c+c1}{//proc.imprimirBonos(nodoRaiz);}
                \PYG{n}{proc}\PYG{o}{.}\PYG{n+na}{imprimirMismaCiudad}\PYG{o}{(}\PYG{n}{nodoRaiz}\PYG{o}{)}\PYG{o}{;}
                \PYG{n}{String}\PYG{o}{[}\PYG{o}{]} \PYG{n}{resultados}\PYG{o}{=}\PYG{n}{proc}\PYG{o}{.}\PYG{n+na}{obtenerListaBonos}\PYG{o}{(}\PYG{n}{nodoRaiz}\PYG{o}{)}\PYG{o}{;}
                \PYG{n}{System}\PYG{o}{.}\PYG{n+na}{out}\PYG{o}{.}\PYG{n+na}{println}\PYG{o}{(}\PYG{l+s}{\PYGZdq{}La ciudad 0 es:\PYGZdq{}}\PYG{o}{+}\PYG{n}{resultados}\PYG{o}{[}\PYG{l+m+mi}{0}\PYG{o}{]}\PYG{o}{)}\PYG{o}{;}
                \PYG{n}{proc}\PYG{o}{.}\PYG{n+na}{crearRSS}\PYG{o}{(}\PYG{o}{)}\PYG{o}{;}
        \PYG{o}{\PYGZcb{}}
\PYG{o}{\PYGZcb{}}
\end{sphinxVerbatim}


\subsection{Archivo XML}
\label{\detokenize{tema6:archivo-xml}}
\begin{sphinxVerbatim}[commandchars=\\\{\}]
\PYG{c+cp}{\PYGZlt{}?xml version=\PYGZdq{}1.0\PYGZdq{} encoding=\PYGZdq{}utf\PYGZhy{}8\PYGZdq{}?\PYGZgt{}}
\PYG{c+cp}{\PYGZlt{}!DOCTYPE listado [}
\PYG{c+cp}{        \PYGZlt{}!ELEMENT listado (futuro+, divisa+, bono+, letra+)\PYGZgt{}}
        \PYG{c+cp}{\PYGZlt{}!ATTLIST futuro precio CDATA \PYGZsh{}REQUIRED\PYGZgt{}}
        \PYG{c+cp}{\PYGZlt{}!ATTLIST divisa precio CDATA \PYGZsh{}REQUIRED\PYGZgt{}}
        \PYG{c+cp}{\PYGZlt{}!ATTLIST bono precio CDATA \PYGZsh{}REQUIRED\PYGZgt{}}
        \PYG{c+cp}{\PYGZlt{}!ATTLIST letra precio CDATA \PYGZsh{}REQUIRED\PYGZgt{}}
        \PYG{c+cp}{\PYGZlt{}!ELEMENT ciudad\PYGZus{}procedencia (madrid\textbar{}nyork\textbar{}frankfurt\textbar{}tokio)\PYGZgt{}}
        \PYG{c+cp}{\PYGZlt{}!ELEMENT madrid EMPTY\PYGZgt{}}
        \PYG{c+cp}{\PYGZlt{}!ELEMENT nyork EMPTY\PYGZgt{}}
        \PYG{c+cp}{\PYGZlt{}!ELEMENT frankfurt EMPTY\PYGZgt{}}
        \PYG{c+cp}{\PYGZlt{}!ELEMENT tokio EMPTY\PYGZgt{}}
        \PYG{c+cp}{\PYGZlt{}!ATTLIST divisa estable CDATA \PYGZsh{}IMPLIED\PYGZgt{}}
        \PYG{c+cp}{\PYGZlt{}!ATTLIST bono estable CDATA \PYGZsh{}IMPLIED\PYGZgt{}}
        \PYG{c+cp}{\PYGZlt{}!ELEMENT futuro (producto, mercado?, ciudad\PYGZus{}procedencia)\PYGZgt{}}
        \PYG{c+cp}{\PYGZlt{}!ELEMENT producto (\PYGZsh{}PCDATA)\PYGZgt{}}
        \PYG{c+cp}{\PYGZlt{}!ELEMENT mercado (\PYGZsh{}PCDATA)\PYGZgt{}}
        \PYG{c+cp}{\PYGZlt{}!ELEMENT bono (pais\PYGZus{}de\PYGZus{}procedencia,valor\PYGZus{}deseado,}
\PYG{c+cp}{                        valor\PYGZus{}minimo, valor\PYGZus{}maximo, ciudad\PYGZus{}procedencia)\PYGZgt{}}
        \PYG{c+cp}{\PYGZlt{}!ELEMENT valor\PYGZus{}deseado (\PYGZsh{}PCDATA)\PYGZgt{}}
        \PYG{c+cp}{\PYGZlt{}!ELEMENT valor\PYGZus{}minimo (\PYGZsh{}PCDATA)\PYGZgt{}}
        \PYG{c+cp}{\PYGZlt{}!ELEMENT valor\PYGZus{}maximo (\PYGZsh{}PCDATA)\PYGZgt{}}
        \PYG{c+cp}{\PYGZlt{}!ELEMENT pais\PYGZus{}de\PYGZus{}procedencia (\PYGZsh{}PCDATA)\PYGZgt{}}
        \PYG{c+cp}{\PYGZlt{}!ELEMENT divisa (nombre\PYGZus{}divisa,}
\PYG{c+cp}{                        tipo\PYGZus{}de\PYGZus{}cambio+, ciudad\PYGZus{}procedencia)\PYGZgt{}}
        \PYG{c+cp}{\PYGZlt{}!ELEMENT nombre\PYGZus{}divisa (\PYGZsh{}PCDATA)\PYGZgt{}}
        \PYG{c+cp}{\PYGZlt{}!ELEMENT tipo\PYGZus{}de\PYGZus{}cambio (\PYGZsh{}PCDATA)\PYGZgt{}}
        \PYG{c+cp}{\PYGZlt{}!ELEMENT letra (tipo\PYGZus{}de\PYGZus{}interes, pais\PYGZus{}emisor,ciudad\PYGZus{}procedencia)\PYGZgt{}}
        \PYG{c+cp}{\PYGZlt{}!ELEMENT tipo\PYGZus{}de\PYGZus{}interes (\PYGZsh{}PCDATA)\PYGZgt{}}
        \PYG{c+cp}{\PYGZlt{}!ELEMENT pais\PYGZus{}emisor (espania\textbar{}eeuu\textbar{}alemania\textbar{}japon)\PYGZgt{}}
        \PYG{c+cp}{\PYGZlt{}!ELEMENT espania     EMPTY\PYGZgt{}}
        \PYG{c+cp}{\PYGZlt{}!ELEMENT eeuu        EMPTY\PYGZgt{}}
        \PYG{c+cp}{\PYGZlt{}!ELEMENT alemania    EMPTY\PYGZgt{}}
        \PYG{c+cp}{\PYGZlt{}!ELEMENT japon       EMPTY\PYGZgt{}}
]\PYGZgt{}


\PYG{n+nt}{\PYGZlt{}listado}\PYG{n+nt}{\PYGZgt{}}\PYG{n+nt}{\PYGZlt{}futuro} \PYG{n+na}{precio=}\PYG{l+s}{\PYGZdq{}11.28\PYGZdq{}}\PYG{n+nt}{\PYGZgt{}}
                \PYG{n+nt}{\PYGZlt{}producto}\PYG{n+nt}{\PYGZgt{}}Cafe\PYG{n+nt}{\PYGZlt{}/producto\PYGZgt{}}
                \PYG{n+nt}{\PYGZlt{}mercado}\PYG{n+nt}{\PYGZgt{}}América Latina\PYG{n+nt}{\PYGZlt{}/mercado\PYGZgt{}}
                \PYG{n+nt}{\PYGZlt{}ciudad\PYGZus{}procedencia}\PYG{n+nt}{\PYGZgt{}}
                        \PYG{n+nt}{\PYGZlt{}madrid}\PYG{n+nt}{/\PYGZgt{}}
                \PYG{n+nt}{\PYGZlt{}/ciudad\PYGZus{}procedencia\PYGZgt{}}
        \PYG{n+nt}{\PYGZlt{}/futuro\PYGZgt{}}
        \PYG{n+nt}{\PYGZlt{}divisa} \PYG{n+na}{precio=}\PYG{l+s}{\PYGZdq{}183\PYGZdq{}}\PYG{n+nt}{\PYGZgt{}}
                \PYG{n+nt}{\PYGZlt{}nombre\PYGZus{}divisa}\PYG{n+nt}{\PYGZgt{}}Libra esterlina\PYG{n+nt}{\PYGZlt{}/nombre\PYGZus{}divisa\PYGZgt{}}
                \PYG{n+nt}{\PYGZlt{}tipo\PYGZus{}de\PYGZus{}cambio}\PYG{n+nt}{\PYGZgt{}}2.7:1 euros\PYG{n+nt}{\PYGZlt{}/tipo\PYGZus{}de\PYGZus{}cambio\PYGZgt{}}
                \PYG{n+nt}{\PYGZlt{}tipo\PYGZus{}de\PYGZus{}cambio}\PYG{n+nt}{\PYGZgt{}}1:0.87 dólares\PYG{n+nt}{\PYGZlt{}/tipo\PYGZus{}de\PYGZus{}cambio\PYGZgt{}}
                \PYG{n+nt}{\PYGZlt{}ciudad\PYGZus{}procedencia}\PYG{n+nt}{\PYGZgt{}}
                        \PYG{n+nt}{\PYGZlt{}madrid}\PYG{n+nt}{/\PYGZgt{}}
                \PYG{n+nt}{\PYGZlt{}/ciudad\PYGZus{}procedencia\PYGZgt{}}
        \PYG{n+nt}{\PYGZlt{}/divisa\PYGZgt{}}
        \PYG{n+nt}{\PYGZlt{}bono} \PYG{n+na}{precio=}\PYG{l+s}{\PYGZdq{}100\PYGZdq{}} \PYG{n+na}{estable=}\PYG{l+s}{\PYGZdq{}si\PYGZdq{}}\PYG{n+nt}{\PYGZgt{}}
                \PYG{n+nt}{\PYGZlt{}pais\PYGZus{}de\PYGZus{}procedencia}\PYG{n+nt}{\PYGZgt{}}
                        España
                \PYG{n+nt}{\PYGZlt{}/pais\PYGZus{}de\PYGZus{}procedencia\PYGZgt{}}
                \PYG{n+nt}{\PYGZlt{}valor\PYGZus{}deseado}\PYG{n+nt}{\PYGZgt{}}9980\PYG{n+nt}{\PYGZlt{}/valor\PYGZus{}deseado\PYGZgt{}}
                \PYG{n+nt}{\PYGZlt{}valor\PYGZus{}minimo}\PYG{n+nt}{\PYGZgt{}}9950\PYG{n+nt}{\PYGZlt{}/valor\PYGZus{}minimo\PYGZgt{}}
                \PYG{n+nt}{\PYGZlt{}valor\PYGZus{}maximo}\PYG{n+nt}{\PYGZgt{}}10020\PYG{n+nt}{\PYGZlt{}/valor\PYGZus{}maximo\PYGZgt{}}
                \PYG{n+nt}{\PYGZlt{}ciudad\PYGZus{}procedencia}\PYG{n+nt}{\PYGZgt{}}
                        \PYG{n+nt}{\PYGZlt{}tokio}\PYG{n+nt}{/\PYGZgt{}}
                \PYG{n+nt}{\PYGZlt{}/ciudad\PYGZus{}procedencia\PYGZgt{}}
        \PYG{n+nt}{\PYGZlt{}/bono\PYGZgt{}}
        \PYG{n+nt}{\PYGZlt{}bono} \PYG{n+na}{precio=}\PYG{l+s}{\PYGZdq{}10000\PYGZdq{}} \PYG{n+na}{estable=}\PYG{l+s}{\PYGZdq{}si\PYGZdq{}}\PYG{n+nt}{\PYGZgt{}}
                \PYG{n+nt}{\PYGZlt{}pais\PYGZus{}de\PYGZus{}procedencia}\PYG{n+nt}{\PYGZgt{}}
                        España
                \PYG{n+nt}{\PYGZlt{}/pais\PYGZus{}de\PYGZus{}procedencia\PYGZgt{}}
                \PYG{n+nt}{\PYGZlt{}valor\PYGZus{}deseado}\PYG{n+nt}{\PYGZgt{}}9980\PYG{n+nt}{\PYGZlt{}/valor\PYGZus{}deseado\PYGZgt{}}
                \PYG{n+nt}{\PYGZlt{}valor\PYGZus{}minimo}\PYG{n+nt}{\PYGZgt{}}9950\PYG{n+nt}{\PYGZlt{}/valor\PYGZus{}minimo\PYGZgt{}}
                \PYG{n+nt}{\PYGZlt{}valor\PYGZus{}maximo}\PYG{n+nt}{\PYGZgt{}}10020\PYG{n+nt}{\PYGZlt{}/valor\PYGZus{}maximo\PYGZgt{}}
                \PYG{n+nt}{\PYGZlt{}ciudad\PYGZus{}procedencia}\PYG{n+nt}{\PYGZgt{}}
                        \PYG{n+nt}{\PYGZlt{}tokio}\PYG{n+nt}{/\PYGZgt{}}
                \PYG{n+nt}{\PYGZlt{}/ciudad\PYGZus{}procedencia\PYGZgt{}}
        \PYG{n+nt}{\PYGZlt{}/bono\PYGZgt{}}
        \PYG{n+nt}{\PYGZlt{}bono} \PYG{n+na}{precio=}\PYG{l+s}{\PYGZdq{}10000\PYGZdq{}} \PYG{n+na}{estable=}\PYG{l+s}{\PYGZdq{}no\PYGZdq{}}\PYG{n+nt}{\PYGZgt{}}
                \PYG{n+nt}{\PYGZlt{}pais\PYGZus{}de\PYGZus{}procedencia}\PYG{n+nt}{\PYGZgt{}}
                        España
                \PYG{n+nt}{\PYGZlt{}/pais\PYGZus{}de\PYGZus{}procedencia\PYGZgt{}}
                \PYG{n+nt}{\PYGZlt{}valor\PYGZus{}deseado}\PYG{n+nt}{\PYGZgt{}}9980\PYG{n+nt}{\PYGZlt{}/valor\PYGZus{}deseado\PYGZgt{}}
                \PYG{n+nt}{\PYGZlt{}valor\PYGZus{}minimo}\PYG{n+nt}{\PYGZgt{}}9950\PYG{n+nt}{\PYGZlt{}/valor\PYGZus{}minimo\PYGZgt{}}
                \PYG{n+nt}{\PYGZlt{}valor\PYGZus{}maximo}\PYG{n+nt}{\PYGZgt{}}10020\PYG{n+nt}{\PYGZlt{}/valor\PYGZus{}maximo\PYGZgt{}}
                \PYG{n+nt}{\PYGZlt{}ciudad\PYGZus{}procedencia}\PYG{n+nt}{\PYGZgt{}}
                        \PYG{n+nt}{\PYGZlt{}tokio}\PYG{n+nt}{/\PYGZgt{}}
                \PYG{n+nt}{\PYGZlt{}/ciudad\PYGZus{}procedencia\PYGZgt{}}
        \PYG{n+nt}{\PYGZlt{}/bono\PYGZgt{}}
        \PYG{n+nt}{\PYGZlt{}bono} \PYG{n+na}{precio=}\PYG{l+s}{\PYGZdq{}10000\PYGZdq{}} \PYG{n+na}{estable=}\PYG{l+s}{\PYGZdq{}no\PYGZdq{}}\PYG{n+nt}{\PYGZgt{}}
                \PYG{n+nt}{\PYGZlt{}pais\PYGZus{}de\PYGZus{}procedencia}\PYG{n+nt}{\PYGZgt{}}
                        España
                \PYG{n+nt}{\PYGZlt{}/pais\PYGZus{}de\PYGZus{}procedencia\PYGZgt{}}
                \PYG{n+nt}{\PYGZlt{}valor\PYGZus{}deseado}\PYG{n+nt}{\PYGZgt{}}9980\PYG{n+nt}{\PYGZlt{}/valor\PYGZus{}deseado\PYGZgt{}}
                \PYG{n+nt}{\PYGZlt{}valor\PYGZus{}minimo}\PYG{n+nt}{\PYGZgt{}}9950\PYG{n+nt}{\PYGZlt{}/valor\PYGZus{}minimo\PYGZgt{}}
                \PYG{n+nt}{\PYGZlt{}valor\PYGZus{}maximo}\PYG{n+nt}{\PYGZgt{}}10020\PYG{n+nt}{\PYGZlt{}/valor\PYGZus{}maximo\PYGZgt{}}
                \PYG{n+nt}{\PYGZlt{}ciudad\PYGZus{}procedencia}\PYG{n+nt}{\PYGZgt{}}
                        \PYG{n+nt}{\PYGZlt{}tokio}\PYG{n+nt}{/\PYGZgt{}}
                \PYG{n+nt}{\PYGZlt{}/ciudad\PYGZus{}procedencia\PYGZgt{}}
        \PYG{n+nt}{\PYGZlt{}/bono\PYGZgt{}}
        \PYG{n+nt}{\PYGZlt{}letra} \PYG{n+na}{precio=}\PYG{l+s}{\PYGZdq{}45020\PYGZdq{}}\PYG{n+nt}{\PYGZgt{}}
                \PYG{n+nt}{\PYGZlt{}tipo\PYGZus{}de\PYGZus{}interes}\PYG{n+nt}{\PYGZgt{}}4.54\PYGZpc{}\PYG{n+nt}{\PYGZlt{}/tipo\PYGZus{}de\PYGZus{}interes\PYGZgt{}}
                \PYG{n+nt}{\PYGZlt{}pais\PYGZus{}emisor}\PYG{n+nt}{\PYGZgt{}}
                        \PYG{n+nt}{\PYGZlt{}espania}\PYG{n+nt}{/\PYGZgt{}}
                \PYG{n+nt}{\PYGZlt{}/pais\PYGZus{}emisor\PYGZgt{}}
                \PYG{n+nt}{\PYGZlt{}ciudad\PYGZus{}procedencia}\PYG{n+nt}{\PYGZgt{}}
                        \PYG{n+nt}{\PYGZlt{}madrid}\PYG{n+nt}{/\PYGZgt{}}
                \PYG{n+nt}{\PYGZlt{}/ciudad\PYGZus{}procedencia\PYGZgt{}}
        \PYG{n+nt}{\PYGZlt{}/letra\PYGZgt{}}
\PYG{n+nt}{\PYGZlt{}/listado\PYGZgt{}}
\end{sphinxVerbatim}


\section{Procesamiento con SAX}
\label{\detokenize{tema6:procesamiento-con-sax}}
Simple Api for XML (o SAX) es un conjunto de clases para procesar XML de una forma muchísimo más eficiente (pero también más incómoda). Consiste en un \sphinxstyleemphasis{parser} que va leyendo etiqueta por etiqueta. Cada vez que el parser encuentra una etiqueta nueva nos lo comunicará mediante un evento y tendremos que incluir código de gestión de eventos que decidan que hacer.

La forma más sencilla de trabajar es hacer que nuestra clase herede de \sphinxcode{DefaultHandler} y sobrecargar el código de los métodos \sphinxcode{startElement} y \sphinxcode{endElement}.

Cuando se procesa XML podemos encontrarnos con que se usen o no espacios de nombres. Si usamos espacios de nombres SAX nos devolverá un argumento pero si no los usamos SAX nos devolverá otro argumento. Observemos el siguiente código:

\begin{sphinxVerbatim}[commandchars=\\\{\}]
\PYG{k+kd}{public} \PYG{k+kd}{class} \PYG{n+nc}{ProcesadorSAX} \PYG{k+kd}{extends} \PYG{n}{DefaultHandler}\PYG{o}{\PYGZob{}}

  \PYG{n+nd}{@Override}
  \PYG{k+kd}{public} \PYG{k+kt}{void} \PYG{n+nf}{startElement}\PYG{o}{(}
      \PYG{n}{String} \PYG{n}{ns}\PYG{o}{,} \PYG{n}{String} \PYG{n}{nombreCuandoHayNS}\PYG{o}{,}
      \PYG{n}{String} \PYG{n}{nombreCuandoNoHayNS}\PYG{o}{,}
      \PYG{n}{Attributes} \PYG{n}{atributos}\PYG{o}{)}
      \PYG{k+kd}{throws} \PYG{n}{SAXException}
  \PYG{o}{\PYGZob{}}
    \PYG{n}{System}\PYG{o}{.}\PYG{n+na}{out}\PYG{o}{.}\PYG{n+na}{println}\PYG{o}{(}\PYG{n}{nombreCuandoNoHayNS}\PYG{o}{)}\PYG{o}{;}
  \PYG{o}{\PYGZcb{}}
\PYG{o}{\PYGZcb{}}
\end{sphinxVerbatim}

En este código SAX avisará a nuestro método \sphinxcode{startElement} (el nombre debe ser así), cada vez que
encuentre una etiqueta. Como en nuestros documentos no estamos usando espacios de nombres nos interesa imprimir el tercer parámetro.

Para hacer que Java procese un fichero mediante SAX usando nuestra clase como procesador de etiquetas haremos lo siguiente:

\begin{sphinxVerbatim}[commandchars=\\\{\}]
\PYG{k+kd}{public} \PYG{k+kd}{class} \PYG{n+nc}{ProcesadorSAX} \PYG{k+kd}{extends} \PYG{n}{DefaultHandler}\PYG{o}{\PYGZob{}}

  \PYG{n+nd}{@Override}
  \PYG{k+kd}{public} \PYG{k+kt}{void} \PYG{n+nf}{startElement}\PYG{o}{(}
      \PYG{n}{String} \PYG{n}{ns}\PYG{o}{,} \PYG{n}{String} \PYG{n}{nombreCuandoHayNS}\PYG{o}{,}
      \PYG{n}{String} \PYG{n}{nombreCuandoNoHayNS}\PYG{o}{,}
      \PYG{n}{Attributes} \PYG{n}{atributos}\PYG{o}{)}
      \PYG{k+kd}{throws} \PYG{n}{SAXException}
  \PYG{o}{\PYGZob{}}
    \PYG{n}{System}\PYG{o}{.}\PYG{n+na}{out}\PYG{o}{.}\PYG{n+na}{println}\PYG{o}{(}\PYG{n}{nombreCuandoNoHayNS}\PYG{o}{)}\PYG{o}{;}
  \PYG{o}{\PYGZcb{}}

  \PYG{k+kd}{public} \PYG{k+kd}{static} \PYG{k+kt}{void} \PYG{n+nf}{main}\PYG{o}{(}\PYG{n}{String}\PYG{o}{[}\PYG{o}{]} \PYG{n}{args}\PYG{o}{)}
      \PYG{k+kd}{throws} \PYG{n}{ParserConfigurationException}\PYG{o}{,}
      \PYG{n}{SAXException}\PYG{o}{,} \PYG{n}{IOException} \PYG{o}{\PYGZob{}}

    \PYG{n}{SAXParserFactory} \PYG{n}{fabrica}\PYG{o}{;}
    \PYG{n}{fabrica}\PYG{o}{=}\PYG{n}{SAXParserFactory}\PYG{o}{.}\PYG{n+na}{newInstance}\PYG{o}{(}\PYG{o}{)}\PYG{o}{;}
    \PYG{n}{SAXParser} \PYG{n}{parser}\PYG{o}{=}\PYG{n}{fabrica}\PYG{o}{.}\PYG{n+na}{newSAXParser}\PYG{o}{(}\PYG{o}{)}\PYG{o}{;}
    \PYG{n}{XMLReader} \PYG{n}{lector}\PYG{o}{=}\PYG{n}{parser}\PYG{o}{.}\PYG{n+na}{getXMLReader}\PYG{o}{(}\PYG{o}{)}\PYG{o}{;}
    \PYG{n}{lector}\PYG{o}{.}\PYG{n+na}{setContentHandler}\PYG{o}{(}\PYG{k}{new} \PYG{n}{ProcesadorSAX}\PYG{o}{(}\PYG{o}{)}\PYG{o}{)}\PYG{o}{;}
    \PYG{n}{lector}\PYG{o}{.}\PYG{n+na}{parse}\PYG{o}{(}
      \PYG{l+s}{\PYGZdq{}c:/users/ogomez/documents/finanzas.xml\PYGZdq{}}\PYG{o}{)}\PYG{o}{;}
  \PYG{o}{\PYGZcb{}}
\PYG{o}{\PYGZcb{}}
\end{sphinxVerbatim}

Ahora Java irá leyendo el fichero etiqueta por etiqueta y mostrándonos los nombres de todas. No importará que el fichero ocupe varios GB, ya que SAX no cargará el fichero completo en memoria.


\subsection{Ejercicio: encontrar producto}
\label{\detokenize{tema6:ejercicio-encontrar-producto}}
Encontrar todos los productos cuyo nombre sea «Cafe» y su mercado «América Latina».

\begin{sphinxVerbatim}[commandchars=\\\{\}]
\PYG{k+kd}{public} \PYG{k+kt}{void} \PYG{n+nf}{characters}\PYG{o}{(}
      \PYG{k+kt}{char}\PYG{o}{[}\PYG{o}{]} \PYG{n}{letras}\PYG{o}{,} \PYG{k+kt}{int} \PYG{n}{ini}\PYG{o}{,} \PYG{k+kt}{int} \PYG{n}{longitud}\PYG{o}{)}\PYG{o}{\PYGZob{}}
    \PYG{k}{if} \PYG{o}{(}\PYG{o}{(}\PYG{n}{mercadoEncontrado}\PYG{o}{)} \PYG{o}{\PYGZam{}}\PYG{o}{\PYGZam{}} \PYG{o}{(}\PYG{n}{cafeEncontrado}\PYG{o}{)}\PYG{o}{)}\PYG{o}{\PYGZob{}}
      \PYG{n}{String} \PYG{n}{cadena}\PYG{o}{=}\PYG{k}{new} \PYG{n}{String}\PYG{o}{(}\PYG{n}{letras}\PYG{o}{,} \PYG{n}{ini}\PYG{o}{,} \PYG{n}{longitud}\PYG{o}{)}\PYG{o}{;}
      \PYG{k}{if} \PYG{o}{(}\PYG{n}{cadena}\PYG{o}{.}\PYG{n+na}{equals}\PYG{o}{(}\PYG{l+s}{\PYGZdq{}América Latina\PYGZdq{}}\PYG{o}{)}\PYG{o}{)}\PYG{o}{\PYGZob{}}
        \PYG{n}{System}\PYG{o}{.}\PYG{n+na}{out}\PYG{o}{.}\PYG{n+na}{println}\PYG{o}{(}\PYG{l+s}{\PYGZdq{}Encontrado Cafe de AL\PYGZdq{}}\PYG{o}{)}\PYG{o}{;}
        \PYG{n}{cafeEncontrado}\PYG{o}{=}\PYG{k+kc}{false}\PYG{o}{;}
        \PYG{n}{mercadoEncontrado}\PYG{o}{=}\PYG{k+kc}{false}\PYG{o}{;}
        \PYG{n}{productoEncontrado}\PYG{o}{=}\PYG{k+kc}{false}\PYG{o}{;}
        \PYG{n}{futuroEncontrado}\PYG{o}{=}\PYG{k+kc}{false}\PYG{o}{;}
      \PYG{o}{\PYGZcb{}}
    \PYG{o}{\PYGZcb{}}
    \PYG{k}{if} \PYG{o}{(}\PYG{o}{(}\PYG{n}{productoEncontrado}\PYG{o}{)} \PYG{o}{\PYGZam{}}\PYG{o}{\PYGZam{}} \PYG{o}{(}\PYG{n}{futuroEncontrado}\PYG{o}{)}\PYG{o}{)}\PYG{o}{\PYGZob{}}
      \PYG{n}{String} \PYG{n}{cadena}\PYG{o}{=}\PYG{k}{new} \PYG{n}{String}\PYG{o}{(}\PYG{n}{letras}\PYG{o}{,} \PYG{n}{ini}\PYG{o}{,} \PYG{n}{longitud}\PYG{o}{)}\PYG{o}{;}
      \PYG{k}{if} \PYG{o}{(}\PYG{n}{cadena}\PYG{o}{.}\PYG{n+na}{equals}\PYG{o}{(}\PYG{l+s}{\PYGZdq{}Cafe\PYGZdq{}}\PYG{o}{)}\PYG{o}{)}\PYG{o}{\PYGZob{}}
        \PYG{n}{cafeEncontrado}\PYG{o}{=}\PYG{k+kc}{true}\PYG{o}{;}
      \PYG{o}{\PYGZcb{}} \PYG{c+c1}{//Fin del if interno}
    \PYG{o}{\PYGZcb{}} \PYG{c+c1}{//Fin del if externo}
  \PYG{o}{\PYGZcb{}} \PYG{c+c1}{//Fin del método characters}
\end{sphinxVerbatim}


\subsection{Ejercicio: precios}
\label{\detokenize{tema6:ejercicio-precios}}
Encontrar todos las divisas cuya ciudad de procedencia sea «Madrid».


\section{Ejercicios para preparar examen}
\label{\detokenize{tema6:ejercicios-para-preparar-examen}}\begin{enumerate}
\item {} 
Indicar cuantas letras tienen un tipo de interés inferior al 3\% e indicar para cada una de ellas el país emisor.

\item {} 
Comprobar si hay alguna divisa con nombre «Euro» cuyo precio sea mayor de 195.

\item {} 
Indicar todos los productos que tengan la misma ciudad de procedencia.

\end{enumerate}


\chapter{Sindicación y transformación de contenidos}
\label{\detokenize{tema7::doc}}\label{\detokenize{tema7:sindicacion-y-transformacion-de-contenidos}}

\section{Introducción}
\label{\detokenize{tema7:introduccion}}
Muchas páginas web disponen de «feeds». Un «feed» es un mecanismo de suscripción que facilita la recepción de información.

En general, los feed utilizan un vocabulario XML llamado RSS o uno llamado Atom.


\section{RSS}
\label{\detokenize{tema7:rss}}
RSS es un estándar para la «sindicación» o «agregación» de recursos (recursos web normalmente).

Su objetivo principal era permitir a un sitio web publicar las novedades con facilidad y que el usuario puede ir derectamente al lugar que le interese.


\subsection{Formato de archivo RSS}
\label{\detokenize{tema7:formato-de-archivo-rss}}
Todo archivo RSS es, por supuesto, un XML

\begin{sphinxVerbatim}[commandchars=\\\{\}]
\PYG{c+cp}{\PYGZlt{}?xml version=\PYGZdq{}1.0\PYGZdq{} encoding=\PYGZdq{}utf\PYGZhy{}8\PYGZdq{}?\PYGZgt{}}
\PYG{n+nt}{\PYGZlt{}rss} \PYG{n+na}{version=}\PYG{l+s}{\PYGZdq{}2.0\PYGZdq{}}\PYG{n+nt}{\PYGZgt{}}
        \PYG{n+nt}{\PYGZlt{}channel}\PYG{n+nt}{\PYGZgt{}}
                \PYG{n+nt}{\PYGZlt{}title}\PYG{n+nt}{\PYGZgt{}}
                        Canal de noticias de SSOO de DAM
                \PYG{n+nt}{\PYGZlt{}/title\PYGZgt{}}
                \PYG{n+nt}{\PYGZlt{}link}\PYG{n+nt}{\PYGZgt{}}
                        http://ssoo.iesmaestredecalatrava.es
                \PYG{n+nt}{\PYGZlt{}/link\PYGZgt{}}
                \PYG{n+nt}{\PYGZlt{}description}\PYG{n+nt}{\PYGZgt{}}
                        En este canal...
                \PYG{n+nt}{\PYGZlt{}/description\PYGZgt{}}
                \PYG{n+nt}{\PYGZlt{}item}\PYG{n+nt}{\PYGZgt{}}
                        \PYG{n+nt}{\PYGZlt{}title}\PYG{n+nt}{\PYGZgt{}}Nueva versión de Ubuntu\PYG{n+nt}{\PYGZlt{}/title\PYGZgt{}}
                        \PYG{n+nt}{\PYGZlt{}link}\PYG{n+nt}{\PYGZgt{}}http://ubuntu.org\PYG{n+nt}{\PYGZlt{}/link\PYGZgt{}}
                        \PYG{n+nt}{\PYGZlt{}description}\PYG{n+nt}{\PYGZgt{}}
                                Nueva versión...
                        \PYG{n+nt}{\PYGZlt{}/description\PYGZgt{}}
                \PYG{n+nt}{\PYGZlt{}/item\PYGZgt{}}
        \PYG{n+nt}{\PYGZlt{}/channel\PYGZgt{}}
        \PYG{n+nt}{\PYGZlt{}channel}\PYG{n+nt}{\PYGZgt{}}
                \PYG{n+nt}{\PYGZlt{}title}\PYG{n+nt}{\PYGZgt{}}
                        Canal de Lenguajes de marcas
                \PYG{n+nt}{\PYGZlt{}/title\PYGZgt{}}
                \PYG{n+nt}{\PYGZlt{}link}\PYG{n+nt}{\PYGZgt{}}
                        http://xml.iesmaestredecalatrava.es
                \PYG{n+nt}{\PYGZlt{}/link\PYGZgt{}}
                \PYG{n+nt}{\PYGZlt{}description}\PYG{n+nt}{\PYGZgt{}}
                        En este canal...
                \PYG{n+nt}{\PYGZlt{}/description\PYGZgt{}}
                \PYG{n+nt}{\PYGZlt{}item}\PYG{n+nt}{\PYGZgt{}}
                        \PYG{n+nt}{\PYGZlt{}title}\PYG{n+nt}{\PYGZgt{}}Publicado nuevo validador del W3C\PYG{n+nt}{\PYGZlt{}/title\PYGZgt{}}
                        \PYG{n+nt}{\PYGZlt{}link}\PYG{n+nt}{\PYGZgt{}}http://validator.w3c.org\PYG{n+nt}{\PYGZlt{}/link\PYGZgt{}}
                        \PYG{n+nt}{\PYGZlt{}description}\PYG{n+nt}{\PYGZgt{}}
                                Hay nuevo validador...
                        \PYG{n+nt}{\PYGZlt{}/description\PYGZgt{}}
                \PYG{n+nt}{\PYGZlt{}/item\PYGZgt{}}
        \PYG{n+nt}{\PYGZlt{}/channel\PYGZgt{}}
\PYG{n+nt}{\PYGZlt{}/rss\PYGZgt{}}
\end{sphinxVerbatim}

Las reglas serían las siguientes:
\begin{itemize}
\item {} 
Todo archivo de descripción de recursos en RSS utiliza el preámbulo típico de los documentos xml

\end{itemize}

\begin{sphinxVerbatim}[commandchars=\\\{\}]
\PYG{c+cp}{\PYGZlt{}?xml version=\PYGZdq{}1.0\PYGZdq{} encoding=\PYGZdq{}UTF\PYGZhy{}8\PYGZdq{}?\PYGZgt{}}
\end{sphinxVerbatim}
\begin{itemize}
\item {} 
Todo RSS tiene un solo elemento raíz en el cual se puede indicar la versión RSS a la que nos ceñimos

\end{itemize}

\begin{sphinxVerbatim}[commandchars=\\\{\}]
\PYG{n+nt}{\PYGZlt{}rss} \PYG{n+na}{version=}\PYG{l+s}{\PYGZdq{}2.0\PYGZdq{}}\PYG{n+nt}{\PYGZgt{}}
\end{sphinxVerbatim}
\begin{itemize}
\item {} 
Un RSS tiene uno o más canales

\end{itemize}

\begin{sphinxVerbatim}[commandchars=\\\{\}]
\PYG{n+nt}{\PYGZlt{}channel}\PYG{n+nt}{\PYGZgt{}}
\PYG{n+nt}{\PYGZlt{}/channel\PYGZgt{}}
\end{sphinxVerbatim}
\begin{itemize}
\item {} 
Todo canal debe tener, al menos un título, un enlace base (la dirección del propio sitio web) y una descripción:

\end{itemize}

\begin{sphinxVerbatim}[commandchars=\\\{\}]
\PYG{n+nt}{\PYGZlt{}channel}\PYG{n+nt}{\PYGZgt{}}
        \PYG{n+nt}{\PYGZlt{}title}\PYG{n+nt}{\PYGZgt{}}
        \PYG{n+nt}{\PYGZlt{}/title\PYGZgt{}}
        \PYG{n+nt}{\PYGZlt{}link}\PYG{n+nt}{\PYGZgt{}}
        \PYG{n+nt}{\PYGZlt{}/link\PYGZgt{}}
        \PYG{n+nt}{\PYGZlt{}description}\PYG{n+nt}{\PYGZgt{}}
        \PYG{n+nt}{\PYGZlt{}/description\PYGZgt{}}
        \PYG{n+nt}{\PYGZlt{}item}\PYG{n+nt}{\PYGZgt{}}
                ...
        \PYG{n+nt}{\PYGZlt{}/item\PYGZgt{}}
        \PYG{n+nt}{\PYGZlt{}item}\PYG{n+nt}{\PYGZgt{}}
                ...
        \PYG{n+nt}{\PYGZlt{}/item\PYGZgt{}}
\PYG{n+nt}{\PYGZlt{}/channel\PYGZgt{}}
\end{sphinxVerbatim}

Resumiendo los puntos más importantes:
\begin{enumerate}
\item {} 
Usar como elemento raíz \sphinxcode{rss}.

\item {} 
Todo RSS tiene uno o más \sphinxcode{channel}

\item {} 
Todo \sphinxcode{channel} tiene al menos \sphinxcode{title}, \sphinxcode{link} y \sphinxcode{description}

\item {} 
Despues de estos elementos, un \sphinxcode{channel} puede tener uno o más elementos \sphinxcode{item} (que son los que contienen las noticias)

\item {} 
Todo \sphinxcode{item} también debe tener un \sphinxcode{title}, un \sphinxcode{link} y \sphinxcode{description}

\end{enumerate}


\subsection{Ejercicio}
\label{\detokenize{tema7:ejercicio}}
Crear un fichero Java que construya el siguiente fichero XML:

\begin{sphinxVerbatim}[commandchars=\\\{\}]
\PYG{c+cp}{\PYGZlt{}?xml version=\PYGZdq{}1.0\PYGZdq{}?\PYGZgt{}}
\PYG{n+nt}{\PYGZlt{}rss} \PYG{n+na}{version=}\PYG{l+s}{\PYGZdq{}2.0\PYGZdq{}}\PYG{n+nt}{\PYGZgt{}}
        \PYG{n+nt}{\PYGZlt{}channel}\PYG{n+nt}{\PYGZgt{}}
                \PYG{n+nt}{\PYGZlt{}title}\PYG{n+nt}{\PYGZgt{}}Prueba\PYG{n+nt}{\PYGZlt{}/title\PYGZgt{}}
                \PYG{n+nt}{\PYGZlt{}link}\PYG{n+nt}{\PYGZgt{}}http://www.google.es\PYG{n+nt}{\PYGZlt{}/link\PYGZgt{}}
                \PYG{n+nt}{\PYGZlt{}description}\PYG{n+nt}{\PYGZgt{}}Prueba de descripcion\PYG{n+nt}{\PYGZlt{}/description\PYGZgt{}}
        \PYG{n+nt}{\PYGZlt{}/channel\PYGZgt{}}
\PYG{n+nt}{\PYGZlt{}/rss\PYGZgt{}}
\end{sphinxVerbatim}

Una posible solución es esta:

\begin{sphinxVerbatim}[commandchars=\\\{\}]
\PYG{k+kd}{public} \PYG{k+kd}{class} \PYG{n+nc}{CreadorRSS} \PYG{o}{\PYGZob{}}
        \PYG{k+kd}{public} \PYG{k+kt}{byte}\PYG{o}{[}\PYG{o}{]} \PYG{n+nf}{getEtiquetas}\PYG{o}{(}
                        \PYG{n}{String} \PYG{n}{titulo}\PYG{o}{,}
                        \PYG{n}{String} \PYG{n}{enlace}\PYG{o}{,}
                        \PYG{n}{String} \PYG{n}{descripcion}\PYG{o}{)}
        \PYG{o}{\PYGZob{}}
                \PYG{n}{String} \PYG{n}{resultado}\PYG{o}{=}\PYG{l+s}{\PYGZdq{}\PYGZdq{}}\PYG{o}{;}
                \PYG{n}{resultado}\PYG{o}{+}\PYG{o}{=}\PYG{l+s}{\PYGZdq{}\PYGZlt{}title\PYGZgt{}\PYGZdq{}}\PYG{o}{;}
                \PYG{n}{resultado}\PYG{o}{+}\PYG{o}{=}\PYG{n}{titulo}\PYG{o}{;}
                \PYG{n}{resultado}\PYG{o}{+}\PYG{o}{=}\PYG{l+s}{\PYGZdq{}\PYGZlt{}/title\PYGZgt{}\PYGZbs{}n\PYGZdq{}}\PYG{o}{;}
                \PYG{n}{resultado}\PYG{o}{+}\PYG{o}{=}\PYG{l+s}{\PYGZdq{}\PYGZlt{}link\PYGZgt{}\PYGZdq{}}\PYG{o}{;}
                \PYG{n}{resultado}\PYG{o}{+}\PYG{o}{=}\PYG{n}{enlace}\PYG{o}{;}
                \PYG{n}{resultado}\PYG{o}{+}\PYG{o}{=}\PYG{l+s}{\PYGZdq{}\PYGZlt{}/link\PYGZgt{}\PYGZbs{}n\PYGZdq{}}\PYG{o}{;}
                \PYG{n}{resultado}\PYG{o}{+}\PYG{o}{=}\PYG{l+s}{\PYGZdq{}\PYGZlt{}description\PYGZgt{}\PYGZdq{}}\PYG{o}{;}
                \PYG{n}{resultado}\PYG{o}{+}\PYG{o}{=}\PYG{n}{descripcion}\PYG{o}{;}
                \PYG{n}{resultado}\PYG{o}{+}\PYG{o}{=}\PYG{l+s}{\PYGZdq{}\PYGZlt{}/description\PYGZgt{}\PYGZdq{}}\PYG{o}{;}
                \PYG{k}{return} \PYG{n}{resultado}\PYG{o}{.}\PYG{n+na}{getBytes}\PYG{o}{(}\PYG{o}{)}\PYG{o}{;}
        \PYG{o}{\PYGZcb{}}
        \PYG{k+kd}{public} \PYG{k+kt}{void} \PYG{n+nf}{crearArchivo}\PYG{o}{(}\PYG{n}{String} \PYG{n}{nombre}\PYG{o}{)}
                        \PYG{k+kd}{throws} \PYG{n}{IOException}\PYG{o}{\PYGZob{}}
                \PYG{n}{FileOutputStream} \PYG{n}{fos}\PYG{o}{=}
                                \PYG{k}{new} \PYG{n}{FileOutputStream}\PYG{o}{(}\PYG{n}{nombre}\PYG{o}{)}\PYG{o}{;}
                \PYG{n}{String} \PYG{n}{cabecera}\PYG{o}{=}\PYG{l+s}{\PYGZdq{}\PYGZlt{}?xml version=\PYGZsq{}1.0\PYGZsq{}?\PYGZgt{}\PYGZbs{}n\PYGZdq{}}\PYG{o}{;}
                \PYG{n}{fos}\PYG{o}{.}\PYG{n+na}{write}\PYG{o}{(}\PYG{n}{cabecera}\PYG{o}{.}\PYG{n+na}{getBytes}\PYG{o}{(}\PYG{o}{)}\PYG{o}{)}\PYG{o}{;}
                \PYG{n}{String} \PYG{n}{rss}\PYG{o}{=}\PYG{l+s}{\PYGZdq{}\PYGZlt{}rss version=\PYGZsq{}1.0\PYGZsq{}\PYGZgt{}\PYGZbs{}n\PYGZdq{}}\PYG{o}{;}
                \PYG{n}{fos}\PYG{o}{.}\PYG{n+na}{write}\PYG{o}{(}\PYG{n}{rss}\PYG{o}{.}\PYG{n+na}{getBytes}\PYG{o}{(}\PYG{o}{)}\PYG{o}{)}\PYG{o}{;}
                \PYG{k+kt}{byte}\PYG{o}{[}\PYG{o}{]} \PYG{n}{etiquetas}\PYG{o}{=}\PYG{k}{this}\PYG{o}{.}\PYG{n+na}{getEtiquetas}\PYG{o}{(}
                                \PYG{l+s}{\PYGZdq{}Titulo de la noticia\PYGZdq{}}\PYG{o}{,}
                                \PYG{l+s}{\PYGZdq{}http://www.algo.com\PYGZdq{}}\PYG{o}{,}
                                \PYG{l+s}{\PYGZdq{}Noticia muy importante\PYGZdq{}}\PYG{o}{)}\PYG{o}{;}
                \PYG{n}{fos}\PYG{o}{.}\PYG{n+na}{write}\PYG{o}{(}\PYG{n}{etiquetas}\PYG{o}{)}\PYG{o}{;}
                \PYG{n}{String} \PYG{n}{rssCierre}\PYG{o}{=}\PYG{l+s}{\PYGZdq{}\PYGZlt{}/rss\PYGZgt{}\PYGZdq{}}\PYG{o}{;}
                \PYG{n}{fos}\PYG{o}{.}\PYG{n+na}{write}\PYG{o}{(}\PYG{n}{rssCierre}\PYG{o}{.}\PYG{n+na}{getBytes}\PYG{o}{(}\PYG{o}{)}\PYG{o}{)}\PYG{o}{;}
                \PYG{n}{fos}\PYG{o}{.}\PYG{n+na}{close}\PYG{o}{(}\PYG{o}{)}\PYG{o}{;}
        \PYG{o}{\PYGZcb{}}

        \PYG{k+kd}{public} \PYG{k+kd}{static} \PYG{k+kt}{void} \PYG{n+nf}{main}\PYG{o}{(}\PYG{n}{String}\PYG{o}{[}\PYG{o}{]} \PYG{n}{args}\PYG{o}{)}
                        \PYG{k+kd}{throws} \PYG{n}{IOException} \PYG{o}{\PYGZob{}}
                \PYG{n}{CreadorRSS} \PYG{n}{creador}\PYG{o}{=}\PYG{k}{new} \PYG{n}{CreadorRSS}\PYG{o}{(}\PYG{o}{)}\PYG{o}{;}
                \PYG{n}{creador}\PYG{o}{.}\PYG{n+na}{crearArchivo}\PYG{o}{(}\PYG{l+s}{\PYGZdq{}D:/oscar/archivo.rss\PYGZdq{}}\PYG{o}{)}\PYG{o}{;}
        \PYG{o}{\PYGZcb{}}
\PYG{o}{\PYGZcb{}}
\end{sphinxVerbatim}

El siguiente código Java ilustra otra forma de hacerlo:

\begin{sphinxVerbatim}[commandchars=\\\{\}]
\PYG{k+kd}{public} \PYG{k+kt}{void} \PYG{n+nf}{crearRSS}\PYG{o}{(}\PYG{o}{)}\PYG{o}{\PYGZob{}}
        \PYG{n}{DocumentBuilderFactory} \PYG{n}{fabrica}\PYG{o}{;}
        \PYG{n}{DocumentBuilder} \PYG{n}{constructor}\PYG{o}{;}
        \PYG{n}{Document} \PYG{n}{documentoXML}\PYG{o}{;}
        \PYG{k}{try}\PYG{o}{\PYGZob{}}
                \PYG{n}{fabrica}\PYG{o}{=}
                                \PYG{n}{DocumentBuilderFactory}\PYG{o}{.}\PYG{n+na}{newInstance}\PYG{o}{(}\PYG{o}{)}\PYG{o}{;}
                \PYG{n}{constructor}\PYG{o}{=}\PYG{n}{fabrica}\PYG{o}{.}\PYG{n+na}{newDocumentBuilder}\PYG{o}{(}\PYG{o}{)}\PYG{o}{;}
                \PYG{n}{documentoXML}\PYG{o}{=}\PYG{n}{constructor}\PYG{o}{.}\PYG{n+na}{newDocument}\PYG{o}{(}\PYG{o}{)}\PYG{o}{;}
                \PYG{n}{TransformerFactory} \PYG{n}{fabricaConv} \PYG{o}{=}
                                \PYG{n}{TransformerFactory}\PYG{o}{.}\PYG{n+na}{newInstance}\PYG{o}{(}\PYG{o}{)}\PYG{o}{;}
                \PYG{n}{Transformer} \PYG{n}{transformador} \PYG{o}{=}
                                \PYG{n}{fabricaConv}\PYG{o}{.}\PYG{n+na}{newTransformer}\PYG{o}{(}\PYG{o}{)}\PYG{o}{;}
                \PYG{n}{DOMSource} \PYG{n}{origenDOM} \PYG{o}{=}
                                \PYG{k}{new} \PYG{n}{DOMSource}\PYG{o}{(}\PYG{n}{documentoXML}\PYG{o}{)}\PYG{o}{;}
                \PYG{n}{Element} \PYG{n}{e}\PYG{o}{=}\PYG{n}{documentoXML}\PYG{o}{.}\PYG{n+na}{createElement}\PYG{o}{(}\PYG{l+s}{\PYGZdq{}rss\PYGZdq{}}\PYG{o}{)}\PYG{o}{;}
                \PYG{n}{documentoXML}\PYG{o}{.}\PYG{n+na}{appendChild}\PYG{o}{(}\PYG{n}{e}\PYG{o}{)}\PYG{o}{;}
                \PYG{n}{StreamResult} \PYG{n}{resultado}\PYG{o}{=}
                                \PYG{k}{new} \PYG{n}{StreamResult}\PYG{o}{(}
                                        \PYG{k}{new} \PYG{n}{File}\PYG{o}{(}\PYG{l+s}{\PYGZdq{}D:\PYGZbs{}\PYGZbs{}resul\PYGZbs{}\PYGZbs{}archivo.rss\PYGZdq{}}\PYG{o}{)}\PYG{o}{)}\PYG{o}{;}
                \PYG{n}{transformador}\PYG{o}{.}\PYG{n+na}{transform}\PYG{o}{(}\PYG{n}{origenDOM}\PYG{o}{,} \PYG{n}{resultado}\PYG{o}{)}\PYG{o}{;}
        \PYG{o}{\PYGZcb{}}
        \PYG{k}{catch} \PYG{o}{(}\PYG{n}{Exception} \PYG{n}{e}\PYG{o}{)}\PYG{o}{\PYGZob{}}
                \PYG{n}{System}\PYG{o}{.}\PYG{n+na}{out}\PYG{o}{.}\PYG{n+na}{print}\PYG{o}{(}\PYG{l+s}{\PYGZdq{}No se han podido crear los\PYGZdq{}}\PYG{o}{)}\PYG{o}{;}
                \PYG{n}{System}\PYG{o}{.}\PYG{n+na}{out}\PYG{o}{.}\PYG{n+na}{println}\PYG{o}{(}\PYG{l+s}{\PYGZdq{} objetos necesarios.\PYGZdq{}}\PYG{o}{)}\PYG{o}{;}
                \PYG{n}{e}\PYG{o}{.}\PYG{n+na}{printStackTrace}\PYG{o}{(}\PYG{o}{)}\PYG{o}{;}
                \PYG{k}{return} \PYG{o}{;}
        \PYG{o}{\PYGZcb{}}
\PYG{o}{\PYGZcb{}}
\end{sphinxVerbatim}


\section{XPath}
\label{\detokenize{tema7:xpath}}
Según el W3C, XPath (que ya va por su versión 3.0) es un lenguaje diseñado para acceder a las distintas partes de un archivo XML. En nuestro caso nos va a resultar de mucha utilidad combinado con XSLT, que se verá un poco despues.

XPath se basa en expresiones. Así, dado un archivo XML y una expresión XPath se dice que la expresión «se evalúa» y se obtiene un resultado que puede ser:
\begin{itemize}
\item {} 
Una lista de nodos.

\item {} 
Un \sphinxcode{boolean} (true o false)

\item {} 
Un \sphinxcode{float}.

\item {} 
Una cadena.

\end{itemize}

XPath también ofrece algunas funciones de utilidad que se asemejan a las de algunos lenguajes de programación.


\subsection{Acceso a elementos}
\label{\detokenize{tema7:acceso-a-elementos}}
El mecanismo de acceso en XPath es muy similar al acceso a directorios que ofrecen algunos sistemas operativos. Para los ejemplos siguientes se usará el siguiente archivo XML

\begin{sphinxVerbatim}[commandchars=\\\{\}]
\PYG{n+nt}{\PYGZlt{}inventario}\PYG{n+nt}{\PYGZgt{}}
    \PYG{n+nt}{\PYGZlt{}producto} \PYG{n+na}{codigo=}\PYG{l+s}{\PYGZdq{}AAA\PYGZhy{}111\PYGZdq{}}\PYG{n+nt}{\PYGZgt{}}
        \PYG{n+nt}{\PYGZlt{}nombre}\PYG{n+nt}{\PYGZgt{}}Teclado\PYG{n+nt}{\PYGZlt{}/nombre\PYGZgt{}}
        \PYG{n+nt}{\PYGZlt{}peso} \PYG{n+na}{unidad=}\PYG{l+s}{\PYGZdq{}g\PYGZdq{}}\PYG{n+nt}{\PYGZgt{}}480\PYG{n+nt}{\PYGZlt{}/peso\PYGZgt{}}
    \PYG{n+nt}{\PYGZlt{}/producto\PYGZgt{}}
    \PYG{n+nt}{\PYGZlt{}producto} \PYG{n+na}{codigo=}\PYG{l+s}{\PYGZdq{}ACD\PYGZhy{}981\PYGZdq{}}\PYG{n+nt}{\PYGZgt{}}
        \PYG{n+nt}{\PYGZlt{}nombre}\PYG{n+nt}{\PYGZgt{}}Monitor\PYG{n+nt}{\PYGZlt{}/nombre\PYGZgt{}}
        \PYG{n+nt}{\PYGZlt{}peso} \PYG{n+na}{unidad=}\PYG{l+s}{\PYGZdq{}kg\PYGZdq{}}\PYG{n+nt}{\PYGZgt{}}1.8\PYG{n+nt}{\PYGZlt{}/peso\PYGZgt{}}
    \PYG{n+nt}{\PYGZlt{}/producto\PYGZgt{}}
    \PYG{n+nt}{\PYGZlt{}producto} \PYG{n+na}{codigo=}\PYG{l+s}{\PYGZdq{}DEZ\PYGZhy{}138\PYGZdq{}}\PYG{n+nt}{\PYGZgt{}}
        \PYG{n+nt}{\PYGZlt{}nombre}\PYG{n+nt}{\PYGZgt{}}Raton\PYG{n+nt}{\PYGZlt{}/nombre\PYGZgt{}}
        \PYG{n+nt}{\PYGZlt{}peso} \PYG{n+na}{unidad=}\PYG{l+s}{\PYGZdq{}g\PYGZdq{}}\PYG{n+nt}{\PYGZgt{}}50\PYG{n+nt}{\PYGZlt{}/peso\PYGZgt{}}
    \PYG{n+nt}{\PYGZlt{}/producto\PYGZgt{}}
\PYG{n+nt}{\PYGZlt{}/inventario\PYGZgt{}}
\end{sphinxVerbatim}

Así dado este archivo tenemos las expresiones siguientes:

Si usamos la expresión \sphinxcode{/inventario} se selecciona \sphinxstyleemphasis{el nodo inventario que cuelga de la raíz}. Como puede verse la raíz en XPath es un elemento conceptual, no existe como elemento. Además, dado como es XML solo puede haber un elemento en la raíz. Así, el resultado de evaluar la expresión \sphinxcode{/inventario} para el archivo de ejemplo produce el resultado siguiente:

\begin{sphinxVerbatim}[commandchars=\\\{\}]
\PYG{n+nt}{\PYGZlt{}inventario}\PYG{n+nt}{\PYGZgt{}}
    \PYG{n+nt}{\PYGZlt{}producto} \PYG{n+na}{codigo=}\PYG{l+s}{\PYGZdq{}AAA\PYGZhy{}111\PYGZdq{}}\PYG{n+nt}{\PYGZgt{}}
        \PYG{n+nt}{\PYGZlt{}nombre}\PYG{n+nt}{\PYGZgt{}}Teclado\PYG{n+nt}{\PYGZlt{}/nombre\PYGZgt{}}
        \PYG{n+nt}{\PYGZlt{}peso} \PYG{n+na}{unidad=}\PYG{l+s}{\PYGZdq{}g\PYGZdq{}}\PYG{n+nt}{\PYGZgt{}}480\PYG{n+nt}{\PYGZlt{}/peso\PYGZgt{}}
    \PYG{n+nt}{\PYGZlt{}/producto\PYGZgt{}}
    \PYG{n+nt}{\PYGZlt{}producto} \PYG{n+na}{codigo=}\PYG{l+s}{\PYGZdq{}ACD\PYGZhy{}981\PYGZdq{}}\PYG{n+nt}{\PYGZgt{}}
        \PYG{n+nt}{\PYGZlt{}nombre}\PYG{n+nt}{\PYGZgt{}}Monitor\PYG{n+nt}{\PYGZlt{}/nombre\PYGZgt{}}
        \PYG{n+nt}{\PYGZlt{}peso} \PYG{n+na}{unidad=}\PYG{l+s}{\PYGZdq{}kg\PYGZdq{}}\PYG{n+nt}{\PYGZgt{}}1.8\PYG{n+nt}{\PYGZlt{}/peso\PYGZgt{}}
    \PYG{n+nt}{\PYGZlt{}/producto\PYGZgt{}}
    \PYG{n+nt}{\PYGZlt{}producto} \PYG{n+na}{codigo=}\PYG{l+s}{\PYGZdq{}DEZ\PYGZhy{}138\PYGZdq{}}\PYG{n+nt}{\PYGZgt{}}
        \PYG{n+nt}{\PYGZlt{}nombre}\PYG{n+nt}{\PYGZgt{}}Raton\PYG{n+nt}{\PYGZlt{}/nombre\PYGZgt{}}
        \PYG{n+nt}{\PYGZlt{}peso} \PYG{n+na}{unidad=}\PYG{l+s}{\PYGZdq{}g\PYGZdq{}}\PYG{n+nt}{\PYGZgt{}}50\PYG{n+nt}{\PYGZlt{}/peso\PYGZgt{}}
    \PYG{n+nt}{\PYGZlt{}/producto\PYGZgt{}}
\PYG{n+nt}{\PYGZlt{}/inventario\PYGZgt{}}
\end{sphinxVerbatim}

Como puede verse, obtenemos el propio archivo original. Sin embargo, podemos movernos a través del árbol XML de forma similar a un árbol de directorios. Y obsérvese que decimos «similar». Observemos por ejemplo que dentro de \sphinxcode{\textless{}inventario\textgreater{}} hay 3 elementos \sphinxcode{\textless{}producto\textgreater{}}. Si pensamos en la expresión XPath \sphinxcode{/inventario/producto} puede que pensemos que obtendremos el primer producto (el que tiene el código AAA-111), sin embargo \sphinxstylestrong{una expresión XPath se parece a una consulta SQL}, y lo que obtiene la expresión es «todo elemento \sphinxcode{\textless{}producto\textgreater{}} que sea hijo de \sphinxcode{\textless{}inventario\textgreater{}}. Es decir, el fichero siguiente (que no es XML, sino una lista de nodos):

\begin{sphinxVerbatim}[commandchars=\\\{\}]
\PYG{n+nt}{\PYGZlt{}producto} \PYG{n+na}{codigo=}\PYG{l+s}{\PYGZdq{}AAA\PYGZhy{}111\PYGZdq{}}\PYG{n+nt}{\PYGZgt{}}
    \PYG{n+nt}{\PYGZlt{}nombre}\PYG{n+nt}{\PYGZgt{}}Teclado\PYG{n+nt}{\PYGZlt{}/nombre\PYGZgt{}}
    \PYG{n+nt}{\PYGZlt{}peso} \PYG{n+na}{unidad=}\PYG{l+s}{\PYGZdq{}g\PYGZdq{}}\PYG{n+nt}{\PYGZgt{}}480\PYG{n+nt}{\PYGZlt{}/peso\PYGZgt{}}
\PYG{n+nt}{\PYGZlt{}/producto\PYGZgt{}}


\PYG{n+nt}{\PYGZlt{}producto} \PYG{n+na}{codigo=}\PYG{l+s}{\PYGZdq{}ACD\PYGZhy{}981\PYGZdq{}}\PYG{n+nt}{\PYGZgt{}}
    \PYG{n+nt}{\PYGZlt{}nombre}\PYG{n+nt}{\PYGZgt{}}Monitor\PYG{n+nt}{\PYGZlt{}/nombre\PYGZgt{}}
    \PYG{n+nt}{\PYGZlt{}peso} \PYG{n+na}{unidad=}\PYG{l+s}{\PYGZdq{}kg\PYGZdq{}}\PYG{n+nt}{\PYGZgt{}}1.8\PYG{n+nt}{\PYGZlt{}/peso\PYGZgt{}}
\PYG{n+nt}{\PYGZlt{}/producto\PYGZgt{}}


\PYG{n+nt}{\PYGZlt{}producto} \PYG{n+na}{codigo=}\PYG{l+s}{\PYGZdq{}DEZ\PYGZhy{}138\PYGZdq{}}\PYG{n+nt}{\PYGZgt{}}
    \PYG{n+nt}{\PYGZlt{}nombre}\PYG{n+nt}{\PYGZgt{}}Raton\PYG{n+nt}{\PYGZlt{}/nombre\PYGZgt{}}
    \PYG{n+nt}{\PYGZlt{}peso} \PYG{n+na}{unidad=}\PYG{l+s}{\PYGZdq{}g\PYGZdq{}}\PYG{n+nt}{\PYGZgt{}}50\PYG{n+nt}{\PYGZlt{}/peso\PYGZgt{}}
\PYG{n+nt}{\PYGZlt{}/producto\PYGZgt{}}
\end{sphinxVerbatim}

En cualquier lista podemos acceder a sus elementos como si fuese un vector. Sin embargo en XPath \sphinxstylestrong{los vectores empiezan por 1}. Por lo cual la expresión \sphinxcode{/inventario/producto{[}1{]}} produce este resultado:

\begin{sphinxVerbatim}[commandchars=\\\{\}]
\PYG{n+nt}{\PYGZlt{}producto} \PYG{n+na}{codigo=}\PYG{l+s}{\PYGZdq{}AAA\PYGZhy{}111\PYGZdq{}}\PYG{n+nt}{\PYGZgt{}}
    \PYG{n+nt}{\PYGZlt{}nombre}\PYG{n+nt}{\PYGZgt{}}Teclado\PYG{n+nt}{\PYGZlt{}/nombre\PYGZgt{}}
    \PYG{n+nt}{\PYGZlt{}peso} \PYG{n+na}{unidad=}\PYG{l+s}{\PYGZdq{}g\PYGZdq{}}\PYG{n+nt}{\PYGZgt{}}480\PYG{n+nt}{\PYGZlt{}/peso\PYGZgt{}}
\PYG{n+nt}{\PYGZlt{}/producto\PYGZgt{}}
\end{sphinxVerbatim}

Y la expresión \sphinxcode{/inventario/producto{[}3{]}} produce este:

\begin{sphinxVerbatim}[commandchars=\\\{\}]
\PYG{n+nt}{\PYGZlt{}producto} \PYG{n+na}{codigo=}\PYG{l+s}{\PYGZdq{}DEZ\PYGZhy{}138\PYGZdq{}}\PYG{n+nt}{\PYGZgt{}}
    \PYG{n+nt}{\PYGZlt{}nombre}\PYG{n+nt}{\PYGZgt{}}Raton\PYG{n+nt}{\PYGZlt{}/nombre\PYGZgt{}}
    \PYG{n+nt}{\PYGZlt{}peso} \PYG{n+na}{unidad=}\PYG{l+s}{\PYGZdq{}g\PYGZdq{}}\PYG{n+nt}{\PYGZgt{}}50\PYG{n+nt}{\PYGZlt{}/peso\PYGZgt{}}
\PYG{n+nt}{\PYGZlt{}/producto\PYGZgt{}}
\end{sphinxVerbatim}

Obsérvese que no existe el elemento 4 y que por tanto la expresión \sphinxcode{/inventario/producto{[}4{]}} producirá un error. Otro aspecto relevante es que no deben confundirse los vectores con las condiciones (que el W3C llama «predicados»), y con las cuales podremos seleccionar nodos que cumplan ciertas condiciones De hecho, una buena forma de verlos es asumir que en los corchetes \sphinxstylestrong{siempre se ponen condiciones y que si hay un número como por ejemplo el 2 nos referimos a la condicion «extraer el elemento cuya posición es igual a 2}.

Dado un elemento, también podemos extraer un cierto atributo usando la arroba @. Así, la expresión \sphinxcode{/inventario/producto{[}3{]}/@codigo} devuelve como resultado \sphinxcode{ACD-981}, que es el atributo código del tercer elemento \sphinxcode{producto} que está dentro de \sphinxcode{inventario} el cual cuelga de la raíz.

Supongamos que deseamos extraer el producto cuyo código sea «AAA-111». Si usamos \sphinxcode{/inventario/producto} extraemos todos los elementos producto hijos de inventario, pero recordemos que entre corchetes podemos poner condiciones. Dado que queremos comprobar si @codigo = «AAA-111», la expresión correcta será \sphinxcode{/inventario/producto{[}@codigo="AAA-111"{]}}, la cual nos devuelve lo siguiente:

\begin{sphinxVerbatim}[commandchars=\\\{\}]
\PYG{n+nt}{\PYGZlt{}producto} \PYG{n+na}{codigo=}\PYG{l+s}{\PYGZdq{}AAA\PYGZhy{}111\PYGZdq{}}\PYG{n+nt}{\PYGZgt{}}
    \PYG{n+nt}{\PYGZlt{}nombre}\PYG{n+nt}{\PYGZgt{}}Teclado\PYG{n+nt}{\PYGZlt{}/nombre\PYGZgt{}}
    \PYG{n+nt}{\PYGZlt{}peso} \PYG{n+na}{unidad=}\PYG{l+s}{\PYGZdq{}g\PYGZdq{}}\PYG{n+nt}{\PYGZgt{}}480\PYG{n+nt}{\PYGZlt{}/peso\PYGZgt{}}
\PYG{n+nt}{\PYGZlt{}/producto\PYGZgt{}}
\end{sphinxVerbatim}

De hecho se puede profundizar aún más y usar la expresión \sphinxcode{/inventario/producto{[}@codigo="AAA-111"{]}/nombre} que extrae los nombres de los elementos producto cuyo código sea «AAA-111». Y aún más para extraer solo el texto de los elementos nombre usando la expresión \sphinxcode{/inventario/producto{[}@codigo="AAA-111"{]}/nombre/text()}. Como vemos en esta última expresión ya hemos usado una función, en concreto \sphinxcode{text()}.

En una condicion podemos referirnos a cualquier hijo de un nodo, así por ejemplo, podemos extraer los productos cuyo peso sea mayor de 50 usando \sphinxcode{/inventario/producto{[}peso\textgreater{}=50{]}}. Sin embargo, sabemos que la unidad es importante, por lo que en realidad podemos extraer los que pesen más de 50 gramos usando esto \sphinxcode{/inventario/producto{[}peso\textgreater{}=50 and peso/@unidad="g"{]}}.

Si se observa despacio el fichero, se observará que en realidad el tercer producto debería aparecer también. Para ello debemos ampliar la expresión convirtiendo los 50 g a kg, es decir comparando con 0.005 kg y la expresión siguiente \sphinxcode{/inventario/producto{[}(peso\textgreater{}=50 and peso/@unidad="g") or (peso\textgreater{}=0.005 and peso/@unidad="kg"){]}}.

Utilizando XPath y XSLT veremos que podemos transformar un XML en casi cualquier otro XML utilizando la potencia combinada de ambos lenguajes.


\section{Adaptación y transformación de XML}
\label{\detokenize{tema7:adaptacion-y-transformacion-de-xml}}
Muy a menudo va a ocurrir que un cierto formato XML va a ampliarse o a modificarse o simplemente se necesita convertir un documento XML en otro con un formato distinto.

Supongamos una estructura como la siguiente:

\begin{sphinxVerbatim}[commandchars=\\\{\}]
\PYG{c+cp}{\PYGZlt{}?xml version=\PYGZdq{}1.0\PYGZdq{} encoding=\PYGZdq{}UTF\PYGZhy{}8\PYGZdq{}?\PYGZgt{}}
\PYG{n+nt}{\PYGZlt{}catalogo}\PYG{n+nt}{\PYGZgt{}}
        \PYG{n+nt}{\PYGZlt{}libro}\PYG{n+nt}{\PYGZgt{}}
                \PYG{n+nt}{\PYGZlt{}title}\PYG{n+nt}{\PYGZgt{}}Don Quijote\PYG{n+nt}{\PYGZlt{}/title\PYGZgt{}}
                \PYG{n+nt}{\PYGZlt{}autor}\PYG{n+nt}{\PYGZgt{}}Cervantes\PYG{n+nt}{\PYGZlt{}/autor\PYGZgt{}}
        \PYG{n+nt}{\PYGZlt{}/libro\PYGZgt{}}
        \PYG{n+nt}{\PYGZlt{}libro}\PYG{n+nt}{\PYGZgt{}}
                \PYG{n+nt}{\PYGZlt{}title}\PYG{n+nt}{\PYGZgt{}}Poeta en Nueva York\PYG{n+nt}{\PYGZlt{}/title\PYGZgt{}}
                \PYG{n+nt}{\PYGZlt{}autor}\PYG{n+nt}{\PYGZgt{}}Lorca\PYG{n+nt}{\PYGZlt{}/autor\PYGZgt{}}
        \PYG{n+nt}{\PYGZlt{}/libro\PYGZgt{}}
\PYG{n+nt}{\PYGZlt{}/catalogo\PYGZgt{}}
\end{sphinxVerbatim}

Supongamos que un cierto sitio se necesita almacenar la información de esta forma:

\begin{sphinxVerbatim}[commandchars=\\\{\}]
\PYG{c+cp}{\PYGZlt{}?xml version=\PYGZdq{}1.0\PYGZdq{} encoding=\PYGZdq{}UTF\PYGZhy{}8\PYGZdq{}?\PYGZgt{}}
\PYG{n+nt}{\PYGZlt{}listadolibros}\PYG{n+nt}{\PYGZgt{}}
        \PYG{n+nt}{\PYGZlt{}libro}\PYG{n+nt}{\PYGZgt{}}
                \PYG{n+nt}{\PYGZlt{}titulo} \PYG{n+na}{autor=}\PYG{l+s}{\PYGZdq{}Cervantes\PYGZdq{}}\PYG{n+nt}{\PYGZgt{}}Don Quijote\PYG{n+nt}{\PYGZlt{}/titulo\PYGZgt{}}
        \PYG{n+nt}{\PYGZlt{}/libro\PYGZgt{}}
        \PYG{n+nt}{\PYGZlt{}libro}\PYG{n+nt}{\PYGZgt{}}
                \PYG{n+nt}{\PYGZlt{}titulo} \PYG{n+na}{autor=}\PYG{l+s}{\PYGZdq{}Lorca\PYGZdq{}}\PYG{n+nt}{\PYGZgt{}}
                Poeta en Nueva York
                \PYG{n+nt}{\PYGZlt{}/titulo\PYGZgt{}}
        \PYG{n+nt}{\PYGZlt{}/libro\PYGZgt{}}
\PYG{n+nt}{\PYGZlt{}/listadolibros\PYGZgt{}}
\end{sphinxVerbatim}

En general, para poder modificar o presentar los XML se puede hacen varias cosas:
\begin{itemize}
\item {} 
En primer lugar, se puede usar CSS para poder cargar los documentos XML en un navegador y mostrarlos de forma aceptable.

\item {} 
Se pueden utilizar otras tecnologías para transformar por completo la estructura del XML.
\begin{itemize}
\item {} 
Se puede usar un lenguaje llamado XSLT (Xml Stylesheet Language Transformation) para convertir el XML en otro distinto.

\item {} 
Se puede utilizar XSL:FO (Xml Stylesheet Language: Formatting Objects) cuando se desee convertir el documento en algo que se desee imprimir (normalmente un PDF)

\end{itemize}

\end{itemize}


\subsection{CSS con XML}
\label{\detokenize{tema7:css-con-xml}}
Supongamos de nuevo el archivo anterior, el cual ahora queremos mostrar en un navegador:

\begin{sphinxVerbatim}[commandchars=\\\{\}]
\PYG{c+cp}{\PYGZlt{}?xml version=\PYGZdq{}1.0\PYGZdq{} encoding=\PYGZdq{}UTF\PYGZhy{}8\PYGZdq{}?\PYGZgt{}}
\PYG{n+nt}{\PYGZlt{}catalogo}\PYG{n+nt}{\PYGZgt{}}
        \PYG{n+nt}{\PYGZlt{}libro}\PYG{n+nt}{\PYGZgt{}}
                \PYG{n+nt}{\PYGZlt{}title}\PYG{n+nt}{\PYGZgt{}}Don Quijote\PYG{n+nt}{\PYGZlt{}/title\PYGZgt{}}
                \PYG{n+nt}{\PYGZlt{}autor}\PYG{n+nt}{\PYGZgt{}}Cervantes\PYG{n+nt}{\PYGZlt{}/autor\PYGZgt{}}
        \PYG{n+nt}{\PYGZlt{}/libro\PYGZgt{}}
        \PYG{n+nt}{\PYGZlt{}libro}\PYG{n+nt}{\PYGZgt{}}
                \PYG{n+nt}{\PYGZlt{}title}\PYG{n+nt}{\PYGZgt{}}
                Poeta en Nueva York
                \PYG{n+nt}{\PYGZlt{}/title\PYGZgt{}}
                \PYG{n+nt}{\PYGZlt{}autor}\PYG{n+nt}{\PYGZgt{}}Lorca\PYG{n+nt}{\PYGZlt{}/autor\PYGZgt{}}
        \PYG{n+nt}{\PYGZlt{}/libro\PYGZgt{}}
\PYG{n+nt}{\PYGZlt{}/catalogo\PYGZgt{}}
\end{sphinxVerbatim}

Si usamos el archivo \sphinxcode{estilo.css} de esta forma:

\begin{sphinxVerbatim}[commandchars=\\\{\}]
\PYG{n+nt}{catalogo}\PYG{p}{\PYGZob{}}
        \PYG{k}{background\PYGZhy{}color}\PYG{p}{:}\PYG{n+nb}{rgb}\PYG{p}{(}\PYG{l+m+mi}{220}\PYG{p}{,} \PYG{l+m+mi}{230}\PYG{p}{,} \PYG{l+m+mi}{220}\PYG{p}{)}\PYG{p}{;}
        \PYG{k}{display}\PYG{p}{:}\PYG{k+kc}{block}\PYG{p}{;}
\PYG{p}{\PYGZcb{}}

\PYG{n+nt}{libro}\PYG{p}{\PYGZob{}}
        \PYG{k}{display}\PYG{p}{:}\PYG{k+kc}{block}\PYG{p}{;}
        \PYG{k}{border}\PYG{p}{:} \PYG{k+kc}{solid} \PYG{k+kc}{black} \PYG{l+m+mi}{1}\PYG{k+kt}{px}\PYG{p}{;}
        \PYG{k}{margin\PYGZhy{}bottom}\PYG{p}{:}\PYG{l+m+mi}{20}\PYG{k+kt}{px}\PYG{p}{;}
\PYG{p}{\PYGZcb{}}
\PYG{n+nt}{title}\PYG{p}{\PYGZob{}}
        \PYG{k}{margin}\PYG{p}{:} \PYG{l+m+mi}{10}\PYG{k+kt}{px}\PYG{p}{;}
        \PYG{k}{display}\PYG{p}{:}\PYG{k+kc}{block}\PYG{p}{;}
\PYG{p}{\PYGZcb{}}
\PYG{n+nt}{autor}\PYG{p}{\PYGZob{}}
        \PYG{k}{display}\PYG{p}{:}\PYG{k+kc}{block}\PYG{p}{;}
        \PYG{n}{font\PYGZhy{}face}\PYG{p}{:}\PYG{n}{Arial}\PYG{p}{;}
        \PYG{k}{text\PYGZhy{}decoration}\PYG{p}{:}\PYG{k+kc}{underline}\PYG{p}{;}
\PYG{p}{\PYGZcb{}}
\end{sphinxVerbatim}

Veremos algo como esto:

\noindent{\hspace*{\fill}\sphinxincludegraphics[scale=0.5]{{estilo-xml}.png}\hspace*{\fill}}


\subsubsection{Ejercicio}
\label{\detokenize{tema7:id1}}
Crear una hoja de estilo asociada al catálogo anterior, que muestre la información de cada libro de forma parecida a una tabla, en la que el \sphinxcode{title} utilice un color de fondo distinto del \sphinxcode{autor}.

\noindent{\hspace*{\fill}\sphinxincludegraphics[scale=0.5]{{estilo-xml2}.png}\hspace*{\fill}}

\begin{sphinxVerbatim}[commandchars=\\\{\}]
\PYG{n+nt}{catalogo}\PYG{p}{\PYGZob{}}
        \PYG{k}{background\PYGZhy{}color}\PYG{p}{:}\PYG{n+nb}{rgb}\PYG{p}{(}\PYG{l+m+mi}{220}\PYG{p}{,} \PYG{l+m+mi}{230}\PYG{p}{,} \PYG{l+m+mi}{220}\PYG{p}{)}\PYG{p}{;}
        \PYG{k}{display}\PYG{p}{:}\PYG{k+kc}{block}\PYG{p}{;}
\PYG{p}{\PYGZcb{}}

\PYG{n+nt}{libro}\PYG{p}{\PYGZob{}}
        \PYG{k}{display}\PYG{p}{:}\PYG{k+kc}{block}\PYG{p}{;}
        \PYG{k}{width}\PYG{p}{:}\PYG{l+m+mi}{100}\PYG{k+kt}{\PYGZpc{}}\PYG{p}{;}
        \PYG{k}{margin\PYGZhy{}bottom}\PYG{p}{:}\PYG{l+m+mi}{40}\PYG{k+kt}{px}\PYG{p}{;}
        \PYG{k}{clear}\PYG{p}{:}\PYG{k+kc}{both}\PYG{p}{;}
\PYG{p}{\PYGZcb{}}
\PYG{n+nt}{title}\PYG{p}{\PYGZob{}}
        \PYG{k}{float}\PYG{p}{:}\PYG{k+kc}{left}\PYG{p}{;}
        \PYG{k}{width}\PYG{p}{:}\PYG{l+m+mi}{45}\PYG{k+kt}{\PYGZpc{}}\PYG{p}{;}
        \PYG{k}{border}\PYG{p}{:}\PYG{k+kc}{solid} \PYG{k+kc}{black} \PYG{l+m+mi}{1}\PYG{k+kt}{px}\PYG{p}{;}
        \PYG{k}{padding}\PYG{p}{:}\PYG{l+m+mi}{5}\PYG{k+kt}{px}\PYG{p}{;}
        \PYG{k}{text\PYGZhy{}align}\PYG{p}{:}\PYG{k+kc}{center}\PYG{p}{;}
        \PYG{k}{background\PYGZhy{}color}\PYG{p}{:}\PYG{n+nb}{rgb}\PYG{p}{(}\PYG{l+m+mi}{180}\PYG{p}{,}\PYG{l+m+mi}{180}\PYG{p}{,}\PYG{l+m+mi}{240}\PYG{p}{)}\PYG{p}{;}
\PYG{p}{\PYGZcb{}}
\PYG{n+nt}{autor}\PYG{p}{\PYGZob{}}
        \PYG{k}{float}\PYG{p}{:}\PYG{k+kc}{left}\PYG{p}{;}
        \PYG{k}{text\PYGZhy{}align}\PYG{p}{:}\PYG{k+kc}{center}\PYG{p}{;}
        \PYG{k}{width}\PYG{p}{:}\PYG{l+m+mi}{45}\PYG{k+kt}{\PYGZpc{}}\PYG{p}{;}
        \PYG{k}{border}\PYG{p}{:}\PYG{k+kc}{solid} \PYG{k+kc}{black} \PYG{l+m+mi}{1}\PYG{k+kt}{px}\PYG{p}{;}
        \PYG{k}{padding}\PYG{p}{:}\PYG{l+m+mi}{5}\PYG{k+kt}{px}\PYG{p}{;}
        \PYG{k}{background\PYGZhy{}color}\PYG{p}{:}\PYG{n+nb}{rgb}\PYG{p}{(}\PYG{l+m+mi}{340}\PYG{p}{,}\PYG{l+m+mi}{180}\PYG{p}{,}\PYG{l+m+mi}{240}\PYG{p}{)}\PYG{p}{;}
\PYG{p}{\PYGZcb{}}
\end{sphinxVerbatim}


\subsection{Transformación de XML}
\label{\detokenize{tema7:transformacion-de-xml}}
Si deseamos \sphinxstyleemphasis{transformar un XML en otro XML} necesitaremos usar XSLT. Un archivo XSLT tiene la extensión XSL e indica las reglas para convertir entre formatos XML.

El documento XSL básico sería así (los navegadores la darán por mala, ya que no hace absolutamente
nada):

\begin{sphinxVerbatim}[commandchars=\\\{\}]
\PYG{c+cp}{\PYGZlt{}?xml version=\PYGZdq{}1.0\PYGZdq{} encoding=\PYGZdq{}UTF\PYGZhy{}8\PYGZdq{}?\PYGZgt{}}
\PYG{n+nt}{\PYGZlt{}xsl:stylesheet}\PYG{n+nt}{\PYGZgt{}}
\PYG{n+nt}{\PYGZlt{}/xsl:stylesheet\PYGZgt{}}
\end{sphinxVerbatim}

En este caso xsl es el espacio de nombres. Un espacio de nombres es un contenedor que
permite evitar que haya confusiones entre unas etiquetas y otras que se llamen igual. En este
caso, queremos usar la etiqueta \textless{}stylesheet\textgreater{} definida por el W3C. Una hoja básica sería
esta

\begin{sphinxVerbatim}[commandchars=\\\{\}]
\PYG{c+cp}{\PYGZlt{}?xml version=\PYGZdq{}1.0\PYGZdq{} encoding=\PYGZdq{}UTF\PYGZhy{}8\PYGZdq{}?\PYGZgt{}}
\PYG{n+nt}{\PYGZlt{}xsl:stylesheet} \PYG{n+na}{version=}\PYG{l+s}{\PYGZdq{}1.0\PYGZdq{}}
        \PYG{n+na}{xmlns:xsl=}
        \PYG{l+s}{\PYGZdq{}http://www.w3.org/1999/XSL/Transform\PYGZdq{}}\PYG{n+nt}{\PYGZgt{}}
        \PYG{n+nt}{\PYGZlt{}xsl:template} \PYG{n+na}{match=}\PYG{l+s}{\PYGZdq{}/\PYGZdq{}}\PYG{n+nt}{\PYGZgt{}}
                \PYG{n+nt}{\PYGZlt{}html}\PYG{n+nt}{\PYGZgt{}}
                        \PYG{n+nt}{\PYGZlt{}head}\PYG{n+nt}{\PYGZgt{}}
                                \PYG{n+nt}{\PYGZlt{}title}\PYG{n+nt}{\PYGZgt{}}
                                        Resultado
                                \PYG{n+nt}{\PYGZlt{}/title\PYGZgt{}}
                        \PYG{n+nt}{\PYGZlt{}/head\PYGZgt{}}
                        \PYG{n+nt}{\PYGZlt{}body}\PYG{n+nt}{\PYGZgt{}}
                                Documento resultado
                        \PYG{n+nt}{\PYGZlt{}/body\PYGZgt{}}
                \PYG{n+nt}{\PYGZlt{}/html\PYGZgt{}}
        \PYG{n+nt}{\PYGZlt{}/xsl:template\PYGZgt{}}
\PYG{n+nt}{\PYGZlt{}/xsl:stylesheet\PYGZgt{}}
\end{sphinxVerbatim}

Algunos navegadores no ejecutan XSL por seguridad. Los detalles de como “abrir” la seguridad de cada uno de estos navegadores deben investigarse en el manual de cada uno de ellos.

Cabe destacar que esta hoja simplemente genera HTML básico pero no recoge ningún dato del XML original.


\subsection{Ejercicio (carga de estilos)}
\label{\detokenize{tema7:ejercicio-carga-de-estilos}}
Hacer que el archivo XML de libros cargue esta hoja de estilos.

Solución: consiste en añadir una línea al archivo que referencie el archivo de transformación y el tipo de lenguaje usado para transformar.

\begin{sphinxVerbatim}[commandchars=\\\{\}]
\PYG{c+cp}{\PYGZlt{}?xml version=\PYGZdq{}1.0\PYGZdq{} encoding=\PYGZdq{}UTF\PYGZhy{}8\PYGZdq{}?\PYGZgt{}}
\PYG{c+cp}{\PYGZlt{}?xml\PYGZhy{}stylesheet href=\PYGZdq{}hoja1.xsl\PYGZdq{} type=\PYGZdq{}text/xsl\PYGZdq{}?\PYGZgt{}}

\PYG{n+nt}{\PYGZlt{}catalogo}\PYG{n+nt}{\PYGZgt{}}
        ... (El resto es igual)
\PYG{n+nt}{\PYGZlt{}/catalogo\PYGZgt{}}
\end{sphinxVerbatim}


\subsection{Ejercicio (conversion entre XMLs)}
\label{\detokenize{tema7:ejercicio-conversion-entre-xmls}}
Dado el fichero de información del catálogo, transformar dicho XML en otro fichero en el que la etiqueta \sphinxcode{title} vaya en español, es decir, que el resultado quede así:

\begin{sphinxVerbatim}[commandchars=\\\{\}]
\PYG{n+nt}{\PYGZlt{}catalogo}\PYG{n+nt}{\PYGZgt{}}
  \PYG{n+nt}{\PYGZlt{}libro}\PYG{n+nt}{\PYGZgt{}}
    \PYG{n+nt}{\PYGZlt{}title}\PYG{n+nt}{\PYGZgt{}}Don Quijote\PYG{n+nt}{\PYGZlt{}/title\PYGZgt{}}
    \PYG{n+nt}{\PYGZlt{}autor}\PYG{n+nt}{\PYGZgt{}}Cervantes\PYG{n+nt}{\PYGZlt{}/autor\PYGZgt{}}
  \PYG{n+nt}{\PYGZlt{}/libro\PYGZgt{}}
  \PYG{n+nt}{\PYGZlt{}libro}\PYG{n+nt}{\PYGZgt{}}
    \PYG{n+nt}{\PYGZlt{}title}\PYG{n+nt}{\PYGZgt{}}
    Poeta en Nueva York
    \PYG{n+nt}{\PYGZlt{}/title\PYGZgt{}}
    \PYG{n+nt}{\PYGZlt{}autor}\PYG{n+nt}{\PYGZgt{}}Lorca\PYG{n+nt}{\PYGZlt{}/autor\PYGZgt{}}
  \PYG{n+nt}{\PYGZlt{}/libro\PYGZgt{}}
\PYG{n+nt}{\PYGZlt{}/catalogo\PYGZgt{}}
\end{sphinxVerbatim}

La solución podría ser algo así:

\begin{sphinxVerbatim}[commandchars=\\\{\}]
\PYG{n+nt}{\PYGZlt{}xsl:stylesheet}
    \PYG{n+na}{version=}\PYG{l+s}{\PYGZdq{}1.0\PYGZdq{}}
    \PYG{n+na}{xmlns:xsl=}\PYG{l+s}{\PYGZdq{}http://www.w3.org/1999/XSL/Transform\PYGZdq{}}\PYG{n+nt}{\PYGZgt{}}
  \PYG{n+nt}{\PYGZlt{}xsl:template} \PYG{n+na}{match=}\PYG{l+s}{\PYGZdq{}/\PYGZdq{}}\PYG{n+nt}{\PYGZgt{}}
    \PYG{n+nt}{\PYGZlt{}catalogo}\PYG{n+nt}{\PYGZgt{}}
      \PYG{n+nt}{\PYGZlt{}xsl:for\PYGZhy{}each} \PYG{n+na}{select=}\PYG{l+s}{\PYGZdq{}/catalogo/libro\PYGZdq{}}\PYG{n+nt}{\PYGZgt{}}
        \PYG{n+nt}{\PYGZlt{}libro}\PYG{n+nt}{\PYGZgt{}}
          \PYG{n+nt}{\PYGZlt{}titulo}\PYG{n+nt}{\PYGZgt{}}
            \PYG{n+nt}{\PYGZlt{}xsl:value\PYGZhy{}of} \PYG{n+na}{select=}\PYG{l+s}{\PYGZdq{}title\PYGZdq{}}\PYG{n+nt}{/\PYGZgt{}}
          \PYG{n+nt}{\PYGZlt{}/titulo\PYGZgt{}}
          \PYG{n+nt}{\PYGZlt{}autor}\PYG{n+nt}{\PYGZgt{}}
            \PYG{n+nt}{\PYGZlt{}xsl:value\PYGZhy{}of} \PYG{n+na}{select=}\PYG{l+s}{\PYGZdq{}autor\PYGZdq{}}\PYG{n+nt}{/\PYGZgt{}}
          \PYG{n+nt}{\PYGZlt{}/autor\PYGZgt{}}
        \PYG{n+nt}{\PYGZlt{}/libro\PYGZgt{}}
      \PYG{n+nt}{\PYGZlt{}/xsl:for\PYGZhy{}each\PYGZgt{}}
    \PYG{n+nt}{\PYGZlt{}/catalogo\PYGZgt{}}
  \PYG{n+nt}{\PYGZlt{}/xsl:template\PYGZgt{}}
\PYG{n+nt}{\PYGZlt{}/xsl:stylesheet\PYGZgt{}}
\end{sphinxVerbatim}


\subsection{Ejercicio (generación de atributos)}
\label{\detokenize{tema7:ejercicio-generacion-de-atributos}}
Dado el archivo XML del catálogo generar un XML en el que el autor vaya como un atributo del título, es decir, que quede algo así:

\begin{sphinxVerbatim}[commandchars=\\\{\}]
\PYG{n+nt}{\PYGZlt{}catalogo}\PYG{n+nt}{\PYGZgt{}}
  \PYG{n+nt}{\PYGZlt{}libro}\PYG{n+nt}{\PYGZgt{}}
    \PYG{n+nt}{\PYGZlt{}titulo} \PYG{n+na}{escritor=}\PYG{l+s}{\PYGZdq{}Cervantes\PYGZdq{}}\PYG{n+nt}{\PYGZgt{}}Don Quijote\PYG{n+nt}{\PYGZlt{}/titulo\PYGZgt{}}
  \PYG{n+nt}{\PYGZlt{}/libro\PYGZgt{}}
  \PYG{n+nt}{\PYGZlt{}libro}\PYG{n+nt}{\PYGZgt{}}
    \PYG{n+nt}{\PYGZlt{}titulo} \PYG{n+na}{escritor=}\PYG{l+s}{\PYGZdq{}Lorca\PYGZdq{}}\PYG{n+nt}{\PYGZgt{}}
    Poeta en Nueva York
    \PYG{n+nt}{\PYGZlt{}/titulo\PYGZgt{}}
  \PYG{n+nt}{\PYGZlt{}/libro\PYGZgt{}}
\PYG{n+nt}{\PYGZlt{}/catalogo\PYGZgt{}}
\end{sphinxVerbatim}

La solución:

\begin{sphinxVerbatim}[commandchars=\\\{\}]
\PYG{n+nt}{\PYGZlt{}xsl:stylesheet}
    \PYG{n+na}{version=}\PYG{l+s}{\PYGZdq{}1.0\PYGZdq{}}
    \PYG{n+na}{xmlns:xsl=}\PYG{l+s}{\PYGZdq{}http://www.w3.org/1999/XSL/Transform\PYGZdq{}}\PYG{n+nt}{\PYGZgt{}}

  \PYG{n+nt}{\PYGZlt{}xsl:template} \PYG{n+na}{match=}\PYG{l+s}{\PYGZdq{}/\PYGZdq{}}\PYG{n+nt}{\PYGZgt{}}
    \PYG{n+nt}{\PYGZlt{}catalogo}\PYG{n+nt}{\PYGZgt{}}
      \PYG{n+nt}{\PYGZlt{}xsl:for\PYGZhy{}each} \PYG{n+na}{select=}\PYG{l+s}{\PYGZdq{}/catalogo/libro\PYGZdq{}}\PYG{n+nt}{\PYGZgt{}}
        \PYG{n+nt}{\PYGZlt{}libro}\PYG{n+nt}{\PYGZgt{}}
          \PYG{n+nt}{\PYGZlt{}titulo}\PYG{n+nt}{\PYGZgt{}}
            \PYG{n+nt}{\PYGZlt{}xsl:attribute} \PYG{n+na}{name=}\PYG{l+s}{\PYGZdq{}escritor\PYGZdq{}}\PYG{n+nt}{\PYGZgt{}}
              \PYG{n+nt}{\PYGZlt{}xsl:value\PYGZhy{}of} \PYG{n+na}{select=}\PYG{l+s}{\PYGZdq{}autor\PYGZdq{}}\PYG{n+nt}{/\PYGZgt{}}
            \PYG{n+nt}{\PYGZlt{}/xsl:attribute\PYGZgt{}}
            \PYG{n+nt}{\PYGZlt{}xsl:value\PYGZhy{}of} \PYG{n+na}{select=}\PYG{l+s}{\PYGZdq{}title\PYGZdq{}}\PYG{n+nt}{/\PYGZgt{}}
          \PYG{n+nt}{\PYGZlt{}/titulo\PYGZgt{}}
        \PYG{n+nt}{\PYGZlt{}/libro\PYGZgt{}}
      \PYG{n+nt}{\PYGZlt{}/xsl:for\PYGZhy{}each\PYGZgt{}}
    \PYG{n+nt}{\PYGZlt{}/catalogo\PYGZgt{}}
  \PYG{n+nt}{\PYGZlt{}/xsl:template\PYGZgt{}}
\PYG{n+nt}{\PYGZlt{}/xsl:stylesheet\PYGZgt{}}
\end{sphinxVerbatim}


\subsection{Ejercicio (tabla HTML)}
\label{\detokenize{tema7:ejercicio-tabla-html}}
Convertir el catalogo XML en una tabla HTML

Solución:

\begin{sphinxVerbatim}[commandchars=\\\{\}]
\PYG{n+nt}{\PYGZlt{}xsl:stylesheet} \PYG{n+na}{xmlns:xsl=}\PYG{l+s}{\PYGZdq{}http://www.w3.org/1999/XSL/Transform\PYGZdq{}}\PYG{n+nt}{\PYGZgt{}}
  \PYG{n+nt}{\PYGZlt{}xsl:template} \PYG{n+na}{match=}\PYG{l+s}{\PYGZdq{}/\PYGZdq{}}\PYG{n+nt}{\PYGZgt{}}
    \PYG{n+nt}{\PYGZlt{}html}\PYG{n+nt}{\PYGZgt{}}
      \PYG{n+nt}{\PYGZlt{}head}\PYG{n+nt}{\PYGZgt{}}
        \PYG{n+nt}{\PYGZlt{}title}\PYG{n+nt}{\PYGZgt{}}Catalogo de libros\PYG{n+nt}{\PYGZlt{}/title\PYGZgt{}}
      \PYG{n+nt}{\PYGZlt{}/head\PYGZgt{}}
      \PYG{n+nt}{\PYGZlt{}body}\PYG{n+nt}{\PYGZgt{}}
        \PYG{n+nt}{\PYGZlt{}h1}\PYG{n+nt}{\PYGZgt{}}Listado de libros\PYG{n+nt}{\PYGZlt{}/h1\PYGZgt{}}
        \PYG{n+nt}{\PYGZlt{}table} \PYG{n+na}{border=}\PYG{l+s}{\PYGZdq{}1\PYGZdq{}}\PYG{n+nt}{\PYGZgt{}}
          \PYG{n+nt}{\PYGZlt{}xsl:for\PYGZhy{}each} \PYG{n+na}{select=}\PYG{l+s}{\PYGZdq{}catalogo/libro\PYGZdq{}}\PYG{n+nt}{\PYGZgt{}}
            \PYG{n+nt}{\PYGZlt{}tr}\PYG{n+nt}{\PYGZgt{}}
              \PYG{n+nt}{\PYGZlt{}td}\PYG{n+nt}{\PYGZgt{}}
                \PYG{n+nt}{\PYGZlt{}xsl:value\PYGZhy{}of} \PYG{n+na}{select=}\PYG{l+s}{\PYGZdq{}title\PYGZdq{}}\PYG{n+nt}{/\PYGZgt{}}
              \PYG{n+nt}{\PYGZlt{}/td\PYGZgt{}}
              \PYG{n+nt}{\PYGZlt{}td}\PYG{n+nt}{\PYGZgt{}}
                \PYG{n+nt}{\PYGZlt{}xsl:value\PYGZhy{}of} \PYG{n+na}{select=}\PYG{l+s}{\PYGZdq{}autor\PYGZdq{}}\PYG{n+nt}{/\PYGZgt{}}
              \PYG{n+nt}{\PYGZlt{}/td\PYGZgt{}}
            \PYG{n+nt}{\PYGZlt{}/tr\PYGZgt{}}
          \PYG{n+nt}{\PYGZlt{}/xsl:for\PYGZhy{}each\PYGZgt{}}
        \PYG{n+nt}{\PYGZlt{}/table\PYGZgt{}}
      \PYG{n+nt}{\PYGZlt{}/body\PYGZgt{}}
    \PYG{n+nt}{\PYGZlt{}/html\PYGZgt{}}
  \PYG{n+nt}{\PYGZlt{}/xsl:template\PYGZgt{}}
\PYG{n+nt}{\PYGZlt{}/xsl:stylesheet\PYGZgt{}}
\end{sphinxVerbatim}


\subsection{Ejercicio (generacion)}
\label{\detokenize{tema7:ejercicio-generacion}}
Hacer que el XSL genere un HTML con información del archivo XML de libro.

\begin{sphinxVerbatim}[commandchars=\\\{\}]
\PYG{c+cp}{\PYGZlt{}?xml version=\PYGZdq{}1.0\PYGZdq{}?\PYGZgt{}}
\PYG{n+nt}{\PYGZlt{}xsl:stylesheet} \PYG{n+na}{version=}\PYG{l+s}{\PYGZdq{}1.1\PYGZdq{}} \PYG{n+na}{xmlns:xsl=}
                \PYG{l+s}{\PYGZdq{}http://www.w3.org/1999/XSL/Transform\PYGZdq{}}\PYG{n+nt}{\PYGZgt{}}
        \PYG{n+nt}{\PYGZlt{}xsl:template} \PYG{n+na}{match=}\PYG{l+s}{\PYGZdq{}/\PYGZdq{}}\PYG{n+nt}{\PYGZgt{}}
                \PYG{n+nt}{\PYGZlt{}html}\PYG{n+nt}{\PYGZgt{}}
                        \PYG{n+nt}{\PYGZlt{}head}\PYG{n+nt}{\PYGZgt{}}
                                \PYG{n+nt}{\PYGZlt{}title}\PYG{n+nt}{\PYGZgt{}}Ejemplo de transformación\PYG{n+nt}{\PYGZlt{}/title\PYGZgt{}}
                        \PYG{n+nt}{\PYGZlt{}/head\PYGZgt{}}
                        \PYG{n+nt}{\PYGZlt{}body}\PYG{n+nt}{\PYGZgt{}}
                                \PYG{n+nt}{\PYGZlt{}h1}\PYG{n+nt}{\PYGZgt{}}Resultado\PYG{n+nt}{\PYGZlt{}/h1\PYGZgt{}}
                                \PYG{n+nt}{\PYGZlt{}xsl:for\PYGZhy{}each} \PYG{n+na}{select=}\PYG{l+s}{\PYGZdq{}catalogo/libro\PYGZdq{}}\PYG{n+nt}{\PYGZgt{}}
                                        \PYG{n+nt}{\PYGZlt{}p}\PYG{n+nt}{\PYGZgt{}}
                                                \PYG{n+nt}{\PYGZlt{}xsl:value\PYGZhy{}of} \PYG{n+na}{select=}\PYG{l+s}{\PYGZdq{}title\PYGZdq{}}\PYG{n+nt}{/\PYGZgt{}}
                                        \PYG{n+nt}{\PYGZlt{}/p\PYGZgt{}}
                                \PYG{n+nt}{\PYGZlt{}/xsl:for\PYGZhy{}each\PYGZgt{}}
                        \PYG{n+nt}{\PYGZlt{}/body\PYGZgt{}}
                \PYG{n+nt}{\PYGZlt{}/html\PYGZgt{}}
        \PYG{n+nt}{\PYGZlt{}/xsl:template\PYGZgt{}}
\PYG{n+nt}{\PYGZlt{}/xsl:stylesheet\PYGZgt{}}
\end{sphinxVerbatim}


\subsection{Ejercicio}
\label{\detokenize{tema7:id2}}
Extraer los títulos de los libros pero consiguiendo encerrarlos en una lista ordenada HTML para que aparezcan numerados.

\begin{sphinxVerbatim}[commandchars=\\\{\}]
\PYG{c+cp}{\PYGZlt{}?xml version=\PYGZdq{}1.0\PYGZdq{}?\PYGZgt{}}
\PYG{n+nt}{\PYGZlt{}xsl:stylesheet} \PYG{n+na}{version=}\PYG{l+s}{\PYGZdq{}1.1\PYGZdq{}} \PYG{n+na}{xmlns:xsl=}
                \PYG{l+s}{\PYGZdq{}http://www.w3.org/1999/XSL/Transform\PYGZdq{}}\PYG{n+nt}{\PYGZgt{}}
        \PYG{n+nt}{\PYGZlt{}xsl:template} \PYG{n+na}{match=}\PYG{l+s}{\PYGZdq{}/\PYGZdq{}}\PYG{n+nt}{\PYGZgt{}}
                \PYG{n+nt}{\PYGZlt{}html}\PYG{n+nt}{\PYGZgt{}}
                        \PYG{n+nt}{\PYGZlt{}head}\PYG{n+nt}{\PYGZgt{}}
                                \PYG{n+nt}{\PYGZlt{}title}\PYG{n+nt}{\PYGZgt{}}Ejemplo de transformación\PYG{n+nt}{\PYGZlt{}/title\PYGZgt{}}
                        \PYG{n+nt}{\PYGZlt{}/head\PYGZgt{}}
                        \PYG{n+nt}{\PYGZlt{}body}\PYG{n+nt}{\PYGZgt{}}
                                \PYG{n+nt}{\PYGZlt{}h1}\PYG{n+nt}{\PYGZgt{}}Resultado\PYG{n+nt}{\PYGZlt{}/h1\PYGZgt{}}
                                \PYG{n+nt}{\PYGZlt{}ol}\PYG{n+nt}{\PYGZgt{}}
                                \PYG{n+nt}{\PYGZlt{}xsl:for\PYGZhy{}each} \PYG{n+na}{select=}\PYG{l+s}{\PYGZdq{}catalogo/libro\PYGZdq{}}\PYG{n+nt}{\PYGZgt{}}
                                        \PYG{n+nt}{\PYGZlt{}li}\PYG{n+nt}{\PYGZgt{}}
                                                \PYG{n+nt}{\PYGZlt{}xsl:value\PYGZhy{}of} \PYG{n+na}{select=}\PYG{l+s}{\PYGZdq{}title\PYGZdq{}}\PYG{n+nt}{/\PYGZgt{}}
                                        \PYG{n+nt}{\PYGZlt{}/li\PYGZgt{}}
                                \PYG{n+nt}{\PYGZlt{}/xsl:for\PYGZhy{}each\PYGZgt{}}
                                \PYG{n+nt}{\PYGZlt{}/ol\PYGZgt{}}
                        \PYG{n+nt}{\PYGZlt{}/body\PYGZgt{}}
                \PYG{n+nt}{\PYGZlt{}/html\PYGZgt{}}
        \PYG{n+nt}{\PYGZlt{}/xsl:template\PYGZgt{}}
\PYG{n+nt}{\PYGZlt{}/xsl:stylesheet\PYGZgt{}}
\end{sphinxVerbatim}


\subsection{Ejercicio}
\label{\detokenize{tema7:id3}}
Supongamos que ahora un libro tiene varios autores y el XML es algo así:

\begin{sphinxVerbatim}[commandchars=\\\{\}]
\PYG{c+cp}{\PYGZlt{}?xml version=\PYGZdq{}1.0\PYGZdq{} encoding=\PYGZdq{}UTF\PYGZhy{}8\PYGZdq{}?\PYGZgt{}}
\PYG{c+cp}{\PYGZlt{}?xml\PYGZhy{}stylesheet href=\PYGZdq{}hoja1.xsl\PYGZdq{} type=\PYGZdq{}text/xsl\PYGZdq{}?\PYGZgt{}}

\PYG{n+nt}{\PYGZlt{}catalogo}\PYG{n+nt}{\PYGZgt{}}
                \PYG{n+nt}{\PYGZlt{}libro}\PYG{n+nt}{\PYGZgt{}}
                                \PYG{n+nt}{\PYGZlt{}title}\PYG{n+nt}{\PYGZgt{}}Don Quijote\PYG{n+nt}{\PYGZlt{}/title\PYGZgt{}}
                                \PYG{n+nt}{\PYGZlt{}autores}\PYG{n+nt}{\PYGZgt{}}
                                                \PYG{n+nt}{\PYGZlt{}autor}\PYG{n+nt}{\PYGZgt{}}Cervantes\PYG{n+nt}{\PYGZlt{}/autor\PYGZgt{}}
                                \PYG{n+nt}{\PYGZlt{}/autores\PYGZgt{}}
                \PYG{n+nt}{\PYGZlt{}/libro\PYGZgt{}}
                \PYG{n+nt}{\PYGZlt{}libro}\PYG{n+nt}{\PYGZgt{}}
                        \PYG{n+nt}{\PYGZlt{}title}\PYG{n+nt}{\PYGZgt{}}Patrones de diseño en programación\PYG{n+nt}{\PYGZlt{}/title\PYGZgt{}}
                        \PYG{n+nt}{\PYGZlt{}autores}\PYG{n+nt}{\PYGZgt{}}
                                \PYG{n+nt}{\PYGZlt{}autor}\PYG{n+nt}{\PYGZgt{}}Erich Gamma\PYG{n+nt}{\PYGZlt{}/autor\PYGZgt{}}
                                \PYG{n+nt}{\PYGZlt{}autor}\PYG{n+nt}{\PYGZgt{}}John Vlissides\PYG{n+nt}{\PYGZlt{}/autor\PYGZgt{}}
                                \PYG{n+nt}{\PYGZlt{}autor}\PYG{n+nt}{\PYGZgt{}}Ralph Johnson\PYG{n+nt}{\PYGZlt{}/autor\PYGZgt{}}
                        \PYG{n+nt}{\PYGZlt{}/autores\PYGZgt{}}
                \PYG{n+nt}{\PYGZlt{}/libro\PYGZgt{}}
\PYG{n+nt}{\PYGZlt{}/catalogo\PYGZgt{}}
\end{sphinxVerbatim}

¿Como mostrar en HTML el título y todos los autores de cada libro?

\begin{sphinxVerbatim}[commandchars=\\\{\}]
\PYG{c+cp}{\PYGZlt{}?xml version=\PYGZdq{}1.0\PYGZdq{}?\PYGZgt{}}
\PYG{n+nt}{\PYGZlt{}xsl:stylesheet} \PYG{n+na}{version=}\PYG{l+s}{\PYGZdq{}1.1\PYGZdq{}} \PYG{n+na}{xmlns:xsl=}
                \PYG{l+s}{\PYGZdq{}http://www.w3.org/1999/XSL/Transform\PYGZdq{}}\PYG{n+nt}{\PYGZgt{}}
        \PYG{n+nt}{\PYGZlt{}xsl:template} \PYG{n+na}{match=}\PYG{l+s}{\PYGZdq{}/\PYGZdq{}}\PYG{n+nt}{\PYGZgt{}}
                \PYG{n+nt}{\PYGZlt{}html}\PYG{n+nt}{\PYGZgt{}}
                        \PYG{n+nt}{\PYGZlt{}head}\PYG{n+nt}{\PYGZgt{}}
                                \PYG{n+nt}{\PYGZlt{}title}\PYG{n+nt}{\PYGZgt{}}Ejemplo de transformación\PYG{n+nt}{\PYGZlt{}/title\PYGZgt{}}
                        \PYG{n+nt}{\PYGZlt{}/head\PYGZgt{}}
                        \PYG{n+nt}{\PYGZlt{}body}\PYG{n+nt}{\PYGZgt{}}
                                \PYG{n+nt}{\PYGZlt{}h1}\PYG{n+nt}{\PYGZgt{}}Resultado\PYG{n+nt}{\PYGZlt{}/h1\PYGZgt{}}
                                \PYG{n+nt}{\PYGZlt{}ol}\PYG{n+nt}{\PYGZgt{}}
                                \PYG{n+nt}{\PYGZlt{}xsl:for\PYGZhy{}each} \PYG{n+na}{select=}\PYG{l+s}{\PYGZdq{}catalogo/libro\PYGZdq{}}\PYG{n+nt}{\PYGZgt{}}
                                        \PYG{n+nt}{\PYGZlt{}li}\PYG{n+nt}{\PYGZgt{}}
                                                \PYG{n+nt}{\PYGZlt{}xsl:value\PYGZhy{}of} \PYG{n+na}{select=}\PYG{l+s}{\PYGZdq{}title\PYGZdq{}}\PYG{n+nt}{/\PYGZgt{}}
                                                \PYG{n+nt}{\PYGZlt{}ol}\PYG{n+nt}{\PYGZgt{}}
                                                        \PYG{n+nt}{\PYGZlt{}xsl:for\PYGZhy{}each} \PYG{n+na}{select=}\PYG{l+s}{\PYGZdq{}autores/autor\PYGZdq{}}\PYG{n+nt}{\PYGZgt{}}
                                                                \PYG{n+nt}{\PYGZlt{}li}\PYG{n+nt}{\PYGZgt{}}
                                                                        \PYG{n+nt}{\PYGZlt{}xsl:value\PYGZhy{}of} \PYG{n+na}{select=}\PYG{l+s}{\PYGZdq{}.\PYGZdq{}}\PYG{n+nt}{/\PYGZgt{}}
                                                                \PYG{n+nt}{\PYGZlt{}/li\PYGZgt{}}
                                                        \PYG{n+nt}{\PYGZlt{}/xsl:for\PYGZhy{}each\PYGZgt{}} \PYG{c}{\PYGZlt{}!\PYGZhy{}\PYGZhy{}}\PYG{c}{Fin del bucle autores}\PYG{c}{\PYGZhy{}\PYGZhy{}\PYGZgt{}}
                                                \PYG{n+nt}{\PYGZlt{}/ol\PYGZgt{}}
                                        \PYG{n+nt}{\PYGZlt{}/li\PYGZgt{}}
                                \PYG{n+nt}{\PYGZlt{}/xsl:for\PYGZhy{}each\PYGZgt{}} \PYG{c}{\PYGZlt{}!\PYGZhy{}\PYGZhy{}}\PYG{c}{Fin del recorrido de libro}\PYG{c}{\PYGZhy{}\PYGZhy{}\PYGZgt{}}
                                \PYG{n+nt}{\PYGZlt{}/ol\PYGZgt{}}
                        \PYG{n+nt}{\PYGZlt{}/body\PYGZgt{}}
                \PYG{n+nt}{\PYGZlt{}/html\PYGZgt{}}
        \PYG{n+nt}{\PYGZlt{}/xsl:template\PYGZgt{}}
\PYG{n+nt}{\PYGZlt{}/xsl:stylesheet\PYGZgt{}}
\end{sphinxVerbatim}


\subsection{Ejercicio}
\label{\detokenize{tema7:id4}}
Se desea hacer lo mismo que en el ejercicio anterior pero haciendo que los autores aparezcan de forma ordenada.

La solución está fundamentada en el uso de la etiqueta siguiente:

\begin{sphinxVerbatim}[commandchars=\\\{\}]
\PYG{n+nt}{\PYGZlt{}xsl:for\PYGZhy{}each} \PYG{n+na}{select=}\PYG{l+s}{\PYGZdq{}...\PYGZdq{}}\PYG{n+nt}{\PYGZgt{}}
        \PYG{n+nt}{\PYGZlt{}xsl:sort} \PYG{n+na}{select=}\PYG{l+s}{\PYGZdq{}...\PYGZdq{}} \PYG{n+na}{ordering=}\PYG{l+s}{\PYGZdq{}...\PYGZdq{}}\PYG{n+nt}{\PYGZgt{}}
                ..cosas del bucle...
        \PYG{n+nt}{\PYGZlt{}/xsl:sort\PYGZgt{}}
\PYG{n+nt}{\PYGZlt{}/xsl:for\PYGZhy{}each\PYGZgt{}}
\end{sphinxVerbatim}

La solución completa sería así:

\begin{sphinxVerbatim}[commandchars=\\\{\}]
\PYG{c+cp}{\PYGZlt{}?xml version=\PYGZdq{}1.0\PYGZdq{}?\PYGZgt{}}
\PYG{n+nt}{\PYGZlt{}xsl:stylesheet} \PYG{n+na}{version=}\PYG{l+s}{\PYGZdq{}1.1\PYGZdq{}} \PYG{n+na}{xmlns:xsl=}
                \PYG{l+s}{\PYGZdq{}http://www.w3.org/1999/XSL/Transform\PYGZdq{}}\PYG{n+nt}{\PYGZgt{}}
        \PYG{n+nt}{\PYGZlt{}xsl:template} \PYG{n+na}{match=}\PYG{l+s}{\PYGZdq{}/\PYGZdq{}}\PYG{n+nt}{\PYGZgt{}}
                \PYG{n+nt}{\PYGZlt{}html}\PYG{n+nt}{\PYGZgt{}}
                        \PYG{n+nt}{\PYGZlt{}head}\PYG{n+nt}{\PYGZgt{}}
                                \PYG{n+nt}{\PYGZlt{}title}\PYG{n+nt}{\PYGZgt{}}Ejemplo de transformación\PYG{n+nt}{\PYGZlt{}/title\PYGZgt{}}
                        \PYG{n+nt}{\PYGZlt{}/head\PYGZgt{}}
                        \PYG{n+nt}{\PYGZlt{}body}\PYG{n+nt}{\PYGZgt{}}
                                \PYG{n+nt}{\PYGZlt{}h1}\PYG{n+nt}{\PYGZgt{}}Resultado\PYG{n+nt}{\PYGZlt{}/h1\PYGZgt{}}
                                \PYG{n+nt}{\PYGZlt{}ol}\PYG{n+nt}{\PYGZgt{}}
                                \PYG{n+nt}{\PYGZlt{}xsl:for\PYGZhy{}each} \PYG{n+na}{select=}\PYG{l+s}{\PYGZdq{}catalogo/libro\PYGZdq{}}\PYG{n+nt}{\PYGZgt{}}
                                        \PYG{n+nt}{\PYGZlt{}li}\PYG{n+nt}{\PYGZgt{}}
                                                \PYG{n+nt}{\PYGZlt{}xsl:value\PYGZhy{}of} \PYG{n+na}{select=}\PYG{l+s}{\PYGZdq{}title\PYGZdq{}}\PYG{n+nt}{/\PYGZgt{}}
                                                \PYG{n+nt}{\PYGZlt{}ol}\PYG{n+nt}{\PYGZgt{}}
                                                        \PYG{n+nt}{\PYGZlt{}xsl:for\PYGZhy{}each} \PYG{n+na}{select=}\PYG{l+s}{\PYGZdq{}autores/autor\PYGZdq{}}\PYG{n+nt}{\PYGZgt{}}
                                                        \PYG{n+nt}{\PYGZlt{}xsl:sort} \PYG{n+na}{order=}\PYG{l+s}{\PYGZdq{}descending\PYGZdq{}}\PYG{n+nt}{/\PYGZgt{}}
                                                                \PYG{n+nt}{\PYGZlt{}li}\PYG{n+nt}{\PYGZgt{}}
                                                                        \PYG{n+nt}{\PYGZlt{}xsl:value\PYGZhy{}of} \PYG{n+na}{select=}\PYG{l+s}{\PYGZdq{}.\PYGZdq{}}\PYG{n+nt}{/\PYGZgt{}}
                                                                \PYG{n+nt}{\PYGZlt{}/li\PYGZgt{}}
                                                        \PYG{n+nt}{\PYGZlt{}/xsl:for\PYGZhy{}each\PYGZgt{}}
                                                \PYG{n+nt}{\PYGZlt{}/ol\PYGZgt{}}
                                        \PYG{n+nt}{\PYGZlt{}/li\PYGZgt{}}
                                \PYG{n+nt}{\PYGZlt{}/xsl:for\PYGZhy{}each\PYGZgt{}} \PYG{c}{\PYGZlt{}!\PYGZhy{}\PYGZhy{}}\PYG{c}{Fin del recorrido de libro}\PYG{c}{\PYGZhy{}\PYGZhy{}\PYGZgt{}}
                                \PYG{n+nt}{\PYGZlt{}/ol\PYGZgt{}}
                        \PYG{n+nt}{\PYGZlt{}/body\PYGZgt{}}
                \PYG{n+nt}{\PYGZlt{}/html\PYGZgt{}}
        \PYG{n+nt}{\PYGZlt{}/xsl:template\PYGZgt{}}
\PYG{n+nt}{\PYGZlt{}/xsl:stylesheet\PYGZgt{}}
\end{sphinxVerbatim}


\subsection{Ejercicio}
\label{\detokenize{tema7:id5}}
Suponiendo que además todos los libros tienen además un elemento \sphinxcode{\textless{}fechaedicion\textgreater{}} mostrar los libros editados despues del 2000.

\begin{sphinxVerbatim}[commandchars=\\\{\}]
\PYG{c+cp}{\PYGZlt{}?xml version=\PYGZdq{}1.0\PYGZdq{} encoding=\PYGZdq{}UTF\PYGZhy{}8\PYGZdq{}?\PYGZgt{}}
\PYG{c+cp}{\PYGZlt{}?xml\PYGZhy{}stylesheet href=\PYGZdq{}hoja1.xsl\PYGZdq{} type=\PYGZdq{}text/xsl\PYGZdq{}?\PYGZgt{}}

\PYG{n+nt}{\PYGZlt{}catalogo}\PYG{n+nt}{\PYGZgt{}}
                \PYG{n+nt}{\PYGZlt{}libro}\PYG{n+nt}{\PYGZgt{}}
                                \PYG{n+nt}{\PYGZlt{}title}\PYG{n+nt}{\PYGZgt{}}Don Quijote\PYG{n+nt}{\PYGZlt{}/title\PYGZgt{}}
                                \PYG{n+nt}{\PYGZlt{}autores}\PYG{n+nt}{\PYGZgt{}}
                                                \PYG{n+nt}{\PYGZlt{}autor}\PYG{n+nt}{\PYGZgt{}}Cervantes\PYG{n+nt}{\PYGZlt{}/autor\PYGZgt{}}
                                \PYG{n+nt}{\PYGZlt{}/autores\PYGZgt{}}
                                \PYG{n+nt}{\PYGZlt{}fechaedicion}\PYG{n+nt}{\PYGZgt{}}1984\PYG{n+nt}{\PYGZlt{}/fechaedicion\PYGZgt{}}
                \PYG{n+nt}{\PYGZlt{}/libro\PYGZgt{}}
                \PYG{n+nt}{\PYGZlt{}libro}\PYG{n+nt}{\PYGZgt{}}
                        \PYG{n+nt}{\PYGZlt{}title}\PYG{n+nt}{\PYGZgt{}}Patrones de diseño en programación\PYG{n+nt}{\PYGZlt{}/title\PYGZgt{}}
                        \PYG{n+nt}{\PYGZlt{}autores}\PYG{n+nt}{\PYGZgt{}}
                                \PYG{n+nt}{\PYGZlt{}autor}\PYG{n+nt}{\PYGZgt{}}Ralph Johnson\PYG{n+nt}{\PYGZlt{}/autor\PYGZgt{}}
                                \PYG{n+nt}{\PYGZlt{}autor}\PYG{n+nt}{\PYGZgt{}}Erich Gamma\PYG{n+nt}{\PYGZlt{}/autor\PYGZgt{}}
                                \PYG{n+nt}{\PYGZlt{}autor}\PYG{n+nt}{\PYGZgt{}}John Vlissides\PYG{n+nt}{\PYGZlt{}/autor\PYGZgt{}}
                        \PYG{n+nt}{\PYGZlt{}/autores\PYGZgt{}}
                        \PYG{n+nt}{\PYGZlt{}fechaedicion}\PYG{n+nt}{\PYGZgt{}}2007\PYG{n+nt}{\PYGZlt{}/fechaedicion\PYGZgt{}}
                \PYG{n+nt}{\PYGZlt{}/libro\PYGZgt{}}
\PYG{n+nt}{\PYGZlt{}/catalogo\PYGZgt{}}
\end{sphinxVerbatim}

\begin{sphinxVerbatim}[commandchars=\\\{\}]
\PYG{c+cp}{\PYGZlt{}?xml version=\PYGZdq{}1.0\PYGZdq{}?\PYGZgt{}}
\PYG{n+nt}{\PYGZlt{}xsl:stylesheet} \PYG{n+na}{version=}\PYG{l+s}{\PYGZdq{}1.1\PYGZdq{}} \PYG{n+na}{xmlns:xsl=}
                \PYG{l+s}{\PYGZdq{}http://www.w3.org/1999/XSL/Transform\PYGZdq{}}\PYG{n+nt}{\PYGZgt{}}
        \PYG{n+nt}{\PYGZlt{}xsl:template} \PYG{n+na}{match=}\PYG{l+s}{\PYGZdq{}/\PYGZdq{}}\PYG{n+nt}{\PYGZgt{}}
                \PYG{n+nt}{\PYGZlt{}html}\PYG{n+nt}{\PYGZgt{}}
                        \PYG{n+nt}{\PYGZlt{}head}\PYG{n+nt}{\PYGZgt{}}
                                \PYG{n+nt}{\PYGZlt{}title}\PYG{n+nt}{\PYGZgt{}}Ejemplo de transformación\PYG{n+nt}{\PYGZlt{}/title\PYGZgt{}}
                        \PYG{n+nt}{\PYGZlt{}/head\PYGZgt{}}
                        \PYG{n+nt}{\PYGZlt{}body}\PYG{n+nt}{\PYGZgt{}}
                                \PYG{n+nt}{\PYGZlt{}h1}\PYG{n+nt}{\PYGZgt{}}Resultado\PYG{n+nt}{\PYGZlt{}/h1\PYGZgt{}}
                                \PYG{n+nt}{\PYGZlt{}ol}\PYG{n+nt}{\PYGZgt{}}
                                \PYG{n+nt}{\PYGZlt{}xsl:for\PYGZhy{}each} \PYG{n+na}{select=}\PYG{l+s}{\PYGZdq{}catalogo/libro\PYGZdq{}}\PYG{n+nt}{\PYGZgt{}}
                                        \PYG{n+nt}{\PYGZlt{}xsl:if} \PYG{n+na}{test=}\PYG{l+s}{\PYGZdq{}fechaedicion \PYGZam{}gt; 2000\PYGZdq{}}\PYG{n+nt}{\PYGZgt{}}
                                        \PYG{n+nt}{\PYGZlt{}li}\PYG{n+nt}{\PYGZgt{}}
                                                \PYG{n+nt}{\PYGZlt{}xsl:value\PYGZhy{}of} \PYG{n+na}{select=}\PYG{l+s}{\PYGZdq{}title\PYGZdq{}}\PYG{n+nt}{/\PYGZgt{}}
                                                \PYG{n+nt}{\PYGZlt{}ol}\PYG{n+nt}{\PYGZgt{}}
                                                        \PYG{n+nt}{\PYGZlt{}xsl:for\PYGZhy{}each} \PYG{n+na}{select=}\PYG{l+s}{\PYGZdq{}autores/autor\PYGZdq{}}\PYG{n+nt}{\PYGZgt{}}
                                                        \PYG{n+nt}{\PYGZlt{}xsl:sort} \PYG{n+na}{order=}\PYG{l+s}{\PYGZdq{}descending\PYGZdq{}}\PYG{n+nt}{/\PYGZgt{}}
                                                                \PYG{n+nt}{\PYGZlt{}li}\PYG{n+nt}{\PYGZgt{}}
                                                                        \PYG{n+nt}{\PYGZlt{}xsl:value\PYGZhy{}of} \PYG{n+na}{select=}\PYG{l+s}{\PYGZdq{}.\PYGZdq{}}\PYG{n+nt}{/\PYGZgt{}}
                                                                \PYG{n+nt}{\PYGZlt{}/li\PYGZgt{}}

                                                        \PYG{n+nt}{\PYGZlt{}/xsl:for\PYGZhy{}each\PYGZgt{}}
                                                \PYG{n+nt}{\PYGZlt{}/ol\PYGZgt{}}
                                        \PYG{n+nt}{\PYGZlt{}/li\PYGZgt{}}
                                        \PYG{n+nt}{\PYGZlt{}/xsl:if\PYGZgt{}}

                                \PYG{n+nt}{\PYGZlt{}/xsl:for\PYGZhy{}each\PYGZgt{}} \PYG{c}{\PYGZlt{}!\PYGZhy{}\PYGZhy{}}\PYG{c}{Fin del recorrido de libro}\PYG{c}{\PYGZhy{}\PYGZhy{}\PYGZgt{}}
                                \PYG{n+nt}{\PYGZlt{}/ol\PYGZgt{}}
                        \PYG{n+nt}{\PYGZlt{}/body\PYGZgt{}}
                \PYG{n+nt}{\PYGZlt{}/html\PYGZgt{}}
        \PYG{n+nt}{\PYGZlt{}/xsl:template\PYGZgt{}}
\PYG{n+nt}{\PYGZlt{}/xsl:stylesheet\PYGZgt{}}
\end{sphinxVerbatim}

En general, las condiciones se escriben así:
\begin{itemize}
\item {} 
\textgreater{} o mayor que o \sphinxcode{\&gt;}

\item {} 
\textless{} o menor que o \sphinxcode{\&lt;}

\item {} 
\textgreater{}= o mayor o igual o \sphinxcode{\&ge;}

\item {} 
\textless{}= o menor o igual o \sphinxcode{\&le;}

\item {} 
\textless{}\textgreater{} o distinto o \sphinxcode{\&neq;}

\end{itemize}


\subsection{Ejercicio XSL, paso a paso}
\label{\detokenize{tema7:ejercicio-xsl-paso-a-paso}}
Dado el siguiente XML crear un programa con XSLT que muestre los titulos y los autores de los libros cuya fecha de edicion sea posterior al 2000.

\begin{sphinxVerbatim}[commandchars=\\\{\}]
\PYG{c+cp}{\PYGZlt{}?xml version=\PYGZdq{}1.0\PYGZdq{} encoding=\PYGZdq{}UTF\PYGZhy{}8\PYGZdq{}?\PYGZgt{}}
\PYG{c+cp}{\PYGZlt{}?xml\PYGZhy{}stylesheet     type=\PYGZdq{}text/xsl\PYGZdq{}         href=\PYGZdq{}ejercicio1.xsl\PYGZdq{}?\PYGZgt{}}
     \PYG{n+nt}{\PYGZlt{}catalogo}\PYG{n+nt}{\PYGZgt{}}
             \PYG{n+nt}{\PYGZlt{}libro} \PYG{n+na}{fechaedicion=}\PYG{l+s}{\PYGZdq{}1999\PYGZdq{}}\PYG{n+nt}{\PYGZgt{}}
                     \PYG{n+nt}{\PYGZlt{}titulo}\PYG{n+nt}{\PYGZgt{}}Don Quijote\PYG{n+nt}{\PYGZlt{}/titulo\PYGZgt{}}
                     \PYG{n+nt}{\PYGZlt{}autores}\PYG{n+nt}{\PYGZgt{}}
                             \PYG{n+nt}{\PYGZlt{}autor}\PYG{n+nt}{\PYGZgt{}}Cervantes\PYG{n+nt}{\PYGZlt{}/autor\PYGZgt{}}
                     \PYG{n+nt}{\PYGZlt{}/autores\PYGZgt{}}
             \PYG{n+nt}{\PYGZlt{}/libro\PYGZgt{}}
             \PYG{n+nt}{\PYGZlt{}libro} \PYG{n+na}{fechaedicion=}\PYG{l+s}{\PYGZdq{}2005\PYGZdq{}}\PYG{n+nt}{\PYGZgt{}}
                     \PYG{n+nt}{\PYGZlt{}titulo}\PYG{n+nt}{\PYGZgt{}}
                     La sociedad civil moderna
                     \PYG{n+nt}{\PYGZlt{}/titulo\PYGZgt{}}
                     \PYG{n+nt}{\PYGZlt{}autores}\PYG{n+nt}{\PYGZgt{}}
                             \PYG{n+nt}{\PYGZlt{}autor}\PYG{n+nt}{\PYGZgt{}}Luis Diaz\PYG{n+nt}{\PYGZlt{}/autor\PYGZgt{}}
                             \PYG{n+nt}{\PYGZlt{}autor}\PYG{n+nt}{\PYGZgt{}}Pedro Campos\PYG{n+nt}{\PYGZlt{}/autor\PYGZgt{}}
                     \PYG{n+nt}{\PYGZlt{}/autores\PYGZgt{}}
             \PYG{n+nt}{\PYGZlt{}/libro\PYGZgt{}}
     \PYG{n+nt}{\PYGZlt{}/catalogo\PYGZgt{}}
\end{sphinxVerbatim}

Hagámoslo paso a paso. En primer lugar tendremos que crear el fichero \sphinxcode{ejercicio1.xsl} y crear la estructura básica:

\begin{sphinxVerbatim}[commandchars=\\\{\}]
\PYG{c+cp}{\PYGZlt{}?xml version=\PYGZdq{}1.0\PYGZdq{} encoding=\PYGZdq{}utf\PYGZhy{}8\PYGZdq{}?\PYGZgt{}}
\PYG{n+nt}{\PYGZlt{}xsl:stylesheet} \PYG{n+na}{xmlns:xsl=}\PYG{l+s}{\PYGZdq{}http://www.w3.org/1999/XSL/Transform\PYGZdq{}} \PYG{n+na}{version=}\PYG{l+s}{\PYGZdq{}1.1\PYGZdq{}}\PYG{n+nt}{\PYGZgt{}}
\PYG{n+nt}{\PYGZlt{}xsl:template} \PYG{n+na}{match=}\PYG{l+s}{\PYGZdq{}/\PYGZdq{}}\PYG{n+nt}{\PYGZgt{}}

\PYG{n+nt}{\PYGZlt{}/xsl:template\PYGZgt{}}
\PYG{n+nt}{\PYGZlt{}/xsl:stylesheet\PYGZgt{}}
\end{sphinxVerbatim}

Ahora recorramos los libros que hay en el catalogo (recordemos que la estructura es \sphinxcode{catalogo/libro}. Simplemente por ver si funciona, de momento el navegado solo muestra los títulos y en una sola línea.

\begin{sphinxVerbatim}[commandchars=\\\{\}]
\PYG{c+cp}{\PYGZlt{}?xml version=\PYGZdq{}1.0\PYGZdq{} encoding=\PYGZdq{}utf\PYGZhy{}8\PYGZdq{}?\PYGZgt{}}
\PYG{n+nt}{\PYGZlt{}xsl:stylesheet} \PYG{n+na}{xmlns:xsl=}\PYG{l+s}{\PYGZdq{}http://www.w3.org/1999/XSL/Transform\PYGZdq{}} \PYG{n+na}{version=}\PYG{l+s}{\PYGZdq{}1.1\PYGZdq{}}\PYG{n+nt}{\PYGZgt{}}
\PYG{n+nt}{\PYGZlt{}xsl:template} \PYG{n+na}{match=}\PYG{l+s}{\PYGZdq{}/\PYGZdq{}}\PYG{n+nt}{\PYGZgt{}}
  \PYG{n+nt}{\PYGZlt{}xsl:for\PYGZhy{}each} \PYG{n+na}{select=}\PYG{l+s}{\PYGZdq{}catalogo/libro\PYGZdq{}}\PYG{n+nt}{\PYGZgt{}}
        \PYG{n+nt}{\PYGZlt{}xsl:value\PYGZhy{}of} \PYG{n+na}{select=}\PYG{l+s}{\PYGZdq{}titulo\PYGZdq{}}\PYG{n+nt}{/\PYGZgt{}}
  \PYG{n+nt}{\PYGZlt{}/xsl:for\PYGZhy{}each\PYGZgt{}}
\PYG{n+nt}{\PYGZlt{}/xsl:template\PYGZgt{}}
\PYG{n+nt}{\PYGZlt{}/xsl:stylesheet\PYGZgt{}}
\end{sphinxVerbatim}

\begin{figure}[htbp]
\centering
\capstart

\noindent\sphinxincludegraphics{{ejercicio1xslpaso1}.png}
\caption{Paso inicial del XSL}\label{\detokenize{tema7:id6}}\end{figure}

Avancemos un poco más y creemos una estructura HTML válida

\begin{sphinxVerbatim}[commandchars=\\\{\}]
\PYG{c+cp}{\PYGZlt{}?xml version=\PYGZdq{}1.0\PYGZdq{} encoding=\PYGZdq{}utf\PYGZhy{}8\PYGZdq{}?\PYGZgt{}}
\PYG{n+nt}{\PYGZlt{}xsl:stylesheet} \PYG{n+na}{xmlns:xsl=}\PYG{l+s}{\PYGZdq{}http://www.w3.org/1999/XSL/Transform\PYGZdq{}} \PYG{n+na}{version=}\PYG{l+s}{\PYGZdq{}1.1\PYGZdq{}}\PYG{n+nt}{\PYGZgt{}}
\PYG{n+nt}{\PYGZlt{}xsl:template} \PYG{n+na}{match=}\PYG{l+s}{\PYGZdq{}/\PYGZdq{}}\PYG{n+nt}{\PYGZgt{}}
\PYG{n+nt}{\PYGZlt{}html}\PYG{n+nt}{\PYGZgt{}}
  \PYG{n+nt}{\PYGZlt{}head}\PYG{n+nt}{\PYGZgt{}}
         \PYG{n+nt}{\PYGZlt{}title}\PYG{n+nt}{\PYGZgt{}}Filtrado con XSLT\PYG{n+nt}{\PYGZlt{}/title\PYGZgt{}}
  \PYG{n+nt}{\PYGZlt{}/head\PYGZgt{}}
  \PYG{n+nt}{\PYGZlt{}body}\PYG{n+nt}{\PYGZgt{}}
  \PYG{n+nt}{\PYGZlt{}h1}\PYG{n+nt}{\PYGZgt{}}Filtrado con XSLT\PYG{n+nt}{\PYGZlt{}/h1\PYGZgt{}}
  \PYG{n+nt}{\PYGZlt{}ol}\PYG{n+nt}{\PYGZgt{}}
        \PYG{n+nt}{\PYGZlt{}xsl:for\PYGZhy{}each} \PYG{n+na}{select=}\PYG{l+s}{\PYGZdq{}catalogo/libro\PYGZdq{}}\PYG{n+nt}{\PYGZgt{}}
                \PYG{n+nt}{\PYGZlt{}li}\PYG{n+nt}{\PYGZgt{}}
                \PYG{n+nt}{\PYGZlt{}xsl:value\PYGZhy{}of} \PYG{n+na}{select=}\PYG{l+s}{\PYGZdq{}titulo\PYGZdq{}}\PYG{n+nt}{/\PYGZgt{}}
                \PYG{n+nt}{\PYGZlt{}/li\PYGZgt{}}
        \PYG{n+nt}{\PYGZlt{}/xsl:for\PYGZhy{}each\PYGZgt{}}
  \PYG{n+nt}{\PYGZlt{}/ol\PYGZgt{}}
  \PYG{n+nt}{\PYGZlt{}/body\PYGZgt{}}
\PYG{n+nt}{\PYGZlt{}/html\PYGZgt{}}
\PYG{n+nt}{\PYGZlt{}/xsl:template\PYGZgt{}}
\PYG{n+nt}{\PYGZlt{}/xsl:stylesheet\PYGZgt{}}
\end{sphinxVerbatim}

\begin{figure}[htbp]
\centering
\capstart

\noindent\sphinxincludegraphics{{ejercicio1xslpaso2}.png}
\caption{Extrayendo los titulos con XSL}\label{\detokenize{tema7:id7}}\end{figure}

Ahora vamos a procesar solo los libros cuya \sphinxcode{fechaedicion} sea posterior al 2000. Añadamos un \sphinxcode{if}

\begin{sphinxVerbatim}[commandchars=\\\{\}]
\PYG{c+cp}{\PYGZlt{}?xml version=\PYGZdq{}1.0\PYGZdq{} encoding=\PYGZdq{}utf\PYGZhy{}8\PYGZdq{}?\PYGZgt{}}
\PYG{n+nt}{\PYGZlt{}xsl:stylesheet} \PYG{n+na}{xmlns:xsl=}\PYG{l+s}{\PYGZdq{}http://www.w3.org/1999/XSL/Transform\PYGZdq{}} \PYG{n+na}{version=}\PYG{l+s}{\PYGZdq{}1.1\PYGZdq{}}\PYG{n+nt}{\PYGZgt{}}
\PYG{n+nt}{\PYGZlt{}xsl:template} \PYG{n+na}{match=}\PYG{l+s}{\PYGZdq{}/\PYGZdq{}}\PYG{n+nt}{\PYGZgt{}}
\PYG{n+nt}{\PYGZlt{}html}\PYG{n+nt}{\PYGZgt{}}
  \PYG{n+nt}{\PYGZlt{}head}\PYG{n+nt}{\PYGZgt{}}
         \PYG{n+nt}{\PYGZlt{}title}\PYG{n+nt}{\PYGZgt{}}Filtrado con XSLT\PYG{n+nt}{\PYGZlt{}/title\PYGZgt{}}
  \PYG{n+nt}{\PYGZlt{}/head\PYGZgt{}}
  \PYG{n+nt}{\PYGZlt{}body}\PYG{n+nt}{\PYGZgt{}}
  \PYG{n+nt}{\PYGZlt{}h1}\PYG{n+nt}{\PYGZgt{}}Filtrado con XSLT\PYG{n+nt}{\PYGZlt{}/h1\PYGZgt{}}
  \PYG{n+nt}{\PYGZlt{}ol}\PYG{n+nt}{\PYGZgt{}}
        \PYG{n+nt}{\PYGZlt{}xsl:for\PYGZhy{}each} \PYG{n+na}{select=}\PYG{l+s}{\PYGZdq{}catalogo/libro\PYGZdq{}}\PYG{n+nt}{\PYGZgt{}}

                \PYG{n+nt}{\PYGZlt{}xsl:if} \PYG{n+na}{test=}\PYG{l+s}{\PYGZdq{}@fechaedicion \PYGZam{}gt; 2000\PYGZdq{}}\PYG{n+nt}{\PYGZgt{}}

                \PYG{n+nt}{\PYGZlt{}li}\PYG{n+nt}{\PYGZgt{}}
                \PYG{n+nt}{\PYGZlt{}xsl:value\PYGZhy{}of} \PYG{n+na}{select=}\PYG{l+s}{\PYGZdq{}titulo\PYGZdq{}}\PYG{n+nt}{/\PYGZgt{}}
                \PYG{n+nt}{\PYGZlt{}/li\PYGZgt{}}

                \PYG{n+nt}{\PYGZlt{}/xsl:if\PYGZgt{}}
        \PYG{n+nt}{\PYGZlt{}/xsl:for\PYGZhy{}each\PYGZgt{}}
  \PYG{n+nt}{\PYGZlt{}/ol\PYGZgt{}}
  \PYG{n+nt}{\PYGZlt{}/body\PYGZgt{}}
\PYG{n+nt}{\PYGZlt{}/html\PYGZgt{}}
\PYG{n+nt}{\PYGZlt{}/xsl:template\PYGZgt{}}
\PYG{n+nt}{\PYGZlt{}/xsl:stylesheet\PYGZgt{}}
\end{sphinxVerbatim}

\begin{figure}[htbp]
\centering
\capstart

\noindent\sphinxincludegraphics{{ejercicio1xslpaso3}.png}
\caption{Procesando los que son \textgreater{} 2000}\label{\detokenize{tema7:id8}}\end{figure}

Ahora para cada libro queremos también mostrar los elementos autor con su propia lista

\begin{sphinxVerbatim}[commandchars=\\\{\}]
\PYG{c+cp}{\PYGZlt{}?xml version=\PYGZdq{}1.0\PYGZdq{} encoding=\PYGZdq{}utf\PYGZhy{}8\PYGZdq{}?\PYGZgt{}}
\PYG{n+nt}{\PYGZlt{}xsl:stylesheet} \PYG{n+na}{xmlns:xsl=}\PYG{l+s}{\PYGZdq{}http://www.w3.org/1999/XSL/Transform\PYGZdq{}} \PYG{n+na}{version=}\PYG{l+s}{\PYGZdq{}1.1\PYGZdq{}}\PYG{n+nt}{\PYGZgt{}}
\PYG{n+nt}{\PYGZlt{}xsl:template} \PYG{n+na}{match=}\PYG{l+s}{\PYGZdq{}/\PYGZdq{}}\PYG{n+nt}{\PYGZgt{}}
\PYG{n+nt}{\PYGZlt{}html}\PYG{n+nt}{\PYGZgt{}}
  \PYG{n+nt}{\PYGZlt{}head}\PYG{n+nt}{\PYGZgt{}}
         \PYG{n+nt}{\PYGZlt{}title}\PYG{n+nt}{\PYGZgt{}}Filtrado con XSLT\PYG{n+nt}{\PYGZlt{}/title\PYGZgt{}}
  \PYG{n+nt}{\PYGZlt{}/head\PYGZgt{}}
  \PYG{n+nt}{\PYGZlt{}body}\PYG{n+nt}{\PYGZgt{}}
  \PYG{n+nt}{\PYGZlt{}h1}\PYG{n+nt}{\PYGZgt{}}Filtrado con XSLT\PYG{n+nt}{\PYGZlt{}/h1\PYGZgt{}}
  \PYG{n+nt}{\PYGZlt{}ol}\PYG{n+nt}{\PYGZgt{}}
        \PYG{n+nt}{\PYGZlt{}xsl:for\PYGZhy{}each} \PYG{n+na}{select=}\PYG{l+s}{\PYGZdq{}catalogo/libro\PYGZdq{}}\PYG{n+nt}{\PYGZgt{}}

                \PYG{n+nt}{\PYGZlt{}xsl:if} \PYG{n+na}{test=}\PYG{l+s}{\PYGZdq{}@fechaedicion \PYGZam{}gt; 2000\PYGZdq{}}\PYG{n+nt}{\PYGZgt{}}

                \PYG{n+nt}{\PYGZlt{}li}\PYG{n+nt}{\PYGZgt{}}
                \PYG{n+nt}{\PYGZlt{}xsl:value\PYGZhy{}of} \PYG{n+na}{select=}\PYG{l+s}{\PYGZdq{}titulo\PYGZdq{}}\PYG{n+nt}{/\PYGZgt{}}
                \PYG{n+nt}{\PYGZlt{}/li\PYGZgt{}}

                \PYG{n+nt}{\PYGZlt{}ol}\PYG{n+nt}{\PYGZgt{}}
                        \PYG{n+nt}{\PYGZlt{}xsl:for\PYGZhy{}each} \PYG{n+na}{select=}\PYG{l+s}{\PYGZdq{}autores/autor\PYGZdq{}}\PYG{n+nt}{\PYGZgt{}}
                                \PYG{n+nt}{\PYGZlt{}li}\PYG{n+nt}{\PYGZgt{}}
                                        \PYG{c}{\PYGZlt{}!\PYGZhy{}\PYGZhy{}}\PYG{c}{El elemento actual es .}\PYG{c}{\PYGZhy{}\PYGZhy{}\PYGZgt{}}
                                        \PYG{n+nt}{\PYGZlt{}xsl:value\PYGZhy{}of} \PYG{n+na}{select=}\PYG{l+s}{\PYGZdq{}.\PYGZdq{}}\PYG{n+nt}{/\PYGZgt{}}
                                \PYG{n+nt}{\PYGZlt{}/li\PYGZgt{}}
                        \PYG{n+nt}{\PYGZlt{}/xsl:for\PYGZhy{}each\PYGZgt{}}
                \PYG{n+nt}{\PYGZlt{}/ol\PYGZgt{}}

                \PYG{n+nt}{\PYGZlt{}/xsl:if\PYGZgt{}}
        \PYG{n+nt}{\PYGZlt{}/xsl:for\PYGZhy{}each\PYGZgt{}}
  \PYG{n+nt}{\PYGZlt{}/ol\PYGZgt{}}
  \PYG{n+nt}{\PYGZlt{}/body\PYGZgt{}}
\PYG{n+nt}{\PYGZlt{}/html\PYGZgt{}}
\PYG{n+nt}{\PYGZlt{}/xsl:template\PYGZgt{}}
\PYG{n+nt}{\PYGZlt{}/xsl:stylesheet\PYGZgt{}}
\end{sphinxVerbatim}

Y el navegador muestra lo siguiente

\begin{figure}[htbp]
\centering
\capstart

\noindent\sphinxincludegraphics{{ejercicio1xslpaso4}.png}
\caption{Mostrando también los autores}\label{\detokenize{tema7:id9}}\end{figure}


\subsection{Ejercicio: condiciones complejas}
\label{\detokenize{tema7:ejercicio-condiciones-complejas}}
Supongamos que nos dan el siguiente fichero de inventario:

Y supongamos que nos dicen que se necesita extraer la información relativa a los productos que pesan más de 5. Una primera aproximación equivocada sería esta:

\begin{sphinxVerbatim}[commandchars=\\\{\}]
\PYG{n+nt}{\PYGZlt{}xsl:template} \PYG{n+na}{match=}\PYG{l+s}{\PYGZdq{}/\PYGZdq{}}\PYG{n+nt}{\PYGZgt{}}
  \PYG{n+nt}{\PYGZlt{}inventario}\PYG{n+nt}{\PYGZgt{}}
    \PYG{n+nt}{\PYGZlt{}xsl:for\PYGZhy{}each} \PYG{n+na}{select=}\PYG{l+s}{\PYGZdq{}inventario/elemento\PYGZdq{}}\PYG{n+nt}{\PYGZgt{}}
      \PYG{n+nt}{\PYGZlt{}xsl:if} \PYG{n+na}{test=}\PYG{l+s}{\PYGZdq{}peso \PYGZam{}gt; 5\PYGZdq{}}\PYG{n+nt}{\PYGZgt{}}
        \PYG{n+nt}{\PYGZlt{}nombre}\PYG{n+nt}{\PYGZgt{}}
          \PYG{n+nt}{\PYGZlt{}xsl:value\PYGZhy{}of} \PYG{n+na}{select=}\PYG{l+s}{\PYGZdq{}nombre\PYGZdq{}}\PYG{n+nt}{/\PYGZgt{}}
        \PYG{n+nt}{\PYGZlt{}/nombre\PYGZgt{}}
      \PYG{n+nt}{\PYGZlt{}/xsl:if\PYGZgt{}}
    \PYG{n+nt}{\PYGZlt{}/xsl:for\PYGZhy{}each\PYGZgt{}}
  \PYG{n+nt}{\PYGZlt{}/inventario\PYGZgt{}}
\PYG{n+nt}{\PYGZlt{}/xsl:template\PYGZgt{}}
\PYG{n+nt}{\PYGZlt{}/xsl:stylesheet\PYGZgt{}}
\end{sphinxVerbatim}

Esta solución está equivocada porque de entrada \sphinxstyleemphasis{la pregunta está mal} Si se refieren a 5kg solo debería mostrarse el ordenador y si se refieren a 5g solo debería mostrarse el altavoz.

Una solución correcta sería esta. Obsérvese como se meten unos if dentro de otros para extraer la información deseada.

\begin{sphinxVerbatim}[commandchars=\\\{\}]
\PYG{n+nt}{\PYGZlt{}xsl:template} \PYG{n+na}{match=}\PYG{l+s}{\PYGZdq{}/\PYGZdq{}}\PYG{n+nt}{\PYGZgt{}}
  \PYG{n+nt}{\PYGZlt{}inventario}\PYG{n+nt}{\PYGZgt{}}
    \PYG{n+nt}{\PYGZlt{}xsl:for\PYGZhy{}each} \PYG{n+na}{select=}\PYG{l+s}{\PYGZdq{}inventario/elemento\PYGZdq{}}\PYG{n+nt}{\PYGZgt{}}
      \PYG{n+nt}{\PYGZlt{}xsl:if} \PYG{n+na}{test=}\PYG{l+s}{\PYGZdq{}./peso/@unidad = \PYGZsq{}kg\PYGZsq{}\PYGZdq{}}\PYG{n+nt}{\PYGZgt{}}
        \PYG{n+nt}{\PYGZlt{}xsl:if} \PYG{n+na}{test=}\PYG{l+s}{\PYGZdq{}peso \PYGZam{}gt; 5\PYGZdq{}}\PYG{n+nt}{\PYGZgt{}}
          \PYG{n+nt}{\PYGZlt{}nombre}\PYG{n+nt}{\PYGZgt{}}
            \PYG{n+nt}{\PYGZlt{}xsl:value\PYGZhy{}of} \PYG{n+na}{select=}\PYG{l+s}{\PYGZdq{}nombre\PYGZdq{}}\PYG{n+nt}{/\PYGZgt{}}
          \PYG{n+nt}{\PYGZlt{}/nombre\PYGZgt{}}
        \PYG{n+nt}{\PYGZlt{}/xsl:if\PYGZgt{}}
      \PYG{n+nt}{\PYGZlt{}/xsl:if\PYGZgt{}}
      \PYG{n+nt}{\PYGZlt{}xsl:if} \PYG{n+na}{test=}\PYG{l+s}{\PYGZdq{}peso/@unidad = \PYGZsq{}g\PYGZsq{}\PYGZdq{}}\PYG{n+nt}{\PYGZgt{}}
        \PYG{n+nt}{\PYGZlt{}xsl:if} \PYG{n+na}{test=}\PYG{l+s}{\PYGZdq{}peso \PYGZam{}gt; 5000\PYGZdq{}}\PYG{n+nt}{\PYGZgt{}}
          \PYG{n+nt}{\PYGZlt{}nombre}\PYG{n+nt}{\PYGZgt{}}
            \PYG{n+nt}{\PYGZlt{}xsl:value\PYGZhy{}of} \PYG{n+na}{select=}\PYG{l+s}{\PYGZdq{}nombre\PYGZdq{}}\PYG{n+nt}{/\PYGZgt{}}
          \PYG{n+nt}{\PYGZlt{}/nombre\PYGZgt{}}
        \PYG{n+nt}{\PYGZlt{}/xsl:if\PYGZgt{}}
      \PYG{n+nt}{\PYGZlt{}/xsl:if\PYGZgt{}}
    \PYG{n+nt}{\PYGZlt{}/xsl:for\PYGZhy{}each\PYGZgt{}}
  \PYG{n+nt}{\PYGZlt{}/inventario\PYGZgt{}}
\PYG{n+nt}{\PYGZlt{}/xsl:template\PYGZgt{}}
\PYG{n+nt}{\PYGZlt{}/xsl:stylesheet\PYGZgt{}}
\end{sphinxVerbatim}


\subsection{Transformación en tabla}
\label{\detokenize{tema7:transformacion-en-tabla}}
Se nos pide convertir el inventario de antes en la tabla siguiente donde el peso debe estar normalizado y aparecer siempre en gramos:

\noindent{\hspace*{\fill}\sphinxincludegraphics[scale=0.5]{{tabla_tras_xslt1}.png}\hspace*{\fill}}

Una posible solución sería:

\begin{sphinxVerbatim}[commandchars=\\\{\}]
\PYG{n+nt}{\PYGZlt{}xsl:stylesheet}
  \PYG{n+na}{xmlns:xsl=}\PYG{l+s}{\PYGZdq{}http://www.w3.org/1999/XSL/Transform\PYGZdq{}}\PYG{n+nt}{\PYGZgt{}}
\PYG{n+nt}{\PYGZlt{}xsl:template} \PYG{n+na}{match=}\PYG{l+s}{\PYGZdq{}/\PYGZdq{}}\PYG{n+nt}{\PYGZgt{}}
\PYG{n+nt}{\PYGZlt{}html}\PYG{n+nt}{\PYGZgt{}}
  \PYG{n+nt}{\PYGZlt{}head}\PYG{n+nt}{\PYGZgt{}}\PYG{n+nt}{\PYGZlt{}title}\PYG{n+nt}{\PYGZgt{}}Tabla de inventario\PYG{n+nt}{\PYGZlt{}/title\PYGZgt{}}\PYG{n+nt}{\PYGZlt{}/head\PYGZgt{}}
  \PYG{n+nt}{\PYGZlt{}body}\PYG{n+nt}{\PYGZgt{}}
    \PYG{n+nt}{\PYGZlt{}table} \PYG{n+na}{border=}\PYG{l+s}{\PYGZsq{}1\PYGZsq{}}\PYG{n+nt}{\PYGZgt{}}
      \PYG{n+nt}{\PYGZlt{}xsl:for\PYGZhy{}each} \PYG{n+na}{select=}\PYG{l+s}{\PYGZdq{}inventario/elemento\PYGZdq{}}\PYG{n+nt}{\PYGZgt{}}
        \PYG{n+nt}{\PYGZlt{}tr}\PYG{n+nt}{\PYGZgt{}}
          \PYG{n+nt}{\PYGZlt{}td}\PYG{n+nt}{\PYGZgt{}}\PYG{n+nt}{\PYGZlt{}xsl:value\PYGZhy{}of} \PYG{n+na}{select=}\PYG{l+s}{\PYGZdq{}nombre\PYGZdq{}}\PYG{n+nt}{/\PYGZgt{}}\PYG{n+nt}{\PYGZlt{}/td\PYGZgt{}}
          \PYG{n+nt}{\PYGZlt{}td}\PYG{n+nt}{\PYGZgt{}}
            \PYG{n+nt}{\PYGZlt{}xsl:if} \PYG{n+na}{test=}\PYG{l+s}{\PYGZdq{}peso/@unidad=\PYGZsq{}kg\PYGZsq{}\PYGZdq{}}\PYG{n+nt}{\PYGZgt{}}
              \PYG{n+nt}{\PYGZlt{}xsl:value\PYGZhy{}of} \PYG{n+na}{select=}\PYG{l+s}{\PYGZdq{}peso * 1000\PYGZdq{}}\PYG{n+nt}{/\PYGZgt{}}
            \PYG{n+nt}{\PYGZlt{}/xsl:if\PYGZgt{}}
            \PYG{n+nt}{\PYGZlt{}xsl:if} \PYG{n+na}{test=}\PYG{l+s}{\PYGZdq{}peso/@unidad=\PYGZsq{}g\PYGZsq{}\PYGZdq{}}\PYG{n+nt}{\PYGZgt{}}
              \PYG{n+nt}{\PYGZlt{}xsl:value\PYGZhy{}of} \PYG{n+na}{select=}\PYG{l+s}{\PYGZdq{}peso\PYGZdq{}}\PYG{n+nt}{/\PYGZgt{}}
            \PYG{n+nt}{\PYGZlt{}/xsl:if\PYGZgt{}}
          \PYG{n+nt}{\PYGZlt{}/td\PYGZgt{}}
        \PYG{n+nt}{\PYGZlt{}/tr\PYGZgt{}}
      \PYG{n+nt}{\PYGZlt{}/xsl:for\PYGZhy{}each\PYGZgt{}}
    \PYG{n+nt}{\PYGZlt{}/table\PYGZgt{}}
  \PYG{n+nt}{\PYGZlt{}/body\PYGZgt{}}
\PYG{n+nt}{\PYGZlt{}/html\PYGZgt{}}
\PYG{n+nt}{\PYGZlt{}/xsl:template\PYGZgt{}}
\PYG{n+nt}{\PYGZlt{}/xsl:stylesheet\PYGZgt{}}
\end{sphinxVerbatim}


\subsection{Transformacion de pedidos}
\label{\detokenize{tema7:transformacion-de-pedidos}}
Dado el siguiente archivo XML:

\begin{sphinxVerbatim}[commandchars=\\\{\}]
\PYG{c+cp}{\PYGZlt{}?xml version=\PYGZdq{}1.0\PYGZdq{} encoding=\PYGZdq{}utf\PYGZhy{}8\PYGZdq{}?\PYGZgt{}}
\PYG{c+cp}{\PYGZlt{}?xml\PYGZhy{}stylesheet href=\PYGZdq{}estilo1.xsl\PYGZdq{} type=\PYGZdq{}text/xsl\PYGZdq{}?\PYGZgt{}}
\PYG{n+nt}{\PYGZlt{}pedido}\PYG{n+nt}{\PYGZgt{}}
        \PYG{n+nt}{\PYGZlt{}portatiles}\PYG{n+nt}{\PYGZgt{}}
                \PYG{n+nt}{\PYGZlt{}portatil}\PYG{n+nt}{\PYGZgt{}}
                        \PYG{n+nt}{\PYGZlt{}peso}\PYG{n+nt}{\PYGZgt{}}1430\PYG{n+nt}{\PYGZlt{}/peso\PYGZgt{}}
                        \PYG{n+nt}{\PYGZlt{}ram} \PYG{n+na}{unidad=}\PYG{l+s}{\PYGZdq{}GB\PYGZdq{}}\PYG{n+nt}{\PYGZgt{}}4\PYG{n+nt}{\PYGZlt{}/ram\PYGZgt{}}
                        \PYG{n+nt}{\PYGZlt{}disco} \PYG{n+na}{tipo=}\PYG{l+s}{\PYGZdq{}ssd\PYGZdq{}}\PYG{n+nt}{\PYGZgt{}}500\PYG{n+nt}{\PYGZlt{}/disco\PYGZgt{}}
                        \PYG{n+nt}{\PYGZlt{}precio}\PYG{n+nt}{\PYGZgt{}}499\PYG{n+nt}{\PYGZlt{}/precio\PYGZgt{}}
                \PYG{n+nt}{\PYGZlt{}/portatil\PYGZgt{}}
                \PYG{n+nt}{\PYGZlt{}portatil}\PYG{n+nt}{\PYGZgt{}}
                        \PYG{n+nt}{\PYGZlt{}peso}\PYG{n+nt}{\PYGZgt{}}1830\PYG{n+nt}{\PYGZlt{}/peso\PYGZgt{}}
                        \PYG{n+nt}{\PYGZlt{}ram} \PYG{n+na}{unidad=}\PYG{l+s}{\PYGZdq{}GB\PYGZdq{}}\PYG{n+nt}{\PYGZgt{}}6\PYG{n+nt}{\PYGZlt{}/ram\PYGZgt{}}
                        \PYG{n+nt}{\PYGZlt{}disco} \PYG{n+na}{tipo=}\PYG{l+s}{\PYGZdq{}ssd\PYGZdq{}}\PYG{n+nt}{\PYGZgt{}}1000\PYG{n+nt}{\PYGZlt{}/disco\PYGZgt{}}
                        \PYG{n+nt}{\PYGZlt{}precio}\PYG{n+nt}{\PYGZgt{}}1199\PYG{n+nt}{\PYGZlt{}/precio\PYGZgt{}}
                \PYG{n+nt}{\PYGZlt{}/portatil\PYGZgt{}}
                \PYG{n+nt}{\PYGZlt{}portatil}\PYG{n+nt}{\PYGZgt{}}
                        \PYG{n+nt}{\PYGZlt{}peso}\PYG{n+nt}{\PYGZgt{}}1250\PYG{n+nt}{\PYGZlt{}/peso\PYGZgt{}}
                        \PYG{n+nt}{\PYGZlt{}ram} \PYG{n+na}{unidad=}\PYG{l+s}{\PYGZdq{}GB\PYGZdq{}}\PYG{n+nt}{\PYGZgt{}}2\PYG{n+nt}{\PYGZlt{}/ram\PYGZgt{}}
                        \PYG{n+nt}{\PYGZlt{}disco} \PYG{n+na}{tipo=}\PYG{l+s}{\PYGZdq{}ssd\PYGZdq{}}\PYG{n+nt}{\PYGZgt{}}750\PYG{n+nt}{\PYGZlt{}/disco\PYGZgt{}}
                        \PYG{n+nt}{\PYGZlt{}precio}\PYG{n+nt}{\PYGZgt{}}699\PYG{n+nt}{\PYGZlt{}/precio\PYGZgt{}}
                \PYG{n+nt}{\PYGZlt{}/portatil\PYGZgt{}}
        \PYG{n+nt}{\PYGZlt{}/portatiles\PYGZgt{}}
        \PYG{n+nt}{\PYGZlt{}tablets}\PYG{n+nt}{\PYGZgt{}}
                \PYG{n+nt}{\PYGZlt{}tablet}\PYG{n+nt}{\PYGZgt{}}
                        \PYG{n+nt}{\PYGZlt{}plataforma}\PYG{n+nt}{\PYGZgt{}}Android\PYG{n+nt}{\PYGZlt{}/plataforma\PYGZgt{}}
                        \PYG{n+nt}{\PYGZlt{}caracteristicas}\PYG{n+nt}{\PYGZgt{}}
                                \PYG{n+nt}{\PYGZlt{}memoria} \PYG{n+na}{medida=}\PYG{l+s}{\PYGZdq{}GB\PYGZdq{}}\PYG{n+nt}{\PYGZgt{}}2\PYG{n+nt}{\PYGZlt{}/memoria\PYGZgt{}}
                                \PYG{n+nt}{\PYGZlt{}tamanio} \PYG{n+na}{medida=}\PYG{l+s}{\PYGZdq{}pulgadas\PYGZdq{}}\PYG{n+nt}{\PYGZgt{}}6\PYG{n+nt}{\PYGZlt{}/tamanio\PYGZgt{}}
                                \PYG{n+nt}{\PYGZlt{}bateria}\PYG{n+nt}{\PYGZgt{}}LiPo\PYG{n+nt}{\PYGZlt{}/bateria\PYGZgt{}}
                        \PYG{n+nt}{\PYGZlt{}/caracteristicas\PYGZgt{}}
                \PYG{n+nt}{\PYGZlt{}/tablet\PYGZgt{}}
                \PYG{n+nt}{\PYGZlt{}tablet}\PYG{n+nt}{\PYGZgt{}}
                        \PYG{n+nt}{\PYGZlt{}plataforma}\PYG{n+nt}{\PYGZgt{}}iOS\PYG{n+nt}{\PYGZlt{}/plataforma\PYGZgt{}}
                        \PYG{n+nt}{\PYGZlt{}caracteristicas}\PYG{n+nt}{\PYGZgt{}}
                                \PYG{n+nt}{\PYGZlt{}memoria} \PYG{n+na}{medida=}\PYG{l+s}{\PYGZdq{}GB\PYGZdq{}}\PYG{n+nt}{\PYGZgt{}}4\PYG{n+nt}{\PYGZlt{}/memoria\PYGZgt{}}
                                \PYG{n+nt}{\PYGZlt{}tamanio} \PYG{n+na}{medida=}\PYG{l+s}{\PYGZdq{}pulgadas\PYGZdq{}}\PYG{n+nt}{\PYGZgt{}}9\PYG{n+nt}{\PYGZlt{}/tamanio\PYGZgt{}}
                                \PYG{n+nt}{\PYGZlt{}bateria}\PYG{n+nt}{\PYGZgt{}}LiIon\PYG{n+nt}{\PYGZlt{}/bateria\PYGZgt{}}
                        \PYG{n+nt}{\PYGZlt{}/caracteristicas\PYGZgt{}}
                \PYG{n+nt}{\PYGZlt{}/tablet\PYGZgt{}}
        \PYG{n+nt}{\PYGZlt{}/tablets\PYGZgt{}}
\PYG{n+nt}{\PYGZlt{}/pedido\PYGZgt{}}
\end{sphinxVerbatim}

Crear un fichero de estilos que permita mostrar la información de los portátiles en forma de tabla.

\begin{figure}[htbp]
\centering
\capstart

\noindent\sphinxincludegraphics{{xsl1}.png}
\caption{Transformacion XSL}\label{\detokenize{tema7:id10}}\end{figure}

Una posible solución sería esta:

\begin{sphinxVerbatim}[commandchars=\\\{\}]
\PYG{c+cp}{\PYGZlt{}?xml version=\PYGZdq{}1.0\PYGZdq{}?\PYGZgt{}}
\PYG{n+nt}{\PYGZlt{}xsl:stylesheet} \PYG{n+na}{version=}\PYG{l+s}{\PYGZdq{}1.1\PYGZdq{}}
        \PYG{n+na}{xmlns:xsl=}\PYG{l+s}{\PYGZdq{}http://www.w3.org/1999/XSL/Transform\PYGZdq{}}\PYG{n+nt}{\PYGZgt{}}
                \PYG{n+nt}{\PYGZlt{}xsl:template} \PYG{n+na}{match=}\PYG{l+s}{\PYGZdq{}/\PYGZdq{}}\PYG{n+nt}{\PYGZgt{}}
                        \PYG{n+nt}{\PYGZlt{}html}\PYG{n+nt}{\PYGZgt{}}
                        \PYG{n+nt}{\PYGZlt{}head}\PYG{n+nt}{\PYGZgt{}}
                                \PYG{n+nt}{\PYGZlt{}title}\PYG{n+nt}{\PYGZgt{}}Ejercicio 1\PYG{n+nt}{\PYGZlt{}/title\PYGZgt{}}
                        \PYG{n+nt}{\PYGZlt{}/head\PYGZgt{}}
                        \PYG{n+nt}{\PYGZlt{}body}\PYG{n+nt}{\PYGZgt{}}
                        \PYG{n+nt}{\PYGZlt{}h1}\PYG{n+nt}{\PYGZgt{}}Resultado\PYG{n+nt}{\PYGZlt{}/h1\PYGZgt{}}
                        \PYG{n+nt}{\PYGZlt{}table} \PYG{n+na}{border=}\PYG{l+s}{\PYGZdq{}1\PYGZdq{}}\PYG{n+nt}{\PYGZgt{}}
                        \PYG{n+nt}{\PYGZlt{}tr}\PYG{n+nt}{\PYGZgt{}}
                                        \PYG{n+nt}{\PYGZlt{}td}\PYG{n+nt}{\PYGZgt{}}Peso\PYG{n+nt}{\PYGZlt{}/td\PYGZgt{}}
                                        \PYG{n+nt}{\PYGZlt{}td}\PYG{n+nt}{\PYGZgt{}}RAM\PYG{n+nt}{\PYGZlt{}/td\PYGZgt{}}
                                        \PYG{n+nt}{\PYGZlt{}td}\PYG{n+nt}{\PYGZgt{}}Disco\PYG{n+nt}{\PYGZlt{}/td\PYGZgt{}}
                                        \PYG{n+nt}{\PYGZlt{}td}\PYG{n+nt}{\PYGZgt{}}Precio\PYG{n+nt}{\PYGZlt{}/td\PYGZgt{}}
                        \PYG{n+nt}{\PYGZlt{}/tr\PYGZgt{}}
                        \PYG{n+nt}{\PYGZlt{}xsl:for\PYGZhy{}each} \PYG{n+na}{select=}
                                \PYG{l+s}{\PYGZdq{}pedido/portatiles/portatil\PYGZdq{}}\PYG{n+nt}{\PYGZgt{}}
                        \PYG{n+nt}{\PYGZlt{}tr}\PYG{n+nt}{\PYGZgt{}}
                                \PYG{n+nt}{\PYGZlt{}td}\PYG{n+nt}{\PYGZgt{}}
                                        \PYG{n+nt}{\PYGZlt{}xsl:value\PYGZhy{}of} \PYG{n+na}{select=}\PYG{l+s}{\PYGZdq{}peso\PYGZdq{}}\PYG{n+nt}{/\PYGZgt{}}
                                \PYG{n+nt}{\PYGZlt{}/td\PYGZgt{}}
                                \PYG{n+nt}{\PYGZlt{}td}\PYG{n+nt}{\PYGZgt{}}
                                        \PYG{n+nt}{\PYGZlt{}xsl:value\PYGZhy{}of} \PYG{n+na}{select=}\PYG{l+s}{\PYGZdq{}ram\PYGZdq{}}\PYG{n+nt}{/\PYGZgt{}}
                                \PYG{n+nt}{\PYGZlt{}/td\PYGZgt{}}
                                \PYG{n+nt}{\PYGZlt{}td}\PYG{n+nt}{\PYGZgt{}}
                                        \PYG{n+nt}{\PYGZlt{}xsl:value\PYGZhy{}of} \PYG{n+na}{select=}\PYG{l+s}{\PYGZdq{}disco\PYGZdq{}}\PYG{n+nt}{/\PYGZgt{}}
                                \PYG{n+nt}{\PYGZlt{}/td\PYGZgt{}}
                                \PYG{n+nt}{\PYGZlt{}td}\PYG{n+nt}{\PYGZgt{}}
                                        \PYG{n+nt}{\PYGZlt{}xsl:value\PYGZhy{}of} \PYG{n+na}{select=}\PYG{l+s}{\PYGZdq{}precio\PYGZdq{}}\PYG{n+nt}{\PYGZgt{}}
                                \PYG{n+nt}{\PYGZlt{}/td\PYGZgt{}}
                        \PYG{n+nt}{\PYGZlt{}/tr\PYGZgt{}}
                        \PYG{n+nt}{\PYGZlt{}/xsl:for\PYGZhy{}each\PYGZgt{}}
                        \PYG{n+nt}{\PYGZlt{}/table\PYGZgt{}}
                \PYG{n+nt}{\PYGZlt{}/body\PYGZgt{}}
                \PYG{n+nt}{\PYGZlt{}/html\PYGZgt{}}
        \PYG{n+nt}{\PYGZlt{}/xsl:template\PYGZgt{}}
\PYG{n+nt}{\PYGZlt{}/xsl:stylesheet\PYGZgt{}}
\end{sphinxVerbatim}


\subsection{Transformación de pedidos (II)}
\label{\detokenize{tema7:transformacion-de-pedidos-ii}}
Con el mismo fichero de pedidos crear una sola tabla que tenga 3 columnas y aglutine información tanto de portátiles como de tablets:
\begin{itemize}
\item {} 
Cuando procesemos portátiles, las columnas serán respectivamente «precio», «ram» y «disco». Solo se procesan portátiles con más de 2GB de RAM.

\item {} 
Cuando procesemos tablets, las columnas serán «plataforma», «ram» y «batería». Solo se procesan los tablets con más de 2GB de RAM y que además tengan un tamaño superior a 7 pulgadas.

\end{itemize}

El fichero siguiente ilustra una posible forma de hacerlo:

\begin{sphinxVerbatim}[commandchars=\\\{\}]
\PYG{c+cp}{\PYGZlt{}?xml version=\PYGZdq{}1.0\PYGZdq{}?\PYGZgt{}}
\PYG{n+nt}{\PYGZlt{}xsl:stylesheet} \PYG{n+na}{version=}\PYG{l+s}{\PYGZdq{}1.1\PYGZdq{}}
\PYG{n+na}{xmlns:xsl=}\PYG{l+s}{\PYGZdq{}http://www.w3.org/1999/XSL/Transform\PYGZdq{}}\PYG{n+nt}{\PYGZgt{}}
        \PYG{n+nt}{\PYGZlt{}xsl:template} \PYG{n+na}{match=}\PYG{l+s}{\PYGZdq{}/\PYGZdq{}}\PYG{n+nt}{\PYGZgt{}}
                \PYG{n+nt}{\PYGZlt{}html}\PYG{n+nt}{\PYGZgt{}}
                \PYG{n+nt}{\PYGZlt{}head}\PYG{n+nt}{\PYGZgt{}}
                        \PYG{n+nt}{\PYGZlt{}title}\PYG{n+nt}{\PYGZgt{}}Ejercicio 1\PYG{n+nt}{\PYGZlt{}/title\PYGZgt{}}
                \PYG{n+nt}{\PYGZlt{}/head\PYGZgt{}}
                \PYG{n+nt}{\PYGZlt{}body}\PYG{n+nt}{\PYGZgt{}}
                        \PYG{n+nt}{\PYGZlt{}h1}\PYG{n+nt}{\PYGZgt{}}Resultado\PYG{n+nt}{\PYGZlt{}/h1\PYGZgt{}}
                        \PYG{n+nt}{\PYGZlt{}table} \PYG{n+na}{border=}\PYG{l+s}{\PYGZdq{}1\PYGZdq{}}\PYG{n+nt}{\PYGZgt{}}
                        \PYG{n+nt}{\PYGZlt{}xsl:for\PYGZhy{}each} \PYG{n+na}{select=}
                        \PYG{l+s}{\PYGZdq{}pedido/portatiles/portatil\PYGZdq{}}\PYG{n+nt}{\PYGZgt{}}
                           \PYG{n+nt}{\PYGZlt{}xsl:if} \PYG{n+na}{test=}\PYG{l+s}{\PYGZdq{}ram \PYGZam{}gt; 2\PYGZdq{}}\PYG{n+nt}{\PYGZgt{}}
                                \PYG{n+nt}{\PYGZlt{}tr}\PYG{n+nt}{\PYGZgt{}}
                                  \PYG{n+nt}{\PYGZlt{}td}\PYG{n+nt}{\PYGZgt{}}
                                        Precio:\PYG{n+nt}{\PYGZlt{}xsl:value\PYGZhy{}of} \PYG{n+na}{select=}\PYG{l+s}{\PYGZdq{}precio\PYGZdq{}}\PYG{n+nt}{/\PYGZgt{}}
                                  \PYG{n+nt}{\PYGZlt{}/td\PYGZgt{}}
                                  \PYG{n+nt}{\PYGZlt{}td}\PYG{n+nt}{\PYGZgt{}}
                                        Memoria:\PYG{n+nt}{\PYGZlt{}xsl:value\PYGZhy{}of} \PYG{n+na}{select=}\PYG{l+s}{\PYGZdq{}ram\PYGZdq{}}\PYG{n+nt}{/\PYGZgt{}}
                              \PYG{n+nt}{\PYGZlt{}/td\PYGZgt{}}
                                  \PYG{n+nt}{\PYGZlt{}td}\PYG{n+nt}{\PYGZgt{}}
                                        Disco duro:\PYG{n+nt}{\PYGZlt{}xsl:value\PYGZhy{}of} \PYG{n+na}{select=}\PYG{l+s}{\PYGZdq{}disco\PYGZdq{}}\PYG{n+nt}{/\PYGZgt{}}
                                  \PYG{n+nt}{\PYGZlt{}/td\PYGZgt{}}
                                \PYG{n+nt}{\PYGZlt{}/tr\PYGZgt{}}
                        \PYG{n+nt}{\PYGZlt{}/xsl:if\PYGZgt{}}
                        \PYG{n+nt}{\PYGZlt{}/xsl:for\PYGZhy{}each\PYGZgt{}}
                        \PYG{n+nt}{\PYGZlt{}xsl:for\PYGZhy{}each} \PYG{n+na}{select=}\PYG{l+s}{\PYGZdq{}pedido/tablets/tablet\PYGZdq{}}\PYG{n+nt}{\PYGZgt{}}
                           \PYG{n+nt}{\PYGZlt{}xsl:if} \PYG{n+na}{test=}\PYG{l+s}{\PYGZdq{}caracteristicas/memoria \PYGZam{}gt; 2\PYGZdq{}}\PYG{n+nt}{\PYGZgt{}}
                                \PYG{n+nt}{\PYGZlt{}xsl:if} \PYG{n+na}{test=}\PYG{l+s}{\PYGZdq{}caracteristicas/tamanio \PYGZam{}gt; 7\PYGZdq{}}\PYG{n+nt}{\PYGZgt{}}
                                        \PYG{n+nt}{\PYGZlt{}tr}\PYG{n+nt}{\PYGZgt{}}
                                        \PYG{n+nt}{\PYGZlt{}td}\PYG{n+nt}{\PYGZgt{}}
                                        \PYG{n+nt}{\PYGZlt{}xsl:value\PYGZhy{}of} \PYG{n+na}{select=}\PYG{l+s}{\PYGZdq{}plataforma\PYGZdq{}}\PYG{n+nt}{/\PYGZgt{}}
                                        \PYG{n+nt}{\PYGZlt{}/td\PYGZgt{}}                                           \PYG{n+nt}{\PYGZlt{}td}\PYG{n+nt}{\PYGZgt{}}
                                        \PYG{n+nt}{\PYGZlt{}xsl:value\PYGZhy{}of} \PYG{n+na}{select=}\PYG{l+s}{\PYGZdq{}caracteristicas/memoria\PYGZdq{}}\PYG{n+nt}{/\PYGZgt{}}
                                        \PYG{n+nt}{\PYGZlt{}/td\PYGZgt{}}
                                        \PYG{n+nt}{\PYGZlt{}td}\PYG{n+nt}{\PYGZgt{}}
                                        \PYG{n+nt}{\PYGZlt{}xsl:value\PYGZhy{}of} \PYG{n+na}{select=}\PYG{l+s}{\PYGZdq{}caracteristicas/bateria\PYGZdq{}}\PYG{n+nt}{/\PYGZgt{}}
                                        \PYG{n+nt}{\PYGZlt{}/td\PYGZgt{}}
                                \PYG{n+nt}{\PYGZlt{}/tr\PYGZgt{}}
                        \PYG{n+nt}{\PYGZlt{}/xsl:if\PYGZgt{}}
                                \PYG{n+nt}{\PYGZlt{}/xsl:if\PYGZgt{}}
                                \PYG{n+nt}{\PYGZlt{}/xsl:for\PYGZhy{}each\PYGZgt{}}
                        \PYG{n+nt}{\PYGZlt{}/table\PYGZgt{}}
                \PYG{n+nt}{\PYGZlt{}/body\PYGZgt{}}
                \PYG{n+nt}{\PYGZlt{}/html\PYGZgt{}}
        \PYG{n+nt}{\PYGZlt{}/xsl:template\PYGZgt{}}
\PYG{n+nt}{\PYGZlt{}/xsl:stylesheet\PYGZgt{}}
\end{sphinxVerbatim}


\subsection{Ejercicio (no se da la solución)}
\label{\detokenize{tema7:ejercicio-no-se-da-la-solucion}}
Poner en una lista ordenada (elemento \sphinxcode{ol}) todas las capacidades RAM que se encuentren en el fichero XML.


\chapter{Sistemas de gestión de información}
\label{\detokenize{tema8::doc}}\label{\detokenize{tema8:sistemas-de-gestion-de-informacion}}

\section{Introducción}
\label{\detokenize{tema8:introduccion}}
Primero definiremos los términos:
\begin{itemize}
\item {} 
Sistema: «conjunto de elementos interrelacionados que colaboran para la consecución de un objetivo».

\item {} 
Gestión de información: movimientos de información que facilitan la consecución de los objetivos de la empresa.

\end{itemize}

Entre los objetivos más habituales están:
\begin{itemize}
\item {} 
Objetivos económicos.

\item {} 
Objetivos temporales (supervivencia).

\item {} 
Objetivos legales, entre los que destacan los fiscales.

\end{itemize}

Procesar la información hoy en día es muy difícil, por lo que la informática es de gran ayuda en dicho procesamiento: (INFORmación+autoMÁTICA).

¿Todos los sistemas de una empresa son informáticos? NO. Un ejemplo es el sistema postal que no requiere (y no siempre ha requerido) parte informática.


\section{Niveles en la empresa}
\label{\detokenize{tema8:niveles-en-la-empresa}}
Toda empresa se puede ver desde tres puntos de vista temporales distintos
\begin{enumerate}
\item {} 
Nivel estratégico: en este nivel están las personas o elementos que tienen un punto de vista más global y más largo plazo (normalmente plazos de varios años).

\item {} 
Nivel táctico: en este nivel están los mandos intermedios de la empresa con plazos medios (máximo un año)

\item {} 
Nivel operativo: está el personal «de a pie», que recoge la información a diario y la procesa a diario.

\end{enumerate}

Ejemplo: Mercadona.


\subsection{Tipos de SGI informáticos}
\label{\detokenize{tema8:tipos-de-sgi-informaticos}}\begin{itemize}
\item {} 
DSS o Decision Support System: ayudan a tomar decisiones en función de la información disponible.

\item {} 
CRM o Customer Relationship Managements o Gestores de la relación con clientes. Son sistemas que gestionan información muy personalizada a los clientes, tales como sistemas de recomendación, sistemas de fidelización. Los CRM manejan la comunicación entre información interna y externa de la empresa.

\item {} 
ERP o Enterprise Resource Planning o sistemas de planificación de recursos corporativos gestionan toda la información interna de la empresa.

\end{itemize}

Los ERP han ganado mucho auge con el tiempo, ya que pueden hacer operaciones muy sofisticadas, entre ellas:
\begin{enumerate}
\item {} 
Optimización de procesos: implica resolver complejos problemas matemáticos que con una herramienta informática se resuelven al instante.

\item {} 
Logística: implica resolver problemas de transporte de productos en base a diversas restricciones (no todas matemáticas). Un problema muy común es la optimización de rutas que implica resolver el «problema del viajante».

\item {} 
Fiscalidad: permite aprovechar las diferentes ventajas que en materia de fiscalidad se ofrecen en distintos casos, en distintos territorios…

\item {} 
Contabilidad: la participación en diversas sociedades empresariales puede complicar las operaciones de contabilidad, operaciones que un ERP puede registrar.

\item {} 
Optimización de recursos humanos: optimizar el volumen de contrataciones y despidos, nóminas, etc…

\item {} 
Operaciones de informes y estadística: para conocer el estado real de la empresa en sus distintos departamentos y períodos.

\item {} 
Trazabilidad: permite conocer el punto exacto de ubicación de un producto a lo largo de toda la cadena de abastecimiento (supply chain).

\end{enumerate}


\section{Almacenamiento en los SGI}
\label{\detokenize{tema8:almacenamiento-en-los-sgi}}
La mayor parte de SGI se apoyan en tecnologías conocidas como SGBD con SQL e incluso XML.

La mayor parte de SGBD permite procesar y extraer información en forma de SQL.

En general, cualquier consulta sobre cualquier tabla puede convertirse en XML. Cuando por ejemplo hacemos

\begin{sphinxVerbatim}[commandchars=\\\{\}]
\PYG{k}{select} \PYG{o}{*} \PYG{k}{from} \PYG{n}{marcas}\PYG{p}{;}
\end{sphinxVerbatim}

obtenemos un resultado como este:

\begin{sphinxVerbatim}[commandchars=\\\{\}]
\PYG{o}{+}\PYG{o}{\PYGZhy{}}\PYG{o}{\PYGZhy{}}\PYG{o}{\PYGZhy{}}\PYG{o}{\PYGZhy{}}\PYG{o}{+}\PYG{o}{\PYGZhy{}}\PYG{o}{\PYGZhy{}}\PYG{o}{\PYGZhy{}}\PYG{o}{\PYGZhy{}}\PYG{o}{\PYGZhy{}}\PYG{o}{\PYGZhy{}}\PYG{o}{\PYGZhy{}}\PYG{o}{\PYGZhy{}}\PYG{o}{+}\PYG{o}{\PYGZhy{}}\PYG{o}{\PYGZhy{}}\PYG{o}{\PYGZhy{}}\PYG{o}{\PYGZhy{}}\PYG{o}{\PYGZhy{}}\PYG{o}{\PYGZhy{}}\PYG{o}{\PYGZhy{}}\PYG{o}{\PYGZhy{}}\PYG{o}{\PYGZhy{}}\PYG{o}{\PYGZhy{}}\PYG{o}{\PYGZhy{}}\PYG{o}{\PYGZhy{}}\PYG{o}{\PYGZhy{}}\PYG{o}{+}
\PYG{o}{\textbar{}} \PYG{n+nb}{id} \PYG{o}{\textbar{}} \PYG{n}{nombre} \PYG{o}{\textbar{}} \PYG{n}{ap}          \PYG{o}{\textbar{}}
\PYG{o}{+}\PYG{o}{\PYGZhy{}}\PYG{o}{\PYGZhy{}}\PYG{o}{\PYGZhy{}}\PYG{o}{\PYGZhy{}}\PYG{o}{+}\PYG{o}{\PYGZhy{}}\PYG{o}{\PYGZhy{}}\PYG{o}{\PYGZhy{}}\PYG{o}{\PYGZhy{}}\PYG{o}{\PYGZhy{}}\PYG{o}{\PYGZhy{}}\PYG{o}{\PYGZhy{}}\PYG{o}{\PYGZhy{}}\PYG{o}{+}\PYG{o}{\PYGZhy{}}\PYG{o}{\PYGZhy{}}\PYG{o}{\PYGZhy{}}\PYG{o}{\PYGZhy{}}\PYG{o}{\PYGZhy{}}\PYG{o}{\PYGZhy{}}\PYG{o}{\PYGZhy{}}\PYG{o}{\PYGZhy{}}\PYG{o}{\PYGZhy{}}\PYG{o}{\PYGZhy{}}\PYG{o}{\PYGZhy{}}\PYG{o}{\PYGZhy{}}\PYG{o}{\PYGZhy{}}\PYG{o}{+}
\PYG{o}{\textbar{}}  \PYG{l+m+mi}{1} \PYG{o}{\textbar{}} \PYG{n}{Juan}   \PYG{o}{\textbar{}} \PYG{n}{Lopez} \PYG{n}{Lopez} \PYG{o}{\textbar{}}
\PYG{o}{\textbar{}}  \PYG{l+m+mi}{2} \PYG{o}{\textbar{}} \PYG{n}{Andres} \PYG{o}{\textbar{}} \PYG{n}{Ruiz} \PYG{n}{Gomez}  \PYG{o}{\textbar{}}
\PYG{o}{\textbar{}}  \PYG{l+m+mi}{3} \PYG{o}{\textbar{}} \PYG{n}{Tomas}  \PYG{o}{\textbar{}} \PYG{n}{Perez} \PYG{n}{Diaz}  \PYG{o}{\textbar{}}
\PYG{o}{+}\PYG{o}{\PYGZhy{}}\PYG{o}{\PYGZhy{}}\PYG{o}{\PYGZhy{}}\PYG{o}{\PYGZhy{}}\PYG{o}{+}\PYG{o}{\PYGZhy{}}\PYG{o}{\PYGZhy{}}\PYG{o}{\PYGZhy{}}\PYG{o}{\PYGZhy{}}\PYG{o}{\PYGZhy{}}\PYG{o}{\PYGZhy{}}\PYG{o}{\PYGZhy{}}\PYG{o}{\PYGZhy{}}\PYG{o}{+}\PYG{o}{\PYGZhy{}}\PYG{o}{\PYGZhy{}}\PYG{o}{\PYGZhy{}}\PYG{o}{\PYGZhy{}}\PYG{o}{\PYGZhy{}}\PYG{o}{\PYGZhy{}}\PYG{o}{\PYGZhy{}}\PYG{o}{\PYGZhy{}}\PYG{o}{\PYGZhy{}}\PYG{o}{\PYGZhy{}}\PYG{o}{\PYGZhy{}}\PYG{o}{\PYGZhy{}}\PYG{o}{\PYGZhy{}}\PYG{o}{+}

\PYG{l+m+mi}{3} \PYG{n}{rows} \PYG{o+ow}{in} \PYG{n+nb}{set} \PYG{p}{(}\PYG{l+m+mf}{0.00} \PYG{n}{sec}\PYG{p}{)}
\end{sphinxVerbatim}

Sin embargo, podemos ejecutar el siguiente comando que se conecta a MySQL ejecuta la consulta dada y devuelve un fichero XML:

\begin{sphinxVerbatim}[commandchars=\\\{\}]
\PYG{n}{mysql} \PYG{o}{\PYGZhy{}}\PYG{n}{uroot} \PYG{o}{\PYGZhy{}}\PYG{n}{Ddatos} \PYG{o}{\PYGZhy{}}\PYG{n}{e}\PYG{l+s+s2}{\PYGZdq{}}\PYG{l+s+s2}{select * from usuarios}\PYG{l+s+s2}{\PYGZdq{}} \PYG{o}{\PYGZhy{}}\PYG{o}{\PYGZhy{}}\PYG{n}{xml}
\end{sphinxVerbatim}

El resultado será algo como esto:

\begin{sphinxVerbatim}[commandchars=\\\{\}]
 \PYG{n+nt}{\PYGZlt{}resultset} \PYG{n+na}{statement=}\PYG{l+s}{\PYGZdq{}select * from usuarios\PYGZdq{}}
\PYG{n+na}{xmlns:xsi=}\PYG{l+s}{\PYGZdq{}http://www.w3.org/2001/XMLSchema\PYGZhy{}instance\PYGZdq{}}\PYG{n+nt}{\PYGZgt{}}
   \PYG{n+nt}{\PYGZlt{}row}\PYG{n+nt}{\PYGZgt{}}
                 \PYG{n+nt}{\PYGZlt{}field} \PYG{n+na}{name=}\PYG{l+s}{\PYGZdq{}id\PYGZdq{}}\PYG{n+nt}{\PYGZgt{}}1\PYG{n+nt}{\PYGZlt{}/field\PYGZgt{}}
                 \PYG{n+nt}{\PYGZlt{}field} \PYG{n+na}{name=}\PYG{l+s}{\PYGZdq{}nombre\PYGZdq{}}\PYG{n+nt}{\PYGZgt{}}Juan\PYG{n+nt}{\PYGZlt{}/field\PYGZgt{}}
                 \PYG{n+nt}{\PYGZlt{}field} \PYG{n+na}{name=}\PYG{l+s}{\PYGZdq{}ap\PYGZdq{}}\PYG{n+nt}{\PYGZgt{}}Lopez Lopez\PYG{n+nt}{\PYGZlt{}/field\PYGZgt{}}
   \PYG{n+nt}{\PYGZlt{}/row\PYGZgt{}}

   \PYG{n+nt}{\PYGZlt{}row}\PYG{n+nt}{\PYGZgt{}}
                 \PYG{n+nt}{\PYGZlt{}field} \PYG{n+na}{name=}\PYG{l+s}{\PYGZdq{}id\PYGZdq{}}\PYG{n+nt}{\PYGZgt{}}2\PYG{n+nt}{\PYGZlt{}/field\PYGZgt{}}
                 \PYG{n+nt}{\PYGZlt{}field} \PYG{n+na}{name=}\PYG{l+s}{\PYGZdq{}nombre\PYGZdq{}}\PYG{n+nt}{\PYGZgt{}}Andres\PYG{n+nt}{\PYGZlt{}/field\PYGZgt{}}
                 \PYG{n+nt}{\PYGZlt{}field} \PYG{n+na}{name=}\PYG{l+s}{\PYGZdq{}ap\PYGZdq{}}\PYG{n+nt}{\PYGZgt{}}Ruiz Gomez\PYG{n+nt}{\PYGZlt{}/field\PYGZgt{}}
   \PYG{n+nt}{\PYGZlt{}/row\PYGZgt{}}

   \PYG{n+nt}{\PYGZlt{}row}\PYG{n+nt}{\PYGZgt{}}
                 \PYG{n+nt}{\PYGZlt{}field} \PYG{n+na}{name=}\PYG{l+s}{\PYGZdq{}id\PYGZdq{}}\PYG{n+nt}{\PYGZgt{}}3\PYG{n+nt}{\PYGZlt{}/field\PYGZgt{}}
                 \PYG{n+nt}{\PYGZlt{}field} \PYG{n+na}{name=}\PYG{l+s}{\PYGZdq{}nombre\PYGZdq{}}\PYG{n+nt}{\PYGZgt{}}Tomas\PYG{n+nt}{\PYGZlt{}/field\PYGZgt{}}
                 \PYG{n+nt}{\PYGZlt{}field} \PYG{n+na}{name=}\PYG{l+s}{\PYGZdq{}ap\PYGZdq{}}\PYG{n+nt}{\PYGZgt{}}Perez Diaz\PYG{n+nt}{\PYGZlt{}/field\PYGZgt{}}
   \PYG{n+nt}{\PYGZlt{}/row\PYGZgt{}}
 \PYG{n+nt}{\PYGZlt{}/resultset\PYGZgt{}}
\end{sphinxVerbatim}


\section{Puntos de vista en los SGI}
\label{\detokenize{tema8:puntos-de-vista-en-los-sgi}}

\subsection{Desarrollador}
\label{\detokenize{tema8:desarrollador}}
Para un desarrollador hay ciertos temas fundamentales a conocer del SGI con el que trabaje:
\begin{itemize}
\item {} 
Programable ¿qué lenguaje de programación usa? ¿usa OOP? ¿es un tipo de lenguaje distinto?.

\item {} 
¿Se puede conectar desde algún lenguaje general?

\item {} 
¿Se pueden usar otras bibliotecas de uso general? (Hibernate)

\end{itemize}


\subsection{Administrador}
\label{\detokenize{tema8:administrador}}\begin{itemize}
\item {} 
¿Se pueden controlar  los accesos?

\item {} 
¿Como cumplir la LOPD y sus 3 niveles?

\end{itemize}


\subsection{Usuario}
\label{\detokenize{tema8:usuario}}\begin{itemize}
\item {} 
¿Como se usa el SGI? En unos casos hay muchos requisitos pero cada vez más abundan los SGI con interfaz Web.

\item {} 
La adaptación específica que ofrece el sistema a la empresa. El PGC en España es un elemento al cual un programa puede estar adaptado o no. Otra posibilidad es que el programa permita INCOTERMS.

\end{itemize}


\subsection{Ejercicio}
\label{\detokenize{tema8:ejercicio}}
Modelar el diagrama de estados para el cajero en el caso «Sacar dinero»


\chapter{Anexo: ejercicios sobre tablas}
\label{\detokenize{ejercicios/html/anexo_tablas::doc}}\label{\detokenize{ejercicios/html/anexo_tablas:anexo-ejercicios-sobre-tablas}}
Las tablas HTML muestran cierta complejidad cuando se anidan. En los ejercicios
siguientes se muestran algunas tablas junto con su resolución en HTML. Los
ejercicios no se muestran con ningún orden de dificultad.


\section{Tabla 1}
\label{\detokenize{ejercicios/html/anexo_tablas:tabla-1}}
Generar la tabla siguiente

\noindent{\hspace*{\fill}\sphinxincludegraphics[scale=0.6]{{foto_01}.png}\hspace*{\fill}}

Solución:

\begin{sphinxVerbatim}[commandchars=\\\{\}]
\PYG{c+cp}{\PYGZlt{}!DOCTYPE html\PYGZgt{}}
\PYG{p}{\PYGZlt{}}\PYG{n+nt}{html}\PYG{p}{\PYGZgt{}}
\PYG{p}{\PYGZlt{}}\PYG{n+nt}{head}\PYG{p}{\PYGZgt{}}
\PYG{p}{\PYGZlt{}}\PYG{n+nt}{meta} \PYG{n+na}{charset}\PYG{o}{=}\PYG{l+s}{\PYGZsq{}utf\PYGZhy{}8\PYGZsq{}}\PYG{p}{\PYGZgt{}}
\PYG{p}{\PYGZlt{}}\PYG{n+nt}{title}\PYG{p}{\PYGZgt{}}Ejercicio\PYG{p}{\PYGZlt{}}\PYG{p}{/}\PYG{n+nt}{title}\PYG{p}{\PYGZgt{}}
\PYG{p}{\PYGZlt{}}\PYG{p}{/}\PYG{n+nt}{head}\PYG{p}{\PYGZgt{}}
\PYG{p}{\PYGZlt{}}\PYG{n+nt}{body}\PYG{p}{\PYGZgt{}}
	\PYG{p}{\PYGZlt{}}\PYG{n+nt}{table} \PYG{n+na}{border}\PYG{o}{=}\PYG{l+s}{\PYGZsq{}1\PYGZsq{}}\PYG{p}{\PYGZgt{}}
		\PYG{p}{\PYGZlt{}}\PYG{n+nt}{tr}\PYG{p}{\PYGZgt{}}
			\PYG{p}{\PYGZlt{}}\PYG{n+nt}{td}\PYG{p}{\PYGZgt{}} Celda \PYG{p}{\PYGZlt{}}\PYG{p}{/}\PYG{n+nt}{td}\PYG{p}{\PYGZgt{}}
			\PYG{p}{\PYGZlt{}}\PYG{n+nt}{td}\PYG{p}{\PYGZgt{}} Celda \PYG{p}{\PYGZlt{}}\PYG{p}{/}\PYG{n+nt}{td}\PYG{p}{\PYGZgt{}}
			\PYG{p}{\PYGZlt{}}\PYG{n+nt}{td}\PYG{p}{\PYGZgt{}} Celda \PYG{p}{\PYGZlt{}}\PYG{p}{/}\PYG{n+nt}{td}\PYG{p}{\PYGZgt{}}
		\PYG{p}{\PYGZlt{}}\PYG{p}{/}\PYG{n+nt}{tr}\PYG{p}{\PYGZgt{}}
		\PYG{p}{\PYGZlt{}}\PYG{n+nt}{tr}\PYG{p}{\PYGZgt{}}
			\PYG{p}{\PYGZlt{}}\PYG{n+nt}{td}\PYG{p}{\PYGZgt{}} Celda \PYG{p}{\PYGZlt{}}\PYG{p}{/}\PYG{n+nt}{td}\PYG{p}{\PYGZgt{}}
			\PYG{p}{\PYGZlt{}}\PYG{n+nt}{td}\PYG{p}{\PYGZgt{}}
			\PYG{p}{\PYGZlt{}}\PYG{n+nt}{table} \PYG{n+na}{border}\PYG{o}{=}\PYG{l+s}{\PYGZsq{}1\PYGZsq{}}\PYG{p}{\PYGZgt{}}
				\PYG{p}{\PYGZlt{}}\PYG{n+nt}{tr}\PYG{p}{\PYGZgt{}}
					\PYG{p}{\PYGZlt{}}\PYG{n+nt}{td}\PYG{p}{\PYGZgt{}} Celda \PYG{p}{\PYGZlt{}}\PYG{p}{/}\PYG{n+nt}{td}\PYG{p}{\PYGZgt{}}
					\PYG{p}{\PYGZlt{}}\PYG{n+nt}{td}\PYG{p}{\PYGZgt{}} Celda \PYG{p}{\PYGZlt{}}\PYG{p}{/}\PYG{n+nt}{td}\PYG{p}{\PYGZgt{}}
					\PYG{p}{\PYGZlt{}}\PYG{n+nt}{td}\PYG{p}{\PYGZgt{}} Celda \PYG{p}{\PYGZlt{}}\PYG{p}{/}\PYG{n+nt}{td}\PYG{p}{\PYGZgt{}}
					\PYG{p}{\PYGZlt{}}\PYG{n+nt}{td}\PYG{p}{\PYGZgt{}} Celda \PYG{p}{\PYGZlt{}}\PYG{p}{/}\PYG{n+nt}{td}\PYG{p}{\PYGZgt{}}
				\PYG{p}{\PYGZlt{}}\PYG{p}{/}\PYG{n+nt}{tr}\PYG{p}{\PYGZgt{}}
				\PYG{p}{\PYGZlt{}}\PYG{n+nt}{tr}\PYG{p}{\PYGZgt{}}
					\PYG{p}{\PYGZlt{}}\PYG{n+nt}{td}\PYG{p}{\PYGZgt{}} Celda \PYG{p}{\PYGZlt{}}\PYG{p}{/}\PYG{n+nt}{td}\PYG{p}{\PYGZgt{}}
					\PYG{p}{\PYGZlt{}}\PYG{n+nt}{td}\PYG{p}{\PYGZgt{}} Celda \PYG{p}{\PYGZlt{}}\PYG{p}{/}\PYG{n+nt}{td}\PYG{p}{\PYGZgt{}}
					\PYG{p}{\PYGZlt{}}\PYG{n+nt}{td}\PYG{p}{\PYGZgt{}} Celda \PYG{p}{\PYGZlt{}}\PYG{p}{/}\PYG{n+nt}{td}\PYG{p}{\PYGZgt{}}
					\PYG{p}{\PYGZlt{}}\PYG{n+nt}{td}\PYG{p}{\PYGZgt{}} Celda \PYG{p}{\PYGZlt{}}\PYG{p}{/}\PYG{n+nt}{td}\PYG{p}{\PYGZgt{}}
				\PYG{p}{\PYGZlt{}}\PYG{p}{/}\PYG{n+nt}{tr}\PYG{p}{\PYGZgt{}}
				\PYG{p}{\PYGZlt{}}\PYG{n+nt}{tr}\PYG{p}{\PYGZgt{}}
					\PYG{p}{\PYGZlt{}}\PYG{n+nt}{td}\PYG{p}{\PYGZgt{}} Celda \PYG{p}{\PYGZlt{}}\PYG{p}{/}\PYG{n+nt}{td}\PYG{p}{\PYGZgt{}}
					\PYG{p}{\PYGZlt{}}\PYG{n+nt}{td}\PYG{p}{\PYGZgt{}} Celda \PYG{p}{\PYGZlt{}}\PYG{p}{/}\PYG{n+nt}{td}\PYG{p}{\PYGZgt{}}
					\PYG{p}{\PYGZlt{}}\PYG{n+nt}{td}\PYG{p}{\PYGZgt{}} Celda \PYG{p}{\PYGZlt{}}\PYG{p}{/}\PYG{n+nt}{td}\PYG{p}{\PYGZgt{}}
					\PYG{p}{\PYGZlt{}}\PYG{n+nt}{td}\PYG{p}{\PYGZgt{}} Celda \PYG{p}{\PYGZlt{}}\PYG{p}{/}\PYG{n+nt}{td}\PYG{p}{\PYGZgt{}}
				\PYG{p}{\PYGZlt{}}\PYG{p}{/}\PYG{n+nt}{tr}\PYG{p}{\PYGZgt{}}
				\PYG{p}{\PYGZlt{}}\PYG{n+nt}{tr}\PYG{p}{\PYGZgt{}}
					\PYG{p}{\PYGZlt{}}\PYG{n+nt}{td}\PYG{p}{\PYGZgt{}} Celda \PYG{p}{\PYGZlt{}}\PYG{p}{/}\PYG{n+nt}{td}\PYG{p}{\PYGZgt{}}
					\PYG{p}{\PYGZlt{}}\PYG{n+nt}{td}\PYG{p}{\PYGZgt{}} Celda \PYG{p}{\PYGZlt{}}\PYG{p}{/}\PYG{n+nt}{td}\PYG{p}{\PYGZgt{}}
					\PYG{p}{\PYGZlt{}}\PYG{n+nt}{td}\PYG{p}{\PYGZgt{}} Celda \PYG{p}{\PYGZlt{}}\PYG{p}{/}\PYG{n+nt}{td}\PYG{p}{\PYGZgt{}}
					\PYG{p}{\PYGZlt{}}\PYG{n+nt}{td}\PYG{p}{\PYGZgt{}} Celda \PYG{p}{\PYGZlt{}}\PYG{p}{/}\PYG{n+nt}{td}\PYG{p}{\PYGZgt{}}
				\PYG{p}{\PYGZlt{}}\PYG{p}{/}\PYG{n+nt}{tr}\PYG{p}{\PYGZgt{}}
			\PYG{p}{\PYGZlt{}}\PYG{p}{/}\PYG{n+nt}{table}\PYG{p}{\PYGZgt{}}
			\PYG{p}{\PYGZlt{}}\PYG{p}{/}\PYG{n+nt}{td}\PYG{p}{\PYGZgt{}}
			\PYG{p}{\PYGZlt{}}\PYG{n+nt}{td}\PYG{p}{\PYGZgt{}} Celda \PYG{p}{\PYGZlt{}}\PYG{p}{/}\PYG{n+nt}{td}\PYG{p}{\PYGZgt{}}
		\PYG{p}{\PYGZlt{}}\PYG{p}{/}\PYG{n+nt}{tr}\PYG{p}{\PYGZgt{}}
		\PYG{p}{\PYGZlt{}}\PYG{n+nt}{tr}\PYG{p}{\PYGZgt{}}
			\PYG{p}{\PYGZlt{}}\PYG{n+nt}{td}\PYG{p}{\PYGZgt{}} Celda \PYG{p}{\PYGZlt{}}\PYG{p}{/}\PYG{n+nt}{td}\PYG{p}{\PYGZgt{}}
			\PYG{p}{\PYGZlt{}}\PYG{n+nt}{td}\PYG{p}{\PYGZgt{}}
			\PYG{p}{\PYGZlt{}}\PYG{n+nt}{table} \PYG{n+na}{border}\PYG{o}{=}\PYG{l+s}{\PYGZsq{}1\PYGZsq{}}\PYG{p}{\PYGZgt{}}
				\PYG{p}{\PYGZlt{}}\PYG{n+nt}{tr}\PYG{p}{\PYGZgt{}}
					\PYG{p}{\PYGZlt{}}\PYG{n+nt}{td}\PYG{p}{\PYGZgt{}} Celda \PYG{p}{\PYGZlt{}}\PYG{p}{/}\PYG{n+nt}{td}\PYG{p}{\PYGZgt{}}
					\PYG{p}{\PYGZlt{}}\PYG{n+nt}{td}\PYG{p}{\PYGZgt{}} Celda \PYG{p}{\PYGZlt{}}\PYG{p}{/}\PYG{n+nt}{td}\PYG{p}{\PYGZgt{}}
					\PYG{p}{\PYGZlt{}}\PYG{n+nt}{td}\PYG{p}{\PYGZgt{}} Celda \PYG{p}{\PYGZlt{}}\PYG{p}{/}\PYG{n+nt}{td}\PYG{p}{\PYGZgt{}}
				\PYG{p}{\PYGZlt{}}\PYG{p}{/}\PYG{n+nt}{tr}\PYG{p}{\PYGZgt{}}
				\PYG{p}{\PYGZlt{}}\PYG{n+nt}{tr}\PYG{p}{\PYGZgt{}}
					\PYG{p}{\PYGZlt{}}\PYG{n+nt}{td}\PYG{p}{\PYGZgt{}} Celda \PYG{p}{\PYGZlt{}}\PYG{p}{/}\PYG{n+nt}{td}\PYG{p}{\PYGZgt{}}
					\PYG{p}{\PYGZlt{}}\PYG{n+nt}{td}\PYG{p}{\PYGZgt{}} Celda \PYG{p}{\PYGZlt{}}\PYG{p}{/}\PYG{n+nt}{td}\PYG{p}{\PYGZgt{}}
					\PYG{p}{\PYGZlt{}}\PYG{n+nt}{td}\PYG{p}{\PYGZgt{}} Celda \PYG{p}{\PYGZlt{}}\PYG{p}{/}\PYG{n+nt}{td}\PYG{p}{\PYGZgt{}}
				\PYG{p}{\PYGZlt{}}\PYG{p}{/}\PYG{n+nt}{tr}\PYG{p}{\PYGZgt{}}
				\PYG{p}{\PYGZlt{}}\PYG{n+nt}{tr}\PYG{p}{\PYGZgt{}}
					\PYG{p}{\PYGZlt{}}\PYG{n+nt}{td}\PYG{p}{\PYGZgt{}} Celda \PYG{p}{\PYGZlt{}}\PYG{p}{/}\PYG{n+nt}{td}\PYG{p}{\PYGZgt{}}
					\PYG{p}{\PYGZlt{}}\PYG{n+nt}{td}\PYG{p}{\PYGZgt{}} Celda \PYG{p}{\PYGZlt{}}\PYG{p}{/}\PYG{n+nt}{td}\PYG{p}{\PYGZgt{}}
					\PYG{p}{\PYGZlt{}}\PYG{n+nt}{td}\PYG{p}{\PYGZgt{}} Celda \PYG{p}{\PYGZlt{}}\PYG{p}{/}\PYG{n+nt}{td}\PYG{p}{\PYGZgt{}}
				\PYG{p}{\PYGZlt{}}\PYG{p}{/}\PYG{n+nt}{tr}\PYG{p}{\PYGZgt{}}
			\PYG{p}{\PYGZlt{}}\PYG{p}{/}\PYG{n+nt}{table}\PYG{p}{\PYGZgt{}}
			\PYG{p}{\PYGZlt{}}\PYG{p}{/}\PYG{n+nt}{td}\PYG{p}{\PYGZgt{}}
			\PYG{p}{\PYGZlt{}}\PYG{n+nt}{td}\PYG{p}{\PYGZgt{}} Celda \PYG{p}{\PYGZlt{}}\PYG{p}{/}\PYG{n+nt}{td}\PYG{p}{\PYGZgt{}}
		\PYG{p}{\PYGZlt{}}\PYG{p}{/}\PYG{n+nt}{tr}\PYG{p}{\PYGZgt{}}
	\PYG{p}{\PYGZlt{}}\PYG{p}{/}\PYG{n+nt}{table}\PYG{p}{\PYGZgt{}}
\PYG{p}{\PYGZlt{}}\PYG{p}{/}\PYG{n+nt}{body}\PYG{p}{\PYGZgt{}}
\PYG{p}{\PYGZlt{}}\PYG{p}{/}\PYG{n+nt}{html}\PYG{p}{\PYGZgt{}}
        
\end{sphinxVerbatim}


\section{Tabla 2}
\label{\detokenize{ejercicios/html/anexo_tablas:tabla-2}}
Generar la tabla siguiente

\noindent{\hspace*{\fill}\sphinxincludegraphics[scale=0.6]{{foto_02}.png}\hspace*{\fill}}

Solución:

\begin{sphinxVerbatim}[commandchars=\\\{\}]
\PYG{c+cp}{\PYGZlt{}!DOCTYPE html\PYGZgt{}}
\PYG{p}{\PYGZlt{}}\PYG{n+nt}{html}\PYG{p}{\PYGZgt{}}
\PYG{p}{\PYGZlt{}}\PYG{n+nt}{head}\PYG{p}{\PYGZgt{}}
\PYG{p}{\PYGZlt{}}\PYG{n+nt}{meta} \PYG{n+na}{charset}\PYG{o}{=}\PYG{l+s}{\PYGZsq{}utf\PYGZhy{}8\PYGZsq{}}\PYG{p}{\PYGZgt{}}
\PYG{p}{\PYGZlt{}}\PYG{n+nt}{title}\PYG{p}{\PYGZgt{}}Ejercicio\PYG{p}{\PYGZlt{}}\PYG{p}{/}\PYG{n+nt}{title}\PYG{p}{\PYGZgt{}}
\PYG{p}{\PYGZlt{}}\PYG{p}{/}\PYG{n+nt}{head}\PYG{p}{\PYGZgt{}}
\PYG{p}{\PYGZlt{}}\PYG{n+nt}{body}\PYG{p}{\PYGZgt{}}
	\PYG{p}{\PYGZlt{}}\PYG{n+nt}{table} \PYG{n+na}{border}\PYG{o}{=}\PYG{l+s}{\PYGZsq{}1\PYGZsq{}}\PYG{p}{\PYGZgt{}}
		\PYG{p}{\PYGZlt{}}\PYG{n+nt}{tr}\PYG{p}{\PYGZgt{}}
			\PYG{p}{\PYGZlt{}}\PYG{n+nt}{td}\PYG{p}{\PYGZgt{}}
			\PYG{p}{\PYGZlt{}}\PYG{n+nt}{table} \PYG{n+na}{border}\PYG{o}{=}\PYG{l+s}{\PYGZsq{}1\PYGZsq{}}\PYG{p}{\PYGZgt{}}
				\PYG{p}{\PYGZlt{}}\PYG{n+nt}{tr}\PYG{p}{\PYGZgt{}}
					\PYG{p}{\PYGZlt{}}\PYG{n+nt}{td}\PYG{p}{\PYGZgt{}} Celda \PYG{p}{\PYGZlt{}}\PYG{p}{/}\PYG{n+nt}{td}\PYG{p}{\PYGZgt{}}
					\PYG{p}{\PYGZlt{}}\PYG{n+nt}{td}\PYG{p}{\PYGZgt{}} Celda \PYG{p}{\PYGZlt{}}\PYG{p}{/}\PYG{n+nt}{td}\PYG{p}{\PYGZgt{}}
				\PYG{p}{\PYGZlt{}}\PYG{p}{/}\PYG{n+nt}{tr}\PYG{p}{\PYGZgt{}}
				\PYG{p}{\PYGZlt{}}\PYG{n+nt}{tr}\PYG{p}{\PYGZgt{}}
					\PYG{p}{\PYGZlt{}}\PYG{n+nt}{td}\PYG{p}{\PYGZgt{}} Celda \PYG{p}{\PYGZlt{}}\PYG{p}{/}\PYG{n+nt}{td}\PYG{p}{\PYGZgt{}}
					\PYG{p}{\PYGZlt{}}\PYG{n+nt}{td}\PYG{p}{\PYGZgt{}} Celda \PYG{p}{\PYGZlt{}}\PYG{p}{/}\PYG{n+nt}{td}\PYG{p}{\PYGZgt{}}
				\PYG{p}{\PYGZlt{}}\PYG{p}{/}\PYG{n+nt}{tr}\PYG{p}{\PYGZgt{}}
				\PYG{p}{\PYGZlt{}}\PYG{n+nt}{tr}\PYG{p}{\PYGZgt{}}
					\PYG{p}{\PYGZlt{}}\PYG{n+nt}{td}\PYG{p}{\PYGZgt{}} Celda \PYG{p}{\PYGZlt{}}\PYG{p}{/}\PYG{n+nt}{td}\PYG{p}{\PYGZgt{}}
					\PYG{p}{\PYGZlt{}}\PYG{n+nt}{td}\PYG{p}{\PYGZgt{}} Celda \PYG{p}{\PYGZlt{}}\PYG{p}{/}\PYG{n+nt}{td}\PYG{p}{\PYGZgt{}}
				\PYG{p}{\PYGZlt{}}\PYG{p}{/}\PYG{n+nt}{tr}\PYG{p}{\PYGZgt{}}
				\PYG{p}{\PYGZlt{}}\PYG{n+nt}{tr}\PYG{p}{\PYGZgt{}}
					\PYG{p}{\PYGZlt{}}\PYG{n+nt}{td}\PYG{p}{\PYGZgt{}} Celda \PYG{p}{\PYGZlt{}}\PYG{p}{/}\PYG{n+nt}{td}\PYG{p}{\PYGZgt{}}
					\PYG{p}{\PYGZlt{}}\PYG{n+nt}{td}\PYG{p}{\PYGZgt{}} Celda \PYG{p}{\PYGZlt{}}\PYG{p}{/}\PYG{n+nt}{td}\PYG{p}{\PYGZgt{}}
				\PYG{p}{\PYGZlt{}}\PYG{p}{/}\PYG{n+nt}{tr}\PYG{p}{\PYGZgt{}}
			\PYG{p}{\PYGZlt{}}\PYG{p}{/}\PYG{n+nt}{table}\PYG{p}{\PYGZgt{}}
			\PYG{p}{\PYGZlt{}}\PYG{p}{/}\PYG{n+nt}{td}\PYG{p}{\PYGZgt{}}
			\PYG{p}{\PYGZlt{}}\PYG{n+nt}{td}\PYG{p}{\PYGZgt{}}
			\PYG{p}{\PYGZlt{}}\PYG{n+nt}{table} \PYG{n+na}{border}\PYG{o}{=}\PYG{l+s}{\PYGZsq{}1\PYGZsq{}}\PYG{p}{\PYGZgt{}}
				\PYG{p}{\PYGZlt{}}\PYG{n+nt}{tr}\PYG{p}{\PYGZgt{}}
					\PYG{p}{\PYGZlt{}}\PYG{n+nt}{td}\PYG{p}{\PYGZgt{}} Celda \PYG{p}{\PYGZlt{}}\PYG{p}{/}\PYG{n+nt}{td}\PYG{p}{\PYGZgt{}}
					\PYG{p}{\PYGZlt{}}\PYG{n+nt}{td}\PYG{p}{\PYGZgt{}} Celda \PYG{p}{\PYGZlt{}}\PYG{p}{/}\PYG{n+nt}{td}\PYG{p}{\PYGZgt{}}
				\PYG{p}{\PYGZlt{}}\PYG{p}{/}\PYG{n+nt}{tr}\PYG{p}{\PYGZgt{}}
				\PYG{p}{\PYGZlt{}}\PYG{n+nt}{tr}\PYG{p}{\PYGZgt{}}
					\PYG{p}{\PYGZlt{}}\PYG{n+nt}{td}\PYG{p}{\PYGZgt{}} Celda \PYG{p}{\PYGZlt{}}\PYG{p}{/}\PYG{n+nt}{td}\PYG{p}{\PYGZgt{}}
					\PYG{p}{\PYGZlt{}}\PYG{n+nt}{td}\PYG{p}{\PYGZgt{}} Celda \PYG{p}{\PYGZlt{}}\PYG{p}{/}\PYG{n+nt}{td}\PYG{p}{\PYGZgt{}}
				\PYG{p}{\PYGZlt{}}\PYG{p}{/}\PYG{n+nt}{tr}\PYG{p}{\PYGZgt{}}
				\PYG{p}{\PYGZlt{}}\PYG{n+nt}{tr}\PYG{p}{\PYGZgt{}}
					\PYG{p}{\PYGZlt{}}\PYG{n+nt}{td}\PYG{p}{\PYGZgt{}} Celda \PYG{p}{\PYGZlt{}}\PYG{p}{/}\PYG{n+nt}{td}\PYG{p}{\PYGZgt{}}
					\PYG{p}{\PYGZlt{}}\PYG{n+nt}{td}\PYG{p}{\PYGZgt{}} Celda \PYG{p}{\PYGZlt{}}\PYG{p}{/}\PYG{n+nt}{td}\PYG{p}{\PYGZgt{}}
				\PYG{p}{\PYGZlt{}}\PYG{p}{/}\PYG{n+nt}{tr}\PYG{p}{\PYGZgt{}}
			\PYG{p}{\PYGZlt{}}\PYG{p}{/}\PYG{n+nt}{table}\PYG{p}{\PYGZgt{}}
			\PYG{p}{\PYGZlt{}}\PYG{p}{/}\PYG{n+nt}{td}\PYG{p}{\PYGZgt{}}
		\PYG{p}{\PYGZlt{}}\PYG{p}{/}\PYG{n+nt}{tr}\PYG{p}{\PYGZgt{}}
		\PYG{p}{\PYGZlt{}}\PYG{n+nt}{tr}\PYG{p}{\PYGZgt{}}
			\PYG{p}{\PYGZlt{}}\PYG{n+nt}{td}\PYG{p}{\PYGZgt{}} Celda \PYG{p}{\PYGZlt{}}\PYG{p}{/}\PYG{n+nt}{td}\PYG{p}{\PYGZgt{}}
			\PYG{p}{\PYGZlt{}}\PYG{n+nt}{td}\PYG{p}{\PYGZgt{}} Celda \PYG{p}{\PYGZlt{}}\PYG{p}{/}\PYG{n+nt}{td}\PYG{p}{\PYGZgt{}}
		\PYG{p}{\PYGZlt{}}\PYG{p}{/}\PYG{n+nt}{tr}\PYG{p}{\PYGZgt{}}
		\PYG{p}{\PYGZlt{}}\PYG{n+nt}{tr}\PYG{p}{\PYGZgt{}}
			\PYG{p}{\PYGZlt{}}\PYG{n+nt}{td}\PYG{p}{\PYGZgt{}} Celda \PYG{p}{\PYGZlt{}}\PYG{p}{/}\PYG{n+nt}{td}\PYG{p}{\PYGZgt{}}
			\PYG{p}{\PYGZlt{}}\PYG{n+nt}{td}\PYG{p}{\PYGZgt{}} Celda \PYG{p}{\PYGZlt{}}\PYG{p}{/}\PYG{n+nt}{td}\PYG{p}{\PYGZgt{}}
		\PYG{p}{\PYGZlt{}}\PYG{p}{/}\PYG{n+nt}{tr}\PYG{p}{\PYGZgt{}}
		\PYG{p}{\PYGZlt{}}\PYG{n+nt}{tr}\PYG{p}{\PYGZgt{}}
			\PYG{p}{\PYGZlt{}}\PYG{n+nt}{td}\PYG{p}{\PYGZgt{}} Celda \PYG{p}{\PYGZlt{}}\PYG{p}{/}\PYG{n+nt}{td}\PYG{p}{\PYGZgt{}}
			\PYG{p}{\PYGZlt{}}\PYG{n+nt}{td}\PYG{p}{\PYGZgt{}} Celda \PYG{p}{\PYGZlt{}}\PYG{p}{/}\PYG{n+nt}{td}\PYG{p}{\PYGZgt{}}
		\PYG{p}{\PYGZlt{}}\PYG{p}{/}\PYG{n+nt}{tr}\PYG{p}{\PYGZgt{}}
	\PYG{p}{\PYGZlt{}}\PYG{p}{/}\PYG{n+nt}{table}\PYG{p}{\PYGZgt{}}
\PYG{p}{\PYGZlt{}}\PYG{p}{/}\PYG{n+nt}{body}\PYG{p}{\PYGZgt{}}
\PYG{p}{\PYGZlt{}}\PYG{p}{/}\PYG{n+nt}{html}\PYG{p}{\PYGZgt{}}
        
\end{sphinxVerbatim}


\section{Tabla 3}
\label{\detokenize{ejercicios/html/anexo_tablas:tabla-3}}
Generar la tabla siguiente

\noindent{\hspace*{\fill}\sphinxincludegraphics[scale=0.6]{{foto_03}.png}\hspace*{\fill}}

Solución:

\begin{sphinxVerbatim}[commandchars=\\\{\}]
\PYG{c+cp}{\PYGZlt{}!DOCTYPE html\PYGZgt{}}
\PYG{p}{\PYGZlt{}}\PYG{n+nt}{html}\PYG{p}{\PYGZgt{}}
\PYG{p}{\PYGZlt{}}\PYG{n+nt}{head}\PYG{p}{\PYGZgt{}}
\PYG{p}{\PYGZlt{}}\PYG{n+nt}{meta} \PYG{n+na}{charset}\PYG{o}{=}\PYG{l+s}{\PYGZsq{}utf\PYGZhy{}8\PYGZsq{}}\PYG{p}{\PYGZgt{}}
\PYG{p}{\PYGZlt{}}\PYG{n+nt}{title}\PYG{p}{\PYGZgt{}}Ejercicio\PYG{p}{\PYGZlt{}}\PYG{p}{/}\PYG{n+nt}{title}\PYG{p}{\PYGZgt{}}
\PYG{p}{\PYGZlt{}}\PYG{p}{/}\PYG{n+nt}{head}\PYG{p}{\PYGZgt{}}
\PYG{p}{\PYGZlt{}}\PYG{n+nt}{body}\PYG{p}{\PYGZgt{}}
	\PYG{p}{\PYGZlt{}}\PYG{n+nt}{table} \PYG{n+na}{border}\PYG{o}{=}\PYG{l+s}{\PYGZsq{}1\PYGZsq{}}\PYG{p}{\PYGZgt{}}
		\PYG{p}{\PYGZlt{}}\PYG{n+nt}{tr}\PYG{p}{\PYGZgt{}}
			\PYG{p}{\PYGZlt{}}\PYG{n+nt}{td}\PYG{p}{\PYGZgt{}} Celda \PYG{p}{\PYGZlt{}}\PYG{p}{/}\PYG{n+nt}{td}\PYG{p}{\PYGZgt{}}
			\PYG{p}{\PYGZlt{}}\PYG{n+nt}{td}\PYG{p}{\PYGZgt{}} Celda \PYG{p}{\PYGZlt{}}\PYG{p}{/}\PYG{n+nt}{td}\PYG{p}{\PYGZgt{}}
			\PYG{p}{\PYGZlt{}}\PYG{n+nt}{td}\PYG{p}{\PYGZgt{}} Celda \PYG{p}{\PYGZlt{}}\PYG{p}{/}\PYG{n+nt}{td}\PYG{p}{\PYGZgt{}}
			\PYG{p}{\PYGZlt{}}\PYG{n+nt}{td}\PYG{p}{\PYGZgt{}} Celda \PYG{p}{\PYGZlt{}}\PYG{p}{/}\PYG{n+nt}{td}\PYG{p}{\PYGZgt{}}
		\PYG{p}{\PYGZlt{}}\PYG{p}{/}\PYG{n+nt}{tr}\PYG{p}{\PYGZgt{}}
		\PYG{p}{\PYGZlt{}}\PYG{n+nt}{tr}\PYG{p}{\PYGZgt{}}
			\PYG{p}{\PYGZlt{}}\PYG{n+nt}{td}\PYG{p}{\PYGZgt{}} Celda \PYG{p}{\PYGZlt{}}\PYG{p}{/}\PYG{n+nt}{td}\PYG{p}{\PYGZgt{}}
			\PYG{p}{\PYGZlt{}}\PYG{n+nt}{td}\PYG{p}{\PYGZgt{}} Celda \PYG{p}{\PYGZlt{}}\PYG{p}{/}\PYG{n+nt}{td}\PYG{p}{\PYGZgt{}}
			\PYG{p}{\PYGZlt{}}\PYG{n+nt}{td}\PYG{p}{\PYGZgt{}} Celda \PYG{p}{\PYGZlt{}}\PYG{p}{/}\PYG{n+nt}{td}\PYG{p}{\PYGZgt{}}
			\PYG{p}{\PYGZlt{}}\PYG{n+nt}{td}\PYG{p}{\PYGZgt{}} Celda \PYG{p}{\PYGZlt{}}\PYG{p}{/}\PYG{n+nt}{td}\PYG{p}{\PYGZgt{}}
		\PYG{p}{\PYGZlt{}}\PYG{p}{/}\PYG{n+nt}{tr}\PYG{p}{\PYGZgt{}}
		\PYG{p}{\PYGZlt{}}\PYG{n+nt}{tr}\PYG{p}{\PYGZgt{}}
			\PYG{p}{\PYGZlt{}}\PYG{n+nt}{td}\PYG{p}{\PYGZgt{}} Celda \PYG{p}{\PYGZlt{}}\PYG{p}{/}\PYG{n+nt}{td}\PYG{p}{\PYGZgt{}}
			\PYG{p}{\PYGZlt{}}\PYG{n+nt}{td}\PYG{p}{\PYGZgt{}} Celda \PYG{p}{\PYGZlt{}}\PYG{p}{/}\PYG{n+nt}{td}\PYG{p}{\PYGZgt{}}
			\PYG{p}{\PYGZlt{}}\PYG{n+nt}{td}\PYG{p}{\PYGZgt{}} Celda \PYG{p}{\PYGZlt{}}\PYG{p}{/}\PYG{n+nt}{td}\PYG{p}{\PYGZgt{}}
			\PYG{p}{\PYGZlt{}}\PYG{n+nt}{td}\PYG{p}{\PYGZgt{}} Celda \PYG{p}{\PYGZlt{}}\PYG{p}{/}\PYG{n+nt}{td}\PYG{p}{\PYGZgt{}}
		\PYG{p}{\PYGZlt{}}\PYG{p}{/}\PYG{n+nt}{tr}\PYG{p}{\PYGZgt{}}
		\PYG{p}{\PYGZlt{}}\PYG{n+nt}{tr}\PYG{p}{\PYGZgt{}}
			\PYG{p}{\PYGZlt{}}\PYG{n+nt}{td}\PYG{p}{\PYGZgt{}} Celda \PYG{p}{\PYGZlt{}}\PYG{p}{/}\PYG{n+nt}{td}\PYG{p}{\PYGZgt{}}
			\PYG{p}{\PYGZlt{}}\PYG{n+nt}{td}\PYG{p}{\PYGZgt{}}
			\PYG{p}{\PYGZlt{}}\PYG{n+nt}{table} \PYG{n+na}{border}\PYG{o}{=}\PYG{l+s}{\PYGZsq{}1\PYGZsq{}}\PYG{p}{\PYGZgt{}}
				\PYG{p}{\PYGZlt{}}\PYG{n+nt}{tr}\PYG{p}{\PYGZgt{}}
					\PYG{p}{\PYGZlt{}}\PYG{n+nt}{td}\PYG{p}{\PYGZgt{}} Celda \PYG{p}{\PYGZlt{}}\PYG{p}{/}\PYG{n+nt}{td}\PYG{p}{\PYGZgt{}}
					\PYG{p}{\PYGZlt{}}\PYG{n+nt}{td}\PYG{p}{\PYGZgt{}} Celda \PYG{p}{\PYGZlt{}}\PYG{p}{/}\PYG{n+nt}{td}\PYG{p}{\PYGZgt{}}
				\PYG{p}{\PYGZlt{}}\PYG{p}{/}\PYG{n+nt}{tr}\PYG{p}{\PYGZgt{}}
				\PYG{p}{\PYGZlt{}}\PYG{n+nt}{tr}\PYG{p}{\PYGZgt{}}
					\PYG{p}{\PYGZlt{}}\PYG{n+nt}{td}\PYG{p}{\PYGZgt{}} Celda \PYG{p}{\PYGZlt{}}\PYG{p}{/}\PYG{n+nt}{td}\PYG{p}{\PYGZgt{}}
					\PYG{p}{\PYGZlt{}}\PYG{n+nt}{td}\PYG{p}{\PYGZgt{}} Celda \PYG{p}{\PYGZlt{}}\PYG{p}{/}\PYG{n+nt}{td}\PYG{p}{\PYGZgt{}}
				\PYG{p}{\PYGZlt{}}\PYG{p}{/}\PYG{n+nt}{tr}\PYG{p}{\PYGZgt{}}
				\PYG{p}{\PYGZlt{}}\PYG{n+nt}{tr}\PYG{p}{\PYGZgt{}}
					\PYG{p}{\PYGZlt{}}\PYG{n+nt}{td}\PYG{p}{\PYGZgt{}} Celda \PYG{p}{\PYGZlt{}}\PYG{p}{/}\PYG{n+nt}{td}\PYG{p}{\PYGZgt{}}
					\PYG{p}{\PYGZlt{}}\PYG{n+nt}{td}\PYG{p}{\PYGZgt{}} Celda \PYG{p}{\PYGZlt{}}\PYG{p}{/}\PYG{n+nt}{td}\PYG{p}{\PYGZgt{}}
				\PYG{p}{\PYGZlt{}}\PYG{p}{/}\PYG{n+nt}{tr}\PYG{p}{\PYGZgt{}}
			\PYG{p}{\PYGZlt{}}\PYG{p}{/}\PYG{n+nt}{table}\PYG{p}{\PYGZgt{}}
			\PYG{p}{\PYGZlt{}}\PYG{p}{/}\PYG{n+nt}{td}\PYG{p}{\PYGZgt{}}
			\PYG{p}{\PYGZlt{}}\PYG{n+nt}{td}\PYG{p}{\PYGZgt{}} Celda \PYG{p}{\PYGZlt{}}\PYG{p}{/}\PYG{n+nt}{td}\PYG{p}{\PYGZgt{}}
			\PYG{p}{\PYGZlt{}}\PYG{n+nt}{td}\PYG{p}{\PYGZgt{}} Celda \PYG{p}{\PYGZlt{}}\PYG{p}{/}\PYG{n+nt}{td}\PYG{p}{\PYGZgt{}}
		\PYG{p}{\PYGZlt{}}\PYG{p}{/}\PYG{n+nt}{tr}\PYG{p}{\PYGZgt{}}
	\PYG{p}{\PYGZlt{}}\PYG{p}{/}\PYG{n+nt}{table}\PYG{p}{\PYGZgt{}}
\PYG{p}{\PYGZlt{}}\PYG{p}{/}\PYG{n+nt}{body}\PYG{p}{\PYGZgt{}}
\PYG{p}{\PYGZlt{}}\PYG{p}{/}\PYG{n+nt}{html}\PYG{p}{\PYGZgt{}}
        
\end{sphinxVerbatim}


\section{Tabla 4}
\label{\detokenize{ejercicios/html/anexo_tablas:tabla-4}}
Generar la tabla siguiente

\noindent{\hspace*{\fill}\sphinxincludegraphics[scale=0.6]{{foto_04}.png}\hspace*{\fill}}

Solución:

\begin{sphinxVerbatim}[commandchars=\\\{\}]
\PYG{c+cp}{\PYGZlt{}!DOCTYPE html\PYGZgt{}}
\PYG{p}{\PYGZlt{}}\PYG{n+nt}{html}\PYG{p}{\PYGZgt{}}
\PYG{p}{\PYGZlt{}}\PYG{n+nt}{head}\PYG{p}{\PYGZgt{}}
\PYG{p}{\PYGZlt{}}\PYG{n+nt}{meta} \PYG{n+na}{charset}\PYG{o}{=}\PYG{l+s}{\PYGZsq{}utf\PYGZhy{}8\PYGZsq{}}\PYG{p}{\PYGZgt{}}
\PYG{p}{\PYGZlt{}}\PYG{n+nt}{title}\PYG{p}{\PYGZgt{}}Ejercicio\PYG{p}{\PYGZlt{}}\PYG{p}{/}\PYG{n+nt}{title}\PYG{p}{\PYGZgt{}}
\PYG{p}{\PYGZlt{}}\PYG{p}{/}\PYG{n+nt}{head}\PYG{p}{\PYGZgt{}}
\PYG{p}{\PYGZlt{}}\PYG{n+nt}{body}\PYG{p}{\PYGZgt{}}
	\PYG{p}{\PYGZlt{}}\PYG{n+nt}{table} \PYG{n+na}{border}\PYG{o}{=}\PYG{l+s}{\PYGZsq{}1\PYGZsq{}}\PYG{p}{\PYGZgt{}}
		\PYG{p}{\PYGZlt{}}\PYG{n+nt}{tr}\PYG{p}{\PYGZgt{}}
			\PYG{p}{\PYGZlt{}}\PYG{n+nt}{td}\PYG{p}{\PYGZgt{}} Celda \PYG{p}{\PYGZlt{}}\PYG{p}{/}\PYG{n+nt}{td}\PYG{p}{\PYGZgt{}}
			\PYG{p}{\PYGZlt{}}\PYG{n+nt}{td}\PYG{p}{\PYGZgt{}}
			\PYG{p}{\PYGZlt{}}\PYG{n+nt}{table} \PYG{n+na}{border}\PYG{o}{=}\PYG{l+s}{\PYGZsq{}1\PYGZsq{}}\PYG{p}{\PYGZgt{}}
				\PYG{p}{\PYGZlt{}}\PYG{n+nt}{tr}\PYG{p}{\PYGZgt{}}
					\PYG{p}{\PYGZlt{}}\PYG{n+nt}{td}\PYG{p}{\PYGZgt{}} Celda \PYG{p}{\PYGZlt{}}\PYG{p}{/}\PYG{n+nt}{td}\PYG{p}{\PYGZgt{}}
					\PYG{p}{\PYGZlt{}}\PYG{n+nt}{td}\PYG{p}{\PYGZgt{}} Celda \PYG{p}{\PYGZlt{}}\PYG{p}{/}\PYG{n+nt}{td}\PYG{p}{\PYGZgt{}}
					\PYG{p}{\PYGZlt{}}\PYG{n+nt}{td}\PYG{p}{\PYGZgt{}} Celda \PYG{p}{\PYGZlt{}}\PYG{p}{/}\PYG{n+nt}{td}\PYG{p}{\PYGZgt{}}
					\PYG{p}{\PYGZlt{}}\PYG{n+nt}{td}\PYG{p}{\PYGZgt{}} Celda \PYG{p}{\PYGZlt{}}\PYG{p}{/}\PYG{n+nt}{td}\PYG{p}{\PYGZgt{}}
				\PYG{p}{\PYGZlt{}}\PYG{p}{/}\PYG{n+nt}{tr}\PYG{p}{\PYGZgt{}}
				\PYG{p}{\PYGZlt{}}\PYG{n+nt}{tr}\PYG{p}{\PYGZgt{}}
					\PYG{p}{\PYGZlt{}}\PYG{n+nt}{td}\PYG{p}{\PYGZgt{}} Celda \PYG{p}{\PYGZlt{}}\PYG{p}{/}\PYG{n+nt}{td}\PYG{p}{\PYGZgt{}}
					\PYG{p}{\PYGZlt{}}\PYG{n+nt}{td}\PYG{p}{\PYGZgt{}} Celda \PYG{p}{\PYGZlt{}}\PYG{p}{/}\PYG{n+nt}{td}\PYG{p}{\PYGZgt{}}
					\PYG{p}{\PYGZlt{}}\PYG{n+nt}{td}\PYG{p}{\PYGZgt{}} Celda \PYG{p}{\PYGZlt{}}\PYG{p}{/}\PYG{n+nt}{td}\PYG{p}{\PYGZgt{}}
					\PYG{p}{\PYGZlt{}}\PYG{n+nt}{td}\PYG{p}{\PYGZgt{}} Celda \PYG{p}{\PYGZlt{}}\PYG{p}{/}\PYG{n+nt}{td}\PYG{p}{\PYGZgt{}}
				\PYG{p}{\PYGZlt{}}\PYG{p}{/}\PYG{n+nt}{tr}\PYG{p}{\PYGZgt{}}
				\PYG{p}{\PYGZlt{}}\PYG{n+nt}{tr}\PYG{p}{\PYGZgt{}}
					\PYG{p}{\PYGZlt{}}\PYG{n+nt}{td}\PYG{p}{\PYGZgt{}} Celda \PYG{p}{\PYGZlt{}}\PYG{p}{/}\PYG{n+nt}{td}\PYG{p}{\PYGZgt{}}
					\PYG{p}{\PYGZlt{}}\PYG{n+nt}{td}\PYG{p}{\PYGZgt{}} Celda \PYG{p}{\PYGZlt{}}\PYG{p}{/}\PYG{n+nt}{td}\PYG{p}{\PYGZgt{}}
					\PYG{p}{\PYGZlt{}}\PYG{n+nt}{td}\PYG{p}{\PYGZgt{}} Celda \PYG{p}{\PYGZlt{}}\PYG{p}{/}\PYG{n+nt}{td}\PYG{p}{\PYGZgt{}}
					\PYG{p}{\PYGZlt{}}\PYG{n+nt}{td}\PYG{p}{\PYGZgt{}} Celda \PYG{p}{\PYGZlt{}}\PYG{p}{/}\PYG{n+nt}{td}\PYG{p}{\PYGZgt{}}
				\PYG{p}{\PYGZlt{}}\PYG{p}{/}\PYG{n+nt}{tr}\PYG{p}{\PYGZgt{}}
			\PYG{p}{\PYGZlt{}}\PYG{p}{/}\PYG{n+nt}{table}\PYG{p}{\PYGZgt{}}
			\PYG{p}{\PYGZlt{}}\PYG{p}{/}\PYG{n+nt}{td}\PYG{p}{\PYGZgt{}}
			\PYG{p}{\PYGZlt{}}\PYG{n+nt}{td}\PYG{p}{\PYGZgt{}} Celda \PYG{p}{\PYGZlt{}}\PYG{p}{/}\PYG{n+nt}{td}\PYG{p}{\PYGZgt{}}
			\PYG{p}{\PYGZlt{}}\PYG{n+nt}{td}\PYG{p}{\PYGZgt{}}
			\PYG{p}{\PYGZlt{}}\PYG{n+nt}{table} \PYG{n+na}{border}\PYG{o}{=}\PYG{l+s}{\PYGZsq{}1\PYGZsq{}}\PYG{p}{\PYGZgt{}}
				\PYG{p}{\PYGZlt{}}\PYG{n+nt}{tr}\PYG{p}{\PYGZgt{}}
					\PYG{p}{\PYGZlt{}}\PYG{n+nt}{td}\PYG{p}{\PYGZgt{}} Celda \PYG{p}{\PYGZlt{}}\PYG{p}{/}\PYG{n+nt}{td}\PYG{p}{\PYGZgt{}}
					\PYG{p}{\PYGZlt{}}\PYG{n+nt}{td}\PYG{p}{\PYGZgt{}} Celda \PYG{p}{\PYGZlt{}}\PYG{p}{/}\PYG{n+nt}{td}\PYG{p}{\PYGZgt{}}
					\PYG{p}{\PYGZlt{}}\PYG{n+nt}{td}\PYG{p}{\PYGZgt{}} Celda \PYG{p}{\PYGZlt{}}\PYG{p}{/}\PYG{n+nt}{td}\PYG{p}{\PYGZgt{}}
				\PYG{p}{\PYGZlt{}}\PYG{p}{/}\PYG{n+nt}{tr}\PYG{p}{\PYGZgt{}}
				\PYG{p}{\PYGZlt{}}\PYG{n+nt}{tr}\PYG{p}{\PYGZgt{}}
					\PYG{p}{\PYGZlt{}}\PYG{n+nt}{td}\PYG{p}{\PYGZgt{}} Celda \PYG{p}{\PYGZlt{}}\PYG{p}{/}\PYG{n+nt}{td}\PYG{p}{\PYGZgt{}}
					\PYG{p}{\PYGZlt{}}\PYG{n+nt}{td}\PYG{p}{\PYGZgt{}} Celda \PYG{p}{\PYGZlt{}}\PYG{p}{/}\PYG{n+nt}{td}\PYG{p}{\PYGZgt{}}
					\PYG{p}{\PYGZlt{}}\PYG{n+nt}{td}\PYG{p}{\PYGZgt{}} Celda \PYG{p}{\PYGZlt{}}\PYG{p}{/}\PYG{n+nt}{td}\PYG{p}{\PYGZgt{}}
				\PYG{p}{\PYGZlt{}}\PYG{p}{/}\PYG{n+nt}{tr}\PYG{p}{\PYGZgt{}}
				\PYG{p}{\PYGZlt{}}\PYG{n+nt}{tr}\PYG{p}{\PYGZgt{}}
					\PYG{p}{\PYGZlt{}}\PYG{n+nt}{td}\PYG{p}{\PYGZgt{}} Celda \PYG{p}{\PYGZlt{}}\PYG{p}{/}\PYG{n+nt}{td}\PYG{p}{\PYGZgt{}}
					\PYG{p}{\PYGZlt{}}\PYG{n+nt}{td}\PYG{p}{\PYGZgt{}} Celda \PYG{p}{\PYGZlt{}}\PYG{p}{/}\PYG{n+nt}{td}\PYG{p}{\PYGZgt{}}
					\PYG{p}{\PYGZlt{}}\PYG{n+nt}{td}\PYG{p}{\PYGZgt{}} Celda \PYG{p}{\PYGZlt{}}\PYG{p}{/}\PYG{n+nt}{td}\PYG{p}{\PYGZgt{}}
				\PYG{p}{\PYGZlt{}}\PYG{p}{/}\PYG{n+nt}{tr}\PYG{p}{\PYGZgt{}}
				\PYG{p}{\PYGZlt{}}\PYG{n+nt}{tr}\PYG{p}{\PYGZgt{}}
					\PYG{p}{\PYGZlt{}}\PYG{n+nt}{td}\PYG{p}{\PYGZgt{}} Celda \PYG{p}{\PYGZlt{}}\PYG{p}{/}\PYG{n+nt}{td}\PYG{p}{\PYGZgt{}}
					\PYG{p}{\PYGZlt{}}\PYG{n+nt}{td}\PYG{p}{\PYGZgt{}} Celda \PYG{p}{\PYGZlt{}}\PYG{p}{/}\PYG{n+nt}{td}\PYG{p}{\PYGZgt{}}
					\PYG{p}{\PYGZlt{}}\PYG{n+nt}{td}\PYG{p}{\PYGZgt{}} Celda \PYG{p}{\PYGZlt{}}\PYG{p}{/}\PYG{n+nt}{td}\PYG{p}{\PYGZgt{}}
				\PYG{p}{\PYGZlt{}}\PYG{p}{/}\PYG{n+nt}{tr}\PYG{p}{\PYGZgt{}}
			\PYG{p}{\PYGZlt{}}\PYG{p}{/}\PYG{n+nt}{table}\PYG{p}{\PYGZgt{}}
			\PYG{p}{\PYGZlt{}}\PYG{p}{/}\PYG{n+nt}{td}\PYG{p}{\PYGZgt{}}
		\PYG{p}{\PYGZlt{}}\PYG{p}{/}\PYG{n+nt}{tr}\PYG{p}{\PYGZgt{}}
		\PYG{p}{\PYGZlt{}}\PYG{n+nt}{tr}\PYG{p}{\PYGZgt{}}
			\PYG{p}{\PYGZlt{}}\PYG{n+nt}{td}\PYG{p}{\PYGZgt{}}
			\PYG{p}{\PYGZlt{}}\PYG{n+nt}{table} \PYG{n+na}{border}\PYG{o}{=}\PYG{l+s}{\PYGZsq{}1\PYGZsq{}}\PYG{p}{\PYGZgt{}}
				\PYG{p}{\PYGZlt{}}\PYG{n+nt}{tr}\PYG{p}{\PYGZgt{}}
					\PYG{p}{\PYGZlt{}}\PYG{n+nt}{td}\PYG{p}{\PYGZgt{}} Celda \PYG{p}{\PYGZlt{}}\PYG{p}{/}\PYG{n+nt}{td}\PYG{p}{\PYGZgt{}}
					\PYG{p}{\PYGZlt{}}\PYG{n+nt}{td}\PYG{p}{\PYGZgt{}} Celda \PYG{p}{\PYGZlt{}}\PYG{p}{/}\PYG{n+nt}{td}\PYG{p}{\PYGZgt{}}
					\PYG{p}{\PYGZlt{}}\PYG{n+nt}{td}\PYG{p}{\PYGZgt{}} Celda \PYG{p}{\PYGZlt{}}\PYG{p}{/}\PYG{n+nt}{td}\PYG{p}{\PYGZgt{}}
					\PYG{p}{\PYGZlt{}}\PYG{n+nt}{td}\PYG{p}{\PYGZgt{}} Celda \PYG{p}{\PYGZlt{}}\PYG{p}{/}\PYG{n+nt}{td}\PYG{p}{\PYGZgt{}}
				\PYG{p}{\PYGZlt{}}\PYG{p}{/}\PYG{n+nt}{tr}\PYG{p}{\PYGZgt{}}
				\PYG{p}{\PYGZlt{}}\PYG{n+nt}{tr}\PYG{p}{\PYGZgt{}}
					\PYG{p}{\PYGZlt{}}\PYG{n+nt}{td}\PYG{p}{\PYGZgt{}} Celda \PYG{p}{\PYGZlt{}}\PYG{p}{/}\PYG{n+nt}{td}\PYG{p}{\PYGZgt{}}
					\PYG{p}{\PYGZlt{}}\PYG{n+nt}{td}\PYG{p}{\PYGZgt{}} Celda \PYG{p}{\PYGZlt{}}\PYG{p}{/}\PYG{n+nt}{td}\PYG{p}{\PYGZgt{}}
					\PYG{p}{\PYGZlt{}}\PYG{n+nt}{td}\PYG{p}{\PYGZgt{}} Celda \PYG{p}{\PYGZlt{}}\PYG{p}{/}\PYG{n+nt}{td}\PYG{p}{\PYGZgt{}}
					\PYG{p}{\PYGZlt{}}\PYG{n+nt}{td}\PYG{p}{\PYGZgt{}} Celda \PYG{p}{\PYGZlt{}}\PYG{p}{/}\PYG{n+nt}{td}\PYG{p}{\PYGZgt{}}
				\PYG{p}{\PYGZlt{}}\PYG{p}{/}\PYG{n+nt}{tr}\PYG{p}{\PYGZgt{}}
				\PYG{p}{\PYGZlt{}}\PYG{n+nt}{tr}\PYG{p}{\PYGZgt{}}
					\PYG{p}{\PYGZlt{}}\PYG{n+nt}{td}\PYG{p}{\PYGZgt{}} Celda \PYG{p}{\PYGZlt{}}\PYG{p}{/}\PYG{n+nt}{td}\PYG{p}{\PYGZgt{}}
					\PYG{p}{\PYGZlt{}}\PYG{n+nt}{td}\PYG{p}{\PYGZgt{}} Celda \PYG{p}{\PYGZlt{}}\PYG{p}{/}\PYG{n+nt}{td}\PYG{p}{\PYGZgt{}}
					\PYG{p}{\PYGZlt{}}\PYG{n+nt}{td}\PYG{p}{\PYGZgt{}} Celda \PYG{p}{\PYGZlt{}}\PYG{p}{/}\PYG{n+nt}{td}\PYG{p}{\PYGZgt{}}
					\PYG{p}{\PYGZlt{}}\PYG{n+nt}{td}\PYG{p}{\PYGZgt{}} Celda \PYG{p}{\PYGZlt{}}\PYG{p}{/}\PYG{n+nt}{td}\PYG{p}{\PYGZgt{}}
				\PYG{p}{\PYGZlt{}}\PYG{p}{/}\PYG{n+nt}{tr}\PYG{p}{\PYGZgt{}}
			\PYG{p}{\PYGZlt{}}\PYG{p}{/}\PYG{n+nt}{table}\PYG{p}{\PYGZgt{}}
			\PYG{p}{\PYGZlt{}}\PYG{p}{/}\PYG{n+nt}{td}\PYG{p}{\PYGZgt{}}
			\PYG{p}{\PYGZlt{}}\PYG{n+nt}{td}\PYG{p}{\PYGZgt{}} Celda \PYG{p}{\PYGZlt{}}\PYG{p}{/}\PYG{n+nt}{td}\PYG{p}{\PYGZgt{}}
			\PYG{p}{\PYGZlt{}}\PYG{n+nt}{td}\PYG{p}{\PYGZgt{}} Celda \PYG{p}{\PYGZlt{}}\PYG{p}{/}\PYG{n+nt}{td}\PYG{p}{\PYGZgt{}}
			\PYG{p}{\PYGZlt{}}\PYG{n+nt}{td}\PYG{p}{\PYGZgt{}} Celda \PYG{p}{\PYGZlt{}}\PYG{p}{/}\PYG{n+nt}{td}\PYG{p}{\PYGZgt{}}
		\PYG{p}{\PYGZlt{}}\PYG{p}{/}\PYG{n+nt}{tr}\PYG{p}{\PYGZgt{}}
		\PYG{p}{\PYGZlt{}}\PYG{n+nt}{tr}\PYG{p}{\PYGZgt{}}
			\PYG{p}{\PYGZlt{}}\PYG{n+nt}{td}\PYG{p}{\PYGZgt{}}
			\PYG{p}{\PYGZlt{}}\PYG{n+nt}{table} \PYG{n+na}{border}\PYG{o}{=}\PYG{l+s}{\PYGZsq{}1\PYGZsq{}}\PYG{p}{\PYGZgt{}}
				\PYG{p}{\PYGZlt{}}\PYG{n+nt}{tr}\PYG{p}{\PYGZgt{}}
					\PYG{p}{\PYGZlt{}}\PYG{n+nt}{td}\PYG{p}{\PYGZgt{}} Celda \PYG{p}{\PYGZlt{}}\PYG{p}{/}\PYG{n+nt}{td}\PYG{p}{\PYGZgt{}}
					\PYG{p}{\PYGZlt{}}\PYG{n+nt}{td}\PYG{p}{\PYGZgt{}} Celda \PYG{p}{\PYGZlt{}}\PYG{p}{/}\PYG{n+nt}{td}\PYG{p}{\PYGZgt{}}
				\PYG{p}{\PYGZlt{}}\PYG{p}{/}\PYG{n+nt}{tr}\PYG{p}{\PYGZgt{}}
				\PYG{p}{\PYGZlt{}}\PYG{n+nt}{tr}\PYG{p}{\PYGZgt{}}
					\PYG{p}{\PYGZlt{}}\PYG{n+nt}{td}\PYG{p}{\PYGZgt{}} Celda \PYG{p}{\PYGZlt{}}\PYG{p}{/}\PYG{n+nt}{td}\PYG{p}{\PYGZgt{}}
					\PYG{p}{\PYGZlt{}}\PYG{n+nt}{td}\PYG{p}{\PYGZgt{}} Celda \PYG{p}{\PYGZlt{}}\PYG{p}{/}\PYG{n+nt}{td}\PYG{p}{\PYGZgt{}}
				\PYG{p}{\PYGZlt{}}\PYG{p}{/}\PYG{n+nt}{tr}\PYG{p}{\PYGZgt{}}
			\PYG{p}{\PYGZlt{}}\PYG{p}{/}\PYG{n+nt}{table}\PYG{p}{\PYGZgt{}}
			\PYG{p}{\PYGZlt{}}\PYG{p}{/}\PYG{n+nt}{td}\PYG{p}{\PYGZgt{}}
			\PYG{p}{\PYGZlt{}}\PYG{n+nt}{td}\PYG{p}{\PYGZgt{}} Celda \PYG{p}{\PYGZlt{}}\PYG{p}{/}\PYG{n+nt}{td}\PYG{p}{\PYGZgt{}}
			\PYG{p}{\PYGZlt{}}\PYG{n+nt}{td}\PYG{p}{\PYGZgt{}} Celda \PYG{p}{\PYGZlt{}}\PYG{p}{/}\PYG{n+nt}{td}\PYG{p}{\PYGZgt{}}
			\PYG{p}{\PYGZlt{}}\PYG{n+nt}{td}\PYG{p}{\PYGZgt{}} Celda \PYG{p}{\PYGZlt{}}\PYG{p}{/}\PYG{n+nt}{td}\PYG{p}{\PYGZgt{}}
		\PYG{p}{\PYGZlt{}}\PYG{p}{/}\PYG{n+nt}{tr}\PYG{p}{\PYGZgt{}}
		\PYG{p}{\PYGZlt{}}\PYG{n+nt}{tr}\PYG{p}{\PYGZgt{}}
			\PYG{p}{\PYGZlt{}}\PYG{n+nt}{td}\PYG{p}{\PYGZgt{}} Celda \PYG{p}{\PYGZlt{}}\PYG{p}{/}\PYG{n+nt}{td}\PYG{p}{\PYGZgt{}}
			\PYG{p}{\PYGZlt{}}\PYG{n+nt}{td}\PYG{p}{\PYGZgt{}} Celda \PYG{p}{\PYGZlt{}}\PYG{p}{/}\PYG{n+nt}{td}\PYG{p}{\PYGZgt{}}
			\PYG{p}{\PYGZlt{}}\PYG{n+nt}{td}\PYG{p}{\PYGZgt{}} Celda \PYG{p}{\PYGZlt{}}\PYG{p}{/}\PYG{n+nt}{td}\PYG{p}{\PYGZgt{}}
			\PYG{p}{\PYGZlt{}}\PYG{n+nt}{td}\PYG{p}{\PYGZgt{}} Celda \PYG{p}{\PYGZlt{}}\PYG{p}{/}\PYG{n+nt}{td}\PYG{p}{\PYGZgt{}}
		\PYG{p}{\PYGZlt{}}\PYG{p}{/}\PYG{n+nt}{tr}\PYG{p}{\PYGZgt{}}
	\PYG{p}{\PYGZlt{}}\PYG{p}{/}\PYG{n+nt}{table}\PYG{p}{\PYGZgt{}}
\PYG{p}{\PYGZlt{}}\PYG{p}{/}\PYG{n+nt}{body}\PYG{p}{\PYGZgt{}}
\PYG{p}{\PYGZlt{}}\PYG{p}{/}\PYG{n+nt}{html}\PYG{p}{\PYGZgt{}}
        
\end{sphinxVerbatim}


\section{Tabla 5}
\label{\detokenize{ejercicios/html/anexo_tablas:tabla-5}}
Generar la tabla siguiente

\noindent{\hspace*{\fill}\sphinxincludegraphics[scale=0.6]{{foto_05}.png}\hspace*{\fill}}

Solución:

\begin{sphinxVerbatim}[commandchars=\\\{\}]
\PYG{c+cp}{\PYGZlt{}!DOCTYPE html\PYGZgt{}}
\PYG{p}{\PYGZlt{}}\PYG{n+nt}{html}\PYG{p}{\PYGZgt{}}
\PYG{p}{\PYGZlt{}}\PYG{n+nt}{head}\PYG{p}{\PYGZgt{}}
\PYG{p}{\PYGZlt{}}\PYG{n+nt}{meta} \PYG{n+na}{charset}\PYG{o}{=}\PYG{l+s}{\PYGZsq{}utf\PYGZhy{}8\PYGZsq{}}\PYG{p}{\PYGZgt{}}
\PYG{p}{\PYGZlt{}}\PYG{n+nt}{title}\PYG{p}{\PYGZgt{}}Ejercicio\PYG{p}{\PYGZlt{}}\PYG{p}{/}\PYG{n+nt}{title}\PYG{p}{\PYGZgt{}}
\PYG{p}{\PYGZlt{}}\PYG{p}{/}\PYG{n+nt}{head}\PYG{p}{\PYGZgt{}}
\PYG{p}{\PYGZlt{}}\PYG{n+nt}{body}\PYG{p}{\PYGZgt{}}
	\PYG{p}{\PYGZlt{}}\PYG{n+nt}{table} \PYG{n+na}{border}\PYG{o}{=}\PYG{l+s}{\PYGZsq{}1\PYGZsq{}}\PYG{p}{\PYGZgt{}}
		\PYG{p}{\PYGZlt{}}\PYG{n+nt}{tr}\PYG{p}{\PYGZgt{}}
			\PYG{p}{\PYGZlt{}}\PYG{n+nt}{td}\PYG{p}{\PYGZgt{}}
			\PYG{p}{\PYGZlt{}}\PYG{n+nt}{table} \PYG{n+na}{border}\PYG{o}{=}\PYG{l+s}{\PYGZsq{}1\PYGZsq{}}\PYG{p}{\PYGZgt{}}
				\PYG{p}{\PYGZlt{}}\PYG{n+nt}{tr}\PYG{p}{\PYGZgt{}}
					\PYG{p}{\PYGZlt{}}\PYG{n+nt}{td}\PYG{p}{\PYGZgt{}} Celda \PYG{p}{\PYGZlt{}}\PYG{p}{/}\PYG{n+nt}{td}\PYG{p}{\PYGZgt{}}
					\PYG{p}{\PYGZlt{}}\PYG{n+nt}{td}\PYG{p}{\PYGZgt{}} Celda \PYG{p}{\PYGZlt{}}\PYG{p}{/}\PYG{n+nt}{td}\PYG{p}{\PYGZgt{}}
					\PYG{p}{\PYGZlt{}}\PYG{n+nt}{td}\PYG{p}{\PYGZgt{}} Celda \PYG{p}{\PYGZlt{}}\PYG{p}{/}\PYG{n+nt}{td}\PYG{p}{\PYGZgt{}}
					\PYG{p}{\PYGZlt{}}\PYG{n+nt}{td}\PYG{p}{\PYGZgt{}} Celda \PYG{p}{\PYGZlt{}}\PYG{p}{/}\PYG{n+nt}{td}\PYG{p}{\PYGZgt{}}
				\PYG{p}{\PYGZlt{}}\PYG{p}{/}\PYG{n+nt}{tr}\PYG{p}{\PYGZgt{}}
				\PYG{p}{\PYGZlt{}}\PYG{n+nt}{tr}\PYG{p}{\PYGZgt{}}
					\PYG{p}{\PYGZlt{}}\PYG{n+nt}{td}\PYG{p}{\PYGZgt{}} Celda \PYG{p}{\PYGZlt{}}\PYG{p}{/}\PYG{n+nt}{td}\PYG{p}{\PYGZgt{}}
					\PYG{p}{\PYGZlt{}}\PYG{n+nt}{td}\PYG{p}{\PYGZgt{}} Celda \PYG{p}{\PYGZlt{}}\PYG{p}{/}\PYG{n+nt}{td}\PYG{p}{\PYGZgt{}}
					\PYG{p}{\PYGZlt{}}\PYG{n+nt}{td}\PYG{p}{\PYGZgt{}} Celda \PYG{p}{\PYGZlt{}}\PYG{p}{/}\PYG{n+nt}{td}\PYG{p}{\PYGZgt{}}
					\PYG{p}{\PYGZlt{}}\PYG{n+nt}{td}\PYG{p}{\PYGZgt{}} Celda \PYG{p}{\PYGZlt{}}\PYG{p}{/}\PYG{n+nt}{td}\PYG{p}{\PYGZgt{}}
				\PYG{p}{\PYGZlt{}}\PYG{p}{/}\PYG{n+nt}{tr}\PYG{p}{\PYGZgt{}}
				\PYG{p}{\PYGZlt{}}\PYG{n+nt}{tr}\PYG{p}{\PYGZgt{}}
					\PYG{p}{\PYGZlt{}}\PYG{n+nt}{td}\PYG{p}{\PYGZgt{}} Celda \PYG{p}{\PYGZlt{}}\PYG{p}{/}\PYG{n+nt}{td}\PYG{p}{\PYGZgt{}}
					\PYG{p}{\PYGZlt{}}\PYG{n+nt}{td}\PYG{p}{\PYGZgt{}} Celda \PYG{p}{\PYGZlt{}}\PYG{p}{/}\PYG{n+nt}{td}\PYG{p}{\PYGZgt{}}
					\PYG{p}{\PYGZlt{}}\PYG{n+nt}{td}\PYG{p}{\PYGZgt{}} Celda \PYG{p}{\PYGZlt{}}\PYG{p}{/}\PYG{n+nt}{td}\PYG{p}{\PYGZgt{}}
					\PYG{p}{\PYGZlt{}}\PYG{n+nt}{td}\PYG{p}{\PYGZgt{}} Celda \PYG{p}{\PYGZlt{}}\PYG{p}{/}\PYG{n+nt}{td}\PYG{p}{\PYGZgt{}}
				\PYG{p}{\PYGZlt{}}\PYG{p}{/}\PYG{n+nt}{tr}\PYG{p}{\PYGZgt{}}
			\PYG{p}{\PYGZlt{}}\PYG{p}{/}\PYG{n+nt}{table}\PYG{p}{\PYGZgt{}}
			\PYG{p}{\PYGZlt{}}\PYG{p}{/}\PYG{n+nt}{td}\PYG{p}{\PYGZgt{}}
			\PYG{p}{\PYGZlt{}}\PYG{n+nt}{td}\PYG{p}{\PYGZgt{}} Celda \PYG{p}{\PYGZlt{}}\PYG{p}{/}\PYG{n+nt}{td}\PYG{p}{\PYGZgt{}}
		\PYG{p}{\PYGZlt{}}\PYG{p}{/}\PYG{n+nt}{tr}\PYG{p}{\PYGZgt{}}
		\PYG{p}{\PYGZlt{}}\PYG{n+nt}{tr}\PYG{p}{\PYGZgt{}}
			\PYG{p}{\PYGZlt{}}\PYG{n+nt}{td}\PYG{p}{\PYGZgt{}} Celda \PYG{p}{\PYGZlt{}}\PYG{p}{/}\PYG{n+nt}{td}\PYG{p}{\PYGZgt{}}
			\PYG{p}{\PYGZlt{}}\PYG{n+nt}{td}\PYG{p}{\PYGZgt{}}
			\PYG{p}{\PYGZlt{}}\PYG{n+nt}{table} \PYG{n+na}{border}\PYG{o}{=}\PYG{l+s}{\PYGZsq{}1\PYGZsq{}}\PYG{p}{\PYGZgt{}}
				\PYG{p}{\PYGZlt{}}\PYG{n+nt}{tr}\PYG{p}{\PYGZgt{}}
					\PYG{p}{\PYGZlt{}}\PYG{n+nt}{td}\PYG{p}{\PYGZgt{}} Celda \PYG{p}{\PYGZlt{}}\PYG{p}{/}\PYG{n+nt}{td}\PYG{p}{\PYGZgt{}}
					\PYG{p}{\PYGZlt{}}\PYG{n+nt}{td}\PYG{p}{\PYGZgt{}} Celda \PYG{p}{\PYGZlt{}}\PYG{p}{/}\PYG{n+nt}{td}\PYG{p}{\PYGZgt{}}
					\PYG{p}{\PYGZlt{}}\PYG{n+nt}{td}\PYG{p}{\PYGZgt{}} Celda \PYG{p}{\PYGZlt{}}\PYG{p}{/}\PYG{n+nt}{td}\PYG{p}{\PYGZgt{}}
					\PYG{p}{\PYGZlt{}}\PYG{n+nt}{td}\PYG{p}{\PYGZgt{}} Celda \PYG{p}{\PYGZlt{}}\PYG{p}{/}\PYG{n+nt}{td}\PYG{p}{\PYGZgt{}}
				\PYG{p}{\PYGZlt{}}\PYG{p}{/}\PYG{n+nt}{tr}\PYG{p}{\PYGZgt{}}
				\PYG{p}{\PYGZlt{}}\PYG{n+nt}{tr}\PYG{p}{\PYGZgt{}}
					\PYG{p}{\PYGZlt{}}\PYG{n+nt}{td}\PYG{p}{\PYGZgt{}} Celda \PYG{p}{\PYGZlt{}}\PYG{p}{/}\PYG{n+nt}{td}\PYG{p}{\PYGZgt{}}
					\PYG{p}{\PYGZlt{}}\PYG{n+nt}{td}\PYG{p}{\PYGZgt{}} Celda \PYG{p}{\PYGZlt{}}\PYG{p}{/}\PYG{n+nt}{td}\PYG{p}{\PYGZgt{}}
					\PYG{p}{\PYGZlt{}}\PYG{n+nt}{td}\PYG{p}{\PYGZgt{}} Celda \PYG{p}{\PYGZlt{}}\PYG{p}{/}\PYG{n+nt}{td}\PYG{p}{\PYGZgt{}}
					\PYG{p}{\PYGZlt{}}\PYG{n+nt}{td}\PYG{p}{\PYGZgt{}} Celda \PYG{p}{\PYGZlt{}}\PYG{p}{/}\PYG{n+nt}{td}\PYG{p}{\PYGZgt{}}
				\PYG{p}{\PYGZlt{}}\PYG{p}{/}\PYG{n+nt}{tr}\PYG{p}{\PYGZgt{}}
				\PYG{p}{\PYGZlt{}}\PYG{n+nt}{tr}\PYG{p}{\PYGZgt{}}
					\PYG{p}{\PYGZlt{}}\PYG{n+nt}{td}\PYG{p}{\PYGZgt{}} Celda \PYG{p}{\PYGZlt{}}\PYG{p}{/}\PYG{n+nt}{td}\PYG{p}{\PYGZgt{}}
					\PYG{p}{\PYGZlt{}}\PYG{n+nt}{td}\PYG{p}{\PYGZgt{}} Celda \PYG{p}{\PYGZlt{}}\PYG{p}{/}\PYG{n+nt}{td}\PYG{p}{\PYGZgt{}}
					\PYG{p}{\PYGZlt{}}\PYG{n+nt}{td}\PYG{p}{\PYGZgt{}} Celda \PYG{p}{\PYGZlt{}}\PYG{p}{/}\PYG{n+nt}{td}\PYG{p}{\PYGZgt{}}
					\PYG{p}{\PYGZlt{}}\PYG{n+nt}{td}\PYG{p}{\PYGZgt{}} Celda \PYG{p}{\PYGZlt{}}\PYG{p}{/}\PYG{n+nt}{td}\PYG{p}{\PYGZgt{}}
				\PYG{p}{\PYGZlt{}}\PYG{p}{/}\PYG{n+nt}{tr}\PYG{p}{\PYGZgt{}}
			\PYG{p}{\PYGZlt{}}\PYG{p}{/}\PYG{n+nt}{table}\PYG{p}{\PYGZgt{}}
			\PYG{p}{\PYGZlt{}}\PYG{p}{/}\PYG{n+nt}{td}\PYG{p}{\PYGZgt{}}
		\PYG{p}{\PYGZlt{}}\PYG{p}{/}\PYG{n+nt}{tr}\PYG{p}{\PYGZgt{}}
		\PYG{p}{\PYGZlt{}}\PYG{n+nt}{tr}\PYG{p}{\PYGZgt{}}
			\PYG{p}{\PYGZlt{}}\PYG{n+nt}{td}\PYG{p}{\PYGZgt{}} Celda \PYG{p}{\PYGZlt{}}\PYG{p}{/}\PYG{n+nt}{td}\PYG{p}{\PYGZgt{}}
			\PYG{p}{\PYGZlt{}}\PYG{n+nt}{td}\PYG{p}{\PYGZgt{}} Celda \PYG{p}{\PYGZlt{}}\PYG{p}{/}\PYG{n+nt}{td}\PYG{p}{\PYGZgt{}}
		\PYG{p}{\PYGZlt{}}\PYG{p}{/}\PYG{n+nt}{tr}\PYG{p}{\PYGZgt{}}
		\PYG{p}{\PYGZlt{}}\PYG{n+nt}{tr}\PYG{p}{\PYGZgt{}}
			\PYG{p}{\PYGZlt{}}\PYG{n+nt}{td}\PYG{p}{\PYGZgt{}}
			\PYG{p}{\PYGZlt{}}\PYG{n+nt}{table} \PYG{n+na}{border}\PYG{o}{=}\PYG{l+s}{\PYGZsq{}1\PYGZsq{}}\PYG{p}{\PYGZgt{}}
				\PYG{p}{\PYGZlt{}}\PYG{n+nt}{tr}\PYG{p}{\PYGZgt{}}
					\PYG{p}{\PYGZlt{}}\PYG{n+nt}{td}\PYG{p}{\PYGZgt{}} Celda \PYG{p}{\PYGZlt{}}\PYG{p}{/}\PYG{n+nt}{td}\PYG{p}{\PYGZgt{}}
					\PYG{p}{\PYGZlt{}}\PYG{n+nt}{td}\PYG{p}{\PYGZgt{}} Celda \PYG{p}{\PYGZlt{}}\PYG{p}{/}\PYG{n+nt}{td}\PYG{p}{\PYGZgt{}}
				\PYG{p}{\PYGZlt{}}\PYG{p}{/}\PYG{n+nt}{tr}\PYG{p}{\PYGZgt{}}
				\PYG{p}{\PYGZlt{}}\PYG{n+nt}{tr}\PYG{p}{\PYGZgt{}}
					\PYG{p}{\PYGZlt{}}\PYG{n+nt}{td}\PYG{p}{\PYGZgt{}} Celda \PYG{p}{\PYGZlt{}}\PYG{p}{/}\PYG{n+nt}{td}\PYG{p}{\PYGZgt{}}
					\PYG{p}{\PYGZlt{}}\PYG{n+nt}{td}\PYG{p}{\PYGZgt{}} Celda \PYG{p}{\PYGZlt{}}\PYG{p}{/}\PYG{n+nt}{td}\PYG{p}{\PYGZgt{}}
				\PYG{p}{\PYGZlt{}}\PYG{p}{/}\PYG{n+nt}{tr}\PYG{p}{\PYGZgt{}}
				\PYG{p}{\PYGZlt{}}\PYG{n+nt}{tr}\PYG{p}{\PYGZgt{}}
					\PYG{p}{\PYGZlt{}}\PYG{n+nt}{td}\PYG{p}{\PYGZgt{}} Celda \PYG{p}{\PYGZlt{}}\PYG{p}{/}\PYG{n+nt}{td}\PYG{p}{\PYGZgt{}}
					\PYG{p}{\PYGZlt{}}\PYG{n+nt}{td}\PYG{p}{\PYGZgt{}} Celda \PYG{p}{\PYGZlt{}}\PYG{p}{/}\PYG{n+nt}{td}\PYG{p}{\PYGZgt{}}
				\PYG{p}{\PYGZlt{}}\PYG{p}{/}\PYG{n+nt}{tr}\PYG{p}{\PYGZgt{}}
			\PYG{p}{\PYGZlt{}}\PYG{p}{/}\PYG{n+nt}{table}\PYG{p}{\PYGZgt{}}
			\PYG{p}{\PYGZlt{}}\PYG{p}{/}\PYG{n+nt}{td}\PYG{p}{\PYGZgt{}}
			\PYG{p}{\PYGZlt{}}\PYG{n+nt}{td}\PYG{p}{\PYGZgt{}} Celda \PYG{p}{\PYGZlt{}}\PYG{p}{/}\PYG{n+nt}{td}\PYG{p}{\PYGZgt{}}
		\PYG{p}{\PYGZlt{}}\PYG{p}{/}\PYG{n+nt}{tr}\PYG{p}{\PYGZgt{}}
	\PYG{p}{\PYGZlt{}}\PYG{p}{/}\PYG{n+nt}{table}\PYG{p}{\PYGZgt{}}
\PYG{p}{\PYGZlt{}}\PYG{p}{/}\PYG{n+nt}{body}\PYG{p}{\PYGZgt{}}
\PYG{p}{\PYGZlt{}}\PYG{p}{/}\PYG{n+nt}{html}\PYG{p}{\PYGZgt{}}
        
\end{sphinxVerbatim}


\section{Tabla 6}
\label{\detokenize{ejercicios/html/anexo_tablas:tabla-6}}
Generar la tabla siguiente

\noindent{\hspace*{\fill}\sphinxincludegraphics[scale=0.6]{{foto_06}.png}\hspace*{\fill}}

Solución:

\begin{sphinxVerbatim}[commandchars=\\\{\}]
\PYG{c+cp}{\PYGZlt{}!DOCTYPE html\PYGZgt{}}
\PYG{p}{\PYGZlt{}}\PYG{n+nt}{html}\PYG{p}{\PYGZgt{}}
\PYG{p}{\PYGZlt{}}\PYG{n+nt}{head}\PYG{p}{\PYGZgt{}}
\PYG{p}{\PYGZlt{}}\PYG{n+nt}{meta} \PYG{n+na}{charset}\PYG{o}{=}\PYG{l+s}{\PYGZsq{}utf\PYGZhy{}8\PYGZsq{}}\PYG{p}{\PYGZgt{}}
\PYG{p}{\PYGZlt{}}\PYG{n+nt}{title}\PYG{p}{\PYGZgt{}}Ejercicio\PYG{p}{\PYGZlt{}}\PYG{p}{/}\PYG{n+nt}{title}\PYG{p}{\PYGZgt{}}
\PYG{p}{\PYGZlt{}}\PYG{p}{/}\PYG{n+nt}{head}\PYG{p}{\PYGZgt{}}
\PYG{p}{\PYGZlt{}}\PYG{n+nt}{body}\PYG{p}{\PYGZgt{}}
	\PYG{p}{\PYGZlt{}}\PYG{n+nt}{table} \PYG{n+na}{border}\PYG{o}{=}\PYG{l+s}{\PYGZsq{}1\PYGZsq{}}\PYG{p}{\PYGZgt{}}
		\PYG{p}{\PYGZlt{}}\PYG{n+nt}{tr}\PYG{p}{\PYGZgt{}}
			\PYG{p}{\PYGZlt{}}\PYG{n+nt}{td}\PYG{p}{\PYGZgt{}} Celda \PYG{p}{\PYGZlt{}}\PYG{p}{/}\PYG{n+nt}{td}\PYG{p}{\PYGZgt{}}
			\PYG{p}{\PYGZlt{}}\PYG{n+nt}{td}\PYG{p}{\PYGZgt{}}
			\PYG{p}{\PYGZlt{}}\PYG{n+nt}{table} \PYG{n+na}{border}\PYG{o}{=}\PYG{l+s}{\PYGZsq{}1\PYGZsq{}}\PYG{p}{\PYGZgt{}}
				\PYG{p}{\PYGZlt{}}\PYG{n+nt}{tr}\PYG{p}{\PYGZgt{}}
					\PYG{p}{\PYGZlt{}}\PYG{n+nt}{td}\PYG{p}{\PYGZgt{}} Celda \PYG{p}{\PYGZlt{}}\PYG{p}{/}\PYG{n+nt}{td}\PYG{p}{\PYGZgt{}}
					\PYG{p}{\PYGZlt{}}\PYG{n+nt}{td}\PYG{p}{\PYGZgt{}} Celda \PYG{p}{\PYGZlt{}}\PYG{p}{/}\PYG{n+nt}{td}\PYG{p}{\PYGZgt{}}
					\PYG{p}{\PYGZlt{}}\PYG{n+nt}{td}\PYG{p}{\PYGZgt{}} Celda \PYG{p}{\PYGZlt{}}\PYG{p}{/}\PYG{n+nt}{td}\PYG{p}{\PYGZgt{}}
				\PYG{p}{\PYGZlt{}}\PYG{p}{/}\PYG{n+nt}{tr}\PYG{p}{\PYGZgt{}}
				\PYG{p}{\PYGZlt{}}\PYG{n+nt}{tr}\PYG{p}{\PYGZgt{}}
					\PYG{p}{\PYGZlt{}}\PYG{n+nt}{td}\PYG{p}{\PYGZgt{}} Celda \PYG{p}{\PYGZlt{}}\PYG{p}{/}\PYG{n+nt}{td}\PYG{p}{\PYGZgt{}}
					\PYG{p}{\PYGZlt{}}\PYG{n+nt}{td}\PYG{p}{\PYGZgt{}} Celda \PYG{p}{\PYGZlt{}}\PYG{p}{/}\PYG{n+nt}{td}\PYG{p}{\PYGZgt{}}
					\PYG{p}{\PYGZlt{}}\PYG{n+nt}{td}\PYG{p}{\PYGZgt{}} Celda \PYG{p}{\PYGZlt{}}\PYG{p}{/}\PYG{n+nt}{td}\PYG{p}{\PYGZgt{}}
				\PYG{p}{\PYGZlt{}}\PYG{p}{/}\PYG{n+nt}{tr}\PYG{p}{\PYGZgt{}}
			\PYG{p}{\PYGZlt{}}\PYG{p}{/}\PYG{n+nt}{table}\PYG{p}{\PYGZgt{}}
			\PYG{p}{\PYGZlt{}}\PYG{p}{/}\PYG{n+nt}{td}\PYG{p}{\PYGZgt{}}
			\PYG{p}{\PYGZlt{}}\PYG{n+nt}{td}\PYG{p}{\PYGZgt{}} Celda \PYG{p}{\PYGZlt{}}\PYG{p}{/}\PYG{n+nt}{td}\PYG{p}{\PYGZgt{}}
		\PYG{p}{\PYGZlt{}}\PYG{p}{/}\PYG{n+nt}{tr}\PYG{p}{\PYGZgt{}}
		\PYG{p}{\PYGZlt{}}\PYG{n+nt}{tr}\PYG{p}{\PYGZgt{}}
			\PYG{p}{\PYGZlt{}}\PYG{n+nt}{td}\PYG{p}{\PYGZgt{}} Celda \PYG{p}{\PYGZlt{}}\PYG{p}{/}\PYG{n+nt}{td}\PYG{p}{\PYGZgt{}}
			\PYG{p}{\PYGZlt{}}\PYG{n+nt}{td}\PYG{p}{\PYGZgt{}} Celda \PYG{p}{\PYGZlt{}}\PYG{p}{/}\PYG{n+nt}{td}\PYG{p}{\PYGZgt{}}
			\PYG{p}{\PYGZlt{}}\PYG{n+nt}{td}\PYG{p}{\PYGZgt{}} Celda \PYG{p}{\PYGZlt{}}\PYG{p}{/}\PYG{n+nt}{td}\PYG{p}{\PYGZgt{}}
		\PYG{p}{\PYGZlt{}}\PYG{p}{/}\PYG{n+nt}{tr}\PYG{p}{\PYGZgt{}}
		\PYG{p}{\PYGZlt{}}\PYG{n+nt}{tr}\PYG{p}{\PYGZgt{}}
			\PYG{p}{\PYGZlt{}}\PYG{n+nt}{td}\PYG{p}{\PYGZgt{}} Celda \PYG{p}{\PYGZlt{}}\PYG{p}{/}\PYG{n+nt}{td}\PYG{p}{\PYGZgt{}}
			\PYG{p}{\PYGZlt{}}\PYG{n+nt}{td}\PYG{p}{\PYGZgt{}}
			\PYG{p}{\PYGZlt{}}\PYG{n+nt}{table} \PYG{n+na}{border}\PYG{o}{=}\PYG{l+s}{\PYGZsq{}1\PYGZsq{}}\PYG{p}{\PYGZgt{}}
				\PYG{p}{\PYGZlt{}}\PYG{n+nt}{tr}\PYG{p}{\PYGZgt{}}
					\PYG{p}{\PYGZlt{}}\PYG{n+nt}{td}\PYG{p}{\PYGZgt{}} Celda \PYG{p}{\PYGZlt{}}\PYG{p}{/}\PYG{n+nt}{td}\PYG{p}{\PYGZgt{}}
					\PYG{p}{\PYGZlt{}}\PYG{n+nt}{td}\PYG{p}{\PYGZgt{}} Celda \PYG{p}{\PYGZlt{}}\PYG{p}{/}\PYG{n+nt}{td}\PYG{p}{\PYGZgt{}}
					\PYG{p}{\PYGZlt{}}\PYG{n+nt}{td}\PYG{p}{\PYGZgt{}} Celda \PYG{p}{\PYGZlt{}}\PYG{p}{/}\PYG{n+nt}{td}\PYG{p}{\PYGZgt{}}
					\PYG{p}{\PYGZlt{}}\PYG{n+nt}{td}\PYG{p}{\PYGZgt{}} Celda \PYG{p}{\PYGZlt{}}\PYG{p}{/}\PYG{n+nt}{td}\PYG{p}{\PYGZgt{}}
				\PYG{p}{\PYGZlt{}}\PYG{p}{/}\PYG{n+nt}{tr}\PYG{p}{\PYGZgt{}}
				\PYG{p}{\PYGZlt{}}\PYG{n+nt}{tr}\PYG{p}{\PYGZgt{}}
					\PYG{p}{\PYGZlt{}}\PYG{n+nt}{td}\PYG{p}{\PYGZgt{}} Celda \PYG{p}{\PYGZlt{}}\PYG{p}{/}\PYG{n+nt}{td}\PYG{p}{\PYGZgt{}}
					\PYG{p}{\PYGZlt{}}\PYG{n+nt}{td}\PYG{p}{\PYGZgt{}} Celda \PYG{p}{\PYGZlt{}}\PYG{p}{/}\PYG{n+nt}{td}\PYG{p}{\PYGZgt{}}
					\PYG{p}{\PYGZlt{}}\PYG{n+nt}{td}\PYG{p}{\PYGZgt{}} Celda \PYG{p}{\PYGZlt{}}\PYG{p}{/}\PYG{n+nt}{td}\PYG{p}{\PYGZgt{}}
					\PYG{p}{\PYGZlt{}}\PYG{n+nt}{td}\PYG{p}{\PYGZgt{}} Celda \PYG{p}{\PYGZlt{}}\PYG{p}{/}\PYG{n+nt}{td}\PYG{p}{\PYGZgt{}}
				\PYG{p}{\PYGZlt{}}\PYG{p}{/}\PYG{n+nt}{tr}\PYG{p}{\PYGZgt{}}
			\PYG{p}{\PYGZlt{}}\PYG{p}{/}\PYG{n+nt}{table}\PYG{p}{\PYGZgt{}}
			\PYG{p}{\PYGZlt{}}\PYG{p}{/}\PYG{n+nt}{td}\PYG{p}{\PYGZgt{}}
			\PYG{p}{\PYGZlt{}}\PYG{n+nt}{td}\PYG{p}{\PYGZgt{}} Celda \PYG{p}{\PYGZlt{}}\PYG{p}{/}\PYG{n+nt}{td}\PYG{p}{\PYGZgt{}}
		\PYG{p}{\PYGZlt{}}\PYG{p}{/}\PYG{n+nt}{tr}\PYG{p}{\PYGZgt{}}
	\PYG{p}{\PYGZlt{}}\PYG{p}{/}\PYG{n+nt}{table}\PYG{p}{\PYGZgt{}}
\PYG{p}{\PYGZlt{}}\PYG{p}{/}\PYG{n+nt}{body}\PYG{p}{\PYGZgt{}}
\PYG{p}{\PYGZlt{}}\PYG{p}{/}\PYG{n+nt}{html}\PYG{p}{\PYGZgt{}}
        
\end{sphinxVerbatim}


\section{Tabla 7}
\label{\detokenize{ejercicios/html/anexo_tablas:tabla-7}}
Generar la tabla siguiente

\noindent{\hspace*{\fill}\sphinxincludegraphics[scale=0.6]{{foto_07}.png}\hspace*{\fill}}

Solución:

\begin{sphinxVerbatim}[commandchars=\\\{\}]
\PYG{c+cp}{\PYGZlt{}!DOCTYPE html\PYGZgt{}}
\PYG{p}{\PYGZlt{}}\PYG{n+nt}{html}\PYG{p}{\PYGZgt{}}
\PYG{p}{\PYGZlt{}}\PYG{n+nt}{head}\PYG{p}{\PYGZgt{}}
\PYG{p}{\PYGZlt{}}\PYG{n+nt}{meta} \PYG{n+na}{charset}\PYG{o}{=}\PYG{l+s}{\PYGZsq{}utf\PYGZhy{}8\PYGZsq{}}\PYG{p}{\PYGZgt{}}
\PYG{p}{\PYGZlt{}}\PYG{n+nt}{title}\PYG{p}{\PYGZgt{}}Ejercicio\PYG{p}{\PYGZlt{}}\PYG{p}{/}\PYG{n+nt}{title}\PYG{p}{\PYGZgt{}}
\PYG{p}{\PYGZlt{}}\PYG{p}{/}\PYG{n+nt}{head}\PYG{p}{\PYGZgt{}}
\PYG{p}{\PYGZlt{}}\PYG{n+nt}{body}\PYG{p}{\PYGZgt{}}
	\PYG{p}{\PYGZlt{}}\PYG{n+nt}{table} \PYG{n+na}{border}\PYG{o}{=}\PYG{l+s}{\PYGZsq{}1\PYGZsq{}}\PYG{p}{\PYGZgt{}}
		\PYG{p}{\PYGZlt{}}\PYG{n+nt}{tr}\PYG{p}{\PYGZgt{}}
			\PYG{p}{\PYGZlt{}}\PYG{n+nt}{td}\PYG{p}{\PYGZgt{}} Celda \PYG{p}{\PYGZlt{}}\PYG{p}{/}\PYG{n+nt}{td}\PYG{p}{\PYGZgt{}}
			\PYG{p}{\PYGZlt{}}\PYG{n+nt}{td}\PYG{p}{\PYGZgt{}} Celda \PYG{p}{\PYGZlt{}}\PYG{p}{/}\PYG{n+nt}{td}\PYG{p}{\PYGZgt{}}
			\PYG{p}{\PYGZlt{}}\PYG{n+nt}{td}\PYG{p}{\PYGZgt{}} Celda \PYG{p}{\PYGZlt{}}\PYG{p}{/}\PYG{n+nt}{td}\PYG{p}{\PYGZgt{}}
		\PYG{p}{\PYGZlt{}}\PYG{p}{/}\PYG{n+nt}{tr}\PYG{p}{\PYGZgt{}}
		\PYG{p}{\PYGZlt{}}\PYG{n+nt}{tr}\PYG{p}{\PYGZgt{}}
			\PYG{p}{\PYGZlt{}}\PYG{n+nt}{td}\PYG{p}{\PYGZgt{}} Celda \PYG{p}{\PYGZlt{}}\PYG{p}{/}\PYG{n+nt}{td}\PYG{p}{\PYGZgt{}}
			\PYG{p}{\PYGZlt{}}\PYG{n+nt}{td}\PYG{p}{\PYGZgt{}} Celda \PYG{p}{\PYGZlt{}}\PYG{p}{/}\PYG{n+nt}{td}\PYG{p}{\PYGZgt{}}
			\PYG{p}{\PYGZlt{}}\PYG{n+nt}{td}\PYG{p}{\PYGZgt{}} Celda \PYG{p}{\PYGZlt{}}\PYG{p}{/}\PYG{n+nt}{td}\PYG{p}{\PYGZgt{}}
		\PYG{p}{\PYGZlt{}}\PYG{p}{/}\PYG{n+nt}{tr}\PYG{p}{\PYGZgt{}}
		\PYG{p}{\PYGZlt{}}\PYG{n+nt}{tr}\PYG{p}{\PYGZgt{}}
			\PYG{p}{\PYGZlt{}}\PYG{n+nt}{td}\PYG{p}{\PYGZgt{}}
			\PYG{p}{\PYGZlt{}}\PYG{n+nt}{table} \PYG{n+na}{border}\PYG{o}{=}\PYG{l+s}{\PYGZsq{}1\PYGZsq{}}\PYG{p}{\PYGZgt{}}
				\PYG{p}{\PYGZlt{}}\PYG{n+nt}{tr}\PYG{p}{\PYGZgt{}}
					\PYG{p}{\PYGZlt{}}\PYG{n+nt}{td}\PYG{p}{\PYGZgt{}} Celda \PYG{p}{\PYGZlt{}}\PYG{p}{/}\PYG{n+nt}{td}\PYG{p}{\PYGZgt{}}
					\PYG{p}{\PYGZlt{}}\PYG{n+nt}{td}\PYG{p}{\PYGZgt{}} Celda \PYG{p}{\PYGZlt{}}\PYG{p}{/}\PYG{n+nt}{td}\PYG{p}{\PYGZgt{}}
					\PYG{p}{\PYGZlt{}}\PYG{n+nt}{td}\PYG{p}{\PYGZgt{}} Celda \PYG{p}{\PYGZlt{}}\PYG{p}{/}\PYG{n+nt}{td}\PYG{p}{\PYGZgt{}}
					\PYG{p}{\PYGZlt{}}\PYG{n+nt}{td}\PYG{p}{\PYGZgt{}} Celda \PYG{p}{\PYGZlt{}}\PYG{p}{/}\PYG{n+nt}{td}\PYG{p}{\PYGZgt{}}
				\PYG{p}{\PYGZlt{}}\PYG{p}{/}\PYG{n+nt}{tr}\PYG{p}{\PYGZgt{}}
				\PYG{p}{\PYGZlt{}}\PYG{n+nt}{tr}\PYG{p}{\PYGZgt{}}
					\PYG{p}{\PYGZlt{}}\PYG{n+nt}{td}\PYG{p}{\PYGZgt{}} Celda \PYG{p}{\PYGZlt{}}\PYG{p}{/}\PYG{n+nt}{td}\PYG{p}{\PYGZgt{}}
					\PYG{p}{\PYGZlt{}}\PYG{n+nt}{td}\PYG{p}{\PYGZgt{}} Celda \PYG{p}{\PYGZlt{}}\PYG{p}{/}\PYG{n+nt}{td}\PYG{p}{\PYGZgt{}}
					\PYG{p}{\PYGZlt{}}\PYG{n+nt}{td}\PYG{p}{\PYGZgt{}} Celda \PYG{p}{\PYGZlt{}}\PYG{p}{/}\PYG{n+nt}{td}\PYG{p}{\PYGZgt{}}
					\PYG{p}{\PYGZlt{}}\PYG{n+nt}{td}\PYG{p}{\PYGZgt{}} Celda \PYG{p}{\PYGZlt{}}\PYG{p}{/}\PYG{n+nt}{td}\PYG{p}{\PYGZgt{}}
				\PYG{p}{\PYGZlt{}}\PYG{p}{/}\PYG{n+nt}{tr}\PYG{p}{\PYGZgt{}}
			\PYG{p}{\PYGZlt{}}\PYG{p}{/}\PYG{n+nt}{table}\PYG{p}{\PYGZgt{}}
			\PYG{p}{\PYGZlt{}}\PYG{p}{/}\PYG{n+nt}{td}\PYG{p}{\PYGZgt{}}
			\PYG{p}{\PYGZlt{}}\PYG{n+nt}{td}\PYG{p}{\PYGZgt{}} Celda \PYG{p}{\PYGZlt{}}\PYG{p}{/}\PYG{n+nt}{td}\PYG{p}{\PYGZgt{}}
			\PYG{p}{\PYGZlt{}}\PYG{n+nt}{td}\PYG{p}{\PYGZgt{}} Celda \PYG{p}{\PYGZlt{}}\PYG{p}{/}\PYG{n+nt}{td}\PYG{p}{\PYGZgt{}}
		\PYG{p}{\PYGZlt{}}\PYG{p}{/}\PYG{n+nt}{tr}\PYG{p}{\PYGZgt{}}
		\PYG{p}{\PYGZlt{}}\PYG{n+nt}{tr}\PYG{p}{\PYGZgt{}}
			\PYG{p}{\PYGZlt{}}\PYG{n+nt}{td}\PYG{p}{\PYGZgt{}} Celda \PYG{p}{\PYGZlt{}}\PYG{p}{/}\PYG{n+nt}{td}\PYG{p}{\PYGZgt{}}
			\PYG{p}{\PYGZlt{}}\PYG{n+nt}{td}\PYG{p}{\PYGZgt{}}
			\PYG{p}{\PYGZlt{}}\PYG{n+nt}{table} \PYG{n+na}{border}\PYG{o}{=}\PYG{l+s}{\PYGZsq{}1\PYGZsq{}}\PYG{p}{\PYGZgt{}}
				\PYG{p}{\PYGZlt{}}\PYG{n+nt}{tr}\PYG{p}{\PYGZgt{}}
					\PYG{p}{\PYGZlt{}}\PYG{n+nt}{td}\PYG{p}{\PYGZgt{}} Celda \PYG{p}{\PYGZlt{}}\PYG{p}{/}\PYG{n+nt}{td}\PYG{p}{\PYGZgt{}}
					\PYG{p}{\PYGZlt{}}\PYG{n+nt}{td}\PYG{p}{\PYGZgt{}} Celda \PYG{p}{\PYGZlt{}}\PYG{p}{/}\PYG{n+nt}{td}\PYG{p}{\PYGZgt{}}
				\PYG{p}{\PYGZlt{}}\PYG{p}{/}\PYG{n+nt}{tr}\PYG{p}{\PYGZgt{}}
				\PYG{p}{\PYGZlt{}}\PYG{n+nt}{tr}\PYG{p}{\PYGZgt{}}
					\PYG{p}{\PYGZlt{}}\PYG{n+nt}{td}\PYG{p}{\PYGZgt{}} Celda \PYG{p}{\PYGZlt{}}\PYG{p}{/}\PYG{n+nt}{td}\PYG{p}{\PYGZgt{}}
					\PYG{p}{\PYGZlt{}}\PYG{n+nt}{td}\PYG{p}{\PYGZgt{}} Celda \PYG{p}{\PYGZlt{}}\PYG{p}{/}\PYG{n+nt}{td}\PYG{p}{\PYGZgt{}}
				\PYG{p}{\PYGZlt{}}\PYG{p}{/}\PYG{n+nt}{tr}\PYG{p}{\PYGZgt{}}
			\PYG{p}{\PYGZlt{}}\PYG{p}{/}\PYG{n+nt}{table}\PYG{p}{\PYGZgt{}}
			\PYG{p}{\PYGZlt{}}\PYG{p}{/}\PYG{n+nt}{td}\PYG{p}{\PYGZgt{}}
			\PYG{p}{\PYGZlt{}}\PYG{n+nt}{td}\PYG{p}{\PYGZgt{}} Celda \PYG{p}{\PYGZlt{}}\PYG{p}{/}\PYG{n+nt}{td}\PYG{p}{\PYGZgt{}}
		\PYG{p}{\PYGZlt{}}\PYG{p}{/}\PYG{n+nt}{tr}\PYG{p}{\PYGZgt{}}
	\PYG{p}{\PYGZlt{}}\PYG{p}{/}\PYG{n+nt}{table}\PYG{p}{\PYGZgt{}}
\PYG{p}{\PYGZlt{}}\PYG{p}{/}\PYG{n+nt}{body}\PYG{p}{\PYGZgt{}}
\PYG{p}{\PYGZlt{}}\PYG{p}{/}\PYG{n+nt}{html}\PYG{p}{\PYGZgt{}}
        
\end{sphinxVerbatim}


\section{Tabla 8}
\label{\detokenize{ejercicios/html/anexo_tablas:tabla-8}}
Generar la tabla siguiente

\noindent{\hspace*{\fill}\sphinxincludegraphics[scale=0.6]{{foto_08}.png}\hspace*{\fill}}

Solución:

\begin{sphinxVerbatim}[commandchars=\\\{\}]
\PYG{c+cp}{\PYGZlt{}!DOCTYPE html\PYGZgt{}}
\PYG{p}{\PYGZlt{}}\PYG{n+nt}{html}\PYG{p}{\PYGZgt{}}
\PYG{p}{\PYGZlt{}}\PYG{n+nt}{head}\PYG{p}{\PYGZgt{}}
\PYG{p}{\PYGZlt{}}\PYG{n+nt}{meta} \PYG{n+na}{charset}\PYG{o}{=}\PYG{l+s}{\PYGZsq{}utf\PYGZhy{}8\PYGZsq{}}\PYG{p}{\PYGZgt{}}
\PYG{p}{\PYGZlt{}}\PYG{n+nt}{title}\PYG{p}{\PYGZgt{}}Ejercicio\PYG{p}{\PYGZlt{}}\PYG{p}{/}\PYG{n+nt}{title}\PYG{p}{\PYGZgt{}}
\PYG{p}{\PYGZlt{}}\PYG{p}{/}\PYG{n+nt}{head}\PYG{p}{\PYGZgt{}}
\PYG{p}{\PYGZlt{}}\PYG{n+nt}{body}\PYG{p}{\PYGZgt{}}
	\PYG{p}{\PYGZlt{}}\PYG{n+nt}{table} \PYG{n+na}{border}\PYG{o}{=}\PYG{l+s}{\PYGZsq{}1\PYGZsq{}}\PYG{p}{\PYGZgt{}}
		\PYG{p}{\PYGZlt{}}\PYG{n+nt}{tr}\PYG{p}{\PYGZgt{}}
			\PYG{p}{\PYGZlt{}}\PYG{n+nt}{td}\PYG{p}{\PYGZgt{}}
			\PYG{p}{\PYGZlt{}}\PYG{n+nt}{table} \PYG{n+na}{border}\PYG{o}{=}\PYG{l+s}{\PYGZsq{}1\PYGZsq{}}\PYG{p}{\PYGZgt{}}
				\PYG{p}{\PYGZlt{}}\PYG{n+nt}{tr}\PYG{p}{\PYGZgt{}}
					\PYG{p}{\PYGZlt{}}\PYG{n+nt}{td}\PYG{p}{\PYGZgt{}} Celda \PYG{p}{\PYGZlt{}}\PYG{p}{/}\PYG{n+nt}{td}\PYG{p}{\PYGZgt{}}
					\PYG{p}{\PYGZlt{}}\PYG{n+nt}{td}\PYG{p}{\PYGZgt{}} Celda \PYG{p}{\PYGZlt{}}\PYG{p}{/}\PYG{n+nt}{td}\PYG{p}{\PYGZgt{}}
					\PYG{p}{\PYGZlt{}}\PYG{n+nt}{td}\PYG{p}{\PYGZgt{}} Celda \PYG{p}{\PYGZlt{}}\PYG{p}{/}\PYG{n+nt}{td}\PYG{p}{\PYGZgt{}}
					\PYG{p}{\PYGZlt{}}\PYG{n+nt}{td}\PYG{p}{\PYGZgt{}} Celda \PYG{p}{\PYGZlt{}}\PYG{p}{/}\PYG{n+nt}{td}\PYG{p}{\PYGZgt{}}
				\PYG{p}{\PYGZlt{}}\PYG{p}{/}\PYG{n+nt}{tr}\PYG{p}{\PYGZgt{}}
				\PYG{p}{\PYGZlt{}}\PYG{n+nt}{tr}\PYG{p}{\PYGZgt{}}
					\PYG{p}{\PYGZlt{}}\PYG{n+nt}{td}\PYG{p}{\PYGZgt{}} Celda \PYG{p}{\PYGZlt{}}\PYG{p}{/}\PYG{n+nt}{td}\PYG{p}{\PYGZgt{}}
					\PYG{p}{\PYGZlt{}}\PYG{n+nt}{td}\PYG{p}{\PYGZgt{}} Celda \PYG{p}{\PYGZlt{}}\PYG{p}{/}\PYG{n+nt}{td}\PYG{p}{\PYGZgt{}}
					\PYG{p}{\PYGZlt{}}\PYG{n+nt}{td}\PYG{p}{\PYGZgt{}} Celda \PYG{p}{\PYGZlt{}}\PYG{p}{/}\PYG{n+nt}{td}\PYG{p}{\PYGZgt{}}
					\PYG{p}{\PYGZlt{}}\PYG{n+nt}{td}\PYG{p}{\PYGZgt{}} Celda \PYG{p}{\PYGZlt{}}\PYG{p}{/}\PYG{n+nt}{td}\PYG{p}{\PYGZgt{}}
				\PYG{p}{\PYGZlt{}}\PYG{p}{/}\PYG{n+nt}{tr}\PYG{p}{\PYGZgt{}}
			\PYG{p}{\PYGZlt{}}\PYG{p}{/}\PYG{n+nt}{table}\PYG{p}{\PYGZgt{}}
			\PYG{p}{\PYGZlt{}}\PYG{p}{/}\PYG{n+nt}{td}\PYG{p}{\PYGZgt{}}
			\PYG{p}{\PYGZlt{}}\PYG{n+nt}{td}\PYG{p}{\PYGZgt{}} Celda \PYG{p}{\PYGZlt{}}\PYG{p}{/}\PYG{n+nt}{td}\PYG{p}{\PYGZgt{}}
			\PYG{p}{\PYGZlt{}}\PYG{n+nt}{td}\PYG{p}{\PYGZgt{}} Celda \PYG{p}{\PYGZlt{}}\PYG{p}{/}\PYG{n+nt}{td}\PYG{p}{\PYGZgt{}}
			\PYG{p}{\PYGZlt{}}\PYG{n+nt}{td}\PYG{p}{\PYGZgt{}}
			\PYG{p}{\PYGZlt{}}\PYG{n+nt}{table} \PYG{n+na}{border}\PYG{o}{=}\PYG{l+s}{\PYGZsq{}1\PYGZsq{}}\PYG{p}{\PYGZgt{}}
				\PYG{p}{\PYGZlt{}}\PYG{n+nt}{tr}\PYG{p}{\PYGZgt{}}
					\PYG{p}{\PYGZlt{}}\PYG{n+nt}{td}\PYG{p}{\PYGZgt{}} Celda \PYG{p}{\PYGZlt{}}\PYG{p}{/}\PYG{n+nt}{td}\PYG{p}{\PYGZgt{}}
					\PYG{p}{\PYGZlt{}}\PYG{n+nt}{td}\PYG{p}{\PYGZgt{}} Celda \PYG{p}{\PYGZlt{}}\PYG{p}{/}\PYG{n+nt}{td}\PYG{p}{\PYGZgt{}}
					\PYG{p}{\PYGZlt{}}\PYG{n+nt}{td}\PYG{p}{\PYGZgt{}} Celda \PYG{p}{\PYGZlt{}}\PYG{p}{/}\PYG{n+nt}{td}\PYG{p}{\PYGZgt{}}
				\PYG{p}{\PYGZlt{}}\PYG{p}{/}\PYG{n+nt}{tr}\PYG{p}{\PYGZgt{}}
				\PYG{p}{\PYGZlt{}}\PYG{n+nt}{tr}\PYG{p}{\PYGZgt{}}
					\PYG{p}{\PYGZlt{}}\PYG{n+nt}{td}\PYG{p}{\PYGZgt{}} Celda \PYG{p}{\PYGZlt{}}\PYG{p}{/}\PYG{n+nt}{td}\PYG{p}{\PYGZgt{}}
					\PYG{p}{\PYGZlt{}}\PYG{n+nt}{td}\PYG{p}{\PYGZgt{}} Celda \PYG{p}{\PYGZlt{}}\PYG{p}{/}\PYG{n+nt}{td}\PYG{p}{\PYGZgt{}}
					\PYG{p}{\PYGZlt{}}\PYG{n+nt}{td}\PYG{p}{\PYGZgt{}} Celda \PYG{p}{\PYGZlt{}}\PYG{p}{/}\PYG{n+nt}{td}\PYG{p}{\PYGZgt{}}
				\PYG{p}{\PYGZlt{}}\PYG{p}{/}\PYG{n+nt}{tr}\PYG{p}{\PYGZgt{}}
				\PYG{p}{\PYGZlt{}}\PYG{n+nt}{tr}\PYG{p}{\PYGZgt{}}
					\PYG{p}{\PYGZlt{}}\PYG{n+nt}{td}\PYG{p}{\PYGZgt{}} Celda \PYG{p}{\PYGZlt{}}\PYG{p}{/}\PYG{n+nt}{td}\PYG{p}{\PYGZgt{}}
					\PYG{p}{\PYGZlt{}}\PYG{n+nt}{td}\PYG{p}{\PYGZgt{}} Celda \PYG{p}{\PYGZlt{}}\PYG{p}{/}\PYG{n+nt}{td}\PYG{p}{\PYGZgt{}}
					\PYG{p}{\PYGZlt{}}\PYG{n+nt}{td}\PYG{p}{\PYGZgt{}} Celda \PYG{p}{\PYGZlt{}}\PYG{p}{/}\PYG{n+nt}{td}\PYG{p}{\PYGZgt{}}
				\PYG{p}{\PYGZlt{}}\PYG{p}{/}\PYG{n+nt}{tr}\PYG{p}{\PYGZgt{}}
				\PYG{p}{\PYGZlt{}}\PYG{n+nt}{tr}\PYG{p}{\PYGZgt{}}
					\PYG{p}{\PYGZlt{}}\PYG{n+nt}{td}\PYG{p}{\PYGZgt{}} Celda \PYG{p}{\PYGZlt{}}\PYG{p}{/}\PYG{n+nt}{td}\PYG{p}{\PYGZgt{}}
					\PYG{p}{\PYGZlt{}}\PYG{n+nt}{td}\PYG{p}{\PYGZgt{}} Celda \PYG{p}{\PYGZlt{}}\PYG{p}{/}\PYG{n+nt}{td}\PYG{p}{\PYGZgt{}}
					\PYG{p}{\PYGZlt{}}\PYG{n+nt}{td}\PYG{p}{\PYGZgt{}} Celda \PYG{p}{\PYGZlt{}}\PYG{p}{/}\PYG{n+nt}{td}\PYG{p}{\PYGZgt{}}
				\PYG{p}{\PYGZlt{}}\PYG{p}{/}\PYG{n+nt}{tr}\PYG{p}{\PYGZgt{}}
			\PYG{p}{\PYGZlt{}}\PYG{p}{/}\PYG{n+nt}{table}\PYG{p}{\PYGZgt{}}
			\PYG{p}{\PYGZlt{}}\PYG{p}{/}\PYG{n+nt}{td}\PYG{p}{\PYGZgt{}}
		\PYG{p}{\PYGZlt{}}\PYG{p}{/}\PYG{n+nt}{tr}\PYG{p}{\PYGZgt{}}
		\PYG{p}{\PYGZlt{}}\PYG{n+nt}{tr}\PYG{p}{\PYGZgt{}}
			\PYG{p}{\PYGZlt{}}\PYG{n+nt}{td}\PYG{p}{\PYGZgt{}} Celda \PYG{p}{\PYGZlt{}}\PYG{p}{/}\PYG{n+nt}{td}\PYG{p}{\PYGZgt{}}
			\PYG{p}{\PYGZlt{}}\PYG{n+nt}{td}\PYG{p}{\PYGZgt{}} Celda \PYG{p}{\PYGZlt{}}\PYG{p}{/}\PYG{n+nt}{td}\PYG{p}{\PYGZgt{}}
			\PYG{p}{\PYGZlt{}}\PYG{n+nt}{td}\PYG{p}{\PYGZgt{}} Celda \PYG{p}{\PYGZlt{}}\PYG{p}{/}\PYG{n+nt}{td}\PYG{p}{\PYGZgt{}}
			\PYG{p}{\PYGZlt{}}\PYG{n+nt}{td}\PYG{p}{\PYGZgt{}} Celda \PYG{p}{\PYGZlt{}}\PYG{p}{/}\PYG{n+nt}{td}\PYG{p}{\PYGZgt{}}
		\PYG{p}{\PYGZlt{}}\PYG{p}{/}\PYG{n+nt}{tr}\PYG{p}{\PYGZgt{}}
		\PYG{p}{\PYGZlt{}}\PYG{n+nt}{tr}\PYG{p}{\PYGZgt{}}
			\PYG{p}{\PYGZlt{}}\PYG{n+nt}{td}\PYG{p}{\PYGZgt{}} Celda \PYG{p}{\PYGZlt{}}\PYG{p}{/}\PYG{n+nt}{td}\PYG{p}{\PYGZgt{}}
			\PYG{p}{\PYGZlt{}}\PYG{n+nt}{td}\PYG{p}{\PYGZgt{}} Celda \PYG{p}{\PYGZlt{}}\PYG{p}{/}\PYG{n+nt}{td}\PYG{p}{\PYGZgt{}}
			\PYG{p}{\PYGZlt{}}\PYG{n+nt}{td}\PYG{p}{\PYGZgt{}} Celda \PYG{p}{\PYGZlt{}}\PYG{p}{/}\PYG{n+nt}{td}\PYG{p}{\PYGZgt{}}
			\PYG{p}{\PYGZlt{}}\PYG{n+nt}{td}\PYG{p}{\PYGZgt{}} Celda \PYG{p}{\PYGZlt{}}\PYG{p}{/}\PYG{n+nt}{td}\PYG{p}{\PYGZgt{}}
		\PYG{p}{\PYGZlt{}}\PYG{p}{/}\PYG{n+nt}{tr}\PYG{p}{\PYGZgt{}}
		\PYG{p}{\PYGZlt{}}\PYG{n+nt}{tr}\PYG{p}{\PYGZgt{}}
			\PYG{p}{\PYGZlt{}}\PYG{n+nt}{td}\PYG{p}{\PYGZgt{}} Celda \PYG{p}{\PYGZlt{}}\PYG{p}{/}\PYG{n+nt}{td}\PYG{p}{\PYGZgt{}}
			\PYG{p}{\PYGZlt{}}\PYG{n+nt}{td}\PYG{p}{\PYGZgt{}} Celda \PYG{p}{\PYGZlt{}}\PYG{p}{/}\PYG{n+nt}{td}\PYG{p}{\PYGZgt{}}
			\PYG{p}{\PYGZlt{}}\PYG{n+nt}{td}\PYG{p}{\PYGZgt{}} Celda \PYG{p}{\PYGZlt{}}\PYG{p}{/}\PYG{n+nt}{td}\PYG{p}{\PYGZgt{}}
			\PYG{p}{\PYGZlt{}}\PYG{n+nt}{td}\PYG{p}{\PYGZgt{}} Celda \PYG{p}{\PYGZlt{}}\PYG{p}{/}\PYG{n+nt}{td}\PYG{p}{\PYGZgt{}}
		\PYG{p}{\PYGZlt{}}\PYG{p}{/}\PYG{n+nt}{tr}\PYG{p}{\PYGZgt{}}
	\PYG{p}{\PYGZlt{}}\PYG{p}{/}\PYG{n+nt}{table}\PYG{p}{\PYGZgt{}}
\PYG{p}{\PYGZlt{}}\PYG{p}{/}\PYG{n+nt}{body}\PYG{p}{\PYGZgt{}}
\PYG{p}{\PYGZlt{}}\PYG{p}{/}\PYG{n+nt}{html}\PYG{p}{\PYGZgt{}}
        
\end{sphinxVerbatim}


\section{Tabla 9}
\label{\detokenize{ejercicios/html/anexo_tablas:tabla-9}}
Generar la tabla siguiente

\noindent{\hspace*{\fill}\sphinxincludegraphics[scale=0.6]{{foto_09}.png}\hspace*{\fill}}

Solución:

\begin{sphinxVerbatim}[commandchars=\\\{\}]
\PYG{c+cp}{\PYGZlt{}!DOCTYPE html\PYGZgt{}}
\PYG{p}{\PYGZlt{}}\PYG{n+nt}{html}\PYG{p}{\PYGZgt{}}
\PYG{p}{\PYGZlt{}}\PYG{n+nt}{head}\PYG{p}{\PYGZgt{}}
\PYG{p}{\PYGZlt{}}\PYG{n+nt}{meta} \PYG{n+na}{charset}\PYG{o}{=}\PYG{l+s}{\PYGZsq{}utf\PYGZhy{}8\PYGZsq{}}\PYG{p}{\PYGZgt{}}
\PYG{p}{\PYGZlt{}}\PYG{n+nt}{title}\PYG{p}{\PYGZgt{}}Ejercicio\PYG{p}{\PYGZlt{}}\PYG{p}{/}\PYG{n+nt}{title}\PYG{p}{\PYGZgt{}}
\PYG{p}{\PYGZlt{}}\PYG{p}{/}\PYG{n+nt}{head}\PYG{p}{\PYGZgt{}}
\PYG{p}{\PYGZlt{}}\PYG{n+nt}{body}\PYG{p}{\PYGZgt{}}
	\PYG{p}{\PYGZlt{}}\PYG{n+nt}{table} \PYG{n+na}{border}\PYG{o}{=}\PYG{l+s}{\PYGZsq{}1\PYGZsq{}}\PYG{p}{\PYGZgt{}}
		\PYG{p}{\PYGZlt{}}\PYG{n+nt}{tr}\PYG{p}{\PYGZgt{}}
			\PYG{p}{\PYGZlt{}}\PYG{n+nt}{td}\PYG{p}{\PYGZgt{}} Celda \PYG{p}{\PYGZlt{}}\PYG{p}{/}\PYG{n+nt}{td}\PYG{p}{\PYGZgt{}}
			\PYG{p}{\PYGZlt{}}\PYG{n+nt}{td}\PYG{p}{\PYGZgt{}}
			\PYG{p}{\PYGZlt{}}\PYG{n+nt}{table} \PYG{n+na}{border}\PYG{o}{=}\PYG{l+s}{\PYGZsq{}1\PYGZsq{}}\PYG{p}{\PYGZgt{}}
				\PYG{p}{\PYGZlt{}}\PYG{n+nt}{tr}\PYG{p}{\PYGZgt{}}
					\PYG{p}{\PYGZlt{}}\PYG{n+nt}{td}\PYG{p}{\PYGZgt{}} Celda \PYG{p}{\PYGZlt{}}\PYG{p}{/}\PYG{n+nt}{td}\PYG{p}{\PYGZgt{}}
					\PYG{p}{\PYGZlt{}}\PYG{n+nt}{td}\PYG{p}{\PYGZgt{}} Celda \PYG{p}{\PYGZlt{}}\PYG{p}{/}\PYG{n+nt}{td}\PYG{p}{\PYGZgt{}}
				\PYG{p}{\PYGZlt{}}\PYG{p}{/}\PYG{n+nt}{tr}\PYG{p}{\PYGZgt{}}
				\PYG{p}{\PYGZlt{}}\PYG{n+nt}{tr}\PYG{p}{\PYGZgt{}}
					\PYG{p}{\PYGZlt{}}\PYG{n+nt}{td}\PYG{p}{\PYGZgt{}} Celda \PYG{p}{\PYGZlt{}}\PYG{p}{/}\PYG{n+nt}{td}\PYG{p}{\PYGZgt{}}
					\PYG{p}{\PYGZlt{}}\PYG{n+nt}{td}\PYG{p}{\PYGZgt{}} Celda \PYG{p}{\PYGZlt{}}\PYG{p}{/}\PYG{n+nt}{td}\PYG{p}{\PYGZgt{}}
				\PYG{p}{\PYGZlt{}}\PYG{p}{/}\PYG{n+nt}{tr}\PYG{p}{\PYGZgt{}}
			\PYG{p}{\PYGZlt{}}\PYG{p}{/}\PYG{n+nt}{table}\PYG{p}{\PYGZgt{}}
			\PYG{p}{\PYGZlt{}}\PYG{p}{/}\PYG{n+nt}{td}\PYG{p}{\PYGZgt{}}
		\PYG{p}{\PYGZlt{}}\PYG{p}{/}\PYG{n+nt}{tr}\PYG{p}{\PYGZgt{}}
		\PYG{p}{\PYGZlt{}}\PYG{n+nt}{tr}\PYG{p}{\PYGZgt{}}
			\PYG{p}{\PYGZlt{}}\PYG{n+nt}{td}\PYG{p}{\PYGZgt{}} Celda \PYG{p}{\PYGZlt{}}\PYG{p}{/}\PYG{n+nt}{td}\PYG{p}{\PYGZgt{}}
			\PYG{p}{\PYGZlt{}}\PYG{n+nt}{td}\PYG{p}{\PYGZgt{}} Celda \PYG{p}{\PYGZlt{}}\PYG{p}{/}\PYG{n+nt}{td}\PYG{p}{\PYGZgt{}}
		\PYG{p}{\PYGZlt{}}\PYG{p}{/}\PYG{n+nt}{tr}\PYG{p}{\PYGZgt{}}
		\PYG{p}{\PYGZlt{}}\PYG{n+nt}{tr}\PYG{p}{\PYGZgt{}}
			\PYG{p}{\PYGZlt{}}\PYG{n+nt}{td}\PYG{p}{\PYGZgt{}} Celda \PYG{p}{\PYGZlt{}}\PYG{p}{/}\PYG{n+nt}{td}\PYG{p}{\PYGZgt{}}
			\PYG{p}{\PYGZlt{}}\PYG{n+nt}{td}\PYG{p}{\PYGZgt{}} Celda \PYG{p}{\PYGZlt{}}\PYG{p}{/}\PYG{n+nt}{td}\PYG{p}{\PYGZgt{}}
		\PYG{p}{\PYGZlt{}}\PYG{p}{/}\PYG{n+nt}{tr}\PYG{p}{\PYGZgt{}}
	\PYG{p}{\PYGZlt{}}\PYG{p}{/}\PYG{n+nt}{table}\PYG{p}{\PYGZgt{}}
\PYG{p}{\PYGZlt{}}\PYG{p}{/}\PYG{n+nt}{body}\PYG{p}{\PYGZgt{}}
\PYG{p}{\PYGZlt{}}\PYG{p}{/}\PYG{n+nt}{html}\PYG{p}{\PYGZgt{}}
        
\end{sphinxVerbatim}


\section{Tabla 10}
\label{\detokenize{ejercicios/html/anexo_tablas:tabla-10}}
Generar la tabla siguiente

\noindent{\hspace*{\fill}\sphinxincludegraphics[scale=0.6]{{foto_10}.png}\hspace*{\fill}}

Solución:

\begin{sphinxVerbatim}[commandchars=\\\{\}]
\PYG{c+cp}{\PYGZlt{}!DOCTYPE html\PYGZgt{}}
\PYG{p}{\PYGZlt{}}\PYG{n+nt}{html}\PYG{p}{\PYGZgt{}}
\PYG{p}{\PYGZlt{}}\PYG{n+nt}{head}\PYG{p}{\PYGZgt{}}
\PYG{p}{\PYGZlt{}}\PYG{n+nt}{meta} \PYG{n+na}{charset}\PYG{o}{=}\PYG{l+s}{\PYGZsq{}utf\PYGZhy{}8\PYGZsq{}}\PYG{p}{\PYGZgt{}}
\PYG{p}{\PYGZlt{}}\PYG{n+nt}{title}\PYG{p}{\PYGZgt{}}Ejercicio\PYG{p}{\PYGZlt{}}\PYG{p}{/}\PYG{n+nt}{title}\PYG{p}{\PYGZgt{}}
\PYG{p}{\PYGZlt{}}\PYG{p}{/}\PYG{n+nt}{head}\PYG{p}{\PYGZgt{}}
\PYG{p}{\PYGZlt{}}\PYG{n+nt}{body}\PYG{p}{\PYGZgt{}}
	\PYG{p}{\PYGZlt{}}\PYG{n+nt}{table} \PYG{n+na}{border}\PYG{o}{=}\PYG{l+s}{\PYGZsq{}1\PYGZsq{}}\PYG{p}{\PYGZgt{}}
		\PYG{p}{\PYGZlt{}}\PYG{n+nt}{tr}\PYG{p}{\PYGZgt{}}
			\PYG{p}{\PYGZlt{}}\PYG{n+nt}{td}\PYG{p}{\PYGZgt{}} Celda \PYG{p}{\PYGZlt{}}\PYG{p}{/}\PYG{n+nt}{td}\PYG{p}{\PYGZgt{}}
			\PYG{p}{\PYGZlt{}}\PYG{n+nt}{td}\PYG{p}{\PYGZgt{}}
			\PYG{p}{\PYGZlt{}}\PYG{n+nt}{table} \PYG{n+na}{border}\PYG{o}{=}\PYG{l+s}{\PYGZsq{}1\PYGZsq{}}\PYG{p}{\PYGZgt{}}
				\PYG{p}{\PYGZlt{}}\PYG{n+nt}{tr}\PYG{p}{\PYGZgt{}}
					\PYG{p}{\PYGZlt{}}\PYG{n+nt}{td}\PYG{p}{\PYGZgt{}} Celda \PYG{p}{\PYGZlt{}}\PYG{p}{/}\PYG{n+nt}{td}\PYG{p}{\PYGZgt{}}
					\PYG{p}{\PYGZlt{}}\PYG{n+nt}{td}\PYG{p}{\PYGZgt{}} Celda \PYG{p}{\PYGZlt{}}\PYG{p}{/}\PYG{n+nt}{td}\PYG{p}{\PYGZgt{}}
				\PYG{p}{\PYGZlt{}}\PYG{p}{/}\PYG{n+nt}{tr}\PYG{p}{\PYGZgt{}}
				\PYG{p}{\PYGZlt{}}\PYG{n+nt}{tr}\PYG{p}{\PYGZgt{}}
					\PYG{p}{\PYGZlt{}}\PYG{n+nt}{td}\PYG{p}{\PYGZgt{}} Celda \PYG{p}{\PYGZlt{}}\PYG{p}{/}\PYG{n+nt}{td}\PYG{p}{\PYGZgt{}}
					\PYG{p}{\PYGZlt{}}\PYG{n+nt}{td}\PYG{p}{\PYGZgt{}} Celda \PYG{p}{\PYGZlt{}}\PYG{p}{/}\PYG{n+nt}{td}\PYG{p}{\PYGZgt{}}
				\PYG{p}{\PYGZlt{}}\PYG{p}{/}\PYG{n+nt}{tr}\PYG{p}{\PYGZgt{}}
				\PYG{p}{\PYGZlt{}}\PYG{n+nt}{tr}\PYG{p}{\PYGZgt{}}
					\PYG{p}{\PYGZlt{}}\PYG{n+nt}{td}\PYG{p}{\PYGZgt{}} Celda \PYG{p}{\PYGZlt{}}\PYG{p}{/}\PYG{n+nt}{td}\PYG{p}{\PYGZgt{}}
					\PYG{p}{\PYGZlt{}}\PYG{n+nt}{td}\PYG{p}{\PYGZgt{}} Celda \PYG{p}{\PYGZlt{}}\PYG{p}{/}\PYG{n+nt}{td}\PYG{p}{\PYGZgt{}}
				\PYG{p}{\PYGZlt{}}\PYG{p}{/}\PYG{n+nt}{tr}\PYG{p}{\PYGZgt{}}
				\PYG{p}{\PYGZlt{}}\PYG{n+nt}{tr}\PYG{p}{\PYGZgt{}}
					\PYG{p}{\PYGZlt{}}\PYG{n+nt}{td}\PYG{p}{\PYGZgt{}} Celda \PYG{p}{\PYGZlt{}}\PYG{p}{/}\PYG{n+nt}{td}\PYG{p}{\PYGZgt{}}
					\PYG{p}{\PYGZlt{}}\PYG{n+nt}{td}\PYG{p}{\PYGZgt{}} Celda \PYG{p}{\PYGZlt{}}\PYG{p}{/}\PYG{n+nt}{td}\PYG{p}{\PYGZgt{}}
				\PYG{p}{\PYGZlt{}}\PYG{p}{/}\PYG{n+nt}{tr}\PYG{p}{\PYGZgt{}}
			\PYG{p}{\PYGZlt{}}\PYG{p}{/}\PYG{n+nt}{table}\PYG{p}{\PYGZgt{}}
			\PYG{p}{\PYGZlt{}}\PYG{p}{/}\PYG{n+nt}{td}\PYG{p}{\PYGZgt{}}
			\PYG{p}{\PYGZlt{}}\PYG{n+nt}{td}\PYG{p}{\PYGZgt{}} Celda \PYG{p}{\PYGZlt{}}\PYG{p}{/}\PYG{n+nt}{td}\PYG{p}{\PYGZgt{}}
			\PYG{p}{\PYGZlt{}}\PYG{n+nt}{td}\PYG{p}{\PYGZgt{}} Celda \PYG{p}{\PYGZlt{}}\PYG{p}{/}\PYG{n+nt}{td}\PYG{p}{\PYGZgt{}}
		\PYG{p}{\PYGZlt{}}\PYG{p}{/}\PYG{n+nt}{tr}\PYG{p}{\PYGZgt{}}
		\PYG{p}{\PYGZlt{}}\PYG{n+nt}{tr}\PYG{p}{\PYGZgt{}}
			\PYG{p}{\PYGZlt{}}\PYG{n+nt}{td}\PYG{p}{\PYGZgt{}} Celda \PYG{p}{\PYGZlt{}}\PYG{p}{/}\PYG{n+nt}{td}\PYG{p}{\PYGZgt{}}
			\PYG{p}{\PYGZlt{}}\PYG{n+nt}{td}\PYG{p}{\PYGZgt{}} Celda \PYG{p}{\PYGZlt{}}\PYG{p}{/}\PYG{n+nt}{td}\PYG{p}{\PYGZgt{}}
			\PYG{p}{\PYGZlt{}}\PYG{n+nt}{td}\PYG{p}{\PYGZgt{}}
			\PYG{p}{\PYGZlt{}}\PYG{n+nt}{table} \PYG{n+na}{border}\PYG{o}{=}\PYG{l+s}{\PYGZsq{}1\PYGZsq{}}\PYG{p}{\PYGZgt{}}
				\PYG{p}{\PYGZlt{}}\PYG{n+nt}{tr}\PYG{p}{\PYGZgt{}}
					\PYG{p}{\PYGZlt{}}\PYG{n+nt}{td}\PYG{p}{\PYGZgt{}} Celda \PYG{p}{\PYGZlt{}}\PYG{p}{/}\PYG{n+nt}{td}\PYG{p}{\PYGZgt{}}
					\PYG{p}{\PYGZlt{}}\PYG{n+nt}{td}\PYG{p}{\PYGZgt{}} Celda \PYG{p}{\PYGZlt{}}\PYG{p}{/}\PYG{n+nt}{td}\PYG{p}{\PYGZgt{}}
					\PYG{p}{\PYGZlt{}}\PYG{n+nt}{td}\PYG{p}{\PYGZgt{}} Celda \PYG{p}{\PYGZlt{}}\PYG{p}{/}\PYG{n+nt}{td}\PYG{p}{\PYGZgt{}}
				\PYG{p}{\PYGZlt{}}\PYG{p}{/}\PYG{n+nt}{tr}\PYG{p}{\PYGZgt{}}
				\PYG{p}{\PYGZlt{}}\PYG{n+nt}{tr}\PYG{p}{\PYGZgt{}}
					\PYG{p}{\PYGZlt{}}\PYG{n+nt}{td}\PYG{p}{\PYGZgt{}} Celda \PYG{p}{\PYGZlt{}}\PYG{p}{/}\PYG{n+nt}{td}\PYG{p}{\PYGZgt{}}
					\PYG{p}{\PYGZlt{}}\PYG{n+nt}{td}\PYG{p}{\PYGZgt{}} Celda \PYG{p}{\PYGZlt{}}\PYG{p}{/}\PYG{n+nt}{td}\PYG{p}{\PYGZgt{}}
					\PYG{p}{\PYGZlt{}}\PYG{n+nt}{td}\PYG{p}{\PYGZgt{}} Celda \PYG{p}{\PYGZlt{}}\PYG{p}{/}\PYG{n+nt}{td}\PYG{p}{\PYGZgt{}}
				\PYG{p}{\PYGZlt{}}\PYG{p}{/}\PYG{n+nt}{tr}\PYG{p}{\PYGZgt{}}
				\PYG{p}{\PYGZlt{}}\PYG{n+nt}{tr}\PYG{p}{\PYGZgt{}}
					\PYG{p}{\PYGZlt{}}\PYG{n+nt}{td}\PYG{p}{\PYGZgt{}} Celda \PYG{p}{\PYGZlt{}}\PYG{p}{/}\PYG{n+nt}{td}\PYG{p}{\PYGZgt{}}
					\PYG{p}{\PYGZlt{}}\PYG{n+nt}{td}\PYG{p}{\PYGZgt{}} Celda \PYG{p}{\PYGZlt{}}\PYG{p}{/}\PYG{n+nt}{td}\PYG{p}{\PYGZgt{}}
					\PYG{p}{\PYGZlt{}}\PYG{n+nt}{td}\PYG{p}{\PYGZgt{}} Celda \PYG{p}{\PYGZlt{}}\PYG{p}{/}\PYG{n+nt}{td}\PYG{p}{\PYGZgt{}}
				\PYG{p}{\PYGZlt{}}\PYG{p}{/}\PYG{n+nt}{tr}\PYG{p}{\PYGZgt{}}
				\PYG{p}{\PYGZlt{}}\PYG{n+nt}{tr}\PYG{p}{\PYGZgt{}}
					\PYG{p}{\PYGZlt{}}\PYG{n+nt}{td}\PYG{p}{\PYGZgt{}} Celda \PYG{p}{\PYGZlt{}}\PYG{p}{/}\PYG{n+nt}{td}\PYG{p}{\PYGZgt{}}
					\PYG{p}{\PYGZlt{}}\PYG{n+nt}{td}\PYG{p}{\PYGZgt{}} Celda \PYG{p}{\PYGZlt{}}\PYG{p}{/}\PYG{n+nt}{td}\PYG{p}{\PYGZgt{}}
					\PYG{p}{\PYGZlt{}}\PYG{n+nt}{td}\PYG{p}{\PYGZgt{}} Celda \PYG{p}{\PYGZlt{}}\PYG{p}{/}\PYG{n+nt}{td}\PYG{p}{\PYGZgt{}}
				\PYG{p}{\PYGZlt{}}\PYG{p}{/}\PYG{n+nt}{tr}\PYG{p}{\PYGZgt{}}
			\PYG{p}{\PYGZlt{}}\PYG{p}{/}\PYG{n+nt}{table}\PYG{p}{\PYGZgt{}}
			\PYG{p}{\PYGZlt{}}\PYG{p}{/}\PYG{n+nt}{td}\PYG{p}{\PYGZgt{}}
			\PYG{p}{\PYGZlt{}}\PYG{n+nt}{td}\PYG{p}{\PYGZgt{}} Celda \PYG{p}{\PYGZlt{}}\PYG{p}{/}\PYG{n+nt}{td}\PYG{p}{\PYGZgt{}}
		\PYG{p}{\PYGZlt{}}\PYG{p}{/}\PYG{n+nt}{tr}\PYG{p}{\PYGZgt{}}
		\PYG{p}{\PYGZlt{}}\PYG{n+nt}{tr}\PYG{p}{\PYGZgt{}}
			\PYG{p}{\PYGZlt{}}\PYG{n+nt}{td}\PYG{p}{\PYGZgt{}}
			\PYG{p}{\PYGZlt{}}\PYG{n+nt}{table} \PYG{n+na}{border}\PYG{o}{=}\PYG{l+s}{\PYGZsq{}1\PYGZsq{}}\PYG{p}{\PYGZgt{}}
				\PYG{p}{\PYGZlt{}}\PYG{n+nt}{tr}\PYG{p}{\PYGZgt{}}
					\PYG{p}{\PYGZlt{}}\PYG{n+nt}{td}\PYG{p}{\PYGZgt{}} Celda \PYG{p}{\PYGZlt{}}\PYG{p}{/}\PYG{n+nt}{td}\PYG{p}{\PYGZgt{}}
					\PYG{p}{\PYGZlt{}}\PYG{n+nt}{td}\PYG{p}{\PYGZgt{}} Celda \PYG{p}{\PYGZlt{}}\PYG{p}{/}\PYG{n+nt}{td}\PYG{p}{\PYGZgt{}}
				\PYG{p}{\PYGZlt{}}\PYG{p}{/}\PYG{n+nt}{tr}\PYG{p}{\PYGZgt{}}
				\PYG{p}{\PYGZlt{}}\PYG{n+nt}{tr}\PYG{p}{\PYGZgt{}}
					\PYG{p}{\PYGZlt{}}\PYG{n+nt}{td}\PYG{p}{\PYGZgt{}} Celda \PYG{p}{\PYGZlt{}}\PYG{p}{/}\PYG{n+nt}{td}\PYG{p}{\PYGZgt{}}
					\PYG{p}{\PYGZlt{}}\PYG{n+nt}{td}\PYG{p}{\PYGZgt{}} Celda \PYG{p}{\PYGZlt{}}\PYG{p}{/}\PYG{n+nt}{td}\PYG{p}{\PYGZgt{}}
				\PYG{p}{\PYGZlt{}}\PYG{p}{/}\PYG{n+nt}{tr}\PYG{p}{\PYGZgt{}}
			\PYG{p}{\PYGZlt{}}\PYG{p}{/}\PYG{n+nt}{table}\PYG{p}{\PYGZgt{}}
			\PYG{p}{\PYGZlt{}}\PYG{p}{/}\PYG{n+nt}{td}\PYG{p}{\PYGZgt{}}
			\PYG{p}{\PYGZlt{}}\PYG{n+nt}{td}\PYG{p}{\PYGZgt{}} Celda \PYG{p}{\PYGZlt{}}\PYG{p}{/}\PYG{n+nt}{td}\PYG{p}{\PYGZgt{}}
			\PYG{p}{\PYGZlt{}}\PYG{n+nt}{td}\PYG{p}{\PYGZgt{}} Celda \PYG{p}{\PYGZlt{}}\PYG{p}{/}\PYG{n+nt}{td}\PYG{p}{\PYGZgt{}}
			\PYG{p}{\PYGZlt{}}\PYG{n+nt}{td}\PYG{p}{\PYGZgt{}}
			\PYG{p}{\PYGZlt{}}\PYG{n+nt}{table} \PYG{n+na}{border}\PYG{o}{=}\PYG{l+s}{\PYGZsq{}1\PYGZsq{}}\PYG{p}{\PYGZgt{}}
				\PYG{p}{\PYGZlt{}}\PYG{n+nt}{tr}\PYG{p}{\PYGZgt{}}
					\PYG{p}{\PYGZlt{}}\PYG{n+nt}{td}\PYG{p}{\PYGZgt{}} Celda \PYG{p}{\PYGZlt{}}\PYG{p}{/}\PYG{n+nt}{td}\PYG{p}{\PYGZgt{}}
					\PYG{p}{\PYGZlt{}}\PYG{n+nt}{td}\PYG{p}{\PYGZgt{}} Celda \PYG{p}{\PYGZlt{}}\PYG{p}{/}\PYG{n+nt}{td}\PYG{p}{\PYGZgt{}}
					\PYG{p}{\PYGZlt{}}\PYG{n+nt}{td}\PYG{p}{\PYGZgt{}} Celda \PYG{p}{\PYGZlt{}}\PYG{p}{/}\PYG{n+nt}{td}\PYG{p}{\PYGZgt{}}
				\PYG{p}{\PYGZlt{}}\PYG{p}{/}\PYG{n+nt}{tr}\PYG{p}{\PYGZgt{}}
				\PYG{p}{\PYGZlt{}}\PYG{n+nt}{tr}\PYG{p}{\PYGZgt{}}
					\PYG{p}{\PYGZlt{}}\PYG{n+nt}{td}\PYG{p}{\PYGZgt{}} Celda \PYG{p}{\PYGZlt{}}\PYG{p}{/}\PYG{n+nt}{td}\PYG{p}{\PYGZgt{}}
					\PYG{p}{\PYGZlt{}}\PYG{n+nt}{td}\PYG{p}{\PYGZgt{}} Celda \PYG{p}{\PYGZlt{}}\PYG{p}{/}\PYG{n+nt}{td}\PYG{p}{\PYGZgt{}}
					\PYG{p}{\PYGZlt{}}\PYG{n+nt}{td}\PYG{p}{\PYGZgt{}} Celda \PYG{p}{\PYGZlt{}}\PYG{p}{/}\PYG{n+nt}{td}\PYG{p}{\PYGZgt{}}
				\PYG{p}{\PYGZlt{}}\PYG{p}{/}\PYG{n+nt}{tr}\PYG{p}{\PYGZgt{}}
			\PYG{p}{\PYGZlt{}}\PYG{p}{/}\PYG{n+nt}{table}\PYG{p}{\PYGZgt{}}
			\PYG{p}{\PYGZlt{}}\PYG{p}{/}\PYG{n+nt}{td}\PYG{p}{\PYGZgt{}}
		\PYG{p}{\PYGZlt{}}\PYG{p}{/}\PYG{n+nt}{tr}\PYG{p}{\PYGZgt{}}
		\PYG{p}{\PYGZlt{}}\PYG{n+nt}{tr}\PYG{p}{\PYGZgt{}}
			\PYG{p}{\PYGZlt{}}\PYG{n+nt}{td}\PYG{p}{\PYGZgt{}} Celda \PYG{p}{\PYGZlt{}}\PYG{p}{/}\PYG{n+nt}{td}\PYG{p}{\PYGZgt{}}
			\PYG{p}{\PYGZlt{}}\PYG{n+nt}{td}\PYG{p}{\PYGZgt{}} Celda \PYG{p}{\PYGZlt{}}\PYG{p}{/}\PYG{n+nt}{td}\PYG{p}{\PYGZgt{}}
			\PYG{p}{\PYGZlt{}}\PYG{n+nt}{td}\PYG{p}{\PYGZgt{}} Celda \PYG{p}{\PYGZlt{}}\PYG{p}{/}\PYG{n+nt}{td}\PYG{p}{\PYGZgt{}}
			\PYG{p}{\PYGZlt{}}\PYG{n+nt}{td}\PYG{p}{\PYGZgt{}} Celda \PYG{p}{\PYGZlt{}}\PYG{p}{/}\PYG{n+nt}{td}\PYG{p}{\PYGZgt{}}
		\PYG{p}{\PYGZlt{}}\PYG{p}{/}\PYG{n+nt}{tr}\PYG{p}{\PYGZgt{}}
	\PYG{p}{\PYGZlt{}}\PYG{p}{/}\PYG{n+nt}{table}\PYG{p}{\PYGZgt{}}
\PYG{p}{\PYGZlt{}}\PYG{p}{/}\PYG{n+nt}{body}\PYG{p}{\PYGZgt{}}
\PYG{p}{\PYGZlt{}}\PYG{p}{/}\PYG{n+nt}{html}\PYG{p}{\PYGZgt{}}
        
\end{sphinxVerbatim}


\section{Tabla 11}
\label{\detokenize{ejercicios/html/anexo_tablas:tabla-11}}
Generar la tabla siguiente

\noindent{\hspace*{\fill}\sphinxincludegraphics[scale=0.6]{{foto_11}.png}\hspace*{\fill}}

Solución:

\begin{sphinxVerbatim}[commandchars=\\\{\}]
\PYG{c+cp}{\PYGZlt{}!DOCTYPE html\PYGZgt{}}
\PYG{p}{\PYGZlt{}}\PYG{n+nt}{html}\PYG{p}{\PYGZgt{}}
\PYG{p}{\PYGZlt{}}\PYG{n+nt}{head}\PYG{p}{\PYGZgt{}}
\PYG{p}{\PYGZlt{}}\PYG{n+nt}{meta} \PYG{n+na}{charset}\PYG{o}{=}\PYG{l+s}{\PYGZsq{}utf\PYGZhy{}8\PYGZsq{}}\PYG{p}{\PYGZgt{}}
\PYG{p}{\PYGZlt{}}\PYG{n+nt}{title}\PYG{p}{\PYGZgt{}}Ejercicio\PYG{p}{\PYGZlt{}}\PYG{p}{/}\PYG{n+nt}{title}\PYG{p}{\PYGZgt{}}
\PYG{p}{\PYGZlt{}}\PYG{p}{/}\PYG{n+nt}{head}\PYG{p}{\PYGZgt{}}
\PYG{p}{\PYGZlt{}}\PYG{n+nt}{body}\PYG{p}{\PYGZgt{}}
	\PYG{p}{\PYGZlt{}}\PYG{n+nt}{table} \PYG{n+na}{border}\PYG{o}{=}\PYG{l+s}{\PYGZsq{}1\PYGZsq{}}\PYG{p}{\PYGZgt{}}
		\PYG{p}{\PYGZlt{}}\PYG{n+nt}{tr}\PYG{p}{\PYGZgt{}}
			\PYG{p}{\PYGZlt{}}\PYG{n+nt}{td}\PYG{p}{\PYGZgt{}}
			\PYG{p}{\PYGZlt{}}\PYG{n+nt}{table} \PYG{n+na}{border}\PYG{o}{=}\PYG{l+s}{\PYGZsq{}1\PYGZsq{}}\PYG{p}{\PYGZgt{}}
				\PYG{p}{\PYGZlt{}}\PYG{n+nt}{tr}\PYG{p}{\PYGZgt{}}
					\PYG{p}{\PYGZlt{}}\PYG{n+nt}{td}\PYG{p}{\PYGZgt{}} Celda \PYG{p}{\PYGZlt{}}\PYG{p}{/}\PYG{n+nt}{td}\PYG{p}{\PYGZgt{}}
					\PYG{p}{\PYGZlt{}}\PYG{n+nt}{td}\PYG{p}{\PYGZgt{}} Celda \PYG{p}{\PYGZlt{}}\PYG{p}{/}\PYG{n+nt}{td}\PYG{p}{\PYGZgt{}}
					\PYG{p}{\PYGZlt{}}\PYG{n+nt}{td}\PYG{p}{\PYGZgt{}} Celda \PYG{p}{\PYGZlt{}}\PYG{p}{/}\PYG{n+nt}{td}\PYG{p}{\PYGZgt{}}
					\PYG{p}{\PYGZlt{}}\PYG{n+nt}{td}\PYG{p}{\PYGZgt{}} Celda \PYG{p}{\PYGZlt{}}\PYG{p}{/}\PYG{n+nt}{td}\PYG{p}{\PYGZgt{}}
				\PYG{p}{\PYGZlt{}}\PYG{p}{/}\PYG{n+nt}{tr}\PYG{p}{\PYGZgt{}}
				\PYG{p}{\PYGZlt{}}\PYG{n+nt}{tr}\PYG{p}{\PYGZgt{}}
					\PYG{p}{\PYGZlt{}}\PYG{n+nt}{td}\PYG{p}{\PYGZgt{}} Celda \PYG{p}{\PYGZlt{}}\PYG{p}{/}\PYG{n+nt}{td}\PYG{p}{\PYGZgt{}}
					\PYG{p}{\PYGZlt{}}\PYG{n+nt}{td}\PYG{p}{\PYGZgt{}} Celda \PYG{p}{\PYGZlt{}}\PYG{p}{/}\PYG{n+nt}{td}\PYG{p}{\PYGZgt{}}
					\PYG{p}{\PYGZlt{}}\PYG{n+nt}{td}\PYG{p}{\PYGZgt{}} Celda \PYG{p}{\PYGZlt{}}\PYG{p}{/}\PYG{n+nt}{td}\PYG{p}{\PYGZgt{}}
					\PYG{p}{\PYGZlt{}}\PYG{n+nt}{td}\PYG{p}{\PYGZgt{}} Celda \PYG{p}{\PYGZlt{}}\PYG{p}{/}\PYG{n+nt}{td}\PYG{p}{\PYGZgt{}}
				\PYG{p}{\PYGZlt{}}\PYG{p}{/}\PYG{n+nt}{tr}\PYG{p}{\PYGZgt{}}
				\PYG{p}{\PYGZlt{}}\PYG{n+nt}{tr}\PYG{p}{\PYGZgt{}}
					\PYG{p}{\PYGZlt{}}\PYG{n+nt}{td}\PYG{p}{\PYGZgt{}} Celda \PYG{p}{\PYGZlt{}}\PYG{p}{/}\PYG{n+nt}{td}\PYG{p}{\PYGZgt{}}
					\PYG{p}{\PYGZlt{}}\PYG{n+nt}{td}\PYG{p}{\PYGZgt{}} Celda \PYG{p}{\PYGZlt{}}\PYG{p}{/}\PYG{n+nt}{td}\PYG{p}{\PYGZgt{}}
					\PYG{p}{\PYGZlt{}}\PYG{n+nt}{td}\PYG{p}{\PYGZgt{}} Celda \PYG{p}{\PYGZlt{}}\PYG{p}{/}\PYG{n+nt}{td}\PYG{p}{\PYGZgt{}}
					\PYG{p}{\PYGZlt{}}\PYG{n+nt}{td}\PYG{p}{\PYGZgt{}} Celda \PYG{p}{\PYGZlt{}}\PYG{p}{/}\PYG{n+nt}{td}\PYG{p}{\PYGZgt{}}
				\PYG{p}{\PYGZlt{}}\PYG{p}{/}\PYG{n+nt}{tr}\PYG{p}{\PYGZgt{}}
				\PYG{p}{\PYGZlt{}}\PYG{n+nt}{tr}\PYG{p}{\PYGZgt{}}
					\PYG{p}{\PYGZlt{}}\PYG{n+nt}{td}\PYG{p}{\PYGZgt{}} Celda \PYG{p}{\PYGZlt{}}\PYG{p}{/}\PYG{n+nt}{td}\PYG{p}{\PYGZgt{}}
					\PYG{p}{\PYGZlt{}}\PYG{n+nt}{td}\PYG{p}{\PYGZgt{}} Celda \PYG{p}{\PYGZlt{}}\PYG{p}{/}\PYG{n+nt}{td}\PYG{p}{\PYGZgt{}}
					\PYG{p}{\PYGZlt{}}\PYG{n+nt}{td}\PYG{p}{\PYGZgt{}} Celda \PYG{p}{\PYGZlt{}}\PYG{p}{/}\PYG{n+nt}{td}\PYG{p}{\PYGZgt{}}
					\PYG{p}{\PYGZlt{}}\PYG{n+nt}{td}\PYG{p}{\PYGZgt{}} Celda \PYG{p}{\PYGZlt{}}\PYG{p}{/}\PYG{n+nt}{td}\PYG{p}{\PYGZgt{}}
				\PYG{p}{\PYGZlt{}}\PYG{p}{/}\PYG{n+nt}{tr}\PYG{p}{\PYGZgt{}}
			\PYG{p}{\PYGZlt{}}\PYG{p}{/}\PYG{n+nt}{table}\PYG{p}{\PYGZgt{}}
			\PYG{p}{\PYGZlt{}}\PYG{p}{/}\PYG{n+nt}{td}\PYG{p}{\PYGZgt{}}
			\PYG{p}{\PYGZlt{}}\PYG{n+nt}{td}\PYG{p}{\PYGZgt{}} Celda \PYG{p}{\PYGZlt{}}\PYG{p}{/}\PYG{n+nt}{td}\PYG{p}{\PYGZgt{}}
		\PYG{p}{\PYGZlt{}}\PYG{p}{/}\PYG{n+nt}{tr}\PYG{p}{\PYGZgt{}}
		\PYG{p}{\PYGZlt{}}\PYG{n+nt}{tr}\PYG{p}{\PYGZgt{}}
			\PYG{p}{\PYGZlt{}}\PYG{n+nt}{td}\PYG{p}{\PYGZgt{}}
			\PYG{p}{\PYGZlt{}}\PYG{n+nt}{table} \PYG{n+na}{border}\PYG{o}{=}\PYG{l+s}{\PYGZsq{}1\PYGZsq{}}\PYG{p}{\PYGZgt{}}
				\PYG{p}{\PYGZlt{}}\PYG{n+nt}{tr}\PYG{p}{\PYGZgt{}}
					\PYG{p}{\PYGZlt{}}\PYG{n+nt}{td}\PYG{p}{\PYGZgt{}} Celda \PYG{p}{\PYGZlt{}}\PYG{p}{/}\PYG{n+nt}{td}\PYG{p}{\PYGZgt{}}
					\PYG{p}{\PYGZlt{}}\PYG{n+nt}{td}\PYG{p}{\PYGZgt{}} Celda \PYG{p}{\PYGZlt{}}\PYG{p}{/}\PYG{n+nt}{td}\PYG{p}{\PYGZgt{}}
				\PYG{p}{\PYGZlt{}}\PYG{p}{/}\PYG{n+nt}{tr}\PYG{p}{\PYGZgt{}}
				\PYG{p}{\PYGZlt{}}\PYG{n+nt}{tr}\PYG{p}{\PYGZgt{}}
					\PYG{p}{\PYGZlt{}}\PYG{n+nt}{td}\PYG{p}{\PYGZgt{}} Celda \PYG{p}{\PYGZlt{}}\PYG{p}{/}\PYG{n+nt}{td}\PYG{p}{\PYGZgt{}}
					\PYG{p}{\PYGZlt{}}\PYG{n+nt}{td}\PYG{p}{\PYGZgt{}} Celda \PYG{p}{\PYGZlt{}}\PYG{p}{/}\PYG{n+nt}{td}\PYG{p}{\PYGZgt{}}
				\PYG{p}{\PYGZlt{}}\PYG{p}{/}\PYG{n+nt}{tr}\PYG{p}{\PYGZgt{}}
				\PYG{p}{\PYGZlt{}}\PYG{n+nt}{tr}\PYG{p}{\PYGZgt{}}
					\PYG{p}{\PYGZlt{}}\PYG{n+nt}{td}\PYG{p}{\PYGZgt{}} Celda \PYG{p}{\PYGZlt{}}\PYG{p}{/}\PYG{n+nt}{td}\PYG{p}{\PYGZgt{}}
					\PYG{p}{\PYGZlt{}}\PYG{n+nt}{td}\PYG{p}{\PYGZgt{}} Celda \PYG{p}{\PYGZlt{}}\PYG{p}{/}\PYG{n+nt}{td}\PYG{p}{\PYGZgt{}}
				\PYG{p}{\PYGZlt{}}\PYG{p}{/}\PYG{n+nt}{tr}\PYG{p}{\PYGZgt{}}
			\PYG{p}{\PYGZlt{}}\PYG{p}{/}\PYG{n+nt}{table}\PYG{p}{\PYGZgt{}}
			\PYG{p}{\PYGZlt{}}\PYG{p}{/}\PYG{n+nt}{td}\PYG{p}{\PYGZgt{}}
			\PYG{p}{\PYGZlt{}}\PYG{n+nt}{td}\PYG{p}{\PYGZgt{}} Celda \PYG{p}{\PYGZlt{}}\PYG{p}{/}\PYG{n+nt}{td}\PYG{p}{\PYGZgt{}}
		\PYG{p}{\PYGZlt{}}\PYG{p}{/}\PYG{n+nt}{tr}\PYG{p}{\PYGZgt{}}
		\PYG{p}{\PYGZlt{}}\PYG{n+nt}{tr}\PYG{p}{\PYGZgt{}}
			\PYG{p}{\PYGZlt{}}\PYG{n+nt}{td}\PYG{p}{\PYGZgt{}} Celda \PYG{p}{\PYGZlt{}}\PYG{p}{/}\PYG{n+nt}{td}\PYG{p}{\PYGZgt{}}
			\PYG{p}{\PYGZlt{}}\PYG{n+nt}{td}\PYG{p}{\PYGZgt{}} Celda \PYG{p}{\PYGZlt{}}\PYG{p}{/}\PYG{n+nt}{td}\PYG{p}{\PYGZgt{}}
		\PYG{p}{\PYGZlt{}}\PYG{p}{/}\PYG{n+nt}{tr}\PYG{p}{\PYGZgt{}}
		\PYG{p}{\PYGZlt{}}\PYG{n+nt}{tr}\PYG{p}{\PYGZgt{}}
			\PYG{p}{\PYGZlt{}}\PYG{n+nt}{td}\PYG{p}{\PYGZgt{}} Celda \PYG{p}{\PYGZlt{}}\PYG{p}{/}\PYG{n+nt}{td}\PYG{p}{\PYGZgt{}}
			\PYG{p}{\PYGZlt{}}\PYG{n+nt}{td}\PYG{p}{\PYGZgt{}} Celda \PYG{p}{\PYGZlt{}}\PYG{p}{/}\PYG{n+nt}{td}\PYG{p}{\PYGZgt{}}
		\PYG{p}{\PYGZlt{}}\PYG{p}{/}\PYG{n+nt}{tr}\PYG{p}{\PYGZgt{}}
	\PYG{p}{\PYGZlt{}}\PYG{p}{/}\PYG{n+nt}{table}\PYG{p}{\PYGZgt{}}
\PYG{p}{\PYGZlt{}}\PYG{p}{/}\PYG{n+nt}{body}\PYG{p}{\PYGZgt{}}
\PYG{p}{\PYGZlt{}}\PYG{p}{/}\PYG{n+nt}{html}\PYG{p}{\PYGZgt{}}
        
\end{sphinxVerbatim}


\section{Tabla 12}
\label{\detokenize{ejercicios/html/anexo_tablas:tabla-12}}
Generar la tabla siguiente

\noindent{\hspace*{\fill}\sphinxincludegraphics[scale=0.6]{{foto_12}.png}\hspace*{\fill}}

Solución:

\begin{sphinxVerbatim}[commandchars=\\\{\}]
\PYG{c+cp}{\PYGZlt{}!DOCTYPE html\PYGZgt{}}
\PYG{p}{\PYGZlt{}}\PYG{n+nt}{html}\PYG{p}{\PYGZgt{}}
\PYG{p}{\PYGZlt{}}\PYG{n+nt}{head}\PYG{p}{\PYGZgt{}}
\PYG{p}{\PYGZlt{}}\PYG{n+nt}{meta} \PYG{n+na}{charset}\PYG{o}{=}\PYG{l+s}{\PYGZsq{}utf\PYGZhy{}8\PYGZsq{}}\PYG{p}{\PYGZgt{}}
\PYG{p}{\PYGZlt{}}\PYG{n+nt}{title}\PYG{p}{\PYGZgt{}}Ejercicio\PYG{p}{\PYGZlt{}}\PYG{p}{/}\PYG{n+nt}{title}\PYG{p}{\PYGZgt{}}
\PYG{p}{\PYGZlt{}}\PYG{p}{/}\PYG{n+nt}{head}\PYG{p}{\PYGZgt{}}
\PYG{p}{\PYGZlt{}}\PYG{n+nt}{body}\PYG{p}{\PYGZgt{}}
	\PYG{p}{\PYGZlt{}}\PYG{n+nt}{table} \PYG{n+na}{border}\PYG{o}{=}\PYG{l+s}{\PYGZsq{}1\PYGZsq{}}\PYG{p}{\PYGZgt{}}
		\PYG{p}{\PYGZlt{}}\PYG{n+nt}{tr}\PYG{p}{\PYGZgt{}}
			\PYG{p}{\PYGZlt{}}\PYG{n+nt}{td}\PYG{p}{\PYGZgt{}} Celda \PYG{p}{\PYGZlt{}}\PYG{p}{/}\PYG{n+nt}{td}\PYG{p}{\PYGZgt{}}
			\PYG{p}{\PYGZlt{}}\PYG{n+nt}{td}\PYG{p}{\PYGZgt{}} Celda \PYG{p}{\PYGZlt{}}\PYG{p}{/}\PYG{n+nt}{td}\PYG{p}{\PYGZgt{}}
			\PYG{p}{\PYGZlt{}}\PYG{n+nt}{td}\PYG{p}{\PYGZgt{}} Celda \PYG{p}{\PYGZlt{}}\PYG{p}{/}\PYG{n+nt}{td}\PYG{p}{\PYGZgt{}}
			\PYG{p}{\PYGZlt{}}\PYG{n+nt}{td}\PYG{p}{\PYGZgt{}} Celda \PYG{p}{\PYGZlt{}}\PYG{p}{/}\PYG{n+nt}{td}\PYG{p}{\PYGZgt{}}
		\PYG{p}{\PYGZlt{}}\PYG{p}{/}\PYG{n+nt}{tr}\PYG{p}{\PYGZgt{}}
		\PYG{p}{\PYGZlt{}}\PYG{n+nt}{tr}\PYG{p}{\PYGZgt{}}
			\PYG{p}{\PYGZlt{}}\PYG{n+nt}{td}\PYG{p}{\PYGZgt{}} Celda \PYG{p}{\PYGZlt{}}\PYG{p}{/}\PYG{n+nt}{td}\PYG{p}{\PYGZgt{}}
			\PYG{p}{\PYGZlt{}}\PYG{n+nt}{td}\PYG{p}{\PYGZgt{}} Celda \PYG{p}{\PYGZlt{}}\PYG{p}{/}\PYG{n+nt}{td}\PYG{p}{\PYGZgt{}}
			\PYG{p}{\PYGZlt{}}\PYG{n+nt}{td}\PYG{p}{\PYGZgt{}} Celda \PYG{p}{\PYGZlt{}}\PYG{p}{/}\PYG{n+nt}{td}\PYG{p}{\PYGZgt{}}
			\PYG{p}{\PYGZlt{}}\PYG{n+nt}{td}\PYG{p}{\PYGZgt{}} Celda \PYG{p}{\PYGZlt{}}\PYG{p}{/}\PYG{n+nt}{td}\PYG{p}{\PYGZgt{}}
		\PYG{p}{\PYGZlt{}}\PYG{p}{/}\PYG{n+nt}{tr}\PYG{p}{\PYGZgt{}}
	\PYG{p}{\PYGZlt{}}\PYG{p}{/}\PYG{n+nt}{table}\PYG{p}{\PYGZgt{}}
\PYG{p}{\PYGZlt{}}\PYG{p}{/}\PYG{n+nt}{body}\PYG{p}{\PYGZgt{}}
\PYG{p}{\PYGZlt{}}\PYG{p}{/}\PYG{n+nt}{html}\PYG{p}{\PYGZgt{}}
        
\end{sphinxVerbatim}


\section{Tabla 13}
\label{\detokenize{ejercicios/html/anexo_tablas:tabla-13}}
Generar la tabla siguiente

\noindent{\hspace*{\fill}\sphinxincludegraphics[scale=0.6]{{foto_13}.png}\hspace*{\fill}}

Solución:

\begin{sphinxVerbatim}[commandchars=\\\{\}]
\PYG{c+cp}{\PYGZlt{}!DOCTYPE html\PYGZgt{}}
\PYG{p}{\PYGZlt{}}\PYG{n+nt}{html}\PYG{p}{\PYGZgt{}}
\PYG{p}{\PYGZlt{}}\PYG{n+nt}{head}\PYG{p}{\PYGZgt{}}
\PYG{p}{\PYGZlt{}}\PYG{n+nt}{meta} \PYG{n+na}{charset}\PYG{o}{=}\PYG{l+s}{\PYGZsq{}utf\PYGZhy{}8\PYGZsq{}}\PYG{p}{\PYGZgt{}}
\PYG{p}{\PYGZlt{}}\PYG{n+nt}{title}\PYG{p}{\PYGZgt{}}Ejercicio\PYG{p}{\PYGZlt{}}\PYG{p}{/}\PYG{n+nt}{title}\PYG{p}{\PYGZgt{}}
\PYG{p}{\PYGZlt{}}\PYG{p}{/}\PYG{n+nt}{head}\PYG{p}{\PYGZgt{}}
\PYG{p}{\PYGZlt{}}\PYG{n+nt}{body}\PYG{p}{\PYGZgt{}}
	\PYG{p}{\PYGZlt{}}\PYG{n+nt}{table} \PYG{n+na}{border}\PYG{o}{=}\PYG{l+s}{\PYGZsq{}1\PYGZsq{}}\PYG{p}{\PYGZgt{}}
		\PYG{p}{\PYGZlt{}}\PYG{n+nt}{tr}\PYG{p}{\PYGZgt{}}
			\PYG{p}{\PYGZlt{}}\PYG{n+nt}{td}\PYG{p}{\PYGZgt{}} Celda \PYG{p}{\PYGZlt{}}\PYG{p}{/}\PYG{n+nt}{td}\PYG{p}{\PYGZgt{}}
			\PYG{p}{\PYGZlt{}}\PYG{n+nt}{td}\PYG{p}{\PYGZgt{}} Celda \PYG{p}{\PYGZlt{}}\PYG{p}{/}\PYG{n+nt}{td}\PYG{p}{\PYGZgt{}}
			\PYG{p}{\PYGZlt{}}\PYG{n+nt}{td}\PYG{p}{\PYGZgt{}} Celda \PYG{p}{\PYGZlt{}}\PYG{p}{/}\PYG{n+nt}{td}\PYG{p}{\PYGZgt{}}
		\PYG{p}{\PYGZlt{}}\PYG{p}{/}\PYG{n+nt}{tr}\PYG{p}{\PYGZgt{}}
		\PYG{p}{\PYGZlt{}}\PYG{n+nt}{tr}\PYG{p}{\PYGZgt{}}
			\PYG{p}{\PYGZlt{}}\PYG{n+nt}{td}\PYG{p}{\PYGZgt{}} Celda \PYG{p}{\PYGZlt{}}\PYG{p}{/}\PYG{n+nt}{td}\PYG{p}{\PYGZgt{}}
			\PYG{p}{\PYGZlt{}}\PYG{n+nt}{td}\PYG{p}{\PYGZgt{}} Celda \PYG{p}{\PYGZlt{}}\PYG{p}{/}\PYG{n+nt}{td}\PYG{p}{\PYGZgt{}}
			\PYG{p}{\PYGZlt{}}\PYG{n+nt}{td}\PYG{p}{\PYGZgt{}} Celda \PYG{p}{\PYGZlt{}}\PYG{p}{/}\PYG{n+nt}{td}\PYG{p}{\PYGZgt{}}
		\PYG{p}{\PYGZlt{}}\PYG{p}{/}\PYG{n+nt}{tr}\PYG{p}{\PYGZgt{}}
		\PYG{p}{\PYGZlt{}}\PYG{n+nt}{tr}\PYG{p}{\PYGZgt{}}
			\PYG{p}{\PYGZlt{}}\PYG{n+nt}{td}\PYG{p}{\PYGZgt{}} Celda \PYG{p}{\PYGZlt{}}\PYG{p}{/}\PYG{n+nt}{td}\PYG{p}{\PYGZgt{}}
			\PYG{p}{\PYGZlt{}}\PYG{n+nt}{td}\PYG{p}{\PYGZgt{}} Celda \PYG{p}{\PYGZlt{}}\PYG{p}{/}\PYG{n+nt}{td}\PYG{p}{\PYGZgt{}}
			\PYG{p}{\PYGZlt{}}\PYG{n+nt}{td}\PYG{p}{\PYGZgt{}} Celda \PYG{p}{\PYGZlt{}}\PYG{p}{/}\PYG{n+nt}{td}\PYG{p}{\PYGZgt{}}
		\PYG{p}{\PYGZlt{}}\PYG{p}{/}\PYG{n+nt}{tr}\PYG{p}{\PYGZgt{}}
		\PYG{p}{\PYGZlt{}}\PYG{n+nt}{tr}\PYG{p}{\PYGZgt{}}
			\PYG{p}{\PYGZlt{}}\PYG{n+nt}{td}\PYG{p}{\PYGZgt{}} Celda \PYG{p}{\PYGZlt{}}\PYG{p}{/}\PYG{n+nt}{td}\PYG{p}{\PYGZgt{}}
			\PYG{p}{\PYGZlt{}}\PYG{n+nt}{td}\PYG{p}{\PYGZgt{}}
			\PYG{p}{\PYGZlt{}}\PYG{n+nt}{table} \PYG{n+na}{border}\PYG{o}{=}\PYG{l+s}{\PYGZsq{}1\PYGZsq{}}\PYG{p}{\PYGZgt{}}
				\PYG{p}{\PYGZlt{}}\PYG{n+nt}{tr}\PYG{p}{\PYGZgt{}}
					\PYG{p}{\PYGZlt{}}\PYG{n+nt}{td}\PYG{p}{\PYGZgt{}} Celda \PYG{p}{\PYGZlt{}}\PYG{p}{/}\PYG{n+nt}{td}\PYG{p}{\PYGZgt{}}
					\PYG{p}{\PYGZlt{}}\PYG{n+nt}{td}\PYG{p}{\PYGZgt{}} Celda \PYG{p}{\PYGZlt{}}\PYG{p}{/}\PYG{n+nt}{td}\PYG{p}{\PYGZgt{}}
					\PYG{p}{\PYGZlt{}}\PYG{n+nt}{td}\PYG{p}{\PYGZgt{}} Celda \PYG{p}{\PYGZlt{}}\PYG{p}{/}\PYG{n+nt}{td}\PYG{p}{\PYGZgt{}}
					\PYG{p}{\PYGZlt{}}\PYG{n+nt}{td}\PYG{p}{\PYGZgt{}} Celda \PYG{p}{\PYGZlt{}}\PYG{p}{/}\PYG{n+nt}{td}\PYG{p}{\PYGZgt{}}
				\PYG{p}{\PYGZlt{}}\PYG{p}{/}\PYG{n+nt}{tr}\PYG{p}{\PYGZgt{}}
				\PYG{p}{\PYGZlt{}}\PYG{n+nt}{tr}\PYG{p}{\PYGZgt{}}
					\PYG{p}{\PYGZlt{}}\PYG{n+nt}{td}\PYG{p}{\PYGZgt{}} Celda \PYG{p}{\PYGZlt{}}\PYG{p}{/}\PYG{n+nt}{td}\PYG{p}{\PYGZgt{}}
					\PYG{p}{\PYGZlt{}}\PYG{n+nt}{td}\PYG{p}{\PYGZgt{}} Celda \PYG{p}{\PYGZlt{}}\PYG{p}{/}\PYG{n+nt}{td}\PYG{p}{\PYGZgt{}}
					\PYG{p}{\PYGZlt{}}\PYG{n+nt}{td}\PYG{p}{\PYGZgt{}} Celda \PYG{p}{\PYGZlt{}}\PYG{p}{/}\PYG{n+nt}{td}\PYG{p}{\PYGZgt{}}
					\PYG{p}{\PYGZlt{}}\PYG{n+nt}{td}\PYG{p}{\PYGZgt{}} Celda \PYG{p}{\PYGZlt{}}\PYG{p}{/}\PYG{n+nt}{td}\PYG{p}{\PYGZgt{}}
				\PYG{p}{\PYGZlt{}}\PYG{p}{/}\PYG{n+nt}{tr}\PYG{p}{\PYGZgt{}}
				\PYG{p}{\PYGZlt{}}\PYG{n+nt}{tr}\PYG{p}{\PYGZgt{}}
					\PYG{p}{\PYGZlt{}}\PYG{n+nt}{td}\PYG{p}{\PYGZgt{}} Celda \PYG{p}{\PYGZlt{}}\PYG{p}{/}\PYG{n+nt}{td}\PYG{p}{\PYGZgt{}}
					\PYG{p}{\PYGZlt{}}\PYG{n+nt}{td}\PYG{p}{\PYGZgt{}} Celda \PYG{p}{\PYGZlt{}}\PYG{p}{/}\PYG{n+nt}{td}\PYG{p}{\PYGZgt{}}
					\PYG{p}{\PYGZlt{}}\PYG{n+nt}{td}\PYG{p}{\PYGZgt{}} Celda \PYG{p}{\PYGZlt{}}\PYG{p}{/}\PYG{n+nt}{td}\PYG{p}{\PYGZgt{}}
					\PYG{p}{\PYGZlt{}}\PYG{n+nt}{td}\PYG{p}{\PYGZgt{}} Celda \PYG{p}{\PYGZlt{}}\PYG{p}{/}\PYG{n+nt}{td}\PYG{p}{\PYGZgt{}}
				\PYG{p}{\PYGZlt{}}\PYG{p}{/}\PYG{n+nt}{tr}\PYG{p}{\PYGZgt{}}
				\PYG{p}{\PYGZlt{}}\PYG{n+nt}{tr}\PYG{p}{\PYGZgt{}}
					\PYG{p}{\PYGZlt{}}\PYG{n+nt}{td}\PYG{p}{\PYGZgt{}} Celda \PYG{p}{\PYGZlt{}}\PYG{p}{/}\PYG{n+nt}{td}\PYG{p}{\PYGZgt{}}
					\PYG{p}{\PYGZlt{}}\PYG{n+nt}{td}\PYG{p}{\PYGZgt{}} Celda \PYG{p}{\PYGZlt{}}\PYG{p}{/}\PYG{n+nt}{td}\PYG{p}{\PYGZgt{}}
					\PYG{p}{\PYGZlt{}}\PYG{n+nt}{td}\PYG{p}{\PYGZgt{}} Celda \PYG{p}{\PYGZlt{}}\PYG{p}{/}\PYG{n+nt}{td}\PYG{p}{\PYGZgt{}}
					\PYG{p}{\PYGZlt{}}\PYG{n+nt}{td}\PYG{p}{\PYGZgt{}} Celda \PYG{p}{\PYGZlt{}}\PYG{p}{/}\PYG{n+nt}{td}\PYG{p}{\PYGZgt{}}
				\PYG{p}{\PYGZlt{}}\PYG{p}{/}\PYG{n+nt}{tr}\PYG{p}{\PYGZgt{}}
			\PYG{p}{\PYGZlt{}}\PYG{p}{/}\PYG{n+nt}{table}\PYG{p}{\PYGZgt{}}
			\PYG{p}{\PYGZlt{}}\PYG{p}{/}\PYG{n+nt}{td}\PYG{p}{\PYGZgt{}}
			\PYG{p}{\PYGZlt{}}\PYG{n+nt}{td}\PYG{p}{\PYGZgt{}} Celda \PYG{p}{\PYGZlt{}}\PYG{p}{/}\PYG{n+nt}{td}\PYG{p}{\PYGZgt{}}
		\PYG{p}{\PYGZlt{}}\PYG{p}{/}\PYG{n+nt}{tr}\PYG{p}{\PYGZgt{}}
	\PYG{p}{\PYGZlt{}}\PYG{p}{/}\PYG{n+nt}{table}\PYG{p}{\PYGZgt{}}
\PYG{p}{\PYGZlt{}}\PYG{p}{/}\PYG{n+nt}{body}\PYG{p}{\PYGZgt{}}
\PYG{p}{\PYGZlt{}}\PYG{p}{/}\PYG{n+nt}{html}\PYG{p}{\PYGZgt{}}
        
\end{sphinxVerbatim}


\section{Tabla 14}
\label{\detokenize{ejercicios/html/anexo_tablas:tabla-14}}
Generar la tabla siguiente

\noindent{\hspace*{\fill}\sphinxincludegraphics[scale=0.6]{{foto_14}.png}\hspace*{\fill}}

Solución:

\begin{sphinxVerbatim}[commandchars=\\\{\}]
\PYG{c+cp}{\PYGZlt{}!DOCTYPE html\PYGZgt{}}
\PYG{p}{\PYGZlt{}}\PYG{n+nt}{html}\PYG{p}{\PYGZgt{}}
\PYG{p}{\PYGZlt{}}\PYG{n+nt}{head}\PYG{p}{\PYGZgt{}}
\PYG{p}{\PYGZlt{}}\PYG{n+nt}{meta} \PYG{n+na}{charset}\PYG{o}{=}\PYG{l+s}{\PYGZsq{}utf\PYGZhy{}8\PYGZsq{}}\PYG{p}{\PYGZgt{}}
\PYG{p}{\PYGZlt{}}\PYG{n+nt}{title}\PYG{p}{\PYGZgt{}}Ejercicio\PYG{p}{\PYGZlt{}}\PYG{p}{/}\PYG{n+nt}{title}\PYG{p}{\PYGZgt{}}
\PYG{p}{\PYGZlt{}}\PYG{p}{/}\PYG{n+nt}{head}\PYG{p}{\PYGZgt{}}
\PYG{p}{\PYGZlt{}}\PYG{n+nt}{body}\PYG{p}{\PYGZgt{}}
	\PYG{p}{\PYGZlt{}}\PYG{n+nt}{table} \PYG{n+na}{border}\PYG{o}{=}\PYG{l+s}{\PYGZsq{}1\PYGZsq{}}\PYG{p}{\PYGZgt{}}
		\PYG{p}{\PYGZlt{}}\PYG{n+nt}{tr}\PYG{p}{\PYGZgt{}}
			\PYG{p}{\PYGZlt{}}\PYG{n+nt}{td}\PYG{p}{\PYGZgt{}} Celda \PYG{p}{\PYGZlt{}}\PYG{p}{/}\PYG{n+nt}{td}\PYG{p}{\PYGZgt{}}
			\PYG{p}{\PYGZlt{}}\PYG{n+nt}{td}\PYG{p}{\PYGZgt{}} Celda \PYG{p}{\PYGZlt{}}\PYG{p}{/}\PYG{n+nt}{td}\PYG{p}{\PYGZgt{}}
			\PYG{p}{\PYGZlt{}}\PYG{n+nt}{td}\PYG{p}{\PYGZgt{}} Celda \PYG{p}{\PYGZlt{}}\PYG{p}{/}\PYG{n+nt}{td}\PYG{p}{\PYGZgt{}}
			\PYG{p}{\PYGZlt{}}\PYG{n+nt}{td}\PYG{p}{\PYGZgt{}}
			\PYG{p}{\PYGZlt{}}\PYG{n+nt}{table} \PYG{n+na}{border}\PYG{o}{=}\PYG{l+s}{\PYGZsq{}1\PYGZsq{}}\PYG{p}{\PYGZgt{}}
				\PYG{p}{\PYGZlt{}}\PYG{n+nt}{tr}\PYG{p}{\PYGZgt{}}
					\PYG{p}{\PYGZlt{}}\PYG{n+nt}{td}\PYG{p}{\PYGZgt{}} Celda \PYG{p}{\PYGZlt{}}\PYG{p}{/}\PYG{n+nt}{td}\PYG{p}{\PYGZgt{}}
					\PYG{p}{\PYGZlt{}}\PYG{n+nt}{td}\PYG{p}{\PYGZgt{}} Celda \PYG{p}{\PYGZlt{}}\PYG{p}{/}\PYG{n+nt}{td}\PYG{p}{\PYGZgt{}}
				\PYG{p}{\PYGZlt{}}\PYG{p}{/}\PYG{n+nt}{tr}\PYG{p}{\PYGZgt{}}
				\PYG{p}{\PYGZlt{}}\PYG{n+nt}{tr}\PYG{p}{\PYGZgt{}}
					\PYG{p}{\PYGZlt{}}\PYG{n+nt}{td}\PYG{p}{\PYGZgt{}} Celda \PYG{p}{\PYGZlt{}}\PYG{p}{/}\PYG{n+nt}{td}\PYG{p}{\PYGZgt{}}
					\PYG{p}{\PYGZlt{}}\PYG{n+nt}{td}\PYG{p}{\PYGZgt{}} Celda \PYG{p}{\PYGZlt{}}\PYG{p}{/}\PYG{n+nt}{td}\PYG{p}{\PYGZgt{}}
				\PYG{p}{\PYGZlt{}}\PYG{p}{/}\PYG{n+nt}{tr}\PYG{p}{\PYGZgt{}}
				\PYG{p}{\PYGZlt{}}\PYG{n+nt}{tr}\PYG{p}{\PYGZgt{}}
					\PYG{p}{\PYGZlt{}}\PYG{n+nt}{td}\PYG{p}{\PYGZgt{}} Celda \PYG{p}{\PYGZlt{}}\PYG{p}{/}\PYG{n+nt}{td}\PYG{p}{\PYGZgt{}}
					\PYG{p}{\PYGZlt{}}\PYG{n+nt}{td}\PYG{p}{\PYGZgt{}} Celda \PYG{p}{\PYGZlt{}}\PYG{p}{/}\PYG{n+nt}{td}\PYG{p}{\PYGZgt{}}
				\PYG{p}{\PYGZlt{}}\PYG{p}{/}\PYG{n+nt}{tr}\PYG{p}{\PYGZgt{}}
				\PYG{p}{\PYGZlt{}}\PYG{n+nt}{tr}\PYG{p}{\PYGZgt{}}
					\PYG{p}{\PYGZlt{}}\PYG{n+nt}{td}\PYG{p}{\PYGZgt{}} Celda \PYG{p}{\PYGZlt{}}\PYG{p}{/}\PYG{n+nt}{td}\PYG{p}{\PYGZgt{}}
					\PYG{p}{\PYGZlt{}}\PYG{n+nt}{td}\PYG{p}{\PYGZgt{}} Celda \PYG{p}{\PYGZlt{}}\PYG{p}{/}\PYG{n+nt}{td}\PYG{p}{\PYGZgt{}}
				\PYG{p}{\PYGZlt{}}\PYG{p}{/}\PYG{n+nt}{tr}\PYG{p}{\PYGZgt{}}
			\PYG{p}{\PYGZlt{}}\PYG{p}{/}\PYG{n+nt}{table}\PYG{p}{\PYGZgt{}}
			\PYG{p}{\PYGZlt{}}\PYG{p}{/}\PYG{n+nt}{td}\PYG{p}{\PYGZgt{}}
		\PYG{p}{\PYGZlt{}}\PYG{p}{/}\PYG{n+nt}{tr}\PYG{p}{\PYGZgt{}}
		\PYG{p}{\PYGZlt{}}\PYG{n+nt}{tr}\PYG{p}{\PYGZgt{}}
			\PYG{p}{\PYGZlt{}}\PYG{n+nt}{td}\PYG{p}{\PYGZgt{}} Celda \PYG{p}{\PYGZlt{}}\PYG{p}{/}\PYG{n+nt}{td}\PYG{p}{\PYGZgt{}}
			\PYG{p}{\PYGZlt{}}\PYG{n+nt}{td}\PYG{p}{\PYGZgt{}} Celda \PYG{p}{\PYGZlt{}}\PYG{p}{/}\PYG{n+nt}{td}\PYG{p}{\PYGZgt{}}
			\PYG{p}{\PYGZlt{}}\PYG{n+nt}{td}\PYG{p}{\PYGZgt{}} Celda \PYG{p}{\PYGZlt{}}\PYG{p}{/}\PYG{n+nt}{td}\PYG{p}{\PYGZgt{}}
			\PYG{p}{\PYGZlt{}}\PYG{n+nt}{td}\PYG{p}{\PYGZgt{}} Celda \PYG{p}{\PYGZlt{}}\PYG{p}{/}\PYG{n+nt}{td}\PYG{p}{\PYGZgt{}}
		\PYG{p}{\PYGZlt{}}\PYG{p}{/}\PYG{n+nt}{tr}\PYG{p}{\PYGZgt{}}
	\PYG{p}{\PYGZlt{}}\PYG{p}{/}\PYG{n+nt}{table}\PYG{p}{\PYGZgt{}}
\PYG{p}{\PYGZlt{}}\PYG{p}{/}\PYG{n+nt}{body}\PYG{p}{\PYGZgt{}}
\PYG{p}{\PYGZlt{}}\PYG{p}{/}\PYG{n+nt}{html}\PYG{p}{\PYGZgt{}}
        
\end{sphinxVerbatim}


\section{Tabla 15}
\label{\detokenize{ejercicios/html/anexo_tablas:tabla-15}}
Generar la tabla siguiente

\noindent{\hspace*{\fill}\sphinxincludegraphics[scale=0.6]{{foto_15}.png}\hspace*{\fill}}

Solución:

\begin{sphinxVerbatim}[commandchars=\\\{\}]
\PYG{c+cp}{\PYGZlt{}!DOCTYPE html\PYGZgt{}}
\PYG{p}{\PYGZlt{}}\PYG{n+nt}{html}\PYG{p}{\PYGZgt{}}
\PYG{p}{\PYGZlt{}}\PYG{n+nt}{head}\PYG{p}{\PYGZgt{}}
\PYG{p}{\PYGZlt{}}\PYG{n+nt}{meta} \PYG{n+na}{charset}\PYG{o}{=}\PYG{l+s}{\PYGZsq{}utf\PYGZhy{}8\PYGZsq{}}\PYG{p}{\PYGZgt{}}
\PYG{p}{\PYGZlt{}}\PYG{n+nt}{title}\PYG{p}{\PYGZgt{}}Ejercicio\PYG{p}{\PYGZlt{}}\PYG{p}{/}\PYG{n+nt}{title}\PYG{p}{\PYGZgt{}}
\PYG{p}{\PYGZlt{}}\PYG{p}{/}\PYG{n+nt}{head}\PYG{p}{\PYGZgt{}}
\PYG{p}{\PYGZlt{}}\PYG{n+nt}{body}\PYG{p}{\PYGZgt{}}
	\PYG{p}{\PYGZlt{}}\PYG{n+nt}{table} \PYG{n+na}{border}\PYG{o}{=}\PYG{l+s}{\PYGZsq{}1\PYGZsq{}}\PYG{p}{\PYGZgt{}}
		\PYG{p}{\PYGZlt{}}\PYG{n+nt}{tr}\PYG{p}{\PYGZgt{}}
			\PYG{p}{\PYGZlt{}}\PYG{n+nt}{td}\PYG{p}{\PYGZgt{}} Celda \PYG{p}{\PYGZlt{}}\PYG{p}{/}\PYG{n+nt}{td}\PYG{p}{\PYGZgt{}}
			\PYG{p}{\PYGZlt{}}\PYG{n+nt}{td}\PYG{p}{\PYGZgt{}} Celda \PYG{p}{\PYGZlt{}}\PYG{p}{/}\PYG{n+nt}{td}\PYG{p}{\PYGZgt{}}
			\PYG{p}{\PYGZlt{}}\PYG{n+nt}{td}\PYG{p}{\PYGZgt{}} Celda \PYG{p}{\PYGZlt{}}\PYG{p}{/}\PYG{n+nt}{td}\PYG{p}{\PYGZgt{}}
		\PYG{p}{\PYGZlt{}}\PYG{p}{/}\PYG{n+nt}{tr}\PYG{p}{\PYGZgt{}}
		\PYG{p}{\PYGZlt{}}\PYG{n+nt}{tr}\PYG{p}{\PYGZgt{}}
			\PYG{p}{\PYGZlt{}}\PYG{n+nt}{td}\PYG{p}{\PYGZgt{}} Celda \PYG{p}{\PYGZlt{}}\PYG{p}{/}\PYG{n+nt}{td}\PYG{p}{\PYGZgt{}}
			\PYG{p}{\PYGZlt{}}\PYG{n+nt}{td}\PYG{p}{\PYGZgt{}} Celda \PYG{p}{\PYGZlt{}}\PYG{p}{/}\PYG{n+nt}{td}\PYG{p}{\PYGZgt{}}
			\PYG{p}{\PYGZlt{}}\PYG{n+nt}{td}\PYG{p}{\PYGZgt{}} Celda \PYG{p}{\PYGZlt{}}\PYG{p}{/}\PYG{n+nt}{td}\PYG{p}{\PYGZgt{}}
		\PYG{p}{\PYGZlt{}}\PYG{p}{/}\PYG{n+nt}{tr}\PYG{p}{\PYGZgt{}}
	\PYG{p}{\PYGZlt{}}\PYG{p}{/}\PYG{n+nt}{table}\PYG{p}{\PYGZgt{}}
\PYG{p}{\PYGZlt{}}\PYG{p}{/}\PYG{n+nt}{body}\PYG{p}{\PYGZgt{}}
\PYG{p}{\PYGZlt{}}\PYG{p}{/}\PYG{n+nt}{html}\PYG{p}{\PYGZgt{}}
        
\end{sphinxVerbatim}


\section{Tabla 16}
\label{\detokenize{ejercicios/html/anexo_tablas:tabla-16}}
Generar la tabla siguiente

\noindent{\hspace*{\fill}\sphinxincludegraphics[scale=0.6]{{foto_16}.png}\hspace*{\fill}}

Solución:

\begin{sphinxVerbatim}[commandchars=\\\{\}]
\PYG{c+cp}{\PYGZlt{}!DOCTYPE html\PYGZgt{}}
\PYG{p}{\PYGZlt{}}\PYG{n+nt}{html}\PYG{p}{\PYGZgt{}}
\PYG{p}{\PYGZlt{}}\PYG{n+nt}{head}\PYG{p}{\PYGZgt{}}
\PYG{p}{\PYGZlt{}}\PYG{n+nt}{meta} \PYG{n+na}{charset}\PYG{o}{=}\PYG{l+s}{\PYGZsq{}utf\PYGZhy{}8\PYGZsq{}}\PYG{p}{\PYGZgt{}}
\PYG{p}{\PYGZlt{}}\PYG{n+nt}{title}\PYG{p}{\PYGZgt{}}Ejercicio\PYG{p}{\PYGZlt{}}\PYG{p}{/}\PYG{n+nt}{title}\PYG{p}{\PYGZgt{}}
\PYG{p}{\PYGZlt{}}\PYG{p}{/}\PYG{n+nt}{head}\PYG{p}{\PYGZgt{}}
\PYG{p}{\PYGZlt{}}\PYG{n+nt}{body}\PYG{p}{\PYGZgt{}}
	\PYG{p}{\PYGZlt{}}\PYG{n+nt}{table} \PYG{n+na}{border}\PYG{o}{=}\PYG{l+s}{\PYGZsq{}1\PYGZsq{}}\PYG{p}{\PYGZgt{}}
		\PYG{p}{\PYGZlt{}}\PYG{n+nt}{tr}\PYG{p}{\PYGZgt{}}
			\PYG{p}{\PYGZlt{}}\PYG{n+nt}{td}\PYG{p}{\PYGZgt{}} Celda \PYG{p}{\PYGZlt{}}\PYG{p}{/}\PYG{n+nt}{td}\PYG{p}{\PYGZgt{}}
			\PYG{p}{\PYGZlt{}}\PYG{n+nt}{td}\PYG{p}{\PYGZgt{}} Celda \PYG{p}{\PYGZlt{}}\PYG{p}{/}\PYG{n+nt}{td}\PYG{p}{\PYGZgt{}}
		\PYG{p}{\PYGZlt{}}\PYG{p}{/}\PYG{n+nt}{tr}\PYG{p}{\PYGZgt{}}
		\PYG{p}{\PYGZlt{}}\PYG{n+nt}{tr}\PYG{p}{\PYGZgt{}}
			\PYG{p}{\PYGZlt{}}\PYG{n+nt}{td}\PYG{p}{\PYGZgt{}}
			\PYG{p}{\PYGZlt{}}\PYG{n+nt}{table} \PYG{n+na}{border}\PYG{o}{=}\PYG{l+s}{\PYGZsq{}1\PYGZsq{}}\PYG{p}{\PYGZgt{}}
				\PYG{p}{\PYGZlt{}}\PYG{n+nt}{tr}\PYG{p}{\PYGZgt{}}
					\PYG{p}{\PYGZlt{}}\PYG{n+nt}{td}\PYG{p}{\PYGZgt{}} Celda \PYG{p}{\PYGZlt{}}\PYG{p}{/}\PYG{n+nt}{td}\PYG{p}{\PYGZgt{}}
					\PYG{p}{\PYGZlt{}}\PYG{n+nt}{td}\PYG{p}{\PYGZgt{}} Celda \PYG{p}{\PYGZlt{}}\PYG{p}{/}\PYG{n+nt}{td}\PYG{p}{\PYGZgt{}}
					\PYG{p}{\PYGZlt{}}\PYG{n+nt}{td}\PYG{p}{\PYGZgt{}} Celda \PYG{p}{\PYGZlt{}}\PYG{p}{/}\PYG{n+nt}{td}\PYG{p}{\PYGZgt{}}
					\PYG{p}{\PYGZlt{}}\PYG{n+nt}{td}\PYG{p}{\PYGZgt{}} Celda \PYG{p}{\PYGZlt{}}\PYG{p}{/}\PYG{n+nt}{td}\PYG{p}{\PYGZgt{}}
				\PYG{p}{\PYGZlt{}}\PYG{p}{/}\PYG{n+nt}{tr}\PYG{p}{\PYGZgt{}}
				\PYG{p}{\PYGZlt{}}\PYG{n+nt}{tr}\PYG{p}{\PYGZgt{}}
					\PYG{p}{\PYGZlt{}}\PYG{n+nt}{td}\PYG{p}{\PYGZgt{}} Celda \PYG{p}{\PYGZlt{}}\PYG{p}{/}\PYG{n+nt}{td}\PYG{p}{\PYGZgt{}}
					\PYG{p}{\PYGZlt{}}\PYG{n+nt}{td}\PYG{p}{\PYGZgt{}} Celda \PYG{p}{\PYGZlt{}}\PYG{p}{/}\PYG{n+nt}{td}\PYG{p}{\PYGZgt{}}
					\PYG{p}{\PYGZlt{}}\PYG{n+nt}{td}\PYG{p}{\PYGZgt{}} Celda \PYG{p}{\PYGZlt{}}\PYG{p}{/}\PYG{n+nt}{td}\PYG{p}{\PYGZgt{}}
					\PYG{p}{\PYGZlt{}}\PYG{n+nt}{td}\PYG{p}{\PYGZgt{}} Celda \PYG{p}{\PYGZlt{}}\PYG{p}{/}\PYG{n+nt}{td}\PYG{p}{\PYGZgt{}}
				\PYG{p}{\PYGZlt{}}\PYG{p}{/}\PYG{n+nt}{tr}\PYG{p}{\PYGZgt{}}
				\PYG{p}{\PYGZlt{}}\PYG{n+nt}{tr}\PYG{p}{\PYGZgt{}}
					\PYG{p}{\PYGZlt{}}\PYG{n+nt}{td}\PYG{p}{\PYGZgt{}} Celda \PYG{p}{\PYGZlt{}}\PYG{p}{/}\PYG{n+nt}{td}\PYG{p}{\PYGZgt{}}
					\PYG{p}{\PYGZlt{}}\PYG{n+nt}{td}\PYG{p}{\PYGZgt{}} Celda \PYG{p}{\PYGZlt{}}\PYG{p}{/}\PYG{n+nt}{td}\PYG{p}{\PYGZgt{}}
					\PYG{p}{\PYGZlt{}}\PYG{n+nt}{td}\PYG{p}{\PYGZgt{}} Celda \PYG{p}{\PYGZlt{}}\PYG{p}{/}\PYG{n+nt}{td}\PYG{p}{\PYGZgt{}}
					\PYG{p}{\PYGZlt{}}\PYG{n+nt}{td}\PYG{p}{\PYGZgt{}} Celda \PYG{p}{\PYGZlt{}}\PYG{p}{/}\PYG{n+nt}{td}\PYG{p}{\PYGZgt{}}
				\PYG{p}{\PYGZlt{}}\PYG{p}{/}\PYG{n+nt}{tr}\PYG{p}{\PYGZgt{}}
			\PYG{p}{\PYGZlt{}}\PYG{p}{/}\PYG{n+nt}{table}\PYG{p}{\PYGZgt{}}
			\PYG{p}{\PYGZlt{}}\PYG{p}{/}\PYG{n+nt}{td}\PYG{p}{\PYGZgt{}}
			\PYG{p}{\PYGZlt{}}\PYG{n+nt}{td}\PYG{p}{\PYGZgt{}} Celda \PYG{p}{\PYGZlt{}}\PYG{p}{/}\PYG{n+nt}{td}\PYG{p}{\PYGZgt{}}
		\PYG{p}{\PYGZlt{}}\PYG{p}{/}\PYG{n+nt}{tr}\PYG{p}{\PYGZgt{}}
		\PYG{p}{\PYGZlt{}}\PYG{n+nt}{tr}\PYG{p}{\PYGZgt{}}
			\PYG{p}{\PYGZlt{}}\PYG{n+nt}{td}\PYG{p}{\PYGZgt{}} Celda \PYG{p}{\PYGZlt{}}\PYG{p}{/}\PYG{n+nt}{td}\PYG{p}{\PYGZgt{}}
			\PYG{p}{\PYGZlt{}}\PYG{n+nt}{td}\PYG{p}{\PYGZgt{}} Celda \PYG{p}{\PYGZlt{}}\PYG{p}{/}\PYG{n+nt}{td}\PYG{p}{\PYGZgt{}}
		\PYG{p}{\PYGZlt{}}\PYG{p}{/}\PYG{n+nt}{tr}\PYG{p}{\PYGZgt{}}
		\PYG{p}{\PYGZlt{}}\PYG{n+nt}{tr}\PYG{p}{\PYGZgt{}}
			\PYG{p}{\PYGZlt{}}\PYG{n+nt}{td}\PYG{p}{\PYGZgt{}} Celda \PYG{p}{\PYGZlt{}}\PYG{p}{/}\PYG{n+nt}{td}\PYG{p}{\PYGZgt{}}
			\PYG{p}{\PYGZlt{}}\PYG{n+nt}{td}\PYG{p}{\PYGZgt{}} Celda \PYG{p}{\PYGZlt{}}\PYG{p}{/}\PYG{n+nt}{td}\PYG{p}{\PYGZgt{}}
		\PYG{p}{\PYGZlt{}}\PYG{p}{/}\PYG{n+nt}{tr}\PYG{p}{\PYGZgt{}}
	\PYG{p}{\PYGZlt{}}\PYG{p}{/}\PYG{n+nt}{table}\PYG{p}{\PYGZgt{}}
\PYG{p}{\PYGZlt{}}\PYG{p}{/}\PYG{n+nt}{body}\PYG{p}{\PYGZgt{}}
\PYG{p}{\PYGZlt{}}\PYG{p}{/}\PYG{n+nt}{html}\PYG{p}{\PYGZgt{}}
        
\end{sphinxVerbatim}


\section{Tabla 17}
\label{\detokenize{ejercicios/html/anexo_tablas:tabla-17}}
Generar la tabla siguiente

\noindent{\hspace*{\fill}\sphinxincludegraphics[scale=0.6]{{foto_17}.png}\hspace*{\fill}}

Solución:

\begin{sphinxVerbatim}[commandchars=\\\{\}]
\PYG{c+cp}{\PYGZlt{}!DOCTYPE html\PYGZgt{}}
\PYG{p}{\PYGZlt{}}\PYG{n+nt}{html}\PYG{p}{\PYGZgt{}}
\PYG{p}{\PYGZlt{}}\PYG{n+nt}{head}\PYG{p}{\PYGZgt{}}
\PYG{p}{\PYGZlt{}}\PYG{n+nt}{meta} \PYG{n+na}{charset}\PYG{o}{=}\PYG{l+s}{\PYGZsq{}utf\PYGZhy{}8\PYGZsq{}}\PYG{p}{\PYGZgt{}}
\PYG{p}{\PYGZlt{}}\PYG{n+nt}{title}\PYG{p}{\PYGZgt{}}Ejercicio\PYG{p}{\PYGZlt{}}\PYG{p}{/}\PYG{n+nt}{title}\PYG{p}{\PYGZgt{}}
\PYG{p}{\PYGZlt{}}\PYG{p}{/}\PYG{n+nt}{head}\PYG{p}{\PYGZgt{}}
\PYG{p}{\PYGZlt{}}\PYG{n+nt}{body}\PYG{p}{\PYGZgt{}}
	\PYG{p}{\PYGZlt{}}\PYG{n+nt}{table} \PYG{n+na}{border}\PYG{o}{=}\PYG{l+s}{\PYGZsq{}1\PYGZsq{}}\PYG{p}{\PYGZgt{}}
		\PYG{p}{\PYGZlt{}}\PYG{n+nt}{tr}\PYG{p}{\PYGZgt{}}
			\PYG{p}{\PYGZlt{}}\PYG{n+nt}{td}\PYG{p}{\PYGZgt{}} Celda \PYG{p}{\PYGZlt{}}\PYG{p}{/}\PYG{n+nt}{td}\PYG{p}{\PYGZgt{}}
			\PYG{p}{\PYGZlt{}}\PYG{n+nt}{td}\PYG{p}{\PYGZgt{}}
			\PYG{p}{\PYGZlt{}}\PYG{n+nt}{table} \PYG{n+na}{border}\PYG{o}{=}\PYG{l+s}{\PYGZsq{}1\PYGZsq{}}\PYG{p}{\PYGZgt{}}
				\PYG{p}{\PYGZlt{}}\PYG{n+nt}{tr}\PYG{p}{\PYGZgt{}}
					\PYG{p}{\PYGZlt{}}\PYG{n+nt}{td}\PYG{p}{\PYGZgt{}} Celda \PYG{p}{\PYGZlt{}}\PYG{p}{/}\PYG{n+nt}{td}\PYG{p}{\PYGZgt{}}
					\PYG{p}{\PYGZlt{}}\PYG{n+nt}{td}\PYG{p}{\PYGZgt{}} Celda \PYG{p}{\PYGZlt{}}\PYG{p}{/}\PYG{n+nt}{td}\PYG{p}{\PYGZgt{}}
					\PYG{p}{\PYGZlt{}}\PYG{n+nt}{td}\PYG{p}{\PYGZgt{}} Celda \PYG{p}{\PYGZlt{}}\PYG{p}{/}\PYG{n+nt}{td}\PYG{p}{\PYGZgt{}}
				\PYG{p}{\PYGZlt{}}\PYG{p}{/}\PYG{n+nt}{tr}\PYG{p}{\PYGZgt{}}
				\PYG{p}{\PYGZlt{}}\PYG{n+nt}{tr}\PYG{p}{\PYGZgt{}}
					\PYG{p}{\PYGZlt{}}\PYG{n+nt}{td}\PYG{p}{\PYGZgt{}} Celda \PYG{p}{\PYGZlt{}}\PYG{p}{/}\PYG{n+nt}{td}\PYG{p}{\PYGZgt{}}
					\PYG{p}{\PYGZlt{}}\PYG{n+nt}{td}\PYG{p}{\PYGZgt{}} Celda \PYG{p}{\PYGZlt{}}\PYG{p}{/}\PYG{n+nt}{td}\PYG{p}{\PYGZgt{}}
					\PYG{p}{\PYGZlt{}}\PYG{n+nt}{td}\PYG{p}{\PYGZgt{}} Celda \PYG{p}{\PYGZlt{}}\PYG{p}{/}\PYG{n+nt}{td}\PYG{p}{\PYGZgt{}}
				\PYG{p}{\PYGZlt{}}\PYG{p}{/}\PYG{n+nt}{tr}\PYG{p}{\PYGZgt{}}
			\PYG{p}{\PYGZlt{}}\PYG{p}{/}\PYG{n+nt}{table}\PYG{p}{\PYGZgt{}}
			\PYG{p}{\PYGZlt{}}\PYG{p}{/}\PYG{n+nt}{td}\PYG{p}{\PYGZgt{}}
			\PYG{p}{\PYGZlt{}}\PYG{n+nt}{td}\PYG{p}{\PYGZgt{}} Celda \PYG{p}{\PYGZlt{}}\PYG{p}{/}\PYG{n+nt}{td}\PYG{p}{\PYGZgt{}}
			\PYG{p}{\PYGZlt{}}\PYG{n+nt}{td}\PYG{p}{\PYGZgt{}} Celda \PYG{p}{\PYGZlt{}}\PYG{p}{/}\PYG{n+nt}{td}\PYG{p}{\PYGZgt{}}
		\PYG{p}{\PYGZlt{}}\PYG{p}{/}\PYG{n+nt}{tr}\PYG{p}{\PYGZgt{}}
		\PYG{p}{\PYGZlt{}}\PYG{n+nt}{tr}\PYG{p}{\PYGZgt{}}
			\PYG{p}{\PYGZlt{}}\PYG{n+nt}{td}\PYG{p}{\PYGZgt{}} Celda \PYG{p}{\PYGZlt{}}\PYG{p}{/}\PYG{n+nt}{td}\PYG{p}{\PYGZgt{}}
			\PYG{p}{\PYGZlt{}}\PYG{n+nt}{td}\PYG{p}{\PYGZgt{}} Celda \PYG{p}{\PYGZlt{}}\PYG{p}{/}\PYG{n+nt}{td}\PYG{p}{\PYGZgt{}}
			\PYG{p}{\PYGZlt{}}\PYG{n+nt}{td}\PYG{p}{\PYGZgt{}} Celda \PYG{p}{\PYGZlt{}}\PYG{p}{/}\PYG{n+nt}{td}\PYG{p}{\PYGZgt{}}
			\PYG{p}{\PYGZlt{}}\PYG{n+nt}{td}\PYG{p}{\PYGZgt{}} Celda \PYG{p}{\PYGZlt{}}\PYG{p}{/}\PYG{n+nt}{td}\PYG{p}{\PYGZgt{}}
		\PYG{p}{\PYGZlt{}}\PYG{p}{/}\PYG{n+nt}{tr}\PYG{p}{\PYGZgt{}}
	\PYG{p}{\PYGZlt{}}\PYG{p}{/}\PYG{n+nt}{table}\PYG{p}{\PYGZgt{}}
\PYG{p}{\PYGZlt{}}\PYG{p}{/}\PYG{n+nt}{body}\PYG{p}{\PYGZgt{}}
\PYG{p}{\PYGZlt{}}\PYG{p}{/}\PYG{n+nt}{html}\PYG{p}{\PYGZgt{}}
        
\end{sphinxVerbatim}


\section{Tabla 18}
\label{\detokenize{ejercicios/html/anexo_tablas:tabla-18}}
Generar la tabla siguiente

\noindent{\hspace*{\fill}\sphinxincludegraphics[scale=0.6]{{foto_18}.png}\hspace*{\fill}}

Solución:

\begin{sphinxVerbatim}[commandchars=\\\{\}]
\PYG{c+cp}{\PYGZlt{}!DOCTYPE html\PYGZgt{}}
\PYG{p}{\PYGZlt{}}\PYG{n+nt}{html}\PYG{p}{\PYGZgt{}}
\PYG{p}{\PYGZlt{}}\PYG{n+nt}{head}\PYG{p}{\PYGZgt{}}
\PYG{p}{\PYGZlt{}}\PYG{n+nt}{meta} \PYG{n+na}{charset}\PYG{o}{=}\PYG{l+s}{\PYGZsq{}utf\PYGZhy{}8\PYGZsq{}}\PYG{p}{\PYGZgt{}}
\PYG{p}{\PYGZlt{}}\PYG{n+nt}{title}\PYG{p}{\PYGZgt{}}Ejercicio\PYG{p}{\PYGZlt{}}\PYG{p}{/}\PYG{n+nt}{title}\PYG{p}{\PYGZgt{}}
\PYG{p}{\PYGZlt{}}\PYG{p}{/}\PYG{n+nt}{head}\PYG{p}{\PYGZgt{}}
\PYG{p}{\PYGZlt{}}\PYG{n+nt}{body}\PYG{p}{\PYGZgt{}}
	\PYG{p}{\PYGZlt{}}\PYG{n+nt}{table} \PYG{n+na}{border}\PYG{o}{=}\PYG{l+s}{\PYGZsq{}1\PYGZsq{}}\PYG{p}{\PYGZgt{}}
		\PYG{p}{\PYGZlt{}}\PYG{n+nt}{tr}\PYG{p}{\PYGZgt{}}
			\PYG{p}{\PYGZlt{}}\PYG{n+nt}{td}\PYG{p}{\PYGZgt{}} Celda \PYG{p}{\PYGZlt{}}\PYG{p}{/}\PYG{n+nt}{td}\PYG{p}{\PYGZgt{}}
			\PYG{p}{\PYGZlt{}}\PYG{n+nt}{td}\PYG{p}{\PYGZgt{}} Celda \PYG{p}{\PYGZlt{}}\PYG{p}{/}\PYG{n+nt}{td}\PYG{p}{\PYGZgt{}}
			\PYG{p}{\PYGZlt{}}\PYG{n+nt}{td}\PYG{p}{\PYGZgt{}} Celda \PYG{p}{\PYGZlt{}}\PYG{p}{/}\PYG{n+nt}{td}\PYG{p}{\PYGZgt{}}
		\PYG{p}{\PYGZlt{}}\PYG{p}{/}\PYG{n+nt}{tr}\PYG{p}{\PYGZgt{}}
		\PYG{p}{\PYGZlt{}}\PYG{n+nt}{tr}\PYG{p}{\PYGZgt{}}
			\PYG{p}{\PYGZlt{}}\PYG{n+nt}{td}\PYG{p}{\PYGZgt{}} Celda \PYG{p}{\PYGZlt{}}\PYG{p}{/}\PYG{n+nt}{td}\PYG{p}{\PYGZgt{}}
			\PYG{p}{\PYGZlt{}}\PYG{n+nt}{td}\PYG{p}{\PYGZgt{}} Celda \PYG{p}{\PYGZlt{}}\PYG{p}{/}\PYG{n+nt}{td}\PYG{p}{\PYGZgt{}}
			\PYG{p}{\PYGZlt{}}\PYG{n+nt}{td}\PYG{p}{\PYGZgt{}} Celda \PYG{p}{\PYGZlt{}}\PYG{p}{/}\PYG{n+nt}{td}\PYG{p}{\PYGZgt{}}
		\PYG{p}{\PYGZlt{}}\PYG{p}{/}\PYG{n+nt}{tr}\PYG{p}{\PYGZgt{}}
		\PYG{p}{\PYGZlt{}}\PYG{n+nt}{tr}\PYG{p}{\PYGZgt{}}
			\PYG{p}{\PYGZlt{}}\PYG{n+nt}{td}\PYG{p}{\PYGZgt{}} Celda \PYG{p}{\PYGZlt{}}\PYG{p}{/}\PYG{n+nt}{td}\PYG{p}{\PYGZgt{}}
			\PYG{p}{\PYGZlt{}}\PYG{n+nt}{td}\PYG{p}{\PYGZgt{}} Celda \PYG{p}{\PYGZlt{}}\PYG{p}{/}\PYG{n+nt}{td}\PYG{p}{\PYGZgt{}}
			\PYG{p}{\PYGZlt{}}\PYG{n+nt}{td}\PYG{p}{\PYGZgt{}} Celda \PYG{p}{\PYGZlt{}}\PYG{p}{/}\PYG{n+nt}{td}\PYG{p}{\PYGZgt{}}
		\PYG{p}{\PYGZlt{}}\PYG{p}{/}\PYG{n+nt}{tr}\PYG{p}{\PYGZgt{}}
	\PYG{p}{\PYGZlt{}}\PYG{p}{/}\PYG{n+nt}{table}\PYG{p}{\PYGZgt{}}
\PYG{p}{\PYGZlt{}}\PYG{p}{/}\PYG{n+nt}{body}\PYG{p}{\PYGZgt{}}
\PYG{p}{\PYGZlt{}}\PYG{p}{/}\PYG{n+nt}{html}\PYG{p}{\PYGZgt{}}
        
\end{sphinxVerbatim}


\section{Tabla 19}
\label{\detokenize{ejercicios/html/anexo_tablas:tabla-19}}
Generar la tabla siguiente

\noindent{\hspace*{\fill}\sphinxincludegraphics[scale=0.6]{{foto_19}.png}\hspace*{\fill}}

Solución:

\begin{sphinxVerbatim}[commandchars=\\\{\}]
\PYG{c+cp}{\PYGZlt{}!DOCTYPE html\PYGZgt{}}
\PYG{p}{\PYGZlt{}}\PYG{n+nt}{html}\PYG{p}{\PYGZgt{}}
\PYG{p}{\PYGZlt{}}\PYG{n+nt}{head}\PYG{p}{\PYGZgt{}}
\PYG{p}{\PYGZlt{}}\PYG{n+nt}{meta} \PYG{n+na}{charset}\PYG{o}{=}\PYG{l+s}{\PYGZsq{}utf\PYGZhy{}8\PYGZsq{}}\PYG{p}{\PYGZgt{}}
\PYG{p}{\PYGZlt{}}\PYG{n+nt}{title}\PYG{p}{\PYGZgt{}}Ejercicio\PYG{p}{\PYGZlt{}}\PYG{p}{/}\PYG{n+nt}{title}\PYG{p}{\PYGZgt{}}
\PYG{p}{\PYGZlt{}}\PYG{p}{/}\PYG{n+nt}{head}\PYG{p}{\PYGZgt{}}
\PYG{p}{\PYGZlt{}}\PYG{n+nt}{body}\PYG{p}{\PYGZgt{}}
	\PYG{p}{\PYGZlt{}}\PYG{n+nt}{table} \PYG{n+na}{border}\PYG{o}{=}\PYG{l+s}{\PYGZsq{}1\PYGZsq{}}\PYG{p}{\PYGZgt{}}
		\PYG{p}{\PYGZlt{}}\PYG{n+nt}{tr}\PYG{p}{\PYGZgt{}}
			\PYG{p}{\PYGZlt{}}\PYG{n+nt}{td}\PYG{p}{\PYGZgt{}} Celda \PYG{p}{\PYGZlt{}}\PYG{p}{/}\PYG{n+nt}{td}\PYG{p}{\PYGZgt{}}
			\PYG{p}{\PYGZlt{}}\PYG{n+nt}{td}\PYG{p}{\PYGZgt{}} Celda \PYG{p}{\PYGZlt{}}\PYG{p}{/}\PYG{n+nt}{td}\PYG{p}{\PYGZgt{}}
		\PYG{p}{\PYGZlt{}}\PYG{p}{/}\PYG{n+nt}{tr}\PYG{p}{\PYGZgt{}}
		\PYG{p}{\PYGZlt{}}\PYG{n+nt}{tr}\PYG{p}{\PYGZgt{}}
			\PYG{p}{\PYGZlt{}}\PYG{n+nt}{td}\PYG{p}{\PYGZgt{}} Celda \PYG{p}{\PYGZlt{}}\PYG{p}{/}\PYG{n+nt}{td}\PYG{p}{\PYGZgt{}}
			\PYG{p}{\PYGZlt{}}\PYG{n+nt}{td}\PYG{p}{\PYGZgt{}} Celda \PYG{p}{\PYGZlt{}}\PYG{p}{/}\PYG{n+nt}{td}\PYG{p}{\PYGZgt{}}
		\PYG{p}{\PYGZlt{}}\PYG{p}{/}\PYG{n+nt}{tr}\PYG{p}{\PYGZgt{}}
		\PYG{p}{\PYGZlt{}}\PYG{n+nt}{tr}\PYG{p}{\PYGZgt{}}
			\PYG{p}{\PYGZlt{}}\PYG{n+nt}{td}\PYG{p}{\PYGZgt{}} Celda \PYG{p}{\PYGZlt{}}\PYG{p}{/}\PYG{n+nt}{td}\PYG{p}{\PYGZgt{}}
			\PYG{p}{\PYGZlt{}}\PYG{n+nt}{td}\PYG{p}{\PYGZgt{}} Celda \PYG{p}{\PYGZlt{}}\PYG{p}{/}\PYG{n+nt}{td}\PYG{p}{\PYGZgt{}}
		\PYG{p}{\PYGZlt{}}\PYG{p}{/}\PYG{n+nt}{tr}\PYG{p}{\PYGZgt{}}
	\PYG{p}{\PYGZlt{}}\PYG{p}{/}\PYG{n+nt}{table}\PYG{p}{\PYGZgt{}}
\PYG{p}{\PYGZlt{}}\PYG{p}{/}\PYG{n+nt}{body}\PYG{p}{\PYGZgt{}}
\PYG{p}{\PYGZlt{}}\PYG{p}{/}\PYG{n+nt}{html}\PYG{p}{\PYGZgt{}}
        
\end{sphinxVerbatim}


\chapter{Anexo: ejercicios sobre formularios}
\label{\detokenize{ejercicios/formularios/anexo_formularios::doc}}\label{\detokenize{ejercicios/formularios/anexo_formularios:anexo-ejercicios-sobre-formularios}}
En los ejercicios siguientes se han muestra el diseño básico de algunos formularios junto con el HTML que los resuelve.


\section{Formulario 1}
\label{\detokenize{ejercicios/formularios/anexo_formularios:formulario-1}}
Generar el formulario siguiente de acuerdo a los siguientes requisitos
\begin{itemize}
\item {} 
Contiene los siguientes \sphinxcode{checkboxes}:checkbox con el \sphinxcode{name}  «escritorio» , \sphinxcode{value}  «escritoriokde»  y el texto «KDE», checkbox con el \sphinxcode{name}  «escritorio» , \sphinxcode{value}  «escritoriognome»  y el texto «GNOME», checkbox con el \sphinxcode{name}  «escritorio» , \sphinxcode{value}  «escritoriounity»  y el texto «Unity».

\end{itemize}

\noindent{\hspace*{\fill}\sphinxincludegraphics[scale=0.6]{{foto_formulario_01}.png}\hspace*{\fill}}

Solución:

\begin{sphinxVerbatim}[commandchars=\\\{\}]
\PYG{p}{\PYGZlt{}}\PYG{n+nt}{form}\PYG{p}{\PYGZgt{}}
\PYG{p}{\PYGZlt{}}\PYG{n+nt}{fieldset}\PYG{p}{\PYGZgt{}}
  \PYG{p}{\PYGZlt{}}\PYG{n+nt}{legend}\PYG{p}{\PYGZgt{}}Complete, por favor\PYG{p}{\PYGZlt{}}\PYG{p}{/}\PYG{n+nt}{legend}\PYG{p}{\PYGZgt{}}
  \PYG{p}{\PYGZlt{}}\PYG{n+nt}{input} \PYG{n+na}{type}\PYG{o}{=}\PYG{l+s}{\PYGZsq{}checkbox\PYGZsq{}} \PYG{n+na}{name}\PYG{o}{=}\PYG{l+s}{\PYGZsq{}escritorio\PYGZsq{}} \PYG{n+na}{value}\PYG{o}{=}\PYG{l+s}{\PYGZsq{}escritoriokde\PYGZsq{}}\PYG{p}{\PYGZgt{}} KDE   \PYG{p}{\PYGZlt{}}\PYG{n+nt}{br}\PYG{p}{/}\PYG{p}{\PYGZgt{}}
  \PYG{p}{\PYGZlt{}}\PYG{n+nt}{input} \PYG{n+na}{type}\PYG{o}{=}\PYG{l+s}{\PYGZsq{}checkbox\PYGZsq{}} \PYG{n+na}{name}\PYG{o}{=}\PYG{l+s}{\PYGZsq{}escritorio\PYGZsq{}} \PYG{n+na}{value}\PYG{o}{=}\PYG{l+s}{\PYGZsq{}escritoriognome\PYGZsq{}}\PYG{p}{\PYGZgt{}} GNOME   \PYG{p}{\PYGZlt{}}\PYG{n+nt}{br}\PYG{p}{/}\PYG{p}{\PYGZgt{}}
  \PYG{p}{\PYGZlt{}}\PYG{n+nt}{input} \PYG{n+na}{type}\PYG{o}{=}\PYG{l+s}{\PYGZsq{}checkbox\PYGZsq{}} \PYG{n+na}{name}\PYG{o}{=}\PYG{l+s}{\PYGZsq{}escritorio\PYGZsq{}} \PYG{n+na}{value}\PYG{o}{=}\PYG{l+s}{\PYGZsq{}escritoriounity\PYGZsq{}}\PYG{p}{\PYGZgt{}} Unity   \PYG{p}{\PYGZlt{}}\PYG{n+nt}{br}\PYG{p}{/}\PYG{p}{\PYGZgt{}}
  \PYG{p}{\PYGZlt{}}\PYG{n+nt}{br}\PYG{p}{/}\PYG{p}{\PYGZgt{}}
\PYG{p}{\PYGZlt{}}\PYG{p}{/}\PYG{n+nt}{fieldset}\PYG{p}{\PYGZgt{}}
\PYG{p}{\PYGZlt{}}\PYG{p}{/}\PYG{n+nt}{form}\PYG{p}{\PYGZgt{}}
\end{sphinxVerbatim}


\section{Formulario 2}
\label{\detokenize{ejercicios/formularios/anexo_formularios:formulario-2}}
Generar el formulario siguiente de acuerdo a los siguientes requisitos
\begin{itemize}
\item {} 
Hay los siguientes cuadros de texto:cuadro de texto con el texto «Instituto» y el \sphinxcode{name} instituto, cuadro de texto con el texto «Estudios elegidos» y el \sphinxcode{name} estudios

\item {} 
Hay una lista desplegable múltiple con el \sphinxcode{name} «sexo» y con las siguientes opciones: opción «Mujer» con el \sphinxcode{value} mujer, opción «Hombre» con el \sphinxcode{value} hombre.

\end{itemize}

\noindent{\hspace*{\fill}\sphinxincludegraphics[scale=0.6]{{foto_formulario_02}.png}\hspace*{\fill}}

Solución:

\begin{sphinxVerbatim}[commandchars=\\\{\}]
\PYG{p}{\PYGZlt{}}\PYG{n+nt}{form}\PYG{p}{\PYGZgt{}}
\PYG{p}{\PYGZlt{}}\PYG{n+nt}{fieldset}\PYG{p}{\PYGZgt{}}
  \PYG{p}{\PYGZlt{}}\PYG{n+nt}{legend}\PYG{p}{\PYGZgt{}}Indique\PYG{p}{\PYGZlt{}}\PYG{p}{/}\PYG{n+nt}{legend}\PYG{p}{\PYGZgt{}}
  Instituto\PYG{p}{\PYGZlt{}}\PYG{n+nt}{input} \PYG{n+na}{type}\PYG{o}{=}\PYG{l+s}{\PYGZsq{}text\PYGZsq{}} \PYG{n+na}{name}\PYG{o}{=}\PYG{l+s}{\PYGZsq{}instituto\PYGZsq{}}\PYG{p}{\PYGZgt{}}
  Estudios elegidos\PYG{p}{\PYGZlt{}}\PYG{n+nt}{input} \PYG{n+na}{type}\PYG{o}{=}\PYG{l+s}{\PYGZsq{}text\PYGZsq{}} \PYG{n+na}{name}\PYG{o}{=}\PYG{l+s}{\PYGZsq{}estudios\PYGZsq{}}\PYG{p}{\PYGZgt{}}
  \PYG{p}{\PYGZlt{}}\PYG{n+nt}{br}\PYG{p}{/}\PYG{p}{\PYGZgt{}}
  \PYG{p}{\PYGZlt{}}\PYG{n+nt}{select} \PYG{n+na}{name}\PYG{o}{=}\PYG{l+s}{\PYGZsq{}sexo\PYGZsq{}} \PYG{n+na}{multiple}\PYG{o}{=}\PYG{l+s}{\PYGZsq{}multiple\PYGZsq{}}\PYG{p}{\PYGZgt{}}
    \PYG{p}{\PYGZlt{}}\PYG{n+nt}{option} \PYG{n+na}{value}\PYG{o}{=}\PYG{l+s}{\PYGZsq{}mujer\PYGZsq{}}\PYG{p}{\PYGZgt{}}Mujer\PYG{p}{\PYGZlt{}}\PYG{p}{/}\PYG{n+nt}{option}\PYG{p}{\PYGZgt{}}
    \PYG{p}{\PYGZlt{}}\PYG{n+nt}{option} \PYG{n+na}{value}\PYG{o}{=}\PYG{l+s}{\PYGZsq{}hombre\PYGZsq{}}\PYG{p}{\PYGZgt{}}Hombre\PYG{p}{\PYGZlt{}}\PYG{p}{/}\PYG{n+nt}{option}\PYG{p}{\PYGZgt{}}
  \PYG{p}{\PYGZlt{}}\PYG{p}{/}\PYG{n+nt}{select}\PYG{p}{\PYGZgt{}}
  \PYG{p}{\PYGZlt{}}\PYG{n+nt}{br}\PYG{p}{/}\PYG{p}{\PYGZgt{}}
\PYG{p}{\PYGZlt{}}\PYG{p}{/}\PYG{n+nt}{fieldset}\PYG{p}{\PYGZgt{}}
\PYG{p}{\PYGZlt{}}\PYG{p}{/}\PYG{n+nt}{form}\PYG{p}{\PYGZgt{}}
\end{sphinxVerbatim}


\section{Formulario 3}
\label{\detokenize{ejercicios/formularios/anexo_formularios:formulario-3}}
Generar el formulario siguiente de acuerdo a los siguientes requisitos
\begin{itemize}
\item {} 
Contiene los siguientes \sphinxcode{checkboxes}:checkbox con el \sphinxcode{name}  «procesador» , \sphinxcode{value}  «procesadorintel»  y el texto «Intel», checkbox con el \sphinxcode{name}  «procesador» , \sphinxcode{value}  «procesadoramd»  y el texto «AMD».

\end{itemize}

\noindent{\hspace*{\fill}\sphinxincludegraphics[scale=0.6]{{foto_formulario_03}.png}\hspace*{\fill}}

Solución:

\begin{sphinxVerbatim}[commandchars=\\\{\}]
\PYG{p}{\PYGZlt{}}\PYG{n+nt}{form}\PYG{p}{\PYGZgt{}}
\PYG{p}{\PYGZlt{}}\PYG{n+nt}{fieldset}\PYG{p}{\PYGZgt{}}
  \PYG{p}{\PYGZlt{}}\PYG{n+nt}{legend}\PYG{p}{\PYGZgt{}}Opciones\PYG{p}{\PYGZlt{}}\PYG{p}{/}\PYG{n+nt}{legend}\PYG{p}{\PYGZgt{}}
  \PYG{p}{\PYGZlt{}}\PYG{n+nt}{input} \PYG{n+na}{type}\PYG{o}{=}\PYG{l+s}{\PYGZsq{}checkbox\PYGZsq{}} \PYG{n+na}{name}\PYG{o}{=}\PYG{l+s}{\PYGZsq{}procesador\PYGZsq{}} \PYG{n+na}{value}\PYG{o}{=}\PYG{l+s}{\PYGZsq{}procesadorintel\PYGZsq{}}\PYG{p}{\PYGZgt{}} Intel   \PYG{p}{\PYGZlt{}}\PYG{n+nt}{br}\PYG{p}{/}\PYG{p}{\PYGZgt{}}
  \PYG{p}{\PYGZlt{}}\PYG{n+nt}{input} \PYG{n+na}{type}\PYG{o}{=}\PYG{l+s}{\PYGZsq{}checkbox\PYGZsq{}} \PYG{n+na}{name}\PYG{o}{=}\PYG{l+s}{\PYGZsq{}procesador\PYGZsq{}} \PYG{n+na}{value}\PYG{o}{=}\PYG{l+s}{\PYGZsq{}procesadoramd\PYGZsq{}}\PYG{p}{\PYGZgt{}} AMD   \PYG{p}{\PYGZlt{}}\PYG{n+nt}{br}\PYG{p}{/}\PYG{p}{\PYGZgt{}}
  \PYG{p}{\PYGZlt{}}\PYG{n+nt}{br}\PYG{p}{/}\PYG{p}{\PYGZgt{}}
\PYG{p}{\PYGZlt{}}\PYG{p}{/}\PYG{n+nt}{fieldset}\PYG{p}{\PYGZgt{}}
\PYG{p}{\PYGZlt{}}\PYG{p}{/}\PYG{n+nt}{form}\PYG{p}{\PYGZgt{}}
\end{sphinxVerbatim}


\section{Formulario 4}
\label{\detokenize{ejercicios/formularios/anexo_formularios:formulario-4}}
Generar el formulario siguiente de acuerdo a los siguientes requisitos
\begin{itemize}
\item {} 
Hay los siguientes cuadros de texto:cuadro de texto con el texto «Nombre» y el \sphinxcode{name} nombre, cuadro de texto con el texto «Apellidos» y el \sphinxcode{name} apellidos

\end{itemize}

\noindent{\hspace*{\fill}\sphinxincludegraphics[scale=0.6]{{foto_formulario_04}.png}\hspace*{\fill}}

Solución:

\begin{sphinxVerbatim}[commandchars=\\\{\}]
\PYG{p}{\PYGZlt{}}\PYG{n+nt}{form}\PYG{p}{\PYGZgt{}}
\PYG{p}{\PYGZlt{}}\PYG{n+nt}{fieldset}\PYG{p}{\PYGZgt{}}
  \PYG{p}{\PYGZlt{}}\PYG{n+nt}{legend}\PYG{p}{\PYGZgt{}}Complete, por favor\PYG{p}{\PYGZlt{}}\PYG{p}{/}\PYG{n+nt}{legend}\PYG{p}{\PYGZgt{}}
  Nombre\PYG{p}{\PYGZlt{}}\PYG{n+nt}{input} \PYG{n+na}{type}\PYG{o}{=}\PYG{l+s}{\PYGZsq{}text\PYGZsq{}} \PYG{n+na}{name}\PYG{o}{=}\PYG{l+s}{\PYGZsq{}nombre\PYGZsq{}}\PYG{p}{\PYGZgt{}}
  Apellidos\PYG{p}{\PYGZlt{}}\PYG{n+nt}{input} \PYG{n+na}{type}\PYG{o}{=}\PYG{l+s}{\PYGZsq{}text\PYGZsq{}} \PYG{n+na}{name}\PYG{o}{=}\PYG{l+s}{\PYGZsq{}apellidos\PYGZsq{}}\PYG{p}{\PYGZgt{}}
  \PYG{p}{\PYGZlt{}}\PYG{n+nt}{br}\PYG{p}{/}\PYG{p}{\PYGZgt{}}
\PYG{p}{\PYGZlt{}}\PYG{p}{/}\PYG{n+nt}{fieldset}\PYG{p}{\PYGZgt{}}
\PYG{p}{\PYGZlt{}}\PYG{p}{/}\PYG{n+nt}{form}\PYG{p}{\PYGZgt{}}
\end{sphinxVerbatim}


\section{Formulario 5}
\label{\detokenize{ejercicios/formularios/anexo_formularios:formulario-5}}
Generar el formulario siguiente de acuerdo a los siguientes requisitos
\begin{itemize}
\item {} 
Hay un \sphinxcode{textarea} que mide 6 filas y 54 columnas que lleva dentro el texto «Inserte aqui el texto»

\item {} 
Contiene los siguientes \sphinxcode{checkboxes}:checkbox con el \sphinxcode{name}  «idioma» , \sphinxcode{value}  «idiomaespanol»  y el texto «Español», checkbox con el \sphinxcode{name}  «idioma» , \sphinxcode{value}  «idiomaingles»  y el texto «Inglés», checkbox con el \sphinxcode{name}  «idioma» , \sphinxcode{value}  «idiomaaleman»  y el texto «Alemán», checkbox con el \sphinxcode{name}  «idioma» , \sphinxcode{value}  «idiomafrances»  y el texto «Francés».

\item {} 
Hay un \sphinxcode{textarea} que mide 7 filas y 55 columnas que lleva dentro el texto «Escriba aquí, por favor»

\item {} 
Contiene los siguientes \sphinxcode{checkboxes}:checkbox con el \sphinxcode{name}  «ciclo» , \sphinxcode{value}  «ciclosmir»  y el texto «SMIR», checkbox con el \sphinxcode{name}  «ciclo» , \sphinxcode{value}  «cicloasir»  y el texto «ASIR», checkbox con el \sphinxcode{name}  «ciclo» , \sphinxcode{value}  «ciclodam»  y el texto «DAM», checkbox con el \sphinxcode{name}  «ciclo» , \sphinxcode{value}  «ciclodaw»  y el texto «DAW».

\item {} 
Hay los siguientes cuadros de texto:cuadro de texto con el texto «Nombre» y el \sphinxcode{name} nombre, cuadro de texto con el texto «Apellidos» y el \sphinxcode{name} apellidos, cuadro de texto con el texto «Direccion» y el \sphinxcode{name} direccion

\end{itemize}

\noindent{\hspace*{\fill}\sphinxincludegraphics[scale=0.6]{{foto_formulario_05}.png}\hspace*{\fill}}

Solución:

\begin{sphinxVerbatim}[commandchars=\\\{\}]
\PYG{p}{\PYGZlt{}}\PYG{n+nt}{form}\PYG{p}{\PYGZgt{}}
\PYG{p}{\PYGZlt{}}\PYG{n+nt}{fieldset}\PYG{p}{\PYGZgt{}}
  \PYG{p}{\PYGZlt{}}\PYG{n+nt}{legend}\PYG{p}{\PYGZgt{}}Completar estas opciones\PYG{p}{\PYGZlt{}}\PYG{p}{/}\PYG{n+nt}{legend}\PYG{p}{\PYGZgt{}}
  \PYG{p}{\PYGZlt{}}\PYG{n+nt}{textarea} \PYG{n+na}{rows}\PYG{o}{=}\PYG{l+s}{\PYGZsq{}6\PYGZsq{}} \PYG{n+na}{cols}\PYG{o}{=}\PYG{l+s}{\PYGZsq{}54\PYGZsq{}}\PYG{p}{\PYGZgt{}}
    Inserte aqui el texto
  \PYG{p}{\PYGZlt{}}\PYG{p}{/}\PYG{n+nt}{textarea}\PYG{p}{\PYGZgt{}}  \PYG{p}{\PYGZlt{}}\PYG{n+nt}{br}\PYG{p}{/}\PYG{p}{\PYGZgt{}}
  \PYG{p}{\PYGZlt{}}\PYG{n+nt}{input} \PYG{n+na}{type}\PYG{o}{=}\PYG{l+s}{\PYGZsq{}checkbox\PYGZsq{}} \PYG{n+na}{name}\PYG{o}{=}\PYG{l+s}{\PYGZsq{}idioma\PYGZsq{}} \PYG{n+na}{value}\PYG{o}{=}\PYG{l+s}{\PYGZsq{}idiomaespanol\PYGZsq{}}\PYG{p}{\PYGZgt{}} Español   \PYG{p}{\PYGZlt{}}\PYG{n+nt}{br}\PYG{p}{/}\PYG{p}{\PYGZgt{}}
  \PYG{p}{\PYGZlt{}}\PYG{n+nt}{input} \PYG{n+na}{type}\PYG{o}{=}\PYG{l+s}{\PYGZsq{}checkbox\PYGZsq{}} \PYG{n+na}{name}\PYG{o}{=}\PYG{l+s}{\PYGZsq{}idioma\PYGZsq{}} \PYG{n+na}{value}\PYG{o}{=}\PYG{l+s}{\PYGZsq{}idiomaingles\PYGZsq{}}\PYG{p}{\PYGZgt{}} Inglés   \PYG{p}{\PYGZlt{}}\PYG{n+nt}{br}\PYG{p}{/}\PYG{p}{\PYGZgt{}}
  \PYG{p}{\PYGZlt{}}\PYG{n+nt}{input} \PYG{n+na}{type}\PYG{o}{=}\PYG{l+s}{\PYGZsq{}checkbox\PYGZsq{}} \PYG{n+na}{name}\PYG{o}{=}\PYG{l+s}{\PYGZsq{}idioma\PYGZsq{}} \PYG{n+na}{value}\PYG{o}{=}\PYG{l+s}{\PYGZsq{}idiomaaleman\PYGZsq{}}\PYG{p}{\PYGZgt{}} Alemán   \PYG{p}{\PYGZlt{}}\PYG{n+nt}{br}\PYG{p}{/}\PYG{p}{\PYGZgt{}}
  \PYG{p}{\PYGZlt{}}\PYG{n+nt}{input} \PYG{n+na}{type}\PYG{o}{=}\PYG{l+s}{\PYGZsq{}checkbox\PYGZsq{}} \PYG{n+na}{name}\PYG{o}{=}\PYG{l+s}{\PYGZsq{}idioma\PYGZsq{}} \PYG{n+na}{value}\PYG{o}{=}\PYG{l+s}{\PYGZsq{}idiomafrances\PYGZsq{}}\PYG{p}{\PYGZgt{}} Francés   \PYG{p}{\PYGZlt{}}\PYG{n+nt}{br}\PYG{p}{/}\PYG{p}{\PYGZgt{}}
  \PYG{p}{\PYGZlt{}}\PYG{n+nt}{br}\PYG{p}{/}\PYG{p}{\PYGZgt{}}
  \PYG{p}{\PYGZlt{}}\PYG{n+nt}{textarea} \PYG{n+na}{rows}\PYG{o}{=}\PYG{l+s}{\PYGZsq{}7\PYGZsq{}} \PYG{n+na}{cols}\PYG{o}{=}\PYG{l+s}{\PYGZsq{}55\PYGZsq{}}\PYG{p}{\PYGZgt{}}
    Escriba aquí, por favor
  \PYG{p}{\PYGZlt{}}\PYG{p}{/}\PYG{n+nt}{textarea}\PYG{p}{\PYGZgt{}}  \PYG{p}{\PYGZlt{}}\PYG{n+nt}{br}\PYG{p}{/}\PYG{p}{\PYGZgt{}}
\PYG{p}{\PYGZlt{}}\PYG{p}{/}\PYG{n+nt}{fieldset}\PYG{p}{\PYGZgt{}}
\PYG{p}{\PYGZlt{}}\PYG{n+nt}{fieldset}\PYG{p}{\PYGZgt{}}
  \PYG{p}{\PYGZlt{}}\PYG{n+nt}{legend}\PYG{p}{\PYGZgt{}}Rellene las opciones siguientes\PYG{p}{\PYGZlt{}}\PYG{p}{/}\PYG{n+nt}{legend}\PYG{p}{\PYGZgt{}}
  \PYG{p}{\PYGZlt{}}\PYG{n+nt}{input} \PYG{n+na}{type}\PYG{o}{=}\PYG{l+s}{\PYGZsq{}checkbox\PYGZsq{}} \PYG{n+na}{name}\PYG{o}{=}\PYG{l+s}{\PYGZsq{}ciclo\PYGZsq{}} \PYG{n+na}{value}\PYG{o}{=}\PYG{l+s}{\PYGZsq{}ciclosmir\PYGZsq{}}\PYG{p}{\PYGZgt{}} SMIR 
  \PYG{p}{\PYGZlt{}}\PYG{n+nt}{input} \PYG{n+na}{type}\PYG{o}{=}\PYG{l+s}{\PYGZsq{}checkbox\PYGZsq{}} \PYG{n+na}{name}\PYG{o}{=}\PYG{l+s}{\PYGZsq{}ciclo\PYGZsq{}} \PYG{n+na}{value}\PYG{o}{=}\PYG{l+s}{\PYGZsq{}cicloasir\PYGZsq{}}\PYG{p}{\PYGZgt{}} ASIR 
  \PYG{p}{\PYGZlt{}}\PYG{n+nt}{input} \PYG{n+na}{type}\PYG{o}{=}\PYG{l+s}{\PYGZsq{}checkbox\PYGZsq{}} \PYG{n+na}{name}\PYG{o}{=}\PYG{l+s}{\PYGZsq{}ciclo\PYGZsq{}} \PYG{n+na}{value}\PYG{o}{=}\PYG{l+s}{\PYGZsq{}ciclodam\PYGZsq{}}\PYG{p}{\PYGZgt{}} DAM 
  \PYG{p}{\PYGZlt{}}\PYG{n+nt}{input} \PYG{n+na}{type}\PYG{o}{=}\PYG{l+s}{\PYGZsq{}checkbox\PYGZsq{}} \PYG{n+na}{name}\PYG{o}{=}\PYG{l+s}{\PYGZsq{}ciclo\PYGZsq{}} \PYG{n+na}{value}\PYG{o}{=}\PYG{l+s}{\PYGZsq{}ciclodaw\PYGZsq{}}\PYG{p}{\PYGZgt{}} DAW 
  \PYG{p}{\PYGZlt{}}\PYG{n+nt}{br}\PYG{p}{/}\PYG{p}{\PYGZgt{}}
  Nombre\PYG{p}{\PYGZlt{}}\PYG{n+nt}{input} \PYG{n+na}{type}\PYG{o}{=}\PYG{l+s}{\PYGZsq{}text\PYGZsq{}} \PYG{n+na}{name}\PYG{o}{=}\PYG{l+s}{\PYGZsq{}nombre\PYGZsq{}}\PYG{p}{\PYGZgt{}} \PYG{p}{\PYGZlt{}}\PYG{n+nt}{br}\PYG{p}{/}\PYG{p}{\PYGZgt{}}
  Apellidos\PYG{p}{\PYGZlt{}}\PYG{n+nt}{input} \PYG{n+na}{type}\PYG{o}{=}\PYG{l+s}{\PYGZsq{}text\PYGZsq{}} \PYG{n+na}{name}\PYG{o}{=}\PYG{l+s}{\PYGZsq{}apellidos\PYGZsq{}}\PYG{p}{\PYGZgt{}} \PYG{p}{\PYGZlt{}}\PYG{n+nt}{br}\PYG{p}{/}\PYG{p}{\PYGZgt{}}
  Direccion\PYG{p}{\PYGZlt{}}\PYG{n+nt}{input} \PYG{n+na}{type}\PYG{o}{=}\PYG{l+s}{\PYGZsq{}text\PYGZsq{}} \PYG{n+na}{name}\PYG{o}{=}\PYG{l+s}{\PYGZsq{}direccion\PYGZsq{}}\PYG{p}{\PYGZgt{}} \PYG{p}{\PYGZlt{}}\PYG{n+nt}{br}\PYG{p}{/}\PYG{p}{\PYGZgt{}}
  \PYG{p}{\PYGZlt{}}\PYG{n+nt}{br}\PYG{p}{/}\PYG{p}{\PYGZgt{}}
\PYG{p}{\PYGZlt{}}\PYG{p}{/}\PYG{n+nt}{fieldset}\PYG{p}{\PYGZgt{}}
\PYG{p}{\PYGZlt{}}\PYG{p}{/}\PYG{n+nt}{form}\PYG{p}{\PYGZgt{}}
\end{sphinxVerbatim}


\section{Formulario 6}
\label{\detokenize{ejercicios/formularios/anexo_formularios:formulario-6}}
Generar el formulario siguiente de acuerdo a los siguientes requisitos
\begin{itemize}
\item {} 
Hay un \sphinxcode{textarea} que mide 4 filas y 48 columnas que lleva dentro el texto «Escriba aquí, por favor»

\item {} 
Contiene los siguientes \sphinxcode{radiobuttons}:radio con el \sphinxcode{name}  «red» , \sphinxcode{value}  «red2g»  y el texto «2G», radio con el \sphinxcode{name}  «red» , \sphinxcode{value}  «red3g»  y el texto «3G», radio con el \sphinxcode{name}  «red» , \sphinxcode{value}  «red4g»  y el texto «4G».

\item {} 
Hay los siguientes cuadros de texto:cuadro de texto con el texto «Nombre» y el \sphinxcode{name} nombre, cuadro de texto con el texto «Apellidos» y el \sphinxcode{name} apellidos

\item {} 
Hay una lista desplegable múltiple con el \sphinxcode{name} «red» y con las siguientes opciones: opción «2G» con el \sphinxcode{value} 2g, opción «3G» con el \sphinxcode{value} 3g, opción «4G» con el \sphinxcode{value} 4g.

\end{itemize}

\noindent{\hspace*{\fill}\sphinxincludegraphics[scale=0.6]{{foto_formulario_06}.png}\hspace*{\fill}}

Solución:

\begin{sphinxVerbatim}[commandchars=\\\{\}]
\PYG{p}{\PYGZlt{}}\PYG{n+nt}{form}\PYG{p}{\PYGZgt{}}
\PYG{p}{\PYGZlt{}}\PYG{n+nt}{fieldset}\PYG{p}{\PYGZgt{}}
  \PYG{p}{\PYGZlt{}}\PYG{n+nt}{legend}\PYG{p}{\PYGZgt{}}Indique\PYG{p}{\PYGZlt{}}\PYG{p}{/}\PYG{n+nt}{legend}\PYG{p}{\PYGZgt{}}
  \PYG{p}{\PYGZlt{}}\PYG{n+nt}{textarea} \PYG{n+na}{rows}\PYG{o}{=}\PYG{l+s}{\PYGZsq{}4\PYGZsq{}} \PYG{n+na}{cols}\PYG{o}{=}\PYG{l+s}{\PYGZsq{}48\PYGZsq{}}\PYG{p}{\PYGZgt{}}
    Escriba aquí, por favor
  \PYG{p}{\PYGZlt{}}\PYG{p}{/}\PYG{n+nt}{textarea}\PYG{p}{\PYGZgt{}}  \PYG{p}{\PYGZlt{}}\PYG{n+nt}{br}\PYG{p}{/}\PYG{p}{\PYGZgt{}}
  \PYG{p}{\PYGZlt{}}\PYG{n+nt}{input} \PYG{n+na}{type}\PYG{o}{=}\PYG{l+s}{\PYGZsq{}radio\PYGZsq{}} \PYG{n+na}{name}\PYG{o}{=}\PYG{l+s}{\PYGZsq{}red\PYGZsq{}} \PYG{n+na}{value}\PYG{o}{=}\PYG{l+s}{\PYGZsq{}red2g\PYGZsq{}}\PYG{p}{\PYGZgt{}} 2G   \PYG{p}{\PYGZlt{}}\PYG{n+nt}{br}\PYG{p}{/}\PYG{p}{\PYGZgt{}}
  \PYG{p}{\PYGZlt{}}\PYG{n+nt}{input} \PYG{n+na}{type}\PYG{o}{=}\PYG{l+s}{\PYGZsq{}radio\PYGZsq{}} \PYG{n+na}{name}\PYG{o}{=}\PYG{l+s}{\PYGZsq{}red\PYGZsq{}} \PYG{n+na}{value}\PYG{o}{=}\PYG{l+s}{\PYGZsq{}red3g\PYGZsq{}}\PYG{p}{\PYGZgt{}} 3G   \PYG{p}{\PYGZlt{}}\PYG{n+nt}{br}\PYG{p}{/}\PYG{p}{\PYGZgt{}}
  \PYG{p}{\PYGZlt{}}\PYG{n+nt}{input} \PYG{n+na}{type}\PYG{o}{=}\PYG{l+s}{\PYGZsq{}radio\PYGZsq{}} \PYG{n+na}{name}\PYG{o}{=}\PYG{l+s}{\PYGZsq{}red\PYGZsq{}} \PYG{n+na}{value}\PYG{o}{=}\PYG{l+s}{\PYGZsq{}red4g\PYGZsq{}}\PYG{p}{\PYGZgt{}} 4G   \PYG{p}{\PYGZlt{}}\PYG{n+nt}{br}\PYG{p}{/}\PYG{p}{\PYGZgt{}}
  \PYG{p}{\PYGZlt{}}\PYG{n+nt}{br}\PYG{p}{/}\PYG{p}{\PYGZgt{}}
\PYG{p}{\PYGZlt{}}\PYG{p}{/}\PYG{n+nt}{fieldset}\PYG{p}{\PYGZgt{}}
\PYG{p}{\PYGZlt{}}\PYG{n+nt}{fieldset}\PYG{p}{\PYGZgt{}}
  \PYG{p}{\PYGZlt{}}\PYG{n+nt}{legend}\PYG{p}{\PYGZgt{}}Indique sus preferencias por favor\PYG{p}{\PYGZlt{}}\PYG{p}{/}\PYG{n+nt}{legend}\PYG{p}{\PYGZgt{}}
  Nombre\PYG{p}{\PYGZlt{}}\PYG{n+nt}{input} \PYG{n+na}{type}\PYG{o}{=}\PYG{l+s}{\PYGZsq{}text\PYGZsq{}} \PYG{n+na}{name}\PYG{o}{=}\PYG{l+s}{\PYGZsq{}nombre\PYGZsq{}}\PYG{p}{\PYGZgt{}}
  Apellidos\PYG{p}{\PYGZlt{}}\PYG{n+nt}{input} \PYG{n+na}{type}\PYG{o}{=}\PYG{l+s}{\PYGZsq{}text\PYGZsq{}} \PYG{n+na}{name}\PYG{o}{=}\PYG{l+s}{\PYGZsq{}apellidos\PYGZsq{}}\PYG{p}{\PYGZgt{}}
  \PYG{p}{\PYGZlt{}}\PYG{n+nt}{br}\PYG{p}{/}\PYG{p}{\PYGZgt{}}
  \PYG{p}{\PYGZlt{}}\PYG{n+nt}{select} \PYG{n+na}{name}\PYG{o}{=}\PYG{l+s}{\PYGZsq{}red\PYGZsq{}} \PYG{n+na}{multiple}\PYG{o}{=}\PYG{l+s}{\PYGZsq{}multiple\PYGZsq{}}\PYG{p}{\PYGZgt{}}
    \PYG{p}{\PYGZlt{}}\PYG{n+nt}{option} \PYG{n+na}{value}\PYG{o}{=}\PYG{l+s}{\PYGZsq{}2g\PYGZsq{}}\PYG{p}{\PYGZgt{}}2G\PYG{p}{\PYGZlt{}}\PYG{p}{/}\PYG{n+nt}{option}\PYG{p}{\PYGZgt{}}
    \PYG{p}{\PYGZlt{}}\PYG{n+nt}{option} \PYG{n+na}{value}\PYG{o}{=}\PYG{l+s}{\PYGZsq{}3g\PYGZsq{}}\PYG{p}{\PYGZgt{}}3G\PYG{p}{\PYGZlt{}}\PYG{p}{/}\PYG{n+nt}{option}\PYG{p}{\PYGZgt{}}
    \PYG{p}{\PYGZlt{}}\PYG{n+nt}{option} \PYG{n+na}{value}\PYG{o}{=}\PYG{l+s}{\PYGZsq{}4g\PYGZsq{}}\PYG{p}{\PYGZgt{}}4G\PYG{p}{\PYGZlt{}}\PYG{p}{/}\PYG{n+nt}{option}\PYG{p}{\PYGZgt{}}
  \PYG{p}{\PYGZlt{}}\PYG{p}{/}\PYG{n+nt}{select}\PYG{p}{\PYGZgt{}}
  \PYG{p}{\PYGZlt{}}\PYG{n+nt}{br}\PYG{p}{/}\PYG{p}{\PYGZgt{}}
\PYG{p}{\PYGZlt{}}\PYG{p}{/}\PYG{n+nt}{fieldset}\PYG{p}{\PYGZgt{}}
\PYG{p}{\PYGZlt{}}\PYG{p}{/}\PYG{n+nt}{form}\PYG{p}{\PYGZgt{}}
\end{sphinxVerbatim}


\section{Formulario 7}
\label{\detokenize{ejercicios/formularios/anexo_formularios:formulario-7}}
Generar el formulario siguiente de acuerdo a los siguientes requisitos
\begin{itemize}
\item {} 
Hay los siguientes cuadros de texto:cuadro de texto con el texto «Nombre» y el \sphinxcode{name} nombre, cuadro de texto con el texto «Apellidos» y el \sphinxcode{name} apellidos, cuadro de texto con el texto «Direccion» y el \sphinxcode{name} direccion

\item {} 
Hay los siguientes cuadros de texto:cuadro de texto con el texto «Nombre» y el \sphinxcode{name} nombre, cuadro de texto con el texto «Apellidos» y el \sphinxcode{name} apellidos, cuadro de texto con el texto «Direccion» y el \sphinxcode{name} direccion

\item {} 
Contiene los siguientes \sphinxcode{checkboxes}:checkbox con el \sphinxcode{name}  «lenguaje» , \sphinxcode{value}  «lenguajejava»  y el texto «Java», checkbox con el \sphinxcode{name}  «lenguaje» , \sphinxcode{value}  «lenguajepython»  y el texto «Python», checkbox con el \sphinxcode{name}  «lenguaje» , \sphinxcode{value}  «lenguajehtml»  y el texto «HTML», checkbox con el \sphinxcode{name}  «lenguaje» , \sphinxcode{value}  «lenguajevisual\_basic»  y el texto «Visual Basic», checkbox con el \sphinxcode{name}  «lenguaje» , \sphinxcode{value}  «lenguajecss»  y el texto «CSS».

\item {} 
Hay una lista desplegable múltiple con el \sphinxcode{name} «asignatura» y con las siguientes opciones: opción «Geografía» con el \sphinxcode{value} geografia, opción «Lengua» con el \sphinxcode{value} lengua, opción «Matemáticas» con el \sphinxcode{value} matematicas, opción «Historia» con el \sphinxcode{value} historia.

\item {} 
Hay los siguientes cuadros de texto:cuadro de texto con el texto «Nombre» y el \sphinxcode{name} nombre, cuadro de texto con el texto «Apellidos» y el \sphinxcode{name} apellidos

\end{itemize}

\noindent{\hspace*{\fill}\sphinxincludegraphics[scale=0.6]{{foto_formulario_07}.png}\hspace*{\fill}}

Solución:

\begin{sphinxVerbatim}[commandchars=\\\{\}]
\PYG{p}{\PYGZlt{}}\PYG{n+nt}{form}\PYG{p}{\PYGZgt{}}
\PYG{p}{\PYGZlt{}}\PYG{n+nt}{fieldset}\PYG{p}{\PYGZgt{}}
  \PYG{p}{\PYGZlt{}}\PYG{n+nt}{legend}\PYG{p}{\PYGZgt{}}Completar\PYG{p}{\PYGZlt{}}\PYG{p}{/}\PYG{n+nt}{legend}\PYG{p}{\PYGZgt{}}
  Nombre\PYG{p}{\PYGZlt{}}\PYG{n+nt}{input} \PYG{n+na}{type}\PYG{o}{=}\PYG{l+s}{\PYGZsq{}text\PYGZsq{}} \PYG{n+na}{name}\PYG{o}{=}\PYG{l+s}{\PYGZsq{}nombre\PYGZsq{}}\PYG{p}{\PYGZgt{}}
  Apellidos\PYG{p}{\PYGZlt{}}\PYG{n+nt}{input} \PYG{n+na}{type}\PYG{o}{=}\PYG{l+s}{\PYGZsq{}text\PYGZsq{}} \PYG{n+na}{name}\PYG{o}{=}\PYG{l+s}{\PYGZsq{}apellidos\PYGZsq{}}\PYG{p}{\PYGZgt{}}
  Direccion\PYG{p}{\PYGZlt{}}\PYG{n+nt}{input} \PYG{n+na}{type}\PYG{o}{=}\PYG{l+s}{\PYGZsq{}text\PYGZsq{}} \PYG{n+na}{name}\PYG{o}{=}\PYG{l+s}{\PYGZsq{}direccion\PYGZsq{}}\PYG{p}{\PYGZgt{}}
  \PYG{p}{\PYGZlt{}}\PYG{n+nt}{br}\PYG{p}{/}\PYG{p}{\PYGZgt{}}
  Nombre\PYG{p}{\PYGZlt{}}\PYG{n+nt}{input} \PYG{n+na}{type}\PYG{o}{=}\PYG{l+s}{\PYGZsq{}text\PYGZsq{}} \PYG{n+na}{name}\PYG{o}{=}\PYG{l+s}{\PYGZsq{}nombre\PYGZsq{}}\PYG{p}{\PYGZgt{}} \PYG{p}{\PYGZlt{}}\PYG{n+nt}{br}\PYG{p}{/}\PYG{p}{\PYGZgt{}}
  Apellidos\PYG{p}{\PYGZlt{}}\PYG{n+nt}{input} \PYG{n+na}{type}\PYG{o}{=}\PYG{l+s}{\PYGZsq{}text\PYGZsq{}} \PYG{n+na}{name}\PYG{o}{=}\PYG{l+s}{\PYGZsq{}apellidos\PYGZsq{}}\PYG{p}{\PYGZgt{}} \PYG{p}{\PYGZlt{}}\PYG{n+nt}{br}\PYG{p}{/}\PYG{p}{\PYGZgt{}}
  Direccion\PYG{p}{\PYGZlt{}}\PYG{n+nt}{input} \PYG{n+na}{type}\PYG{o}{=}\PYG{l+s}{\PYGZsq{}text\PYGZsq{}} \PYG{n+na}{name}\PYG{o}{=}\PYG{l+s}{\PYGZsq{}direccion\PYGZsq{}}\PYG{p}{\PYGZgt{}} \PYG{p}{\PYGZlt{}}\PYG{n+nt}{br}\PYG{p}{/}\PYG{p}{\PYGZgt{}}
  \PYG{p}{\PYGZlt{}}\PYG{n+nt}{br}\PYG{p}{/}\PYG{p}{\PYGZgt{}}
  \PYG{p}{\PYGZlt{}}\PYG{n+nt}{input} \PYG{n+na}{type}\PYG{o}{=}\PYG{l+s}{\PYGZsq{}checkbox\PYGZsq{}} \PYG{n+na}{name}\PYG{o}{=}\PYG{l+s}{\PYGZsq{}lenguaje\PYGZsq{}} \PYG{n+na}{value}\PYG{o}{=}\PYG{l+s}{\PYGZsq{}lenguajejava\PYGZsq{}}\PYG{p}{\PYGZgt{}} Java 
  \PYG{p}{\PYGZlt{}}\PYG{n+nt}{input} \PYG{n+na}{type}\PYG{o}{=}\PYG{l+s}{\PYGZsq{}checkbox\PYGZsq{}} \PYG{n+na}{name}\PYG{o}{=}\PYG{l+s}{\PYGZsq{}lenguaje\PYGZsq{}} \PYG{n+na}{value}\PYG{o}{=}\PYG{l+s}{\PYGZsq{}lenguajepython\PYGZsq{}}\PYG{p}{\PYGZgt{}} Python 
  \PYG{p}{\PYGZlt{}}\PYG{n+nt}{input} \PYG{n+na}{type}\PYG{o}{=}\PYG{l+s}{\PYGZsq{}checkbox\PYGZsq{}} \PYG{n+na}{name}\PYG{o}{=}\PYG{l+s}{\PYGZsq{}lenguaje\PYGZsq{}} \PYG{n+na}{value}\PYG{o}{=}\PYG{l+s}{\PYGZsq{}lenguajehtml\PYGZsq{}}\PYG{p}{\PYGZgt{}} HTML 
  \PYG{p}{\PYGZlt{}}\PYG{n+nt}{input} \PYG{n+na}{type}\PYG{o}{=}\PYG{l+s}{\PYGZsq{}checkbox\PYGZsq{}} \PYG{n+na}{name}\PYG{o}{=}\PYG{l+s}{\PYGZsq{}lenguaje\PYGZsq{}} \PYG{n+na}{value}\PYG{o}{=}\PYG{l+s}{\PYGZsq{}lenguajevisual\PYGZus{}basic\PYGZsq{}}\PYG{p}{\PYGZgt{}} Visual Basic 
  \PYG{p}{\PYGZlt{}}\PYG{n+nt}{input} \PYG{n+na}{type}\PYG{o}{=}\PYG{l+s}{\PYGZsq{}checkbox\PYGZsq{}} \PYG{n+na}{name}\PYG{o}{=}\PYG{l+s}{\PYGZsq{}lenguaje\PYGZsq{}} \PYG{n+na}{value}\PYG{o}{=}\PYG{l+s}{\PYGZsq{}lenguajecss\PYGZsq{}}\PYG{p}{\PYGZgt{}} CSS 
  \PYG{p}{\PYGZlt{}}\PYG{n+nt}{br}\PYG{p}{/}\PYG{p}{\PYGZgt{}}
  \PYG{p}{\PYGZlt{}}\PYG{n+nt}{select} \PYG{n+na}{name}\PYG{o}{=}\PYG{l+s}{\PYGZsq{}asignatura\PYGZsq{}} \PYG{n+na}{multiple}\PYG{o}{=}\PYG{l+s}{\PYGZsq{}multiple\PYGZsq{}}\PYG{p}{\PYGZgt{}}
    \PYG{p}{\PYGZlt{}}\PYG{n+nt}{option} \PYG{n+na}{value}\PYG{o}{=}\PYG{l+s}{\PYGZsq{}geografia\PYGZsq{}}\PYG{p}{\PYGZgt{}}Geografía\PYG{p}{\PYGZlt{}}\PYG{p}{/}\PYG{n+nt}{option}\PYG{p}{\PYGZgt{}}
    \PYG{p}{\PYGZlt{}}\PYG{n+nt}{option} \PYG{n+na}{value}\PYG{o}{=}\PYG{l+s}{\PYGZsq{}lengua\PYGZsq{}}\PYG{p}{\PYGZgt{}}Lengua\PYG{p}{\PYGZlt{}}\PYG{p}{/}\PYG{n+nt}{option}\PYG{p}{\PYGZgt{}}
    \PYG{p}{\PYGZlt{}}\PYG{n+nt}{option} \PYG{n+na}{value}\PYG{o}{=}\PYG{l+s}{\PYGZsq{}matematicas\PYGZsq{}}\PYG{p}{\PYGZgt{}}Matemáticas\PYG{p}{\PYGZlt{}}\PYG{p}{/}\PYG{n+nt}{option}\PYG{p}{\PYGZgt{}}
    \PYG{p}{\PYGZlt{}}\PYG{n+nt}{option} \PYG{n+na}{value}\PYG{o}{=}\PYG{l+s}{\PYGZsq{}historia\PYGZsq{}}\PYG{p}{\PYGZgt{}}Historia\PYG{p}{\PYGZlt{}}\PYG{p}{/}\PYG{n+nt}{option}\PYG{p}{\PYGZgt{}}
  \PYG{p}{\PYGZlt{}}\PYG{p}{/}\PYG{n+nt}{select}\PYG{p}{\PYGZgt{}}
  \PYG{p}{\PYGZlt{}}\PYG{n+nt}{br}\PYG{p}{/}\PYG{p}{\PYGZgt{}}
\PYG{p}{\PYGZlt{}}\PYG{p}{/}\PYG{n+nt}{fieldset}\PYG{p}{\PYGZgt{}}
\PYG{p}{\PYGZlt{}}\PYG{n+nt}{fieldset}\PYG{p}{\PYGZgt{}}
  \PYG{p}{\PYGZlt{}}\PYG{n+nt}{legend}\PYG{p}{\PYGZgt{}}Rellenar\PYG{p}{\PYGZlt{}}\PYG{p}{/}\PYG{n+nt}{legend}\PYG{p}{\PYGZgt{}}
  Nombre\PYG{p}{\PYGZlt{}}\PYG{n+nt}{input} \PYG{n+na}{type}\PYG{o}{=}\PYG{l+s}{\PYGZsq{}text\PYGZsq{}} \PYG{n+na}{name}\PYG{o}{=}\PYG{l+s}{\PYGZsq{}nombre\PYGZsq{}}\PYG{p}{\PYGZgt{}} \PYG{p}{\PYGZlt{}}\PYG{n+nt}{br}\PYG{p}{/}\PYG{p}{\PYGZgt{}}
  Apellidos\PYG{p}{\PYGZlt{}}\PYG{n+nt}{input} \PYG{n+na}{type}\PYG{o}{=}\PYG{l+s}{\PYGZsq{}text\PYGZsq{}} \PYG{n+na}{name}\PYG{o}{=}\PYG{l+s}{\PYGZsq{}apellidos\PYGZsq{}}\PYG{p}{\PYGZgt{}} \PYG{p}{\PYGZlt{}}\PYG{n+nt}{br}\PYG{p}{/}\PYG{p}{\PYGZgt{}}
  \PYG{p}{\PYGZlt{}}\PYG{n+nt}{br}\PYG{p}{/}\PYG{p}{\PYGZgt{}}
\PYG{p}{\PYGZlt{}}\PYG{p}{/}\PYG{n+nt}{fieldset}\PYG{p}{\PYGZgt{}}
\PYG{p}{\PYGZlt{}}\PYG{p}{/}\PYG{n+nt}{form}\PYG{p}{\PYGZgt{}}
\end{sphinxVerbatim}


\section{Formulario 8}
\label{\detokenize{ejercicios/formularios/anexo_formularios:formulario-8}}
Generar el formulario siguiente de acuerdo a los siguientes requisitos
\begin{itemize}
\item {} 
Contiene los siguientes \sphinxcode{radiobuttons}:radio con el \sphinxcode{name}  «conector» , \sphinxcode{value}  «conectorusb»  y el texto «USB», radio con el \sphinxcode{name}  «conector» , \sphinxcode{value}  «conectorparalelo»  y el texto «Paralelo», radio con el \sphinxcode{name}  «conector» , \sphinxcode{value}  «conectorps2»  y el texto «PS2».

\item {} 
Contiene los siguientes \sphinxcode{radiobuttons}:radio con el \sphinxcode{name}  «velocidad» , \sphinxcode{value}  «velocidad100\_mbits»  y el texto «100 Mbits», radio con el \sphinxcode{name}  «velocidad» , \sphinxcode{value}  «velocidad1000\_mbits»  y el texto «1000 Mbits».

\item {} 
Hay los siguientes cuadros de texto:cuadro de texto con el texto «Instituto» y el \sphinxcode{name} instituto, cuadro de texto con el texto «Estudios elegidos» y el \sphinxcode{name} estudios

\end{itemize}

\noindent{\hspace*{\fill}\sphinxincludegraphics[scale=0.6]{{foto_formulario_08}.png}\hspace*{\fill}}

Solución:

\begin{sphinxVerbatim}[commandchars=\\\{\}]
\PYG{p}{\PYGZlt{}}\PYG{n+nt}{form}\PYG{p}{\PYGZgt{}}
\PYG{p}{\PYGZlt{}}\PYG{n+nt}{fieldset}\PYG{p}{\PYGZgt{}}
  \PYG{p}{\PYGZlt{}}\PYG{n+nt}{legend}\PYG{p}{\PYGZgt{}}Completar estas opciones\PYG{p}{\PYGZlt{}}\PYG{p}{/}\PYG{n+nt}{legend}\PYG{p}{\PYGZgt{}}
  \PYG{p}{\PYGZlt{}}\PYG{n+nt}{input} \PYG{n+na}{type}\PYG{o}{=}\PYG{l+s}{\PYGZsq{}radio\PYGZsq{}} \PYG{n+na}{name}\PYG{o}{=}\PYG{l+s}{\PYGZsq{}conector\PYGZsq{}} \PYG{n+na}{value}\PYG{o}{=}\PYG{l+s}{\PYGZsq{}conectorusb\PYGZsq{}}\PYG{p}{\PYGZgt{}} USB   \PYG{p}{\PYGZlt{}}\PYG{n+nt}{br}\PYG{p}{/}\PYG{p}{\PYGZgt{}}
  \PYG{p}{\PYGZlt{}}\PYG{n+nt}{input} \PYG{n+na}{type}\PYG{o}{=}\PYG{l+s}{\PYGZsq{}radio\PYGZsq{}} \PYG{n+na}{name}\PYG{o}{=}\PYG{l+s}{\PYGZsq{}conector\PYGZsq{}} \PYG{n+na}{value}\PYG{o}{=}\PYG{l+s}{\PYGZsq{}conectorparalelo\PYGZsq{}}\PYG{p}{\PYGZgt{}} Paralelo   \PYG{p}{\PYGZlt{}}\PYG{n+nt}{br}\PYG{p}{/}\PYG{p}{\PYGZgt{}}
  \PYG{p}{\PYGZlt{}}\PYG{n+nt}{input} \PYG{n+na}{type}\PYG{o}{=}\PYG{l+s}{\PYGZsq{}radio\PYGZsq{}} \PYG{n+na}{name}\PYG{o}{=}\PYG{l+s}{\PYGZsq{}conector\PYGZsq{}} \PYG{n+na}{value}\PYG{o}{=}\PYG{l+s}{\PYGZsq{}conectorps2\PYGZsq{}}\PYG{p}{\PYGZgt{}} PS2   \PYG{p}{\PYGZlt{}}\PYG{n+nt}{br}\PYG{p}{/}\PYG{p}{\PYGZgt{}}
  \PYG{p}{\PYGZlt{}}\PYG{n+nt}{br}\PYG{p}{/}\PYG{p}{\PYGZgt{}}
  \PYG{p}{\PYGZlt{}}\PYG{n+nt}{input} \PYG{n+na}{type}\PYG{o}{=}\PYG{l+s}{\PYGZsq{}radio\PYGZsq{}} \PYG{n+na}{name}\PYG{o}{=}\PYG{l+s}{\PYGZsq{}velocidad\PYGZsq{}} \PYG{n+na}{value}\PYG{o}{=}\PYG{l+s}{\PYGZsq{}velocidad100\PYGZus{}mbits\PYGZsq{}}\PYG{p}{\PYGZgt{}} 100 Mbits   \PYG{p}{\PYGZlt{}}\PYG{n+nt}{br}\PYG{p}{/}\PYG{p}{\PYGZgt{}}
  \PYG{p}{\PYGZlt{}}\PYG{n+nt}{input} \PYG{n+na}{type}\PYG{o}{=}\PYG{l+s}{\PYGZsq{}radio\PYGZsq{}} \PYG{n+na}{name}\PYG{o}{=}\PYG{l+s}{\PYGZsq{}velocidad\PYGZsq{}} \PYG{n+na}{value}\PYG{o}{=}\PYG{l+s}{\PYGZsq{}velocidad1000\PYGZus{}mbits\PYGZsq{}}\PYG{p}{\PYGZgt{}} 1000 Mbits   \PYG{p}{\PYGZlt{}}\PYG{n+nt}{br}\PYG{p}{/}\PYG{p}{\PYGZgt{}}
  \PYG{p}{\PYGZlt{}}\PYG{n+nt}{br}\PYG{p}{/}\PYG{p}{\PYGZgt{}}
\PYG{p}{\PYGZlt{}}\PYG{p}{/}\PYG{n+nt}{fieldset}\PYG{p}{\PYGZgt{}}
\PYG{p}{\PYGZlt{}}\PYG{n+nt}{fieldset}\PYG{p}{\PYGZgt{}}
  \PYG{p}{\PYGZlt{}}\PYG{n+nt}{legend}\PYG{p}{\PYGZgt{}}Completar\PYG{p}{\PYGZlt{}}\PYG{p}{/}\PYG{n+nt}{legend}\PYG{p}{\PYGZgt{}}
  Instituto\PYG{p}{\PYGZlt{}}\PYG{n+nt}{input} \PYG{n+na}{type}\PYG{o}{=}\PYG{l+s}{\PYGZsq{}text\PYGZsq{}} \PYG{n+na}{name}\PYG{o}{=}\PYG{l+s}{\PYGZsq{}instituto\PYGZsq{}}\PYG{p}{\PYGZgt{}} \PYG{p}{\PYGZlt{}}\PYG{n+nt}{br}\PYG{p}{/}\PYG{p}{\PYGZgt{}}
  Estudios elegidos\PYG{p}{\PYGZlt{}}\PYG{n+nt}{input} \PYG{n+na}{type}\PYG{o}{=}\PYG{l+s}{\PYGZsq{}text\PYGZsq{}} \PYG{n+na}{name}\PYG{o}{=}\PYG{l+s}{\PYGZsq{}estudios\PYGZsq{}}\PYG{p}{\PYGZgt{}} \PYG{p}{\PYGZlt{}}\PYG{n+nt}{br}\PYG{p}{/}\PYG{p}{\PYGZgt{}}
  \PYG{p}{\PYGZlt{}}\PYG{n+nt}{br}\PYG{p}{/}\PYG{p}{\PYGZgt{}}
\PYG{p}{\PYGZlt{}}\PYG{p}{/}\PYG{n+nt}{fieldset}\PYG{p}{\PYGZgt{}}
\PYG{p}{\PYGZlt{}}\PYG{p}{/}\PYG{n+nt}{form}\PYG{p}{\PYGZgt{}}
\end{sphinxVerbatim}


\section{Formulario 9}
\label{\detokenize{ejercicios/formularios/anexo_formularios:formulario-9}}
Generar el formulario siguiente de acuerdo a los siguientes requisitos
\begin{itemize}
\item {} 
Contiene los siguientes \sphinxcode{radiobuttons}:radio con el \sphinxcode{name}  «aula» , \sphinxcode{value}  «aulaa01»  y el texto «A01», radio con el \sphinxcode{name}  «aula» , \sphinxcode{value}  «aulaa02»  y el texto «A02», radio con el \sphinxcode{name}  «aula» , \sphinxcode{value}  «aulaa03»  y el texto «A03».

\item {} 
Contiene los siguientes \sphinxcode{radiobuttons}:radio con el \sphinxcode{name}  «idioma» , \sphinxcode{value}  «idiomaespanol»  y el texto «Español», radio con el \sphinxcode{name}  «idioma» , \sphinxcode{value}  «idiomaingles»  y el texto «Inglés», radio con el \sphinxcode{name}  «idioma» , \sphinxcode{value}  «idiomaaleman»  y el texto «Alemán», radio con el \sphinxcode{name}  «idioma» , \sphinxcode{value}  «idiomafrances»  y el texto «Francés».

\item {} 
Hay los siguientes cuadros de texto:cuadro de texto con el texto «Nombre» y el \sphinxcode{name} nombre, cuadro de texto con el texto «Apellidos» y el \sphinxcode{name} apellidos

\end{itemize}

\noindent{\hspace*{\fill}\sphinxincludegraphics[scale=0.6]{{foto_formulario_09}.png}\hspace*{\fill}}

Solución:

\begin{sphinxVerbatim}[commandchars=\\\{\}]
\PYG{p}{\PYGZlt{}}\PYG{n+nt}{form}\PYG{p}{\PYGZgt{}}
\PYG{p}{\PYGZlt{}}\PYG{n+nt}{fieldset}\PYG{p}{\PYGZgt{}}
  \PYG{p}{\PYGZlt{}}\PYG{n+nt}{legend}\PYG{p}{\PYGZgt{}}Formulario de opciones\PYG{p}{\PYGZlt{}}\PYG{p}{/}\PYG{n+nt}{legend}\PYG{p}{\PYGZgt{}}
  \PYG{p}{\PYGZlt{}}\PYG{n+nt}{input} \PYG{n+na}{type}\PYG{o}{=}\PYG{l+s}{\PYGZsq{}radio\PYGZsq{}} \PYG{n+na}{name}\PYG{o}{=}\PYG{l+s}{\PYGZsq{}aula\PYGZsq{}} \PYG{n+na}{value}\PYG{o}{=}\PYG{l+s}{\PYGZsq{}aulaa01\PYGZsq{}}\PYG{p}{\PYGZgt{}} A01 
  \PYG{p}{\PYGZlt{}}\PYG{n+nt}{input} \PYG{n+na}{type}\PYG{o}{=}\PYG{l+s}{\PYGZsq{}radio\PYGZsq{}} \PYG{n+na}{name}\PYG{o}{=}\PYG{l+s}{\PYGZsq{}aula\PYGZsq{}} \PYG{n+na}{value}\PYG{o}{=}\PYG{l+s}{\PYGZsq{}aulaa02\PYGZsq{}}\PYG{p}{\PYGZgt{}} A02 
  \PYG{p}{\PYGZlt{}}\PYG{n+nt}{input} \PYG{n+na}{type}\PYG{o}{=}\PYG{l+s}{\PYGZsq{}radio\PYGZsq{}} \PYG{n+na}{name}\PYG{o}{=}\PYG{l+s}{\PYGZsq{}aula\PYGZsq{}} \PYG{n+na}{value}\PYG{o}{=}\PYG{l+s}{\PYGZsq{}aulaa03\PYGZsq{}}\PYG{p}{\PYGZgt{}} A03 
  \PYG{p}{\PYGZlt{}}\PYG{n+nt}{br}\PYG{p}{/}\PYG{p}{\PYGZgt{}}
  \PYG{p}{\PYGZlt{}}\PYG{n+nt}{input} \PYG{n+na}{type}\PYG{o}{=}\PYG{l+s}{\PYGZsq{}radio\PYGZsq{}} \PYG{n+na}{name}\PYG{o}{=}\PYG{l+s}{\PYGZsq{}idioma\PYGZsq{}} \PYG{n+na}{value}\PYG{o}{=}\PYG{l+s}{\PYGZsq{}idiomaespanol\PYGZsq{}}\PYG{p}{\PYGZgt{}} Español   \PYG{p}{\PYGZlt{}}\PYG{n+nt}{br}\PYG{p}{/}\PYG{p}{\PYGZgt{}}
  \PYG{p}{\PYGZlt{}}\PYG{n+nt}{input} \PYG{n+na}{type}\PYG{o}{=}\PYG{l+s}{\PYGZsq{}radio\PYGZsq{}} \PYG{n+na}{name}\PYG{o}{=}\PYG{l+s}{\PYGZsq{}idioma\PYGZsq{}} \PYG{n+na}{value}\PYG{o}{=}\PYG{l+s}{\PYGZsq{}idiomaingles\PYGZsq{}}\PYG{p}{\PYGZgt{}} Inglés   \PYG{p}{\PYGZlt{}}\PYG{n+nt}{br}\PYG{p}{/}\PYG{p}{\PYGZgt{}}
  \PYG{p}{\PYGZlt{}}\PYG{n+nt}{input} \PYG{n+na}{type}\PYG{o}{=}\PYG{l+s}{\PYGZsq{}radio\PYGZsq{}} \PYG{n+na}{name}\PYG{o}{=}\PYG{l+s}{\PYGZsq{}idioma\PYGZsq{}} \PYG{n+na}{value}\PYG{o}{=}\PYG{l+s}{\PYGZsq{}idiomaaleman\PYGZsq{}}\PYG{p}{\PYGZgt{}} Alemán   \PYG{p}{\PYGZlt{}}\PYG{n+nt}{br}\PYG{p}{/}\PYG{p}{\PYGZgt{}}
  \PYG{p}{\PYGZlt{}}\PYG{n+nt}{input} \PYG{n+na}{type}\PYG{o}{=}\PYG{l+s}{\PYGZsq{}radio\PYGZsq{}} \PYG{n+na}{name}\PYG{o}{=}\PYG{l+s}{\PYGZsq{}idioma\PYGZsq{}} \PYG{n+na}{value}\PYG{o}{=}\PYG{l+s}{\PYGZsq{}idiomafrances\PYGZsq{}}\PYG{p}{\PYGZgt{}} Francés   \PYG{p}{\PYGZlt{}}\PYG{n+nt}{br}\PYG{p}{/}\PYG{p}{\PYGZgt{}}
  \PYG{p}{\PYGZlt{}}\PYG{n+nt}{br}\PYG{p}{/}\PYG{p}{\PYGZgt{}}
  Nombre\PYG{p}{\PYGZlt{}}\PYG{n+nt}{input} \PYG{n+na}{type}\PYG{o}{=}\PYG{l+s}{\PYGZsq{}text\PYGZsq{}} \PYG{n+na}{name}\PYG{o}{=}\PYG{l+s}{\PYGZsq{}nombre\PYGZsq{}}\PYG{p}{\PYGZgt{}} \PYG{p}{\PYGZlt{}}\PYG{n+nt}{br}\PYG{p}{/}\PYG{p}{\PYGZgt{}}
  Apellidos\PYG{p}{\PYGZlt{}}\PYG{n+nt}{input} \PYG{n+na}{type}\PYG{o}{=}\PYG{l+s}{\PYGZsq{}text\PYGZsq{}} \PYG{n+na}{name}\PYG{o}{=}\PYG{l+s}{\PYGZsq{}apellidos\PYGZsq{}}\PYG{p}{\PYGZgt{}} \PYG{p}{\PYGZlt{}}\PYG{n+nt}{br}\PYG{p}{/}\PYG{p}{\PYGZgt{}}
  \PYG{p}{\PYGZlt{}}\PYG{n+nt}{br}\PYG{p}{/}\PYG{p}{\PYGZgt{}}
\PYG{p}{\PYGZlt{}}\PYG{p}{/}\PYG{n+nt}{fieldset}\PYG{p}{\PYGZgt{}}
\PYG{p}{\PYGZlt{}}\PYG{p}{/}\PYG{n+nt}{form}\PYG{p}{\PYGZgt{}}
\end{sphinxVerbatim}


\section{Formulario 10}
\label{\detokenize{ejercicios/formularios/anexo_formularios:formulario-10}}
Generar el formulario siguiente de acuerdo a los siguientes requisitos
\begin{itemize}
\item {} 
Hay los siguientes cuadros de texto:cuadro de texto con el texto «Nombre» y el \sphinxcode{name} nombre, cuadro de texto con el texto «Apellidos» y el \sphinxcode{name} apellidos, cuadro de texto con el texto «Direccion» y el \sphinxcode{name} direccion

\item {} 
Hay los siguientes cuadros de texto:cuadro de texto con el texto «Nombre» y el \sphinxcode{name} nombre, cuadro de texto con el texto «Apellidos» y el \sphinxcode{name} apellidos, cuadro de texto con el texto «Direccion» y el \sphinxcode{name} direccion

\item {} 
Hay una lista desplegable múltiple con el \sphinxcode{name} «asignatura» y con las siguientes opciones: opción «Matemáticas» con el \sphinxcode{value} matematicas, opción «Historia» con el \sphinxcode{value} historia, opción «Geografía» con el \sphinxcode{value} geografia, opción «Lengua» con el \sphinxcode{value} lengua.

\item {} 
Hay un \sphinxcode{textarea} que mide 8 filas y 49 columnas que lleva dentro el texto «Utilice este recuadro por favor»

\item {} 
Contiene los siguientes \sphinxcode{radiobuttons}:radio con el \sphinxcode{name}  «idioma» , \sphinxcode{value}  «idiomaespanol»  y el texto «Español», radio con el \sphinxcode{name}  «idioma» , \sphinxcode{value}  «idiomaingles»  y el texto «Inglés», radio con el \sphinxcode{name}  «idioma» , \sphinxcode{value}  «idiomaaleman»  y el texto «Alemán», radio con el \sphinxcode{name}  «idioma» , \sphinxcode{value}  «idiomafrances»  y el texto «Francés».

\item {} 
Contiene los siguientes \sphinxcode{radiobuttons}:radio con el \sphinxcode{name}  «navegador» , \sphinxcode{value}  «navegadorfirefox»  y el texto «Firefox», radio con el \sphinxcode{name}  «navegador» , \sphinxcode{value}  «navegadorchrome»  y el texto «Chrome», radio con el \sphinxcode{name}  «navegador» , \sphinxcode{value}  «navegadoropera»  y el texto «Opera», radio con el \sphinxcode{name}  «navegador» , \sphinxcode{value}  «navegadorie»  y el texto «IE».

\item {} 
Hay los siguientes cuadros de texto:cuadro de texto con el texto «Nombre» y el \sphinxcode{name} nombre, cuadro de texto con el texto «Apellidos» y el \sphinxcode{name} apellidos, cuadro de texto con el texto «Direccion» y el \sphinxcode{name} direccion

\end{itemize}

\noindent{\hspace*{\fill}\sphinxincludegraphics[scale=0.6]{{foto_formulario_10}.png}\hspace*{\fill}}

Solución:

\begin{sphinxVerbatim}[commandchars=\\\{\}]
\PYG{p}{\PYGZlt{}}\PYG{n+nt}{form}\PYG{p}{\PYGZgt{}}
\PYG{p}{\PYGZlt{}}\PYG{n+nt}{fieldset}\PYG{p}{\PYGZgt{}}
  \PYG{p}{\PYGZlt{}}\PYG{n+nt}{legend}\PYG{p}{\PYGZgt{}}Rellenar\PYG{p}{\PYGZlt{}}\PYG{p}{/}\PYG{n+nt}{legend}\PYG{p}{\PYGZgt{}}
  Nombre\PYG{p}{\PYGZlt{}}\PYG{n+nt}{input} \PYG{n+na}{type}\PYG{o}{=}\PYG{l+s}{\PYGZsq{}text\PYGZsq{}} \PYG{n+na}{name}\PYG{o}{=}\PYG{l+s}{\PYGZsq{}nombre\PYGZsq{}}\PYG{p}{\PYGZgt{}}
  Apellidos\PYG{p}{\PYGZlt{}}\PYG{n+nt}{input} \PYG{n+na}{type}\PYG{o}{=}\PYG{l+s}{\PYGZsq{}text\PYGZsq{}} \PYG{n+na}{name}\PYG{o}{=}\PYG{l+s}{\PYGZsq{}apellidos\PYGZsq{}}\PYG{p}{\PYGZgt{}}
  Direccion\PYG{p}{\PYGZlt{}}\PYG{n+nt}{input} \PYG{n+na}{type}\PYG{o}{=}\PYG{l+s}{\PYGZsq{}text\PYGZsq{}} \PYG{n+na}{name}\PYG{o}{=}\PYG{l+s}{\PYGZsq{}direccion\PYGZsq{}}\PYG{p}{\PYGZgt{}}
  \PYG{p}{\PYGZlt{}}\PYG{n+nt}{br}\PYG{p}{/}\PYG{p}{\PYGZgt{}}
  Nombre\PYG{p}{\PYGZlt{}}\PYG{n+nt}{input} \PYG{n+na}{type}\PYG{o}{=}\PYG{l+s}{\PYGZsq{}text\PYGZsq{}} \PYG{n+na}{name}\PYG{o}{=}\PYG{l+s}{\PYGZsq{}nombre\PYGZsq{}}\PYG{p}{\PYGZgt{}} \PYG{p}{\PYGZlt{}}\PYG{n+nt}{br}\PYG{p}{/}\PYG{p}{\PYGZgt{}}
  Apellidos\PYG{p}{\PYGZlt{}}\PYG{n+nt}{input} \PYG{n+na}{type}\PYG{o}{=}\PYG{l+s}{\PYGZsq{}text\PYGZsq{}} \PYG{n+na}{name}\PYG{o}{=}\PYG{l+s}{\PYGZsq{}apellidos\PYGZsq{}}\PYG{p}{\PYGZgt{}} \PYG{p}{\PYGZlt{}}\PYG{n+nt}{br}\PYG{p}{/}\PYG{p}{\PYGZgt{}}
  Direccion\PYG{p}{\PYGZlt{}}\PYG{n+nt}{input} \PYG{n+na}{type}\PYG{o}{=}\PYG{l+s}{\PYGZsq{}text\PYGZsq{}} \PYG{n+na}{name}\PYG{o}{=}\PYG{l+s}{\PYGZsq{}direccion\PYGZsq{}}\PYG{p}{\PYGZgt{}} \PYG{p}{\PYGZlt{}}\PYG{n+nt}{br}\PYG{p}{/}\PYG{p}{\PYGZgt{}}
  \PYG{p}{\PYGZlt{}}\PYG{n+nt}{br}\PYG{p}{/}\PYG{p}{\PYGZgt{}}
  \PYG{p}{\PYGZlt{}}\PYG{n+nt}{select} \PYG{n+na}{name}\PYG{o}{=}\PYG{l+s}{\PYGZsq{}asignatura\PYGZsq{}} \PYG{p}{\PYGZgt{}}
    \PYG{p}{\PYGZlt{}}\PYG{n+nt}{option} \PYG{n+na}{value}\PYG{o}{=}\PYG{l+s}{\PYGZsq{}matematicas\PYGZsq{}}\PYG{p}{\PYGZgt{}}Matemáticas\PYG{p}{\PYGZlt{}}\PYG{p}{/}\PYG{n+nt}{option}\PYG{p}{\PYGZgt{}}
    \PYG{p}{\PYGZlt{}}\PYG{n+nt}{option} \PYG{n+na}{value}\PYG{o}{=}\PYG{l+s}{\PYGZsq{}historia\PYGZsq{}}\PYG{p}{\PYGZgt{}}Historia\PYG{p}{\PYGZlt{}}\PYG{p}{/}\PYG{n+nt}{option}\PYG{p}{\PYGZgt{}}
    \PYG{p}{\PYGZlt{}}\PYG{n+nt}{option} \PYG{n+na}{value}\PYG{o}{=}\PYG{l+s}{\PYGZsq{}geografia\PYGZsq{}}\PYG{p}{\PYGZgt{}}Geografía\PYG{p}{\PYGZlt{}}\PYG{p}{/}\PYG{n+nt}{option}\PYG{p}{\PYGZgt{}}
    \PYG{p}{\PYGZlt{}}\PYG{n+nt}{option} \PYG{n+na}{value}\PYG{o}{=}\PYG{l+s}{\PYGZsq{}lengua\PYGZsq{}}\PYG{p}{\PYGZgt{}}Lengua\PYG{p}{\PYGZlt{}}\PYG{p}{/}\PYG{n+nt}{option}\PYG{p}{\PYGZgt{}}
  \PYG{p}{\PYGZlt{}}\PYG{p}{/}\PYG{n+nt}{select}\PYG{p}{\PYGZgt{}}
  \PYG{p}{\PYGZlt{}}\PYG{n+nt}{br}\PYG{p}{/}\PYG{p}{\PYGZgt{}}
\PYG{p}{\PYGZlt{}}\PYG{p}{/}\PYG{n+nt}{fieldset}\PYG{p}{\PYGZgt{}}
\PYG{p}{\PYGZlt{}}\PYG{n+nt}{fieldset}\PYG{p}{\PYGZgt{}}
  \PYG{p}{\PYGZlt{}}\PYG{n+nt}{legend}\PYG{p}{\PYGZgt{}}Complete, por favor\PYG{p}{\PYGZlt{}}\PYG{p}{/}\PYG{n+nt}{legend}\PYG{p}{\PYGZgt{}}
  \PYG{p}{\PYGZlt{}}\PYG{n+nt}{textarea} \PYG{n+na}{rows}\PYG{o}{=}\PYG{l+s}{\PYGZsq{}8\PYGZsq{}} \PYG{n+na}{cols}\PYG{o}{=}\PYG{l+s}{\PYGZsq{}49\PYGZsq{}}\PYG{p}{\PYGZgt{}}
    Utilice este recuadro por favor
  \PYG{p}{\PYGZlt{}}\PYG{p}{/}\PYG{n+nt}{textarea}\PYG{p}{\PYGZgt{}}  \PYG{p}{\PYGZlt{}}\PYG{n+nt}{br}\PYG{p}{/}\PYG{p}{\PYGZgt{}}
  \PYG{p}{\PYGZlt{}}\PYG{n+nt}{input} \PYG{n+na}{type}\PYG{o}{=}\PYG{l+s}{\PYGZsq{}radio\PYGZsq{}} \PYG{n+na}{name}\PYG{o}{=}\PYG{l+s}{\PYGZsq{}idioma\PYGZsq{}} \PYG{n+na}{value}\PYG{o}{=}\PYG{l+s}{\PYGZsq{}idiomaespanol\PYGZsq{}}\PYG{p}{\PYGZgt{}} Español   \PYG{p}{\PYGZlt{}}\PYG{n+nt}{br}\PYG{p}{/}\PYG{p}{\PYGZgt{}}
  \PYG{p}{\PYGZlt{}}\PYG{n+nt}{input} \PYG{n+na}{type}\PYG{o}{=}\PYG{l+s}{\PYGZsq{}radio\PYGZsq{}} \PYG{n+na}{name}\PYG{o}{=}\PYG{l+s}{\PYGZsq{}idioma\PYGZsq{}} \PYG{n+na}{value}\PYG{o}{=}\PYG{l+s}{\PYGZsq{}idiomaingles\PYGZsq{}}\PYG{p}{\PYGZgt{}} Inglés   \PYG{p}{\PYGZlt{}}\PYG{n+nt}{br}\PYG{p}{/}\PYG{p}{\PYGZgt{}}
  \PYG{p}{\PYGZlt{}}\PYG{n+nt}{input} \PYG{n+na}{type}\PYG{o}{=}\PYG{l+s}{\PYGZsq{}radio\PYGZsq{}} \PYG{n+na}{name}\PYG{o}{=}\PYG{l+s}{\PYGZsq{}idioma\PYGZsq{}} \PYG{n+na}{value}\PYG{o}{=}\PYG{l+s}{\PYGZsq{}idiomaaleman\PYGZsq{}}\PYG{p}{\PYGZgt{}} Alemán   \PYG{p}{\PYGZlt{}}\PYG{n+nt}{br}\PYG{p}{/}\PYG{p}{\PYGZgt{}}
  \PYG{p}{\PYGZlt{}}\PYG{n+nt}{input} \PYG{n+na}{type}\PYG{o}{=}\PYG{l+s}{\PYGZsq{}radio\PYGZsq{}} \PYG{n+na}{name}\PYG{o}{=}\PYG{l+s}{\PYGZsq{}idioma\PYGZsq{}} \PYG{n+na}{value}\PYG{o}{=}\PYG{l+s}{\PYGZsq{}idiomafrances\PYGZsq{}}\PYG{p}{\PYGZgt{}} Francés   \PYG{p}{\PYGZlt{}}\PYG{n+nt}{br}\PYG{p}{/}\PYG{p}{\PYGZgt{}}
  \PYG{p}{\PYGZlt{}}\PYG{n+nt}{br}\PYG{p}{/}\PYG{p}{\PYGZgt{}}
  \PYG{p}{\PYGZlt{}}\PYG{n+nt}{input} \PYG{n+na}{type}\PYG{o}{=}\PYG{l+s}{\PYGZsq{}radio\PYGZsq{}} \PYG{n+na}{name}\PYG{o}{=}\PYG{l+s}{\PYGZsq{}navegador\PYGZsq{}} \PYG{n+na}{value}\PYG{o}{=}\PYG{l+s}{\PYGZsq{}navegadorfirefox\PYGZsq{}}\PYG{p}{\PYGZgt{}} Firefox 
  \PYG{p}{\PYGZlt{}}\PYG{n+nt}{input} \PYG{n+na}{type}\PYG{o}{=}\PYG{l+s}{\PYGZsq{}radio\PYGZsq{}} \PYG{n+na}{name}\PYG{o}{=}\PYG{l+s}{\PYGZsq{}navegador\PYGZsq{}} \PYG{n+na}{value}\PYG{o}{=}\PYG{l+s}{\PYGZsq{}navegadorchrome\PYGZsq{}}\PYG{p}{\PYGZgt{}} Chrome 
  \PYG{p}{\PYGZlt{}}\PYG{n+nt}{input} \PYG{n+na}{type}\PYG{o}{=}\PYG{l+s}{\PYGZsq{}radio\PYGZsq{}} \PYG{n+na}{name}\PYG{o}{=}\PYG{l+s}{\PYGZsq{}navegador\PYGZsq{}} \PYG{n+na}{value}\PYG{o}{=}\PYG{l+s}{\PYGZsq{}navegadoropera\PYGZsq{}}\PYG{p}{\PYGZgt{}} Opera 
  \PYG{p}{\PYGZlt{}}\PYG{n+nt}{input} \PYG{n+na}{type}\PYG{o}{=}\PYG{l+s}{\PYGZsq{}radio\PYGZsq{}} \PYG{n+na}{name}\PYG{o}{=}\PYG{l+s}{\PYGZsq{}navegador\PYGZsq{}} \PYG{n+na}{value}\PYG{o}{=}\PYG{l+s}{\PYGZsq{}navegadorie\PYGZsq{}}\PYG{p}{\PYGZgt{}} IE 
  \PYG{p}{\PYGZlt{}}\PYG{n+nt}{br}\PYG{p}{/}\PYG{p}{\PYGZgt{}}
  Nombre\PYG{p}{\PYGZlt{}}\PYG{n+nt}{input} \PYG{n+na}{type}\PYG{o}{=}\PYG{l+s}{\PYGZsq{}text\PYGZsq{}} \PYG{n+na}{name}\PYG{o}{=}\PYG{l+s}{\PYGZsq{}nombre\PYGZsq{}}\PYG{p}{\PYGZgt{}}
  Apellidos\PYG{p}{\PYGZlt{}}\PYG{n+nt}{input} \PYG{n+na}{type}\PYG{o}{=}\PYG{l+s}{\PYGZsq{}text\PYGZsq{}} \PYG{n+na}{name}\PYG{o}{=}\PYG{l+s}{\PYGZsq{}apellidos\PYGZsq{}}\PYG{p}{\PYGZgt{}}
  Direccion\PYG{p}{\PYGZlt{}}\PYG{n+nt}{input} \PYG{n+na}{type}\PYG{o}{=}\PYG{l+s}{\PYGZsq{}text\PYGZsq{}} \PYG{n+na}{name}\PYG{o}{=}\PYG{l+s}{\PYGZsq{}direccion\PYGZsq{}}\PYG{p}{\PYGZgt{}}
  \PYG{p}{\PYGZlt{}}\PYG{n+nt}{br}\PYG{p}{/}\PYG{p}{\PYGZgt{}}
\PYG{p}{\PYGZlt{}}\PYG{p}{/}\PYG{n+nt}{fieldset}\PYG{p}{\PYGZgt{}}
\PYG{p}{\PYGZlt{}}\PYG{p}{/}\PYG{n+nt}{form}\PYG{p}{\PYGZgt{}}
\end{sphinxVerbatim}


\section{Formulario 11}
\label{\detokenize{ejercicios/formularios/anexo_formularios:formulario-11}}
Generar el formulario siguiente de acuerdo a los siguientes requisitos
\begin{itemize}
\item {} 
Hay una lista desplegable múltiple con el \sphinxcode{name} «lenguaje» y con las siguientes opciones: opción «Python» con el \sphinxcode{value} python, opción «HTML» con el \sphinxcode{value} html, opción «Visual Basic» con el \sphinxcode{value} visual\_basic, opción «Java» con el \sphinxcode{value} java.

\item {} 
Hay una lista desplegable múltiple con el \sphinxcode{name} «lenguaje» y con las siguientes opciones: opción «Java» con el \sphinxcode{value} java, opción «Python» con el \sphinxcode{value} python, opción «HTML» con el \sphinxcode{value} html, opción «Visual Basic» con el \sphinxcode{value} visual\_basic.

\item {} 
Hay los siguientes cuadros de texto:cuadro de texto con el texto «Instituto» y el \sphinxcode{name} instituto, cuadro de texto con el texto «Estudios elegidos» y el \sphinxcode{name} estudios

\item {} 
Contiene los siguientes \sphinxcode{radiobuttons}:radio con el \sphinxcode{name}  «idioma» , \sphinxcode{value}  «idiomaingles»  y el texto «Inglés», radio con el \sphinxcode{name}  «idioma» , \sphinxcode{value}  «idiomaaleman»  y el texto «Alemán», radio con el \sphinxcode{name}  «idioma» , \sphinxcode{value}  «idiomafrances»  y el texto «Francés».

\item {} 
Hay un \sphinxcode{textarea} que mide 8 filas y 60 columnas que lleva dentro el texto «Inserte aqui el texto»

\end{itemize}

\noindent{\hspace*{\fill}\sphinxincludegraphics[scale=0.6]{{foto_formulario_11}.png}\hspace*{\fill}}

Solución:

\begin{sphinxVerbatim}[commandchars=\\\{\}]
\PYG{p}{\PYGZlt{}}\PYG{n+nt}{form}\PYG{p}{\PYGZgt{}}
\PYG{p}{\PYGZlt{}}\PYG{n+nt}{fieldset}\PYG{p}{\PYGZgt{}}
  \PYG{p}{\PYGZlt{}}\PYG{n+nt}{legend}\PYG{p}{\PYGZgt{}}Rellenar\PYG{p}{\PYGZlt{}}\PYG{p}{/}\PYG{n+nt}{legend}\PYG{p}{\PYGZgt{}}
  \PYG{p}{\PYGZlt{}}\PYG{n+nt}{select} \PYG{n+na}{name}\PYG{o}{=}\PYG{l+s}{\PYGZsq{}lenguaje\PYGZsq{}} \PYG{p}{\PYGZgt{}}
    \PYG{p}{\PYGZlt{}}\PYG{n+nt}{option} \PYG{n+na}{value}\PYG{o}{=}\PYG{l+s}{\PYGZsq{}python\PYGZsq{}}\PYG{p}{\PYGZgt{}}Python\PYG{p}{\PYGZlt{}}\PYG{p}{/}\PYG{n+nt}{option}\PYG{p}{\PYGZgt{}}
    \PYG{p}{\PYGZlt{}}\PYG{n+nt}{option} \PYG{n+na}{value}\PYG{o}{=}\PYG{l+s}{\PYGZsq{}html\PYGZsq{}}\PYG{p}{\PYGZgt{}}HTML\PYG{p}{\PYGZlt{}}\PYG{p}{/}\PYG{n+nt}{option}\PYG{p}{\PYGZgt{}}
    \PYG{p}{\PYGZlt{}}\PYG{n+nt}{option} \PYG{n+na}{value}\PYG{o}{=}\PYG{l+s}{\PYGZsq{}visual\PYGZus{}basic\PYGZsq{}}\PYG{p}{\PYGZgt{}}Visual Basic\PYG{p}{\PYGZlt{}}\PYG{p}{/}\PYG{n+nt}{option}\PYG{p}{\PYGZgt{}}
    \PYG{p}{\PYGZlt{}}\PYG{n+nt}{option} \PYG{n+na}{value}\PYG{o}{=}\PYG{l+s}{\PYGZsq{}java\PYGZsq{}}\PYG{p}{\PYGZgt{}}Java\PYG{p}{\PYGZlt{}}\PYG{p}{/}\PYG{n+nt}{option}\PYG{p}{\PYGZgt{}}
  \PYG{p}{\PYGZlt{}}\PYG{p}{/}\PYG{n+nt}{select}\PYG{p}{\PYGZgt{}}
  \PYG{p}{\PYGZlt{}}\PYG{n+nt}{br}\PYG{p}{/}\PYG{p}{\PYGZgt{}}
  \PYG{p}{\PYGZlt{}}\PYG{n+nt}{select} \PYG{n+na}{name}\PYG{o}{=}\PYG{l+s}{\PYGZsq{}lenguaje\PYGZsq{}} \PYG{n+na}{multiple}\PYG{o}{=}\PYG{l+s}{\PYGZsq{}multiple\PYGZsq{}}\PYG{p}{\PYGZgt{}}
    \PYG{p}{\PYGZlt{}}\PYG{n+nt}{option} \PYG{n+na}{value}\PYG{o}{=}\PYG{l+s}{\PYGZsq{}java\PYGZsq{}}\PYG{p}{\PYGZgt{}}Java\PYG{p}{\PYGZlt{}}\PYG{p}{/}\PYG{n+nt}{option}\PYG{p}{\PYGZgt{}}
    \PYG{p}{\PYGZlt{}}\PYG{n+nt}{option} \PYG{n+na}{value}\PYG{o}{=}\PYG{l+s}{\PYGZsq{}python\PYGZsq{}}\PYG{p}{\PYGZgt{}}Python\PYG{p}{\PYGZlt{}}\PYG{p}{/}\PYG{n+nt}{option}\PYG{p}{\PYGZgt{}}
    \PYG{p}{\PYGZlt{}}\PYG{n+nt}{option} \PYG{n+na}{value}\PYG{o}{=}\PYG{l+s}{\PYGZsq{}html\PYGZsq{}}\PYG{p}{\PYGZgt{}}HTML\PYG{p}{\PYGZlt{}}\PYG{p}{/}\PYG{n+nt}{option}\PYG{p}{\PYGZgt{}}
    \PYG{p}{\PYGZlt{}}\PYG{n+nt}{option} \PYG{n+na}{value}\PYG{o}{=}\PYG{l+s}{\PYGZsq{}visual\PYGZus{}basic\PYGZsq{}}\PYG{p}{\PYGZgt{}}Visual Basic\PYG{p}{\PYGZlt{}}\PYG{p}{/}\PYG{n+nt}{option}\PYG{p}{\PYGZgt{}}
  \PYG{p}{\PYGZlt{}}\PYG{p}{/}\PYG{n+nt}{select}\PYG{p}{\PYGZgt{}}
  \PYG{p}{\PYGZlt{}}\PYG{n+nt}{br}\PYG{p}{/}\PYG{p}{\PYGZgt{}}
  Instituto\PYG{p}{\PYGZlt{}}\PYG{n+nt}{input} \PYG{n+na}{type}\PYG{o}{=}\PYG{l+s}{\PYGZsq{}text\PYGZsq{}} \PYG{n+na}{name}\PYG{o}{=}\PYG{l+s}{\PYGZsq{}instituto\PYGZsq{}}\PYG{p}{\PYGZgt{}} \PYG{p}{\PYGZlt{}}\PYG{n+nt}{br}\PYG{p}{/}\PYG{p}{\PYGZgt{}}
  Estudios elegidos\PYG{p}{\PYGZlt{}}\PYG{n+nt}{input} \PYG{n+na}{type}\PYG{o}{=}\PYG{l+s}{\PYGZsq{}text\PYGZsq{}} \PYG{n+na}{name}\PYG{o}{=}\PYG{l+s}{\PYGZsq{}estudios\PYGZsq{}}\PYG{p}{\PYGZgt{}} \PYG{p}{\PYGZlt{}}\PYG{n+nt}{br}\PYG{p}{/}\PYG{p}{\PYGZgt{}}
  \PYG{p}{\PYGZlt{}}\PYG{n+nt}{br}\PYG{p}{/}\PYG{p}{\PYGZgt{}}
  \PYG{p}{\PYGZlt{}}\PYG{n+nt}{input} \PYG{n+na}{type}\PYG{o}{=}\PYG{l+s}{\PYGZsq{}radio\PYGZsq{}} \PYG{n+na}{name}\PYG{o}{=}\PYG{l+s}{\PYGZsq{}idioma\PYGZsq{}} \PYG{n+na}{value}\PYG{o}{=}\PYG{l+s}{\PYGZsq{}idiomaingles\PYGZsq{}}\PYG{p}{\PYGZgt{}} Inglés   \PYG{p}{\PYGZlt{}}\PYG{n+nt}{br}\PYG{p}{/}\PYG{p}{\PYGZgt{}}
  \PYG{p}{\PYGZlt{}}\PYG{n+nt}{input} \PYG{n+na}{type}\PYG{o}{=}\PYG{l+s}{\PYGZsq{}radio\PYGZsq{}} \PYG{n+na}{name}\PYG{o}{=}\PYG{l+s}{\PYGZsq{}idioma\PYGZsq{}} \PYG{n+na}{value}\PYG{o}{=}\PYG{l+s}{\PYGZsq{}idiomaaleman\PYGZsq{}}\PYG{p}{\PYGZgt{}} Alemán   \PYG{p}{\PYGZlt{}}\PYG{n+nt}{br}\PYG{p}{/}\PYG{p}{\PYGZgt{}}
  \PYG{p}{\PYGZlt{}}\PYG{n+nt}{input} \PYG{n+na}{type}\PYG{o}{=}\PYG{l+s}{\PYGZsq{}radio\PYGZsq{}} \PYG{n+na}{name}\PYG{o}{=}\PYG{l+s}{\PYGZsq{}idioma\PYGZsq{}} \PYG{n+na}{value}\PYG{o}{=}\PYG{l+s}{\PYGZsq{}idiomafrances\PYGZsq{}}\PYG{p}{\PYGZgt{}} Francés   \PYG{p}{\PYGZlt{}}\PYG{n+nt}{br}\PYG{p}{/}\PYG{p}{\PYGZgt{}}
  \PYG{p}{\PYGZlt{}}\PYG{n+nt}{br}\PYG{p}{/}\PYG{p}{\PYGZgt{}}
\PYG{p}{\PYGZlt{}}\PYG{p}{/}\PYG{n+nt}{fieldset}\PYG{p}{\PYGZgt{}}
\PYG{p}{\PYGZlt{}}\PYG{n+nt}{fieldset}\PYG{p}{\PYGZgt{}}
  \PYG{p}{\PYGZlt{}}\PYG{n+nt}{legend}\PYG{p}{\PYGZgt{}}Por favor, complete estas opciones\PYG{p}{\PYGZlt{}}\PYG{p}{/}\PYG{n+nt}{legend}\PYG{p}{\PYGZgt{}}
  \PYG{p}{\PYGZlt{}}\PYG{n+nt}{textarea} \PYG{n+na}{rows}\PYG{o}{=}\PYG{l+s}{\PYGZsq{}8\PYGZsq{}} \PYG{n+na}{cols}\PYG{o}{=}\PYG{l+s}{\PYGZsq{}60\PYGZsq{}}\PYG{p}{\PYGZgt{}}
    Inserte aqui el texto
  \PYG{p}{\PYGZlt{}}\PYG{p}{/}\PYG{n+nt}{textarea}\PYG{p}{\PYGZgt{}}  \PYG{p}{\PYGZlt{}}\PYG{n+nt}{br}\PYG{p}{/}\PYG{p}{\PYGZgt{}}
\PYG{p}{\PYGZlt{}}\PYG{p}{/}\PYG{n+nt}{fieldset}\PYG{p}{\PYGZgt{}}
\PYG{p}{\PYGZlt{}}\PYG{p}{/}\PYG{n+nt}{form}\PYG{p}{\PYGZgt{}}
\end{sphinxVerbatim}


\section{Formulario 12}
\label{\detokenize{ejercicios/formularios/anexo_formularios:formulario-12}}
Generar el formulario siguiente de acuerdo a los siguientes requisitos
\begin{itemize}
\item {} 
Contiene los siguientes \sphinxcode{radiobuttons}:radio con el \sphinxcode{name}  «procesador» , \sphinxcode{value}  «procesadoramd»  y el texto «AMD», radio con el \sphinxcode{name}  «procesador» , \sphinxcode{value}  «procesadorintel\_i5»  y el texto «Intel i5», radio con el \sphinxcode{name}  «procesador» , \sphinxcode{value}  «procesadorintel\_i7»  y el texto «Intel i7».

\item {} 
Contiene los siguientes \sphinxcode{radiobuttons}:radio con el \sphinxcode{name}  «idioma» , \sphinxcode{value}  «idiomaespanol»  y el texto «Español», radio con el \sphinxcode{name}  «idioma» , \sphinxcode{value}  «idiomaingles»  y el texto «Inglés», radio con el \sphinxcode{name}  «idioma» , \sphinxcode{value}  «idiomaaleman»  y el texto «Alemán», radio con el \sphinxcode{name}  «idioma» , \sphinxcode{value}  «idiomafrances»  y el texto «Francés».

\item {} 
Hay una lista desplegable múltiple con el \sphinxcode{name} «medio\_transporte» y con las siguientes opciones: opción «Automóvil» con el \sphinxcode{value} automovil, opción «Moto» con el \sphinxcode{value} moto, opción «Autobús» con el \sphinxcode{value} autobus.

\item {} 
Hay los siguientes cuadros de texto:cuadro de texto con el texto «Nombre» y el \sphinxcode{name} nombre, cuadro de texto con el texto «Apellidos» y el \sphinxcode{name} apellidos, cuadro de texto con el texto «Direccion» y el \sphinxcode{name} direccion

\item {} 
Contiene los siguientes \sphinxcode{radiobuttons}:radio con el \sphinxcode{name}  «lenguaje» , \sphinxcode{value}  «lenguajejava»  y el texto «Java», radio con el \sphinxcode{name}  «lenguaje» , \sphinxcode{value}  «lenguajepython»  y el texto «Python», radio con el \sphinxcode{name}  «lenguaje» , \sphinxcode{value}  «lenguajehtml»  y el texto «HTML», radio con el \sphinxcode{name}  «lenguaje» , \sphinxcode{value}  «lenguajevisual\_basic»  y el texto «Visual Basic».

\item {} 
Hay un \sphinxcode{textarea} que mide 4 filas y 49 columnas que lleva dentro el texto «Inserte aqui el texto»

\item {} 
Contiene los siguientes \sphinxcode{radiobuttons}:radio con el \sphinxcode{name}  «ide» , \sphinxcode{value}  «ideeclipse»  y el texto «Eclipse», radio con el \sphinxcode{name}  «ide» , \sphinxcode{value}  «idenetbeans»  y el texto «Netbeans», radio con el \sphinxcode{name}  «ide» , \sphinxcode{value}  «idenotepad»  y el texto «Notepad», radio con el \sphinxcode{name}  «ide» , \sphinxcode{value}  «ideidea»  y el texto «IDEA».

\item {} 
Hay una lista desplegable múltiple con el \sphinxcode{name} «ciclo» y con las siguientes opciones: opción «ASIR» con el \sphinxcode{value} asir, opción «DAM» con el \sphinxcode{value} dam, opción «DAW» con el \sphinxcode{value} daw.

\end{itemize}

\noindent{\hspace*{\fill}\sphinxincludegraphics[scale=0.6]{{foto_formulario_12}.png}\hspace*{\fill}}

Solución:

\begin{sphinxVerbatim}[commandchars=\\\{\}]
\PYG{p}{\PYGZlt{}}\PYG{n+nt}{form}\PYG{p}{\PYGZgt{}}
\PYG{p}{\PYGZlt{}}\PYG{n+nt}{fieldset}\PYG{p}{\PYGZgt{}}
  \PYG{p}{\PYGZlt{}}\PYG{n+nt}{legend}\PYG{p}{\PYGZgt{}}Rellene las opciones siguientes\PYG{p}{\PYGZlt{}}\PYG{p}{/}\PYG{n+nt}{legend}\PYG{p}{\PYGZgt{}}
  \PYG{p}{\PYGZlt{}}\PYG{n+nt}{input} \PYG{n+na}{type}\PYG{o}{=}\PYG{l+s}{\PYGZsq{}radio\PYGZsq{}} \PYG{n+na}{name}\PYG{o}{=}\PYG{l+s}{\PYGZsq{}procesador\PYGZsq{}} \PYG{n+na}{value}\PYG{o}{=}\PYG{l+s}{\PYGZsq{}procesadoramd\PYGZsq{}}\PYG{p}{\PYGZgt{}} AMD   \PYG{p}{\PYGZlt{}}\PYG{n+nt}{br}\PYG{p}{/}\PYG{p}{\PYGZgt{}}
  \PYG{p}{\PYGZlt{}}\PYG{n+nt}{input} \PYG{n+na}{type}\PYG{o}{=}\PYG{l+s}{\PYGZsq{}radio\PYGZsq{}} \PYG{n+na}{name}\PYG{o}{=}\PYG{l+s}{\PYGZsq{}procesador\PYGZsq{}} \PYG{n+na}{value}\PYG{o}{=}\PYG{l+s}{\PYGZsq{}procesadorintel\PYGZus{}i5\PYGZsq{}}\PYG{p}{\PYGZgt{}} Intel i5   \PYG{p}{\PYGZlt{}}\PYG{n+nt}{br}\PYG{p}{/}\PYG{p}{\PYGZgt{}}
  \PYG{p}{\PYGZlt{}}\PYG{n+nt}{input} \PYG{n+na}{type}\PYG{o}{=}\PYG{l+s}{\PYGZsq{}radio\PYGZsq{}} \PYG{n+na}{name}\PYG{o}{=}\PYG{l+s}{\PYGZsq{}procesador\PYGZsq{}} \PYG{n+na}{value}\PYG{o}{=}\PYG{l+s}{\PYGZsq{}procesadorintel\PYGZus{}i7\PYGZsq{}}\PYG{p}{\PYGZgt{}} Intel i7   \PYG{p}{\PYGZlt{}}\PYG{n+nt}{br}\PYG{p}{/}\PYG{p}{\PYGZgt{}}
  \PYG{p}{\PYGZlt{}}\PYG{n+nt}{br}\PYG{p}{/}\PYG{p}{\PYGZgt{}}
  \PYG{p}{\PYGZlt{}}\PYG{n+nt}{input} \PYG{n+na}{type}\PYG{o}{=}\PYG{l+s}{\PYGZsq{}radio\PYGZsq{}} \PYG{n+na}{name}\PYG{o}{=}\PYG{l+s}{\PYGZsq{}idioma\PYGZsq{}} \PYG{n+na}{value}\PYG{o}{=}\PYG{l+s}{\PYGZsq{}idiomaespanol\PYGZsq{}}\PYG{p}{\PYGZgt{}} Español 
  \PYG{p}{\PYGZlt{}}\PYG{n+nt}{input} \PYG{n+na}{type}\PYG{o}{=}\PYG{l+s}{\PYGZsq{}radio\PYGZsq{}} \PYG{n+na}{name}\PYG{o}{=}\PYG{l+s}{\PYGZsq{}idioma\PYGZsq{}} \PYG{n+na}{value}\PYG{o}{=}\PYG{l+s}{\PYGZsq{}idiomaingles\PYGZsq{}}\PYG{p}{\PYGZgt{}} Inglés 
  \PYG{p}{\PYGZlt{}}\PYG{n+nt}{input} \PYG{n+na}{type}\PYG{o}{=}\PYG{l+s}{\PYGZsq{}radio\PYGZsq{}} \PYG{n+na}{name}\PYG{o}{=}\PYG{l+s}{\PYGZsq{}idioma\PYGZsq{}} \PYG{n+na}{value}\PYG{o}{=}\PYG{l+s}{\PYGZsq{}idiomaaleman\PYGZsq{}}\PYG{p}{\PYGZgt{}} Alemán 
  \PYG{p}{\PYGZlt{}}\PYG{n+nt}{input} \PYG{n+na}{type}\PYG{o}{=}\PYG{l+s}{\PYGZsq{}radio\PYGZsq{}} \PYG{n+na}{name}\PYG{o}{=}\PYG{l+s}{\PYGZsq{}idioma\PYGZsq{}} \PYG{n+na}{value}\PYG{o}{=}\PYG{l+s}{\PYGZsq{}idiomafrances\PYGZsq{}}\PYG{p}{\PYGZgt{}} Francés 
  \PYG{p}{\PYGZlt{}}\PYG{n+nt}{br}\PYG{p}{/}\PYG{p}{\PYGZgt{}}
  \PYG{p}{\PYGZlt{}}\PYG{n+nt}{select} \PYG{n+na}{name}\PYG{o}{=}\PYG{l+s}{\PYGZsq{}medio\PYGZus{}transporte\PYGZsq{}} \PYG{n+na}{multiple}\PYG{o}{=}\PYG{l+s}{\PYGZsq{}multiple\PYGZsq{}}\PYG{p}{\PYGZgt{}}
    \PYG{p}{\PYGZlt{}}\PYG{n+nt}{option} \PYG{n+na}{value}\PYG{o}{=}\PYG{l+s}{\PYGZsq{}automovil\PYGZsq{}}\PYG{p}{\PYGZgt{}}Automóvil\PYG{p}{\PYGZlt{}}\PYG{p}{/}\PYG{n+nt}{option}\PYG{p}{\PYGZgt{}}
    \PYG{p}{\PYGZlt{}}\PYG{n+nt}{option} \PYG{n+na}{value}\PYG{o}{=}\PYG{l+s}{\PYGZsq{}moto\PYGZsq{}}\PYG{p}{\PYGZgt{}}Moto\PYG{p}{\PYGZlt{}}\PYG{p}{/}\PYG{n+nt}{option}\PYG{p}{\PYGZgt{}}
    \PYG{p}{\PYGZlt{}}\PYG{n+nt}{option} \PYG{n+na}{value}\PYG{o}{=}\PYG{l+s}{\PYGZsq{}autobus\PYGZsq{}}\PYG{p}{\PYGZgt{}}Autobús\PYG{p}{\PYGZlt{}}\PYG{p}{/}\PYG{n+nt}{option}\PYG{p}{\PYGZgt{}}
  \PYG{p}{\PYGZlt{}}\PYG{p}{/}\PYG{n+nt}{select}\PYG{p}{\PYGZgt{}}
  \PYG{p}{\PYGZlt{}}\PYG{n+nt}{br}\PYG{p}{/}\PYG{p}{\PYGZgt{}}
  Nombre\PYG{p}{\PYGZlt{}}\PYG{n+nt}{input} \PYG{n+na}{type}\PYG{o}{=}\PYG{l+s}{\PYGZsq{}text\PYGZsq{}} \PYG{n+na}{name}\PYG{o}{=}\PYG{l+s}{\PYGZsq{}nombre\PYGZsq{}}\PYG{p}{\PYGZgt{}}
  Apellidos\PYG{p}{\PYGZlt{}}\PYG{n+nt}{input} \PYG{n+na}{type}\PYG{o}{=}\PYG{l+s}{\PYGZsq{}text\PYGZsq{}} \PYG{n+na}{name}\PYG{o}{=}\PYG{l+s}{\PYGZsq{}apellidos\PYGZsq{}}\PYG{p}{\PYGZgt{}}
  Direccion\PYG{p}{\PYGZlt{}}\PYG{n+nt}{input} \PYG{n+na}{type}\PYG{o}{=}\PYG{l+s}{\PYGZsq{}text\PYGZsq{}} \PYG{n+na}{name}\PYG{o}{=}\PYG{l+s}{\PYGZsq{}direccion\PYGZsq{}}\PYG{p}{\PYGZgt{}}
  \PYG{p}{\PYGZlt{}}\PYG{n+nt}{br}\PYG{p}{/}\PYG{p}{\PYGZgt{}}
\PYG{p}{\PYGZlt{}}\PYG{p}{/}\PYG{n+nt}{fieldset}\PYG{p}{\PYGZgt{}}
\PYG{p}{\PYGZlt{}}\PYG{n+nt}{fieldset}\PYG{p}{\PYGZgt{}}
  \PYG{p}{\PYGZlt{}}\PYG{n+nt}{legend}\PYG{p}{\PYGZgt{}}Rellenar\PYG{p}{\PYGZlt{}}\PYG{p}{/}\PYG{n+nt}{legend}\PYG{p}{\PYGZgt{}}
  \PYG{p}{\PYGZlt{}}\PYG{n+nt}{input} \PYG{n+na}{type}\PYG{o}{=}\PYG{l+s}{\PYGZsq{}radio\PYGZsq{}} \PYG{n+na}{name}\PYG{o}{=}\PYG{l+s}{\PYGZsq{}lenguaje\PYGZsq{}} \PYG{n+na}{value}\PYG{o}{=}\PYG{l+s}{\PYGZsq{}lenguajejava\PYGZsq{}}\PYG{p}{\PYGZgt{}} Java 
  \PYG{p}{\PYGZlt{}}\PYG{n+nt}{input} \PYG{n+na}{type}\PYG{o}{=}\PYG{l+s}{\PYGZsq{}radio\PYGZsq{}} \PYG{n+na}{name}\PYG{o}{=}\PYG{l+s}{\PYGZsq{}lenguaje\PYGZsq{}} \PYG{n+na}{value}\PYG{o}{=}\PYG{l+s}{\PYGZsq{}lenguajepython\PYGZsq{}}\PYG{p}{\PYGZgt{}} Python 
  \PYG{p}{\PYGZlt{}}\PYG{n+nt}{input} \PYG{n+na}{type}\PYG{o}{=}\PYG{l+s}{\PYGZsq{}radio\PYGZsq{}} \PYG{n+na}{name}\PYG{o}{=}\PYG{l+s}{\PYGZsq{}lenguaje\PYGZsq{}} \PYG{n+na}{value}\PYG{o}{=}\PYG{l+s}{\PYGZsq{}lenguajehtml\PYGZsq{}}\PYG{p}{\PYGZgt{}} HTML 
  \PYG{p}{\PYGZlt{}}\PYG{n+nt}{input} \PYG{n+na}{type}\PYG{o}{=}\PYG{l+s}{\PYGZsq{}radio\PYGZsq{}} \PYG{n+na}{name}\PYG{o}{=}\PYG{l+s}{\PYGZsq{}lenguaje\PYGZsq{}} \PYG{n+na}{value}\PYG{o}{=}\PYG{l+s}{\PYGZsq{}lenguajevisual\PYGZus{}basic\PYGZsq{}}\PYG{p}{\PYGZgt{}} Visual Basic 
  \PYG{p}{\PYGZlt{}}\PYG{n+nt}{br}\PYG{p}{/}\PYG{p}{\PYGZgt{}}
  \PYG{p}{\PYGZlt{}}\PYG{n+nt}{textarea} \PYG{n+na}{rows}\PYG{o}{=}\PYG{l+s}{\PYGZsq{}4\PYGZsq{}} \PYG{n+na}{cols}\PYG{o}{=}\PYG{l+s}{\PYGZsq{}49\PYGZsq{}}\PYG{p}{\PYGZgt{}}
    Inserte aqui el texto
  \PYG{p}{\PYGZlt{}}\PYG{p}{/}\PYG{n+nt}{textarea}\PYG{p}{\PYGZgt{}}  \PYG{p}{\PYGZlt{}}\PYG{n+nt}{br}\PYG{p}{/}\PYG{p}{\PYGZgt{}}
  \PYG{p}{\PYGZlt{}}\PYG{n+nt}{input} \PYG{n+na}{type}\PYG{o}{=}\PYG{l+s}{\PYGZsq{}radio\PYGZsq{}} \PYG{n+na}{name}\PYG{o}{=}\PYG{l+s}{\PYGZsq{}ide\PYGZsq{}} \PYG{n+na}{value}\PYG{o}{=}\PYG{l+s}{\PYGZsq{}ideeclipse\PYGZsq{}}\PYG{p}{\PYGZgt{}} Eclipse   \PYG{p}{\PYGZlt{}}\PYG{n+nt}{br}\PYG{p}{/}\PYG{p}{\PYGZgt{}}
  \PYG{p}{\PYGZlt{}}\PYG{n+nt}{input} \PYG{n+na}{type}\PYG{o}{=}\PYG{l+s}{\PYGZsq{}radio\PYGZsq{}} \PYG{n+na}{name}\PYG{o}{=}\PYG{l+s}{\PYGZsq{}ide\PYGZsq{}} \PYG{n+na}{value}\PYG{o}{=}\PYG{l+s}{\PYGZsq{}idenetbeans\PYGZsq{}}\PYG{p}{\PYGZgt{}} Netbeans   \PYG{p}{\PYGZlt{}}\PYG{n+nt}{br}\PYG{p}{/}\PYG{p}{\PYGZgt{}}
  \PYG{p}{\PYGZlt{}}\PYG{n+nt}{input} \PYG{n+na}{type}\PYG{o}{=}\PYG{l+s}{\PYGZsq{}radio\PYGZsq{}} \PYG{n+na}{name}\PYG{o}{=}\PYG{l+s}{\PYGZsq{}ide\PYGZsq{}} \PYG{n+na}{value}\PYG{o}{=}\PYG{l+s}{\PYGZsq{}idenotepad\PYGZsq{}}\PYG{p}{\PYGZgt{}} Notepad   \PYG{p}{\PYGZlt{}}\PYG{n+nt}{br}\PYG{p}{/}\PYG{p}{\PYGZgt{}}
  \PYG{p}{\PYGZlt{}}\PYG{n+nt}{input} \PYG{n+na}{type}\PYG{o}{=}\PYG{l+s}{\PYGZsq{}radio\PYGZsq{}} \PYG{n+na}{name}\PYG{o}{=}\PYG{l+s}{\PYGZsq{}ide\PYGZsq{}} \PYG{n+na}{value}\PYG{o}{=}\PYG{l+s}{\PYGZsq{}ideidea\PYGZsq{}}\PYG{p}{\PYGZgt{}} IDEA   \PYG{p}{\PYGZlt{}}\PYG{n+nt}{br}\PYG{p}{/}\PYG{p}{\PYGZgt{}}
  \PYG{p}{\PYGZlt{}}\PYG{n+nt}{br}\PYG{p}{/}\PYG{p}{\PYGZgt{}}
  \PYG{p}{\PYGZlt{}}\PYG{n+nt}{select} \PYG{n+na}{name}\PYG{o}{=}\PYG{l+s}{\PYGZsq{}ciclo\PYGZsq{}} \PYG{p}{\PYGZgt{}}
    \PYG{p}{\PYGZlt{}}\PYG{n+nt}{option} \PYG{n+na}{value}\PYG{o}{=}\PYG{l+s}{\PYGZsq{}asir\PYGZsq{}}\PYG{p}{\PYGZgt{}}ASIR\PYG{p}{\PYGZlt{}}\PYG{p}{/}\PYG{n+nt}{option}\PYG{p}{\PYGZgt{}}
    \PYG{p}{\PYGZlt{}}\PYG{n+nt}{option} \PYG{n+na}{value}\PYG{o}{=}\PYG{l+s}{\PYGZsq{}dam\PYGZsq{}}\PYG{p}{\PYGZgt{}}DAM\PYG{p}{\PYGZlt{}}\PYG{p}{/}\PYG{n+nt}{option}\PYG{p}{\PYGZgt{}}
    \PYG{p}{\PYGZlt{}}\PYG{n+nt}{option} \PYG{n+na}{value}\PYG{o}{=}\PYG{l+s}{\PYGZsq{}daw\PYGZsq{}}\PYG{p}{\PYGZgt{}}DAW\PYG{p}{\PYGZlt{}}\PYG{p}{/}\PYG{n+nt}{option}\PYG{p}{\PYGZgt{}}
  \PYG{p}{\PYGZlt{}}\PYG{p}{/}\PYG{n+nt}{select}\PYG{p}{\PYGZgt{}}
  \PYG{p}{\PYGZlt{}}\PYG{n+nt}{br}\PYG{p}{/}\PYG{p}{\PYGZgt{}}
\PYG{p}{\PYGZlt{}}\PYG{p}{/}\PYG{n+nt}{fieldset}\PYG{p}{\PYGZgt{}}
\PYG{p}{\PYGZlt{}}\PYG{p}{/}\PYG{n+nt}{form}\PYG{p}{\PYGZgt{}}
\end{sphinxVerbatim}


\section{Formulario 13}
\label{\detokenize{ejercicios/formularios/anexo_formularios:formulario-13}}
Generar el formulario siguiente de acuerdo a los siguientes requisitos
\begin{itemize}
\item {} 
Hay un \sphinxcode{textarea} que mide 4 filas y 56 columnas que lleva dentro el texto «Utilice este cuadro para escribir»

\item {} 
Hay los siguientes cuadros de texto:cuadro de texto con el texto «Nombre» y el \sphinxcode{name} nombre, cuadro de texto con el texto «Apellidos» y el \sphinxcode{name} apellidos

\item {} 
Contiene los siguientes \sphinxcode{checkboxes}:checkbox con el \sphinxcode{name}  «dia» , \sphinxcode{value}  «dialunes»  y el texto «Lunes», checkbox con el \sphinxcode{name}  «dia» , \sphinxcode{value}  «diamartes»  y el texto «Martes», checkbox con el \sphinxcode{name}  «dia» , \sphinxcode{value}  «diamiercoles»  y el texto «Miércoles», checkbox con el \sphinxcode{name}  «dia» , \sphinxcode{value}  «diajueves»  y el texto «Jueves», checkbox con el \sphinxcode{name}  «dia» , \sphinxcode{value}  «diasabado»  y el texto «Sabado».

\item {} 
Contiene los siguientes \sphinxcode{checkboxes}:checkbox con el \sphinxcode{name}  «idioma» , \sphinxcode{value}  «idiomaingles»  y el texto «Inglés», checkbox con el \sphinxcode{name}  «idioma» , \sphinxcode{value}  «idiomaaleman»  y el texto «Alemán», checkbox con el \sphinxcode{name}  «idioma» , \sphinxcode{value}  «idiomafrances»  y el texto «Francés».

\item {} 
Contiene los siguientes \sphinxcode{checkboxes}:checkbox con el \sphinxcode{name}  «red» , \sphinxcode{value}  «red2g»  y el texto «2G», checkbox con el \sphinxcode{name}  «red» , \sphinxcode{value}  «red3g»  y el texto «3G», checkbox con el \sphinxcode{name}  «red» , \sphinxcode{value}  «red4g»  y el texto «4G».

\item {} 
Hay una lista desplegable múltiple con el \sphinxcode{name} «idioma» y con las siguientes opciones: opción «Inglés» con el \sphinxcode{value} ingles, opción «Alemán» con el \sphinxcode{value} aleman, opción «Francés» con el \sphinxcode{value} frances.

\item {} 
Contiene los siguientes \sphinxcode{radiobuttons}:radio con el \sphinxcode{name}  «procesador» , \sphinxcode{value}  «procesadorintel»  y el texto «Intel», radio con el \sphinxcode{name}  «procesador» , \sphinxcode{value}  «procesadoramd»  y el texto «AMD».

\end{itemize}

\noindent{\hspace*{\fill}\sphinxincludegraphics[scale=0.6]{{foto_formulario_13}.png}\hspace*{\fill}}

Solución:

\begin{sphinxVerbatim}[commandchars=\\\{\}]
\PYG{p}{\PYGZlt{}}\PYG{n+nt}{form}\PYG{p}{\PYGZgt{}}
\PYG{p}{\PYGZlt{}}\PYG{n+nt}{fieldset}\PYG{p}{\PYGZgt{}}
  \PYG{p}{\PYGZlt{}}\PYG{n+nt}{legend}\PYG{p}{\PYGZgt{}}Indique\PYG{p}{\PYGZlt{}}\PYG{p}{/}\PYG{n+nt}{legend}\PYG{p}{\PYGZgt{}}
  \PYG{p}{\PYGZlt{}}\PYG{n+nt}{textarea} \PYG{n+na}{rows}\PYG{o}{=}\PYG{l+s}{\PYGZsq{}4\PYGZsq{}} \PYG{n+na}{cols}\PYG{o}{=}\PYG{l+s}{\PYGZsq{}56\PYGZsq{}}\PYG{p}{\PYGZgt{}}
    Utilice este cuadro para escribir
  \PYG{p}{\PYGZlt{}}\PYG{p}{/}\PYG{n+nt}{textarea}\PYG{p}{\PYGZgt{}}  \PYG{p}{\PYGZlt{}}\PYG{n+nt}{br}\PYG{p}{/}\PYG{p}{\PYGZgt{}}
  Nombre\PYG{p}{\PYGZlt{}}\PYG{n+nt}{input} \PYG{n+na}{type}\PYG{o}{=}\PYG{l+s}{\PYGZsq{}text\PYGZsq{}} \PYG{n+na}{name}\PYG{o}{=}\PYG{l+s}{\PYGZsq{}nombre\PYGZsq{}}\PYG{p}{\PYGZgt{}} \PYG{p}{\PYGZlt{}}\PYG{n+nt}{br}\PYG{p}{/}\PYG{p}{\PYGZgt{}}
  Apellidos\PYG{p}{\PYGZlt{}}\PYG{n+nt}{input} \PYG{n+na}{type}\PYG{o}{=}\PYG{l+s}{\PYGZsq{}text\PYGZsq{}} \PYG{n+na}{name}\PYG{o}{=}\PYG{l+s}{\PYGZsq{}apellidos\PYGZsq{}}\PYG{p}{\PYGZgt{}} \PYG{p}{\PYGZlt{}}\PYG{n+nt}{br}\PYG{p}{/}\PYG{p}{\PYGZgt{}}
  \PYG{p}{\PYGZlt{}}\PYG{n+nt}{br}\PYG{p}{/}\PYG{p}{\PYGZgt{}}
  \PYG{p}{\PYGZlt{}}\PYG{n+nt}{input} \PYG{n+na}{type}\PYG{o}{=}\PYG{l+s}{\PYGZsq{}checkbox\PYGZsq{}} \PYG{n+na}{name}\PYG{o}{=}\PYG{l+s}{\PYGZsq{}dia\PYGZsq{}} \PYG{n+na}{value}\PYG{o}{=}\PYG{l+s}{\PYGZsq{}dialunes\PYGZsq{}}\PYG{p}{\PYGZgt{}} Lunes   \PYG{p}{\PYGZlt{}}\PYG{n+nt}{br}\PYG{p}{/}\PYG{p}{\PYGZgt{}}
  \PYG{p}{\PYGZlt{}}\PYG{n+nt}{input} \PYG{n+na}{type}\PYG{o}{=}\PYG{l+s}{\PYGZsq{}checkbox\PYGZsq{}} \PYG{n+na}{name}\PYG{o}{=}\PYG{l+s}{\PYGZsq{}dia\PYGZsq{}} \PYG{n+na}{value}\PYG{o}{=}\PYG{l+s}{\PYGZsq{}diamartes\PYGZsq{}}\PYG{p}{\PYGZgt{}} Martes   \PYG{p}{\PYGZlt{}}\PYG{n+nt}{br}\PYG{p}{/}\PYG{p}{\PYGZgt{}}
  \PYG{p}{\PYGZlt{}}\PYG{n+nt}{input} \PYG{n+na}{type}\PYG{o}{=}\PYG{l+s}{\PYGZsq{}checkbox\PYGZsq{}} \PYG{n+na}{name}\PYG{o}{=}\PYG{l+s}{\PYGZsq{}dia\PYGZsq{}} \PYG{n+na}{value}\PYG{o}{=}\PYG{l+s}{\PYGZsq{}diamiercoles\PYGZsq{}}\PYG{p}{\PYGZgt{}} Miércoles   \PYG{p}{\PYGZlt{}}\PYG{n+nt}{br}\PYG{p}{/}\PYG{p}{\PYGZgt{}}
  \PYG{p}{\PYGZlt{}}\PYG{n+nt}{input} \PYG{n+na}{type}\PYG{o}{=}\PYG{l+s}{\PYGZsq{}checkbox\PYGZsq{}} \PYG{n+na}{name}\PYG{o}{=}\PYG{l+s}{\PYGZsq{}dia\PYGZsq{}} \PYG{n+na}{value}\PYG{o}{=}\PYG{l+s}{\PYGZsq{}diajueves\PYGZsq{}}\PYG{p}{\PYGZgt{}} Jueves   \PYG{p}{\PYGZlt{}}\PYG{n+nt}{br}\PYG{p}{/}\PYG{p}{\PYGZgt{}}
  \PYG{p}{\PYGZlt{}}\PYG{n+nt}{input} \PYG{n+na}{type}\PYG{o}{=}\PYG{l+s}{\PYGZsq{}checkbox\PYGZsq{}} \PYG{n+na}{name}\PYG{o}{=}\PYG{l+s}{\PYGZsq{}dia\PYGZsq{}} \PYG{n+na}{value}\PYG{o}{=}\PYG{l+s}{\PYGZsq{}diasabado\PYGZsq{}}\PYG{p}{\PYGZgt{}} Sabado   \PYG{p}{\PYGZlt{}}\PYG{n+nt}{br}\PYG{p}{/}\PYG{p}{\PYGZgt{}}
  \PYG{p}{\PYGZlt{}}\PYG{n+nt}{br}\PYG{p}{/}\PYG{p}{\PYGZgt{}}
\PYG{p}{\PYGZlt{}}\PYG{p}{/}\PYG{n+nt}{fieldset}\PYG{p}{\PYGZgt{}}
\PYG{p}{\PYGZlt{}}\PYG{n+nt}{fieldset}\PYG{p}{\PYGZgt{}}
  \PYG{p}{\PYGZlt{}}\PYG{n+nt}{legend}\PYG{p}{\PYGZgt{}}Rellene las opciones siguientes\PYG{p}{\PYGZlt{}}\PYG{p}{/}\PYG{n+nt}{legend}\PYG{p}{\PYGZgt{}}
  \PYG{p}{\PYGZlt{}}\PYG{n+nt}{input} \PYG{n+na}{type}\PYG{o}{=}\PYG{l+s}{\PYGZsq{}checkbox\PYGZsq{}} \PYG{n+na}{name}\PYG{o}{=}\PYG{l+s}{\PYGZsq{}idioma\PYGZsq{}} \PYG{n+na}{value}\PYG{o}{=}\PYG{l+s}{\PYGZsq{}idiomaingles\PYGZsq{}}\PYG{p}{\PYGZgt{}} Inglés 
  \PYG{p}{\PYGZlt{}}\PYG{n+nt}{input} \PYG{n+na}{type}\PYG{o}{=}\PYG{l+s}{\PYGZsq{}checkbox\PYGZsq{}} \PYG{n+na}{name}\PYG{o}{=}\PYG{l+s}{\PYGZsq{}idioma\PYGZsq{}} \PYG{n+na}{value}\PYG{o}{=}\PYG{l+s}{\PYGZsq{}idiomaaleman\PYGZsq{}}\PYG{p}{\PYGZgt{}} Alemán 
  \PYG{p}{\PYGZlt{}}\PYG{n+nt}{input} \PYG{n+na}{type}\PYG{o}{=}\PYG{l+s}{\PYGZsq{}checkbox\PYGZsq{}} \PYG{n+na}{name}\PYG{o}{=}\PYG{l+s}{\PYGZsq{}idioma\PYGZsq{}} \PYG{n+na}{value}\PYG{o}{=}\PYG{l+s}{\PYGZsq{}idiomafrances\PYGZsq{}}\PYG{p}{\PYGZgt{}} Francés 
  \PYG{p}{\PYGZlt{}}\PYG{n+nt}{br}\PYG{p}{/}\PYG{p}{\PYGZgt{}}
  \PYG{p}{\PYGZlt{}}\PYG{n+nt}{input} \PYG{n+na}{type}\PYG{o}{=}\PYG{l+s}{\PYGZsq{}checkbox\PYGZsq{}} \PYG{n+na}{name}\PYG{o}{=}\PYG{l+s}{\PYGZsq{}red\PYGZsq{}} \PYG{n+na}{value}\PYG{o}{=}\PYG{l+s}{\PYGZsq{}red2g\PYGZsq{}}\PYG{p}{\PYGZgt{}} 2G   \PYG{p}{\PYGZlt{}}\PYG{n+nt}{br}\PYG{p}{/}\PYG{p}{\PYGZgt{}}
  \PYG{p}{\PYGZlt{}}\PYG{n+nt}{input} \PYG{n+na}{type}\PYG{o}{=}\PYG{l+s}{\PYGZsq{}checkbox\PYGZsq{}} \PYG{n+na}{name}\PYG{o}{=}\PYG{l+s}{\PYGZsq{}red\PYGZsq{}} \PYG{n+na}{value}\PYG{o}{=}\PYG{l+s}{\PYGZsq{}red3g\PYGZsq{}}\PYG{p}{\PYGZgt{}} 3G   \PYG{p}{\PYGZlt{}}\PYG{n+nt}{br}\PYG{p}{/}\PYG{p}{\PYGZgt{}}
  \PYG{p}{\PYGZlt{}}\PYG{n+nt}{input} \PYG{n+na}{type}\PYG{o}{=}\PYG{l+s}{\PYGZsq{}checkbox\PYGZsq{}} \PYG{n+na}{name}\PYG{o}{=}\PYG{l+s}{\PYGZsq{}red\PYGZsq{}} \PYG{n+na}{value}\PYG{o}{=}\PYG{l+s}{\PYGZsq{}red4g\PYGZsq{}}\PYG{p}{\PYGZgt{}} 4G   \PYG{p}{\PYGZlt{}}\PYG{n+nt}{br}\PYG{p}{/}\PYG{p}{\PYGZgt{}}
  \PYG{p}{\PYGZlt{}}\PYG{n+nt}{br}\PYG{p}{/}\PYG{p}{\PYGZgt{}}
  \PYG{p}{\PYGZlt{}}\PYG{n+nt}{select} \PYG{n+na}{name}\PYG{o}{=}\PYG{l+s}{\PYGZsq{}idioma\PYGZsq{}} \PYG{n+na}{multiple}\PYG{o}{=}\PYG{l+s}{\PYGZsq{}multiple\PYGZsq{}}\PYG{p}{\PYGZgt{}}
    \PYG{p}{\PYGZlt{}}\PYG{n+nt}{option} \PYG{n+na}{value}\PYG{o}{=}\PYG{l+s}{\PYGZsq{}ingles\PYGZsq{}}\PYG{p}{\PYGZgt{}}Inglés\PYG{p}{\PYGZlt{}}\PYG{p}{/}\PYG{n+nt}{option}\PYG{p}{\PYGZgt{}}
    \PYG{p}{\PYGZlt{}}\PYG{n+nt}{option} \PYG{n+na}{value}\PYG{o}{=}\PYG{l+s}{\PYGZsq{}aleman\PYGZsq{}}\PYG{p}{\PYGZgt{}}Alemán\PYG{p}{\PYGZlt{}}\PYG{p}{/}\PYG{n+nt}{option}\PYG{p}{\PYGZgt{}}
    \PYG{p}{\PYGZlt{}}\PYG{n+nt}{option} \PYG{n+na}{value}\PYG{o}{=}\PYG{l+s}{\PYGZsq{}frances\PYGZsq{}}\PYG{p}{\PYGZgt{}}Francés\PYG{p}{\PYGZlt{}}\PYG{p}{/}\PYG{n+nt}{option}\PYG{p}{\PYGZgt{}}
  \PYG{p}{\PYGZlt{}}\PYG{p}{/}\PYG{n+nt}{select}\PYG{p}{\PYGZgt{}}
  \PYG{p}{\PYGZlt{}}\PYG{n+nt}{br}\PYG{p}{/}\PYG{p}{\PYGZgt{}}
  \PYG{p}{\PYGZlt{}}\PYG{n+nt}{input} \PYG{n+na}{type}\PYG{o}{=}\PYG{l+s}{\PYGZsq{}radio\PYGZsq{}} \PYG{n+na}{name}\PYG{o}{=}\PYG{l+s}{\PYGZsq{}procesador\PYGZsq{}} \PYG{n+na}{value}\PYG{o}{=}\PYG{l+s}{\PYGZsq{}procesadorintel\PYGZsq{}}\PYG{p}{\PYGZgt{}} Intel 
  \PYG{p}{\PYGZlt{}}\PYG{n+nt}{input} \PYG{n+na}{type}\PYG{o}{=}\PYG{l+s}{\PYGZsq{}radio\PYGZsq{}} \PYG{n+na}{name}\PYG{o}{=}\PYG{l+s}{\PYGZsq{}procesador\PYGZsq{}} \PYG{n+na}{value}\PYG{o}{=}\PYG{l+s}{\PYGZsq{}procesadoramd\PYGZsq{}}\PYG{p}{\PYGZgt{}} AMD 
  \PYG{p}{\PYGZlt{}}\PYG{n+nt}{br}\PYG{p}{/}\PYG{p}{\PYGZgt{}}
\PYG{p}{\PYGZlt{}}\PYG{p}{/}\PYG{n+nt}{fieldset}\PYG{p}{\PYGZgt{}}
\PYG{p}{\PYGZlt{}}\PYG{p}{/}\PYG{n+nt}{form}\PYG{p}{\PYGZgt{}}
\end{sphinxVerbatim}


\section{Formulario 14}
\label{\detokenize{ejercicios/formularios/anexo_formularios:formulario-14}}
Generar el formulario siguiente de acuerdo a los siguientes requisitos
\begin{itemize}
\item {} 
Contiene los siguientes \sphinxcode{radiobuttons}:radio con el \sphinxcode{name}  «medio\_transporte» , \sphinxcode{value}  «medio\_transporteautomovil»  y el texto «Automóvil», radio con el \sphinxcode{name}  «medio\_transporte» , \sphinxcode{value}  «medio\_transportemoto»  y el texto «Moto», radio con el \sphinxcode{name}  «medio\_transporte» , \sphinxcode{value}  «medio\_transporteautobus»  y el texto «Autobús».

\item {} 
Hay un \sphinxcode{textarea} que mide 6 filas y 47 columnas que lleva dentro el texto «Utilice este recuadro por favor»

\item {} 
Contiene los siguientes \sphinxcode{checkboxes}:checkbox con el \sphinxcode{name}  «formato» , \sphinxcode{value}  «formatojpg»  y el texto «JPG», checkbox con el \sphinxcode{name}  «formato» , \sphinxcode{value}  «formatopng»  y el texto «PNG».

\item {} 
Contiene los siguientes \sphinxcode{radiobuttons}:radio con el \sphinxcode{name}  «idioma» , \sphinxcode{value}  «idiomaespanol»  y el texto «Español», radio con el \sphinxcode{name}  «idioma» , \sphinxcode{value}  «idiomaingles»  y el texto «Inglés», radio con el \sphinxcode{name}  «idioma» , \sphinxcode{value}  «idiomaaleman»  y el texto «Alemán», radio con el \sphinxcode{name}  «idioma» , \sphinxcode{value}  «idiomafrances»  y el texto «Francés».

\end{itemize}

\noindent{\hspace*{\fill}\sphinxincludegraphics[scale=0.6]{{foto_formulario_14}.png}\hspace*{\fill}}

Solución:

\begin{sphinxVerbatim}[commandchars=\\\{\}]
\PYG{p}{\PYGZlt{}}\PYG{n+nt}{form}\PYG{p}{\PYGZgt{}}
\PYG{p}{\PYGZlt{}}\PYG{n+nt}{fieldset}\PYG{p}{\PYGZgt{}}
  \PYG{p}{\PYGZlt{}}\PYG{n+nt}{legend}\PYG{p}{\PYGZgt{}}Preferencias\PYG{p}{\PYGZlt{}}\PYG{p}{/}\PYG{n+nt}{legend}\PYG{p}{\PYGZgt{}}
  \PYG{p}{\PYGZlt{}}\PYG{n+nt}{input} \PYG{n+na}{type}\PYG{o}{=}\PYG{l+s}{\PYGZsq{}radio\PYGZsq{}} \PYG{n+na}{name}\PYG{o}{=}\PYG{l+s}{\PYGZsq{}medio\PYGZus{}transporte\PYGZsq{}} \PYG{n+na}{value}\PYG{o}{=}\PYG{l+s}{\PYGZsq{}medio\PYGZus{}transporteautomovil\PYGZsq{}}\PYG{p}{\PYGZgt{}} Automóvil   \PYG{p}{\PYGZlt{}}\PYG{n+nt}{br}\PYG{p}{/}\PYG{p}{\PYGZgt{}}
  \PYG{p}{\PYGZlt{}}\PYG{n+nt}{input} \PYG{n+na}{type}\PYG{o}{=}\PYG{l+s}{\PYGZsq{}radio\PYGZsq{}} \PYG{n+na}{name}\PYG{o}{=}\PYG{l+s}{\PYGZsq{}medio\PYGZus{}transporte\PYGZsq{}} \PYG{n+na}{value}\PYG{o}{=}\PYG{l+s}{\PYGZsq{}medio\PYGZus{}transportemoto\PYGZsq{}}\PYG{p}{\PYGZgt{}} Moto   \PYG{p}{\PYGZlt{}}\PYG{n+nt}{br}\PYG{p}{/}\PYG{p}{\PYGZgt{}}
  \PYG{p}{\PYGZlt{}}\PYG{n+nt}{input} \PYG{n+na}{type}\PYG{o}{=}\PYG{l+s}{\PYGZsq{}radio\PYGZsq{}} \PYG{n+na}{name}\PYG{o}{=}\PYG{l+s}{\PYGZsq{}medio\PYGZus{}transporte\PYGZsq{}} \PYG{n+na}{value}\PYG{o}{=}\PYG{l+s}{\PYGZsq{}medio\PYGZus{}transporteautobus\PYGZsq{}}\PYG{p}{\PYGZgt{}} Autobús   \PYG{p}{\PYGZlt{}}\PYG{n+nt}{br}\PYG{p}{/}\PYG{p}{\PYGZgt{}}
  \PYG{p}{\PYGZlt{}}\PYG{n+nt}{br}\PYG{p}{/}\PYG{p}{\PYGZgt{}}
  \PYG{p}{\PYGZlt{}}\PYG{n+nt}{textarea} \PYG{n+na}{rows}\PYG{o}{=}\PYG{l+s}{\PYGZsq{}6\PYGZsq{}} \PYG{n+na}{cols}\PYG{o}{=}\PYG{l+s}{\PYGZsq{}47\PYGZsq{}}\PYG{p}{\PYGZgt{}}
    Utilice este recuadro por favor
  \PYG{p}{\PYGZlt{}}\PYG{p}{/}\PYG{n+nt}{textarea}\PYG{p}{\PYGZgt{}}  \PYG{p}{\PYGZlt{}}\PYG{n+nt}{br}\PYG{p}{/}\PYG{p}{\PYGZgt{}}
  \PYG{p}{\PYGZlt{}}\PYG{n+nt}{input} \PYG{n+na}{type}\PYG{o}{=}\PYG{l+s}{\PYGZsq{}checkbox\PYGZsq{}} \PYG{n+na}{name}\PYG{o}{=}\PYG{l+s}{\PYGZsq{}formato\PYGZsq{}} \PYG{n+na}{value}\PYG{o}{=}\PYG{l+s}{\PYGZsq{}formatojpg\PYGZsq{}}\PYG{p}{\PYGZgt{}} JPG   \PYG{p}{\PYGZlt{}}\PYG{n+nt}{br}\PYG{p}{/}\PYG{p}{\PYGZgt{}}
  \PYG{p}{\PYGZlt{}}\PYG{n+nt}{input} \PYG{n+na}{type}\PYG{o}{=}\PYG{l+s}{\PYGZsq{}checkbox\PYGZsq{}} \PYG{n+na}{name}\PYG{o}{=}\PYG{l+s}{\PYGZsq{}formato\PYGZsq{}} \PYG{n+na}{value}\PYG{o}{=}\PYG{l+s}{\PYGZsq{}formatopng\PYGZsq{}}\PYG{p}{\PYGZgt{}} PNG   \PYG{p}{\PYGZlt{}}\PYG{n+nt}{br}\PYG{p}{/}\PYG{p}{\PYGZgt{}}
  \PYG{p}{\PYGZlt{}}\PYG{n+nt}{br}\PYG{p}{/}\PYG{p}{\PYGZgt{}}
  \PYG{p}{\PYGZlt{}}\PYG{n+nt}{input} \PYG{n+na}{type}\PYG{o}{=}\PYG{l+s}{\PYGZsq{}radio\PYGZsq{}} \PYG{n+na}{name}\PYG{o}{=}\PYG{l+s}{\PYGZsq{}idioma\PYGZsq{}} \PYG{n+na}{value}\PYG{o}{=}\PYG{l+s}{\PYGZsq{}idiomaespanol\PYGZsq{}}\PYG{p}{\PYGZgt{}} Español 
  \PYG{p}{\PYGZlt{}}\PYG{n+nt}{input} \PYG{n+na}{type}\PYG{o}{=}\PYG{l+s}{\PYGZsq{}radio\PYGZsq{}} \PYG{n+na}{name}\PYG{o}{=}\PYG{l+s}{\PYGZsq{}idioma\PYGZsq{}} \PYG{n+na}{value}\PYG{o}{=}\PYG{l+s}{\PYGZsq{}idiomaingles\PYGZsq{}}\PYG{p}{\PYGZgt{}} Inglés 
  \PYG{p}{\PYGZlt{}}\PYG{n+nt}{input} \PYG{n+na}{type}\PYG{o}{=}\PYG{l+s}{\PYGZsq{}radio\PYGZsq{}} \PYG{n+na}{name}\PYG{o}{=}\PYG{l+s}{\PYGZsq{}idioma\PYGZsq{}} \PYG{n+na}{value}\PYG{o}{=}\PYG{l+s}{\PYGZsq{}idiomaaleman\PYGZsq{}}\PYG{p}{\PYGZgt{}} Alemán 
  \PYG{p}{\PYGZlt{}}\PYG{n+nt}{input} \PYG{n+na}{type}\PYG{o}{=}\PYG{l+s}{\PYGZsq{}radio\PYGZsq{}} \PYG{n+na}{name}\PYG{o}{=}\PYG{l+s}{\PYGZsq{}idioma\PYGZsq{}} \PYG{n+na}{value}\PYG{o}{=}\PYG{l+s}{\PYGZsq{}idiomafrances\PYGZsq{}}\PYG{p}{\PYGZgt{}} Francés 
  \PYG{p}{\PYGZlt{}}\PYG{n+nt}{br}\PYG{p}{/}\PYG{p}{\PYGZgt{}}
\PYG{p}{\PYGZlt{}}\PYG{p}{/}\PYG{n+nt}{fieldset}\PYG{p}{\PYGZgt{}}
\PYG{p}{\PYGZlt{}}\PYG{p}{/}\PYG{n+nt}{form}\PYG{p}{\PYGZgt{}}
\end{sphinxVerbatim}


\section{Formulario 15}
\label{\detokenize{ejercicios/formularios/anexo_formularios:formulario-15}}
Generar el formulario siguiente de acuerdo a los siguientes requisitos
\begin{itemize}
\item {} 
Hay los siguientes cuadros de texto:cuadro de texto con el texto «Nombre» y el \sphinxcode{name} nombre, cuadro de texto con el texto «Apellidos» y el \sphinxcode{name} apellidos, cuadro de texto con el texto «Direccion» y el \sphinxcode{name} direccion

\item {} 
Contiene los siguientes \sphinxcode{radiobuttons}:radio con el \sphinxcode{name}  «conexion» , \sphinxcode{value}  «conexionwifi»  y el texto «Wifi», radio con el \sphinxcode{name}  «conexion» , \sphinxcode{value}  «conexioncable»  y el texto «Cable», radio con el \sphinxcode{name}  «conexion» , \sphinxcode{value}  «conexionfibra»  y el texto «Fibra».

\item {} 
Hay los siguientes cuadros de texto:cuadro de texto con el texto «Instituto» y el \sphinxcode{name} instituto, cuadro de texto con el texto «Estudios elegidos» y el \sphinxcode{name} estudios

\item {} 
Hay una lista desplegable múltiple con el \sphinxcode{name} «dia» y con las siguientes opciones: opción «Lunes» con el \sphinxcode{value} lunes, opción «Martes» con el \sphinxcode{value} martes, opción «Miércoles» con el \sphinxcode{value} miercoles, opción «Jueves» con el \sphinxcode{value} jueves, opción «Sabado» con el \sphinxcode{value} sabado.

\item {} 
Contiene los siguientes \sphinxcode{checkboxes}:checkbox con el \sphinxcode{name}  «asignatura» , \sphinxcode{value}  «asignaturahistoria»  y el texto «Historia», checkbox con el \sphinxcode{name}  «asignatura» , \sphinxcode{value}  «asignaturageografia»  y el texto «Geografía», checkbox con el \sphinxcode{name}  «asignatura» , \sphinxcode{value}  «asignaturalengua»  y el texto «Lengua», checkbox con el \sphinxcode{name}  «asignatura» , \sphinxcode{value}  «asignaturamatematicas»  y el texto «Matemáticas».

\item {} 
Hay un \sphinxcode{textarea} que mide 6 filas y 50 columnas que lleva dentro el texto «Inserte aqui el texto»

\end{itemize}

\noindent{\hspace*{\fill}\sphinxincludegraphics[scale=0.6]{{foto_formulario_15}.png}\hspace*{\fill}}

Solución:

\begin{sphinxVerbatim}[commandchars=\\\{\}]
\PYG{p}{\PYGZlt{}}\PYG{n+nt}{form}\PYG{p}{\PYGZgt{}}
\PYG{p}{\PYGZlt{}}\PYG{n+nt}{fieldset}\PYG{p}{\PYGZgt{}}
  \PYG{p}{\PYGZlt{}}\PYG{n+nt}{legend}\PYG{p}{\PYGZgt{}}Formulario de opciones\PYG{p}{\PYGZlt{}}\PYG{p}{/}\PYG{n+nt}{legend}\PYG{p}{\PYGZgt{}}
  Nombre\PYG{p}{\PYGZlt{}}\PYG{n+nt}{input} \PYG{n+na}{type}\PYG{o}{=}\PYG{l+s}{\PYGZsq{}text\PYGZsq{}} \PYG{n+na}{name}\PYG{o}{=}\PYG{l+s}{\PYGZsq{}nombre\PYGZsq{}}\PYG{p}{\PYGZgt{}} \PYG{p}{\PYGZlt{}}\PYG{n+nt}{br}\PYG{p}{/}\PYG{p}{\PYGZgt{}}
  Apellidos\PYG{p}{\PYGZlt{}}\PYG{n+nt}{input} \PYG{n+na}{type}\PYG{o}{=}\PYG{l+s}{\PYGZsq{}text\PYGZsq{}} \PYG{n+na}{name}\PYG{o}{=}\PYG{l+s}{\PYGZsq{}apellidos\PYGZsq{}}\PYG{p}{\PYGZgt{}} \PYG{p}{\PYGZlt{}}\PYG{n+nt}{br}\PYG{p}{/}\PYG{p}{\PYGZgt{}}
  Direccion\PYG{p}{\PYGZlt{}}\PYG{n+nt}{input} \PYG{n+na}{type}\PYG{o}{=}\PYG{l+s}{\PYGZsq{}text\PYGZsq{}} \PYG{n+na}{name}\PYG{o}{=}\PYG{l+s}{\PYGZsq{}direccion\PYGZsq{}}\PYG{p}{\PYGZgt{}} \PYG{p}{\PYGZlt{}}\PYG{n+nt}{br}\PYG{p}{/}\PYG{p}{\PYGZgt{}}
  \PYG{p}{\PYGZlt{}}\PYG{n+nt}{br}\PYG{p}{/}\PYG{p}{\PYGZgt{}}
  \PYG{p}{\PYGZlt{}}\PYG{n+nt}{input} \PYG{n+na}{type}\PYG{o}{=}\PYG{l+s}{\PYGZsq{}radio\PYGZsq{}} \PYG{n+na}{name}\PYG{o}{=}\PYG{l+s}{\PYGZsq{}conexion\PYGZsq{}} \PYG{n+na}{value}\PYG{o}{=}\PYG{l+s}{\PYGZsq{}conexionwifi\PYGZsq{}}\PYG{p}{\PYGZgt{}} Wifi 
  \PYG{p}{\PYGZlt{}}\PYG{n+nt}{input} \PYG{n+na}{type}\PYG{o}{=}\PYG{l+s}{\PYGZsq{}radio\PYGZsq{}} \PYG{n+na}{name}\PYG{o}{=}\PYG{l+s}{\PYGZsq{}conexion\PYGZsq{}} \PYG{n+na}{value}\PYG{o}{=}\PYG{l+s}{\PYGZsq{}conexioncable\PYGZsq{}}\PYG{p}{\PYGZgt{}} Cable 
  \PYG{p}{\PYGZlt{}}\PYG{n+nt}{input} \PYG{n+na}{type}\PYG{o}{=}\PYG{l+s}{\PYGZsq{}radio\PYGZsq{}} \PYG{n+na}{name}\PYG{o}{=}\PYG{l+s}{\PYGZsq{}conexion\PYGZsq{}} \PYG{n+na}{value}\PYG{o}{=}\PYG{l+s}{\PYGZsq{}conexionfibra\PYGZsq{}}\PYG{p}{\PYGZgt{}} Fibra 
  \PYG{p}{\PYGZlt{}}\PYG{n+nt}{br}\PYG{p}{/}\PYG{p}{\PYGZgt{}}
  Instituto\PYG{p}{\PYGZlt{}}\PYG{n+nt}{input} \PYG{n+na}{type}\PYG{o}{=}\PYG{l+s}{\PYGZsq{}text\PYGZsq{}} \PYG{n+na}{name}\PYG{o}{=}\PYG{l+s}{\PYGZsq{}instituto\PYGZsq{}}\PYG{p}{\PYGZgt{}} \PYG{p}{\PYGZlt{}}\PYG{n+nt}{br}\PYG{p}{/}\PYG{p}{\PYGZgt{}}
  Estudios elegidos\PYG{p}{\PYGZlt{}}\PYG{n+nt}{input} \PYG{n+na}{type}\PYG{o}{=}\PYG{l+s}{\PYGZsq{}text\PYGZsq{}} \PYG{n+na}{name}\PYG{o}{=}\PYG{l+s}{\PYGZsq{}estudios\PYGZsq{}}\PYG{p}{\PYGZgt{}} \PYG{p}{\PYGZlt{}}\PYG{n+nt}{br}\PYG{p}{/}\PYG{p}{\PYGZgt{}}
  \PYG{p}{\PYGZlt{}}\PYG{n+nt}{br}\PYG{p}{/}\PYG{p}{\PYGZgt{}}
  \PYG{p}{\PYGZlt{}}\PYG{n+nt}{select} \PYG{n+na}{name}\PYG{o}{=}\PYG{l+s}{\PYGZsq{}dia\PYGZsq{}} \PYG{n+na}{multiple}\PYG{o}{=}\PYG{l+s}{\PYGZsq{}multiple\PYGZsq{}}\PYG{p}{\PYGZgt{}}
    \PYG{p}{\PYGZlt{}}\PYG{n+nt}{option} \PYG{n+na}{value}\PYG{o}{=}\PYG{l+s}{\PYGZsq{}lunes\PYGZsq{}}\PYG{p}{\PYGZgt{}}Lunes\PYG{p}{\PYGZlt{}}\PYG{p}{/}\PYG{n+nt}{option}\PYG{p}{\PYGZgt{}}
    \PYG{p}{\PYGZlt{}}\PYG{n+nt}{option} \PYG{n+na}{value}\PYG{o}{=}\PYG{l+s}{\PYGZsq{}martes\PYGZsq{}}\PYG{p}{\PYGZgt{}}Martes\PYG{p}{\PYGZlt{}}\PYG{p}{/}\PYG{n+nt}{option}\PYG{p}{\PYGZgt{}}
    \PYG{p}{\PYGZlt{}}\PYG{n+nt}{option} \PYG{n+na}{value}\PYG{o}{=}\PYG{l+s}{\PYGZsq{}miercoles\PYGZsq{}}\PYG{p}{\PYGZgt{}}Miércoles\PYG{p}{\PYGZlt{}}\PYG{p}{/}\PYG{n+nt}{option}\PYG{p}{\PYGZgt{}}
    \PYG{p}{\PYGZlt{}}\PYG{n+nt}{option} \PYG{n+na}{value}\PYG{o}{=}\PYG{l+s}{\PYGZsq{}jueves\PYGZsq{}}\PYG{p}{\PYGZgt{}}Jueves\PYG{p}{\PYGZlt{}}\PYG{p}{/}\PYG{n+nt}{option}\PYG{p}{\PYGZgt{}}
    \PYG{p}{\PYGZlt{}}\PYG{n+nt}{option} \PYG{n+na}{value}\PYG{o}{=}\PYG{l+s}{\PYGZsq{}sabado\PYGZsq{}}\PYG{p}{\PYGZgt{}}Sabado\PYG{p}{\PYGZlt{}}\PYG{p}{/}\PYG{n+nt}{option}\PYG{p}{\PYGZgt{}}
  \PYG{p}{\PYGZlt{}}\PYG{p}{/}\PYG{n+nt}{select}\PYG{p}{\PYGZgt{}}
  \PYG{p}{\PYGZlt{}}\PYG{n+nt}{br}\PYG{p}{/}\PYG{p}{\PYGZgt{}}
\PYG{p}{\PYGZlt{}}\PYG{p}{/}\PYG{n+nt}{fieldset}\PYG{p}{\PYGZgt{}}
\PYG{p}{\PYGZlt{}}\PYG{n+nt}{fieldset}\PYG{p}{\PYGZgt{}}
  \PYG{p}{\PYGZlt{}}\PYG{n+nt}{legend}\PYG{p}{\PYGZgt{}}Preferencias\PYG{p}{\PYGZlt{}}\PYG{p}{/}\PYG{n+nt}{legend}\PYG{p}{\PYGZgt{}}
  \PYG{p}{\PYGZlt{}}\PYG{n+nt}{input} \PYG{n+na}{type}\PYG{o}{=}\PYG{l+s}{\PYGZsq{}checkbox\PYGZsq{}} \PYG{n+na}{name}\PYG{o}{=}\PYG{l+s}{\PYGZsq{}asignatura\PYGZsq{}} \PYG{n+na}{value}\PYG{o}{=}\PYG{l+s}{\PYGZsq{}asignaturahistoria\PYGZsq{}}\PYG{p}{\PYGZgt{}} Historia 
  \PYG{p}{\PYGZlt{}}\PYG{n+nt}{input} \PYG{n+na}{type}\PYG{o}{=}\PYG{l+s}{\PYGZsq{}checkbox\PYGZsq{}} \PYG{n+na}{name}\PYG{o}{=}\PYG{l+s}{\PYGZsq{}asignatura\PYGZsq{}} \PYG{n+na}{value}\PYG{o}{=}\PYG{l+s}{\PYGZsq{}asignaturageografia\PYGZsq{}}\PYG{p}{\PYGZgt{}} Geografía 
  \PYG{p}{\PYGZlt{}}\PYG{n+nt}{input} \PYG{n+na}{type}\PYG{o}{=}\PYG{l+s}{\PYGZsq{}checkbox\PYGZsq{}} \PYG{n+na}{name}\PYG{o}{=}\PYG{l+s}{\PYGZsq{}asignatura\PYGZsq{}} \PYG{n+na}{value}\PYG{o}{=}\PYG{l+s}{\PYGZsq{}asignaturalengua\PYGZsq{}}\PYG{p}{\PYGZgt{}} Lengua 
  \PYG{p}{\PYGZlt{}}\PYG{n+nt}{input} \PYG{n+na}{type}\PYG{o}{=}\PYG{l+s}{\PYGZsq{}checkbox\PYGZsq{}} \PYG{n+na}{name}\PYG{o}{=}\PYG{l+s}{\PYGZsq{}asignatura\PYGZsq{}} \PYG{n+na}{value}\PYG{o}{=}\PYG{l+s}{\PYGZsq{}asignaturamatematicas\PYGZsq{}}\PYG{p}{\PYGZgt{}} Matemáticas 
  \PYG{p}{\PYGZlt{}}\PYG{n+nt}{br}\PYG{p}{/}\PYG{p}{\PYGZgt{}}
  \PYG{p}{\PYGZlt{}}\PYG{n+nt}{textarea} \PYG{n+na}{rows}\PYG{o}{=}\PYG{l+s}{\PYGZsq{}6\PYGZsq{}} \PYG{n+na}{cols}\PYG{o}{=}\PYG{l+s}{\PYGZsq{}50\PYGZsq{}}\PYG{p}{\PYGZgt{}}
    Inserte aqui el texto
  \PYG{p}{\PYGZlt{}}\PYG{p}{/}\PYG{n+nt}{textarea}\PYG{p}{\PYGZgt{}}  \PYG{p}{\PYGZlt{}}\PYG{n+nt}{br}\PYG{p}{/}\PYG{p}{\PYGZgt{}}
\PYG{p}{\PYGZlt{}}\PYG{p}{/}\PYG{n+nt}{fieldset}\PYG{p}{\PYGZgt{}}
\PYG{p}{\PYGZlt{}}\PYG{p}{/}\PYG{n+nt}{form}\PYG{p}{\PYGZgt{}}
\end{sphinxVerbatim}


\section{Formulario 16}
\label{\detokenize{ejercicios/formularios/anexo_formularios:formulario-16}}
Generar el formulario siguiente de acuerdo a los siguientes requisitos
\begin{itemize}
\item {} 
Contiene los siguientes \sphinxcode{radiobuttons}:radio con el \sphinxcode{name}  «formato» , \sphinxcode{value}  «formatojpg»  y el texto «JPG», radio con el \sphinxcode{name}  «formato» , \sphinxcode{value}  «formatopng»  y el texto «PNG».

\item {} 
Contiene los siguientes \sphinxcode{checkboxes}:checkbox con el \sphinxcode{name}  «escritorio» , \sphinxcode{value}  «escritoriokde»  y el texto «KDE», checkbox con el \sphinxcode{name}  «escritorio» , \sphinxcode{value}  «escritoriognome»  y el texto «GNOME», checkbox con el \sphinxcode{name}  «escritorio» , \sphinxcode{value}  «escritoriounity»  y el texto «Unity».

\end{itemize}

\noindent{\hspace*{\fill}\sphinxincludegraphics[scale=0.6]{{foto_formulario_16}.png}\hspace*{\fill}}

Solución:

\begin{sphinxVerbatim}[commandchars=\\\{\}]
\PYG{p}{\PYGZlt{}}\PYG{n+nt}{form}\PYG{p}{\PYGZgt{}}
\PYG{p}{\PYGZlt{}}\PYG{n+nt}{fieldset}\PYG{p}{\PYGZgt{}}
  \PYG{p}{\PYGZlt{}}\PYG{n+nt}{legend}\PYG{p}{\PYGZgt{}}Preferencias\PYG{p}{\PYGZlt{}}\PYG{p}{/}\PYG{n+nt}{legend}\PYG{p}{\PYGZgt{}}
  \PYG{p}{\PYGZlt{}}\PYG{n+nt}{input} \PYG{n+na}{type}\PYG{o}{=}\PYG{l+s}{\PYGZsq{}radio\PYGZsq{}} \PYG{n+na}{name}\PYG{o}{=}\PYG{l+s}{\PYGZsq{}formato\PYGZsq{}} \PYG{n+na}{value}\PYG{o}{=}\PYG{l+s}{\PYGZsq{}formatojpg\PYGZsq{}}\PYG{p}{\PYGZgt{}} JPG 
  \PYG{p}{\PYGZlt{}}\PYG{n+nt}{input} \PYG{n+na}{type}\PYG{o}{=}\PYG{l+s}{\PYGZsq{}radio\PYGZsq{}} \PYG{n+na}{name}\PYG{o}{=}\PYG{l+s}{\PYGZsq{}formato\PYGZsq{}} \PYG{n+na}{value}\PYG{o}{=}\PYG{l+s}{\PYGZsq{}formatopng\PYGZsq{}}\PYG{p}{\PYGZgt{}} PNG 
  \PYG{p}{\PYGZlt{}}\PYG{n+nt}{br}\PYG{p}{/}\PYG{p}{\PYGZgt{}}
  \PYG{p}{\PYGZlt{}}\PYG{n+nt}{input} \PYG{n+na}{type}\PYG{o}{=}\PYG{l+s}{\PYGZsq{}checkbox\PYGZsq{}} \PYG{n+na}{name}\PYG{o}{=}\PYG{l+s}{\PYGZsq{}escritorio\PYGZsq{}} \PYG{n+na}{value}\PYG{o}{=}\PYG{l+s}{\PYGZsq{}escritoriokde\PYGZsq{}}\PYG{p}{\PYGZgt{}} KDE   \PYG{p}{\PYGZlt{}}\PYG{n+nt}{br}\PYG{p}{/}\PYG{p}{\PYGZgt{}}
  \PYG{p}{\PYGZlt{}}\PYG{n+nt}{input} \PYG{n+na}{type}\PYG{o}{=}\PYG{l+s}{\PYGZsq{}checkbox\PYGZsq{}} \PYG{n+na}{name}\PYG{o}{=}\PYG{l+s}{\PYGZsq{}escritorio\PYGZsq{}} \PYG{n+na}{value}\PYG{o}{=}\PYG{l+s}{\PYGZsq{}escritoriognome\PYGZsq{}}\PYG{p}{\PYGZgt{}} GNOME   \PYG{p}{\PYGZlt{}}\PYG{n+nt}{br}\PYG{p}{/}\PYG{p}{\PYGZgt{}}
  \PYG{p}{\PYGZlt{}}\PYG{n+nt}{input} \PYG{n+na}{type}\PYG{o}{=}\PYG{l+s}{\PYGZsq{}checkbox\PYGZsq{}} \PYG{n+na}{name}\PYG{o}{=}\PYG{l+s}{\PYGZsq{}escritorio\PYGZsq{}} \PYG{n+na}{value}\PYG{o}{=}\PYG{l+s}{\PYGZsq{}escritoriounity\PYGZsq{}}\PYG{p}{\PYGZgt{}} Unity   \PYG{p}{\PYGZlt{}}\PYG{n+nt}{br}\PYG{p}{/}\PYG{p}{\PYGZgt{}}
  \PYG{p}{\PYGZlt{}}\PYG{n+nt}{br}\PYG{p}{/}\PYG{p}{\PYGZgt{}}
\PYG{p}{\PYGZlt{}}\PYG{p}{/}\PYG{n+nt}{fieldset}\PYG{p}{\PYGZgt{}}
\PYG{p}{\PYGZlt{}}\PYG{p}{/}\PYG{n+nt}{form}\PYG{p}{\PYGZgt{}}
\end{sphinxVerbatim}


\section{Formulario 17}
\label{\detokenize{ejercicios/formularios/anexo_formularios:formulario-17}}
Generar el formulario siguiente de acuerdo a los siguientes requisitos
\begin{itemize}
\item {} 
Hay un \sphinxcode{textarea} que mide 7 filas y 46 columnas que lleva dentro el texto «Por favor, escriba aquí»

\item {} 
Hay un \sphinxcode{textarea} que mide 6 filas y 60 columnas que lleva dentro el texto «Inserte aqui el texto»

\end{itemize}

\noindent{\hspace*{\fill}\sphinxincludegraphics[scale=0.6]{{foto_formulario_17}.png}\hspace*{\fill}}

Solución:

\begin{sphinxVerbatim}[commandchars=\\\{\}]
\PYG{p}{\PYGZlt{}}\PYG{n+nt}{form}\PYG{p}{\PYGZgt{}}
\PYG{p}{\PYGZlt{}}\PYG{n+nt}{fieldset}\PYG{p}{\PYGZgt{}}
  \PYG{p}{\PYGZlt{}}\PYG{n+nt}{legend}\PYG{p}{\PYGZgt{}}Por favor, complete estas opciones\PYG{p}{\PYGZlt{}}\PYG{p}{/}\PYG{n+nt}{legend}\PYG{p}{\PYGZgt{}}
  \PYG{p}{\PYGZlt{}}\PYG{n+nt}{textarea} \PYG{n+na}{rows}\PYG{o}{=}\PYG{l+s}{\PYGZsq{}7\PYGZsq{}} \PYG{n+na}{cols}\PYG{o}{=}\PYG{l+s}{\PYGZsq{}46\PYGZsq{}}\PYG{p}{\PYGZgt{}}
    Por favor, escriba aquí
  \PYG{p}{\PYGZlt{}}\PYG{p}{/}\PYG{n+nt}{textarea}\PYG{p}{\PYGZgt{}}  \PYG{p}{\PYGZlt{}}\PYG{n+nt}{br}\PYG{p}{/}\PYG{p}{\PYGZgt{}}
  \PYG{p}{\PYGZlt{}}\PYG{n+nt}{textarea} \PYG{n+na}{rows}\PYG{o}{=}\PYG{l+s}{\PYGZsq{}6\PYGZsq{}} \PYG{n+na}{cols}\PYG{o}{=}\PYG{l+s}{\PYGZsq{}60\PYGZsq{}}\PYG{p}{\PYGZgt{}}
    Inserte aqui el texto
  \PYG{p}{\PYGZlt{}}\PYG{p}{/}\PYG{n+nt}{textarea}\PYG{p}{\PYGZgt{}}  \PYG{p}{\PYGZlt{}}\PYG{n+nt}{br}\PYG{p}{/}\PYG{p}{\PYGZgt{}}
\PYG{p}{\PYGZlt{}}\PYG{p}{/}\PYG{n+nt}{fieldset}\PYG{p}{\PYGZgt{}}
\PYG{p}{\PYGZlt{}}\PYG{p}{/}\PYG{n+nt}{form}\PYG{p}{\PYGZgt{}}
\end{sphinxVerbatim}


\section{Formulario 18}
\label{\detokenize{ejercicios/formularios/anexo_formularios:formulario-18}}
Generar el formulario siguiente de acuerdo a los siguientes requisitos
\begin{itemize}
\item {} 
Contiene los siguientes \sphinxcode{radiobuttons}:radio con el \sphinxcode{name}  «procesador» , \sphinxcode{value}  «procesadoramd»  y el texto «AMD», radio con el \sphinxcode{name}  «procesador» , \sphinxcode{value}  «procesadorintel\_i5»  y el texto «Intel i5», radio con el \sphinxcode{name}  «procesador» , \sphinxcode{value}  «procesadorintel\_i7»  y el texto «Intel i7».

\item {} 
Hay los siguientes cuadros de texto:cuadro de texto con el texto «Nombre» y el \sphinxcode{name} nombre, cuadro de texto con el texto «Apellidos» y el \sphinxcode{name} apellidos, cuadro de texto con el texto «Direccion» y el \sphinxcode{name} direccion

\item {} 
Contiene los siguientes \sphinxcode{checkboxes}:checkbox con el \sphinxcode{name}  «sexo» , \sphinxcode{value}  «sexomujer»  y el texto «Mujer», checkbox con el \sphinxcode{name}  «sexo» , \sphinxcode{value}  «sexohombre»  y el texto «Hombre».

\item {} 
Contiene los siguientes \sphinxcode{checkboxes}:checkbox con el \sphinxcode{name}  «ciclo» , \sphinxcode{value}  «ciclosmir»  y el texto «SMIR», checkbox con el \sphinxcode{name}  «ciclo» , \sphinxcode{value}  «cicloasir»  y el texto «ASIR», checkbox con el \sphinxcode{name}  «ciclo» , \sphinxcode{value}  «ciclodam»  y el texto «DAM», checkbox con el \sphinxcode{name}  «ciclo» , \sphinxcode{value}  «ciclodaw»  y el texto «DAW».

\item {} 
Contiene los siguientes \sphinxcode{radiobuttons}:radio con el \sphinxcode{name}  «navegador» , \sphinxcode{value}  «navegadorfirefox»  y el texto «Firefox», radio con el \sphinxcode{name}  «navegador» , \sphinxcode{value}  «navegadorchrome»  y el texto «Chrome», radio con el \sphinxcode{name}  «navegador» , \sphinxcode{value}  «navegadoropera»  y el texto «Opera», radio con el \sphinxcode{name}  «navegador» , \sphinxcode{value}  «navegadorie»  y el texto «IE».

\end{itemize}

\noindent{\hspace*{\fill}\sphinxincludegraphics[scale=0.6]{{foto_formulario_18}.png}\hspace*{\fill}}

Solución:

\begin{sphinxVerbatim}[commandchars=\\\{\}]
\PYG{p}{\PYGZlt{}}\PYG{n+nt}{form}\PYG{p}{\PYGZgt{}}
\PYG{p}{\PYGZlt{}}\PYG{n+nt}{fieldset}\PYG{p}{\PYGZgt{}}
  \PYG{p}{\PYGZlt{}}\PYG{n+nt}{legend}\PYG{p}{\PYGZgt{}}Opciones\PYG{p}{\PYGZlt{}}\PYG{p}{/}\PYG{n+nt}{legend}\PYG{p}{\PYGZgt{}}
  \PYG{p}{\PYGZlt{}}\PYG{n+nt}{input} \PYG{n+na}{type}\PYG{o}{=}\PYG{l+s}{\PYGZsq{}radio\PYGZsq{}} \PYG{n+na}{name}\PYG{o}{=}\PYG{l+s}{\PYGZsq{}procesador\PYGZsq{}} \PYG{n+na}{value}\PYG{o}{=}\PYG{l+s}{\PYGZsq{}procesadoramd\PYGZsq{}}\PYG{p}{\PYGZgt{}} AMD 
  \PYG{p}{\PYGZlt{}}\PYG{n+nt}{input} \PYG{n+na}{type}\PYG{o}{=}\PYG{l+s}{\PYGZsq{}radio\PYGZsq{}} \PYG{n+na}{name}\PYG{o}{=}\PYG{l+s}{\PYGZsq{}procesador\PYGZsq{}} \PYG{n+na}{value}\PYG{o}{=}\PYG{l+s}{\PYGZsq{}procesadorintel\PYGZus{}i5\PYGZsq{}}\PYG{p}{\PYGZgt{}} Intel i5 
  \PYG{p}{\PYGZlt{}}\PYG{n+nt}{input} \PYG{n+na}{type}\PYG{o}{=}\PYG{l+s}{\PYGZsq{}radio\PYGZsq{}} \PYG{n+na}{name}\PYG{o}{=}\PYG{l+s}{\PYGZsq{}procesador\PYGZsq{}} \PYG{n+na}{value}\PYG{o}{=}\PYG{l+s}{\PYGZsq{}procesadorintel\PYGZus{}i7\PYGZsq{}}\PYG{p}{\PYGZgt{}} Intel i7 
  \PYG{p}{\PYGZlt{}}\PYG{n+nt}{br}\PYG{p}{/}\PYG{p}{\PYGZgt{}}
\PYG{p}{\PYGZlt{}}\PYG{p}{/}\PYG{n+nt}{fieldset}\PYG{p}{\PYGZgt{}}
\PYG{p}{\PYGZlt{}}\PYG{n+nt}{fieldset}\PYG{p}{\PYGZgt{}}
  \PYG{p}{\PYGZlt{}}\PYG{n+nt}{legend}\PYG{p}{\PYGZgt{}}Rellene las opciones siguientes\PYG{p}{\PYGZlt{}}\PYG{p}{/}\PYG{n+nt}{legend}\PYG{p}{\PYGZgt{}}
  Nombre\PYG{p}{\PYGZlt{}}\PYG{n+nt}{input} \PYG{n+na}{type}\PYG{o}{=}\PYG{l+s}{\PYGZsq{}text\PYGZsq{}} \PYG{n+na}{name}\PYG{o}{=}\PYG{l+s}{\PYGZsq{}nombre\PYGZsq{}}\PYG{p}{\PYGZgt{}} \PYG{p}{\PYGZlt{}}\PYG{n+nt}{br}\PYG{p}{/}\PYG{p}{\PYGZgt{}}
  Apellidos\PYG{p}{\PYGZlt{}}\PYG{n+nt}{input} \PYG{n+na}{type}\PYG{o}{=}\PYG{l+s}{\PYGZsq{}text\PYGZsq{}} \PYG{n+na}{name}\PYG{o}{=}\PYG{l+s}{\PYGZsq{}apellidos\PYGZsq{}}\PYG{p}{\PYGZgt{}} \PYG{p}{\PYGZlt{}}\PYG{n+nt}{br}\PYG{p}{/}\PYG{p}{\PYGZgt{}}
  Direccion\PYG{p}{\PYGZlt{}}\PYG{n+nt}{input} \PYG{n+na}{type}\PYG{o}{=}\PYG{l+s}{\PYGZsq{}text\PYGZsq{}} \PYG{n+na}{name}\PYG{o}{=}\PYG{l+s}{\PYGZsq{}direccion\PYGZsq{}}\PYG{p}{\PYGZgt{}} \PYG{p}{\PYGZlt{}}\PYG{n+nt}{br}\PYG{p}{/}\PYG{p}{\PYGZgt{}}
  \PYG{p}{\PYGZlt{}}\PYG{n+nt}{br}\PYG{p}{/}\PYG{p}{\PYGZgt{}}
  \PYG{p}{\PYGZlt{}}\PYG{n+nt}{input} \PYG{n+na}{type}\PYG{o}{=}\PYG{l+s}{\PYGZsq{}checkbox\PYGZsq{}} \PYG{n+na}{name}\PYG{o}{=}\PYG{l+s}{\PYGZsq{}sexo\PYGZsq{}} \PYG{n+na}{value}\PYG{o}{=}\PYG{l+s}{\PYGZsq{}sexomujer\PYGZsq{}}\PYG{p}{\PYGZgt{}} Mujer 
  \PYG{p}{\PYGZlt{}}\PYG{n+nt}{input} \PYG{n+na}{type}\PYG{o}{=}\PYG{l+s}{\PYGZsq{}checkbox\PYGZsq{}} \PYG{n+na}{name}\PYG{o}{=}\PYG{l+s}{\PYGZsq{}sexo\PYGZsq{}} \PYG{n+na}{value}\PYG{o}{=}\PYG{l+s}{\PYGZsq{}sexohombre\PYGZsq{}}\PYG{p}{\PYGZgt{}} Hombre 
  \PYG{p}{\PYGZlt{}}\PYG{n+nt}{br}\PYG{p}{/}\PYG{p}{\PYGZgt{}}
  \PYG{p}{\PYGZlt{}}\PYG{n+nt}{input} \PYG{n+na}{type}\PYG{o}{=}\PYG{l+s}{\PYGZsq{}checkbox\PYGZsq{}} \PYG{n+na}{name}\PYG{o}{=}\PYG{l+s}{\PYGZsq{}ciclo\PYGZsq{}} \PYG{n+na}{value}\PYG{o}{=}\PYG{l+s}{\PYGZsq{}ciclosmir\PYGZsq{}}\PYG{p}{\PYGZgt{}} SMIR 
  \PYG{p}{\PYGZlt{}}\PYG{n+nt}{input} \PYG{n+na}{type}\PYG{o}{=}\PYG{l+s}{\PYGZsq{}checkbox\PYGZsq{}} \PYG{n+na}{name}\PYG{o}{=}\PYG{l+s}{\PYGZsq{}ciclo\PYGZsq{}} \PYG{n+na}{value}\PYG{o}{=}\PYG{l+s}{\PYGZsq{}cicloasir\PYGZsq{}}\PYG{p}{\PYGZgt{}} ASIR 
  \PYG{p}{\PYGZlt{}}\PYG{n+nt}{input} \PYG{n+na}{type}\PYG{o}{=}\PYG{l+s}{\PYGZsq{}checkbox\PYGZsq{}} \PYG{n+na}{name}\PYG{o}{=}\PYG{l+s}{\PYGZsq{}ciclo\PYGZsq{}} \PYG{n+na}{value}\PYG{o}{=}\PYG{l+s}{\PYGZsq{}ciclodam\PYGZsq{}}\PYG{p}{\PYGZgt{}} DAM 
  \PYG{p}{\PYGZlt{}}\PYG{n+nt}{input} \PYG{n+na}{type}\PYG{o}{=}\PYG{l+s}{\PYGZsq{}checkbox\PYGZsq{}} \PYG{n+na}{name}\PYG{o}{=}\PYG{l+s}{\PYGZsq{}ciclo\PYGZsq{}} \PYG{n+na}{value}\PYG{o}{=}\PYG{l+s}{\PYGZsq{}ciclodaw\PYGZsq{}}\PYG{p}{\PYGZgt{}} DAW 
  \PYG{p}{\PYGZlt{}}\PYG{n+nt}{br}\PYG{p}{/}\PYG{p}{\PYGZgt{}}
  \PYG{p}{\PYGZlt{}}\PYG{n+nt}{input} \PYG{n+na}{type}\PYG{o}{=}\PYG{l+s}{\PYGZsq{}radio\PYGZsq{}} \PYG{n+na}{name}\PYG{o}{=}\PYG{l+s}{\PYGZsq{}navegador\PYGZsq{}} \PYG{n+na}{value}\PYG{o}{=}\PYG{l+s}{\PYGZsq{}navegadorfirefox\PYGZsq{}}\PYG{p}{\PYGZgt{}} Firefox 
  \PYG{p}{\PYGZlt{}}\PYG{n+nt}{input} \PYG{n+na}{type}\PYG{o}{=}\PYG{l+s}{\PYGZsq{}radio\PYGZsq{}} \PYG{n+na}{name}\PYG{o}{=}\PYG{l+s}{\PYGZsq{}navegador\PYGZsq{}} \PYG{n+na}{value}\PYG{o}{=}\PYG{l+s}{\PYGZsq{}navegadorchrome\PYGZsq{}}\PYG{p}{\PYGZgt{}} Chrome 
  \PYG{p}{\PYGZlt{}}\PYG{n+nt}{input} \PYG{n+na}{type}\PYG{o}{=}\PYG{l+s}{\PYGZsq{}radio\PYGZsq{}} \PYG{n+na}{name}\PYG{o}{=}\PYG{l+s}{\PYGZsq{}navegador\PYGZsq{}} \PYG{n+na}{value}\PYG{o}{=}\PYG{l+s}{\PYGZsq{}navegadoropera\PYGZsq{}}\PYG{p}{\PYGZgt{}} Opera 
  \PYG{p}{\PYGZlt{}}\PYG{n+nt}{input} \PYG{n+na}{type}\PYG{o}{=}\PYG{l+s}{\PYGZsq{}radio\PYGZsq{}} \PYG{n+na}{name}\PYG{o}{=}\PYG{l+s}{\PYGZsq{}navegador\PYGZsq{}} \PYG{n+na}{value}\PYG{o}{=}\PYG{l+s}{\PYGZsq{}navegadorie\PYGZsq{}}\PYG{p}{\PYGZgt{}} IE 
  \PYG{p}{\PYGZlt{}}\PYG{n+nt}{br}\PYG{p}{/}\PYG{p}{\PYGZgt{}}
\PYG{p}{\PYGZlt{}}\PYG{p}{/}\PYG{n+nt}{fieldset}\PYG{p}{\PYGZgt{}}
\PYG{p}{\PYGZlt{}}\PYG{p}{/}\PYG{n+nt}{form}\PYG{p}{\PYGZgt{}}
\end{sphinxVerbatim}


\section{Formulario 19}
\label{\detokenize{ejercicios/formularios/anexo_formularios:formulario-19}}
Generar el formulario siguiente de acuerdo a los siguientes requisitos
\begin{itemize}
\item {} 
Hay los siguientes cuadros de texto:cuadro de texto con el texto «Nombre» y el \sphinxcode{name} nombre, cuadro de texto con el texto «Apellidos» y el \sphinxcode{name} apellidos

\item {} 
Contiene los siguientes \sphinxcode{checkboxes}:checkbox con el \sphinxcode{name}  «preferencia» , \sphinxcode{value}  «preferenciaciencias»  y el texto «Ciencias», checkbox con el \sphinxcode{name}  «preferencia» , \sphinxcode{value}  «preferencialetras»  y el texto «Letras».

\item {} 
Hay una lista desplegable múltiple con el \sphinxcode{name} «asignatura» y con las siguientes opciones: opción «Historia» con el \sphinxcode{value} historia, opción «Geografía» con el \sphinxcode{value} geografia, opción «Lengua» con el \sphinxcode{value} lengua, opción «Matemáticas» con el \sphinxcode{value} matematicas.

\item {} 
Hay una lista desplegable múltiple con el \sphinxcode{name} «asignatura» y con las siguientes opciones: opción «Geografía» con el \sphinxcode{value} geografia, opción «Lengua» con el \sphinxcode{value} lengua, opción «Matemáticas» con el \sphinxcode{value} matematicas, opción «Historia» con el \sphinxcode{value} historia.

\item {} 
Hay un \sphinxcode{textarea} que mide 4 filas y 49 columnas que lleva dentro el texto «Escriba aquí, por favor»

\item {} 
Hay los siguientes cuadros de texto:cuadro de texto con el texto «Nombre» y el \sphinxcode{name} nombre, cuadro de texto con el texto «Apellidos» y el \sphinxcode{name} apellidos, cuadro de texto con el texto «Direccion» y el \sphinxcode{name} direccion

\item {} 
Contiene los siguientes \sphinxcode{radiobuttons}:radio con el \sphinxcode{name}  «asignatura» , \sphinxcode{value}  «asignaturalengua»  y el texto «Lengua», radio con el \sphinxcode{name}  «asignatura» , \sphinxcode{value}  «asignaturamatematicas»  y el texto «Matemáticas», radio con el \sphinxcode{name}  «asignatura» , \sphinxcode{value}  «asignaturahistoria»  y el texto «Historia», radio con el \sphinxcode{name}  «asignatura» , \sphinxcode{value}  «asignaturageografia»  y el texto «Geografía».

\end{itemize}

\noindent{\hspace*{\fill}\sphinxincludegraphics[scale=0.6]{{foto_formulario_19}.png}\hspace*{\fill}}

Solución:

\begin{sphinxVerbatim}[commandchars=\\\{\}]
\PYG{p}{\PYGZlt{}}\PYG{n+nt}{form}\PYG{p}{\PYGZgt{}}
\PYG{p}{\PYGZlt{}}\PYG{n+nt}{fieldset}\PYG{p}{\PYGZgt{}}
  \PYG{p}{\PYGZlt{}}\PYG{n+nt}{legend}\PYG{p}{\PYGZgt{}}Complete, por favor\PYG{p}{\PYGZlt{}}\PYG{p}{/}\PYG{n+nt}{legend}\PYG{p}{\PYGZgt{}}
  Nombre\PYG{p}{\PYGZlt{}}\PYG{n+nt}{input} \PYG{n+na}{type}\PYG{o}{=}\PYG{l+s}{\PYGZsq{}text\PYGZsq{}} \PYG{n+na}{name}\PYG{o}{=}\PYG{l+s}{\PYGZsq{}nombre\PYGZsq{}}\PYG{p}{\PYGZgt{}}
  Apellidos\PYG{p}{\PYGZlt{}}\PYG{n+nt}{input} \PYG{n+na}{type}\PYG{o}{=}\PYG{l+s}{\PYGZsq{}text\PYGZsq{}} \PYG{n+na}{name}\PYG{o}{=}\PYG{l+s}{\PYGZsq{}apellidos\PYGZsq{}}\PYG{p}{\PYGZgt{}}
  \PYG{p}{\PYGZlt{}}\PYG{n+nt}{br}\PYG{p}{/}\PYG{p}{\PYGZgt{}}
  \PYG{p}{\PYGZlt{}}\PYG{n+nt}{input} \PYG{n+na}{type}\PYG{o}{=}\PYG{l+s}{\PYGZsq{}checkbox\PYGZsq{}} \PYG{n+na}{name}\PYG{o}{=}\PYG{l+s}{\PYGZsq{}preferencia\PYGZsq{}} \PYG{n+na}{value}\PYG{o}{=}\PYG{l+s}{\PYGZsq{}preferenciaciencias\PYGZsq{}}\PYG{p}{\PYGZgt{}} Ciencias   \PYG{p}{\PYGZlt{}}\PYG{n+nt}{br}\PYG{p}{/}\PYG{p}{\PYGZgt{}}
  \PYG{p}{\PYGZlt{}}\PYG{n+nt}{input} \PYG{n+na}{type}\PYG{o}{=}\PYG{l+s}{\PYGZsq{}checkbox\PYGZsq{}} \PYG{n+na}{name}\PYG{o}{=}\PYG{l+s}{\PYGZsq{}preferencia\PYGZsq{}} \PYG{n+na}{value}\PYG{o}{=}\PYG{l+s}{\PYGZsq{}preferencialetras\PYGZsq{}}\PYG{p}{\PYGZgt{}} Letras   \PYG{p}{\PYGZlt{}}\PYG{n+nt}{br}\PYG{p}{/}\PYG{p}{\PYGZgt{}}
  \PYG{p}{\PYGZlt{}}\PYG{n+nt}{br}\PYG{p}{/}\PYG{p}{\PYGZgt{}}
  \PYG{p}{\PYGZlt{}}\PYG{n+nt}{select} \PYG{n+na}{name}\PYG{o}{=}\PYG{l+s}{\PYGZsq{}asignatura\PYGZsq{}} \PYG{n+na}{multiple}\PYG{o}{=}\PYG{l+s}{\PYGZsq{}multiple\PYGZsq{}}\PYG{p}{\PYGZgt{}}
    \PYG{p}{\PYGZlt{}}\PYG{n+nt}{option} \PYG{n+na}{value}\PYG{o}{=}\PYG{l+s}{\PYGZsq{}historia\PYGZsq{}}\PYG{p}{\PYGZgt{}}Historia\PYG{p}{\PYGZlt{}}\PYG{p}{/}\PYG{n+nt}{option}\PYG{p}{\PYGZgt{}}
    \PYG{p}{\PYGZlt{}}\PYG{n+nt}{option} \PYG{n+na}{value}\PYG{o}{=}\PYG{l+s}{\PYGZsq{}geografia\PYGZsq{}}\PYG{p}{\PYGZgt{}}Geografía\PYG{p}{\PYGZlt{}}\PYG{p}{/}\PYG{n+nt}{option}\PYG{p}{\PYGZgt{}}
    \PYG{p}{\PYGZlt{}}\PYG{n+nt}{option} \PYG{n+na}{value}\PYG{o}{=}\PYG{l+s}{\PYGZsq{}lengua\PYGZsq{}}\PYG{p}{\PYGZgt{}}Lengua\PYG{p}{\PYGZlt{}}\PYG{p}{/}\PYG{n+nt}{option}\PYG{p}{\PYGZgt{}}
    \PYG{p}{\PYGZlt{}}\PYG{n+nt}{option} \PYG{n+na}{value}\PYG{o}{=}\PYG{l+s}{\PYGZsq{}matematicas\PYGZsq{}}\PYG{p}{\PYGZgt{}}Matemáticas\PYG{p}{\PYGZlt{}}\PYG{p}{/}\PYG{n+nt}{option}\PYG{p}{\PYGZgt{}}
  \PYG{p}{\PYGZlt{}}\PYG{p}{/}\PYG{n+nt}{select}\PYG{p}{\PYGZgt{}}
  \PYG{p}{\PYGZlt{}}\PYG{n+nt}{br}\PYG{p}{/}\PYG{p}{\PYGZgt{}}
\PYG{p}{\PYGZlt{}}\PYG{p}{/}\PYG{n+nt}{fieldset}\PYG{p}{\PYGZgt{}}
\PYG{p}{\PYGZlt{}}\PYG{n+nt}{fieldset}\PYG{p}{\PYGZgt{}}
  \PYG{p}{\PYGZlt{}}\PYG{n+nt}{legend}\PYG{p}{\PYGZgt{}}Indique\PYG{p}{\PYGZlt{}}\PYG{p}{/}\PYG{n+nt}{legend}\PYG{p}{\PYGZgt{}}
  \PYG{p}{\PYGZlt{}}\PYG{n+nt}{select} \PYG{n+na}{name}\PYG{o}{=}\PYG{l+s}{\PYGZsq{}asignatura\PYGZsq{}} \PYG{p}{\PYGZgt{}}
    \PYG{p}{\PYGZlt{}}\PYG{n+nt}{option} \PYG{n+na}{value}\PYG{o}{=}\PYG{l+s}{\PYGZsq{}geografia\PYGZsq{}}\PYG{p}{\PYGZgt{}}Geografía\PYG{p}{\PYGZlt{}}\PYG{p}{/}\PYG{n+nt}{option}\PYG{p}{\PYGZgt{}}
    \PYG{p}{\PYGZlt{}}\PYG{n+nt}{option} \PYG{n+na}{value}\PYG{o}{=}\PYG{l+s}{\PYGZsq{}lengua\PYGZsq{}}\PYG{p}{\PYGZgt{}}Lengua\PYG{p}{\PYGZlt{}}\PYG{p}{/}\PYG{n+nt}{option}\PYG{p}{\PYGZgt{}}
    \PYG{p}{\PYGZlt{}}\PYG{n+nt}{option} \PYG{n+na}{value}\PYG{o}{=}\PYG{l+s}{\PYGZsq{}matematicas\PYGZsq{}}\PYG{p}{\PYGZgt{}}Matemáticas\PYG{p}{\PYGZlt{}}\PYG{p}{/}\PYG{n+nt}{option}\PYG{p}{\PYGZgt{}}
    \PYG{p}{\PYGZlt{}}\PYG{n+nt}{option} \PYG{n+na}{value}\PYG{o}{=}\PYG{l+s}{\PYGZsq{}historia\PYGZsq{}}\PYG{p}{\PYGZgt{}}Historia\PYG{p}{\PYGZlt{}}\PYG{p}{/}\PYG{n+nt}{option}\PYG{p}{\PYGZgt{}}
  \PYG{p}{\PYGZlt{}}\PYG{p}{/}\PYG{n+nt}{select}\PYG{p}{\PYGZgt{}}
  \PYG{p}{\PYGZlt{}}\PYG{n+nt}{br}\PYG{p}{/}\PYG{p}{\PYGZgt{}}
  \PYG{p}{\PYGZlt{}}\PYG{n+nt}{textarea} \PYG{n+na}{rows}\PYG{o}{=}\PYG{l+s}{\PYGZsq{}4\PYGZsq{}} \PYG{n+na}{cols}\PYG{o}{=}\PYG{l+s}{\PYGZsq{}49\PYGZsq{}}\PYG{p}{\PYGZgt{}}
    Escriba aquí, por favor
  \PYG{p}{\PYGZlt{}}\PYG{p}{/}\PYG{n+nt}{textarea}\PYG{p}{\PYGZgt{}}  \PYG{p}{\PYGZlt{}}\PYG{n+nt}{br}\PYG{p}{/}\PYG{p}{\PYGZgt{}}
  Nombre\PYG{p}{\PYGZlt{}}\PYG{n+nt}{input} \PYG{n+na}{type}\PYG{o}{=}\PYG{l+s}{\PYGZsq{}text\PYGZsq{}} \PYG{n+na}{name}\PYG{o}{=}\PYG{l+s}{\PYGZsq{}nombre\PYGZsq{}}\PYG{p}{\PYGZgt{}}
  Apellidos\PYG{p}{\PYGZlt{}}\PYG{n+nt}{input} \PYG{n+na}{type}\PYG{o}{=}\PYG{l+s}{\PYGZsq{}text\PYGZsq{}} \PYG{n+na}{name}\PYG{o}{=}\PYG{l+s}{\PYGZsq{}apellidos\PYGZsq{}}\PYG{p}{\PYGZgt{}}
  Direccion\PYG{p}{\PYGZlt{}}\PYG{n+nt}{input} \PYG{n+na}{type}\PYG{o}{=}\PYG{l+s}{\PYGZsq{}text\PYGZsq{}} \PYG{n+na}{name}\PYG{o}{=}\PYG{l+s}{\PYGZsq{}direccion\PYGZsq{}}\PYG{p}{\PYGZgt{}}
  \PYG{p}{\PYGZlt{}}\PYG{n+nt}{br}\PYG{p}{/}\PYG{p}{\PYGZgt{}}
  \PYG{p}{\PYGZlt{}}\PYG{n+nt}{input} \PYG{n+na}{type}\PYG{o}{=}\PYG{l+s}{\PYGZsq{}radio\PYGZsq{}} \PYG{n+na}{name}\PYG{o}{=}\PYG{l+s}{\PYGZsq{}asignatura\PYGZsq{}} \PYG{n+na}{value}\PYG{o}{=}\PYG{l+s}{\PYGZsq{}asignaturalengua\PYGZsq{}}\PYG{p}{\PYGZgt{}} Lengua 
  \PYG{p}{\PYGZlt{}}\PYG{n+nt}{input} \PYG{n+na}{type}\PYG{o}{=}\PYG{l+s}{\PYGZsq{}radio\PYGZsq{}} \PYG{n+na}{name}\PYG{o}{=}\PYG{l+s}{\PYGZsq{}asignatura\PYGZsq{}} \PYG{n+na}{value}\PYG{o}{=}\PYG{l+s}{\PYGZsq{}asignaturamatematicas\PYGZsq{}}\PYG{p}{\PYGZgt{}} Matemáticas 
  \PYG{p}{\PYGZlt{}}\PYG{n+nt}{input} \PYG{n+na}{type}\PYG{o}{=}\PYG{l+s}{\PYGZsq{}radio\PYGZsq{}} \PYG{n+na}{name}\PYG{o}{=}\PYG{l+s}{\PYGZsq{}asignatura\PYGZsq{}} \PYG{n+na}{value}\PYG{o}{=}\PYG{l+s}{\PYGZsq{}asignaturahistoria\PYGZsq{}}\PYG{p}{\PYGZgt{}} Historia 
  \PYG{p}{\PYGZlt{}}\PYG{n+nt}{input} \PYG{n+na}{type}\PYG{o}{=}\PYG{l+s}{\PYGZsq{}radio\PYGZsq{}} \PYG{n+na}{name}\PYG{o}{=}\PYG{l+s}{\PYGZsq{}asignatura\PYGZsq{}} \PYG{n+na}{value}\PYG{o}{=}\PYG{l+s}{\PYGZsq{}asignaturageografia\PYGZsq{}}\PYG{p}{\PYGZgt{}} Geografía 
  \PYG{p}{\PYGZlt{}}\PYG{n+nt}{br}\PYG{p}{/}\PYG{p}{\PYGZgt{}}
\PYG{p}{\PYGZlt{}}\PYG{p}{/}\PYG{n+nt}{fieldset}\PYG{p}{\PYGZgt{}}
\PYG{p}{\PYGZlt{}}\PYG{p}{/}\PYG{n+nt}{form}\PYG{p}{\PYGZgt{}}
\end{sphinxVerbatim}


\section{Formulario 20}
\label{\detokenize{ejercicios/formularios/anexo_formularios:formulario-20}}
Generar el formulario siguiente de acuerdo a los siguientes requisitos
\begin{itemize}
\item {} 
Contiene los siguientes \sphinxcode{checkboxes}:checkbox con el \sphinxcode{name}  «ciclo» , \sphinxcode{value}  «cicloasir»  y el texto «ASIR», checkbox con el \sphinxcode{name}  «ciclo» , \sphinxcode{value}  «ciclodam»  y el texto «DAM», checkbox con el \sphinxcode{name}  «ciclo» , \sphinxcode{value}  «ciclodaw»  y el texto «DAW».

\end{itemize}

\noindent{\hspace*{\fill}\sphinxincludegraphics[scale=0.6]{{foto_formulario_20}.png}\hspace*{\fill}}

Solución:

\begin{sphinxVerbatim}[commandchars=\\\{\}]
\PYG{p}{\PYGZlt{}}\PYG{n+nt}{form}\PYG{p}{\PYGZgt{}}
\PYG{p}{\PYGZlt{}}\PYG{n+nt}{fieldset}\PYG{p}{\PYGZgt{}}
  \PYG{p}{\PYGZlt{}}\PYG{n+nt}{legend}\PYG{p}{\PYGZgt{}}Rellenar\PYG{p}{\PYGZlt{}}\PYG{p}{/}\PYG{n+nt}{legend}\PYG{p}{\PYGZgt{}}
  \PYG{p}{\PYGZlt{}}\PYG{n+nt}{input} \PYG{n+na}{type}\PYG{o}{=}\PYG{l+s}{\PYGZsq{}checkbox\PYGZsq{}} \PYG{n+na}{name}\PYG{o}{=}\PYG{l+s}{\PYGZsq{}ciclo\PYGZsq{}} \PYG{n+na}{value}\PYG{o}{=}\PYG{l+s}{\PYGZsq{}cicloasir\PYGZsq{}}\PYG{p}{\PYGZgt{}} ASIR   \PYG{p}{\PYGZlt{}}\PYG{n+nt}{br}\PYG{p}{/}\PYG{p}{\PYGZgt{}}
  \PYG{p}{\PYGZlt{}}\PYG{n+nt}{input} \PYG{n+na}{type}\PYG{o}{=}\PYG{l+s}{\PYGZsq{}checkbox\PYGZsq{}} \PYG{n+na}{name}\PYG{o}{=}\PYG{l+s}{\PYGZsq{}ciclo\PYGZsq{}} \PYG{n+na}{value}\PYG{o}{=}\PYG{l+s}{\PYGZsq{}ciclodam\PYGZsq{}}\PYG{p}{\PYGZgt{}} DAM   \PYG{p}{\PYGZlt{}}\PYG{n+nt}{br}\PYG{p}{/}\PYG{p}{\PYGZgt{}}
  \PYG{p}{\PYGZlt{}}\PYG{n+nt}{input} \PYG{n+na}{type}\PYG{o}{=}\PYG{l+s}{\PYGZsq{}checkbox\PYGZsq{}} \PYG{n+na}{name}\PYG{o}{=}\PYG{l+s}{\PYGZsq{}ciclo\PYGZsq{}} \PYG{n+na}{value}\PYG{o}{=}\PYG{l+s}{\PYGZsq{}ciclodaw\PYGZsq{}}\PYG{p}{\PYGZgt{}} DAW   \PYG{p}{\PYGZlt{}}\PYG{n+nt}{br}\PYG{p}{/}\PYG{p}{\PYGZgt{}}
  \PYG{p}{\PYGZlt{}}\PYG{n+nt}{br}\PYG{p}{/}\PYG{p}{\PYGZgt{}}
\PYG{p}{\PYGZlt{}}\PYG{p}{/}\PYG{n+nt}{fieldset}\PYG{p}{\PYGZgt{}}
\PYG{p}{\PYGZlt{}}\PYG{p}{/}\PYG{n+nt}{form}\PYG{p}{\PYGZgt{}}
\end{sphinxVerbatim}


\chapter{Anexo: ejercicios sobre XSLT}
\label{\detokenize{ejercicios/xslt/anexo_ejercicios_xslt:anexo-ejercicios-sobre-xslt}}\label{\detokenize{ejercicios/xslt/anexo_ejercicios_xslt::doc}}

\section{Fichero origen}
\label{\detokenize{ejercicios/xslt/anexo_ejercicios_xslt:fichero-origen}}
Para los ejercicios siguiente supondremos que se va a trabajar con el fichero que se muestra a continuación:

\begin{sphinxVerbatim}[commandchars=\\\{\}]
\PYG{n+nt}{\PYGZlt{}inventario}\PYG{n+nt}{\PYGZgt{}}
    \PYG{n+nt}{\PYGZlt{}producto} \PYG{n+na}{codigo=}\PYG{l+s}{\PYGZdq{}P1\PYGZdq{}}\PYG{n+nt}{\PYGZgt{}}
        \PYG{n+nt}{\PYGZlt{}peso} \PYG{n+na}{unidad=}\PYG{l+s}{\PYGZdq{}kg\PYGZdq{}}\PYG{n+nt}{\PYGZgt{}}10\PYG{n+nt}{\PYGZlt{}/peso\PYGZgt{}}
        \PYG{n+nt}{\PYGZlt{}nombre}\PYG{n+nt}{\PYGZgt{}}Ordenador\PYG{n+nt}{\PYGZlt{}/nombre\PYGZgt{}}
        \PYG{n+nt}{\PYGZlt{}lugar} \PYG{n+na}{edificio=}\PYG{l+s}{\PYGZdq{}B\PYGZdq{}}\PYG{n+nt}{\PYGZgt{}}
            \PYG{n+nt}{\PYGZlt{}aula}\PYG{n+nt}{\PYGZgt{}}10\PYG{n+nt}{\PYGZlt{}/aula\PYGZgt{}}
        \PYG{n+nt}{\PYGZlt{}/lugar\PYGZgt{}}
    \PYG{n+nt}{\PYGZlt{}/producto\PYGZgt{}}
    \PYG{n+nt}{\PYGZlt{}producto} \PYG{n+na}{codigo=}\PYG{l+s}{\PYGZdq{}P2\PYGZdq{}}\PYG{n+nt}{\PYGZgt{}}
        \PYG{n+nt}{\PYGZlt{}peso} \PYG{n+na}{unidad=}\PYG{l+s}{\PYGZsq{}g\PYGZsq{}}\PYG{n+nt}{\PYGZgt{}}500\PYG{n+nt}{\PYGZlt{}/peso\PYGZgt{}}
        \PYG{n+nt}{\PYGZlt{}nombre}\PYG{n+nt}{\PYGZgt{}}Switch\PYG{n+nt}{\PYGZlt{}/nombre\PYGZgt{}}
        \PYG{n+nt}{\PYGZlt{}lugar} \PYG{n+na}{edificio=}\PYG{l+s}{\PYGZdq{}A\PYGZdq{}}\PYG{n+nt}{\PYGZgt{}}
            \PYG{n+nt}{\PYGZlt{}aula}\PYG{n+nt}{\PYGZgt{}}6\PYG{n+nt}{\PYGZlt{}/aula\PYGZgt{}}
        \PYG{n+nt}{\PYGZlt{}/lugar\PYGZgt{}}
    \PYG{n+nt}{\PYGZlt{}/producto\PYGZgt{}}
\PYG{n+nt}{\PYGZlt{}/inventario\PYGZgt{}}
\end{sphinxVerbatim}


\section{Generación de lista con puntos}
\label{\detokenize{ejercicios/xslt/anexo_ejercicios_xslt:generacion-de-lista-con-puntos}}
Convertir el fichero origen en una lista punteada similar a la que se muestra en la figura:

\noindent{\hspace*{\fill}\sphinxincludegraphics[scale=0.5]{{lista_punteada}.png}\hspace*{\fill}}

\begin{sphinxVerbatim}[commandchars=\\\{\}]
\PYG{n+nt}{\PYGZlt{}xsl:stylesheet} \PYG{n+na}{xmlns:xsl=}\PYG{l+s}{\PYGZdq{}http://www.w3.org/1999/XSL/Transform\PYGZdq{}}\PYG{n+nt}{\PYGZgt{}}
\PYG{n+nt}{\PYGZlt{}xsl:template} \PYG{n+na}{match=}\PYG{l+s}{\PYGZdq{}/\PYGZdq{}}\PYG{n+nt}{\PYGZgt{}}
    \PYG{n+nt}{\PYGZlt{}html}\PYG{n+nt}{\PYGZgt{}}
        \PYG{n+nt}{\PYGZlt{}head}\PYG{n+nt}{\PYGZgt{}}
            \PYG{n+nt}{\PYGZlt{}title}\PYG{n+nt}{\PYGZgt{}}Resultado HTML\PYG{n+nt}{\PYGZlt{}/title\PYGZgt{}}
        \PYG{n+nt}{\PYGZlt{}/head\PYGZgt{}}
        \PYG{n+nt}{\PYGZlt{}body}\PYG{n+nt}{\PYGZgt{}}
            \PYG{n+nt}{\PYGZlt{}ul}\PYG{n+nt}{\PYGZgt{}}
                \PYG{n+nt}{\PYGZlt{}xsl:for\PYGZhy{}each} \PYG{n+na}{select=}\PYG{l+s}{\PYGZdq{}inventario/producto\PYGZdq{}}\PYG{n+nt}{\PYGZgt{}}
                    \PYG{n+nt}{\PYGZlt{}li}\PYG{n+nt}{\PYGZgt{}}
                        Elemento
                        \PYG{n+nt}{\PYGZlt{}xsl:value\PYGZhy{}of} \PYG{n+na}{select=}\PYG{l+s}{\PYGZdq{}./@codigo\PYGZdq{}}\PYG{n+nt}{/\PYGZgt{}}
                        \PYG{n+nt}{\PYGZlt{}ul}\PYG{n+nt}{\PYGZgt{}}
                            \PYG{n+nt}{\PYGZlt{}li}\PYG{n+nt}{\PYGZgt{}}
                                Nombre:
                                \PYG{n+nt}{\PYGZlt{}xsl:value\PYGZhy{}of} \PYG{n+na}{select=}\PYG{l+s}{\PYGZdq{}nombre\PYGZdq{}}\PYG{n+nt}{/\PYGZgt{}}
                            \PYG{n+nt}{\PYGZlt{}/li\PYGZgt{}}
                            \PYG{n+nt}{\PYGZlt{}li}\PYG{n+nt}{\PYGZgt{}}
                                Peso:
                                \PYG{n+nt}{\PYGZlt{}xsl:value\PYGZhy{}of} \PYG{n+na}{select=}\PYG{l+s}{\PYGZdq{}peso\PYGZdq{}}\PYG{n+nt}{/\PYGZgt{}}
                                \PYG{n+nt}{\PYGZlt{}xsl:value\PYGZhy{}of}
                                    \PYG{n+na}{select=}\PYG{l+s}{\PYGZdq{}peso/@unidad\PYGZdq{}}\PYG{n+nt}{/\PYGZgt{}}
                            \PYG{n+nt}{\PYGZlt{}/li\PYGZgt{}}
                        \PYG{n+nt}{\PYGZlt{}/ul\PYGZgt{}}
                    \PYG{n+nt}{\PYGZlt{}/li\PYGZgt{}}
                \PYG{n+nt}{\PYGZlt{}/xsl:for\PYGZhy{}each\PYGZgt{}}
            \PYG{n+nt}{\PYGZlt{}/ul\PYGZgt{}}
        \PYG{n+nt}{\PYGZlt{}/body\PYGZgt{}}
    \PYG{n+nt}{\PYGZlt{}/html\PYGZgt{}}
\PYG{n+nt}{\PYGZlt{}/xsl:template\PYGZgt{}}
\PYG{n+nt}{\PYGZlt{}/xsl:stylesheet\PYGZgt{}}
\end{sphinxVerbatim}


\section{Filtrado}
\label{\detokenize{ejercicios/xslt/anexo_ejercicios_xslt:filtrado}}
Replicar la estructura del fichero pero filtrando
los elementos y haciendo que solo aparezcan los que estén
en el aula A6


\section{Recuperación de elementos pesados}
\label{\detokenize{ejercicios/xslt/anexo_ejercicios_xslt:recuperacion-de-elementos-pesados}}
Se pide un XSLT que muestre exactamente la misma información del fichero origen pero sin mostrar los elementos cuyo peso sea menor de 7.

Una posible solución sería esta:

\begin{sphinxVerbatim}[commandchars=\\\{\}]
\PYG{n+nt}{\PYGZlt{}xsl:stylesheet}
 \PYG{n+na}{xmlns:xsl=}\PYG{l+s}{\PYGZdq{}http://www.w3.org/1999/XSL/Transform\PYGZdq{}}\PYG{n+nt}{\PYGZgt{}}
\PYG{n+nt}{\PYGZlt{}xsl:template} \PYG{n+na}{match=}\PYG{l+s}{\PYGZdq{}/\PYGZdq{}}\PYG{n+nt}{\PYGZgt{}}
    \PYG{n+nt}{\PYGZlt{}inventario}\PYG{n+nt}{\PYGZgt{}}
    \PYG{n+nt}{\PYGZlt{}xsl:for\PYGZhy{}each} \PYG{n+na}{select=}\PYG{l+s}{\PYGZdq{}inventario/producto\PYGZdq{}}\PYG{n+nt}{\PYGZgt{}}
        \PYG{n+nt}{\PYGZlt{}xsl:if} \PYG{n+na}{test=}\PYG{l+s}{\PYGZdq{}peso \PYGZam{}lt; 7\PYGZdq{}}\PYG{n+nt}{\PYGZgt{}}
            \PYG{n+nt}{\PYGZlt{}producto}\PYG{n+nt}{\PYGZgt{}}
                \PYG{n+nt}{\PYGZlt{}peso}\PYG{n+nt}{\PYGZgt{}}
                    \PYG{n+nt}{\PYGZlt{}xsl:value\PYGZhy{}of} \PYG{n+na}{select=}\PYG{l+s}{\PYGZdq{}peso\PYGZdq{}}\PYG{n+nt}{/\PYGZgt{}}
                \PYG{n+nt}{\PYGZlt{}/peso\PYGZgt{}}
                \PYG{n+nt}{\PYGZlt{}nombre}\PYG{n+nt}{\PYGZgt{}}
                    \PYG{n+nt}{\PYGZlt{}xsl:value\PYGZhy{}of} \PYG{n+na}{select=}\PYG{l+s}{\PYGZdq{}nombre\PYGZdq{}}\PYG{n+nt}{/\PYGZgt{}}
                \PYG{n+nt}{\PYGZlt{}/nombre\PYGZgt{}}
                \PYG{n+nt}{\PYGZlt{}lugar}\PYG{n+nt}{\PYGZgt{}}
                    \PYG{n+nt}{\PYGZlt{}xsl:attribute} \PYG{n+na}{name=}\PYG{l+s}{\PYGZdq{}edificio\PYGZdq{}}\PYG{n+nt}{\PYGZgt{}}
                        \PYG{n+nt}{\PYGZlt{}xsl:value\PYGZhy{}of}
                            \PYG{n+na}{select=}\PYG{l+s}{\PYGZdq{}lugar/@edificio\PYGZdq{}}\PYG{n+nt}{/\PYGZgt{}}
                    \PYG{n+nt}{\PYGZlt{}/xsl:attribute\PYGZgt{}}
                    \PYG{n+nt}{\PYGZlt{}aula}\PYG{n+nt}{\PYGZgt{}}
                        \PYG{n+nt}{\PYGZlt{}xsl:value\PYGZhy{}of}
                            \PYG{n+na}{select=}\PYG{l+s}{\PYGZdq{}lugar/aula\PYGZdq{}}\PYG{n+nt}{/\PYGZgt{}}
                    \PYG{n+nt}{\PYGZlt{}/aula\PYGZgt{}}
                \PYG{n+nt}{\PYGZlt{}/lugar\PYGZgt{}}
            \PYG{n+nt}{\PYGZlt{}/producto\PYGZgt{}}
        \PYG{n+nt}{\PYGZlt{}/xsl:if\PYGZgt{}}
    \PYG{n+nt}{\PYGZlt{}/xsl:for\PYGZhy{}each\PYGZgt{}}
    \PYG{n+nt}{\PYGZlt{}/inventario\PYGZgt{}}
\PYG{n+nt}{\PYGZlt{}/xsl:template\PYGZgt{}}
\PYG{n+nt}{\PYGZlt{}/xsl:stylesheet\PYGZgt{}}
\end{sphinxVerbatim}

Sin embargo, \sphinxstylestrong{dicha solución está mal} porque una pregunta básica es «el peso está en kg o en g», por lo que en realidad la condición de filtrado debe refinarse un poco más.

Supongamos entonces que el enunciado correcto pone el peso en kg. Así, la solución entonces podría hacerse de esta manera.


\section{Productos del edificio B}
\label{\detokenize{ejercicios/xslt/anexo_ejercicios_xslt:productos-del-edificio-b}}
Se pide ahora mostrar en el resultado la misma información del fichero origen pero solo en los casos en que el lugar del producto sea el edificio B

La solución es muy parecida, necesitando solamente modificar la condición.

\begin{sphinxVerbatim}[commandchars=\\\{\}]
\PYG{n+nt}{\PYGZlt{}xsl:stylesheet}
 \PYG{n+na}{xmlns:xsl=}\PYG{l+s}{\PYGZdq{}http://www.w3.org/1999/XSL/Transform\PYGZdq{}}\PYG{n+nt}{\PYGZgt{}}
\PYG{n+nt}{\PYGZlt{}xsl:template} \PYG{n+na}{match=}\PYG{l+s}{\PYGZdq{}/\PYGZdq{}}\PYG{n+nt}{\PYGZgt{}}
    \PYG{n+nt}{\PYGZlt{}inventario}\PYG{n+nt}{\PYGZgt{}}
    \PYG{n+nt}{\PYGZlt{}xsl:for\PYGZhy{}each} \PYG{n+na}{select=}\PYG{l+s}{\PYGZdq{}inventario/producto\PYGZdq{}}\PYG{n+nt}{\PYGZgt{}}
        \PYG{n+nt}{\PYGZlt{}xsl:if} \PYG{n+na}{test=}\PYG{l+s}{\PYGZdq{}lugar/@edificio=\PYGZsq{}B\PYGZsq{}\PYGZdq{}}\PYG{n+nt}{\PYGZgt{}}
            \PYG{n+nt}{\PYGZlt{}producto}\PYG{n+nt}{\PYGZgt{}}
                \PYG{n+nt}{\PYGZlt{}peso}\PYG{n+nt}{\PYGZgt{}}
                    \PYG{n+nt}{\PYGZlt{}xsl:value\PYGZhy{}of} \PYG{n+na}{select=}\PYG{l+s}{\PYGZdq{}peso\PYGZdq{}}\PYG{n+nt}{/\PYGZgt{}}
                \PYG{n+nt}{\PYGZlt{}/peso\PYGZgt{}}
                \PYG{n+nt}{\PYGZlt{}nombre}\PYG{n+nt}{\PYGZgt{}}
                    \PYG{n+nt}{\PYGZlt{}xsl:value\PYGZhy{}of} \PYG{n+na}{select=}\PYG{l+s}{\PYGZdq{}nombre\PYGZdq{}}\PYG{n+nt}{/\PYGZgt{}}
                \PYG{n+nt}{\PYGZlt{}/nombre\PYGZgt{}}
                \PYG{n+nt}{\PYGZlt{}lugar}\PYG{n+nt}{\PYGZgt{}}
                    \PYG{n+nt}{\PYGZlt{}xsl:attribute} \PYG{n+na}{name=}\PYG{l+s}{\PYGZdq{}edificio\PYGZdq{}}\PYG{n+nt}{\PYGZgt{}}
                        \PYG{n+nt}{\PYGZlt{}xsl:value\PYGZhy{}of}
                            \PYG{n+na}{select=}\PYG{l+s}{\PYGZdq{}lugar/@edificio\PYGZdq{}}\PYG{n+nt}{/\PYGZgt{}}
                    \PYG{n+nt}{\PYGZlt{}/xsl:attribute\PYGZgt{}}
                    \PYG{n+nt}{\PYGZlt{}aula}\PYG{n+nt}{\PYGZgt{}}
                        \PYG{n+nt}{\PYGZlt{}xsl:value\PYGZhy{}of}
                            \PYG{n+na}{select=}\PYG{l+s}{\PYGZdq{}lugar/aula\PYGZdq{}}\PYG{n+nt}{/\PYGZgt{}}
                    \PYG{n+nt}{\PYGZlt{}/aula\PYGZgt{}}
                \PYG{n+nt}{\PYGZlt{}/lugar\PYGZgt{}}
            \PYG{n+nt}{\PYGZlt{}/producto\PYGZgt{}}
        \PYG{n+nt}{\PYGZlt{}/xsl:if\PYGZgt{}}
    \PYG{n+nt}{\PYGZlt{}/xsl:for\PYGZhy{}each\PYGZgt{}}
    \PYG{n+nt}{\PYGZlt{}/inventario\PYGZgt{}}
\PYG{n+nt}{\PYGZlt{}/xsl:template\PYGZgt{}}
\PYG{n+nt}{\PYGZlt{}/xsl:stylesheet\PYGZgt{}}
\end{sphinxVerbatim}


\section{Tabla de localizaciones}
\label{\detokenize{ejercicios/xslt/anexo_ejercicios_xslt:tabla-de-localizaciones}}
Generar una tabla HTML que muestre la información del fichero origen de la manera siguiente:

\noindent{\hspace*{\fill}\sphinxincludegraphics[scale=0.5]{{xslt_tabla_edificio_aula}.png}\hspace*{\fill}}

\begin{sphinxVerbatim}[commandchars=\\\{\}]
\PYG{n+nt}{\PYGZlt{}xsl:stylesheet}
 \PYG{n+na}{xmlns:xsl=}\PYG{l+s}{\PYGZdq{}http://www.w3.org/1999/XSL/Transform\PYGZdq{}}\PYG{n+nt}{\PYGZgt{}}
\PYG{n+nt}{\PYGZlt{}xsl:template} \PYG{n+na}{match=}\PYG{l+s}{\PYGZdq{}/\PYGZdq{}}\PYG{n+nt}{\PYGZgt{}}
    \PYG{n+nt}{\PYGZlt{}inventario}\PYG{n+nt}{\PYGZgt{}}
    \PYG{n+nt}{\PYGZlt{}xsl:for\PYGZhy{}each} \PYG{n+na}{select=}\PYG{l+s}{\PYGZdq{}inventario/producto\PYGZdq{}}\PYG{n+nt}{\PYGZgt{}}
        \PYG{n+nt}{\PYGZlt{}xsl:if} \PYG{n+na}{test=}\PYG{l+s}{\PYGZdq{}lugar/@edificio=\PYGZsq{}B\PYGZsq{}\PYGZdq{}}\PYG{n+nt}{\PYGZgt{}}
            \PYG{n+nt}{\PYGZlt{}producto}\PYG{n+nt}{\PYGZgt{}}
                \PYG{n+nt}{\PYGZlt{}peso}\PYG{n+nt}{\PYGZgt{}}
                    \PYG{n+nt}{\PYGZlt{}xsl:value\PYGZhy{}of} \PYG{n+na}{select=}\PYG{l+s}{\PYGZdq{}peso\PYGZdq{}}\PYG{n+nt}{/\PYGZgt{}}
                \PYG{n+nt}{\PYGZlt{}/peso\PYGZgt{}}
                \PYG{n+nt}{\PYGZlt{}nombre}\PYG{n+nt}{\PYGZgt{}}
                    \PYG{n+nt}{\PYGZlt{}xsl:value\PYGZhy{}of} \PYG{n+na}{select=}\PYG{l+s}{\PYGZdq{}nombre\PYGZdq{}}\PYG{n+nt}{/\PYGZgt{}}
                \PYG{n+nt}{\PYGZlt{}/nombre\PYGZgt{}}
                \PYG{n+nt}{\PYGZlt{}lugar}\PYG{n+nt}{\PYGZgt{}}
                    \PYG{n+nt}{\PYGZlt{}xsl:attribute} \PYG{n+na}{name=}\PYG{l+s}{\PYGZdq{}edificio\PYGZdq{}}\PYG{n+nt}{\PYGZgt{}}
                        \PYG{n+nt}{\PYGZlt{}xsl:value\PYGZhy{}of}
                            \PYG{n+na}{select=}\PYG{l+s}{\PYGZdq{}lugar/@edificio\PYGZdq{}}\PYG{n+nt}{/\PYGZgt{}}
                    \PYG{n+nt}{\PYGZlt{}/xsl:attribute\PYGZgt{}}
                    \PYG{n+nt}{\PYGZlt{}aula}\PYG{n+nt}{\PYGZgt{}}
                        \PYG{n+nt}{\PYGZlt{}xsl:value\PYGZhy{}of}
                            \PYG{n+na}{select=}\PYG{l+s}{\PYGZdq{}lugar/aula\PYGZdq{}}\PYG{n+nt}{/\PYGZgt{}}
                    \PYG{n+nt}{\PYGZlt{}/aula\PYGZgt{}}
                \PYG{n+nt}{\PYGZlt{}/lugar\PYGZgt{}}
            \PYG{n+nt}{\PYGZlt{}/producto\PYGZgt{}}
        \PYG{n+nt}{\PYGZlt{}/xsl:if\PYGZgt{}}
    \PYG{n+nt}{\PYGZlt{}/xsl:for\PYGZhy{}each\PYGZgt{}}
    \PYG{n+nt}{\PYGZlt{}/inventario\PYGZgt{}}
\PYG{n+nt}{\PYGZlt{}/xsl:template\PYGZgt{}}
\PYG{n+nt}{\PYGZlt{}/xsl:stylesheet\PYGZgt{}}
\end{sphinxVerbatim}


\section{Tablas con edificios separados}
\label{\detokenize{ejercicios/xslt/anexo_ejercicios_xslt:tablas-con-edificios-separados}}
Hacer una plantilla que fabrique una tabla
con los datos de los productos del edificio A
y otra tabla separada para los productos del edificio B

\noindent{\hspace*{\fill}\sphinxincludegraphics[scale=0.5]{{xslt_tabla_edificio_separadas}.png}\hspace*{\fill}}

Una posible solución sería esta:

\begin{sphinxVerbatim}[commandchars=\\\{\}]
\PYG{n+nt}{\PYGZlt{}xsl:stylesheet}
\PYG{n+na}{xmlns:xsl=}\PYG{l+s}{\PYGZdq{}http://www.w3.org/1999/XSL/Transform\PYGZdq{}}\PYG{n+nt}{\PYGZgt{}}
\PYG{n+nt}{\PYGZlt{}xsl:template} \PYG{n+na}{match=}\PYG{l+s}{\PYGZdq{}/\PYGZdq{}}\PYG{n+nt}{\PYGZgt{}}
    \PYG{n+nt}{\PYGZlt{}html}\PYG{n+nt}{\PYGZgt{}}
        \PYG{n+nt}{\PYGZlt{}head}\PYG{n+nt}{\PYGZgt{}}\PYG{n+nt}{\PYGZlt{}title}\PYG{n+nt}{\PYGZgt{}}Datos por edificio\PYG{n+nt}{\PYGZlt{}/title\PYGZgt{}}\PYG{n+nt}{\PYGZlt{}/head\PYGZgt{}}
        \PYG{n+nt}{\PYGZlt{}body}\PYG{n+nt}{\PYGZgt{}}
            \PYG{n+nt}{\PYGZlt{}h1}\PYG{n+nt}{\PYGZgt{}}Edificio A\PYG{n+nt}{\PYGZlt{}/h1\PYGZgt{}}
            \PYG{n+nt}{\PYGZlt{}table} \PYG{n+na}{border=}\PYG{l+s}{\PYGZsq{}1\PYGZsq{}}\PYG{n+nt}{\PYGZgt{}}
            \PYG{n+nt}{\PYGZlt{}xsl:for\PYGZhy{}each} \PYG{n+na}{select=}\PYG{l+s}{\PYGZdq{}inventario/producto\PYGZdq{}}\PYG{n+nt}{\PYGZgt{}}
                \PYG{n+nt}{\PYGZlt{}xsl:if} \PYG{n+na}{test=}\PYG{l+s}{\PYGZdq{}lugar/@edificio=\PYGZsq{}A\PYGZsq{}\PYGZdq{}}\PYG{n+nt}{\PYGZgt{}}
                    \PYG{n+nt}{\PYGZlt{}tr}\PYG{n+nt}{\PYGZgt{}}
                        \PYG{n+nt}{\PYGZlt{}td}\PYG{n+nt}{\PYGZgt{}}
                            \PYG{n+nt}{\PYGZlt{}xsl:value\PYGZhy{}of}
                                \PYG{n+na}{select=}\PYG{l+s}{\PYGZdq{}nombre\PYGZdq{}}\PYG{n+nt}{/\PYGZgt{}}
                        \PYG{n+nt}{\PYGZlt{}/td\PYGZgt{}}
                        \PYG{n+nt}{\PYGZlt{}td}\PYG{n+nt}{\PYGZgt{}}
                            \PYG{n+nt}{\PYGZlt{}xsl:value\PYGZhy{}of}
                                \PYG{n+na}{select=}\PYG{l+s}{\PYGZdq{}peso\PYGZdq{}}\PYG{n+nt}{/\PYGZgt{}}
                        \PYG{n+nt}{\PYGZlt{}/td\PYGZgt{}}
                        \PYG{n+nt}{\PYGZlt{}td}\PYG{n+nt}{\PYGZgt{}}
                            \PYG{n+nt}{\PYGZlt{}xsl:value\PYGZhy{}of}
                            \PYG{n+na}{select=}\PYG{l+s}{\PYGZdq{}lugar/@edificio\PYGZdq{}}\PYG{n+nt}{/\PYGZgt{}}
                            \PYG{n+nt}{\PYGZlt{}xsl:value\PYGZhy{}of}
                            \PYG{n+na}{select=}\PYG{l+s}{\PYGZdq{}lugar/aula\PYGZdq{}}\PYG{n+nt}{/\PYGZgt{}}
                        \PYG{n+nt}{\PYGZlt{}/td\PYGZgt{}}
                    \PYG{n+nt}{\PYGZlt{}/tr\PYGZgt{}}
                \PYG{n+nt}{\PYGZlt{}/xsl:if\PYGZgt{}}
            \PYG{n+nt}{\PYGZlt{}/xsl:for\PYGZhy{}each\PYGZgt{}}
            \PYG{n+nt}{\PYGZlt{}/table\PYGZgt{}}
            \PYG{n+nt}{\PYGZlt{}h1}\PYG{n+nt}{\PYGZgt{}}Edificio B\PYG{n+nt}{\PYGZlt{}/h1\PYGZgt{}}
            \PYG{n+nt}{\PYGZlt{}table} \PYG{n+na}{border=}\PYG{l+s}{\PYGZsq{}1\PYGZsq{}}\PYG{n+nt}{\PYGZgt{}}
            \PYG{n+nt}{\PYGZlt{}xsl:for\PYGZhy{}each} \PYG{n+na}{select=}\PYG{l+s}{\PYGZdq{}inventario/producto\PYGZdq{}}\PYG{n+nt}{\PYGZgt{}}
                \PYG{n+nt}{\PYGZlt{}xsl:if} \PYG{n+na}{test=}\PYG{l+s}{\PYGZdq{}lugar/@edificio=\PYGZsq{}B\PYGZsq{}\PYGZdq{}}\PYG{n+nt}{\PYGZgt{}}
                    \PYG{n+nt}{\PYGZlt{}tr}\PYG{n+nt}{\PYGZgt{}}
                        \PYG{n+nt}{\PYGZlt{}td}\PYG{n+nt}{\PYGZgt{}}
                            \PYG{n+nt}{\PYGZlt{}xsl:value\PYGZhy{}of}
                                \PYG{n+na}{select=}\PYG{l+s}{\PYGZdq{}nombre\PYGZdq{}}\PYG{n+nt}{/\PYGZgt{}}
                        \PYG{n+nt}{\PYGZlt{}/td\PYGZgt{}}
                        \PYG{n+nt}{\PYGZlt{}td}\PYG{n+nt}{\PYGZgt{}}
                            \PYG{n+nt}{\PYGZlt{}xsl:value\PYGZhy{}of}
                                \PYG{n+na}{select=}\PYG{l+s}{\PYGZdq{}peso\PYGZdq{}}\PYG{n+nt}{/\PYGZgt{}}
                        \PYG{n+nt}{\PYGZlt{}/td\PYGZgt{}}
                        \PYG{n+nt}{\PYGZlt{}td}\PYG{n+nt}{\PYGZgt{}}
                            \PYG{n+nt}{\PYGZlt{}xsl:value\PYGZhy{}of}
                            \PYG{n+na}{select=}\PYG{l+s}{\PYGZdq{}lugar/@edificio\PYGZdq{}}\PYG{n+nt}{/\PYGZgt{}}
                            \PYG{n+nt}{\PYGZlt{}xsl:value\PYGZhy{}of}
                            \PYG{n+na}{select=}\PYG{l+s}{\PYGZdq{}lugar/aula\PYGZdq{}}\PYG{n+nt}{/\PYGZgt{}}
                        \PYG{n+nt}{\PYGZlt{}/td\PYGZgt{}}
                    \PYG{n+nt}{\PYGZlt{}/tr\PYGZgt{}}
                \PYG{n+nt}{\PYGZlt{}/xsl:if\PYGZgt{}}
            \PYG{n+nt}{\PYGZlt{}/xsl:for\PYGZhy{}each\PYGZgt{}}
            \PYG{n+nt}{\PYGZlt{}/table\PYGZgt{}}
        \PYG{n+nt}{\PYGZlt{}/body\PYGZgt{}}
    \PYG{n+nt}{\PYGZlt{}/html\PYGZgt{}}
\PYG{n+nt}{\PYGZlt{}/xsl:template\PYGZgt{}}
\PYG{n+nt}{\PYGZlt{}/xsl:stylesheet\PYGZgt{}}
\end{sphinxVerbatim}


\section{Productos del aula 6}
\label{\detokenize{ejercicios/xslt/anexo_ejercicios_xslt:productos-del-aula-6}}
Se pide generar un inventario en el que aparezcan solo los nombres de productos que estén en el aula 6.

\begin{sphinxVerbatim}[commandchars=\\\{\}]
\PYG{n+nt}{\PYGZlt{}xsl:stylesheet}
  \PYG{n+na}{xmlns:xsl=}\PYG{l+s}{\PYGZdq{}http://www.w3.org/1999/XSL/Transform\PYGZdq{}}\PYG{n+nt}{\PYGZgt{}}
\PYG{n+nt}{\PYGZlt{}xsl:template} \PYG{n+na}{match=}\PYG{l+s}{\PYGZdq{}/\PYGZdq{}}\PYG{n+nt}{\PYGZgt{}}
  \PYG{n+nt}{\PYGZlt{}inventario}\PYG{n+nt}{\PYGZgt{}}
  \PYG{n+nt}{\PYGZlt{}xsl:for\PYGZhy{}each} \PYG{n+na}{select=}\PYG{l+s}{\PYGZdq{}/inventario/producto\PYGZdq{}}\PYG{n+nt}{\PYGZgt{}}
    \PYG{n+nt}{\PYGZlt{}xsl:if} \PYG{n+na}{test=}\PYG{l+s}{\PYGZdq{} lugar/aula= \PYGZsq{}6\PYGZsq{} \PYGZdq{}}\PYG{n+nt}{\PYGZgt{}}
      \PYG{n+nt}{\PYGZlt{}nombre}\PYG{n+nt}{\PYGZgt{}}
        \PYG{n+nt}{\PYGZlt{}xsl:value\PYGZhy{}of} \PYG{n+na}{select=}\PYG{l+s}{\PYGZdq{}nombre\PYGZdq{}}\PYG{n+nt}{/\PYGZgt{}}
      \PYG{n+nt}{\PYGZlt{}/nombre\PYGZgt{}}
    \PYG{n+nt}{\PYGZlt{}/xsl:if\PYGZgt{}}
  \PYG{n+nt}{\PYGZlt{}/xsl:for\PYGZhy{}each\PYGZgt{}}
  \PYG{n+nt}{\PYGZlt{}/inventario\PYGZgt{}}
\PYG{n+nt}{\PYGZlt{}/xsl:template\PYGZgt{}}
\PYG{n+nt}{\PYGZlt{}/xsl:stylesheet\PYGZgt{}}
\end{sphinxVerbatim}


\section{Productos del edificio B}
\label{\detokenize{ejercicios/xslt/anexo_ejercicios_xslt:id1}}
El siguiente ejercicio es muy parecido al anterior, con la salvedad de que ahora solo nos piden los nombres de los productos ubicados en el edificio B.

\begin{sphinxVerbatim}[commandchars=\\\{\}]
\PYG{n+nt}{\PYGZlt{}inventario}\PYG{n+nt}{\PYGZgt{}}
  \PYG{n+nt}{\PYGZlt{}xsl:for\PYGZhy{}each} \PYG{n+na}{select=}\PYG{l+s}{\PYGZdq{}/inventario/producto\PYGZdq{}}\PYG{n+nt}{\PYGZgt{}}
    \PYG{n+nt}{\PYGZlt{}xsl:if} \PYG{n+na}{test=}\PYG{l+s}{\PYGZdq{} lugar/@edificio = \PYGZsq{}B\PYGZsq{} \PYGZdq{}}\PYG{n+nt}{\PYGZgt{}}
      \PYG{n+nt}{\PYGZlt{}nombre}\PYG{n+nt}{\PYGZgt{}}
        \PYG{n+nt}{\PYGZlt{}xsl:value\PYGZhy{}of} \PYG{n+na}{select=}\PYG{l+s}{\PYGZdq{}nombre\PYGZdq{}}\PYG{n+nt}{/\PYGZgt{}}
      \PYG{n+nt}{\PYGZlt{}/nombre\PYGZgt{}}
    \PYG{n+nt}{\PYGZlt{}/xsl:if\PYGZgt{}}
  \PYG{n+nt}{\PYGZlt{}/xsl:for\PYGZhy{}each\PYGZgt{}}
  \PYG{n+nt}{\PYGZlt{}/inventario\PYGZgt{}}
\PYG{n+nt}{\PYGZlt{}/xsl:template\PYGZgt{}}
\PYG{n+nt}{\PYGZlt{}/xsl:stylesheet\PYGZgt{}}
\end{sphinxVerbatim}


\section{Condiciones múltiples: peso ligero y edificio A}
\label{\detokenize{ejercicios/xslt/anexo_ejercicios_xslt:condiciones-multiples-peso-ligero-y-edificio-a}}
Se pide ahora generar un fichero HTML con una tabla pero en la que solo aparezcan los productos cuyo edificio sea el A \sphinxstylestrong{y además} pesen menos de 7kg.

\begin{sphinxVerbatim}[commandchars=\\\{\}]
\PYG{n+nt}{\PYGZlt{}xsl:stylesheet}
\PYG{n+na}{xmlns:xsl=}\PYG{l+s}{\PYGZdq{}http://www.w3.org/1999/XSL/Transform\PYGZdq{}}\PYG{n+nt}{\PYGZgt{}}
\PYG{n+nt}{\PYGZlt{}xsl:template} \PYG{n+na}{match=}\PYG{l+s}{\PYGZdq{}/\PYGZdq{}}\PYG{n+nt}{\PYGZgt{}}
    \PYG{n+nt}{\PYGZlt{}html}\PYG{n+nt}{\PYGZgt{}}
        \PYG{n+nt}{\PYGZlt{}head}\PYG{n+nt}{\PYGZgt{}}\PYG{n+nt}{\PYGZlt{}title}\PYG{n+nt}{\PYGZgt{}}Resultados\PYG{n+nt}{\PYGZlt{}/title\PYGZgt{}}\PYG{n+nt}{\PYGZlt{}/head\PYGZgt{}}
        \PYG{n+nt}{\PYGZlt{}body}\PYG{n+nt}{\PYGZgt{}}
            \PYG{n+nt}{\PYGZlt{}xsl:for\PYGZhy{}each} \PYG{n+na}{select=}\PYG{l+s}{\PYGZdq{}inventario/producto\PYGZdq{}}\PYG{n+nt}{\PYGZgt{}}
                \PYG{n+nt}{\PYGZlt{}xsl:if} \PYG{n+na}{test=}\PYG{l+s}{\PYGZdq{}lugar/@edificio = \PYGZsq{}A\PYGZsq{}\PYGZdq{}}\PYG{n+nt}{\PYGZgt{}}
                    \PYG{n+nt}{\PYGZlt{}xsl:if} \PYG{n+na}{test=}\PYG{l+s}{\PYGZdq{}peso/@unidad = \PYGZsq{}g\PYGZsq{}\PYGZdq{}}\PYG{n+nt}{\PYGZgt{}}
                        \PYG{n+nt}{\PYGZlt{}xsl:if} \PYG{n+na}{test=}\PYG{l+s}{\PYGZdq{}peso \PYGZam{}lt; 7000\PYGZdq{}}\PYG{n+nt}{\PYGZgt{}}
                            \PYG{n+nt}{\PYGZlt{}tr}\PYG{n+nt}{\PYGZgt{}}
                                \PYG{n+nt}{\PYGZlt{}td}\PYG{n+nt}{\PYGZgt{}}
                                    \PYG{n+nt}{\PYGZlt{}xsl:value\PYGZhy{}of}
                                    \PYG{n+na}{select=}\PYG{l+s}{\PYGZdq{}nombre\PYGZdq{}}\PYG{n+nt}{/\PYGZgt{}}
                                \PYG{n+nt}{\PYGZlt{}/td\PYGZgt{}}
                            \PYG{n+nt}{\PYGZlt{}/tr\PYGZgt{}}
                        \PYG{n+nt}{\PYGZlt{}/xsl:if\PYGZgt{}}
                    \PYG{n+nt}{\PYGZlt{}/xsl:if\PYGZgt{}}
                    \PYG{n+nt}{\PYGZlt{}xsl:if} \PYG{n+na}{test=}\PYG{l+s}{\PYGZdq{}peso/@unidad = \PYGZsq{}Kg\PYGZsq{}\PYGZdq{}}\PYG{n+nt}{\PYGZgt{}}
                        \PYG{n+nt}{\PYGZlt{}xsl:if} \PYG{n+na}{test=}\PYG{l+s}{\PYGZdq{}peso \PYGZam{}lt; 7\PYGZdq{}}\PYG{n+nt}{\PYGZgt{}}
                            \PYG{n+nt}{\PYGZlt{}tr}\PYG{n+nt}{\PYGZgt{}}
                                \PYG{n+nt}{\PYGZlt{}td}\PYG{n+nt}{\PYGZgt{}}
                                    \PYG{n+nt}{\PYGZlt{}xsl:value\PYGZhy{}of}
                                    \PYG{n+na}{select=}\PYG{l+s}{\PYGZdq{}nombre\PYGZdq{}}\PYG{n+nt}{/\PYGZgt{}}
                                \PYG{n+nt}{\PYGZlt{}/td\PYGZgt{}}
                            \PYG{n+nt}{\PYGZlt{}/tr\PYGZgt{}}
                        \PYG{n+nt}{\PYGZlt{}/xsl:if\PYGZgt{}}
                    \PYG{n+nt}{\PYGZlt{}/xsl:if\PYGZgt{}}
                \PYG{n+nt}{\PYGZlt{}/xsl:if\PYGZgt{}}
            \PYG{n+nt}{\PYGZlt{}/xsl:for\PYGZhy{}each\PYGZgt{}}
        \PYG{n+nt}{\PYGZlt{}/body\PYGZgt{}}
    \PYG{n+nt}{\PYGZlt{}/html\PYGZgt{}}
\PYG{n+nt}{\PYGZlt{}/xsl:template\PYGZgt{}}
\PYG{n+nt}{\PYGZlt{}/xsl:stylesheet\PYGZgt{}}
\end{sphinxVerbatim}


\section{Ejercicio de examen XSLT}
\label{\detokenize{ejercicios/xslt/anexo_ejercicios_xslt:ejercicio-de-examen-xslt}}
Dado el siguiente fichero XML

\begin{sphinxVerbatim}[commandchars=\\\{\}]
\PYG{n+nt}{\PYGZlt{}catalogo}\PYG{n+nt}{\PYGZgt{}}
  \PYG{n+nt}{\PYGZlt{}libro} \PYG{n+na}{isbn=}\PYG{l+s}{\PYGZdq{}i1\PYGZdq{}}\PYG{n+nt}{\PYGZgt{}}
    \PYG{n+nt}{\PYGZlt{}titulo}\PYG{n+nt}{\PYGZgt{}}Don Quijote\PYG{n+nt}{\PYGZlt{}/titulo\PYGZgt{}}
    \PYG{n+nt}{\PYGZlt{}autores}\PYG{n+nt}{\PYGZgt{}}
      \PYG{n+nt}{\PYGZlt{}autor} \PYG{n+na}{nacimiento=}\PYG{l+s}{\PYGZdq{}1547\PYGZdq{}}\PYG{n+nt}{\PYGZgt{}}Cervantes\PYG{n+nt}{\PYGZlt{}/autor\PYGZgt{}}
    \PYG{n+nt}{\PYGZlt{}/autores\PYGZgt{}}
  \PYG{n+nt}{\PYGZlt{}/libro\PYGZgt{}}
  \PYG{n+nt}{\PYGZlt{}libro} \PYG{n+na}{isbn=}\PYG{l+s}{\PYGZdq{}i2\PYGZdq{}}\PYG{n+nt}{\PYGZgt{}}
    \PYG{n+nt}{\PYGZlt{}titulo}\PYG{n+nt}{\PYGZgt{}}Antologia\PYG{n+nt}{\PYGZlt{}/titulo\PYGZgt{}}
    \PYG{n+nt}{\PYGZlt{}autores}\PYG{n+nt}{\PYGZgt{}}
        \PYG{n+nt}{\PYGZlt{}autor} \PYG{n+na}{nacimiento=}\PYG{l+s}{\PYGZdq{}1898\PYGZdq{}}\PYG{n+nt}{\PYGZgt{}}Lorca\PYG{n+nt}{\PYGZlt{}/autor\PYGZgt{}}
        \PYG{n+nt}{\PYGZlt{}autor} \PYG{n+na}{nacimiento=}\PYG{l+s}{\PYGZdq{}1910\PYGZdq{}}\PYG{n+nt}{\PYGZgt{}}Miguel Hernandez\PYG{n+nt}{\PYGZlt{}/autor\PYGZgt{}}
    \PYG{n+nt}{\PYGZlt{}/autores\PYGZgt{}}
  \PYG{n+nt}{\PYGZlt{}/libro\PYGZgt{}}
\PYG{n+nt}{\PYGZlt{}/catalogo\PYGZgt{}}
\end{sphinxVerbatim}

Conseguir lo siguiente:
\begin{enumerate}
\item {} 
Mostrar en un HTML con lista numerada los títulos de los libros con algún autor nacido despues de 1900.

\end{enumerate}

\begin{sphinxVerbatim}[commandchars=\\\{\}]
\PYG{n+nt}{\PYGZlt{}xsl:stylesheet} \PYG{n+na}{xmlns:xsl=}\PYG{l+s}{\PYGZdq{}http://www.w3.org/1999/XSL/Transform\PYGZdq{}}\PYG{n+nt}{\PYGZgt{}}
  \PYG{n+nt}{\PYGZlt{}xsl:template} \PYG{n+na}{match=}\PYG{l+s}{\PYGZdq{}/\PYGZdq{}}\PYG{n+nt}{\PYGZgt{}}
    \PYG{n+nt}{\PYGZlt{}html}\PYG{n+nt}{\PYGZgt{}}
      \PYG{n+nt}{\PYGZlt{}head}\PYG{n+nt}{\PYGZgt{}}
        \PYG{n+nt}{\PYGZlt{}title}\PYG{n+nt}{\PYGZgt{}}Resultado\PYG{n+nt}{\PYGZlt{}/title\PYGZgt{}}
      \PYG{n+nt}{\PYGZlt{}/head\PYGZgt{}}
      \PYG{n+nt}{\PYGZlt{}body}\PYG{n+nt}{\PYGZgt{}}
        \PYG{n+nt}{\PYGZlt{}ol}\PYG{n+nt}{\PYGZgt{}}
          \PYG{c}{\PYGZlt{}!\PYGZhy{}\PYGZhy{}}\PYG{c}{Recorremos los autores}\PYG{c}{\PYGZhy{}\PYGZhy{}\PYGZgt{}}
          \PYG{n+nt}{\PYGZlt{}xsl:for\PYGZhy{}each} \PYG{n+na}{select=}\PYG{l+s}{\PYGZdq{}catalogo/libro/autores/autor\PYGZdq{}}\PYG{n+nt}{\PYGZgt{}}
            \PYG{c}{\PYGZlt{}!\PYGZhy{}\PYGZhy{}}\PYG{c}{Y si nacieron despues de 1900...\PYGZdq{}}\PYG{c}{\PYGZhy{}\PYGZhy{}\PYGZgt{}}
            \PYG{n+nt}{\PYGZlt{}xsl:if} \PYG{n+na}{test=}\PYG{l+s}{\PYGZdq{}@nacimiento \PYGZgt{} 1900\PYGZdq{}}\PYG{n+nt}{\PYGZgt{}}
              \PYG{n+nt}{\PYGZlt{}li}\PYG{n+nt}{\PYGZgt{}}
                \PYG{c}{\PYGZlt{}!\PYGZhy{}\PYGZhy{}}\PYG{c}{Entonces \PYGZdq{}retrocedemos\PYGZdq{}}
\PYG{c}{                para extraer el titulo}\PYG{c}{\PYGZhy{}\PYGZhy{}\PYGZgt{}}
                \PYG{n+nt}{\PYGZlt{}xsl:value\PYGZhy{}of} \PYG{n+na}{select=}\PYG{l+s}{\PYGZdq{}../../titulo\PYGZdq{}}\PYG{n+nt}{/\PYGZgt{}}
              \PYG{n+nt}{\PYGZlt{}/li\PYGZgt{}}
            \PYG{n+nt}{\PYGZlt{}/xsl:if\PYGZgt{}}
          \PYG{n+nt}{\PYGZlt{}/xsl:for\PYGZhy{}each\PYGZgt{}}
        \PYG{n+nt}{\PYGZlt{}/ol\PYGZgt{}}
      \PYG{n+nt}{\PYGZlt{}/body\PYGZgt{}}
    \PYG{n+nt}{\PYGZlt{}/html\PYGZgt{}}
  \PYG{n+nt}{\PYGZlt{}/xsl:template\PYGZgt{}}
\PYG{n+nt}{\PYGZlt{}/xsl:stylesheet\PYGZgt{}}
\end{sphinxVerbatim}
\begin{enumerate}
\setcounter{enumi}{1}
\item {} 
Mostrar en un HTML la lista de los autores ordenada por orden alfabético inverso.

\end{enumerate}

\begin{sphinxVerbatim}[commandchars=\\\{\}]
\PYG{n+nt}{\PYGZlt{}xsl:stylesheet} \PYG{n+na}{xmlns:xsl=}\PYG{l+s}{\PYGZdq{}http://www.w3.org/1999/XSL/Transform\PYGZdq{}}\PYG{n+nt}{\PYGZgt{}}
  \PYG{n+nt}{\PYGZlt{}xsl:template} \PYG{n+na}{match=}\PYG{l+s}{\PYGZdq{}/\PYGZdq{}}\PYG{n+nt}{\PYGZgt{}}
    \PYG{n+nt}{\PYGZlt{}html}\PYG{n+nt}{\PYGZgt{}}
      \PYG{n+nt}{\PYGZlt{}head}\PYG{n+nt}{\PYGZgt{}}
        \PYG{n+nt}{\PYGZlt{}title}\PYG{n+nt}{\PYGZgt{}}Resultado\PYG{n+nt}{\PYGZlt{}/title\PYGZgt{}}
      \PYG{n+nt}{\PYGZlt{}/head\PYGZgt{}}
      \PYG{n+nt}{\PYGZlt{}body}\PYG{n+nt}{\PYGZgt{}}
        \PYG{n+nt}{\PYGZlt{}ol}\PYG{n+nt}{\PYGZgt{}}
          \PYG{n+nt}{\PYGZlt{}xsl:for\PYGZhy{}each} \PYG{n+na}{select=}\PYG{l+s}{\PYGZdq{}catalogo/libro/autores/autor\PYGZdq{}}\PYG{n+nt}{\PYGZgt{}}
            \PYG{n+nt}{\PYGZlt{}xsl:sort} \PYG{n+na}{select=}\PYG{l+s}{\PYGZdq{}.\PYGZdq{}} \PYG{n+na}{order=}\PYG{l+s}{\PYGZdq{}descending\PYGZdq{}}\PYG{n+nt}{/\PYGZgt{}}
            \PYG{n+nt}{\PYGZlt{}li}\PYG{n+nt}{\PYGZgt{}}
              \PYG{n+nt}{\PYGZlt{}xsl:value\PYGZhy{}of} \PYG{n+na}{select=}\PYG{l+s}{\PYGZdq{}.\PYGZdq{}}\PYG{n+nt}{/\PYGZgt{}}
            \PYG{n+nt}{\PYGZlt{}/li\PYGZgt{}}
          \PYG{n+nt}{\PYGZlt{}/xsl:for\PYGZhy{}each\PYGZgt{}}
        \PYG{n+nt}{\PYGZlt{}/ol\PYGZgt{}}
      \PYG{n+nt}{\PYGZlt{}/body\PYGZgt{}}
    \PYG{n+nt}{\PYGZlt{}/html\PYGZgt{}}
  \PYG{n+nt}{\PYGZlt{}/xsl:template\PYGZgt{}}
\PYG{n+nt}{\PYGZlt{}/xsl:stylesheet\PYGZgt{}}
\end{sphinxVerbatim}
\begin{enumerate}
\setcounter{enumi}{2}
\item {} 
Mostrar el nombre de los autores nacidos despues del año 1700.

\end{enumerate}

\begin{sphinxVerbatim}[commandchars=\\\{\}]
\PYG{n+nt}{\PYGZlt{}xsl:stylesheet}
\PYG{n+na}{xmlns:xsl=}\PYG{l+s}{\PYGZdq{}http://www.w3.org/1999/XSL/Transform\PYGZdq{}}\PYG{n+nt}{\PYGZgt{}}
\PYG{n+nt}{\PYGZlt{}xsl:template} \PYG{n+na}{match=}\PYG{l+s}{\PYGZdq{}/\PYGZdq{}}\PYG{n+nt}{\PYGZgt{}}
    \PYG{n+nt}{\PYGZlt{}html}\PYG{n+nt}{\PYGZgt{}}
        \PYG{n+nt}{\PYGZlt{}head}\PYG{n+nt}{\PYGZgt{}}\PYG{n+nt}{\PYGZlt{}title}\PYG{n+nt}{\PYGZgt{}}Resultado\PYG{n+nt}{\PYGZlt{}/title\PYGZgt{}}\PYG{n+nt}{\PYGZlt{}/head\PYGZgt{}}
        \PYG{n+nt}{\PYGZlt{}body}\PYG{n+nt}{\PYGZgt{}}
            \PYG{n+nt}{\PYGZlt{}table} \PYG{n+na}{border=}\PYG{l+s}{\PYGZsq{}1\PYGZsq{}}\PYG{n+nt}{\PYGZgt{}}
                \PYG{n+nt}{\PYGZlt{}tr}\PYG{n+nt}{\PYGZgt{}}
                    \PYG{n+nt}{\PYGZlt{}td}\PYG{n+nt}{\PYGZgt{}}Nombre\PYG{n+nt}{\PYGZlt{}/td\PYGZgt{}}
                    \PYG{n+nt}{\PYGZlt{}td}\PYG{n+nt}{\PYGZgt{}}Año nacimiento\PYG{n+nt}{\PYGZlt{}/td\PYGZgt{}}
                \PYG{n+nt}{\PYGZlt{}/tr\PYGZgt{}}
                \PYG{n+nt}{\PYGZlt{}xsl:for\PYGZhy{}each}
                \PYG{n+na}{select=}\PYG{l+s}{\PYGZdq{}catalogo/libro/autores/autor\PYGZdq{}}\PYG{n+nt}{\PYGZgt{}}
                \PYG{n+nt}{\PYGZlt{}xsl:if} \PYG{n+na}{test=}\PYG{l+s}{\PYGZdq{}@nacimiento \PYGZam{}gt; 1700\PYGZdq{}}\PYG{n+nt}{\PYGZgt{}}
                    \PYG{n+nt}{\PYGZlt{}td}\PYG{n+nt}{\PYGZgt{}}
                        \PYG{n+nt}{\PYGZlt{}xsl:value\PYGZhy{}of} \PYG{n+na}{select=}\PYG{l+s}{\PYGZdq{}.\PYGZdq{}}\PYG{n+nt}{/\PYGZgt{}}
                    \PYG{n+nt}{\PYGZlt{}/td\PYGZgt{}}
                    \PYG{n+nt}{\PYGZlt{}td}\PYG{n+nt}{\PYGZgt{}}
                        \PYG{n+nt}{\PYGZlt{}xsl:value\PYGZhy{}of} \PYG{n+na}{select=}\PYG{l+s}{\PYGZdq{}@nacimiento\PYGZdq{}}\PYG{n+nt}{/\PYGZgt{}}
                    \PYG{n+nt}{\PYGZlt{}/td\PYGZgt{}}
                \PYG{n+nt}{\PYGZlt{}/xsl:if\PYGZgt{}}
                \PYG{n+nt}{\PYGZlt{}/xsl:for\PYGZhy{}each\PYGZgt{}}
            \PYG{n+nt}{\PYGZlt{}/table\PYGZgt{}}
        \PYG{n+nt}{\PYGZlt{}/body\PYGZgt{}}
    \PYG{n+nt}{\PYGZlt{}/html\PYGZgt{}}
\PYG{n+nt}{\PYGZlt{}/xsl:template\PYGZgt{}}
\PYG{n+nt}{\PYGZlt{}/xsl:stylesheet\PYGZgt{}}
\end{sphinxVerbatim}


\section{Transformación de un XML bancario}
\label{\detokenize{ejercicios/xslt/anexo_ejercicios_xslt:transformacion-de-un-xml-bancario}}
Una empresa utiliza el siguiente XML para intercambiar información entre bases de datos de distintos proveedores. Sin embargo han comprado un nuevo sistema que necesita que la información tenga una estructura siguiente. Los dos listados que se ven a continuación ilustran la estructura original y la nueva estructura que deben tener los datos. Crear el XSLT que permita convertir la información original en un formato que pueda entender el nuevo sistema.

\begin{sphinxVerbatim}[commandchars=\\\{\}]
\PYG{c}{\PYGZlt{}!\PYGZhy{}\PYGZhy{}}\PYG{c}{Estructura original de la información}\PYG{c}{\PYGZhy{}\PYGZhy{}\PYGZgt{}}
\PYG{n+nt}{\PYGZlt{}listado}\PYG{n+nt}{\PYGZgt{}}
    \PYG{n+nt}{\PYGZlt{}cuenta}\PYG{n+nt}{\PYGZgt{}}
        \PYG{n+nt}{\PYGZlt{}titular} \PYG{n+na}{dni=}\PYG{l+s}{\PYGZdq{}5671001D\PYGZdq{}}\PYG{n+nt}{\PYGZgt{}}Ramon Perez\PYG{n+nt}{\PYGZlt{}/titular\PYGZgt{}}
        \PYG{n+nt}{\PYGZlt{}saldoactual} \PYG{n+na}{moneda=}\PYG{l+s}{\PYGZdq{}euros\PYGZdq{}}\PYG{n+nt}{\PYGZgt{}}12000\PYG{n+nt}{\PYGZlt{}/saldoactual\PYGZgt{}}
        \PYG{n+nt}{\PYGZlt{}fechacreacion}\PYG{n+nt}{\PYGZgt{}}13\PYGZhy{}abril\PYGZhy{}2012\PYG{n+nt}{\PYGZlt{}/fechacreacion\PYGZgt{}}
    \PYG{n+nt}{\PYGZlt{}/cuenta\PYGZgt{}}
    \PYG{n+nt}{\PYGZlt{}fondo}\PYG{n+nt}{\PYGZgt{}}
        \PYG{n+nt}{\PYGZlt{}cuentaasociada}\PYG{n+nt}{\PYGZgt{}}20\PYGZhy{}A\PYG{n+nt}{\PYGZlt{}/cuentaasociada\PYGZgt{}}
        \PYG{n+nt}{\PYGZlt{}datos}\PYG{n+nt}{\PYGZgt{}}
            \PYG{n+nt}{\PYGZlt{}cantidaddepositada}\PYG{n+nt}{\PYGZgt{}}20000\PYG{n+nt}{\PYGZlt{}/cantidaddepositada\PYGZgt{}}
            \PYG{n+nt}{\PYGZlt{}moneda}\PYG{n+nt}{\PYGZgt{}}Euros\PYG{n+nt}{\PYGZlt{}/moneda\PYGZgt{}}
        \PYG{n+nt}{\PYGZlt{}/datos\PYGZgt{}}
    \PYG{n+nt}{\PYGZlt{}/fondo\PYGZgt{}}
    \PYG{n+nt}{\PYGZlt{}fondo}\PYG{n+nt}{\PYGZgt{}}
        \PYG{n+nt}{\PYGZlt{}cuentaasociada}\PYG{n+nt}{\PYGZgt{}}21\PYGZhy{}DX\PYG{n+nt}{\PYGZlt{}/cuentaasociada\PYGZgt{}}
        \PYG{n+nt}{\PYGZlt{}datos}\PYG{n+nt}{\PYGZgt{}}
            \PYG{n+nt}{\PYGZlt{}cantidaddepositada}\PYG{n+nt}{\PYGZgt{}}4800\PYG{n+nt}{\PYGZlt{}/cantidaddepositada\PYGZgt{}}
            \PYG{n+nt}{\PYGZlt{}moneda}\PYG{n+nt}{\PYGZgt{}}Dolares\PYG{n+nt}{\PYGZlt{}/moneda\PYGZgt{}}
        \PYG{n+nt}{\PYGZlt{}/datos\PYGZgt{}}
    \PYG{n+nt}{\PYGZlt{}/fondo\PYGZgt{}}
    \PYG{n+nt}{\PYGZlt{}cuenta}\PYG{n+nt}{\PYGZgt{}}
        \PYG{n+nt}{\PYGZlt{}titular} \PYG{n+na}{dni=}\PYG{l+s}{\PYGZdq{}39812341C\PYGZdq{}}\PYG{n+nt}{\PYGZgt{}}Carmen Diaz\PYG{n+nt}{\PYGZlt{}/titular\PYGZgt{}}
        \PYG{n+nt}{\PYGZlt{}saldoactual} \PYG{n+na}{moneda=}\PYG{l+s}{\PYGZdq{}euros\PYGZdq{}}\PYG{n+nt}{\PYGZgt{}}1900\PYG{n+nt}{\PYGZlt{}/saldoactual\PYGZgt{}}
        \PYG{n+nt}{\PYGZlt{}fechacreacion}\PYG{n+nt}{\PYGZgt{}}15\PYGZhy{}febrero\PYGZhy{}2011\PYG{n+nt}{\PYGZlt{}/fechacreacion\PYGZgt{}}
    \PYG{n+nt}{\PYGZlt{}/cuenta\PYGZgt{}}
\PYG{n+nt}{\PYGZlt{}/listado\PYGZgt{}}
\end{sphinxVerbatim}

\begin{sphinxVerbatim}[commandchars=\\\{\}]
\PYG{c}{\PYGZlt{}!\PYGZhy{}\PYGZhy{}}\PYG{c}{Estructura final que debemos conseguir}\PYG{c}{\PYGZhy{}\PYGZhy{}\PYGZgt{}}
\PYG{n+nt}{\PYGZlt{}datos}\PYG{n+nt}{\PYGZgt{}}
    \PYG{n+nt}{\PYGZlt{}cuentas}\PYG{n+nt}{\PYGZgt{}}
        \PYG{n+nt}{\PYGZlt{}cuenta} \PYG{n+na}{dnititular=}\PYG{l+s}{\PYGZdq{}5671001D\PYGZdq{}}\PYG{n+nt}{\PYGZgt{}}
            \PYG{n+nt}{\PYGZlt{}creacion}\PYG{n+nt}{\PYGZgt{}}13\PYGZhy{}abril\PYGZhy{}2012\PYG{n+nt}{\PYGZlt{}/creacion\PYGZgt{}}
            \PYG{n+nt}{\PYGZlt{}titular}\PYG{n+nt}{\PYGZgt{}}Ramon Perez\PYG{n+nt}{\PYGZlt{}/titular\PYGZgt{}}
            \PYG{n+nt}{\PYGZlt{}saldoactual}\PYG{n+nt}{\PYGZgt{}}12000 euros\PYG{n+nt}{\PYGZlt{}/saldoactual\PYGZgt{}}

        \PYG{n+nt}{\PYGZlt{}/cuenta\PYGZgt{}}
        \PYG{n+nt}{\PYGZlt{}cuenta} \PYG{n+na}{dnititular=}\PYG{l+s}{\PYGZdq{}39812341C\PYGZdq{}}\PYG{n+nt}{\PYGZgt{}}
            \PYG{n+nt}{\PYGZlt{}creacion}\PYG{n+nt}{\PYGZgt{}}15\PYGZhy{}febrero\PYGZhy{}2011\PYG{n+nt}{\PYGZlt{}/creacion\PYGZgt{}}
            \PYG{n+nt}{\PYGZlt{}titular}\PYG{n+nt}{\PYGZgt{}}Carmen Diaz\PYG{n+nt}{\PYGZlt{}/titular\PYGZgt{}}
            \PYG{n+nt}{\PYGZlt{}saldoactual}\PYG{n+nt}{\PYGZgt{}}1900 euros\PYG{n+nt}{\PYGZlt{}/saldoactual\PYGZgt{}}

        \PYG{n+nt}{\PYGZlt{}/cuenta\PYGZgt{}}
    \PYG{n+nt}{\PYGZlt{}/cuentas\PYGZgt{}}
    \PYG{n+nt}{\PYGZlt{}fondos}\PYG{n+nt}{\PYGZgt{}}
        \PYG{n+nt}{\PYGZlt{}fondo} \PYG{n+na}{cuentaasociada=}\PYG{l+s}{\PYGZdq{}20\PYGZhy{}A\PYGZdq{}}\PYG{n+nt}{\PYGZgt{}}
            \PYG{n+nt}{\PYGZlt{}cantidaddepositada}\PYG{n+nt}{\PYGZgt{}}20000\PYG{n+nt}{\PYGZlt{}/cantidaddepositada\PYGZgt{}}
            \PYG{n+nt}{\PYGZlt{}moneda}\PYG{n+nt}{\PYGZgt{}}Euros\PYG{n+nt}{\PYGZlt{}/moneda\PYGZgt{}}
        \PYG{n+nt}{\PYGZlt{}/fondo\PYGZgt{}}
        \PYG{n+nt}{\PYGZlt{}fondo} \PYG{n+na}{cuentaasociada=}\PYG{l+s}{\PYGZdq{}21\PYGZhy{}DX\PYGZdq{}}\PYG{n+nt}{\PYGZgt{}}
            \PYG{n+nt}{\PYGZlt{}cantidaddepositada}\PYG{n+nt}{\PYGZgt{}}4800\PYG{n+nt}{\PYGZlt{}/cantidaddepositada\PYGZgt{}}
            \PYG{n+nt}{\PYGZlt{}moneda}\PYG{n+nt}{\PYGZgt{}}Dolares\PYG{n+nt}{\PYGZlt{}/moneda\PYGZgt{}}
        \PYG{n+nt}{\PYGZlt{}/fondo\PYGZgt{}}
    \PYG{n+nt}{\PYGZlt{}/fondos\PYGZgt{}}
\PYG{n+nt}{\PYGZlt{}/datos\PYGZgt{}}
\end{sphinxVerbatim}


\subsection{Análisis del problema}
\label{\detokenize{ejercicios/xslt/anexo_ejercicios_xslt:analisis-del-problema}}
Es necesario hacer varios cambios:
\begin{enumerate}
\item {} 
Se ha cambiado el nombre de elemento raíz de \sphinxcode{listado} a \sphinxcode{datos}.

\item {} 
Ahora todos los elementos \sphinxcode{cuenta} van dentro de un nuevo elemento \sphinxcode{cuentas} y todos los elementos \sphinxcode{fondo} van dentro de un nuevo elemento \sphinxcode{fondos}.

\item {} 
El \sphinxcode{dni} se ha movido del elemento \sphinxcode{titular} al elemento \sphinxcode{cuenta}.

\item {} 
La \sphinxcode{fechacreación} se ha movido y se ha renombrado a \sphinxcode{creacion}.

\item {} 
El elememento \sphinxcode{moneda} desaparece y su texto se ha puesto al lado de la cantidad que hay en \sphinxcode{saldoactual}.

\item {} 
En el elemento \sphinxcode{fondo} se ha quitado el elemento \sphinxcode{datos}.

\item {} 
El elemento \sphinxcode{cuentaasociada} ha pasado a ser un atributo.

\end{enumerate}


\subsection{Solución paso a paso}
\label{\detokenize{ejercicios/xslt/anexo_ejercicios_xslt:solucion-paso-a-paso}}
Empecemos por crear una hoja muy básica, que busque el elemento raíz y devuelva como salida el elemento \sphinxcode{datos} (que va a ser la nueva raíz)

\begin{sphinxVerbatim}[commandchars=\\\{\}]
\PYG{n+nt}{\PYGZlt{}xsl:stylesheet} \PYG{n+na}{xmlns:xsl=}\PYG{l+s}{\PYGZdq{}http://www.w3.org/1999/XSL/Transform\PYGZdq{}}\PYG{n+nt}{\PYGZgt{}}
\PYG{n+nt}{\PYGZlt{}xsl:template} \PYG{n+na}{match=}\PYG{l+s}{\PYGZdq{}/\PYGZdq{}}\PYG{n+nt}{\PYGZgt{}}
    \PYG{n+nt}{\PYGZlt{}datos}\PYG{n+nt}{\PYGZgt{}}
    \PYG{n+nt}{\PYGZlt{}/datos\PYGZgt{}}
\PYG{n+nt}{\PYGZlt{}/xsl:template\PYGZgt{}}
\PYG{n+nt}{\PYGZlt{}/xsl:stylesheet\PYGZgt{}}
\end{sphinxVerbatim}

Si probamos dicho XSLT aplicándolo al XML original obtendremos esto:

\begin{sphinxVerbatim}[commandchars=\\\{\}]
\PYG{n+nt}{\PYGZlt{}datos}\PYG{n+nt}{/\PYGZgt{}}
\end{sphinxVerbatim}

No pasa nada porque se obtenga el elemento raíz vacío, el programa de transformación lo hace para ahorrar tiempo y bytes.

Una vez que hemos cambiado el elemento raíz tenemos que generar dos elementos más que agrupen los elementos \sphinxcode{cuenta} y los elementos \sphinxcode{fondo}. Para ello, basta con escribirlos como muestra la siguiente hoja de estilo.

\begin{sphinxVerbatim}[commandchars=\\\{\}]
\PYG{n+nt}{\PYGZlt{}xsl:stylesheet} \PYG{n+na}{xmlns:xsl=}\PYG{l+s}{\PYGZdq{}http://www.w3.org/1999/XSL/Transform\PYGZdq{}}\PYG{n+nt}{\PYGZgt{}}
\PYG{n+nt}{\PYGZlt{}xsl:template} \PYG{n+na}{match=}\PYG{l+s}{\PYGZdq{}/\PYGZdq{}}\PYG{n+nt}{\PYGZgt{}}
    \PYG{n+nt}{\PYGZlt{}datos}\PYG{n+nt}{\PYGZgt{}}
        \PYG{n+nt}{\PYGZlt{}cuentas}\PYG{n+nt}{\PYGZgt{}}\PYG{n+nt}{\PYGZlt{}/cuentas\PYGZgt{}}
        \PYG{n+nt}{\PYGZlt{}fondos}\PYG{n+nt}{\PYGZgt{}}\PYG{n+nt}{\PYGZlt{}/fondos\PYGZgt{}}
    \PYG{n+nt}{\PYGZlt{}/datos\PYGZgt{}}
\PYG{n+nt}{\PYGZlt{}/xsl:template\PYGZgt{}}
\PYG{n+nt}{\PYGZlt{}/xsl:stylesheet\PYGZgt{}}
\end{sphinxVerbatim}

Ahora tenemos que ir buscando todos los elementos \sphinxcode{cuenta} y meterlos dentro de \sphinxcode{cuentas}. Despues resolveremos el problema de los fondos. Para recorrer elementos necesitamos un bucle \sphinxcode{for-each}. Como la plantilla ya nos ha situado en la raíz necesitaremos que el bucle nos vaya dando cada uno de los elementos \sphinxcode{listado/cuenta}. Es decir, le pedimos al bucle que se meta en el elemento hijo \sphinxcode{listado} y nos vaya dando cada uno de los elementos \sphinxcode{cuenta} que hay dentro. Un posible bucle sería este:

\begin{sphinxVerbatim}[commandchars=\\\{\}]
\PYG{n+nt}{\PYGZlt{}xsl:stylesheet} \PYG{n+na}{xmlns:xsl=}\PYG{l+s}{\PYGZdq{}http://www.w3.org/1999/XSL/Transform\PYGZdq{}}\PYG{n+nt}{\PYGZgt{}}
\PYG{n+nt}{\PYGZlt{}xsl:template} \PYG{n+na}{match=}\PYG{l+s}{\PYGZdq{}/\PYGZdq{}}\PYG{n+nt}{\PYGZgt{}}
    \PYG{n+nt}{\PYGZlt{}datos}\PYG{n+nt}{\PYGZgt{}}
        \PYG{n+nt}{\PYGZlt{}cuentas}\PYG{n+nt}{\PYGZgt{}}
            \PYG{n+nt}{\PYGZlt{}xsl:for\PYGZhy{}each} \PYG{n+na}{select=}\PYG{l+s}{\PYGZdq{}listado/cuenta\PYGZdq{}}\PYG{n+nt}{\PYGZgt{}}
                \PYG{n+nt}{\PYGZlt{}cuenta}\PYG{n+nt}{\PYGZgt{}}\PYG{n+nt}{\PYGZlt{}/cuenta\PYGZgt{}}
            \PYG{n+nt}{\PYGZlt{}/xsl:for\PYGZhy{}each\PYGZgt{}}
        \PYG{n+nt}{\PYGZlt{}/cuentas\PYGZgt{}}
        \PYG{n+nt}{\PYGZlt{}fondos}\PYG{n+nt}{\PYGZgt{}}\PYG{n+nt}{\PYGZlt{}/fondos\PYGZgt{}}
    \PYG{n+nt}{\PYGZlt{}/datos\PYGZgt{}}
\PYG{n+nt}{\PYGZlt{}/xsl:template\PYGZgt{}}
\PYG{n+nt}{\PYGZlt{}/xsl:stylesheet\PYGZgt{}}
\end{sphinxVerbatim}

Que al pasárselo a nuestros datos nos da esto:

\begin{sphinxVerbatim}[commandchars=\\\{\}]
\PYG{n+nt}{\PYGZlt{}datos}\PYG{n+nt}{\PYGZgt{}}
  \PYG{n+nt}{\PYGZlt{}cuentas}\PYG{n+nt}{\PYGZgt{}}
    \PYG{n+nt}{\PYGZlt{}cuenta}\PYG{n+nt}{/\PYGZgt{}}
    \PYG{n+nt}{\PYGZlt{}cuenta}\PYG{n+nt}{/\PYGZgt{}}
  \PYG{n+nt}{\PYGZlt{}/cuentas\PYGZgt{}}
  \PYG{n+nt}{\PYGZlt{}fondos}\PYG{n+nt}{/\PYGZgt{}}
\PYG{n+nt}{\PYGZlt{}/datos\PYGZgt{}}
\end{sphinxVerbatim}

Como puede verse, la plantilla genera dos elementos \sphinxcode{cuenta}, uno por cada \sphinxcode{cuenta} que nos da el bucle. Obsérvese que podríamos haber hecho esto para tener un nombre de elemento distinto, \sphinxstylestrong{y este es el «truco» para poder cambiar de nombre un elemento} :

\begin{sphinxVerbatim}[commandchars=\\\{\}]
\PYG{n+nt}{\PYGZlt{}xsl:stylesheet} \PYG{n+na}{xmlns:xsl=}\PYG{l+s}{\PYGZdq{}http://www.w3.org/1999/XSL/Transform\PYGZdq{}}\PYG{n+nt}{\PYGZgt{}}
\PYG{n+nt}{\PYGZlt{}xsl:template} \PYG{n+na}{match=}\PYG{l+s}{\PYGZdq{}/\PYGZdq{}}\PYG{n+nt}{\PYGZgt{}}
    \PYG{n+nt}{\PYGZlt{}datos}\PYG{n+nt}{\PYGZgt{}}
        \PYG{n+nt}{\PYGZlt{}cuentas}\PYG{n+nt}{\PYGZgt{}}
            \PYG{n+nt}{\PYGZlt{}xsl:for\PYGZhy{}each} \PYG{n+na}{select=}\PYG{l+s}{\PYGZdq{}listado/cuenta\PYGZdq{}}\PYG{n+nt}{\PYGZgt{}}
                \PYG{n+nt}{\PYGZlt{}otroelemento}\PYG{n+nt}{\PYGZgt{}}\PYG{n+nt}{\PYGZlt{}/otroelemento\PYGZgt{}}
            \PYG{n+nt}{\PYGZlt{}/xsl:for\PYGZhy{}each\PYGZgt{}}
        \PYG{n+nt}{\PYGZlt{}/cuentas\PYGZgt{}}
        \PYG{n+nt}{\PYGZlt{}fondos}\PYG{n+nt}{\PYGZgt{}}\PYG{n+nt}{\PYGZlt{}/fondos\PYGZgt{}}
    \PYG{n+nt}{\PYGZlt{}/datos\PYGZgt{}}
\PYG{n+nt}{\PYGZlt{}/xsl:template\PYGZgt{}}
\PYG{n+nt}{\PYGZlt{}/xsl:stylesheet\PYGZgt{}}
\end{sphinxVerbatim}

Sigamos con el problema original: ya hemos creado un elemento \sphinxcode{cuentas} que lleva dentro un elemento \sphinxcode{cuenta} para cada una de las cuentas originales. Ahora en dicho elemento \sphinxcode{cuenta} vamos a meter dentro un atributo llamado \sphinxcode{dnititular} usando la etiqueta \sphinxcode{xsl:attribute} que debe ir \sphinxstylestrong{dentro del elemento al que le queramos poner el atributo y además al principio}. Si queremos varios atributos no pasa nada podemos ponerlos todos dentro del elemento pero recordando ponerlos al principio.

Así, el código siguiente nos fabrica el atributo.

\begin{sphinxVerbatim}[commandchars=\\\{\}]
\PYG{n+nt}{\PYGZlt{}xsl:stylesheet} \PYG{n+na}{xmlns:xsl=}\PYG{l+s}{\PYGZdq{}http://www.w3.org/1999/XSL/Transform\PYGZdq{}}\PYG{n+nt}{\PYGZgt{}}
\PYG{n+nt}{\PYGZlt{}xsl:template} \PYG{n+na}{match=}\PYG{l+s}{\PYGZdq{}/\PYGZdq{}}\PYG{n+nt}{\PYGZgt{}}
    \PYG{n+nt}{\PYGZlt{}datos}\PYG{n+nt}{\PYGZgt{}}
        \PYG{n+nt}{\PYGZlt{}cuentas}\PYG{n+nt}{\PYGZgt{}}
            \PYG{n+nt}{\PYGZlt{}xsl:for\PYGZhy{}each} \PYG{n+na}{select=}\PYG{l+s}{\PYGZdq{}listado/cuenta\PYGZdq{}}\PYG{n+nt}{\PYGZgt{}}
                \PYG{n+nt}{\PYGZlt{}cuenta}\PYG{n+nt}{\PYGZgt{}}
                    \PYG{c}{\PYGZlt{}!\PYGZhy{}\PYGZhy{}}\PYG{c}{Esto añade el atributo dnititular a cuenta}\PYG{c}{\PYGZhy{}\PYGZhy{}\PYGZgt{}}
                    \PYG{n+nt}{\PYGZlt{}xsl:attribute} \PYG{n+na}{name=}\PYG{l+s}{\PYGZdq{}dnititular\PYGZdq{}}\PYG{n+nt}{\PYGZgt{}}10\PYG{n+nt}{\PYGZlt{}/xsl:attribute\PYGZgt{}}
                \PYG{n+nt}{\PYGZlt{}/cuenta\PYGZgt{}}
            \PYG{n+nt}{\PYGZlt{}/xsl:for\PYGZhy{}each\PYGZgt{}}
        \PYG{n+nt}{\PYGZlt{}/cuentas\PYGZgt{}}
        \PYG{n+nt}{\PYGZlt{}fondos}\PYG{n+nt}{\PYGZgt{}}\PYG{n+nt}{\PYGZlt{}/fondos\PYGZgt{}}
    \PYG{n+nt}{\PYGZlt{}/datos\PYGZgt{}}
\PYG{n+nt}{\PYGZlt{}/xsl:template\PYGZgt{}}
\PYG{n+nt}{\PYGZlt{}/xsl:stylesheet\PYGZgt{}}
\end{sphinxVerbatim}

Pero hay un problema, todas las cuentas tienen el \sphinxcode{dnititular} a 10. Necesitamos la etiqueta \sphinxcode{value-of} que nos permite \sphinxstylestrong{extraer el contenido de un elemento o atributo}, en este caso queremos extrar el valor del atributo \sphinxcode{dni} que está dentro del elemento \sphinxcode{titular}. Esto se hace con \sphinxcode{titular/@dni} que significa «extraer el atributo dni que debe estar dentro del elemento titular».

Así, el código siguiente:

\begin{sphinxVerbatim}[commandchars=\\\{\}]
\PYG{n+nt}{\PYGZlt{}xsl:stylesheet} \PYG{n+na}{xmlns:xsl=}\PYG{l+s}{\PYGZdq{}http://www.w3.org/1999/XSL/Transform\PYGZdq{}}\PYG{n+nt}{\PYGZgt{}}
\PYG{n+nt}{\PYGZlt{}xsl:template} \PYG{n+na}{match=}\PYG{l+s}{\PYGZdq{}/\PYGZdq{}}\PYG{n+nt}{\PYGZgt{}}
    \PYG{n+nt}{\PYGZlt{}datos}\PYG{n+nt}{\PYGZgt{}}
        \PYG{n+nt}{\PYGZlt{}cuentas}\PYG{n+nt}{\PYGZgt{}}
            \PYG{n+nt}{\PYGZlt{}xsl:for\PYGZhy{}each} \PYG{n+na}{select=}\PYG{l+s}{\PYGZdq{}listado/cuenta\PYGZdq{}}\PYG{n+nt}{\PYGZgt{}}
                \PYG{n+nt}{\PYGZlt{}cuenta}\PYG{n+nt}{\PYGZgt{}}
                    \PYG{n+nt}{\PYGZlt{}xsl:attribute} \PYG{n+na}{name=}\PYG{l+s}{\PYGZdq{}dnititular\PYGZdq{}}\PYG{n+nt}{\PYGZgt{}}
                        \PYG{n+nt}{\PYGZlt{}xsl:value\PYGZhy{}of} \PYG{n+na}{select=}\PYG{l+s}{\PYGZdq{}titular/@dni\PYGZdq{}}\PYG{n+nt}{/\PYGZgt{}}
                    \PYG{n+nt}{\PYGZlt{}/xsl:attribute\PYGZgt{}}
                \PYG{n+nt}{\PYGZlt{}/cuenta\PYGZgt{}}
            \PYG{n+nt}{\PYGZlt{}/xsl:for\PYGZhy{}each\PYGZgt{}}
        \PYG{n+nt}{\PYGZlt{}/cuentas\PYGZgt{}}
        \PYG{n+nt}{\PYGZlt{}fondos}\PYG{n+nt}{\PYGZgt{}}\PYG{n+nt}{\PYGZlt{}/fondos\PYGZgt{}}
    \PYG{n+nt}{\PYGZlt{}/datos\PYGZgt{}}
\PYG{n+nt}{\PYGZlt{}/xsl:template\PYGZgt{}}
\PYG{n+nt}{\PYGZlt{}/xsl:stylesheet\PYGZgt{}}
\end{sphinxVerbatim}

Nos devuelve como resultado:

\begin{sphinxVerbatim}[commandchars=\\\{\}]
\PYG{n+nt}{\PYGZlt{}datos}\PYG{n+nt}{\PYGZgt{}}
  \PYG{n+nt}{\PYGZlt{}cuentas}\PYG{n+nt}{\PYGZgt{}}
    \PYG{n+nt}{\PYGZlt{}cuenta} \PYG{n+na}{dnititular=}\PYG{l+s}{\PYGZdq{}5671001D\PYGZdq{}}\PYG{n+nt}{/\PYGZgt{}}
    \PYG{n+nt}{\PYGZlt{}cuenta} \PYG{n+na}{dnititular=}\PYG{l+s}{\PYGZdq{}39812341C\PYGZdq{}}\PYG{n+nt}{/\PYGZgt{}}
  \PYG{n+nt}{\PYGZlt{}/cuentas\PYGZgt{}}
  \PYG{n+nt}{\PYGZlt{}fondos}\PYG{n+nt}{/\PYGZgt{}}
\PYG{n+nt}{\PYGZlt{}/datos\PYGZgt{}}
\end{sphinxVerbatim}

El paso siguiente va a ser crear el elemento \sphinxcode{titular} que también se llama \sphinxcode{titular} en el archivo original. Para ello lo metemos dentro de \sphinxcode{cuenta} y permitiendo que la creación del atributo \sphinxcode{dnititular} se quede al principio.

El código XSLT es este

\begin{sphinxVerbatim}[commandchars=\\\{\}]
\PYG{n+nt}{\PYGZlt{}xsl:stylesheet} \PYG{n+na}{xmlns:xsl=}\PYG{l+s}{\PYGZdq{}http://www.w3.org/1999/XSL/Transform\PYGZdq{}}\PYG{n+nt}{\PYGZgt{}}
\PYG{n+nt}{\PYGZlt{}xsl:template} \PYG{n+na}{match=}\PYG{l+s}{\PYGZdq{}/\PYGZdq{}}\PYG{n+nt}{\PYGZgt{}}
    \PYG{n+nt}{\PYGZlt{}datos}\PYG{n+nt}{\PYGZgt{}}
        \PYG{n+nt}{\PYGZlt{}cuentas}\PYG{n+nt}{\PYGZgt{}}
            \PYG{n+nt}{\PYGZlt{}xsl:for\PYGZhy{}each} \PYG{n+na}{select=}\PYG{l+s}{\PYGZdq{}listado/cuenta\PYGZdq{}}\PYG{n+nt}{\PYGZgt{}}
                \PYG{n+nt}{\PYGZlt{}cuenta}\PYG{n+nt}{\PYGZgt{}}
                    \PYG{n+nt}{\PYGZlt{}xsl:attribute} \PYG{n+na}{name=}\PYG{l+s}{\PYGZdq{}dnititular\PYGZdq{}}\PYG{n+nt}{\PYGZgt{}}
                        \PYG{n+nt}{\PYGZlt{}xsl:value\PYGZhy{}of} \PYG{n+na}{select=}\PYG{l+s}{\PYGZdq{}titular/@dni\PYGZdq{}}\PYG{n+nt}{/\PYGZgt{}}
                    \PYG{n+nt}{\PYGZlt{}/xsl:attribute\PYGZgt{}}
                    \PYG{c}{\PYGZlt{}!\PYGZhy{}\PYGZhy{}}\PYG{c}{Creamos el elemento titular y metemos}
\PYG{c}{                    dentro del valor original del titular}\PYG{c}{\PYGZhy{}\PYGZhy{}\PYGZgt{}}
                    \PYG{n+nt}{\PYGZlt{}titular}\PYG{n+nt}{\PYGZgt{}}
                        \PYG{n+nt}{\PYGZlt{}xsl:value\PYGZhy{}of} \PYG{n+na}{select=}\PYG{l+s}{\PYGZdq{}titular\PYGZdq{}}\PYG{n+nt}{/\PYGZgt{}}
                    \PYG{n+nt}{\PYGZlt{}/titular\PYGZgt{}}
                \PYG{n+nt}{\PYGZlt{}/cuenta\PYGZgt{}}
            \PYG{n+nt}{\PYGZlt{}/xsl:for\PYGZhy{}each\PYGZgt{}}
        \PYG{n+nt}{\PYGZlt{}/cuentas\PYGZgt{}}
        \PYG{n+nt}{\PYGZlt{}fondos}\PYG{n+nt}{\PYGZgt{}}\PYG{n+nt}{\PYGZlt{}/fondos\PYGZgt{}}
    \PYG{n+nt}{\PYGZlt{}/datos\PYGZgt{}}
\PYG{n+nt}{\PYGZlt{}/xsl:template\PYGZgt{}}
\PYG{n+nt}{\PYGZlt{}/xsl:stylesheet\PYGZgt{}}
\end{sphinxVerbatim}

Que genera el siguiente resultado:

\begin{sphinxVerbatim}[commandchars=\\\{\}]
\PYG{n+nt}{\PYGZlt{}datos}\PYG{n+nt}{\PYGZgt{}}
  \PYG{n+nt}{\PYGZlt{}cuentas}\PYG{n+nt}{\PYGZgt{}}
    \PYG{n+nt}{\PYGZlt{}cuenta} \PYG{n+na}{dnititular=}\PYG{l+s}{\PYGZdq{}5671001D\PYGZdq{}}\PYG{n+nt}{\PYGZgt{}}
      \PYG{n+nt}{\PYGZlt{}titular}\PYG{n+nt}{\PYGZgt{}}Ramon Perez\PYG{n+nt}{\PYGZlt{}/titular\PYGZgt{}}
    \PYG{n+nt}{\PYGZlt{}/cuenta\PYGZgt{}}
    \PYG{n+nt}{\PYGZlt{}cuenta} \PYG{n+na}{dnititular=}\PYG{l+s}{\PYGZdq{}39812341C\PYGZdq{}}\PYG{n+nt}{\PYGZgt{}}
      \PYG{n+nt}{\PYGZlt{}titular}\PYG{n+nt}{\PYGZgt{}}Carmen Diaz\PYG{n+nt}{\PYGZlt{}/titular\PYGZgt{}}
    \PYG{n+nt}{\PYGZlt{}/cuenta\PYGZgt{}}
  \PYG{n+nt}{\PYGZlt{}/cuentas\PYGZgt{}}
  \PYG{n+nt}{\PYGZlt{}fondos}\PYG{n+nt}{/\PYGZgt{}}
\PYG{n+nt}{\PYGZlt{}/datos\PYGZgt{}}
\end{sphinxVerbatim}

Ahora vamos a crear el elemento \sphinxcode{saldoactual}. Dentro de ese elemento debemos escribir el texto que ya tuviese el \sphinxcode{saldoactual} antiguo y escribiendo al lado el atributo \sphinxcode{moneda}.

El código necesario es este

\begin{sphinxVerbatim}[commandchars=\\\{\}]
\PYG{n+nt}{\PYGZlt{}xsl:stylesheet} \PYG{n+na}{xmlns:xsl=}\PYG{l+s}{\PYGZdq{}http://www.w3.org/1999/XSL/Transform\PYGZdq{}}\PYG{n+nt}{\PYGZgt{}}
\PYG{n+nt}{\PYGZlt{}xsl:template} \PYG{n+na}{match=}\PYG{l+s}{\PYGZdq{}/\PYGZdq{}}\PYG{n+nt}{\PYGZgt{}}
    \PYG{n+nt}{\PYGZlt{}datos}\PYG{n+nt}{\PYGZgt{}}
        \PYG{n+nt}{\PYGZlt{}cuentas}\PYG{n+nt}{\PYGZgt{}}
            \PYG{n+nt}{\PYGZlt{}xsl:for\PYGZhy{}each} \PYG{n+na}{select=}\PYG{l+s}{\PYGZdq{}listado/cuenta\PYGZdq{}}\PYG{n+nt}{\PYGZgt{}}
                \PYG{n+nt}{\PYGZlt{}cuenta}\PYG{n+nt}{\PYGZgt{}}
                    \PYG{n+nt}{\PYGZlt{}xsl:attribute} \PYG{n+na}{name=}\PYG{l+s}{\PYGZdq{}dnititular\PYGZdq{}}\PYG{n+nt}{\PYGZgt{}}
                        \PYG{n+nt}{\PYGZlt{}xsl:value\PYGZhy{}of} \PYG{n+na}{select=}\PYG{l+s}{\PYGZdq{}titular/@dni\PYGZdq{}}\PYG{n+nt}{/\PYGZgt{}}
                    \PYG{n+nt}{\PYGZlt{}/xsl:attribute\PYGZgt{}}
                    \PYG{c}{\PYGZlt{}!\PYGZhy{}\PYGZhy{}}\PYG{c}{Creamos el elemento titular y metemos}
\PYG{c}{                    dentro del valor original del titular}\PYG{c}{\PYGZhy{}\PYGZhy{}\PYGZgt{}}
                    \PYG{n+nt}{\PYGZlt{}titular}\PYG{n+nt}{\PYGZgt{}}
                        \PYG{n+nt}{\PYGZlt{}xsl:value\PYGZhy{}of} \PYG{n+na}{select=}\PYG{l+s}{\PYGZdq{}titular\PYGZdq{}}\PYG{n+nt}{/\PYGZgt{}}
                    \PYG{n+nt}{\PYGZlt{}/titular\PYGZgt{}}
                    \PYG{c}{\PYGZlt{}!\PYGZhy{}\PYGZhy{}}\PYG{c}{Creamos el saldo actual}\PYG{c}{\PYGZhy{}\PYGZhy{}\PYGZgt{}}
                    \PYG{n+nt}{\PYGZlt{}saldoactual}\PYG{n+nt}{\PYGZgt{}}
                        \PYG{c}{\PYGZlt{}!\PYGZhy{}\PYGZhy{}}\PYG{c}{Y metemos dentro la cantidad que tuviese}
\PYG{c}{                        el fichero original...}\PYG{c}{\PYGZhy{}\PYGZhy{}\PYGZgt{}}
                        \PYG{n+nt}{\PYGZlt{}xsl:value\PYGZhy{}of} \PYG{n+na}{select=}\PYG{l+s}{\PYGZdq{}saldoactual\PYGZdq{}}\PYG{n+nt}{/\PYGZgt{}}
                        \PYG{c}{\PYGZlt{}!\PYGZhy{}\PYGZhy{}}\PYG{c}{Y extraemos la moneda...}\PYG{c}{\PYGZhy{}\PYGZhy{}\PYGZgt{}}
                         \PYG{n+nt}{\PYGZlt{}xsl:value\PYGZhy{}of} \PYG{n+na}{select=}\PYG{l+s}{\PYGZdq{}saldoactual/@moneda\PYGZdq{}}\PYG{n+nt}{/\PYGZgt{}}
                    \PYG{n+nt}{\PYGZlt{}/saldoactual\PYGZgt{}}
                \PYG{n+nt}{\PYGZlt{}/cuenta\PYGZgt{}}
            \PYG{n+nt}{\PYGZlt{}/xsl:for\PYGZhy{}each\PYGZgt{}}
        \PYG{n+nt}{\PYGZlt{}/cuentas\PYGZgt{}}
        \PYG{n+nt}{\PYGZlt{}fondos}\PYG{n+nt}{\PYGZgt{}}\PYG{n+nt}{\PYGZlt{}/fondos\PYGZgt{}}
    \PYG{n+nt}{\PYGZlt{}/datos\PYGZgt{}}
\PYG{n+nt}{\PYGZlt{}/xsl:template\PYGZgt{}}
\PYG{n+nt}{\PYGZlt{}/xsl:stylesheet\PYGZgt{}}
\end{sphinxVerbatim}

Y el resultado es este:

\begin{sphinxVerbatim}[commandchars=\\\{\}]
\PYG{n+nt}{\PYGZlt{}datos}\PYG{n+nt}{\PYGZgt{}}
  \PYG{n+nt}{\PYGZlt{}cuentas}\PYG{n+nt}{\PYGZgt{}}
    \PYG{n+nt}{\PYGZlt{}cuenta} \PYG{n+na}{dnititular=}\PYG{l+s}{\PYGZdq{}5671001D\PYGZdq{}}\PYG{n+nt}{\PYGZgt{}}
      \PYG{n+nt}{\PYGZlt{}titular}\PYG{n+nt}{\PYGZgt{}}Ramon Perez\PYG{n+nt}{\PYGZlt{}/titular\PYGZgt{}}
      \PYG{n+nt}{\PYGZlt{}saldoactual}\PYG{n+nt}{\PYGZgt{}}12000euros\PYG{n+nt}{\PYGZlt{}/saldoactual\PYGZgt{}}
    \PYG{n+nt}{\PYGZlt{}/cuenta\PYGZgt{}}
    \PYG{n+nt}{\PYGZlt{}cuenta} \PYG{n+na}{dnititular=}\PYG{l+s}{\PYGZdq{}39812341C\PYGZdq{}}\PYG{n+nt}{\PYGZgt{}}
      \PYG{n+nt}{\PYGZlt{}titular}\PYG{n+nt}{\PYGZgt{}}Carmen Diaz\PYG{n+nt}{\PYGZlt{}/titular\PYGZgt{}}
      \PYG{n+nt}{\PYGZlt{}saldoactual}\PYG{n+nt}{\PYGZgt{}}1900euros\PYG{n+nt}{\PYGZlt{}/saldoactual\PYGZgt{}}
    \PYG{n+nt}{\PYGZlt{}/cuenta\PYGZgt{}}
  \PYG{n+nt}{\PYGZlt{}/cuentas\PYGZgt{}}
  \PYG{n+nt}{\PYGZlt{}fondos}\PYG{n+nt}{/\PYGZgt{}}
\PYG{n+nt}{\PYGZlt{}/datos\PYGZgt{}}
\end{sphinxVerbatim}

Por último, añadamos la fecha de creación. En el fichero original se llama \sphinxcode{fechacreación} y en el fichero final se llama \sphinxcode{creación} y además va como primer elemento. El XSLT sería este:

\begin{sphinxVerbatim}[commandchars=\\\{\}]
\PYG{n+nt}{\PYGZlt{}xsl:stylesheet} \PYG{n+na}{xmlns:xsl=}\PYG{l+s}{\PYGZdq{}http://www.w3.org/1999/XSL/Transform\PYGZdq{}}\PYG{n+nt}{\PYGZgt{}}
\PYG{n+nt}{\PYGZlt{}xsl:template} \PYG{n+na}{match=}\PYG{l+s}{\PYGZdq{}/\PYGZdq{}}\PYG{n+nt}{\PYGZgt{}}
    \PYG{n+nt}{\PYGZlt{}datos}\PYG{n+nt}{\PYGZgt{}}
        \PYG{n+nt}{\PYGZlt{}cuentas}\PYG{n+nt}{\PYGZgt{}}
            \PYG{n+nt}{\PYGZlt{}xsl:for\PYGZhy{}each} \PYG{n+na}{select=}\PYG{l+s}{\PYGZdq{}listado/cuenta\PYGZdq{}}\PYG{n+nt}{\PYGZgt{}}
                \PYG{n+nt}{\PYGZlt{}cuenta}\PYG{n+nt}{\PYGZgt{}}
                    \PYG{n+nt}{\PYGZlt{}xsl:attribute} \PYG{n+na}{name=}\PYG{l+s}{\PYGZdq{}dnititular\PYGZdq{}}\PYG{n+nt}{\PYGZgt{}}
                        \PYG{n+nt}{\PYGZlt{}xsl:value\PYGZhy{}of} \PYG{n+na}{select=}\PYG{l+s}{\PYGZdq{}titular/@dni\PYGZdq{}}\PYG{n+nt}{/\PYGZgt{}}
                    \PYG{n+nt}{\PYGZlt{}/xsl:attribute\PYGZgt{}}
                    \PYG{c}{\PYGZlt{}!\PYGZhy{}\PYGZhy{}}\PYG{c}{Creamos el elemento \PYGZdq{}creación\PYGZdq{} que en}
\PYG{c}{                    realidad contiene el texto del elemento}
\PYG{c}{                    \PYGZdq{}fechacreación\PYGZdq{} original}\PYG{c}{\PYGZhy{}\PYGZhy{}\PYGZgt{}}
                    \PYG{n+nt}{\PYGZlt{}creacion}\PYG{n+nt}{\PYGZgt{}}
                        \PYG{n+nt}{\PYGZlt{}xsl:value\PYGZhy{}of} \PYG{n+na}{select=}\PYG{l+s}{\PYGZdq{}fechacreacion\PYGZdq{}}\PYG{n+nt}{/\PYGZgt{}}
                    \PYG{n+nt}{\PYGZlt{}/creacion\PYGZgt{}}
                    \PYG{c}{\PYGZlt{}!\PYGZhy{}\PYGZhy{}}\PYG{c}{Creamos el elemento titular y metemos}
\PYG{c}{                    dentro del valor original del titular}\PYG{c}{\PYGZhy{}\PYGZhy{}\PYGZgt{}}
                    \PYG{n+nt}{\PYGZlt{}titular}\PYG{n+nt}{\PYGZgt{}}
                        \PYG{n+nt}{\PYGZlt{}xsl:value\PYGZhy{}of} \PYG{n+na}{select=}\PYG{l+s}{\PYGZdq{}titular\PYGZdq{}}\PYG{n+nt}{/\PYGZgt{}}
                    \PYG{n+nt}{\PYGZlt{}/titular\PYGZgt{}}
                    \PYG{c}{\PYGZlt{}!\PYGZhy{}\PYGZhy{}}\PYG{c}{Creamos el saldo actual}\PYG{c}{\PYGZhy{}\PYGZhy{}\PYGZgt{}}
                    \PYG{n+nt}{\PYGZlt{}saldoactual}\PYG{n+nt}{\PYGZgt{}}
                        \PYG{c}{\PYGZlt{}!\PYGZhy{}\PYGZhy{}}\PYG{c}{Y metemos dentro la cantidad que tuviese}
\PYG{c}{                        el fichero original...}\PYG{c}{\PYGZhy{}\PYGZhy{}\PYGZgt{}}
                        \PYG{n+nt}{\PYGZlt{}xsl:value\PYGZhy{}of} \PYG{n+na}{select=}\PYG{l+s}{\PYGZdq{}saldoactual\PYGZdq{}}\PYG{n+nt}{/\PYGZgt{}}
                        \PYG{c}{\PYGZlt{}!\PYGZhy{}\PYGZhy{}}\PYG{c}{Y extraemos la moneda...}\PYG{c}{\PYGZhy{}\PYGZhy{}\PYGZgt{}}
                         \PYG{n+nt}{\PYGZlt{}xsl:value\PYGZhy{}of} \PYG{n+na}{select=}\PYG{l+s}{\PYGZdq{}saldoactual/@moneda\PYGZdq{}}\PYG{n+nt}{/\PYGZgt{}}
                    \PYG{n+nt}{\PYGZlt{}/saldoactual\PYGZgt{}}
                \PYG{n+nt}{\PYGZlt{}/cuenta\PYGZgt{}}
            \PYG{n+nt}{\PYGZlt{}/xsl:for\PYGZhy{}each\PYGZgt{}}
        \PYG{n+nt}{\PYGZlt{}/cuentas\PYGZgt{}}
        \PYG{n+nt}{\PYGZlt{}fondos}\PYG{n+nt}{\PYGZgt{}}\PYG{n+nt}{\PYGZlt{}/fondos\PYGZgt{}}
    \PYG{n+nt}{\PYGZlt{}/datos\PYGZgt{}}
\PYG{n+nt}{\PYGZlt{}/xsl:template\PYGZgt{}}
\PYG{n+nt}{\PYGZlt{}/xsl:stylesheet\PYGZgt{}}
\end{sphinxVerbatim}

Que genera el resultado siguiente:

\begin{sphinxVerbatim}[commandchars=\\\{\}]
\PYG{n+nt}{\PYGZlt{}datos}\PYG{n+nt}{\PYGZgt{}}
  \PYG{n+nt}{\PYGZlt{}cuentas}\PYG{n+nt}{\PYGZgt{}}
    \PYG{n+nt}{\PYGZlt{}cuenta} \PYG{n+na}{dnititular=}\PYG{l+s}{\PYGZdq{}5671001D\PYGZdq{}}\PYG{n+nt}{\PYGZgt{}}
      \PYG{n+nt}{\PYGZlt{}creacion}\PYG{n+nt}{\PYGZgt{}}13\PYGZhy{}abril\PYGZhy{}2012\PYG{n+nt}{\PYGZlt{}/creacion\PYGZgt{}}
      \PYG{n+nt}{\PYGZlt{}titular}\PYG{n+nt}{\PYGZgt{}}Ramon Perez\PYG{n+nt}{\PYGZlt{}/titular\PYGZgt{}}
      \PYG{n+nt}{\PYGZlt{}saldoactual}\PYG{n+nt}{\PYGZgt{}}12000euros\PYG{n+nt}{\PYGZlt{}/saldoactual\PYGZgt{}}
    \PYG{n+nt}{\PYGZlt{}/cuenta\PYGZgt{}}
    \PYG{n+nt}{\PYGZlt{}cuenta} \PYG{n+na}{dnititular=}\PYG{l+s}{\PYGZdq{}39812341C\PYGZdq{}}\PYG{n+nt}{\PYGZgt{}}
      \PYG{n+nt}{\PYGZlt{}creacion}\PYG{n+nt}{\PYGZgt{}}15\PYGZhy{}febrero\PYGZhy{}2011\PYG{n+nt}{\PYGZlt{}/creacion\PYGZgt{}}
      \PYG{n+nt}{\PYGZlt{}titular}\PYG{n+nt}{\PYGZgt{}}Carmen Diaz\PYG{n+nt}{\PYGZlt{}/titular\PYGZgt{}}
      \PYG{n+nt}{\PYGZlt{}saldoactual}\PYG{n+nt}{\PYGZgt{}}1900euros\PYG{n+nt}{\PYGZlt{}/saldoactual\PYGZgt{}}
    \PYG{n+nt}{\PYGZlt{}/cuenta\PYGZgt{}}
  \PYG{n+nt}{\PYGZlt{}/cuentas\PYGZgt{}}
  \PYG{n+nt}{\PYGZlt{}fondos}\PYG{n+nt}{/\PYGZgt{}}
\PYG{n+nt}{\PYGZlt{}/datos\PYGZgt{}}
\end{sphinxVerbatim}

Con esto, la parte de las \sphinxcode{cuentas} ya está hecha. Ahora queda lo siguiente
\begin{enumerate}
\item {} 
Hacer un elemento \sphinxcode{fondos} con un bucle que vaya generando elementos \sphinxcode{fondo}.

\item {} 
Poner en el \sphinxcode{fondo} el atributo \sphinxcode{cuentaasociada}.

\item {} 
Crear el elemento \sphinxcode{cantidaddepositada}.

\item {} 
Crear el elemento moneda.

\end{enumerate}

Vamos con el paso 1 \sphinxstyleemphasis{«crear el elemento fondos»}

\begin{sphinxVerbatim}[commandchars=\\\{\}]
\PYG{n+nt}{\PYGZlt{}xsl:stylesheet} \PYG{n+na}{xmlns:xsl=}\PYG{l+s}{\PYGZdq{}http://www.w3.org/1999/XSL/Transform\PYGZdq{}}\PYG{n+nt}{\PYGZgt{}}
\PYG{n+nt}{\PYGZlt{}xsl:template} \PYG{n+na}{match=}\PYG{l+s}{\PYGZdq{}/\PYGZdq{}}\PYG{n+nt}{\PYGZgt{}}
    \PYG{n+nt}{\PYGZlt{}datos}\PYG{n+nt}{\PYGZgt{}}
        \PYG{n+nt}{\PYGZlt{}cuentas}\PYG{n+nt}{\PYGZgt{}}
            \PYG{n+nt}{\PYGZlt{}xsl:for\PYGZhy{}each} \PYG{n+na}{select=}\PYG{l+s}{\PYGZdq{}listado/cuenta\PYGZdq{}}\PYG{n+nt}{\PYGZgt{}}
                \PYG{n+nt}{\PYGZlt{}cuenta}\PYG{n+nt}{\PYGZgt{}}
                    \PYG{n+nt}{\PYGZlt{}xsl:attribute} \PYG{n+na}{name=}\PYG{l+s}{\PYGZdq{}dnititular\PYGZdq{}}\PYG{n+nt}{\PYGZgt{}}
                        \PYG{n+nt}{\PYGZlt{}xsl:value\PYGZhy{}of} \PYG{n+na}{select=}\PYG{l+s}{\PYGZdq{}titular/@dni\PYGZdq{}}\PYG{n+nt}{/\PYGZgt{}}
                    \PYG{n+nt}{\PYGZlt{}/xsl:attribute\PYGZgt{}}
                    \PYG{c}{\PYGZlt{}!\PYGZhy{}\PYGZhy{}}\PYG{c}{Creamos el elemento \PYGZdq{}creación\PYGZdq{} que en}
\PYG{c}{                    realidad contiene el texto del elemento}
\PYG{c}{                    \PYGZdq{}fechacreación\PYGZdq{} original}\PYG{c}{\PYGZhy{}\PYGZhy{}\PYGZgt{}}
                    \PYG{n+nt}{\PYGZlt{}creacion}\PYG{n+nt}{\PYGZgt{}}
                        \PYG{n+nt}{\PYGZlt{}xsl:value\PYGZhy{}of} \PYG{n+na}{select=}\PYG{l+s}{\PYGZdq{}fechacreacion\PYGZdq{}}\PYG{n+nt}{/\PYGZgt{}}
                    \PYG{n+nt}{\PYGZlt{}/creacion\PYGZgt{}}
                    \PYG{c}{\PYGZlt{}!\PYGZhy{}\PYGZhy{}}\PYG{c}{Creamos el elemento titular y metemos}
\PYG{c}{                    dentro del valor original del titular}\PYG{c}{\PYGZhy{}\PYGZhy{}\PYGZgt{}}
                    \PYG{n+nt}{\PYGZlt{}titular}\PYG{n+nt}{\PYGZgt{}}
                        \PYG{n+nt}{\PYGZlt{}xsl:value\PYGZhy{}of} \PYG{n+na}{select=}\PYG{l+s}{\PYGZdq{}titular\PYGZdq{}}\PYG{n+nt}{/\PYGZgt{}}
                    \PYG{n+nt}{\PYGZlt{}/titular\PYGZgt{}}
                    \PYG{c}{\PYGZlt{}!\PYGZhy{}\PYGZhy{}}\PYG{c}{Creamos el saldo actual}\PYG{c}{\PYGZhy{}\PYGZhy{}\PYGZgt{}}
                    \PYG{n+nt}{\PYGZlt{}saldoactual}\PYG{n+nt}{\PYGZgt{}}
                        \PYG{c}{\PYGZlt{}!\PYGZhy{}\PYGZhy{}}\PYG{c}{Y metemos dentro la cantidad que tuviese}
\PYG{c}{                        el fichero original...}\PYG{c}{\PYGZhy{}\PYGZhy{}\PYGZgt{}}
                        \PYG{n+nt}{\PYGZlt{}xsl:value\PYGZhy{}of} \PYG{n+na}{select=}\PYG{l+s}{\PYGZdq{}saldoactual\PYGZdq{}}\PYG{n+nt}{/\PYGZgt{}}
                        \PYG{c}{\PYGZlt{}!\PYGZhy{}\PYGZhy{}}\PYG{c}{Y extraemos la moneda...}\PYG{c}{\PYGZhy{}\PYGZhy{}\PYGZgt{}}
                         \PYG{n+nt}{\PYGZlt{}xsl:value\PYGZhy{}of} \PYG{n+na}{select=}\PYG{l+s}{\PYGZdq{}saldoactual/@moneda\PYGZdq{}}\PYG{n+nt}{/\PYGZgt{}}
                    \PYG{n+nt}{\PYGZlt{}/saldoactual\PYGZgt{}}
                \PYG{n+nt}{\PYGZlt{}/cuenta\PYGZgt{}}
            \PYG{n+nt}{\PYGZlt{}/xsl:for\PYGZhy{}each\PYGZgt{}}
        \PYG{n+nt}{\PYGZlt{}/cuentas\PYGZgt{}}
        \PYG{n+nt}{\PYGZlt{}fondos}\PYG{n+nt}{\PYGZgt{}}
            \PYG{n+nt}{\PYGZlt{}xsl:for\PYGZhy{}each} \PYG{n+na}{select=}\PYG{l+s}{\PYGZdq{}listado/fondo\PYGZdq{}}\PYG{n+nt}{\PYGZgt{}}
                \PYG{c}{\PYGZlt{}!\PYGZhy{}\PYGZhy{}}\PYG{c}{Paso 1: crear un fondo por cada fondo original}\PYG{c}{\PYGZhy{}\PYGZhy{}\PYGZgt{}}
                \PYG{n+nt}{\PYGZlt{}fondo}\PYG{n+nt}{\PYGZgt{}}

                \PYG{n+nt}{\PYGZlt{}/fondo\PYGZgt{}}
            \PYG{n+nt}{\PYGZlt{}/xsl:for\PYGZhy{}each\PYGZgt{}}
        \PYG{n+nt}{\PYGZlt{}/fondos\PYGZgt{}}
    \PYG{n+nt}{\PYGZlt{}/datos\PYGZgt{}}
\PYG{n+nt}{\PYGZlt{}/xsl:template\PYGZgt{}}
\PYG{n+nt}{\PYGZlt{}/xsl:stylesheet\PYGZgt{}}
\end{sphinxVerbatim}

Ahora el paso 2: \sphinxstyleemphasis{poner el atributo «cuentaasociada»}. El XSLT sería así:

\begin{sphinxVerbatim}[commandchars=\\\{\}]
\PYG{n+nt}{\PYGZlt{}xsl:stylesheet} \PYG{n+na}{xmlns:xsl=}\PYG{l+s}{\PYGZdq{}http://www.w3.org/1999/XSL/Transform\PYGZdq{}}\PYG{n+nt}{\PYGZgt{}}
\PYG{n+nt}{\PYGZlt{}xsl:template} \PYG{n+na}{match=}\PYG{l+s}{\PYGZdq{}/\PYGZdq{}}\PYG{n+nt}{\PYGZgt{}}
    \PYG{n+nt}{\PYGZlt{}datos}\PYG{n+nt}{\PYGZgt{}}
        \PYG{n+nt}{\PYGZlt{}cuentas}\PYG{n+nt}{\PYGZgt{}}
            \PYG{n+nt}{\PYGZlt{}xsl:for\PYGZhy{}each} \PYG{n+na}{select=}\PYG{l+s}{\PYGZdq{}listado/cuenta\PYGZdq{}}\PYG{n+nt}{\PYGZgt{}}
                \PYG{n+nt}{\PYGZlt{}cuenta}\PYG{n+nt}{\PYGZgt{}}
                    \PYG{n+nt}{\PYGZlt{}xsl:attribute} \PYG{n+na}{name=}\PYG{l+s}{\PYGZdq{}dnititular\PYGZdq{}}\PYG{n+nt}{\PYGZgt{}}
                        \PYG{n+nt}{\PYGZlt{}xsl:value\PYGZhy{}of} \PYG{n+na}{select=}\PYG{l+s}{\PYGZdq{}titular/@dni\PYGZdq{}}\PYG{n+nt}{/\PYGZgt{}}
                    \PYG{n+nt}{\PYGZlt{}/xsl:attribute\PYGZgt{}}
                    \PYG{c}{\PYGZlt{}!\PYGZhy{}\PYGZhy{}}\PYG{c}{Creamos el elemento \PYGZdq{}creación\PYGZdq{} que en}
\PYG{c}{                    realidad contiene el texto del elemento}
\PYG{c}{                    \PYGZdq{}fechacreación\PYGZdq{} original}\PYG{c}{\PYGZhy{}\PYGZhy{}\PYGZgt{}}
                    \PYG{n+nt}{\PYGZlt{}creacion}\PYG{n+nt}{\PYGZgt{}}
                        \PYG{n+nt}{\PYGZlt{}xsl:value\PYGZhy{}of} \PYG{n+na}{select=}\PYG{l+s}{\PYGZdq{}fechacreacion\PYGZdq{}}\PYG{n+nt}{/\PYGZgt{}}
                    \PYG{n+nt}{\PYGZlt{}/creacion\PYGZgt{}}
                    \PYG{c}{\PYGZlt{}!\PYGZhy{}\PYGZhy{}}\PYG{c}{Creamos el elemento titular y metemos}
\PYG{c}{                    dentro del valor original del titular}\PYG{c}{\PYGZhy{}\PYGZhy{}\PYGZgt{}}
                    \PYG{n+nt}{\PYGZlt{}titular}\PYG{n+nt}{\PYGZgt{}}
                        \PYG{n+nt}{\PYGZlt{}xsl:value\PYGZhy{}of} \PYG{n+na}{select=}\PYG{l+s}{\PYGZdq{}titular\PYGZdq{}}\PYG{n+nt}{/\PYGZgt{}}
                    \PYG{n+nt}{\PYGZlt{}/titular\PYGZgt{}}
                    \PYG{c}{\PYGZlt{}!\PYGZhy{}\PYGZhy{}}\PYG{c}{Creamos el saldo actual}\PYG{c}{\PYGZhy{}\PYGZhy{}\PYGZgt{}}
                    \PYG{n+nt}{\PYGZlt{}saldoactual}\PYG{n+nt}{\PYGZgt{}}
                        \PYG{c}{\PYGZlt{}!\PYGZhy{}\PYGZhy{}}\PYG{c}{Y metemos dentro la cantidad que tuviese}
\PYG{c}{                        el fichero original...}\PYG{c}{\PYGZhy{}\PYGZhy{}\PYGZgt{}}
                        \PYG{n+nt}{\PYGZlt{}xsl:value\PYGZhy{}of} \PYG{n+na}{select=}\PYG{l+s}{\PYGZdq{}saldoactual\PYGZdq{}}\PYG{n+nt}{/\PYGZgt{}}
                        \PYG{c}{\PYGZlt{}!\PYGZhy{}\PYGZhy{}}\PYG{c}{Y extraemos la moneda...}\PYG{c}{\PYGZhy{}\PYGZhy{}\PYGZgt{}}
                         \PYG{n+nt}{\PYGZlt{}xsl:value\PYGZhy{}of} \PYG{n+na}{select=}\PYG{l+s}{\PYGZdq{}saldoactual/@moneda\PYGZdq{}}\PYG{n+nt}{/\PYGZgt{}}
                    \PYG{n+nt}{\PYGZlt{}/saldoactual\PYGZgt{}}
                \PYG{n+nt}{\PYGZlt{}/cuenta\PYGZgt{}}
            \PYG{n+nt}{\PYGZlt{}/xsl:for\PYGZhy{}each\PYGZgt{}}
        \PYG{n+nt}{\PYGZlt{}/cuentas\PYGZgt{}}
        \PYG{n+nt}{\PYGZlt{}fondos}\PYG{n+nt}{\PYGZgt{}}
            \PYG{n+nt}{\PYGZlt{}xsl:for\PYGZhy{}each} \PYG{n+na}{select=}\PYG{l+s}{\PYGZdq{}listado/fondo\PYGZdq{}}\PYG{n+nt}{\PYGZgt{}}
                \PYG{c}{\PYGZlt{}!\PYGZhy{}\PYGZhy{}}\PYG{c}{Paso 1: crear un fondo por cada fondo original}\PYG{c}{\PYGZhy{}\PYGZhy{}\PYGZgt{}}
                \PYG{n+nt}{\PYGZlt{}fondo}\PYG{n+nt}{\PYGZgt{}}
                    \PYG{c}{\PYGZlt{}!\PYGZhy{}\PYGZhy{}}\PYG{c}{Paso 2, crear el atributo cuentaasociada}\PYG{c}{\PYGZhy{}\PYGZhy{}\PYGZgt{}}
                    \PYG{n+nt}{\PYGZlt{}xsl:attribute} \PYG{n+na}{name=}\PYG{l+s}{\PYGZdq{}cuentaasociada\PYGZdq{}}\PYG{n+nt}{\PYGZgt{}}
                        \PYG{n+nt}{\PYGZlt{}xsl:value\PYGZhy{}of} \PYG{n+na}{select=}\PYG{l+s}{\PYGZdq{}cuentaasociada\PYGZdq{}}\PYG{n+nt}{/\PYGZgt{}}
                    \PYG{n+nt}{\PYGZlt{}/xsl:attribute\PYGZgt{}}

                \PYG{n+nt}{\PYGZlt{}/fondo\PYGZgt{}}
            \PYG{n+nt}{\PYGZlt{}/xsl:for\PYGZhy{}each\PYGZgt{}}
        \PYG{n+nt}{\PYGZlt{}/fondos\PYGZgt{}}
    \PYG{n+nt}{\PYGZlt{}/datos\PYGZgt{}}
\PYG{n+nt}{\PYGZlt{}/xsl:template\PYGZgt{}}
\PYG{n+nt}{\PYGZlt{}/xsl:stylesheet\PYGZgt{}}
\end{sphinxVerbatim}

Con el XSLT siguiente conseguimos el paso 3: \sphinxstyleemphasis{«crear el elemento cantidaddepositada»}

\begin{sphinxVerbatim}[commandchars=\\\{\}]
\PYG{n+nt}{\PYGZlt{}xsl:stylesheet} \PYG{n+na}{xmlns:xsl=}\PYG{l+s}{\PYGZdq{}http://www.w3.org/1999/XSL/Transform\PYGZdq{}}\PYG{n+nt}{\PYGZgt{}}
\PYG{n+nt}{\PYGZlt{}xsl:template} \PYG{n+na}{match=}\PYG{l+s}{\PYGZdq{}/\PYGZdq{}}\PYG{n+nt}{\PYGZgt{}}
    \PYG{n+nt}{\PYGZlt{}datos}\PYG{n+nt}{\PYGZgt{}}
        \PYG{n+nt}{\PYGZlt{}cuentas}\PYG{n+nt}{\PYGZgt{}}
            \PYG{n+nt}{\PYGZlt{}xsl:for\PYGZhy{}each} \PYG{n+na}{select=}\PYG{l+s}{\PYGZdq{}listado/cuenta\PYGZdq{}}\PYG{n+nt}{\PYGZgt{}}
                \PYG{n+nt}{\PYGZlt{}cuenta}\PYG{n+nt}{\PYGZgt{}}
                    \PYG{n+nt}{\PYGZlt{}xsl:attribute} \PYG{n+na}{name=}\PYG{l+s}{\PYGZdq{}dnititular\PYGZdq{}}\PYG{n+nt}{\PYGZgt{}}
                        \PYG{n+nt}{\PYGZlt{}xsl:value\PYGZhy{}of} \PYG{n+na}{select=}\PYG{l+s}{\PYGZdq{}titular/@dni\PYGZdq{}}\PYG{n+nt}{/\PYGZgt{}}
                    \PYG{n+nt}{\PYGZlt{}/xsl:attribute\PYGZgt{}}
                    \PYG{c}{\PYGZlt{}!\PYGZhy{}\PYGZhy{}}\PYG{c}{Creamos el elemento \PYGZdq{}creación\PYGZdq{} que en}
\PYG{c}{                    realidad contiene el texto del elemento}
\PYG{c}{                    \PYGZdq{}fechacreación\PYGZdq{} original}\PYG{c}{\PYGZhy{}\PYGZhy{}\PYGZgt{}}
                    \PYG{n+nt}{\PYGZlt{}creacion}\PYG{n+nt}{\PYGZgt{}}
                        \PYG{n+nt}{\PYGZlt{}xsl:value\PYGZhy{}of} \PYG{n+na}{select=}\PYG{l+s}{\PYGZdq{}fechacreacion\PYGZdq{}}\PYG{n+nt}{/\PYGZgt{}}
                    \PYG{n+nt}{\PYGZlt{}/creacion\PYGZgt{}}
                    \PYG{c}{\PYGZlt{}!\PYGZhy{}\PYGZhy{}}\PYG{c}{Creamos el elemento titular y metemos}
\PYG{c}{                    dentro del valor original del titular}\PYG{c}{\PYGZhy{}\PYGZhy{}\PYGZgt{}}
                    \PYG{n+nt}{\PYGZlt{}titular}\PYG{n+nt}{\PYGZgt{}}
                        \PYG{n+nt}{\PYGZlt{}xsl:value\PYGZhy{}of} \PYG{n+na}{select=}\PYG{l+s}{\PYGZdq{}titular\PYGZdq{}}\PYG{n+nt}{/\PYGZgt{}}
                    \PYG{n+nt}{\PYGZlt{}/titular\PYGZgt{}}
                    \PYG{c}{\PYGZlt{}!\PYGZhy{}\PYGZhy{}}\PYG{c}{Creamos el saldo actual}\PYG{c}{\PYGZhy{}\PYGZhy{}\PYGZgt{}}
                    \PYG{n+nt}{\PYGZlt{}saldoactual}\PYG{n+nt}{\PYGZgt{}}
                        \PYG{c}{\PYGZlt{}!\PYGZhy{}\PYGZhy{}}\PYG{c}{Y metemos dentro la cantidad que tuviese}
\PYG{c}{                        el fichero original...}\PYG{c}{\PYGZhy{}\PYGZhy{}\PYGZgt{}}
                        \PYG{n+nt}{\PYGZlt{}xsl:value\PYGZhy{}of} \PYG{n+na}{select=}\PYG{l+s}{\PYGZdq{}saldoactual\PYGZdq{}}\PYG{n+nt}{/\PYGZgt{}}
                        \PYG{c}{\PYGZlt{}!\PYGZhy{}\PYGZhy{}}\PYG{c}{Y extraemos la moneda...}\PYG{c}{\PYGZhy{}\PYGZhy{}\PYGZgt{}}
                         \PYG{n+nt}{\PYGZlt{}xsl:value\PYGZhy{}of} \PYG{n+na}{select=}\PYG{l+s}{\PYGZdq{}saldoactual/@moneda\PYGZdq{}}\PYG{n+nt}{/\PYGZgt{}}
                    \PYG{n+nt}{\PYGZlt{}/saldoactual\PYGZgt{}}
                \PYG{n+nt}{\PYGZlt{}/cuenta\PYGZgt{}}
            \PYG{n+nt}{\PYGZlt{}/xsl:for\PYGZhy{}each\PYGZgt{}}
        \PYG{n+nt}{\PYGZlt{}/cuentas\PYGZgt{}}
        \PYG{n+nt}{\PYGZlt{}fondos}\PYG{n+nt}{\PYGZgt{}}
            \PYG{n+nt}{\PYGZlt{}xsl:for\PYGZhy{}each} \PYG{n+na}{select=}\PYG{l+s}{\PYGZdq{}listado/fondo\PYGZdq{}}\PYG{n+nt}{\PYGZgt{}}
                \PYG{c}{\PYGZlt{}!\PYGZhy{}\PYGZhy{}}\PYG{c}{Paso 1: crear un fondo por cada fondo original}\PYG{c}{\PYGZhy{}\PYGZhy{}\PYGZgt{}}
                \PYG{n+nt}{\PYGZlt{}fondo}\PYG{n+nt}{\PYGZgt{}}
                    \PYG{c}{\PYGZlt{}!\PYGZhy{}\PYGZhy{}}\PYG{c}{Paso 2, crear el atributo cuentaasociada}\PYG{c}{\PYGZhy{}\PYGZhy{}\PYGZgt{}}
                    \PYG{n+nt}{\PYGZlt{}xsl:attribute} \PYG{n+na}{name=}\PYG{l+s}{\PYGZdq{}cuentaasociada\PYGZdq{}}\PYG{n+nt}{\PYGZgt{}}
                        \PYG{n+nt}{\PYGZlt{}xsl:value\PYGZhy{}of} \PYG{n+na}{select=}\PYG{l+s}{\PYGZdq{}cuentaasociada\PYGZdq{}}\PYG{n+nt}{/\PYGZgt{}}
                    \PYG{n+nt}{\PYGZlt{}/xsl:attribute\PYGZgt{}}
                    \PYG{c}{\PYGZlt{}!\PYGZhy{}\PYGZhy{}}\PYG{c}{Paso 3, crear el elemento cantidaddepositada}\PYG{c}{\PYGZhy{}\PYGZhy{}\PYGZgt{}}
                    \PYG{n+nt}{\PYGZlt{}cantidaddepositada}\PYG{n+nt}{\PYGZgt{}}
                        \PYG{n+nt}{\PYGZlt{}xsl:value\PYGZhy{}of} \PYG{n+na}{select=}\PYG{l+s}{\PYGZdq{}datos/cantidaddepositada\PYGZdq{}}\PYG{n+nt}{/\PYGZgt{}}
                    \PYG{n+nt}{\PYGZlt{}/cantidaddepositada\PYGZgt{}}
                \PYG{n+nt}{\PYGZlt{}/fondo\PYGZgt{}}
            \PYG{n+nt}{\PYGZlt{}/xsl:for\PYGZhy{}each\PYGZgt{}}
        \PYG{n+nt}{\PYGZlt{}/fondos\PYGZgt{}}
    \PYG{n+nt}{\PYGZlt{}/datos\PYGZgt{}}
\PYG{n+nt}{\PYGZlt{}/xsl:template\PYGZgt{}}
\PYG{n+nt}{\PYGZlt{}/xsl:stylesheet\PYGZgt{}}
\end{sphinxVerbatim}

Y por último el paso 4 «crear el elemento moneda». El XSLT sería algo así:

\begin{sphinxVerbatim}[commandchars=\\\{\}]
\PYG{n+nt}{\PYGZlt{}xsl:stylesheet} \PYG{n+na}{xmlns:xsl=}\PYG{l+s}{\PYGZdq{}http://www.w3.org/1999/XSL/Transform\PYGZdq{}}\PYG{n+nt}{\PYGZgt{}}
\PYG{n+nt}{\PYGZlt{}xsl:template} \PYG{n+na}{match=}\PYG{l+s}{\PYGZdq{}/\PYGZdq{}}\PYG{n+nt}{\PYGZgt{}}
    \PYG{n+nt}{\PYGZlt{}datos}\PYG{n+nt}{\PYGZgt{}}
        \PYG{n+nt}{\PYGZlt{}cuentas}\PYG{n+nt}{\PYGZgt{}}
            \PYG{n+nt}{\PYGZlt{}xsl:for\PYGZhy{}each} \PYG{n+na}{select=}\PYG{l+s}{\PYGZdq{}listado/cuenta\PYGZdq{}}\PYG{n+nt}{\PYGZgt{}}
                \PYG{n+nt}{\PYGZlt{}cuenta}\PYG{n+nt}{\PYGZgt{}}
                    \PYG{n+nt}{\PYGZlt{}xsl:attribute} \PYG{n+na}{name=}\PYG{l+s}{\PYGZdq{}dnititular\PYGZdq{}}\PYG{n+nt}{\PYGZgt{}}
                        \PYG{n+nt}{\PYGZlt{}xsl:value\PYGZhy{}of} \PYG{n+na}{select=}\PYG{l+s}{\PYGZdq{}titular/@dni\PYGZdq{}}\PYG{n+nt}{/\PYGZgt{}}
                    \PYG{n+nt}{\PYGZlt{}/xsl:attribute\PYGZgt{}}
                    \PYG{c}{\PYGZlt{}!\PYGZhy{}\PYGZhy{}}\PYG{c}{Creamos el elemento \PYGZdq{}creación\PYGZdq{} que en}
\PYG{c}{                    realidad contiene el texto del elemento}
\PYG{c}{                    \PYGZdq{}fechacreación\PYGZdq{} original}\PYG{c}{\PYGZhy{}\PYGZhy{}\PYGZgt{}}
                    \PYG{n+nt}{\PYGZlt{}creacion}\PYG{n+nt}{\PYGZgt{}}
                        \PYG{n+nt}{\PYGZlt{}xsl:value\PYGZhy{}of} \PYG{n+na}{select=}\PYG{l+s}{\PYGZdq{}fechacreacion\PYGZdq{}}\PYG{n+nt}{/\PYGZgt{}}
                    \PYG{n+nt}{\PYGZlt{}/creacion\PYGZgt{}}
                    \PYG{c}{\PYGZlt{}!\PYGZhy{}\PYGZhy{}}\PYG{c}{Creamos el elemento titular y metemos}
\PYG{c}{                    dentro del valor original del titular}\PYG{c}{\PYGZhy{}\PYGZhy{}\PYGZgt{}}
                    \PYG{n+nt}{\PYGZlt{}titular}\PYG{n+nt}{\PYGZgt{}}
                        \PYG{n+nt}{\PYGZlt{}xsl:value\PYGZhy{}of} \PYG{n+na}{select=}\PYG{l+s}{\PYGZdq{}titular\PYGZdq{}}\PYG{n+nt}{/\PYGZgt{}}
                    \PYG{n+nt}{\PYGZlt{}/titular\PYGZgt{}}
                    \PYG{c}{\PYGZlt{}!\PYGZhy{}\PYGZhy{}}\PYG{c}{Creamos el saldo actual}\PYG{c}{\PYGZhy{}\PYGZhy{}\PYGZgt{}}
                    \PYG{n+nt}{\PYGZlt{}saldoactual}\PYG{n+nt}{\PYGZgt{}}
                        \PYG{c}{\PYGZlt{}!\PYGZhy{}\PYGZhy{}}\PYG{c}{Y metemos dentro la cantidad que tuviese}
\PYG{c}{                        el fichero original...}\PYG{c}{\PYGZhy{}\PYGZhy{}\PYGZgt{}}
                        \PYG{n+nt}{\PYGZlt{}xsl:value\PYGZhy{}of} \PYG{n+na}{select=}\PYG{l+s}{\PYGZdq{}saldoactual\PYGZdq{}}\PYG{n+nt}{/\PYGZgt{}}
                        \PYG{c}{\PYGZlt{}!\PYGZhy{}\PYGZhy{}}\PYG{c}{Y extraemos la moneda...}\PYG{c}{\PYGZhy{}\PYGZhy{}\PYGZgt{}}
                         \PYG{n+nt}{\PYGZlt{}xsl:value\PYGZhy{}of} \PYG{n+na}{select=}\PYG{l+s}{\PYGZdq{}saldoactual/@moneda\PYGZdq{}}\PYG{n+nt}{/\PYGZgt{}}
                    \PYG{n+nt}{\PYGZlt{}/saldoactual\PYGZgt{}}
                \PYG{n+nt}{\PYGZlt{}/cuenta\PYGZgt{}}
            \PYG{n+nt}{\PYGZlt{}/xsl:for\PYGZhy{}each\PYGZgt{}}
        \PYG{n+nt}{\PYGZlt{}/cuentas\PYGZgt{}}
        \PYG{n+nt}{\PYGZlt{}fondos}\PYG{n+nt}{\PYGZgt{}}
            \PYG{n+nt}{\PYGZlt{}xsl:for\PYGZhy{}each} \PYG{n+na}{select=}\PYG{l+s}{\PYGZdq{}listado/fondo\PYGZdq{}}\PYG{n+nt}{\PYGZgt{}}
                \PYG{c}{\PYGZlt{}!\PYGZhy{}\PYGZhy{}}\PYG{c}{Paso 1: crear un fondo por cada fondo original}\PYG{c}{\PYGZhy{}\PYGZhy{}\PYGZgt{}}
                \PYG{n+nt}{\PYGZlt{}fondo}\PYG{n+nt}{\PYGZgt{}}
                    \PYG{c}{\PYGZlt{}!\PYGZhy{}\PYGZhy{}}\PYG{c}{Paso 2, crear el atributo cuentaasociada}\PYG{c}{\PYGZhy{}\PYGZhy{}\PYGZgt{}}
                    \PYG{n+nt}{\PYGZlt{}xsl:attribute} \PYG{n+na}{name=}\PYG{l+s}{\PYGZdq{}cuentaasociada\PYGZdq{}}\PYG{n+nt}{\PYGZgt{}}
                        \PYG{n+nt}{\PYGZlt{}xsl:value\PYGZhy{}of} \PYG{n+na}{select=}\PYG{l+s}{\PYGZdq{}cuentaasociada\PYGZdq{}}\PYG{n+nt}{/\PYGZgt{}}
                    \PYG{n+nt}{\PYGZlt{}/xsl:attribute\PYGZgt{}}
                    \PYG{c}{\PYGZlt{}!\PYGZhy{}\PYGZhy{}}\PYG{c}{Paso 3, crear el elemento cantidaddepositada}\PYG{c}{\PYGZhy{}\PYGZhy{}\PYGZgt{}}
                    \PYG{n+nt}{\PYGZlt{}cantidaddepositada}\PYG{n+nt}{\PYGZgt{}}
                        \PYG{n+nt}{\PYGZlt{}xsl:value\PYGZhy{}of} \PYG{n+na}{select=}\PYG{l+s}{\PYGZdq{}datos/cantidaddepositada\PYGZdq{}}\PYG{n+nt}{/\PYGZgt{}}
                    \PYG{n+nt}{\PYGZlt{}/cantidaddepositada\PYGZgt{}}
                    \PYG{c}{\PYGZlt{}!\PYGZhy{}\PYGZhy{}}\PYG{c}{Paso 4:Crear el elemento moneda}\PYG{c}{\PYGZhy{}\PYGZhy{}\PYGZgt{}}
                    \PYG{n+nt}{\PYGZlt{}moneda}\PYG{n+nt}{\PYGZgt{}}
                        \PYG{n+nt}{\PYGZlt{}xsl:value\PYGZhy{}of} \PYG{n+na}{select=}\PYG{l+s}{\PYGZdq{}datos/moneda\PYGZdq{}}\PYG{n+nt}{/\PYGZgt{}}
                    \PYG{n+nt}{\PYGZlt{}/moneda\PYGZgt{}}
                \PYG{n+nt}{\PYGZlt{}/fondo\PYGZgt{}}
            \PYG{n+nt}{\PYGZlt{}/xsl:for\PYGZhy{}each\PYGZgt{}}
        \PYG{n+nt}{\PYGZlt{}/fondos\PYGZgt{}}
    \PYG{n+nt}{\PYGZlt{}/datos\PYGZgt{}}
\PYG{n+nt}{\PYGZlt{}/xsl:template\PYGZgt{}}
\PYG{n+nt}{\PYGZlt{}/xsl:stylesheet\PYGZgt{}}
\end{sphinxVerbatim}

Si probamos el XSLT anterior veremos que efectivamente conseguimos transformar el fichero original en el fichero resultado que nos piden, que es el siguiente:

\begin{sphinxVerbatim}[commandchars=\\\{\}]
\PYG{n+nt}{\PYGZlt{}datos}\PYG{n+nt}{\PYGZgt{}}
  \PYG{n+nt}{\PYGZlt{}cuentas}\PYG{n+nt}{\PYGZgt{}}
    \PYG{n+nt}{\PYGZlt{}cuenta} \PYG{n+na}{dnititular=}\PYG{l+s}{\PYGZdq{}5671001D\PYGZdq{}}\PYG{n+nt}{\PYGZgt{}}
      \PYG{n+nt}{\PYGZlt{}creacion}\PYG{n+nt}{\PYGZgt{}}13\PYGZhy{}abril\PYGZhy{}2012\PYG{n+nt}{\PYGZlt{}/creacion\PYGZgt{}}
      \PYG{n+nt}{\PYGZlt{}titular}\PYG{n+nt}{\PYGZgt{}}Ramon Perez\PYG{n+nt}{\PYGZlt{}/titular\PYGZgt{}}
      \PYG{n+nt}{\PYGZlt{}saldoactual}\PYG{n+nt}{\PYGZgt{}}12000euros\PYG{n+nt}{\PYGZlt{}/saldoactual\PYGZgt{}}
    \PYG{n+nt}{\PYGZlt{}/cuenta\PYGZgt{}}
    \PYG{n+nt}{\PYGZlt{}cuenta} \PYG{n+na}{dnititular=}\PYG{l+s}{\PYGZdq{}39812341C\PYGZdq{}}\PYG{n+nt}{\PYGZgt{}}
      \PYG{n+nt}{\PYGZlt{}creacion}\PYG{n+nt}{\PYGZgt{}}15\PYGZhy{}febrero\PYGZhy{}2011\PYG{n+nt}{\PYGZlt{}/creacion\PYGZgt{}}
      \PYG{n+nt}{\PYGZlt{}titular}\PYG{n+nt}{\PYGZgt{}}Carmen Diaz\PYG{n+nt}{\PYGZlt{}/titular\PYGZgt{}}
      \PYG{n+nt}{\PYGZlt{}saldoactual}\PYG{n+nt}{\PYGZgt{}}1900euros\PYG{n+nt}{\PYGZlt{}/saldoactual\PYGZgt{}}
    \PYG{n+nt}{\PYGZlt{}/cuenta\PYGZgt{}}
  \PYG{n+nt}{\PYGZlt{}/cuentas\PYGZgt{}}
  \PYG{n+nt}{\PYGZlt{}fondos}\PYG{n+nt}{\PYGZgt{}}
    \PYG{n+nt}{\PYGZlt{}fondo} \PYG{n+na}{cuentaasociada=}\PYG{l+s}{\PYGZdq{}20\PYGZhy{}A\PYGZdq{}}\PYG{n+nt}{\PYGZgt{}}
      \PYG{n+nt}{\PYGZlt{}cantidaddepositada}\PYG{n+nt}{\PYGZgt{}}20000\PYG{n+nt}{\PYGZlt{}/cantidaddepositada\PYGZgt{}}
      \PYG{n+nt}{\PYGZlt{}moneda}\PYG{n+nt}{\PYGZgt{}}Euros\PYG{n+nt}{\PYGZlt{}/moneda\PYGZgt{}}
    \PYG{n+nt}{\PYGZlt{}/fondo\PYGZgt{}}
    \PYG{n+nt}{\PYGZlt{}fondo} \PYG{n+na}{cuentaasociada=}\PYG{l+s}{\PYGZdq{}21\PYGZhy{}DX\PYGZdq{}}\PYG{n+nt}{\PYGZgt{}}
      \PYG{n+nt}{\PYGZlt{}cantidaddepositada}\PYG{n+nt}{\PYGZgt{}}4800\PYG{n+nt}{\PYGZlt{}/cantidaddepositada\PYGZgt{}}
      \PYG{n+nt}{\PYGZlt{}moneda}\PYG{n+nt}{\PYGZgt{}}Dolares\PYG{n+nt}{\PYGZlt{}/moneda\PYGZgt{}}
    \PYG{n+nt}{\PYGZlt{}/fondo\PYGZgt{}}
  \PYG{n+nt}{\PYGZlt{}/fondos\PYGZgt{}}
\PYG{n+nt}{\PYGZlt{}/datos\PYGZgt{}}
\end{sphinxVerbatim}


\chapter{Anexo: Ejercicios de XPath}
\label{\detokenize{ejercicios/xpath/anexo_ejercicios_xpath::doc}}\label{\detokenize{ejercicios/xpath/anexo_ejercicios_xpath:anexo-ejercicios-de-xpath}}
En los ejercicios siguientes se asume que se va a utilizar el fichero siguiente:

\begin{sphinxVerbatim}[commandchars=\\\{\}]
\PYG{n+nt}{\PYGZlt{}inventario}\PYG{n+nt}{\PYGZgt{}}
    \PYG{n+nt}{\PYGZlt{}producto} \PYG{n+na}{codigo=}\PYG{l+s}{\PYGZdq{}AAA\PYGZhy{}111\PYGZdq{}}\PYG{n+nt}{\PYGZgt{}}
        \PYG{n+nt}{\PYGZlt{}nombre}\PYG{n+nt}{\PYGZgt{}}Teclado\PYG{n+nt}{\PYGZlt{}/nombre\PYGZgt{}}
        \PYG{n+nt}{\PYGZlt{}peso} \PYG{n+na}{unidad=}\PYG{l+s}{\PYGZdq{}g\PYGZdq{}}\PYG{n+nt}{\PYGZgt{}}480\PYG{n+nt}{\PYGZlt{}/peso\PYGZgt{}}
    \PYG{n+nt}{\PYGZlt{}/producto\PYGZgt{}}
    \PYG{n+nt}{\PYGZlt{}producto} \PYG{n+na}{codigo=}\PYG{l+s}{\PYGZdq{}ACD\PYGZhy{}981\PYGZdq{}}\PYG{n+nt}{\PYGZgt{}}
        \PYG{n+nt}{\PYGZlt{}nombre}\PYG{n+nt}{\PYGZgt{}}Monitor\PYG{n+nt}{\PYGZlt{}/nombre\PYGZgt{}}
        \PYG{n+nt}{\PYGZlt{}peso} \PYG{n+na}{unidad=}\PYG{l+s}{\PYGZdq{}kg\PYGZdq{}}\PYG{n+nt}{\PYGZgt{}}1.8\PYG{n+nt}{\PYGZlt{}/peso\PYGZgt{}}
    \PYG{n+nt}{\PYGZlt{}/producto\PYGZgt{}}
    \PYG{n+nt}{\PYGZlt{}producto} \PYG{n+na}{codigo=}\PYG{l+s}{\PYGZdq{}DEZ\PYGZhy{}138\PYGZdq{}}\PYG{n+nt}{\PYGZgt{}}
        \PYG{n+nt}{\PYGZlt{}nombre}\PYG{n+nt}{\PYGZgt{}}Raton\PYG{n+nt}{\PYGZlt{}/nombre\PYGZgt{}}
        \PYG{n+nt}{\PYGZlt{}peso} \PYG{n+na}{unidad=}\PYG{l+s}{\PYGZdq{}g\PYGZdq{}}\PYG{n+nt}{\PYGZgt{}}50\PYG{n+nt}{\PYGZlt{}/peso\PYGZgt{}}
    \PYG{n+nt}{\PYGZlt{}/producto\PYGZgt{}}
\PYG{n+nt}{\PYGZlt{}/inventario\PYGZgt{}}
\end{sphinxVerbatim}

Resolver los siguientes problemas usando expresiones XPath. Si no nos dicen nada se puede asumir que las etiquetas no deben incluirse.
\begin{itemize}
\item {} 
Extraer todos los elementos peso (etiqueta incluida).

\item {} 
Extraer las cantidades de todos los elementos peso (sin la etiqueta \textless{}peso\textgreater{}).

\item {} 
Extraer el peso del ultimo elemento.

\item {} 
Extraer las distintas unidades en las que se han almacenado los pesos.

\item {} 
Extraer el penúltimo codigo.

\item {} 
Extraer el peso del elemento cuyo codigo sea AAA-111.

\item {} 
Extraer el nombre de los productos que hayan puesto el peso en gramos.

\item {} 
Extraer el codigo de los productos cuyo nombre sea «Monitor».

\item {} 
Extraer el código de los productos que pesen más de un cuarto de kilo.

\end{itemize}


\section{Soluciones}
\label{\detokenize{ejercicios/xpath/anexo_ejercicios_xpath:soluciones}}
Para el enunciado  \sphinxstyleemphasis{Extraer todos los elementos peso (etiqueta incluida)} . La solución sería algo como esto \sphinxcode{/inventario/producto/peso} . De hecho nos devuelve esto:

\begin{sphinxVerbatim}[commandchars=\\\{\}]
\PYG{n+nt}{\PYGZlt{}peso} \PYG{n+na}{unidad=}\PYG{l+s}{\PYGZdq{}g\PYGZdq{}}\PYG{n+nt}{\PYGZgt{}}480\PYG{n+nt}{\PYGZlt{}/peso\PYGZgt{}}
\PYG{n+nt}{\PYGZlt{}peso} \PYG{n+na}{unidad=}\PYG{l+s}{\PYGZdq{}kg\PYGZdq{}}\PYG{n+nt}{\PYGZgt{}}1.8\PYG{n+nt}{\PYGZlt{}/peso\PYGZgt{}}
\PYG{n+nt}{\PYGZlt{}peso} \PYG{n+na}{unidad=}\PYG{l+s}{\PYGZdq{}g\PYGZdq{}}\PYG{n+nt}{\PYGZgt{}}50\PYG{n+nt}{\PYGZlt{}/peso\PYGZgt{}}
\end{sphinxVerbatim}

Enunciado: \sphinxstyleemphasis{Extraer las cantidades de todos los elementos peso (sin la etiqueta \textless{}peso\textgreater{})}. La solución sería \sphinxcode{/inventario/producto/peso/text()}. La expresión devuelve

\begin{sphinxVerbatim}[commandchars=\\\{\}]
480
1.8
50
\end{sphinxVerbatim}

Enunciado:\sphinxstyleemphasis{Extraer el peso del ultimo producto.}. Una posible solución \sphinxstylestrong{equivocada} sería esta \sphinxcode{/inventario/producto/peso{[}last(){]}}, que en realidad recupera «el último peso de cada producto», es decir recupera muchos, como muestra el resultado siguiente:

\begin{sphinxVerbatim}[commandchars=\\\{\}]
\PYG{n+nt}{\PYGZlt{}peso} \PYG{n+na}{unidad=}\PYG{l+s}{\PYGZdq{}g\PYGZdq{}}\PYG{n+nt}{\PYGZgt{}}480\PYG{n+nt}{\PYGZlt{}/peso\PYGZgt{}}
\PYG{n+nt}{\PYGZlt{}peso} \PYG{n+na}{unidad=}\PYG{l+s}{\PYGZdq{}kg\PYGZdq{}}\PYG{n+nt}{\PYGZgt{}}1.8\PYG{n+nt}{\PYGZlt{}/peso\PYGZgt{}}
\PYG{n+nt}{\PYGZlt{}peso} \PYG{n+na}{unidad=}\PYG{l+s}{\PYGZdq{}g\PYGZdq{}}\PYG{n+nt}{\PYGZgt{}}50\PYG{n+nt}{\PYGZlt{}/peso\PYGZgt{}}
\end{sphinxVerbatim}

La expresión correcta sería \sphinxcode{/inventario/producto{[}last(){]}/peso} que devuelve esto

\begin{sphinxVerbatim}[commandchars=\\\{\}]
\PYG{n+nt}{\PYGZlt{}peso} \PYG{n+na}{unidad=}\PYG{l+s}{\PYGZdq{}g\PYGZdq{}}\PYG{n+nt}{\PYGZgt{}}50\PYG{n+nt}{\PYGZlt{}/peso\PYGZgt{}}
\end{sphinxVerbatim}

Enunciado: \sphinxstyleemphasis{Extraer las distintas unidades en las que se han almacenado los pesos}. Una posible solución sería esta \sphinxcode{/inventario/producto/peso/@unidad}, que devuelve esto:

\begin{sphinxVerbatim}[commandchars=\\\{\}]
g
kg
g
\end{sphinxVerbatim}

Enunciado: \sphinxstyleemphasis{Extraer el penúltimo codigo.}. Una posible solucion sería esta:\sphinxcode{/inventario/producto{[}last()-1{]}/@codigo}

Enunciado: \sphinxstyleemphasis{Extraer el peso del elemento cuyo codigo sea AAA-111.} \sphinxcode{/inventario/producto{[}@codigo="AAA-111"{]}/peso}

Esto devuelve como resultado:

\begin{sphinxVerbatim}[commandchars=\\\{\}]
\PYG{n+nt}{\PYGZlt{}peso} \PYG{n+na}{unidad=}\PYG{l+s}{\PYGZdq{}g\PYGZdq{}}\PYG{n+nt}{\PYGZgt{}}480\PYG{n+nt}{\PYGZlt{}/peso\PYGZgt{}}
\end{sphinxVerbatim}

Enunciado: \sphinxstyleemphasis{Extraer el nombre de los productos que hayan puesto el peso en gramos.}

Una idea incorrecta sería esta \sphinxcode{/inventario/producto/peso{[}@unidad="g"{]}}.
Esta está mal, porque recupera «pesos» en lugar de «nombres» . De hecho recupera esto:

\begin{sphinxVerbatim}[commandchars=\\\{\}]
\PYG{n+nt}{\PYGZlt{}peso} \PYG{n+na}{unidad=}\PYG{l+s}{\PYGZdq{}g\PYGZdq{}}\PYG{n+nt}{\PYGZgt{}}480\PYG{n+nt}{\PYGZlt{}/peso\PYGZgt{}}
\PYG{n+nt}{\PYGZlt{}peso} \PYG{n+na}{unidad=}\PYG{l+s}{\PYGZdq{}g\PYGZdq{}}\PYG{n+nt}{\PYGZgt{}}50\PYG{n+nt}{\PYGZlt{}/peso\PYGZgt{}}
\end{sphinxVerbatim}

Una correcta sería

\sphinxcode{/inventario/producto{[}peso/@unidad="g"{]}/nombre}

Y otra posibilidad sería

\sphinxcode{/inventario/producto{[}peso{[}@unidad="g"{]}{]}/nombre}

Que literalmente pone \sphinxstyleemphasis{«extraer el nombre de productos que cumplan la condicion de tener un hijo peso y a su vez ese hijo peso cumpla la condición de tener un atributo unidad con el valor g»}

Esto devuelve

\begin{sphinxVerbatim}[commandchars=\\\{\}]
\PYG{n+nt}{\PYGZlt{}nombre}\PYG{n+nt}{\PYGZgt{}}Teclado\PYG{n+nt}{\PYGZlt{}/nombre\PYGZgt{}}
\PYG{n+nt}{\PYGZlt{}nombre}\PYG{n+nt}{\PYGZgt{}}Raton\PYG{n+nt}{\PYGZlt{}/nombre\PYGZgt{}}
\end{sphinxVerbatim}

Enunciado: \sphinxstyleemphasis{Extraer el codigo de los productos cuyo nombre sea «Monitor»}

Una posible solución sería

\sphinxcode{/inventario/producto{[}nombre/text()="Monitor"{]}/@codigo}

Aunque en realidad también serviría lo siguiente:

\sphinxcode{/inventario/producto{[}nombre="Monitor"{]}/@codigo}

La clave es que \sphinxstylestrong{como el elemento nombre no tiene hijos entonces se permite comparar el elemento como una cadena}. El evaluador XPath sobreentiende que queremos comparar «el contenido del elemento nombre» con la cadena «Monitor».

Enunciado: \sphinxstyleemphasis{Extraer el código de los productos que pesen más de un cuarto de kilo.}.

La solución sería

\sphinxcode{/inventario/producto{[}
(peso/@unidad="g" and peso/text()\textgreater{}"250")
or
(peso/@unidad="kg" and peso/text()\textgreater{}"0.25")
{]}/@codigo}


\section{Información bancaria}
\label{\detokenize{ejercicios/xpath/anexo_ejercicios_xpath:informacion-bancaria}}
Dado el siguiente fichero XML, contestar a las preguntas que se enuncian más abajo usando XPath.

\begin{sphinxVerbatim}[commandchars=\\\{\}]
\PYG{n+nt}{\PYGZlt{}listado}\PYG{n+nt}{\PYGZgt{}}
    \PYG{n+nt}{\PYGZlt{}cuenta}\PYG{n+nt}{\PYGZgt{}}
        \PYG{n+nt}{\PYGZlt{}titular} \PYG{n+na}{dni=}\PYG{l+s}{\PYGZdq{}5671001D\PYGZdq{}}\PYG{n+nt}{\PYGZgt{}}Ramon Perez\PYG{n+nt}{\PYGZlt{}/titular\PYGZgt{}}
        \PYG{n+nt}{\PYGZlt{}saldoactual} \PYG{n+na}{moneda=}\PYG{l+s}{\PYGZdq{}euros\PYGZdq{}}\PYG{n+nt}{\PYGZgt{}}12000\PYG{n+nt}{\PYGZlt{}/saldoactual\PYGZgt{}}
        \PYG{n+nt}{\PYGZlt{}fechacreacion}\PYG{n+nt}{\PYGZgt{}}13\PYGZhy{}abril\PYGZhy{}2012\PYG{n+nt}{\PYGZlt{}/fechacreacion\PYGZgt{}}
    \PYG{n+nt}{\PYGZlt{}/cuenta\PYGZgt{}}
    \PYG{n+nt}{\PYGZlt{}fondo}\PYG{n+nt}{\PYGZgt{}}
        \PYG{n+nt}{\PYGZlt{}cuentaasociada}\PYG{n+nt}{\PYGZgt{}}20\PYGZhy{}A\PYG{n+nt}{\PYGZlt{}/cuentaasociada\PYGZgt{}}
        \PYG{n+nt}{\PYGZlt{}datos}\PYG{n+nt}{\PYGZgt{}}
            \PYG{n+nt}{\PYGZlt{}cantidaddepositada}\PYG{n+nt}{\PYGZgt{}}20000\PYG{n+nt}{\PYGZlt{}/cantidaddepositada\PYGZgt{}}
            \PYG{n+nt}{\PYGZlt{}moneda}\PYG{n+nt}{\PYGZgt{}}Euros\PYG{n+nt}{\PYGZlt{}/moneda\PYGZgt{}}
        \PYG{n+nt}{\PYGZlt{}/datos\PYGZgt{}}
    \PYG{n+nt}{\PYGZlt{}/fondo\PYGZgt{}}
    \PYG{n+nt}{\PYGZlt{}fondo}\PYG{n+nt}{\PYGZgt{}}
        \PYG{n+nt}{\PYGZlt{}cuentaasociada}\PYG{n+nt}{\PYGZgt{}}21\PYGZhy{}DX\PYG{n+nt}{\PYGZlt{}/cuentaasociada\PYGZgt{}}
        \PYG{n+nt}{\PYGZlt{}datos}\PYG{n+nt}{\PYGZgt{}}
            \PYG{n+nt}{\PYGZlt{}cantidaddepositada}\PYG{n+nt}{\PYGZgt{}}4800\PYG{n+nt}{\PYGZlt{}/cantidaddepositada\PYGZgt{}}
            \PYG{n+nt}{\PYGZlt{}moneda}\PYG{n+nt}{\PYGZgt{}}Dolares\PYG{n+nt}{\PYGZlt{}/moneda\PYGZgt{}}
        \PYG{n+nt}{\PYGZlt{}/datos\PYGZgt{}}
    \PYG{n+nt}{\PYGZlt{}/fondo\PYGZgt{}}
    \PYG{n+nt}{\PYGZlt{}cuenta}\PYG{n+nt}{\PYGZgt{}}
        \PYG{n+nt}{\PYGZlt{}titular} \PYG{n+na}{dni=}\PYG{l+s}{\PYGZdq{}39812341C\PYGZdq{}}\PYG{n+nt}{\PYGZgt{}}Carmen Diaz\PYG{n+nt}{\PYGZlt{}/titular\PYGZgt{}}
        \PYG{n+nt}{\PYGZlt{}saldoactual} \PYG{n+na}{moneda=}\PYG{l+s}{\PYGZdq{}euros\PYGZdq{}}\PYG{n+nt}{\PYGZgt{}}1900\PYG{n+nt}{\PYGZlt{}/saldoactual\PYGZgt{}}
        \PYG{n+nt}{\PYGZlt{}fechacreacion}\PYG{n+nt}{\PYGZgt{}}15\PYGZhy{}febrero\PYGZhy{}2011\PYG{n+nt}{\PYGZlt{}/fechacreacion\PYGZgt{}}
    \PYG{n+nt}{\PYGZlt{}/cuenta\PYGZgt{}}
\PYG{n+nt}{\PYGZlt{}/listado\PYGZgt{}}
\end{sphinxVerbatim}


\section{Consulta: «cantidad depositada»}
\label{\detokenize{ejercicios/xpath/anexo_ejercicios_xpath:consulta-cantidad-depositada}}
\sphinxstyleemphasis{Extraer la cantidad depositada en la cuenta 20-A}
\begin{itemize}
\item {} 
\sphinxcode{/listado/fondo} nos devolvería todos los elementos \sphinxcode{fondo}

\item {} 
\sphinxcode{/listado/fondo/datos} nos devolvería todos los elementos \sphinxcode{datos} que sean hijos de \sphinxcode{fondo} los cuales a su vez deben ir dentro de \sphinxcode{inventario}.

\end{itemize}

Como nos piden una cantidad el último elemento XPath debe ser forzosamente \sphinxcode{cantidad}. Como nos ponen una condición tendremos que usar corchetes. El elemento \sphinxcode{cuentaasociada} es hijo de \sphinxcode{fondo} así que una buena posibilidad sería poner la condicion con el elemento fondo, así tendríamos que

Si añadimos la condición parece entonces que una buena posibilidad sería algo como esto:

\begin{sphinxVerbatim}[commandchars=\\\{\}]
\PYG{o}{/}\PYG{n}{listado}\PYG{o}{/}\PYG{n}{fondo}\PYG{p}{[}\PYG{n}{cuentaasociada} \PYG{o}{=} \PYG{l+s+s2}{\PYGZdq{}}\PYG{l+s+s2}{20\PYGZhy{}A}\PYG{l+s+s2}{\PYGZdq{}}\PYG{p}{]}\PYG{o}{/}\PYG{n}{datos}\PYG{o}{/}\PYG{n}{cantidaddepositada}
\end{sphinxVerbatim}

Si realmenten nos piden la cantidad sin etiquetas podemos añadir al final \sphinxcode{/text()} y dejarlo así:

\begin{sphinxVerbatim}[commandchars=\\\{\}]
\PYG{o}{/}\PYG{n}{listado}\PYG{o}{/}\PYG{n}{fondo}\PYG{p}{[}\PYG{n}{cuentaasociada} \PYG{o}{=} \PYG{l+s+s2}{\PYGZdq{}}\PYG{l+s+s2}{20\PYGZhy{}A}\PYG{l+s+s2}{\PYGZdq{}}\PYG{p}{]}\PYG{o}{/}\PYG{n}{datos}\PYG{o}{/}\PYG{n}{cantidaddepositada}\PYG{o}{/}\PYG{n}{text}\PYG{p}{(}\PYG{p}{)}
\end{sphinxVerbatim}


\section{Consulta: «monedas usadas»}
\label{\detokenize{ejercicios/xpath/anexo_ejercicios_xpath:consulta-monedas-usadas}}
\sphinxstyleemphasis{Extraer un listado sin etiquetas de todas las monedas usadas por los distintos fondos}

La consulta sería simplemente:

\begin{sphinxVerbatim}[commandchars=\\\{\}]
\PYG{o}{/}\PYG{n}{listado}\PYG{o}{/}\PYG{n}{fondo}\PYG{o}{/}\PYG{n}{datos}\PYG{o}{/}\PYG{n}{moneda}\PYG{o}{/}\PYG{n}{text}\PYG{p}{(}\PYG{p}{)}
\end{sphinxVerbatim}


\section{Consulta: «dnis con dólares»}
\label{\detokenize{ejercicios/xpath/anexo_ejercicios_xpath:consulta-dnis-con-dolares}}
\sphinxstyleemphasis{Extraer el DNI de las cuentas que usen dolares como moneda de base}

En las cuentas la moneda es un atributo de \sphinxcode{saldoactual}, por lo que la condición debería ser:

\begin{sphinxVerbatim}[commandchars=\\\{\}]
\PYG{o}{.}\PYG{o}{.}\PYG{o}{.}\PYG{p}{[}\PYG{n+nd}{@moneda}\PYG{o}{=}\PYG{l+s+s2}{\PYGZdq{}}\PYG{l+s+s2}{dolares}\PYG{l+s+s2}{\PYGZdq{}}\PYG{p}{]}\PYG{o}{.}\PYG{o}{.}\PYG{o}{.}
\end{sphinxVerbatim}

Como nos piden el DNI  este debe ser el campo que debe aparecer al final del XPath. Aparte de eso el DNI es un atributo de \sphinxcode{titular}, por lo que la consulta debería ser algo así:

\begin{sphinxVerbatim}[commandchars=\\\{\}]
\PYG{o}{.}\PYG{o}{.}\PYG{o}{.}\PYG{p}{[}\PYG{n+nd}{@moneda}\PYG{o}{=}\PYG{l+s+s2}{\PYGZdq{}}\PYG{l+s+s2}{dolares}\PYG{l+s+s2}{\PYGZdq{}}\PYG{p}{]}\PYG{o}{.}\PYG{o}{.}\PYG{o}{.} \PYG{o}{/}\PYG{n+nd}{@dni}
\end{sphinxVerbatim}

Como el \sphinxcode{dni} es un atributo de \sphinxcode{titular} y \sphinxcode{moneda} es un atributo de \sphinxcode{saldoactual} una buena posibilidad es mover la condición al elemento padre \sphinxcode{cuenta} y hacer algo así:

\begin{sphinxVerbatim}[commandchars=\\\{\}]
\PYG{o}{/}\PYG{n}{listado}\PYG{o}{/}\PYG{n}{cuenta}\PYG{p}{[}\PYG{n}{saldoactual}\PYG{o}{/}\PYG{n+nd}{@moneda} \PYG{o}{=} \PYG{l+s+s2}{\PYGZdq{}}\PYG{l+s+s2}{dolares}\PYG{l+s+s2}{\PYGZdq{}}\PYG{p}{]}\PYG{o}{/}\PYG{n}{titular}\PYG{o}{/}\PYG{n+nd}{@dni}
\end{sphinxVerbatim}

Obsérvese que para el ejemplo dado \sphinxstylestrong{no hay ningún resultado} (de hecho el evaluador dará error porque nadie cumple la condición). Si se desea comprobar que el XPath está bien se invita al lector a añadir datos al conjunto para verificar que la expresión XPath funciona correctamente.


\section{Consulta: «fondos con menos de 2500 euros»}
\label{\detokenize{ejercicios/xpath/anexo_ejercicios_xpath:consulta-fondos-con-menos-de-2500-euros}}
\sphinxstyleemphasis{Extraer toda la información de los fondos que usen «euros» por un valor inferior a 2500}

Por el texto del enunciado es evidente que el último elemento del XPath debe ser \sphinxcode{fondo}, por lo que de momento podemos saber que la expresión tendría este aspecto:

\begin{sphinxVerbatim}[commandchars=\\\{\}]
\PYG{o}{.}\PYG{o}{.}\PYG{o}{.}\PYG{o}{/}\PYG{n}{fondo}
\end{sphinxVerbatim}

También es evidente que hay una condición de filtrado. Esta condición tiene dos partes:
\begin{itemize}
\item {} 
Por un lado la divisa debe ser «Dolares». La divisa va dentro del elemento \sphinxcode{moneda} que es hijo de \sphinxcode{datos} que a su vez es hijo de \sphinxcode{fondo}

\item {} 
Por otro lado la cantidad debe ser inferior a 2500. Esta información está en \sphinxcode{cantidaddepositada}.

\item {} 
Para este caso necesitamos que se cumplan ambas condiciones por lo que deberemos usar \sphinxcode{AND}

\end{itemize}

Una posible consulta que resuelve esto es:

\begin{sphinxVerbatim}[commandchars=\\\{\}]
\PYG{o}{/}\PYG{n}{listado}\PYG{o}{/}\PYG{n}{fondo}\PYG{p}{[}\PYG{n}{datos}\PYG{o}{/}\PYG{n}{cantidaddepositada}\PYG{o}{\PYGZlt{}}\PYG{l+m+mi}{2500} \PYG{o+ow}{and} \PYG{n}{datos}\PYG{o}{/}\PYG{n}{moneda} \PYG{o}{=} \PYG{l+s+s2}{\PYGZdq{}}\PYG{l+s+s2}{Dolares}\PYG{l+s+s2}{\PYGZdq{}}\PYG{p}{]}
\end{sphinxVerbatim}

De nuevo nos encontramos con que el programa puede dar une rror, lo que tiene todo el sentido del mundo, ya que le pedimos que extraiga datos de algo que no existe.


\chapter{Anexo: Ejercicios de XQuery}
\label{\detokenize{ejercicios/xquery/anexo_ejercicios_xquery::doc}}\label{\detokenize{ejercicios/xquery/anexo_ejercicios_xquery:anexo-ejercicios-de-xquery}}
En los ejercicios siguientes se asume que se va a utilizar la siguiente base de datos XML (tomada del libro de C.J. Date «Sistemas Gestores de Bases de Datos»)

\begin{sphinxVerbatim}[commandchars=\\\{\}]
\PYG{n+nt}{\PYGZlt{}datos}\PYG{n+nt}{\PYGZgt{}}
    \PYG{n+nt}{\PYGZlt{}proveedores}\PYG{n+nt}{\PYGZgt{}}
        \PYG{n+nt}{\PYGZlt{}proveedor} \PYG{n+na}{numprov=}\PYG{l+s}{\PYGZdq{}v1\PYGZdq{}}\PYG{n+nt}{\PYGZgt{}}
            \PYG{n+nt}{\PYGZlt{}nombreprov}\PYG{n+nt}{\PYGZgt{}}Smith\PYG{n+nt}{\PYGZlt{}/nombreprov\PYGZgt{}}
            \PYG{n+nt}{\PYGZlt{}estado}\PYG{n+nt}{\PYGZgt{}}20\PYG{n+nt}{\PYGZlt{}/estado\PYGZgt{}}
            \PYG{n+nt}{\PYGZlt{}ciudad}\PYG{n+nt}{\PYGZgt{}}Londres\PYG{n+nt}{\PYGZlt{}/ciudad\PYGZgt{}}
        \PYG{n+nt}{\PYGZlt{}/proveedor\PYGZgt{}}
        \PYG{n+nt}{\PYGZlt{}proveedor} \PYG{n+na}{numprov=}\PYG{l+s}{\PYGZdq{}v2\PYGZdq{}}\PYG{n+nt}{\PYGZgt{}}
            \PYG{n+nt}{\PYGZlt{}nombreprov}\PYG{n+nt}{\PYGZgt{}}Jones\PYG{n+nt}{\PYGZlt{}/nombreprov\PYGZgt{}}
            \PYG{n+nt}{\PYGZlt{}estado}\PYG{n+nt}{\PYGZgt{}}10\PYG{n+nt}{\PYGZlt{}/estado\PYGZgt{}}
            \PYG{n+nt}{\PYGZlt{}ciudad}\PYG{n+nt}{\PYGZgt{}}Paris\PYG{n+nt}{\PYGZlt{}/ciudad\PYGZgt{}}
        \PYG{n+nt}{\PYGZlt{}/proveedor\PYGZgt{}}
        \PYG{n+nt}{\PYGZlt{}proveedor} \PYG{n+na}{numprov=}\PYG{l+s}{\PYGZdq{}v3\PYGZdq{}}\PYG{n+nt}{\PYGZgt{}}
            \PYG{n+nt}{\PYGZlt{}nombreprov}\PYG{n+nt}{\PYGZgt{}}Blake\PYG{n+nt}{\PYGZlt{}/nombreprov\PYGZgt{}}
            \PYG{n+nt}{\PYGZlt{}estado}\PYG{n+nt}{\PYGZgt{}}30\PYG{n+nt}{\PYGZlt{}/estado\PYGZgt{}}
            \PYG{n+nt}{\PYGZlt{}ciudad}\PYG{n+nt}{\PYGZgt{}}Paris\PYG{n+nt}{\PYGZlt{}/ciudad\PYGZgt{}}
        \PYG{n+nt}{\PYGZlt{}/proveedor\PYGZgt{}}
        \PYG{n+nt}{\PYGZlt{}proveedor} \PYG{n+na}{numprov=}\PYG{l+s}{\PYGZdq{}v4\PYGZdq{}}\PYG{n+nt}{\PYGZgt{}}
            \PYG{n+nt}{\PYGZlt{}nombreprov}\PYG{n+nt}{\PYGZgt{}}Clarke\PYG{n+nt}{\PYGZlt{}/nombreprov\PYGZgt{}}
            \PYG{n+nt}{\PYGZlt{}estado}\PYG{n+nt}{\PYGZgt{}}20\PYG{n+nt}{\PYGZlt{}/estado\PYGZgt{}}
            \PYG{n+nt}{\PYGZlt{}ciudad}\PYG{n+nt}{\PYGZgt{}}Londres\PYG{n+nt}{\PYGZlt{}/ciudad\PYGZgt{}}
        \PYG{n+nt}{\PYGZlt{}/proveedor\PYGZgt{}}
        \PYG{n+nt}{\PYGZlt{}proveedor} \PYG{n+na}{numprov=}\PYG{l+s}{\PYGZdq{}v5\PYGZdq{}}\PYG{n+nt}{\PYGZgt{}}
            \PYG{n+nt}{\PYGZlt{}nombreprov}\PYG{n+nt}{\PYGZgt{}}Adams\PYG{n+nt}{\PYGZlt{}/nombreprov\PYGZgt{}}
            \PYG{n+nt}{\PYGZlt{}estado}\PYG{n+nt}{\PYGZgt{}}30\PYG{n+nt}{\PYGZlt{}/estado\PYGZgt{}}
            \PYG{n+nt}{\PYGZlt{}ciudad}\PYG{n+nt}{\PYGZgt{}}Atenas\PYG{n+nt}{\PYGZlt{}/ciudad\PYGZgt{}}
        \PYG{n+nt}{\PYGZlt{}/proveedor\PYGZgt{}}       
    \PYG{n+nt}{\PYGZlt{}/proveedores\PYGZgt{}}
    \PYG{n+nt}{\PYGZlt{}partes}\PYG{n+nt}{\PYGZgt{}}
        \PYG{n+nt}{\PYGZlt{}parte} \PYG{n+na}{numparte=}\PYG{l+s}{\PYGZdq{}p1\PYGZdq{}}\PYG{n+nt}{\PYGZgt{}}
            \PYG{n+nt}{\PYGZlt{}nombreparte}\PYG{n+nt}{\PYGZgt{}}Tuerca\PYG{n+nt}{\PYGZlt{}/nombreparte\PYGZgt{}}
            \PYG{n+nt}{\PYGZlt{}color}\PYG{n+nt}{\PYGZgt{}}Rojo\PYG{n+nt}{\PYGZlt{}/color\PYGZgt{}}
            \PYG{n+nt}{\PYGZlt{}peso}\PYG{n+nt}{\PYGZgt{}}12\PYG{n+nt}{\PYGZlt{}/peso\PYGZgt{}}
            \PYG{n+nt}{\PYGZlt{}ciudad}\PYG{n+nt}{\PYGZgt{}}Londres\PYG{n+nt}{\PYGZlt{}/ciudad\PYGZgt{}}
        \PYG{n+nt}{\PYGZlt{}/parte\PYGZgt{}}
        \PYG{n+nt}{\PYGZlt{}parte} \PYG{n+na}{numparte=}\PYG{l+s}{\PYGZdq{}p2\PYGZdq{}}\PYG{n+nt}{\PYGZgt{}}
            \PYG{n+nt}{\PYGZlt{}nombreparte}\PYG{n+nt}{\PYGZgt{}}Perno\PYG{n+nt}{\PYGZlt{}/nombreparte\PYGZgt{}}
            \PYG{n+nt}{\PYGZlt{}color}\PYG{n+nt}{\PYGZgt{}}Verde\PYG{n+nt}{\PYGZlt{}/color\PYGZgt{}}
            \PYG{n+nt}{\PYGZlt{}peso}\PYG{n+nt}{\PYGZgt{}}17\PYG{n+nt}{\PYGZlt{}/peso\PYGZgt{}}
            \PYG{n+nt}{\PYGZlt{}ciudad}\PYG{n+nt}{\PYGZgt{}}Paris\PYG{n+nt}{\PYGZlt{}/ciudad\PYGZgt{}}
        \PYG{n+nt}{\PYGZlt{}/parte\PYGZgt{}}
        \PYG{n+nt}{\PYGZlt{}parte} \PYG{n+na}{numparte=}\PYG{l+s}{\PYGZdq{}p3\PYGZdq{}}\PYG{n+nt}{\PYGZgt{}}
            \PYG{n+nt}{\PYGZlt{}nombreparte}\PYG{n+nt}{\PYGZgt{}}Tornillo\PYG{n+nt}{\PYGZlt{}/nombreparte\PYGZgt{}}
            \PYG{n+nt}{\PYGZlt{}color}\PYG{n+nt}{\PYGZgt{}}Azul\PYG{n+nt}{\PYGZlt{}/color\PYGZgt{}}
            \PYG{n+nt}{\PYGZlt{}peso}\PYG{n+nt}{\PYGZgt{}}17\PYG{n+nt}{\PYGZlt{}/peso\PYGZgt{}}
            \PYG{n+nt}{\PYGZlt{}ciudad}\PYG{n+nt}{\PYGZgt{}}Roma\PYG{n+nt}{\PYGZlt{}/ciudad\PYGZgt{}}
        \PYG{n+nt}{\PYGZlt{}/parte\PYGZgt{}}
        \PYG{n+nt}{\PYGZlt{}parte} \PYG{n+na}{numparte=}\PYG{l+s}{\PYGZdq{}p4\PYGZdq{}}\PYG{n+nt}{\PYGZgt{}}
            \PYG{n+nt}{\PYGZlt{}nombreparte}\PYG{n+nt}{\PYGZgt{}}Tornillo\PYG{n+nt}{\PYGZlt{}/nombreparte\PYGZgt{}}
            \PYG{n+nt}{\PYGZlt{}color}\PYG{n+nt}{\PYGZgt{}}Rojo\PYG{n+nt}{\PYGZlt{}/color\PYGZgt{}}
            \PYG{n+nt}{\PYGZlt{}peso}\PYG{n+nt}{\PYGZgt{}}14\PYG{n+nt}{\PYGZlt{}/peso\PYGZgt{}}
            \PYG{n+nt}{\PYGZlt{}ciudad}\PYG{n+nt}{\PYGZgt{}}Londres\PYG{n+nt}{\PYGZlt{}/ciudad\PYGZgt{}}
        \PYG{n+nt}{\PYGZlt{}/parte\PYGZgt{}}
        \PYG{n+nt}{\PYGZlt{}parte} \PYG{n+na}{numparte=}\PYG{l+s}{\PYGZdq{}p5\PYGZdq{}}\PYG{n+nt}{\PYGZgt{}}
            \PYG{n+nt}{\PYGZlt{}nombreparte}\PYG{n+nt}{\PYGZgt{}}Leva\PYG{n+nt}{\PYGZlt{}/nombreparte\PYGZgt{}}
            \PYG{n+nt}{\PYGZlt{}color}\PYG{n+nt}{\PYGZgt{}}Azul\PYG{n+nt}{\PYGZlt{}/color\PYGZgt{}}
            \PYG{n+nt}{\PYGZlt{}peso}\PYG{n+nt}{\PYGZgt{}}12\PYG{n+nt}{\PYGZlt{}/peso\PYGZgt{}}
            \PYG{n+nt}{\PYGZlt{}ciudad}\PYG{n+nt}{\PYGZgt{}}Paris\PYG{n+nt}{\PYGZlt{}/ciudad\PYGZgt{}}
        \PYG{n+nt}{\PYGZlt{}/parte\PYGZgt{}}
        \PYG{n+nt}{\PYGZlt{}parte} \PYG{n+na}{numparte=}\PYG{l+s}{\PYGZdq{}p6\PYGZdq{}}\PYG{n+nt}{\PYGZgt{}}
            \PYG{n+nt}{\PYGZlt{}nombreparte}\PYG{n+nt}{\PYGZgt{}}Engranaje\PYG{n+nt}{\PYGZlt{}/nombreparte\PYGZgt{}}
            \PYG{n+nt}{\PYGZlt{}color}\PYG{n+nt}{\PYGZgt{}}Rojo\PYG{n+nt}{\PYGZlt{}/color\PYGZgt{}}
            \PYG{n+nt}{\PYGZlt{}peso}\PYG{n+nt}{\PYGZgt{}}19\PYG{n+nt}{\PYGZlt{}/peso\PYGZgt{}}
            \PYG{n+nt}{\PYGZlt{}ciudad}\PYG{n+nt}{\PYGZgt{}}Londres\PYG{n+nt}{\PYGZlt{}/ciudad\PYGZgt{}}
        \PYG{n+nt}{\PYGZlt{}/parte\PYGZgt{}}
    \PYG{n+nt}{\PYGZlt{}/partes\PYGZgt{}}
    \PYG{n+nt}{\PYGZlt{}proyectos}\PYG{n+nt}{\PYGZgt{}}
        \PYG{n+nt}{\PYGZlt{}proyecto} \PYG{n+na}{numproyecto=}\PYG{l+s}{\PYGZdq{}y1\PYGZdq{}}\PYG{n+nt}{\PYGZgt{}}
            \PYG{n+nt}{\PYGZlt{}nombreproyecto}\PYG{n+nt}{\PYGZgt{}}Clasificador\PYG{n+nt}{\PYGZlt{}/nombreproyecto\PYGZgt{}}
            \PYG{n+nt}{\PYGZlt{}ciudad}\PYG{n+nt}{\PYGZgt{}}Paris\PYG{n+nt}{\PYGZlt{}/ciudad\PYGZgt{}}
        \PYG{n+nt}{\PYGZlt{}/proyecto\PYGZgt{}}
        \PYG{n+nt}{\PYGZlt{}proyecto} \PYG{n+na}{numproyecto=}\PYG{l+s}{\PYGZdq{}y2\PYGZdq{}}\PYG{n+nt}{\PYGZgt{}}
            \PYG{n+nt}{\PYGZlt{}nombreproyecto}\PYG{n+nt}{\PYGZgt{}}Monitor\PYG{n+nt}{\PYGZlt{}/nombreproyecto\PYGZgt{}}
            \PYG{n+nt}{\PYGZlt{}ciudad}\PYG{n+nt}{\PYGZgt{}}Roma\PYG{n+nt}{\PYGZlt{}/ciudad\PYGZgt{}}
        \PYG{n+nt}{\PYGZlt{}/proyecto\PYGZgt{}}
        \PYG{n+nt}{\PYGZlt{}proyecto} \PYG{n+na}{numproyecto=}\PYG{l+s}{\PYGZdq{}y3\PYGZdq{}}\PYG{n+nt}{\PYGZgt{}}
            \PYG{n+nt}{\PYGZlt{}nombreproyecto}\PYG{n+nt}{\PYGZgt{}}OCR\PYG{n+nt}{\PYGZlt{}/nombreproyecto\PYGZgt{}}
            \PYG{n+nt}{\PYGZlt{}ciudad}\PYG{n+nt}{\PYGZgt{}}Atenas\PYG{n+nt}{\PYGZlt{}/ciudad\PYGZgt{}}
        \PYG{n+nt}{\PYGZlt{}/proyecto\PYGZgt{}}
        \PYG{n+nt}{\PYGZlt{}proyecto} \PYG{n+na}{numproyecto=}\PYG{l+s}{\PYGZdq{}y4\PYGZdq{}}\PYG{n+nt}{\PYGZgt{}}
            \PYG{n+nt}{\PYGZlt{}nombreproyecto}\PYG{n+nt}{\PYGZgt{}}Consola\PYG{n+nt}{\PYGZlt{}/nombreproyecto\PYGZgt{}}
            \PYG{n+nt}{\PYGZlt{}ciudad}\PYG{n+nt}{\PYGZgt{}}Atenas\PYG{n+nt}{\PYGZlt{}/ciudad\PYGZgt{}}
        \PYG{n+nt}{\PYGZlt{}/proyecto\PYGZgt{}}
        \PYG{n+nt}{\PYGZlt{}proyecto} \PYG{n+na}{numproyecto=}\PYG{l+s}{\PYGZdq{}y5\PYGZdq{}}\PYG{n+nt}{\PYGZgt{}}
            \PYG{n+nt}{\PYGZlt{}nombreproyecto}\PYG{n+nt}{\PYGZgt{}}RAID\PYG{n+nt}{\PYGZlt{}/nombreproyecto\PYGZgt{}}
            \PYG{n+nt}{\PYGZlt{}ciudad}\PYG{n+nt}{\PYGZgt{}}Londres\PYG{n+nt}{\PYGZlt{}/ciudad\PYGZgt{}}
        \PYG{n+nt}{\PYGZlt{}/proyecto\PYGZgt{}}
        \PYG{n+nt}{\PYGZlt{}proyecto} \PYG{n+na}{numproyecto=}\PYG{l+s}{\PYGZdq{}y6\PYGZdq{}}\PYG{n+nt}{\PYGZgt{}}
            \PYG{n+nt}{\PYGZlt{}nombreproyecto}\PYG{n+nt}{\PYGZgt{}}EDS\PYG{n+nt}{\PYGZlt{}/nombreproyecto\PYGZgt{}}
            \PYG{n+nt}{\PYGZlt{}ciudad}\PYG{n+nt}{\PYGZgt{}}Oslo\PYG{n+nt}{\PYGZlt{}/ciudad\PYGZgt{}}
        \PYG{n+nt}{\PYGZlt{}/proyecto\PYGZgt{}}
        \PYG{n+nt}{\PYGZlt{}proyecto} \PYG{n+na}{numproyecto=}\PYG{l+s}{\PYGZdq{}y7\PYGZdq{}}\PYG{n+nt}{\PYGZgt{}}
            \PYG{n+nt}{\PYGZlt{}nombreproyecto}\PYG{n+nt}{\PYGZgt{}}Cinta\PYG{n+nt}{\PYGZlt{}/nombreproyecto\PYGZgt{}}
            \PYG{n+nt}{\PYGZlt{}ciudad}\PYG{n+nt}{\PYGZgt{}}Londres\PYG{n+nt}{\PYGZlt{}/ciudad\PYGZgt{}}
        \PYG{n+nt}{\PYGZlt{}/proyecto\PYGZgt{}}
    \PYG{n+nt}{\PYGZlt{}/proyectos\PYGZgt{}}
    \PYG{n+nt}{\PYGZlt{}suministros}\PYG{n+nt}{\PYGZgt{}}
        \PYG{n+nt}{\PYGZlt{}suministra}\PYG{n+nt}{\PYGZgt{}}
            \PYG{n+nt}{\PYGZlt{}numprov}\PYG{n+nt}{\PYGZgt{}}v1\PYG{n+nt}{\PYGZlt{}/numprov\PYGZgt{}}
            \PYG{n+nt}{\PYGZlt{}numparte}\PYG{n+nt}{\PYGZgt{}}p1\PYG{n+nt}{\PYGZlt{}/numparte\PYGZgt{}}
            \PYG{n+nt}{\PYGZlt{}numproyecto}\PYG{n+nt}{\PYGZgt{}}y1\PYG{n+nt}{\PYGZlt{}/numproyecto\PYGZgt{}}
            \PYG{n+nt}{\PYGZlt{}cantidad}\PYG{n+nt}{\PYGZgt{}}200\PYG{n+nt}{\PYGZlt{}/cantidad\PYGZgt{}}
        \PYG{n+nt}{\PYGZlt{}/suministra\PYGZgt{}}
        \PYG{n+nt}{\PYGZlt{}suministra}\PYG{n+nt}{\PYGZgt{}}
            \PYG{n+nt}{\PYGZlt{}numprov}\PYG{n+nt}{\PYGZgt{}}v1\PYG{n+nt}{\PYGZlt{}/numprov\PYGZgt{}}
            \PYG{n+nt}{\PYGZlt{}numparte}\PYG{n+nt}{\PYGZgt{}}p1\PYG{n+nt}{\PYGZlt{}/numparte\PYGZgt{}}
            \PYG{n+nt}{\PYGZlt{}numproyecto}\PYG{n+nt}{\PYGZgt{}}y4\PYG{n+nt}{\PYGZlt{}/numproyecto\PYGZgt{}}
            \PYG{n+nt}{\PYGZlt{}cantidad}\PYG{n+nt}{\PYGZgt{}}700\PYG{n+nt}{\PYGZlt{}/cantidad\PYGZgt{}}
        \PYG{n+nt}{\PYGZlt{}/suministra\PYGZgt{}}
        \PYG{n+nt}{\PYGZlt{}suministra}\PYG{n+nt}{\PYGZgt{}}
            \PYG{n+nt}{\PYGZlt{}numprov}\PYG{n+nt}{\PYGZgt{}}v2\PYG{n+nt}{\PYGZlt{}/numprov\PYGZgt{}}
            \PYG{n+nt}{\PYGZlt{}numparte}\PYG{n+nt}{\PYGZgt{}}p3\PYG{n+nt}{\PYGZlt{}/numparte\PYGZgt{}}
            \PYG{n+nt}{\PYGZlt{}numproyecto}\PYG{n+nt}{\PYGZgt{}}y1\PYG{n+nt}{\PYGZlt{}/numproyecto\PYGZgt{}}
            \PYG{n+nt}{\PYGZlt{}cantidad}\PYG{n+nt}{\PYGZgt{}}400\PYG{n+nt}{\PYGZlt{}/cantidad\PYGZgt{}}
        \PYG{n+nt}{\PYGZlt{}/suministra\PYGZgt{}}
        \PYG{n+nt}{\PYGZlt{}suministra}\PYG{n+nt}{\PYGZgt{}}
            \PYG{n+nt}{\PYGZlt{}numprov}\PYG{n+nt}{\PYGZgt{}}v2\PYG{n+nt}{\PYGZlt{}/numprov\PYGZgt{}}
            \PYG{n+nt}{\PYGZlt{}numparte}\PYG{n+nt}{\PYGZgt{}}p3\PYG{n+nt}{\PYGZlt{}/numparte\PYGZgt{}}
            \PYG{n+nt}{\PYGZlt{}numproyecto}\PYG{n+nt}{\PYGZgt{}}y2\PYG{n+nt}{\PYGZlt{}/numproyecto\PYGZgt{}}
            \PYG{n+nt}{\PYGZlt{}cantidad}\PYG{n+nt}{\PYGZgt{}}200\PYG{n+nt}{\PYGZlt{}/cantidad\PYGZgt{}}
        \PYG{n+nt}{\PYGZlt{}/suministra\PYGZgt{}}
        \PYG{n+nt}{\PYGZlt{}suministra}\PYG{n+nt}{\PYGZgt{}}
            \PYG{n+nt}{\PYGZlt{}numprov}\PYG{n+nt}{\PYGZgt{}}v2\PYG{n+nt}{\PYGZlt{}/numprov\PYGZgt{}}
            \PYG{n+nt}{\PYGZlt{}numparte}\PYG{n+nt}{\PYGZgt{}}p3\PYG{n+nt}{\PYGZlt{}/numparte\PYGZgt{}}
            \PYG{n+nt}{\PYGZlt{}numproyecto}\PYG{n+nt}{\PYGZgt{}}y3\PYG{n+nt}{\PYGZlt{}/numproyecto\PYGZgt{}}
            \PYG{n+nt}{\PYGZlt{}cantidad}\PYG{n+nt}{\PYGZgt{}}300\PYG{n+nt}{\PYGZlt{}/cantidad\PYGZgt{}}
        \PYG{n+nt}{\PYGZlt{}/suministra\PYGZgt{}}
        \PYG{n+nt}{\PYGZlt{}suministra}\PYG{n+nt}{\PYGZgt{}}
            \PYG{n+nt}{\PYGZlt{}numprov}\PYG{n+nt}{\PYGZgt{}}v2\PYG{n+nt}{\PYGZlt{}/numprov\PYGZgt{}}
            \PYG{n+nt}{\PYGZlt{}numparte}\PYG{n+nt}{\PYGZgt{}}p3\PYG{n+nt}{\PYGZlt{}/numparte\PYGZgt{}}
            \PYG{n+nt}{\PYGZlt{}numproyecto}\PYG{n+nt}{\PYGZgt{}}y4\PYG{n+nt}{\PYGZlt{}/numproyecto\PYGZgt{}}
            \PYG{n+nt}{\PYGZlt{}cantidad}\PYG{n+nt}{\PYGZgt{}}500\PYG{n+nt}{\PYGZlt{}/cantidad\PYGZgt{}}
        \PYG{n+nt}{\PYGZlt{}/suministra\PYGZgt{}}
        \PYG{n+nt}{\PYGZlt{}suministra}\PYG{n+nt}{\PYGZgt{}}
            \PYG{n+nt}{\PYGZlt{}numprov}\PYG{n+nt}{\PYGZgt{}}v2\PYG{n+nt}{\PYGZlt{}/numprov\PYGZgt{}}
            \PYG{n+nt}{\PYGZlt{}numparte}\PYG{n+nt}{\PYGZgt{}}p3\PYG{n+nt}{\PYGZlt{}/numparte\PYGZgt{}}
            \PYG{n+nt}{\PYGZlt{}numproyecto}\PYG{n+nt}{\PYGZgt{}}y5\PYG{n+nt}{\PYGZlt{}/numproyecto\PYGZgt{}}
            \PYG{n+nt}{\PYGZlt{}cantidad}\PYG{n+nt}{\PYGZgt{}}600\PYG{n+nt}{\PYGZlt{}/cantidad\PYGZgt{}}
        \PYG{n+nt}{\PYGZlt{}/suministra\PYGZgt{}}
        \PYG{n+nt}{\PYGZlt{}suministra}\PYG{n+nt}{\PYGZgt{}}
            \PYG{n+nt}{\PYGZlt{}numprov}\PYG{n+nt}{\PYGZgt{}}v2\PYG{n+nt}{\PYGZlt{}/numprov\PYGZgt{}}
            \PYG{n+nt}{\PYGZlt{}numparte}\PYG{n+nt}{\PYGZgt{}}p3\PYG{n+nt}{\PYGZlt{}/numparte\PYGZgt{}}
            \PYG{n+nt}{\PYGZlt{}numproyecto}\PYG{n+nt}{\PYGZgt{}}y6\PYG{n+nt}{\PYGZlt{}/numproyecto\PYGZgt{}}
            \PYG{n+nt}{\PYGZlt{}cantidad}\PYG{n+nt}{\PYGZgt{}}400\PYG{n+nt}{\PYGZlt{}/cantidad\PYGZgt{}}
        \PYG{n+nt}{\PYGZlt{}/suministra\PYGZgt{}}
        \PYG{n+nt}{\PYGZlt{}suministra}\PYG{n+nt}{\PYGZgt{}}
            \PYG{n+nt}{\PYGZlt{}numprov}\PYG{n+nt}{\PYGZgt{}}v2\PYG{n+nt}{\PYGZlt{}/numprov\PYGZgt{}}
            \PYG{n+nt}{\PYGZlt{}numparte}\PYG{n+nt}{\PYGZgt{}}p3\PYG{n+nt}{\PYGZlt{}/numparte\PYGZgt{}}
            \PYG{n+nt}{\PYGZlt{}numproyecto}\PYG{n+nt}{\PYGZgt{}}y7\PYG{n+nt}{\PYGZlt{}/numproyecto\PYGZgt{}}
            \PYG{n+nt}{\PYGZlt{}cantidad}\PYG{n+nt}{\PYGZgt{}}600\PYG{n+nt}{\PYGZlt{}/cantidad\PYGZgt{}}
        \PYG{n+nt}{\PYGZlt{}/suministra\PYGZgt{}}
        \PYG{n+nt}{\PYGZlt{}suministra}\PYG{n+nt}{\PYGZgt{}}
            \PYG{n+nt}{\PYGZlt{}numprov}\PYG{n+nt}{\PYGZgt{}}v2\PYG{n+nt}{\PYGZlt{}/numprov\PYGZgt{}}
            \PYG{n+nt}{\PYGZlt{}numparte}\PYG{n+nt}{\PYGZgt{}}p5\PYG{n+nt}{\PYGZlt{}/numparte\PYGZgt{}}
            \PYG{n+nt}{\PYGZlt{}numproyecto}\PYG{n+nt}{\PYGZgt{}}y2\PYG{n+nt}{\PYGZlt{}/numproyecto\PYGZgt{}}
            \PYG{n+nt}{\PYGZlt{}cantidad}\PYG{n+nt}{\PYGZgt{}}100\PYG{n+nt}{\PYGZlt{}/cantidad\PYGZgt{}}
        \PYG{n+nt}{\PYGZlt{}/suministra\PYGZgt{}}
        \PYG{n+nt}{\PYGZlt{}suministra}\PYG{n+nt}{\PYGZgt{}}
            \PYG{n+nt}{\PYGZlt{}numprov}\PYG{n+nt}{\PYGZgt{}}v3\PYG{n+nt}{\PYGZlt{}/numprov\PYGZgt{}}
            \PYG{n+nt}{\PYGZlt{}numparte}\PYG{n+nt}{\PYGZgt{}}p3\PYG{n+nt}{\PYGZlt{}/numparte\PYGZgt{}}
            \PYG{n+nt}{\PYGZlt{}numproyecto}\PYG{n+nt}{\PYGZgt{}}y1\PYG{n+nt}{\PYGZlt{}/numproyecto\PYGZgt{}}
            \PYG{n+nt}{\PYGZlt{}cantidad}\PYG{n+nt}{\PYGZgt{}}200\PYG{n+nt}{\PYGZlt{}/cantidad\PYGZgt{}}
        \PYG{n+nt}{\PYGZlt{}/suministra\PYGZgt{}}
        \PYG{n+nt}{\PYGZlt{}suministra}\PYG{n+nt}{\PYGZgt{}}
            \PYG{n+nt}{\PYGZlt{}numprov}\PYG{n+nt}{\PYGZgt{}}v3\PYG{n+nt}{\PYGZlt{}/numprov\PYGZgt{}}
            \PYG{n+nt}{\PYGZlt{}numparte}\PYG{n+nt}{\PYGZgt{}}p4\PYG{n+nt}{\PYGZlt{}/numparte\PYGZgt{}}
            \PYG{n+nt}{\PYGZlt{}numproyecto}\PYG{n+nt}{\PYGZgt{}}y2\PYG{n+nt}{\PYGZlt{}/numproyecto\PYGZgt{}}
            \PYG{n+nt}{\PYGZlt{}cantidad}\PYG{n+nt}{\PYGZgt{}}500\PYG{n+nt}{\PYGZlt{}/cantidad\PYGZgt{}}
        \PYG{n+nt}{\PYGZlt{}/suministra\PYGZgt{}}
        \PYG{n+nt}{\PYGZlt{}suministra}\PYG{n+nt}{\PYGZgt{}}
            \PYG{n+nt}{\PYGZlt{}numprov}\PYG{n+nt}{\PYGZgt{}}v4\PYG{n+nt}{\PYGZlt{}/numprov\PYGZgt{}}
            \PYG{n+nt}{\PYGZlt{}numparte}\PYG{n+nt}{\PYGZgt{}}p6\PYG{n+nt}{\PYGZlt{}/numparte\PYGZgt{}}
            \PYG{n+nt}{\PYGZlt{}numproyecto}\PYG{n+nt}{\PYGZgt{}}y3\PYG{n+nt}{\PYGZlt{}/numproyecto\PYGZgt{}}
            \PYG{n+nt}{\PYGZlt{}cantidad}\PYG{n+nt}{\PYGZgt{}}300\PYG{n+nt}{\PYGZlt{}/cantidad\PYGZgt{}}
        \PYG{n+nt}{\PYGZlt{}/suministra\PYGZgt{}}
        \PYG{n+nt}{\PYGZlt{}suministra}\PYG{n+nt}{\PYGZgt{}}
            \PYG{n+nt}{\PYGZlt{}numprov}\PYG{n+nt}{\PYGZgt{}}v4\PYG{n+nt}{\PYGZlt{}/numprov\PYGZgt{}}
            \PYG{n+nt}{\PYGZlt{}numparte}\PYG{n+nt}{\PYGZgt{}}p6\PYG{n+nt}{\PYGZlt{}/numparte\PYGZgt{}}
            \PYG{n+nt}{\PYGZlt{}numproyecto}\PYG{n+nt}{\PYGZgt{}}y7\PYG{n+nt}{\PYGZlt{}/numproyecto\PYGZgt{}}
            \PYG{n+nt}{\PYGZlt{}cantidad}\PYG{n+nt}{\PYGZgt{}}300\PYG{n+nt}{\PYGZlt{}/cantidad\PYGZgt{}}
        \PYG{n+nt}{\PYGZlt{}/suministra\PYGZgt{}}
        \PYG{n+nt}{\PYGZlt{}suministra}\PYG{n+nt}{\PYGZgt{}}
            \PYG{n+nt}{\PYGZlt{}numprov}\PYG{n+nt}{\PYGZgt{}}v5\PYG{n+nt}{\PYGZlt{}/numprov\PYGZgt{}}
            \PYG{n+nt}{\PYGZlt{}numparte}\PYG{n+nt}{\PYGZgt{}}p2\PYG{n+nt}{\PYGZlt{}/numparte\PYGZgt{}}
            \PYG{n+nt}{\PYGZlt{}numproyecto}\PYG{n+nt}{\PYGZgt{}}y2\PYG{n+nt}{\PYGZlt{}/numproyecto\PYGZgt{}}
            \PYG{n+nt}{\PYGZlt{}cantidad}\PYG{n+nt}{\PYGZgt{}}200\PYG{n+nt}{\PYGZlt{}/cantidad\PYGZgt{}}
        \PYG{n+nt}{\PYGZlt{}/suministra\PYGZgt{}}
        \PYG{n+nt}{\PYGZlt{}suministra}\PYG{n+nt}{\PYGZgt{}}
            \PYG{n+nt}{\PYGZlt{}numprov}\PYG{n+nt}{\PYGZgt{}}v5\PYG{n+nt}{\PYGZlt{}/numprov\PYGZgt{}}
            \PYG{n+nt}{\PYGZlt{}numparte}\PYG{n+nt}{\PYGZgt{}}p2\PYG{n+nt}{\PYGZlt{}/numparte\PYGZgt{}}
            \PYG{n+nt}{\PYGZlt{}numproyecto}\PYG{n+nt}{\PYGZgt{}}y4\PYG{n+nt}{\PYGZlt{}/numproyecto\PYGZgt{}}
            \PYG{n+nt}{\PYGZlt{}cantidad}\PYG{n+nt}{\PYGZgt{}}100\PYG{n+nt}{\PYGZlt{}/cantidad\PYGZgt{}}
        \PYG{n+nt}{\PYGZlt{}/suministra\PYGZgt{}}
        \PYG{n+nt}{\PYGZlt{}suministra}\PYG{n+nt}{\PYGZgt{}}
            \PYG{n+nt}{\PYGZlt{}numprov}\PYG{n+nt}{\PYGZgt{}}v5\PYG{n+nt}{\PYGZlt{}/numprov\PYGZgt{}}
            \PYG{n+nt}{\PYGZlt{}numparte}\PYG{n+nt}{\PYGZgt{}}p5\PYG{n+nt}{\PYGZlt{}/numparte\PYGZgt{}}
            \PYG{n+nt}{\PYGZlt{}numproyecto}\PYG{n+nt}{\PYGZgt{}}y5\PYG{n+nt}{\PYGZlt{}/numproyecto\PYGZgt{}}
            \PYG{n+nt}{\PYGZlt{}cantidad}\PYG{n+nt}{\PYGZgt{}}500\PYG{n+nt}{\PYGZlt{}/cantidad\PYGZgt{}}
        \PYG{n+nt}{\PYGZlt{}/suministra\PYGZgt{}}
        \PYG{n+nt}{\PYGZlt{}suministra}\PYG{n+nt}{\PYGZgt{}}
            \PYG{n+nt}{\PYGZlt{}numprov}\PYG{n+nt}{\PYGZgt{}}v5\PYG{n+nt}{\PYGZlt{}/numprov\PYGZgt{}}
            \PYG{n+nt}{\PYGZlt{}numparte}\PYG{n+nt}{\PYGZgt{}}p6\PYG{n+nt}{\PYGZlt{}/numparte\PYGZgt{}}
            \PYG{n+nt}{\PYGZlt{}numproyecto}\PYG{n+nt}{\PYGZgt{}}y2\PYG{n+nt}{\PYGZlt{}/numproyecto\PYGZgt{}}
            \PYG{n+nt}{\PYGZlt{}cantidad}\PYG{n+nt}{\PYGZgt{}}200\PYG{n+nt}{\PYGZlt{}/cantidad\PYGZgt{}}
        \PYG{n+nt}{\PYGZlt{}/suministra\PYGZgt{}}
        \PYG{n+nt}{\PYGZlt{}suministra}\PYG{n+nt}{\PYGZgt{}}
            \PYG{n+nt}{\PYGZlt{}numprov}\PYG{n+nt}{\PYGZgt{}}v5\PYG{n+nt}{\PYGZlt{}/numprov\PYGZgt{}}
            \PYG{n+nt}{\PYGZlt{}numparte}\PYG{n+nt}{\PYGZgt{}}p1\PYG{n+nt}{\PYGZlt{}/numparte\PYGZgt{}}
            \PYG{n+nt}{\PYGZlt{}numproyecto}\PYG{n+nt}{\PYGZgt{}}y4\PYG{n+nt}{\PYGZlt{}/numproyecto\PYGZgt{}}
            \PYG{n+nt}{\PYGZlt{}cantidad}\PYG{n+nt}{\PYGZgt{}}100\PYG{n+nt}{\PYGZlt{}/cantidad\PYGZgt{}}
        \PYG{n+nt}{\PYGZlt{}/suministra\PYGZgt{}}
        \PYG{n+nt}{\PYGZlt{}suministra}\PYG{n+nt}{\PYGZgt{}}
            \PYG{n+nt}{\PYGZlt{}numprov}\PYG{n+nt}{\PYGZgt{}}v5\PYG{n+nt}{\PYGZlt{}/numprov\PYGZgt{}}
            \PYG{n+nt}{\PYGZlt{}numparte}\PYG{n+nt}{\PYGZgt{}}p3\PYG{n+nt}{\PYGZlt{}/numparte\PYGZgt{}}
            \PYG{n+nt}{\PYGZlt{}numproyecto}\PYG{n+nt}{\PYGZgt{}}y4\PYG{n+nt}{\PYGZlt{}/numproyecto\PYGZgt{}}
            \PYG{n+nt}{\PYGZlt{}cantidad}\PYG{n+nt}{\PYGZgt{}}200\PYG{n+nt}{\PYGZlt{}/cantidad\PYGZgt{}}
        \PYG{n+nt}{\PYGZlt{}/suministra\PYGZgt{}}
        \PYG{n+nt}{\PYGZlt{}suministra}\PYG{n+nt}{\PYGZgt{}}
            \PYG{n+nt}{\PYGZlt{}numprov}\PYG{n+nt}{\PYGZgt{}}v5\PYG{n+nt}{\PYGZlt{}/numprov\PYGZgt{}}
            \PYG{n+nt}{\PYGZlt{}numparte}\PYG{n+nt}{\PYGZgt{}}p4\PYG{n+nt}{\PYGZlt{}/numparte\PYGZgt{}}
            \PYG{n+nt}{\PYGZlt{}numproyecto}\PYG{n+nt}{\PYGZgt{}}y4\PYG{n+nt}{\PYGZlt{}/numproyecto\PYGZgt{}}
            \PYG{n+nt}{\PYGZlt{}cantidad}\PYG{n+nt}{\PYGZgt{}}800\PYG{n+nt}{\PYGZlt{}/cantidad\PYGZgt{}}
        \PYG{n+nt}{\PYGZlt{}/suministra\PYGZgt{}}
        \PYG{n+nt}{\PYGZlt{}suministra}\PYG{n+nt}{\PYGZgt{}}
            \PYG{n+nt}{\PYGZlt{}numprov}\PYG{n+nt}{\PYGZgt{}}v5\PYG{n+nt}{\PYGZlt{}/numprov\PYGZgt{}}
            \PYG{n+nt}{\PYGZlt{}numparte}\PYG{n+nt}{\PYGZgt{}}p5\PYG{n+nt}{\PYGZlt{}/numparte\PYGZgt{}}
            \PYG{n+nt}{\PYGZlt{}numproyecto}\PYG{n+nt}{\PYGZgt{}}y4\PYG{n+nt}{\PYGZlt{}/numproyecto\PYGZgt{}}
            \PYG{n+nt}{\PYGZlt{}cantidad}\PYG{n+nt}{\PYGZgt{}}400\PYG{n+nt}{\PYGZlt{}/cantidad\PYGZgt{}}
        \PYG{n+nt}{\PYGZlt{}/suministra\PYGZgt{}}
        \PYG{n+nt}{\PYGZlt{}suministra}\PYG{n+nt}{\PYGZgt{}}
            \PYG{n+nt}{\PYGZlt{}numprov}\PYG{n+nt}{\PYGZgt{}}v5\PYG{n+nt}{\PYGZlt{}/numprov\PYGZgt{}}
            \PYG{n+nt}{\PYGZlt{}numparte}\PYG{n+nt}{\PYGZgt{}}p6\PYG{n+nt}{\PYGZlt{}/numparte\PYGZgt{}}
            \PYG{n+nt}{\PYGZlt{}numproyecto}\PYG{n+nt}{\PYGZgt{}}y4\PYG{n+nt}{\PYGZlt{}/numproyecto\PYGZgt{}}
            \PYG{n+nt}{\PYGZlt{}cantidad}\PYG{n+nt}{\PYGZgt{}}500\PYG{n+nt}{\PYGZlt{}/cantidad\PYGZgt{}}
        \PYG{n+nt}{\PYGZlt{}/suministra\PYGZgt{}}
    \PYG{n+nt}{\PYGZlt{}/suministros\PYGZgt{}}
\PYG{n+nt}{\PYGZlt{}/datos\PYGZgt{}}
\end{sphinxVerbatim}

A continuación se muestra la estructura en forma de tabla de los elementos XML de dicho archivo. Obsérvese que los atributos llevan la arroba delante y que no se han puesto todas las filas:

Tabla proveedores


\begin{savenotes}\sphinxattablestart
\centering
\begin{tabulary}{\linewidth}[t]{|T|T|T|T|}
\hline
\sphinxstylethead{\sphinxstyletheadfamily 
@numprov
\unskip}\relax &\sphinxstylethead{\sphinxstyletheadfamily 
nombre
\unskip}\relax &\sphinxstylethead{\sphinxstyletheadfamily 
estado
\unskip}\relax &\sphinxstylethead{\sphinxstyletheadfamily 
ciudad
\unskip}\relax \\
\hline
v1
&
Smith
&
20
&
Londres
\\
\hline
v2
&
Jones
&
10
&
Paris
\\
\hline
\end{tabulary}
\par
\sphinxattableend\end{savenotes}

Tabla partes:


\begin{savenotes}\sphinxattablestart
\centering
\begin{tabulary}{\linewidth}[t]{|T|T|T|T|T|}
\hline
\sphinxstylethead{\sphinxstyletheadfamily 
@numparte
\unskip}\relax &\sphinxstylethead{\sphinxstyletheadfamily 
nombreparte
\unskip}\relax &\sphinxstylethead{\sphinxstyletheadfamily 
color
\unskip}\relax &\sphinxstylethead{\sphinxstyletheadfamily 
peso
\unskip}\relax &\sphinxstylethead{\sphinxstyletheadfamily 
ciudad
\unskip}\relax \\
\hline
p1
&
Tuerca
&
Rojo
&
12
&
Londres
\\
\hline
p2
&
Perno
&
Verde
&
17
&
Paris
\\
\hline
\end{tabulary}
\par
\sphinxattableend\end{savenotes}

Tabla proyectos


\begin{savenotes}\sphinxattablestart
\centering
\begin{tabulary}{\linewidth}[t]{|T|T|T|}
\hline
\sphinxstylethead{\sphinxstyletheadfamily 
@numproyecto
\unskip}\relax &\sphinxstylethead{\sphinxstyletheadfamily 
nombreproyecto
\unskip}\relax &\sphinxstylethead{\sphinxstyletheadfamily 
ciudad
\unskip}\relax \\
\hline
y1
&
Clasificador
&
Paris
\\
\hline
y2
&
Monitor
&
Roma
\\
\hline
\end{tabulary}
\par
\sphinxattableend\end{savenotes}

Tabla suministra


\begin{savenotes}\sphinxattablestart
\centering
\begin{tabulary}{\linewidth}[t]{|T|T|T|T|}
\hline
\sphinxstylethead{\sphinxstyletheadfamily 
numprov
\unskip}\relax &\sphinxstylethead{\sphinxstyletheadfamily 
numparte
\unskip}\relax &\sphinxstylethead{\sphinxstyletheadfamily 
numproyecto
\unskip}\relax &\sphinxstylethead{\sphinxstyletheadfamily 
cantidad
\unskip}\relax \\
\hline
v1
&
p1
&
y1
&
200
\\
\hline
v1
&
p1
&
y4
&
700
\\
\hline
\end{tabulary}
\par
\sphinxattableend\end{savenotes}

Obsérvese también que:
\begin{itemize}
\item {} 
En la tabla \sphinxcode{suministra} el campo \sphinxcode{numprov} es el mismo que el campo \sphinxcode{numprov} de la tabla \sphinxcode{proveedor}

\item {} 
En la tabla \sphinxcode{suministra} el campo \sphinxcode{numparte} es el mismo que el campo \sphinxcode{numparte} de la tabla \sphinxcode{partes}

\item {} 
En la tabla \sphinxcode{suministra} el campo \sphinxcode{numproyecto} es el mismo que el campo \sphinxcode{numproyecto} de la tabla \sphinxcode{proyectos}

\end{itemize}


\section{Consulta: ciudad de los proveedores}
\label{\detokenize{ejercicios/xquery/anexo_ejercicios_xquery:consulta-ciudad-de-los-proveedores}}
Extraer la ciudad de los proveedores (no debe aparecer la etiqueta) que tengan un estado mayor de 15.

\begin{sphinxVerbatim}[commandchars=\\\{\}]
\PYG{x}{for \PYGZdl{}proveedor in doc(\PYGZdq{}datos.xml\PYGZdq{})/datos/proveedores/proveedor}
\PYG{x}{where \PYGZdl{}proveedor/estado \PYGZgt{} 15}
\PYG{x}{return \PYGZdl{}proveedor/ciudad/text()}
\end{sphinxVerbatim}

Si se ejecuta esto se verá que el resultado es correcto sin embargo la presentación no es muy buena, al mostrarse todo seguido. Usemos por ejemplo la función \sphinxcode{concat} para que cada resultado lleve un espacio detrás:

\begin{sphinxVerbatim}[commandchars=\\\{\}]
\PYG{x}{for \PYGZdl{}proveedor in doc(\PYGZdq{}datos.xml\PYGZdq{})/datos/proveedores/proveedor}
\PYG{x}{where \PYGZdl{}proveedor/estado \PYGZgt{} 15}
\PYG{x}{return concat(\PYGZdl{}proveedor/ciudad/text(), \PYGZsq{} \PYGZsq{})}
\end{sphinxVerbatim}


\section{Consulta: filas de la tabla partes}
\label{\detokenize{ejercicios/xquery/anexo_ejercicios_xquery:consulta-filas-de-la-tabla-partes}}
Averiguar cuantas partes existen, es decir, el total de filas de la «tabla» partes.

Para resolverlo una tentación muy común es resolverlo así:

\begin{sphinxVerbatim}[commandchars=\\\{\}]
\PYG{x}{for \PYGZdl{}partes in doc(\PYGZdq{}datos.xml\PYGZdq{})/datos/partes}
\PYG{x}{return count (\PYGZdl{}partes/parte)}
\end{sphinxVerbatim}

Y aunque esta solución funciona \sphinxstylestrong{en realidad estamos haciendo un bucle de una sola iteración}.

En este caso, se puede recurrir directamente a la función \sphinxcode{count} que permite contar el número de elementos de una consulta parcial sin tener que hacer siquiera el recorrido.

\begin{sphinxVerbatim}[commandchars=\\\{\}]
\PYG{x}{count (doc(\PYGZdq{}datos.xml\PYGZdq{})/datos/partes/parte)}
\end{sphinxVerbatim}


\section{Consulta con join’s}
\label{\detokenize{ejercicios/xquery/anexo_ejercicios_xquery:consulta-con-join-s}}
Obtener el nombre de los proyectos cuya ciudad sea Paris y que reciban una cantidad de partes \textgreater{} 350


\section{Paso 0: análisis}
\label{\detokenize{ejercicios/xquery/anexo_ejercicios_xquery:paso-0-analisis}}
La cantidad está en las filas (elementos XML) \sphinxcode{suministra} pero el \sphinxcode{nombreproyecto} está en las filas \sphinxcode{proyecto}. Será necesario «cruzar» elementos \sphinxcode{proyecto} con elementos \sphinxcode{suministra} usando como condición que \sphinxcode{@numproyecto} de los elementos \sphinxcode{proyecto} sea igual a los campos \sphinxcode{numproyecto} de los elementos \sphinxcode{suministra}


\section{Paso 1: hacemos el cruce}
\label{\detokenize{ejercicios/xquery/anexo_ejercicios_xquery:paso-1-hacemos-el-cruce}}
Una primera aproximación sería esta:

Sin embargo, esto devuelve «todos los proyectos» que de alguna manera aparezcan en la tabla \sphinxcode{suministra}


\section{Paso 2: añadir condiciones}
\label{\detokenize{ejercicios/xquery/anexo_ejercicios_xquery:paso-2-anadir-condiciones}}
Nos han dicho que la cantidad de la tabla \sphinxcode{suministra} debe ser mayor de 350, así que en el where o en el for de \sphinxcode{suministra} podemos añadir una condición de filtrado:

Asimismo necesitamos solamente los proyectos cuyo campo \sphinxcode{ciudad} sea Paris.

Otra variante usando condiciones en el \sphinxcode{where} sería esta:


\section{Consulta: ciudades iguales}
\label{\detokenize{ejercicios/xquery/anexo_ejercicios_xquery:consulta-ciudades-iguales}}
Obtener los nombres de proyecto y nombres de parte que estén en la misma ciudad.

\begin{sphinxVerbatim}[commandchars=\\\{\}]
\PYG{x}{for \PYGZdl{}proyecto in}
\PYG{x}{    doc(\PYGZdq{}datos.xml\PYGZdq{})/datos/proyectos/proyecto}
\PYG{x}{    for \PYGZdl{}parte in}
\PYG{x}{        doc(\PYGZdq{}datos.xml\PYGZdq{})/datos/partes/parte}
\PYG{x}{        where}
\PYG{x}{            \PYGZdl{}parte/ciudad = \PYGZdl{}proyecto/ciudad}
\PYG{x}{    return concat(}
\PYG{x}{        \PYGZdl{}parte/nombreparte, \PYGZdq{} en la misma ciudad que \PYGZdq{},}
\PYG{x}{        \PYGZdl{}proyecto/nombreproyecto, \PYGZdq{}\PYGZhy{}\PYGZhy{}\PYGZhy{}\PYGZhy{}\PYGZhy{}\PYGZdq{}}
\PYG{x}{    )}
\end{sphinxVerbatim}


\section{Consulta: partes con colores iguales}
\label{\detokenize{ejercicios/xquery/anexo_ejercicios_xquery:consulta-partes-con-colores-iguales}}
Obtener parejas de partes que tengan el mismo color (indicando el nombre de ambas partes y el color que comparten)

\begin{sphinxVerbatim}[commandchars=\\\{\}]
\PYG{x}{for \PYGZdl{}p1 in}
\PYG{x}{    doc(\PYGZdq{}datos.xml\PYGZdq{})/datos/partes/parte}
\PYG{x}{for \PYGZdl{}p2 in}
\PYG{x}{    doc(\PYGZdq{}datos.xml\PYGZdq{})/datos/partes/parte}
\PYG{x}{    where}
\PYG{x}{        \PYGZdl{}p1/color = \PYGZdl{}p2/color}
\PYG{x}{    return concat (\PYGZdl{}p1/nombreparte,}
\PYG{x}{                   \PYGZdq{} tiene el mismo color que \PYGZdq{},}
\PYG{x}{                   \PYGZdl{}p2/nombreparte,}
\PYG{x}{                   \PYGZdq{} en concreto el color es:\PYGZdq{},}
\PYG{x}{                   \PYGZdl{}p1/color, \PYGZdq{}}
\PYG{x}{                   \PYGZdq{})}
\end{sphinxVerbatim}

Esta consulta funciona, pero ofrece parejas de partes que no tienen mucho sentido en pantalla, por ejemplo «Tuerca es igual que Tuerca». Para mejorar la consulta, vamos a eliminar parejas en la cuales el \sphinxcode{numparte} sea el mismo, es decir no vamos a contemplar el emparejar una parte consigo misma.

\begin{sphinxVerbatim}[commandchars=\\\{\}]
\PYG{x}{for \PYGZdl{}p1 in}
\PYG{x}{    doc(\PYGZdq{}datos.xml\PYGZdq{})/datos/partes/parte}
\PYG{x}{for \PYGZdl{}p2 in}
\PYG{x}{    doc(\PYGZdq{}datos.xml\PYGZdq{})/datos/partes/parte}
\PYG{x}{where}
\PYG{x}{    \PYGZdl{}p1/color = \PYGZdl{}p2/color}
\PYG{x}{and}
\PYG{x}{    \PYGZdl{}p1/@numparte != \PYGZdl{}p2/@numparte}
\PYG{x}{return concat (\PYGZdl{}p1/nombreparte,}
\PYG{x}{                   \PYGZdq{} tiene el mismo color que \PYGZdq{},}
\PYG{x}{                   \PYGZdl{}p2/nombreparte,}
\PYG{x}{                   \PYGZdq{} en concreto el color es:\PYGZdq{},}
\PYG{x}{                   \PYGZdl{}p1/color, \PYGZdq{}}
\PYG{x}{                   \PYGZdq{})}
\end{sphinxVerbatim}


\section{Consulta: cantidad de partes de Londres}
\label{\detokenize{ejercicios/xquery/anexo_ejercicios_xquery:consulta-cantidad-de-partes-de-londres}}
Averiguar cuantas partes existen cuya ciudad sea «Londres», es decir, el total de filas de la «tabla» partes pero teniendo en cuenta la condición de que el «campo» ciudad debe ser Londres.

\begin{sphinxVerbatim}[commandchars=\\\{\}]
\PYG{x}{count (doc(\PYGZdq{}datos.xml\PYGZdq{})/datos/partes/parte[ciudad=\PYGZsq{}Londres\PYGZsq{}])}
\end{sphinxVerbatim}


\section{Consulta: media de partes rojas}
\label{\detokenize{ejercicios/xquery/anexo_ejercicios_xquery:consulta-media-de-partes-rojas}}
Crear una consulta XQuery que averigüe la media de partes suministradas cuyo color sea “Rojo”


\subsection{Paso 1: cruce de tablas}
\label{\detokenize{ejercicios/xquery/anexo_ejercicios_xquery:paso-1-cruce-de-tablas}}

\subsection{Paso 2: añadir condicion de filtrado}
\label{\detokenize{ejercicios/xquery/anexo_ejercicios_xquery:paso-2-anadir-condicion-de-filtrado}}

\subsection{Paso 3: calcular la media}
\label{\detokenize{ejercicios/xquery/anexo_ejercicios_xquery:paso-3-calcular-la-media}}

\subsection{Comprobación}
\label{\detokenize{ejercicios/xquery/anexo_ejercicios_xquery:comprobacion}}
Si analizamos la tabla partes veremos que las únicas partes cuyo color es “Rojo” son las partes \sphinxcode{p1}, \sphinxcode{p4} y \sphinxcode{p6}.
Esto significa que las unicas filas de \sphinxcode{suministra} que nos interesan son estas:


\begin{savenotes}\sphinxattablestart
\centering
\begin{tabulary}{\linewidth}[t]{|T|T|T|T|}
\hline
\sphinxstylethead{\sphinxstyletheadfamily 
numprov
\unskip}\relax &\sphinxstylethead{\sphinxstyletheadfamily 
numparte
\unskip}\relax &\sphinxstylethead{\sphinxstyletheadfamily 
numproyecto
\unskip}\relax &\sphinxstylethead{\sphinxstyletheadfamily 
cantidad
\unskip}\relax \\
\hline
v1
&
p1
&
y1
&
200
\\
\hline
v1
&
p1
&
y4
&
700
\\
\hline
v3
&
p4
&
y2
&
500
\\
\hline
v4
&
p6
&
y3
&
300
\\
\hline
v4
&
p6
&
y7
&
300
\\
\hline
v5
&
p6
&
y2
&
200
\\
\hline
v5
&
p1
&
y4
&
100
\\
\hline
v5
&
p4
&
y4
&
800
\\
\hline
v5
&
p6
&
y4
&
500
\\
\hline
\end{tabulary}
\par
\sphinxattableend\end{savenotes}

Como vemos hay 9 filas con suministros de partes cuyo color es “Rojo” y la suma de cantidades es 3600 por lo el resultado correcto es 400


\section{Consulta: media de suministros}
\label{\detokenize{ejercicios/xquery/anexo_ejercicios_xquery:consulta-media-de-suministros}}
Averiguar la media de la cantidad de partes que aparecen en la «tabla» suministra

\begin{sphinxVerbatim}[commandchars=\\\{\}]
\PYG{x}{avg (doc(\PYGZdq{}datos.xml\PYGZdq{})/datos/suministros/suministra/cantidad)}
\end{sphinxVerbatim}

Pregunta: ¿por qué no ponemos esta solución?

\begin{sphinxVerbatim}[commandchars=\\\{\}]
\PYG{x}{avg (doc(\PYGZdq{}datos.xml\PYGZdq{})/datos/suministros/suministra/cantidad/text())}
\end{sphinxVerbatim}

Respuesta: las funciones son capaces de «extraer el texto automáticamente» si en el elemento no hay hijos. En este caso, la cantidad no tiene hijos, por lo que podemos ahorrarnos el \sphinxcode{text()}


\section{Consulta: media por proveedor}
\label{\detokenize{ejercicios/xquery/anexo_ejercicios_xquery:consulta-media-por-proveedor}}
Averiguar la media de cantidades por proveedor usando los datos de la tabla suministra.

Hagamos esta consulta por partes. En primer lugar, saquemos los distintos proveedores que hay en suministra

\begin{sphinxVerbatim}[commandchars=\\\{\}]
\PYG{x}{for \PYGZdl{}n in distinct\PYGZhy{}values(}
\PYG{x}{doc(\PYGZdq{}datos.xml\PYGZdq{})/datos/}
\PYG{x}{proveedores/proveedor/@numprov)}
\PYG{x}{return \PYGZdl{}n}
\end{sphinxVerbatim}

Ahora, teniendo los distintos proveedores podemos devolver algo distinto de \sphinxcode{numprov}. Podemos devolver la media para ese proveedor aprovechando los filtrados XPath.

\begin{sphinxVerbatim}[commandchars=\\\{\}]
\PYG{x}{for \PYGZdl{}n in distinct\PYGZhy{}values(}
\PYG{x}{doc(\PYGZdq{}datos.xml\PYGZdq{})/datos/}
\PYG{x}{proveedores/proveedor/@numprov)}
\PYG{x}{return avg}
\PYG{x}{( doc(\PYGZdq{}datos.xml\PYGZdq{})/datos/suministros/suministra}
\PYG{x}{[numprov=\PYGZdl{}n]/cantidad)}
\end{sphinxVerbatim}

Y por último, si usamos \sphinxcode{concat} apropiadamente podemos hacer que aparezca el número de proveedor al lado de dica cantidad.

\begin{sphinxVerbatim}[commandchars=\\\{\}]
\PYG{x}{for \PYGZdl{}n in distinct\PYGZhy{}values(}
\PYG{x}{doc(\PYGZdq{}datos.xml\PYGZdq{})/datos/proveedores/proveedor/@numprov)}
\PYG{x}{return concat (}
\PYG{x}{ \PYGZdl{}n, \PYGZsq{} \PYGZsq{}, avg(}
\PYG{x}{ doc(\PYGZdq{}datos.xml\PYGZdq{})/datos/suministros}
\PYG{x}{/suministra[numprov=\PYGZdl{}n]/cantidad)}
\PYG{x}{)}
\end{sphinxVerbatim}


\section{Consulta: suministros en grandes cantidades}
\label{\detokenize{ejercicios/xquery/anexo_ejercicios_xquery:consulta-suministros-en-grandes-cantidades}}
Averiguar el nombre de los proyectos (sin que haya repeticiones) que reciban una cantidad en la tabla suministra mayor de 650.

\begin{sphinxVerbatim}[commandchars=\\\{\}]
\PYG{x}{for \PYGZdl{}proyecto}
\PYG{x}{in doc(\PYGZdq{}datos.xml\PYGZdq{})/datos/proyectos/proyecto}
\PYG{x}{for \PYGZdl{}suministra}
\PYG{x}{in}
\PYG{x}{doc(\PYGZdq{}datos.xml\PYGZdq{})/datos/suministros/suministra[cantidad\PYGZgt{}650]}
\PYG{x}{where \PYGZdl{}proyecto/@numproyecto = \PYGZdl{}suministra/numproyecto}
\PYG{x}{return (\PYGZdl{}proyecto/nombreproyecto, \PYGZdl{}suministra/cantidad)}
\end{sphinxVerbatim}



\renewcommand{\indexname}{Índice}
\printindex
\end{document}